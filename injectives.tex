\IfFileExists{stacks-project.cls}{%
\documentclass{stacks-project}
}{%
\documentclass{amsart}
}

% The following AMS packages are automatically loaded with
% the amsart documentclass:
%\usepackage{amsmath}
%\usepackage{amssymb}
%\usepackage{amsthm}

\usepackage{graphicx}

% For dealing with references we use the comment environment
\usepackage{verbatim}
\newenvironment{reference}{\comment}{\endcomment}
%\newenvironment{reference}{}{}
\newenvironment{slogan}{\comment}{\endcomment}
\newenvironment{history}{\comment}{\endcomment}

% For commutative diagrams you can use
% \usepackage{amscd}
\usepackage[all]{xy}

% We use 2cell for 2-commutative diagrams.
\xyoption{2cell}
\UseAllTwocells

% To put source file link in headers.
% Change "template.tex" to "this_filename.tex"
% \usepackage{fancyhdr}
% \pagestyle{fancy}
% \lhead{}
% \chead{}
% \rhead{Source file: \url{template.tex}}
% \lfoot{}
% \cfoot{\thepage}
% \rfoot{}
% \renewcommand{\headrulewidth}{0pt}
% \renewcommand{\footrulewidth}{0pt}
% \renewcommand{\headheight}{12pt}

\usepackage{multicol}

% For cross-file-references
\usepackage{xr-hyper}

% Package for hypertext links:
\usepackage{hyperref}

% For any local file, say "hello.tex" you want to link to please
% use \externaldocument[hello-]{hello}
\externaldocument[introduction-]{introduction}
\externaldocument[conventions-]{conventions}
\externaldocument[sets-]{sets}
\externaldocument[categories-]{categories}
\externaldocument[topology-]{topology}
\externaldocument[sheaves-]{sheaves}
\externaldocument[sites-]{sites}
\externaldocument[stacks-]{stacks}
\externaldocument[fields-]{fields}
\externaldocument[algebra-]{algebra}
\externaldocument[brauer-]{brauer}
\externaldocument[homology-]{homology}
\externaldocument[derived-]{derived}
\externaldocument[simplicial-]{simplicial}
\externaldocument[more-algebra-]{more-algebra}
\externaldocument[smoothing-]{smoothing}
\externaldocument[modules-]{modules}
\externaldocument[sites-modules-]{sites-modules}
\externaldocument[injectives-]{injectives}
\externaldocument[cohomology-]{cohomology}
\externaldocument[sites-cohomology-]{sites-cohomology}
\externaldocument[dga-]{dga}
\externaldocument[dpa-]{dpa}
\externaldocument[hypercovering-]{hypercovering}
\externaldocument[schemes-]{schemes}
\externaldocument[constructions-]{constructions}
\externaldocument[properties-]{properties}
\externaldocument[morphisms-]{morphisms}
\externaldocument[coherent-]{coherent}
\externaldocument[divisors-]{divisors}
\externaldocument[limits-]{limits}
\externaldocument[varieties-]{varieties}
\externaldocument[topologies-]{topologies}
\externaldocument[descent-]{descent}
\externaldocument[perfect-]{perfect}
\externaldocument[more-morphisms-]{more-morphisms}
\externaldocument[flat-]{flat}
\externaldocument[groupoids-]{groupoids}
\externaldocument[more-groupoids-]{more-groupoids}
\externaldocument[etale-]{etale}
\externaldocument[chow-]{chow}
\externaldocument[intersection-]{intersection}
\externaldocument[pic-]{pic}
\externaldocument[adequate-]{adequate}
\externaldocument[dualizing-]{dualizing}
\externaldocument[duality-]{duality}
\externaldocument[discriminant-]{discriminant}
\externaldocument[local-cohomology-]{local-cohomology}
\externaldocument[curves-]{curves}
\externaldocument[resolve-]{resolve}
\externaldocument[models-]{models}
\externaldocument[pione-]{pione}
\externaldocument[etale-cohomology-]{etale-cohomology}
\externaldocument[ssgroups-]{ssgroups}
\externaldocument[proetale-]{proetale}
\externaldocument[crystalline-]{crystalline}
\externaldocument[spaces-]{spaces}
\externaldocument[spaces-properties-]{spaces-properties}
\externaldocument[spaces-morphisms-]{spaces-morphisms}
\externaldocument[decent-spaces-]{decent-spaces}
\externaldocument[spaces-cohomology-]{spaces-cohomology}
\externaldocument[spaces-limits-]{spaces-limits}
\externaldocument[spaces-divisors-]{spaces-divisors}
\externaldocument[spaces-over-fields-]{spaces-over-fields}
\externaldocument[spaces-topologies-]{spaces-topologies}
\externaldocument[spaces-descent-]{spaces-descent}
\externaldocument[spaces-perfect-]{spaces-perfect}
\externaldocument[spaces-more-morphisms-]{spaces-more-morphisms}
\externaldocument[spaces-flat-]{spaces-flat}
\externaldocument[spaces-groupoids-]{spaces-groupoids}
\externaldocument[spaces-more-groupoids-]{spaces-more-groupoids}
\externaldocument[bootstrap-]{bootstrap}
\externaldocument[spaces-pushouts-]{spaces-pushouts}
\externaldocument[spaces-chow-]{spaces-chow}
\externaldocument[groupoids-quotients-]{groupoids-quotients}
\externaldocument[spaces-more-cohomology-]{spaces-more-cohomology}
\externaldocument[spaces-simplicial-]{spaces-simplicial}
\externaldocument[spaces-duality-]{spaces-duality}
\externaldocument[formal-spaces-]{formal-spaces}
\externaldocument[restricted-]{restricted}
\externaldocument[spaces-resolve-]{spaces-resolve}
\externaldocument[formal-defos-]{formal-defos}
\externaldocument[defos-]{defos}
\externaldocument[cotangent-]{cotangent}
\externaldocument[examples-defos-]{examples-defos}
\externaldocument[algebraic-]{algebraic}
\externaldocument[examples-stacks-]{examples-stacks}
\externaldocument[stacks-sheaves-]{stacks-sheaves}
\externaldocument[criteria-]{criteria}
\externaldocument[artin-]{artin}
\externaldocument[quot-]{quot}
\externaldocument[stacks-properties-]{stacks-properties}
\externaldocument[stacks-morphisms-]{stacks-morphisms}
\externaldocument[stacks-limits-]{stacks-limits}
\externaldocument[stacks-cohomology-]{stacks-cohomology}
\externaldocument[stacks-perfect-]{stacks-perfect}
\externaldocument[stacks-introduction-]{stacks-introduction}
\externaldocument[stacks-more-morphisms-]{stacks-more-morphisms}
\externaldocument[stacks-geometry-]{stacks-geometry}
\externaldocument[moduli-]{moduli}
\externaldocument[moduli-curves-]{moduli-curves}
\externaldocument[examples-]{examples}
\externaldocument[exercises-]{exercises}
\externaldocument[guide-]{guide}
\externaldocument[desirables-]{desirables}
\externaldocument[coding-]{coding}
\externaldocument[obsolete-]{obsolete}
\externaldocument[fdl-]{fdl}
\externaldocument[index-]{index}

% Theorem environments.
%
\theoremstyle{plain}
\newtheorem{theorem}[subsection]{Theorem}
\newtheorem{proposition}[subsection]{Proposition}
\newtheorem{lemma}[subsection]{Lemma}

\theoremstyle{definition}
\newtheorem{definition}[subsection]{Definition}
\newtheorem{example}[subsection]{Example}
\newtheorem{exercise}[subsection]{Exercise}
\newtheorem{situation}[subsection]{Situation}

\theoremstyle{remark}
\newtheorem{remark}[subsection]{Remark}
\newtheorem{remarks}[subsection]{Remarks}

\numberwithin{equation}{subsection}

% Macros
%
\def\lim{\mathop{\mathrm{lim}}\nolimits}
\def\colim{\mathop{\mathrm{colim}}\nolimits}
\def\Spec{\mathop{\mathrm{Spec}}}
\def\Hom{\mathop{\mathrm{Hom}}\nolimits}
\def\Ext{\mathop{\mathrm{Ext}}\nolimits}
\def\SheafHom{\mathop{\mathcal{H}\!\mathit{om}}\nolimits}
\def\SheafExt{\mathop{\mathcal{E}\!\mathit{xt}}\nolimits}
\def\Sch{\mathit{Sch}}
\def\Mor{\mathop{Mor}\nolimits}
\def\Ob{\mathop{\mathrm{Ob}}\nolimits}
\def\Sh{\mathop{\mathit{Sh}}\nolimits}
\def\NL{\mathop{N\!L}\nolimits}
\def\proetale{{pro\text{-}\acute{e}tale}}
\def\etale{{\acute{e}tale}}
\def\QCoh{\mathit{QCoh}}
\def\Ker{\mathop{\mathrm{Ker}}}
\def\Im{\mathop{\mathrm{Im}}}
\def\Coker{\mathop{\mathrm{Coker}}}
\def\Coim{\mathop{\mathrm{Coim}}}
\def\id{\mathop{\mathrm{id}}\nolimits}

%
% Macros for linear algebraic groups
%
\def\SL{\mathop{\mathrm{SL}}\nolimits}
\def\GL{\mathop{\mathrm{GL}}\nolimits}
\def\ltimes{{\mathchar"256E}}
\def\rtimes{{\mathchar"256F}}
\def\Rrightarrow{{\mathchar"3456}}

%
% Macros for moduli stacks/spaces
%
\def\QCohstack{\mathcal{QC}\!\mathit{oh}}
\def\Cohstack{\mathcal{C}\!\mathit{oh}}
\def\Spacesstack{\mathcal{S}\!\mathit{paces}}
\def\Quotfunctor{\mathrm{Quot}}
\def\Hilbfunctor{\mathrm{Hilb}}
\def\Curvesstack{\mathcal{C}\!\mathit{urves}}
\def\Polarizedstack{\mathcal{P}\!\mathit{olarized}}
\def\Complexesstack{\mathcal{C}\!\mathit{omplexes}}
% \Pic is the operator that assigns to X its picard group, usage \Pic(X)
% \Picardstack_{X/B} denotes the Picard stack of X over B
% \Picardfunctor_{X/B} denotes the Picard functor of X over B
\def\Pic{\mathop{\mathrm{Pic}}\nolimits}
\def\Picardstack{\mathcal{P}\!\mathit{ic}}
\def\Picardfunctor{\mathrm{Pic}}
\def\Deformationcategory{\mathcal{D}\!\mathit{ef}}


% OK, start here.
%
\begin{document}

\title{Injectives}


\maketitle

\phantomsection
\label{section-phantom}

\tableofcontents

\section{Introduction}
\label{section-introduction}

\noindent
In future chapters we will use the existence of injectives and
K-injective complexes to do cohomology of sheaves of modules on
ringed sites. In this chapter we explain how to produce injectives
and K-injective complexes first for modules on sites and later
more generally for Grothendieck abelian categories.

\medskip\noindent
We observe that we already know that the category of
abelian groups and the category of modules over a ring have
enough injectives, see More on Algebra, Sections
\ref{more-algebra-section-abelian-groups} and
\ref{more-algebra-section-injectives-modules}





\section{Baer's argument for modules}
\label{section-baer}

\noindent
There is another, more set-theoretic approach to showing that any $R$-module
$M$ can be imbedded in an injective module. This approach constructs
the injective module by a transfinite colimit of push-outs. While this
method is somewhat abstract and more complicated than the one of
More on Algebra, Section \ref{more-algebra-section-injectives-modules},
it is also more general. Apparently this method originates with Baer,
and was revisited by Cartan and Eilenberg in
\cite{Cartan-Eilenberg} and by Grothendieck in \cite{Tohoku}.
There Grothendieck uses it to show that
many other abelian categories have enough injectives. We will get back to
the general case later (Section \ref{section-grothendieck-categories}).

\medskip\noindent
We begin with a few set theoretic remarks.
Let $\{B_{\beta}\}_{\beta \in \alpha}$ be an inductive system of
objects in some category $\mathcal{C}$, indexed by
an ordinal $\alpha$. Assume that $\colim_{\beta \in \alpha} B_\beta$
exists in $\mathcal{C}$. If $A$ is an object of $\mathcal{C}$, then there is a
natural map
\begin{equation}
\label{equation-compare}
\colim_{\beta \in \alpha} \Mor_\mathcal{C}(A, B_\beta)
\longrightarrow
\Mor_\mathcal{C}(A, \colim_{\beta \in \alpha} B_\beta).
\end{equation}
because if one is given a map $A \to B_\beta$ for some $\beta$, one
naturally gets a map from $A$  into the colimit by composing with
$B_\beta \to \colim_{\beta \in \alpha} B_\alpha$.
Note that the left colimit is one of sets! In general, (\ref{equation-compare})
is neither injective or surjective.

\begin{example}
\label{example-not-surjective}
Consider the category of sets. Let $A = \mathbf{N}$ and
$B_n = \{1, \ldots, n\}$ be the inductive system indexed by the natural numbers
where $B_n \to B_m$ for $n \leq m$ is the obvious map. Then
$\colim B_n = \mathbf{N}$, so there is a map
$A \to \colim B_n$, which does not factor as $A \to B_m$
for any $m$. Consequently,
$\colim \Mor(A, B_n) \to \Mor(A, \colim B_n)$
is not surjective.
\end{example}

\begin{example}
\label{example-not-injective}
Next we give an example where the map fails to be injective. Let $B_n =
\mathbf{N}/\{1,  2, \ldots, n\}$, that is, the quotient set of
$\mathbf{N}$ with the first $n$ elements collapsed to one element.
There are natural maps $B_n \to B_m$ for $n \leq m$, so the
$\{B_n\}$ form a system of sets over $\mathbf{N}$. It is easy to see that
$\colim B_n = \{*\}$: it is the one-point set.
So it follows that $\Mor(A, \colim B_n)$ is a one-element set
for every set $A$.
However, $\colim \Mor(A , B_n)$ is {\bf not} a one-element set.
Consider the family of maps $A \to B_n$ which are just the natural projections
$\mathbf{N} \to \mathbf{N}/\{1, 2, \ldots, n\}$ and the family of
maps $A \to B_n$ which map the whole of $A$ to the class of $1$.
These two families of maps are distinct at each step and thus are distinct in
$\colim \Mor(A, B_n)$, but they induce the same map
$A \to \colim B_n$.
\end{example}

\noindent
Nonetheless, if we map out of a finite set then
(\ref{equation-compare}) is an isomorphism always.

\begin{lemma}
\label{lemma-out-of-finite}
Suppose that, in (\ref{equation-compare}), $\mathcal{C}$ is the category
of sets and $A$ is a {\it finite set}, then the map is a bijection.
\end{lemma}

\begin{proof}
Let $f : A \to \colim B_\beta$.
The range of $f$ is finite, containing say
elements $c_1, \ldots, c_r \in \colim B_\beta$.
These all come from some elements in $B_\beta$ for $\beta \in \alpha$
large by definition of the colimit. Thus we can define
$\widetilde{f} : A \to B_\beta$ lifting $f$ at a finite stage.
This proves that (\ref{equation-compare}) is surjective.
Next, suppose two maps $f : A \to B_\gamma, f' : A \to B_{\gamma'}$
define the same map $A \to \colim B_\beta$.
Then each of the finitely many elements of $A$ gets sent to the same point in
the colimit. By definition of the colimit for sets, there is
$\beta \geq \gamma, \gamma'$ such that the finitely many elements of
$A$ get sent to the same points in $B_\beta$ under $f$ and $f'$.
This proves that (\ref{equation-compare}) is injective.
\end{proof}

\noindent
The most interesting case of the lemma is when $\alpha = \omega$, i.e.,
when the system $\{B_\beta\}$ is a system $\{B_n\}_{n \in \mathbf{N}}$
over the natural numbers as in
Examples \ref{example-not-surjective} and
\ref{example-not-injective}.
The essential idea is that $A$ is ``small'' relative to the long chain of
compositions $B_1 \to B_2 \to \ldots$, so that it has to factor through a
finite step. A more general version of this lemma can be found in
Sets, Lemma \ref{sets-lemma-map-from-set-lifts}.
Next, we generalize this to the category of modules.

\begin{definition}
\label{definition-small}
Let $\mathcal{C}$ be a category, let $I \subset \text{Arrows}(\mathcal{C})$,
and let $\alpha$ be an ordinal. An object $A$ of $\mathcal{C}$ is said to
be {\it $\alpha$-small with respect to $I$} if whenever $\{B_\beta\}$ is
a system over $\alpha$ with transition maps in $I$, then
the map (\ref{equation-compare}) is an isomorphism.
\end{definition}

\noindent
In the rest of this section we shall restrict ourselves
to the category of $R$-modules for a fixed commutative ring $R$.
We shall also take $I$ to be the collection of injective maps, i.e., the
{\it monomorphisms} in the category of modules over $R$. In this case, for
any system $\{B_\beta\}$ as in the definition each of the maps
$$
B_\beta \to \colim_{\beta \in \alpha} B_\beta
$$
is an injection. It follows that the map (\ref{equation-compare}) is an
{\it injection}. We can in fact interpret the $B_\beta$'s as submodules
of the module $B = \colim_{\beta \in \alpha} B_\beta$, and then we
have $B = \bigcup_{\beta \in \alpha} B_\beta$. This is not an abuse of
notation if we identify $B_\alpha$ with the image in the colimit.
We now want to show that modules are always small for ``large'' ordinals
$\alpha$.

\begin{proposition}
\label{proposition-modules-are-small}
Let $R$ be a ring. Let $M$ be an $R$-module.
Let $\kappa$ the cardinality of the set of submodules of $M$.
If $\alpha$ is an ordinal whose cofinality is bigger than $\kappa$,
then $M$ is $\alpha$-small with respect to injections.
\end{proposition}

\begin{proof}
The proof is straightforward, but let us first think about a special case.
If $M$ is finite, then the claim is that for any inductive system
$\{B_\beta\}$ with injections between them, parametrized by a
limit ordinal, any map $M \to \colim B_\beta$ factors through one of
the $B_\beta$. And this we proved in
Lemma \ref{lemma-out-of-finite}.

\medskip\noindent
Now we start the proof in the general case.
We need only show that the map (\ref{equation-compare}) is a surjection.
Let $f : M \to \colim B_\beta$ be a map.
Consider the subobjects $\{f^{-1}(B_\beta)\}$ of $M$, where $B_\beta$
is considered as a subobject of the colimit $B = \bigcup_\beta B_\beta$.
If one of these, say $f^{-1}(B_\beta)$, fills $M$,
then the map factors through $B_\beta$.

\medskip\noindent
So suppose to the contrary that all of the $f^{-1}(B_\beta)$ were proper
subobjects of $M$. However, we know that
$$
\bigcup f^{-1}(B_\beta) = f^{-1}\left(\bigcup B_\beta\right) = M.
$$
Now there are at most $\kappa$ different subobjects of $M$ that occur among
the $f^{-1}(B_\alpha)$, by hypothesis.
Thus we can find a subset $S \subset \alpha$ of cardinality at most
$\kappa$ such that as $\beta'$ ranges over $S$, the
$f^{-1}(B_{\beta'})$ range over \emph{all} the $f^{-1}(B_\alpha)$.

\medskip\noindent
However, $S$ has an upper bound $\widetilde{\alpha} < \alpha$ as
$\alpha$ has cofinality bigger than $\kappa$. In particular, all the
$f^{-1}(B_{\beta'})$, $\beta' \in S$ are contained in
$f^{-1}(B_{\widetilde{\alpha}})$.
It follows that $f^{-1}(B_{\widetilde{\alpha}}) = M$.
In particular, the map $f$ factors through $B_{\widetilde{\alpha}}$.
\end{proof}

\noindent
From this lemma we will be able to deduce the existence of lots of injectives.
Let us recall Baer's criterion.

\begin{lemma}[Baer's criterion]
\label{lemma-criterion-baer}
\begin{reference}
\cite[Theorem 1]{Baer}
\end{reference}
Let $R$ be a ring. An $R$-module $Q$ is injective if and only if in every
commutative diagram
$$
\xymatrix{
\mathfrak{a} \ar[d] \ar[r] &  Q \\
R \ar@{-->}[ru]
}
$$
for $\mathfrak{a} \subset R$ an ideal, the dotted arrow exists.
\end{lemma}

\begin{proof}
This is the equivalence of (1) and (3) in
More on Algebra, Lemma \ref{more-algebra-lemma-characterize-injective-bis};
please observe that the proof given there is elementary
(and does not use $\text{Ext}$ groups or the existence of injectives
or projectives in the category of $R$-modules).
\end{proof}

\noindent
If $M$ is an $R$-module, then in general we may have a semi-complete
diagram as in
Lemma \ref{lemma-criterion-baer}.
In it, we can form the \emph{push-out}
$$
\xymatrix{
\mathfrak{a} \ar[d]  \ar[r] &  M \ar[d] \\
R \ar[r] &  R \oplus_{\mathfrak{a}} M.
}
$$
Here the vertical map is injective, and the diagram commutes. The point is
that we can extend $\mathfrak{a} \to M$ to $R$ \emph{if} we extend $M$ to the
larger module $R \oplus_{\mathfrak{a}} M$.

\medskip\noindent
The key point of Baer's argument is to repeat this procedure
transfinitely many times. To do this we first define, given an $R$-module
$M$ the following (huge) pushout
\begin{equation}
\label{equation-huge-diagram}
\vcenter{
\xymatrix{
\bigoplus_{\mathfrak a}
\bigoplus_{\varphi \in \Hom_R(\mathfrak a, M)}
\mathfrak{a} \ar[r] \ar[d] & M \ar[d] \\
\bigoplus_{\mathfrak a}
\bigoplus_{\varphi \in \Hom_R(\mathfrak a, M)}
R \ar[r] &  \mathbf{M}(M).
}
}
\end{equation}
Here the top horizontal arrow maps the element $a \in \mathfrak a$
in the summand corresponding to $\varphi$ to the element $\varphi(a) \in M$.
The left vertical arrow maps $a \in \mathfrak a$ in the summand corresponding
to $\varphi$ simply to the element $a \in R$ in the summand corresponding
to $\varphi$. The fundamental properties of this construction are
formulated in the following lemma.

\begin{lemma}
\label{lemma-construction}
Let $R$ be a ring.
\begin{enumerate}
\item The construction $M \mapsto (M \to \mathbf{M}(M))$
is functorial in $M$.
\item The map $M \to \mathbf{M}(M)$ is injective.
\item For any ideal $\mathfrak{a}$ and any $R$-module map
$\varphi : \mathfrak a \to M$ there is an $R$-module map
$\varphi' : R \to \mathbf{M}(M)$ such that
$$
\xymatrix{
\mathfrak{a} \ar[d] \ar[r]_\varphi &  M \ar[d] \\
R \ar[r]^{\varphi'} & \mathbf{M}(M)
}
$$
commutes.
\end{enumerate}
\end{lemma}

\begin{proof}
Parts (2) and (3) are immediate from the construction.
To see (1), let $\chi : M \to N$ be an $R$-module map. We claim there exists
a canonical commutative diagram
$$
\xymatrix{
\bigoplus_{\mathfrak a}
\bigoplus_{\varphi \in \Hom_R(\mathfrak a, M)}
\mathfrak{a} \ar[r] \ar[d] \ar[rrd] & M \ar[rrd]^\chi \\
\bigoplus_{\mathfrak a}
\bigoplus_{\varphi \in \Hom_R(\mathfrak a, M)}
R \ar[rrd] & &
\bigoplus_{\mathfrak a}
\bigoplus_{\psi \in \Hom_R(\mathfrak a, N)}
\mathfrak{a} \ar[r] \ar[d] & N \\
& & \bigoplus_{\mathfrak a}
\bigoplus_{\psi \in \Hom_R(\mathfrak a, N)}
R
}
$$
which induces the desired map $\mathbf{M}(M) \to \mathbf{M}(N)$.
The middle east-south-east arrow maps the summand $\mathfrak a$
corresponding to $\varphi$ via $\text{id}_{\mathfrak a}$ to the
summand $\mathfrak a$ corresponding to $\psi = \chi \circ \varphi$.
Similarly for the lower east-south-east arrow. Details omitted.
\end{proof}

\noindent
The idea will now be to apply the functor $\mathbf{M}$ a transfinite number
of times. We define for each ordinal $\alpha$ a functor $\mathbf{M}_\alpha$
on the category of $R$-modules, together with a natural injection $N \to
\mathbf{M}_\alpha(N)$. We do this by transfinite recursion.
First, $\mathbf{M}_1 = \mathbf{M}$ is the functor defined above.
Now, suppose given an ordinal $\alpha$, and suppose $\mathbf{M}_{\alpha'}$
is defined for $\alpha' < \alpha$. If $\alpha$ has an immediate predecessor
$\widetilde{\alpha}$, we let
$$
\mathbf{M}_\alpha = \mathbf{M} \circ \mathbf{M}_{\widetilde{\alpha}}.
$$
If not, i.e., if $\alpha$ is a limit ordinal, we let
$$
\mathbf{M}_{\alpha}(N) =
\colim_{\alpha' < \alpha} \mathbf{M}_{\alpha'}(N).
$$
It is clear (e.g., inductively) that the $\mathbf{M}_{\alpha}(N)$
form an inductive system over ordinals, so this is reasonable.

\begin{theorem}
\label{theorem-baer-grothendieck}
Let $\kappa$ be the cardinality of the set of ideals in $R$, and
let $\alpha$ be an ordinal whose cofinality is greater than
$\kappa$. Then $\mathbf{M}_\alpha(N)$ is an injective $R$-module,
and $N \to \mathbf{M}_\alpha(N)$ is a functorial injective embedding.
\end{theorem}

\begin{proof}
By Baer's criterion
Lemma \ref{lemma-criterion-baer},
it suffices to show that if $\mathfrak{a} \subset R$ is an ideal, then
any map $f : \mathfrak{a} \to \mathbf{M}_\alpha(N)$ extends to
$R \to \mathbf{M}_\alpha(N)$. However, we know since $\alpha$ is a limit
ordinal that
$$
\mathbf{M}_{\alpha}(N) =
\colim_{\beta < \alpha} \mathbf{M}_{\beta}(N),
$$
so by
Proposition \ref{proposition-modules-are-small},
we find that
$$
\Hom_R(\mathfrak{a}, \mathbf{M}_{\alpha}(N)) =
\colim_{\beta < \alpha} \Hom_R(\mathfrak a, \mathbf{M}_{\beta}(N)).
$$
This means in particular that there is some $\beta' < \alpha$
such that $f$ factors through the submodule $\mathbf{M}_{\beta'}(N)$, as
$$
f : \mathfrak{a} \to \mathbf{M}_{\beta'}(N) \to
\mathbf{M}_{\alpha}(N).
$$
However, by the fundamental property of the functor $\mathbf{M}$,
see Lemma \ref{lemma-construction} part (3),
we know that the map $\mathfrak{a} \to \mathbf{M}_{\beta'}(N)$
can be extended to
$$
R \to \mathbf{M}( \mathbf{M}_{\beta'}(N)) =
\mathbf{M}_{\beta' + 1}(N),
$$
and the last object imbeds in $\mathbf{M}_{\alpha}(N)$ (as
$\beta' + 1 < \alpha$ since $\alpha$ is a limit ordinal).
In particular, $f$ can be extended to $\mathbf{M}_{\alpha}(N)$.
\end{proof}




\section{G-modules}
\label{section-G-modules}

\noindent
We will see later
(Differential Graded Algebra, Section \ref{dga-section-modules-noncommutative})
that the category of modules over an algebra has
functorial injective embeddings. The construction is exactly the same
as the construction in
More on Algebra, Section \ref{more-algebra-section-injectives-modules}.

\begin{lemma}
\label{lemma-G-modules}
Let $G$ be a topological group. Let $R$ be a ring.
The category $\text{Mod}_{R, G}$ of $R\text{-}G$-modules, see
\'Etale Cohomology, Definition
\ref{etale-cohomology-definition-G-module-continuous},
has functorial injective embeddings. In particular this holds
for the category of discrete $G$-modules.
\end{lemma}

\begin{proof}
By the remark above the lemma the category $\text{Mod}_{R[G]}$
has functorial injective embeddings.
Consider the forgetful functor
$v : \text{Mod}_{R, G} \to \text{Mod}_{R[G]}$.
This functor is fully faithful, transforms injective maps into
injective maps and has a right adjoint, namely
$$
u : M \mapsto u(M) = \{x \in M \mid \text{stabilizer of }x\text{ is open}\}
$$
Since $v(M) = 0 \Rightarrow M = 0$ we conclude by
Homology, Lemma \ref{homology-lemma-adjoint-functorial-injectives}.
\end{proof}



\section{Abelian sheaves on a space}
\label{section-abelian-sheaves-space}


\begin{lemma}
\label{lemma-abelian-sheaves-space}
Let $X$ be a topological space.
The category of abelian sheaves on $X$ has enough injectives.
In fact it has functorial injective embeddings.
\end{lemma}

\begin{proof}
For an abelian group $A$ we denote $j : A \to J(A)$ the functorial
injective embedding constructed in
More on Algebra, Section \ref{more-algebra-section-injectives-modules}.
Let $\mathcal{F}$ be an abelian sheaf on $X$.
By Sheaves, Example \ref{sheaves-example-sheaf-product-pointwise}
the assignment
$$
\mathcal{I} : U \mapsto
\mathcal{I}(U) = \prod\nolimits_{x\in U} J(\mathcal{F}_x)
$$
is an abelian sheaf. There is a canonical map $\mathcal{F} \to \mathcal{I}$
given by mapping $s \in \mathcal{F}(U)$ to $\prod_{x \in U} j(s_x)$
where $s_x \in \mathcal{F}_x$ denotes the germ of $s$ at $x$.
This map is injective, see
Sheaves, Lemma \ref{sheaves-lemma-sheaf-subset-stalks}
for example.

\medskip\noindent
It remains to prove the following: Given a rule
$x \mapsto I_x$ which assigns to each point $x \in X$ an injective
abelian group the sheaf $\mathcal{I} : U \mapsto \prod_{x \in U} I_x$
is injective. Note that
$$
\mathcal{I} = \prod\nolimits_{x \in X} i_{x, *}I_x
$$
is the product of the skyscraper sheaves $i_{x, *}I_x$ (see
Sheaves, Section \ref{sheaves-section-skyscraper-sheaves} for notation.)
We have
$$
\Mor_{\textit{Ab}}(\mathcal{F}_x, I_x)
=
\Mor_{\textit{Ab}(X)}(\mathcal{F}, i_{x, *}I_x).
$$
see Sheaves, Lemma \ref{sheaves-lemma-stalk-skyscraper-adjoint}. Hence it is
clear that each $i_{x, *}I_x$ is injective. Hence the injectivity of
$\mathcal{I}$ follows from
Homology, Lemma \ref{homology-lemma-product-injectives}.
\end{proof}









\section{Sheaves of modules on a ringed space}
\label{section-sheaves-modules-space}


\begin{lemma}
\label{lemma-sheaves-modules-space}
Let $(X, \mathcal{O}_X)$ be a ringed space, see
Sheaves, Section \ref{sheaves-section-ringed-spaces}.
The category of sheaves of $\mathcal{O}_X$-modules on $X$
has enough injectives. In fact it has functorial injective embeddings.
\end{lemma}

\begin{proof}
For any ring $R$ and any $R$-module $M$ we denote
$j : M \to J_R(M)$ the functorial
injective embedding constructed in
More on Algebra, Section \ref{more-algebra-section-injectives-modules}.
Let $\mathcal{F}$ be a sheaf of $\mathcal{O}_X$-modules on $X$.
By Sheaves, Examples \ref{sheaves-example-sheaf-product-pointwise}
and \ref{sheaves-example-sheaf-product-pointwise-algebraic-structure}
the assignment
$$
\mathcal{I} : U \mapsto
\mathcal{I}(U) = \prod\nolimits_{x\in U} J_{\mathcal{O}_{X, x}}(\mathcal{F}_x)
$$
is an abelian sheaf.
There is a canonical map $\mathcal{F} \to \mathcal{I}$
given by mapping $s \in \mathcal{F}(U)$ to $\prod_{x \in U} j(s_x)$
where $s_x \in \mathcal{F}_x$ denotes the germ of $s$ at $x$.
This map is injective, see
Sheaves, Lemma \ref{sheaves-lemma-sheaf-subset-stalks}
for example.

\medskip\noindent
It remains to prove the following: Given a rule
$x \mapsto I_x$ which assigns to each point $x \in X$ an injective
$\mathcal{O}_{X, x}$-module
the sheaf $\mathcal{I} : U \mapsto \prod_{x \in U} I_x$
is injective. Note that
$$
\mathcal{I} = \prod\nolimits_{x \in X} i_{x, *}I_x
$$
is the product of the skyscraper sheaves $i_{x, *}I_x$ (see
Sheaves, Section \ref{sheaves-section-skyscraper-sheaves} for notation.)
We have
$$
\Hom_{\mathcal{O}_{X, x}}(\mathcal{F}_x, I_x)
=
\Hom_{\mathcal{O}_X}(\mathcal{F}, i_{x, *}I_x).
$$
see Sheaves, Lemma \ref{sheaves-lemma-stalk-skyscraper-adjoint}. Hence it is
clear that each $i_{x, *}I_x$ is an injective $\mathcal{O}_X$-module
(see Homology, Lemma \ref{homology-lemma-adjoint-preserve-injectives} or argue
directly). Hence the injectivity of $\mathcal{I}$ follows from
Homology, Lemma \ref{homology-lemma-product-injectives}.
\end{proof}













\section{Abelian presheaves on a category}
\label{section-injectives-presheaves}

\noindent
Let $\mathcal{C}$ be a category. Recall that this means that
$\Ob(\mathcal{C})$ is a set. On the one hand, consider abelian
presheaves on $\mathcal{C}$, see
Sites, Section \ref{sites-section-presheaves}.
On the other hand, consider families of abelian groups
indexed by elements of $\Ob(\mathcal{C})$; in other
words presheaves on the discrete category with underlying set
of objects $\Ob(\mathcal{C})$. Let us denote this
discrete category simply $\Ob(\mathcal{C})$.
There is a natural functor
$$
i : \Ob(\mathcal{C}) \longrightarrow \mathcal{C}
$$
and hence there is a natural restriction or forgetful functor
$$
v = i^p :
\textit{PAb}(\mathcal{C})
\longrightarrow
\textit{PAb}(\Ob(\mathcal{C}))
$$
compare Sites, Section \ref{sites-section-functoriality-PSh}.
We will denote presheaves
on $\mathcal{C}$ by $B$ and presheaves on
$\Ob(\mathcal{C})$ by $A$.

\medskip\noindent
There are also two functors, namely $i_p$ and ${}_pi$
which assign an abelian presheaf on $\mathcal{C}$
to an abelian presheaf on $\Ob(\mathcal{C})$, see
Sites, Sections \ref{sites-section-functoriality-PSh} and
\ref{sites-section-more-functoriality-PSh}.
Here we will use $u = {}_pi$ which is defined (in the case at hand)
as follows:
$$
uA(U) = \prod\nolimits_{U' \to U} A(U').
$$
So an element is a family $(a_\phi)_\phi$ with $\phi$
ranging through all morphisms in $\mathcal{C}$ with target $U$.
The restriction map on $uA$ corresponding to $g : V \to U$
maps our element $(a_\phi)_\phi$ to the element
$(a_{g \circ \psi})_\psi$.

\medskip\noindent
There is a canonical surjective map $vuA \to A$ and a canonical
injective map $B \to uvB$. We leave it to the reader to show that
$$
\Mor_{\textit{PAb}(\mathcal{C})}(B, uA)
=
\Mor_{\textit{PAb}(\Ob(\mathcal{C}))}(vB, A).
$$
in this simple case; the general case is in
Sites, Section \ref{sites-section-functoriality-PSh}.
Thus the pair $(u, v)$ is an example of a pair of adjoint
functors, see
Categories, Section \ref{categories-section-adjoint}.

\medskip\noindent
At this point we can list the following facts
about the situation above.
\begin{enumerate}
\item The functors $u$ and $v$ are exact. This follows from
the explicit description of these functors given above.
\item In particular the functor $v$ transforms injective maps
into injective maps.
\item The category $\textit{PAb}(\Ob(\mathcal{C}))$
has enough injectives.
\item In fact there is a functorial injective embedding
$A \mapsto \big(A \to J(A)\big)$ as in
Homology, Definition \ref{homology-definition-functorial-injective-embedding}.
Namely, we can take $J(A)$ to be the
presheaf $U\mapsto J(A(U))$, where
$J(-)$ is the functor constructed in
More on Algebra, Section \ref{more-algebra-section-injectives-modules}
for the ring $\mathbf{Z}$.
\end{enumerate}
Putting all of this together gives us the following procedure
for embedding objects $B$ of $\textit{PAb}(\mathcal{C})$ into
an injective object: $B \to uJ(vB)$. See
Homology, Lemma \ref{homology-lemma-adjoint-functorial-injectives}.

\begin{proposition}
\label{proposition-presheaves-injectives}
For abelian presheaves on a category there is a functorial injective
embedding.
\end{proposition}

\begin{proof}
See discussion above.
\end{proof}












\section{Abelian Sheaves on a site}
\label{section-injectives-sheaves}

\noindent
Let $\mathcal{C}$ be a site. In this section we prove that there are
enough injectives for abelian sheaves on $\mathcal{C}$.

\medskip\noindent
Denote
$i : \textit{Ab}(\mathcal{C}) \longrightarrow \textit{PAb}(\mathcal{C})$
the forgetful functor from abelian sheaves to abelian presheaves.
Let
${}^\# : \textit{PAb}(\mathcal{C}) \longrightarrow \textit{Ab}(\mathcal{C})$
denote the sheafification functor. Recall that ${}^\#$ is a left adjoint
to $i$, that ${}^\#$ is exact, and that $i\mathcal{F}^\# = \mathcal{F}$
for any abelian sheaf $\mathcal{F}$. Finally, let
$\mathcal{G} \to J(\mathcal{G})$ denote the canonical
embedding into an injective presheaf we found in
Section \ref{section-injectives-presheaves}.

\medskip\noindent
For any sheaf $\mathcal{F}$ in $\textit{Ab}(\mathcal{C})$ and
any ordinal $\beta$ we define a sheaf
$J_\beta(\mathcal{F})$ by transfinite recursion.
We set $J_0(\mathcal{F}) = \mathcal{F}$.
We define $J_1(\mathcal{F}) = J(i\mathcal{F})^\#$.
Sheafification of the canonical map $i\mathcal{F} \to J(i\mathcal{F})$
gives a functorial map
$$
\mathcal{F} \longrightarrow J_1(\mathcal{F})
$$
which is injective as $\#$ is exact. We set
$J_{\alpha + 1}(\mathcal{F}) = J_1(J_\alpha(\mathcal{F}))$.
So that there are canonical injective maps
$J_\alpha(\mathcal{F}) \to J_{\alpha + 1}(\mathcal{F})$.
For a limit ordinal $\beta$, we define
$$
J_\beta(\mathcal{F}) = \colim_{\alpha < \beta} J_\alpha(\mathcal{F}).
$$
Note that this is a directed colimit. Hence for any ordinals $\alpha < \beta$
we have an injective map $J_\alpha(\mathcal{F}) \to J_\beta(\mathcal{F})$.

\begin{lemma}
\label{lemma-map-into-next-one}
With notation as above.
Suppose that $\mathcal{G}_1 \to \mathcal{G}_2$ is an injective
map of abelian sheaves on $\mathcal{C}$. Let $\alpha$ be an ordinal
and let $\mathcal{G}_1 \to J_\alpha(\mathcal{F})$ be a morphism
of sheaves. There exists a morphism $\mathcal{G}_2 \to
J_{\alpha + 1}(\mathcal{F})$ such that the following diagram commutes
$$
\xymatrix{
\mathcal{G}_1 \ar[d] \ar[r] & \mathcal{G}_2 \ar[d] \\
J_{\alpha}(\mathcal{F}) \ar[r] & J_{\alpha + 1}(\mathcal{F}) }
$$
\end{lemma}

\begin{proof}
This is because the map $i\mathcal{G}_1 \to i\mathcal{G}_2$ is injective
and hence $i\mathcal{G}_1 \to iJ_\alpha(\mathcal{F})$ extends to
$i\mathcal{G}_2 \to J(iJ_\alpha(\mathcal{F}))$ which gives the
desired map after applying the sheafification functor.
\end{proof}

\noindent
This lemma says that somehow the system $\{J_{\alpha}(\mathcal{F})\}$
is an injective embedding of $\mathcal{F}$. Of course
we cannot take the limit over all $\alpha$ because they form a class
and not a set. However, the idea is now that you don't have to check
injectivity on all injections $\mathcal{G}_1 \to \mathcal{G}_2$, plus
the following lemma.

\begin{lemma}
\label{lemma-map-into-smaller}
Suppose that $\mathcal{G}_i$, $i\in I$ is set of abelian sheaves
on $\mathcal{C}$. There exists an ordinal $\beta$ such that
for any sheaf $\mathcal{F}$, any $i\in I$, and any map
$\varphi : \mathcal{G}_i \to J_\beta(\mathcal{F})$ there exists an
$\alpha < \beta$ such that $ \varphi $ factors through
$J_\alpha(\mathcal{F})$.
\end{lemma}

\begin{proof}
This reduces to the case of a single sheaf $\mathcal{G}$
by taking the direct sum of all the $\mathcal{G}_i$.

\medskip\noindent
Consider the sets
$$
S = \coprod\nolimits_{U \in \Ob(\mathcal{C})} \mathcal{G}(U).
$$
and
$$
T_\beta
=
\coprod\nolimits_{U \in \Ob(\mathcal{C})} J_\beta(\mathcal{F})(U)
$$
The transition maps between the sets $T_\beta$ are injective.
If the cofinality of $\beta$ is large enough, then
$T_\beta = \colim_{\alpha < \beta} T_\alpha$, see
Sites, Lemma \ref{sites-lemma-colimit-over-ordinal-sections}.
A morphism $\mathcal{G} \to J_\beta(\mathcal{F})$ factors
through $J_\alpha(\mathcal{F})$ if and only if
the associated map $S \to T_\beta$ factors through $T_\alpha$.
By
Sets, Lemma \ref{sets-lemma-map-from-set-lifts}
if the cofinality of $\beta$ is bigger than the cardinality
of $S$, then the result of the lemma is true. Hence the lemma
follows from the fact that there are ordinals with arbitrarily
large cofinality, see
Sets, Proposition \ref{sets-proposition-exist-ordinals-large-cofinality}.
\end{proof}

\noindent
Recall that for an object $X$ of $\mathcal{C}$ we denote $\mathbf{Z}_X$
the presheaf of abelian groups $\Gamma(U, \mathbf{Z}_X) =
\oplus_{U \to X} \mathbf{Z}$, see
Modules on Sites, Section \ref{sites-modules-section-free-abelian-presheaf}.
The sheaf associated to this presheaf
is denoted $\mathbf{Z}_X^\#$, see
Modules on Sites, Section \ref{sites-modules-section-free-abelian-sheaf}.
It can be characterized by
the property
\begin{equation}
\label{equation-free-sheaf-on}
\Mor_{\textit{Ab}(\mathcal{C})}(\mathbf{Z}_X^\#, \mathcal{G})
=
\mathcal{G}(X)
\end{equation}
where the element $\varphi$ of the left hand side is mapped
to $\varphi(1 \cdot \text{id}_X)$ in the right hand side. We can use these
sheaves to characterize injective abelian sheaves.

\begin{lemma}
\label{lemma-characterize-injectives}
Suppose $\mathcal{J}$ is a sheaf of abelian groups with the following
property: For all $X\in \Ob(\mathcal{C})$, for any abelian subsheaf
$\mathcal{S} \subset \mathbf{Z}_X^\#$ and any morphism
$\varphi : \mathcal{S} \to \mathcal{J}$, there exists a morphism
$\mathbf{Z}_X^\# \to \mathcal{J}$ extending $\varphi$.
Then $\mathcal{J}$ is an injective sheaf of abelian groups.
\end{lemma}

\begin{proof}
Let $\mathcal{F} \to \mathcal{G}$ be an injective map
of abelian sheaves. Suppose $\varphi : \mathcal{F} \to \mathcal{J}$
is a morphism. Arguing as in the proof of
More on Algebra, Lemma \ref{more-algebra-lemma-injective-abelian}
we see that it suffices
to prove that if $\mathcal{F} \not = \mathcal{G}$, then we
can find an abelian sheaf $\mathcal{F}'$,
$\mathcal{F} \subset \mathcal{F}' \subset \mathcal{G}$
such that (a) the inclusion $\mathcal{F} \subset \mathcal{F}'$ is strict,
and (b) $\varphi$ can be extended to $\mathcal{F}'$.
To find $\mathcal{F}'$, let $X$ be an object of $\mathcal{C}$ such
that the inclusion $\mathcal{F}(X) \subset \mathcal{G}(X)$
is strict. Pick $s \in \mathcal{G}(X)$, $s \not \in \mathcal{F}(X)$.
Let $\psi : \mathbf{Z}_X^\# \to \mathcal{G}$ be the morphism corresponding
to the section $s$ via (\ref{equation-free-sheaf-on}). Set
$\mathcal{S} = \psi^{-1}(\mathcal{F})$. By assumption the morphism
$$
\mathcal{S} \xrightarrow{\psi} \mathcal{F} \xrightarrow{\varphi} \mathcal{J}
$$
can be extended to a morphism $\varphi' : \mathbf{Z}_X^\# \to \mathcal{J}$.
Note that $\varphi'$ annihilates the kernel of $\psi$ (as this is true
for $\varphi$). Thus $\varphi'$ gives rise to a morphism
$\varphi'' : \Im(\psi) \to \mathcal{J}$
which agrees with $\varphi$ on the intersection
$\mathcal{F} \cap \Im(\psi)$ by construction.
Thus $\varphi$ and $\varphi''$ glue to give an extension
of $\varphi$ to the strictly bigger subsheaf
$\mathcal{F}' = \mathcal{F} + \Im(\psi)$.
\end{proof}

\begin{theorem}
\label{theorem-sheaves-injectives}
The category of sheaves of abelian groups on a
site has enough injectives. In fact there exists
a functorial injective embedding, see
Homology, Definition \ref{homology-definition-functorial-injective-embedding}.
\end{theorem}

\begin{proof}
Let $\mathcal{G}_i$, $i \in I$ be a set of abelian
sheaves such that every subsheaf of every $\mathbf{Z}_X^\#$
occurs as one of the $\mathcal{G}_i$. Apply
Lemma \ref{lemma-map-into-smaller} to this collection to
get an ordinal $\beta$. We claim that for any sheaf of abelian
groups $\mathcal{F}$ the map $\mathcal{F} \to J_\beta(\mathcal{F})$
is an injection of $\mathcal{F}$ into an injective.
Note that by construction the assignment
$\mathcal{F} \mapsto \big(\mathcal{F} \to J_\beta(\mathcal{F})\big)$
is indeed functorial.

\medskip\noindent
The proof of the claim comes from the fact that by
Lemma \ref{lemma-characterize-injectives} it suffices to extend any
morphism $\gamma : \mathcal{G} \to J_\beta(\mathcal{F})$
from a subsheaf $\mathcal{G}$ of some $\mathbf{Z}_X^\#$ to all of
$\mathbf{Z}_X^\#$. Then by Lemma \ref{lemma-map-into-smaller} the
map $\gamma$ lifts into $J_\alpha(\mathcal{F})$ for some
$\alpha < \beta$. Finally, we apply Lemma \ref{lemma-map-into-next-one}
to get the desired extension of $\gamma$ to a morphism
into $J_{\alpha + 1}(\mathcal{F}) \to J_\beta(\mathcal{F})$.
\end{proof}






\section{Modules on a ringed site}
\label{section-sheaves-modules}

\noindent
Let $\mathcal{C}$ be a site.
Let $\mathcal{O}$ be a sheaf of rings on $\mathcal{C}$.
By analogy with
More on Algebra, Section \ref{more-algebra-section-injectives-modules}
let us try to prove that there are enough injective
$\mathcal{O}$-modules. First of all, we pick an injective
embedding
$$
\bigoplus\nolimits_{U, \mathcal{I}}
j_{U!}\mathcal{O}_U/\mathcal{I}
\longrightarrow
\mathcal{J}
$$
where $\mathcal{J}$ is an injective abelian sheaf (which
exists by the previous section). Here the direct sum is
over all objects $U$ of $\mathcal{C}$ and over all
$\mathcal{O}$-submodules $\mathcal{I} \subset j_{U!}\mathcal{O}_U$.
Please see
Modules on Sites, Section \ref{sites-modules-section-localize}
to read about the functors restriction and
extension by $0$ for the localization functor
$j_U : \mathcal{C}/U \to \mathcal{C}$.

\medskip\noindent
For any sheaf of $\mathcal{O}$-modules $\mathcal{F}$ denote
$$
\mathcal{F}^\vee
=
\SheafHom(\mathcal{F}, \mathcal{J})
$$
with its natural $\mathcal{O}$-module structure.
Insert here future reference to internal hom.
We will also need
a canonical flat resolution of a sheaf of $\mathcal{O}$-modules.
This we can do as follows: For any $\mathcal{O}$-module
$\mathcal{F}$ we denote
$$
F(\mathcal{F})
=
\bigoplus\nolimits_{U \in \Ob(\mathcal{C}), s \in \mathcal{F}(U)}
j_{U!}\mathcal{O}_U.
$$
This is a flat sheaf of $\mathcal{O}$-modules which comes equipped
with a canonical surjection $F(\mathcal{F}) \to \mathcal{F}$, see
Modules on Sites, Lemma \ref{sites-modules-lemma-module-quotient-flat}.
Moreover the construction $\mathcal{F} \mapsto F(\mathcal{F})$
is functorial in $\mathcal{F}$.

\begin{lemma}
\label{lemma-vee-exact-sheaves}
The functor $\mathcal{F} \mapsto \mathcal{F}^\vee$ is exact.
\end{lemma}

\begin{proof}
This because $\mathcal{J}$ is an injective abelian sheaf.
\end{proof}

\noindent
There is a canonical map $ev : \mathcal{F} \to (\mathcal{F}^\vee)^\vee$
given by evaluation: given $x \in \mathcal{F}(U)$ we let
$ev(x) \in (\mathcal{F}^\vee)^\vee =
\SheafHom(\mathcal{F}^\vee, \mathcal{J})$
be the map $\varphi \mapsto \varphi(x)$.

\begin{lemma}
\label{lemma-ev-injective-sheaves}
For any $\mathcal{O}$-module $\mathcal{F}$ the evaluation map
$ev : \mathcal{F} \to (\mathcal{F}^\vee)^\vee$ is injective.
\end{lemma}

\begin{proof}
You can check this using the definition of $\mathcal{J}$.
Namely, if $s \in \mathcal{F}(U)$ is not zero, then let
$j_{U!}\mathcal{O}_U \to \mathcal{F}$ be the map of
$\mathcal{O}$-modules it corresponds to via adjunction.
Let $\mathcal{I}$ be the kernel of this map. There exists
a nonzero map $\mathcal{F} \supset j_{U!}\mathcal{O}_U/\mathcal{I}
\to \mathcal{J}$ which does not annihilate $s$. As $\mathcal{J}$ is
an injective $\mathcal{O}$-module, this extends to a map
$\varphi : \mathcal{F} \to \mathcal{J}$.
Then $ev(s)(\varphi) = \varphi(s) \not = 0$ which is what we had to prove.
\end{proof}

\noindent
The canonical surjection
$F(\mathcal{F}) \to \mathcal{F}$ of $\mathcal{O}$-modules turns into a
canonical injection, see above, of $\mathcal{O}$-modules
$$
(\mathcal{F}^\vee)^\vee \longrightarrow (F(\mathcal{F}^\vee))^\vee.
$$
Set $J(\mathcal{F}) = (F(\mathcal{F}^\vee))^\vee$.
The composition of $ev$ with this
the displayed map gives
$\mathcal{F} \to J(\mathcal{F})$ functorially in $\mathcal{F}$.

\begin{lemma}
\label{lemma-JM-injective-sheaves}
Let $\mathcal{O}$ be a sheaf of rings.
For every $\mathcal{O}$-module $\mathcal{F}$ the
$\mathcal{O}$-module $J(\mathcal{F})$ is injective.
\end{lemma}

\begin{proof}
We have to show that the functor
$\Hom_\mathcal{O}(\mathcal{G}, J(\mathcal{F}))$
is exact. Note that
\begin{eqnarray*}
\Hom_\mathcal{O}(\mathcal{G}, J(\mathcal{F}))
& = &
\Hom_\mathcal{O}(\mathcal{G}, (F(\mathcal{F}^\vee))^\vee) \\
& = &
\Hom_\mathcal{O}
(\mathcal{G}, \SheafHom(F(\mathcal{F}^\vee), \mathcal{J})) \\
& = &
\Hom(\mathcal{G} \otimes_\mathcal{O} F(\mathcal{F}^\vee), \mathcal{J})
\end{eqnarray*}
Thus what we want follows from the fact that $F(\mathcal{F}^\vee)$
is flat and $\mathcal{J}$ is injective.
\end{proof}

\begin{theorem}
\label{theorem-sheaves-modules-injectives}
Let $\mathcal{C}$ be a site.
Let $\mathcal{O}$ be a sheaf of rings on $\mathcal{C}$.
The category of sheaves of $\mathcal{O}$-modules on a
site has enough injectives. In fact there exists
a functorial injective embedding, see
Homology, Definition \ref{homology-definition-functorial-injective-embedding}.
\end{theorem}

\begin{proof}
From the discussion in this section.
\end{proof}

\begin{proposition}
\label{proposition-presheaves-modules}
Let $\mathcal{C}$ be a category.
Let $\mathcal{O}$ be a presheaf of rings on $\mathcal{C}$.
The category $\textit{PMod}(\mathcal{O})$ of presheaves of
$\mathcal{O}$-modules has functorial injective embeddings.
\end{proposition}

\begin{proof}
We could prove this along the lines of the discussion in
Section \ref{section-injectives-presheaves}. But instead we argue using the
theorem above. Endow $\mathcal{C}$ with the structure of a site by letting the
set of coverings of an object $U$ consist of all singletons $\{f : V \to U\}$
where $f$ is an isomorphism. We omit the verification that this defines a site.
A sheaf for this topology is the same as a presheaf (proof omitted). Hence the
theorem applies.
\end{proof}








\section{Embedding abelian categories}
\label{section-embedding}

\noindent
In this section we show that an abelian category embeds in the
category of abelian sheaves on a site having enough points.
The site will be the one described in the following lemma.

\begin{lemma}
\label{lemma-site-abelian-category}
Let $\mathcal{A}$ be an abelian category.
Let
$$
\text{Cov} = \{\{f : V \to U\} \mid f\text{ is surjective}\}.
$$
Then $(\mathcal{A}, \text{Cov})$ is a site, see
Sites, Definition \ref{sites-definition-site}.
\end{lemma}

\begin{proof}
Note that $\Ob(\mathcal{A})$ is a set by our conventions
about categories. An isomorphism is a surjective morphism.
The composition of surjective morphisms is surjective.
And the base change of a surjective morphism in $\mathcal{A}$
is surjective, see
Homology, Lemma \ref{homology-lemma-epimorphism-universal-abelian-category}.
\end{proof}

\noindent
Let $\mathcal{A}$ be a pre-additive category. In this case the
Yoneda embedding $\mathcal{A} \to \textit{PSh}(\mathcal{A})$, $X \mapsto h_X$
factors through a functor $\mathcal{A} \to \textit{PAb}(\mathcal{A})$.

\begin{lemma}
\label{lemma-embedding}
Let $\mathcal{A}$ be an abelian category.
Let $\mathcal{C} = (\mathcal{A}, \text{Cov})$ be the
site defined in
Lemma \ref{lemma-site-abelian-category}.
Then $X \mapsto h_X$ defines a fully faithful, exact functor
$$
\mathcal{A} \longrightarrow \textit{Ab}(\mathcal{C}).
$$
Moreover, the site $\mathcal{C}$ has enough points.
\end{lemma}

\begin{proof}
Suppose that $f : V \to U$ is a surjective morphism of $\mathcal{A}$.
Let $K = \Ker(f)$. Recall that
$V \times_U V = \Ker((f, -f) : V \oplus V \to U)$, see
Homology, Example \ref{homology-example-fibre-product-pushouts}.
In particular there exists an injection $K \oplus K \to V \times_U V$.
Let $p, q : V \times_U V \to V$ be the two projection morphisms.
Note that $p - q : V \times_U V \to V$ is a morphism such that
$f \circ (p  - q) = 0$. Hence $p - q$ factors through $K \to V$.
Let us denote this morphism by $c : V \times_U V \to K$.
And since the composition $K \oplus K \to V \times_U V \to K$
is surjective, we conclude that $c$ is surjective. It follows that
$$
V \times_U V \xrightarrow{p - q} V \to U \to 0
$$
is an exact sequence of $\mathcal{A}$.
Hence for an object $X$ of $\mathcal{A}$ the sequence
$$
0 \to
\Hom_\mathcal{A}(U, X) \to
\Hom_\mathcal{A}(V, X) \to
\Hom_\mathcal{A}(V \times_U V, X)
$$
is an exact sequence of abelian groups, see
Homology, Lemma \ref{homology-lemma-check-exactness}.
This means that $h_X$ satisfies the sheaf condition
on $\mathcal{C}$.

\medskip\noindent
The functor is fully faithful by
Categories, Lemma \ref{categories-lemma-yoneda}.
The functor is a left exact functor between abelian categories by
Homology, Lemma \ref{homology-lemma-check-exactness}.
To show that it is right exact, let $X \to Y$ be a surjective morphism
of $\mathcal{A}$. Let $U$ be an object of $\mathcal{A}$, and let
$s \in h_Y(U) = \Mor_\mathcal{A}(U, Y)$ be a section of $h_Y$
over $U$. By
Homology, Lemma \ref{homology-lemma-epimorphism-universal-abelian-category}
the projection $U \times_Y X \to U$ is surjective.
Hence $\{V = U \times_Y X \to U\}$ is a covering of $U$ such that
$s|_V$ lifts to a section of $h_X$. This proves that
$h_X \to h_Y$ is a surjection of abelian sheaves, see
Sites, Lemma \ref{sites-lemma-mono-epi-sheaves}.

\medskip\noindent
The site $\mathcal{C}$ has enough points by
Sites, Proposition \ref{sites-proposition-criterion-points}.
\end{proof}

\begin{remark}
\label{remark-embedding}
The Freyd-Mitchell embedding theorem says there exists a fully faithful
exact functor from any abelian category $\mathcal{A}$
to the category of modules over a ring.
Lemma \ref{lemma-embedding}
is not quite as strong. But the result is suitable for the
Stacks project as we have to understand sheaves of abelian groups on
sites in detail anyway. Moreover, ``diagram chasing'' works in the category
of abelian sheaves on $\mathcal{C}$, for example by working with sections over
objects, or by working on the level of stalks using that $\mathcal{C}$ has
enough points. To see how to deduce the Freyd-Mitchell embedding theorem from
Lemma \ref{lemma-embedding}
see
Remark \ref{remark-embedding-freyd}.
\end{remark}

\begin{remark}
\label{remark-embedding-big}
If $\mathcal{A}$ is a ``big'' abelian category, i.e., if $\mathcal{A}$
has a class of objects, then
Lemma \ref{lemma-embedding}
does not work. In this case, given any set of objects
$E \subset \Ob(\mathcal{A})$ there exists an abelian full subcategory
$\mathcal{A}' \subset \mathcal{A}$ such that
$\Ob(\mathcal{A}')$ is a set and $E \subset \Ob(\mathcal{A}')$.
Then one can apply
Lemma \ref{lemma-embedding}
to $\mathcal{A}'$. One can use this to prove that results depending on
a diagram chase hold in $\mathcal{A}$.
\end{remark}

\begin{remark}
\label{remark-embedding-freyd}
Let $\mathcal{C}$ be a site.
Note that $\textit{Ab}(\mathcal{C})$ has enough injectives, see
Theorem \ref{theorem-sheaves-injectives}.
(In the case that $\mathcal{C}$ has enough points this is straightforward
because  $p_*I$ is an injective sheaf if $I$ is an injective
$\mathbf{Z}$-module and $p$ is a point.)
Also, $\textit{Ab}(\mathcal{C})$ has a cogenerator (details omitted).
Hence
Lemma \ref{lemma-embedding}
proves that we have a fully faithful, exact embedding
$\mathcal{A} \to \mathcal{B}$ where $\mathcal{B}$ has a
cogenerator and enough injectives.
We can apply this to $\mathcal{A}^{opp}$ and we get a
fully faithful exact functor
$i : \mathcal{A} \to \mathcal{D} = \mathcal{B}^{opp}$
where $\mathcal{D}$ has enough projectives and a generator. Hence
$\mathcal{D}$ has a projective generator $P$.
Set $R = \Mor_\mathcal{D}(P, P)$. Then
$$
\mathcal{A} \longrightarrow \text{Mod}_R, \quad
X \longmapsto \Hom_\mathcal{D}(P, X).
$$
One can check this is a fully faithful, exact functor.
In other words, one retrieves the
Freyd-Mitchell theorem mentioned in
Remark \ref{remark-embedding}
above.
\end{remark}

\begin{remark}
\label{remark-embed-exact-category}
The arguments proving
Lemmas \ref{lemma-site-abelian-category} and
\ref{lemma-embedding}
work also for {\it exact categories}, see
\cite[Appendix A]{Buhler} and
\cite[1.1.4]{BBD}.
We quickly review this here and we add more details if we ever
need it in the Stacks project.

\medskip\noindent
Let $\mathcal{A}$ be an additive category.
A {\it kernel-cokernel} pair is a pair $(i, p)$
of morphisms of $\mathcal{A}$ with
$i : A \to B$, $p : B \to C$ such that $i$ is the kernel of
$p$ and $p$ is the cokernel of $i$.
Given a set $\mathcal{E}$ of kernel-cokernel pairs we say
$i : A \to B$ is an {\it admissible monomorphism}
if $(i, p) \in \mathcal{E}$ for some morphism $p$.
Similarly we say a morphism $p : B \to C$ is an {\it admissible epimorphism}
if $(i, p) \in \mathcal{E}$ for some morphism $i$.
The pair $(\mathcal{A}, \mathcal{E})$ is said to be an
{\it exact category} if the following axioms hold
\begin{enumerate}
\item $\mathcal{E}$ is closed under isomorphisms of kernel-cokernel
pairs,
\item for any object $A$ the morphism $1_A$ is both an admissible epimorphism
and an admissible monomorphism,
\item admissible monomorphisms are stable under composition,
\item admissible epimorphisms are stable under composition,
\item the push-out of an admissible monomorphism $i : A \to B$ via
any morphism $A \to A'$ exist and the induced morphism $i' : A' \to B'$
is an admissible monomorphism, and
\item the base change of an admissible epimorphism $p : B \to C$ via
any morphism $C' \to C$ exist and the induced morphism $p' : B' \to C'$
is an admissible epimorphism.
\end{enumerate}
Given such a structure let $\mathcal{C} = (\mathcal{A}, \text{Cov})$
where coverings (i.e., elements of $\text{Cov}$) are given by
admissible epimorphisms. The axioms listed above
immediately imply that this is a site. Consider the functor
$$
F : \mathcal{A} \longrightarrow \textit{Ab}(\mathcal{C}), \quad
X \longmapsto h_X
$$
exactly as in
Lemma \ref{lemma-embedding}.
It turns out that this functor is fully faithful, exact, and reflects
exactness. Moreover, any extension of objects in the essential image
of $F$ is in the essential image of $F$.
\end{remark}






\section{Grothendieck's AB conditions}
\label{section-grothendieck-conditions}

\noindent
This and the next few sections are mostly interesting for ``big'' abelian
categories, i.e., those categories listed in
Categories, Remark \ref{categories-remark-big-categories}.
A good case to keep in mind is the category of sheaves of modules
on a ringed site.

\medskip\noindent
Grothendieck proved the existence of injectives in great generality
in the paper \cite{Tohoku}. He used the following conditions to single
out abelian categories with special properties.

\begin{definition}
\label{definition-grothendieck-conditions}
Let $\mathcal{A}$ be an abelian category. We name some conditions
\begin{enumerate}
\item[AB3] $\mathcal{A}$ has direct sums,
\item[AB4] $\mathcal{A}$ has AB3 and direct sums are exact,
\item[AB5] $\mathcal{A}$ has AB3 and filtered colimits are exact.
\end{enumerate}
Here are the dual notions
\begin{enumerate}
\item[AB3*] $\mathcal{A}$ has products,
\item[AB4*] $\mathcal{A}$ has AB3* and products are exact,
\item[AB5*] $\mathcal{A}$ has AB3* and cofiltered limits are exact.
\end{enumerate}
We say an object $U$ of $\mathcal{A}$ is a {\it generator} if
for every $N \subset M$, $N \not = M$ in $\mathcal{A}$ there exists a morphism
$U \to M$ which does not factor through $N$.
We say $\mathcal{A}$ is a {\it Grothendieck abelian category} if
it has AB5 and a generator.
\end{definition}

\noindent
Discussion: A direct sum in an abelian category is a coproduct.
If an abelian category has direct sums (i.e., AB3), then it has colimits, see
Categories, Lemma \ref{categories-lemma-colimits-coproducts-coequalizers}. 
Similarly if $\mathcal{A}$ has AB3* then it has limits, see
Categories, Lemma \ref{categories-lemma-limits-products-equalizers}. 
Exactness of direct sums means the following: given an index set $I$
and short exact sequences
$$
0 \to A_i \to B_i \to C_i \to 0,\quad i \in I
$$
in $\mathcal{A}$ then the sequence
$$
0 \to
\bigoplus\nolimits_{i \in I} A_i \to
\bigoplus\nolimits_{i \in I} B_i \to
\bigoplus\nolimits_{i \in I} C_i \to 0
$$
is exact as well. Without assuming AB4 it is only true in general that
the sequence is exact on the right (i.e., taking direct sums is a right
exact functor if direct sums exist). Similarly, exactness of filtered
colimits means the following: given a directed set $I$
and a system of short exact sequences
$$
0 \to A_i \to B_i \to C_i \to 0
$$
over $I$ in $\mathcal{A}$ then the sequence
$$
0 \to
\colim_{i \in I} A_i \to
\colim_{i \in I} B_i \to
\colim_{i \in I} C_i \to 0
$$
is exact as well. Without assuming AB5 it is only true in general that
the sequence is exact on the right (i.e., taking colimits is a right
exact functor if colimits exist). A similar explanation holds for
AB4* and AB5*.



\section{Injectives in Grothendieck categories}
\label{section-grothendieck-categories}

\noindent
The existence of a generator implies that given an object $M$ of a
Grothendieck abelian category $\mathcal{A}$ there is a set of subobjects.
(This may not be true for a general ``big'' abelian category.)

\begin{lemma}
\label{lemma-set-of-subobjects}
Let $\mathcal{A}$ be an abelian category with a generator $U$ and
$X$ and object of $\mathcal{A}$. If $\kappa$ is the cardinality of
$\Mor(U, X)$ then
\begin{enumerate}
\item There does not exist a strictly increasing
(or strictly decreasing) chain of subobjects
of $X$ indexed by a cardinal bigger than $\kappa$.
\item If $\alpha$ is an ordinal of cofinality $> \kappa$
then any increasing (or decreasing) sequence of subobjects
of $X$ indexed by $\alpha$ is eventually constant.
\item The cardinality of the set of subobjects of $X$
is $\leq 2^\kappa$.
\end{enumerate}
\end{lemma}

\begin{proof}
For (1) assume $\kappa' > \kappa$ is a cardinal and assume
$X_i$, $i \in \kappa'$ is strictly increasing. Then take for
each $i$ a $\phi_i \in \Mor(U, X)$ such that $\phi_i$ factors through
$X_{i + 1}$ but not through $X_i$. Then the morphisms $\phi_i$
are distinct, which contradicts the definition of $\kappa$.

\medskip\noindent
Part (2) follows from the definition of cofinality and (1).

\medskip\noindent
Proof of (3). For any subobject $Y \subset X$
define $S_Y \in \mathcal{P}(\Mor(U, X))$ (power set) as
$S_Y = \{\phi \in \Mor(U,X) : \phi)\text{ factors through }Y\}$.
Then $Y = Y'$ if and only if $S_Y = S_{Y'}$. Hence the cardinality
of the set of subobjects is at most the cardinality of this power set.
\end{proof}

\noindent
By Lemma \ref{lemma-set-of-subobjects} the following definition makes sense.

\begin{definition}
\label{definition-size}
Let $\mathcal{A}$ be a Grothendieck abelian category.
Let $M$ be an object of $\mathcal{A}$.
The {\it size} $|M|$ of $M$ is the cardinality of the set of subobjects
of $M$.
\end{definition}

\begin{lemma}
\label{lemma-size-goes-down}
Let $\mathcal{A}$ be a Grothendieck abelian category.
If $0 \to M' \to M \to M'' \to 0$ is a short exact sequence of
$\mathcal{A}$, then $|M'|, |M''| \leq |M|$.
\end{lemma}

\begin{proof}
Immediate from the definitions.
\end{proof}

\begin{lemma}
\label{lemma-set-iso-classes-bounded-size}
Let $\mathcal{A}$ be a Grothendieck abelian category with generator $U$.
\begin{enumerate}
\item If $|M| \leq \kappa$, then $M$ is the quotient of a direct
sum of at most $\kappa$ copies of $U$.
\item For every cardinal $\kappa$ the isomorphism classes
of objects $M$ with $|M| \leq \kappa$ form a set.
\end{enumerate}
\end{lemma}

\begin{proof}
For (1) choose for every proper subobject $M' \subset M$ a morphism
$\varphi_{M'} : U \to M$ whose image is not contained in $M'$. Then
$\bigoplus_{M' \subset M} \varphi_{M'} : \bigoplus_{M' \subset M} U \to M$
is surjective. It is clear that (1) implies (2).
\end{proof}

\begin{proposition}
\label{proposition-objects-are-small}
Let $\mathcal{A}$ be a Grothendieck abelian category. Let $M$ be an
object of $\mathcal{A}$. Let $\kappa = |M|$.
If $\alpha$ is an ordinal whose cofinality is bigger than $\kappa$,
then $M$ is $\alpha$-small with respect to injections.
\end{proposition}

\begin{proof}
Please compare with Proposition \ref{proposition-modules-are-small}.
We need only show that the map (\ref{equation-compare}) is a surjection.
Let $f : M \to \colim B_\beta$ be a map.
Consider the subobjects $\{f^{-1}(B_\beta)\}$ of $M$, where $B_\beta$
is considered as a subobject of the colimit $B = \bigcup_\beta B_\beta$.
If one of these, say $f^{-1}(B_\beta)$, fills $M$,
then the map factors through $B_\beta$.

\medskip\noindent
So suppose to the contrary that all of the $f^{-1}(B_\beta)$ were proper
subobjects of $M$. However, because $\mathcal{A}$ has
AB5 we have
$$
\colim f^{-1}(B_\beta) = f^{-1}\left(\colim B_\beta\right) = M.
$$
Now there are at most $\kappa$ different subobjects of $M$ that occur among
the $f^{-1}(B_\alpha)$, by hypothesis.
Thus we can find a subset $S \subset \alpha$ of cardinality at most
$\kappa$ such that as $\beta'$ ranges over $S$, the
$f^{-1}(B_{\beta'})$ range over \emph{all} the $f^{-1}(B_\alpha)$.

\medskip\noindent
However, $S$ has an upper bound $\widetilde{\alpha} < \alpha$ as
$\alpha$ has cofinality bigger than $\kappa$. In particular, all the
$f^{-1}(B_{\beta'})$, $\beta' \in S$ are contained in
$f^{-1}(B_{\widetilde{\alpha}})$.
It follows that $f^{-1}(B_{\widetilde{\alpha}}) = M$.
In particular, the map $f$ factors through $B_{\widetilde{\alpha}}$.
\end{proof}

\begin{lemma}
\label{lemma-characterize-injective}
\begin{slogan}
To check that an object is injective, one only needs to check that lifting
holds for subobjects of a generator.
\end{slogan}
Let $\mathcal{A}$ be a Grothendieck abelian category with generator $U$.
An object $I$ of $\mathcal{A}$ is injective if and only if in every
commutative diagram
$$
\xymatrix{
M \ar[d] \ar[r] &  I \\
U \ar@{-->}[ru]
}
$$
for $M \subset U$ a subobject, the dotted arrow exists.
\end{lemma}

\begin{proof}
Please see Lemma \ref{lemma-criterion-baer} for the case of modules.
Choose an injection $A \subset B$ and a morphism $\varphi : A \to I$.
Consider the set $S$ of pairs $(A', \varphi')$ consisting of
subobjects $A \subset A' \subset B$ and a morphism $\varphi' : A' \to I$
extending $\varphi$. Define a partial ordering on this set in the obvious
manner. Choose a totally ordered subset $T \subset S$. Then
$$
A' = \colim_{t \in T} A_t \xrightarrow{\colim_{t \in T} \varphi_t} I
$$
is an upper bound. Hence by Zorn's lemma the set $S$ has a maximal element
$(A', \varphi')$. We claim that $A' = B$. If not, then choose a morphism
$\psi : U \to B$ which does not factor through $A'$. Set
$N = A' \cap \psi(U)$. Set $M = \psi^{-1}(N)$. Then the map
$$
M \to N \to A' \xrightarrow{\varphi'} I
$$
can be extended to a morphism $\chi : U \to I$. Since
$\chi|_{\Ker(\psi)} = 0$ we see that $\chi$ factors as
$$
U \to \Im(\psi) \xrightarrow{\varphi''} I
$$
Since $\varphi'$ and $\varphi''$ agree on $N = A' \cap \Im(\psi)$
we see that combined the define a morphism $A' + \Im(\psi) \to I$
contradicting the assumed maximality of $A'$.
\end{proof}

\begin{theorem}
\label{theorem-injective-embedding-grothendieck}
Let $\mathcal{A}$ be a Grothendieck abelian category.
Then $\mathcal{A}$ has functorial injective embeddings.
\end{theorem}

\begin{proof}
Please compare with the proof of
Theorem \ref{theorem-baer-grothendieck}.
Choose a generator $U$ of $\mathcal{A}$. For an object $M$ we define
$\mathbf{M}(M)$ by the following pushout diagram
$$
\xymatrix{
\bigoplus_{N \subset U}
\bigoplus_{\varphi \in \Hom(N, M)}
N \ar[r] \ar[d] & M \ar[d] \\
\bigoplus_{N \subset U}
\bigoplus_{\varphi \in \Hom(N, M)}
U \ar[r] &  \mathbf{M}(M).
}
$$
Note that $M \to \mathbf{M}(N)$ is a functor and that there
exist functorial injective maps $M \to \mathbf{M}(M)$. By transfinite
induction we define functors $\mathbf{M}_\alpha(M)$ for every
ordinal $\alpha$. Namely, set $\mathbf{M}_0(M) = M$. Given
$\mathbf{M}_\alpha(M)$ set
$\mathbf{M}_{\alpha + 1}(M) = \mathbf{M}(\mathbf{M}_\alpha(M))$.
For a limit ordinal $\beta$ set
$$
\mathbf{M}_\beta(M) = \colim_{\alpha < \beta} \mathbf{M}_\alpha(M).
$$
Finally, pick any ordinal $\alpha$ whose cofinality is greater than $|U|$.
Such an ordinal exists by
Sets, Proposition \ref{sets-proposition-exist-ordinals-large-cofinality}.
We claim that $M \to \mathbf{M}_\alpha(M)$ is the desired functorial
injective embedding. Namely, if $N \subset U$ is a subobject and
$\varphi : N \to \mathbf{M}_\alpha(M)$ is a morphism, then we see that
$\varphi$ factors through $\mathbf{M}_{\alpha'}(M)$ for some
$\alpha' < \alpha$ by
Proposition \ref{proposition-objects-are-small}.
By construction of $\mathbf{M}(-)$ we see that $\varphi$ extends to
a morphism from $U$ into $\mathbf{M}_{\alpha' + 1}(M)$ and hence into
$\mathbf{M}_\alpha(M)$. By
Lemma \ref{lemma-characterize-injective}
we conclude that $\mathbf{M}_\alpha(M)$ is injective.
\end{proof}













\section{K-injectives in Grothendieck categories}
\label{section-K-injective}

\noindent
The material in this section is taken from the paper \cite{serpe}
authored by Serp\'e. This paper generalizes some of the results
of \cite{Spaltenstein} by Spaltenstein to general Grothendieck abelian
categories. Our Lemma \ref{lemma-characterize-K-injective}
is only implicit in the paper by Serp\'e. Our approach is to mimic
Grothendieck's proof of
Theorem \ref{theorem-injective-embedding-grothendieck}.

\begin{lemma}
\label{lemma-surjection-bounded-size}
Let $\mathcal{A}$ be a Grothendieck abelian category with generator $U$.
Let $c$ be the function on cardinals defined by
$c(\kappa) = |\bigoplus_{\alpha \in \kappa} U|$. If $\pi : M \to N$ is a
surjection then there exists a subobject $M' \subset M$ which surjects
onto $N$ with $|M'| \leq c(|N|)$.
\end{lemma}

\begin{proof}
For every proper subobject $N' \subset N$ choose a morphism
$\varphi_{N'} : U \to M$ such that $U \to M \to N$ does not factor
through $N'$. Set
$$
M' = \Im\left(
\bigoplus\nolimits_{N' \subset N} \varphi_{N'} :
\bigoplus\nolimits_{N' \subset N} U \longrightarrow M\right)
$$
Then $M'$ works.
\end{proof}

\begin{lemma}
\label{lemma-acyclic-quotient-complexes-bounded-size}
Let $\mathcal{A}$ be a Grothendieck abelian category. There exists a cardinal
$\kappa$ such that given any acyclic complex $M^\bullet$ we have
\begin{enumerate}
\item if $M^\bullet$ is nonzero, there is a nonzero subcomplex
$N^\bullet$ which is bounded above, acyclic, and $|N^n| \leq \kappa$,
\item there exists a surjection of complexes
$$
\bigoplus\nolimits_{i \in I} M_i^\bullet \longrightarrow M^\bullet
$$
where $M_i^\bullet$ is bounded above, acyclic, and $|M_i^n| \leq \kappa$.
\end{enumerate}
\end{lemma}

\begin{proof}
Choose a generator $U$ of $\mathcal{A}$. Denote $c$ the function of
Lemma \ref{lemma-surjection-bounded-size}.
Set $\kappa = \sup \{c^n(|U|), n = 1, 2, 3, \ldots\}$.
Let $n \in \mathbf{Z}$ and let $\psi : U \to M^n$ be a morphism.
In order to prove (1) and (2) it suffices to prove there exists a subcomplex
$N^\bullet \subset M^\bullet$ which is bounded above, acyclic, and
$|N^m| \leq \kappa$, such that $\psi$ factors through $N^n$.
To do this set $N^n = \Im(\psi)$, $N^{n + 1} = \Im(U \to M^n \to M^{n + 1})$,
and $N^m = 0$ for $m \geq n + 2$.
Suppose we have constructed $N^m \subset M^m$ for all $m \geq k$ such that
\begin{enumerate}
\item $\text{d}(N^m) \subset N^{m + 1}$, $m \geq k$,
\item $\Im(N^{m - 1} \to N^m) = \Ker(N^m \to N^{m + 1})$ for
all $m \geq k + 1$, and
\item $|N^m| \leq c^{\max\{n - m, 0\}}(|U|)$
\end{enumerate}
for some $k \leq n$. Because $M^\bullet$ is acyclic, we see that the subobject
$\text{d}^{-1}(\Ker(N^k \to N^{k + 1})) \subset M^{k - 1}$ surjects onto
$\Ker(N^k \to N^{k + 1})$. Thus we can choose $N^{k - 1} \subset M^{k - 1}$
surjecting onto $\Ker(N^k \to N^{k + 1})$ with
$|N^{k - 1}| \leq c^{n - k + 1}(|U|)$ by
Lemma \ref{lemma-surjection-bounded-size}. The proof is finished by
induction on $k$.
\end{proof}

\begin{lemma}
\label{lemma-characterize-K-injective}
Let $\mathcal{A}$ be a Grothendieck abelian category.
Let $\kappa$ be a cardinal as in
Lemma \ref{lemma-acyclic-quotient-complexes-bounded-size}.
Suppose that $I^\bullet$ is a complex such that
\begin{enumerate}
\item each $I^j$ is injective, and
\item for every bounded above acyclic complex $M^\bullet$
such that $|M^n| \leq \kappa$
we have $\Hom_{K(\mathcal{A})}(M^\bullet, I^\bullet) = 0$.
\end{enumerate}
Then $I^\bullet$ is an $K$-injective complex.
\end{lemma}

\begin{proof}
Let $M^\bullet$ be an acyclic complex. We are going to construct by
induction on the ordinal $\alpha$ an acyclic subcomplex
$K_\alpha^\bullet \subset M^\bullet$ as follows.
For $\alpha = 0$ we set $K_0^\bullet = 0$. For $\alpha > 0$
we proceed as follows:
\begin{enumerate}
\item If $\alpha = \beta + 1$ and $K_\beta^\bullet = M^\bullet$
then we choose $K_\alpha^\bullet = K_\beta^\bullet$.
\item If $\alpha = \beta + 1$ and $K_\beta^\bullet \not = M^\bullet$
then $M^\bullet/K_\beta^\bullet$ is a nonzero acyclic complex.
We choose a subcomplex $N_\alpha^\bullet \subset M^\bullet/K_\beta^\bullet$
as in Lemma \ref{lemma-acyclic-quotient-complexes-bounded-size}.
Finally, we let $K_\alpha^\bullet \subset M^\bullet$
be the inverse image of $N_\alpha^\bullet$.
\item If $\alpha$ is a limit ordinal we set
$K_\beta^\bullet = \colim K_\alpha^\bullet$.
\end{enumerate}
It is clear that $M^\bullet = K_\alpha^\bullet$ for a suitably large
ordinal $\alpha$. We will prove that
$$
\Hom_{K(\mathcal{A})}(K_\alpha^\bullet, I^\bullet)
$$
is zero by transfinite induction on $\alpha$. It holds for $\alpha = 0$
since $K_0^\bullet$ is zero. Suppose it holds for $\beta$ and
$\alpha = \beta + 1$. In case (1) of the list above the result is clear.
In case (2) there is a short exact sequence of complexes
$$
0 \to K_\beta^\bullet \to K_\alpha^\bullet \to N_\alpha^\bullet \to 0
$$
Since each component of $I^\bullet$ is injective we see that we obtain
an exact sequence
$$
\Hom_{K(\mathcal{A})}(K_\beta^\bullet, I^\bullet) \leftarrow
\Hom_{K(\mathcal{A})}(K_\alpha^\bullet, I^\bullet) \leftarrow
\Hom_{K(\mathcal{A})}(N_\alpha^\bullet, I^\bullet)
$$
By induction the term on the left is zero and by assumption on $I^\bullet$
the term on the right is zero. Thus the middle group is zero too.
Finally, suppose that $\alpha$ is a limit ordinal. Then we see that
$$
\Hom^\bullet(K_\alpha^\bullet, I^\bullet) =
\lim_{\beta < \alpha} \Hom^\bullet(K_\beta^\bullet, I^\bullet)
$$
with notation as in
More on Algebra, Section \ref{more-algebra-section-hom-complexes}.
These complexes compute morphisms in $K(\mathcal{A})$ by
More on Algebra, Equation
(\ref{more-algebra-equation-cohomology-hom-complex}).
Note that the transition maps in the system are surjective
because $I^j$ is injective for each $j$. Moreover, for a limit
ordinal $\alpha$ we have equality of limit and value
(see displayed formula above). Thus we may apply
Homology, Lemma \ref{homology-lemma-ML-over-ordinals}
to conclude.
\end{proof}

\begin{lemma}
\label{lemma-functorial-homotopies}
Let $\mathcal{A}$ be a Grothendieck abelian category.
Let $(K_i^\bullet)_{i \in I}$ be a set of acyclic complexes.
There exists a functor $M^\bullet \mapsto \mathbf{M}^\bullet(M^\bullet)$
and a natural transformation
$j_{M^\bullet} : M^\bullet \to \mathbf{M}^\bullet(M^\bullet)$
such
\begin{enumerate}
\item $j_{M^\bullet}$ is a (termwise) injective quasi-isomorphism, and
\item for every $i \in I$ and $w : K_i^\bullet \to M^\bullet$
the morphism $j_{M^\bullet} \circ w$ is homotopic to zero.
\end{enumerate}
\end{lemma}

\begin{proof}
For every $i \in I$ choose a (termwise) injective map of complexes
$K_i^\bullet \to L_i^\bullet$ which is homotopic to zero with
$L_i^\bullet$ quasi-isomorphic to zero. For example, take $L_i^\bullet$
to be the cone on the identity of $K_i^\bullet$.
We define $\mathbf{M}^\bullet(M^\bullet)$ by the following pushout diagram
$$
\xymatrix{
\bigoplus_{i \in I}
\bigoplus_{w : K_i^\bullet \to M^\bullet}
K_i^\bullet \ar[r] \ar[d] & M^\bullet \ar[d] \\
\bigoplus_{i \in I}
\bigoplus_{w : K_i^\bullet \to M^\bullet}
L_i^\bullet \ar[r] &  \mathbf{M}^\bullet(M^\bullet).
}
$$
Then $M^\bullet \to \mathbf{M}^\bullet(M^\bullet)$ is a functor. The right
vertical arrow defines the functorial injective map $j_{M^\bullet}$.
The cokernel of $j_{M^\bullet}$ is isomorphic to the direct sum of
the cokernels of the maps $K_i^\bullet \to L_i^\bullet$ hence acyclic.
Thus $j_{M^\bullet}$ is a quasi-isomorphism. Part (2) holds by construction.
\end{proof}

\begin{lemma}
\label{lemma-functorial-injective}
Let $\mathcal{A}$ be a Grothendieck abelian category.
There exists a functor $M^\bullet \mapsto \mathbf{N}^\bullet(M^\bullet)$
and a natural transformation
$j_{M^\bullet} : M^\bullet \to \mathbf{N}^\bullet(M^\bullet)$
such
\begin{enumerate}
\item $j_{M^\bullet}$ is a (termwise) injective quasi-isomorphism, and
\item for every $n \in \mathbf{Z}$ the map $M^n \to \mathbf{N}^n(M^\bullet)$
factors through a subobject $I^n \subset \mathbf{N}^n(M^\bullet)$ where $I^n$
is an injective object of $\mathcal{A}$.
\end{enumerate}
\end{lemma}

\begin{proof}
Choose a functorial injective embeddings $i_M : M \to I(M)$, see
Theorem \ref{theorem-injective-embedding-grothendieck}.
For every complex $M^\bullet$ denote $J^\bullet(M^\bullet)$ the complex
with terms $J^n(M^\bullet) = I(M^n) \oplus I(M^{n + 1})$ and differential
$$
d_{J^\bullet(M^\bullet)} =
\left(
\begin{matrix}
0 & 1 \\
0 & 0
\end{matrix}
\right)
$$
There exists a canonical injective map of complexes
$u_{M^\bullet} : M^\bullet \to J^\bullet(M^\bullet)$ by mapping $M^n$ to
$I(M^n) \oplus I(M^{n + 1})$ via the maps $i_{M^n} : M^n \to I(M^n)$ and
$i_{M^{n + 1}} \circ d : M^n \to M^{n + 1} \to I(M^{n + 1})$. Hence a
short exact sequence of complexes
$$
0 \to M^\bullet \xrightarrow{u_{M^\bullet}}
J^\bullet(M^\bullet) \xrightarrow{v_{M^\bullet}}
Q^\bullet(M^\bullet) \to 0
$$
functorial in $M^\bullet$. Set
$$
\mathbf{N}^\bullet(M^\bullet) = C(v_{M^\bullet})^\bullet[-1].
$$
Note that
$$
\mathbf{N}^n(M^\bullet) = Q^{n - 1}(M^\bullet) \oplus J^n(M^\bullet)
$$
with differential
$$
\left(
\begin{matrix}
- d^{n - 1}_{Q^\bullet(M^\bullet)} & - v^n_{M^\bullet} \\
0 & d^n_{J^\bullet(M)}
\end{matrix}
\right)
$$
Hence we see that there is a map of complexes
$j_{M^\bullet} : M^\bullet \to \mathbf{N}^\bullet(M^\bullet)$
induced by $u$. It is injective and factors through an injective subobject
by construction. The map $j_{M^\bullet}$ is a quasi-isomorphism as one
can prove by looking at the long exact sequence of cohomology associated
to the short exact sequences of complexes above.
\end{proof}

\begin{theorem}
\label{theorem-K-injective-embedding-grothendieck}
\begin{slogan}
Existence of K-injective complexes for Grothendieck abelian categories.
\end{slogan}
Let $\mathcal{A}$ be a Grothendieck abelian category.
For every complex $M^\bullet$ there exists a quasi-isomorphism
$M^\bullet \to I^\bullet$ such that $M^n \to I^n$ is injective and $I^n$
is an injective object of $\mathcal{A}$ for all $n$ and $I^\bullet$
is a K-injective complex. Moreover, the construction is functorial in
$M^\bullet$.
\end{theorem}

\begin{proof}
Please compare with the proof of
Theorem \ref{theorem-baer-grothendieck}
and
Theorem \ref{theorem-injective-embedding-grothendieck}.
Choose a cardinal $\kappa$ as in
Lemmas \ref{lemma-acyclic-quotient-complexes-bounded-size} and
\ref{lemma-characterize-K-injective}.
Choose a set $(K_i^\bullet)_{i \in I}$
of bounded above, acyclic complexes
such that every bounded above acyclic complex $K^\bullet$
such that $|K^n| \leq \kappa$ is isomorphic to $K_i^\bullet$ for some
$i \in I$. This is possible by
Lemma \ref{lemma-set-iso-classes-bounded-size}.
Denote $\mathbf{M}^\bullet(-)$ the functor constructed in
Lemma \ref{lemma-functorial-homotopies}.
Denote $\mathbf{N}^\bullet(-)$ the functor constructed in
Lemma \ref{lemma-functorial-injective}.
Both of these functors come with injective transformations
$\text{id} \to \mathbf{M}$ and $\text{id} \to \mathbf{N}$.

\medskip\noindent
Using transfinite recursion we define a sequence of functors
$\mathbf{T}_\alpha(-)$ and corresponding transformations
$\text{id} \to \mathbf{T}_\alpha$. Namely we set
$\mathbf{T}_0(M^\bullet) = M^\bullet$. If $\mathbf{T}_\alpha$ is
given then we set
$$
\mathbf{T}_{\alpha + 1}(M^\bullet) =
\mathbf{N}^\bullet(\mathbf{M}^\bullet(\mathbf{T}_\alpha(M^\bullet)))
$$
If $\beta$ is a limit ordinal we set
$$
\mathbf{T}_\beta(M^\bullet) =
\colim_{\alpha < \beta} \mathbf{T}_\alpha(M^\bullet)
$$
The transition maps of the system are injective quasi-isomorphisms.
By AB5 we see that the colimit is still quasi-isomorphic to $M^\bullet$.
We claim that $M^\bullet \to \mathbf{T}_\alpha(M^\bullet)$
does the job if the cofinality of $\alpha$ is larger than
$\max(\kappa, |U|)$ where $U$ is a generator of $\mathcal{A}$.
Namely, it suffices to check conditions (1) and (2) of
Lemma \ref{lemma-characterize-K-injective}.

\medskip\noindent
For (1) we use the criterion of
Lemma \ref{lemma-characterize-injective}.
Suppose that $M \subset U$ and $\varphi : M \to \mathbf{T}^n_\alpha(M^\bullet)$
is a morphism for some $n \in \mathbf{Z}$. By
Proposition \ref{proposition-objects-are-small}
we see that $\varphi$ factor through
$\mathbf{T}^n_{\alpha'}(M^\bullet)$ for some $\alpha' < \alpha$.
In particular, by the construction of the functor
$\mathbf{N}^\bullet(-)$ we see that $\varphi$ factors through
an injective object of $\mathcal{A}$ which shows that $\varphi$
lifts to a morphism on $U$.

\medskip\noindent
For (2) let $w : K^\bullet  \to \mathbf{T}_\alpha(M^\bullet)$
be a morphism of complexes where $K^\bullet$ is a bounded above acyclic
complex such that $|K^n| \leq \kappa$. Then $K^\bullet \cong K_i^\bullet$
for some $i \in I$. Moreover, by
Proposition \ref{proposition-objects-are-small}
once again we see that $w$ factor through
$\mathbf{T}^n_{\alpha'}(M^\bullet)$ for some $\alpha' < \alpha$.
In particular, by the construction of the functor
$\mathbf{M}^\bullet(-)$ we see that $w$ is homotopic to zero.
This finishes the proof.
\end{proof}







\section{Additional remarks on Grothendieck abelian categories}
\label{section-additional-Grothendieck}

\noindent
In this section we put some results on Grothendieck abelian categories
which are folklore.

\begin{lemma}
\label{lemma-grothendieck-brown}
Let $\mathcal{A}$ be a Grothendieck abelian category.
Let $F : \mathcal{A}^{opp} \to \textit{Sets}$ be a functor.
Then $F$ is representable if and only if $F$ commutes with colimits, i.e.,
$$
F(\colim_i N_i) = \lim F(N_i)
$$
for any diagram $\mathcal{I} \to \mathcal{A}$, $i \in \mathcal{I}$.
\end{lemma}

\begin{proof}
If $F$ is representable, then it commutes with colimits by definition
of colimits.

\medskip\noindent
Assume that $F$ commutes with colimits. Then $F(M \oplus N) = F(M) \times F(N)$
and we can use this to define a group structure on $F(M)$. Hence we get
$F : \mathcal{A} \to \textit{Ab}$ which is additive and right exact, i.e.,
transforms a short exact sequence $0 \to K \to L \to M \to 0$ into an exact
sequence $F(K) \leftarrow F(L) \leftarrow F(M) \leftarrow 0$ (compare with
Homology, Section \ref{homology-section-functors}).

\medskip\noindent
Let $U$ be a generator for $\mathcal{A}$. Set $A = \bigoplus_{s \in F(U)} U$.
Let $s_{univ} = (s)_{s \in F(U)} \in F(A) = \prod_{s \in F(U)} F(U)$. Let
$A' \subset A$ be the largest subobject such that $s_{univ}$ restricts to
zero on $A'$. This exists because $\mathcal{A}$ is a Grothendieck category
and because $F$ commutes with colimits. Because $F$ commutes with colimits
there exists a unique element $\overline{s}_{univ} \in F(A/A')$ which
maps to $s_{univ}$ in $F(A)$. We claim that $A/A'$ represents $F$, in
other words, the Yoneda map
$$
\overline{s}_{univ} : h_{A/A'} \longrightarrow F
$$
is an isomorphism. Let $M \in \Ob(\mathcal{A})$ and $s \in F(M)$. Consider
the surjection
$$
c_M :
A_M = \bigoplus\nolimits_{\varphi \in \Hom_\mathcal{A}(U, M)} U
\longrightarrow
M.
$$
This gives $F(c_M)(s) = (s_\varphi) \in \prod_\varphi F(U)$.
Consider the map
$$
\psi :
A_M = \bigoplus\nolimits_{\varphi \in \Hom_\mathcal{A}(U, M)} U
\longrightarrow
\bigoplus\nolimits_{s \in F(U)} U = A
$$
which maps the summand corresponding to $\varphi$ to the summand
corresponding to $s_\varphi$ by the identity map on $U$. Then $s_{univ}$
maps to $(s_\varphi)_\varphi$ by construction.
in other words the right square in the diagram
$$
\xymatrix{
A' \ar[r] &
A \ar@{..>}[r]_{s_{univ}} & F \\
K \ar[r] \ar[u]^{?} & A_M \ar[u]^\psi \ar[r] &
M \ar@{..>}[u]_s
}
$$
commutes. Let $K = \Ker(A_M \to M)$. Since $s$ restricts to zero
on $K$ we see that $\psi(K) \subset A'$ by definition of $A'$. Hence there
is an induced morphism $M \to A/A'$. This construction gives an inverse
to the map $h_{A/A'}(M) \to F(M)$ (details omitted).
\end{proof}

\begin{lemma}
\label{lemma-grothendieck-products}
A Grothendieck abelian category has Ab3*.
\end{lemma}

\begin{proof}
Let $M_i$, $i \in I$ be a family of objects of $\mathcal{A}$ indexed
by a set $I$. The functor $F = \prod_{i \in I} h_{M_i}$
commutes with colimits. Hence
Lemma \ref{lemma-grothendieck-brown}
applies.
\end{proof}

\begin{remark}
\label{remark-existence-D}
In the chapter on derived categories we consistently work with
``small'' abelian categories (as is the convention in the Stacks
project). For a ``big'' abelian category $\mathcal{A}$ it isn't clear
that the derived category $D(\mathcal{A})$ exists because it isn't
clear that morphisms in the derived category are sets. In general this
isn't true, see
Examples, Lemma \ref{examples-lemma-big-abelian-category}.
However, if $\mathcal{A}$ is a Grothendieck abelian category, and given
$K^\bullet, L^\bullet$ in $K(\mathcal{A})$, then by
Theorem \ref{theorem-K-injective-embedding-grothendieck}
there exists a quasi-isomorphism $L^\bullet \to I^\bullet$ to a
K-injective complex $I^\bullet$ and
Derived Categories, Lemma \ref{derived-lemma-K-injective} shows that
$$
\Hom_{D(\mathcal{A})}(K^\bullet, L^\bullet) =
\Hom_{K(\mathcal{A})}(K^\bullet, I^\bullet)
$$
which is a set. Some examples of Grothendieck abelian categories
are the category of modules over a ring, or more generally
the category of sheaves of modules on a ringed site.
\end{remark}

\begin{lemma}
\label{lemma-derived-products}
Let $\mathcal{A}$ be a Grothendieck abelian category.
Then
\begin{enumerate}
\item $D(\mathcal{A})$ has both direct sums and products,
\item direct sums are obtained by taking termwise direct sums of
any complexes,
\item products are obtained by taking termwise products of
K-injective complexes.
\end{enumerate}
\end{lemma}

\begin{proof}
Let $K^\bullet_i$, $i \in I$ be a family of objects of $D(\mathcal{A})$
indexed by a set $I$. We claim that the termwise direct sum
$\bigoplus_{i \in I} K^\bullet_i$ is a direct sum in $D(\mathcal{A})$.
Namely, let $I^\bullet$ be a K-injective complex. Then we have
\begin{align*}
\Hom_{D(\mathcal{A})}(\bigoplus\nolimits_{i \in I} K^\bullet_i, I^\bullet)
& =
\Hom_{K(\mathcal{A})}(\bigoplus\nolimits_{i \in I} K^\bullet_i, I^\bullet) \\
& =
\prod\nolimits_{i \in I} \Hom_{K(\mathcal{A})}(K^\bullet_i, I^\bullet) \\
& =
\prod\nolimits_{i \in I} \Hom_{D(\mathcal{A})}(K^\bullet_i, I^\bullet)
\end{align*}
as desired. This is sufficient since any complex can be represented
by a K-injective complex by
Theorem \ref{theorem-K-injective-embedding-grothendieck}.
To construct the product, choose a K-injective resolution
$K_i^\bullet \to I_i^\bullet$ for each $i$. Then we claim that
$\prod_{i \in I} I_i^\bullet$ is a product in $D(\mathcal{A})$.
This follows from
Derived Categories, Lemma \ref{derived-lemma-product-K-injective}.
\end{proof}

\begin{remark}
\label{remark-direct-sum-product-derived}
Let $R$ be a ring. Suppose that $M_n$, $n \in \mathbf{Z}$ are $R$-modules.
Denote $E_n = M_n[-n] \in D(R)$. We claim that $E = \bigoplus M_n[-n]$ is
{\it both} the direct sum and the product of the objects $E_n$ in $D(R)$.
To see that it is the direct sum, take a look at the proof of
Lemma \ref{lemma-derived-products}.
To see that it is the direct product, take injective resolutions
$M_n \to I_n^\bullet$. By the proof of
Lemma \ref{lemma-derived-products}
we have
$$
\prod E_n = \prod I_n^\bullet[-n]
$$
in $D(R)$. Since products in $\text{Mod}_R$ are exact, we see that
$\prod I_n^\bullet[-n]$ is quasi-isomorphic to $E$. This works more generally
in $D(\mathcal{A})$ where $\mathcal{A}$ is a Grothendieck abelian
category with Ab4*.
\end{remark}

\begin{lemma}
\label{lemma-RF-commutes-with-Rlim}
Let $F : \mathcal{A} \to \mathcal{B}$ be an additive functor of
abelian categories. Assume
\begin{enumerate}
\item $\mathcal{A}$ is a Grothendieck abelian category,
\item $\mathcal{B}$ has exact countable products, and
\item $F$ commutes with countable products.
\end{enumerate}
Then
$RF : D(\mathcal{A}) \to D(\mathcal{B})$ commutes with derived limits.
\end{lemma}

\begin{proof}
Observe that $RF$ exists as $\mathcal{A}$ has enough K-injectives
(Theorem \ref{theorem-K-injective-embedding-grothendieck}
and
Derived Categories, Lemma \ref{derived-lemma-K-injective-defined}).
The statement means that if $K = R\lim K_n$, then
$RF(K) = R\lim RF(K_n)$. See
Derived Categories, Definition \ref{derived-definition-derived-limit}
for notation. Since $RF$ is an exact functor of triangulated
categories it suffices to see that $RF$ commutes with countable
products of objects of $D(\mathcal{A})$. In the proof of
Lemma \ref{lemma-derived-products}
we have seen that products in $D(\mathcal{A})$ are computed by
taking products of K-injective complexes and moreover that a
product of K-injective complexes is K-injective.
Moreover, in Derived Categories, Lemma
\ref{derived-lemma-products}
we have seen that products in $D(\mathcal{B})$ are computed
by taking termwise products.
Since $RF$ is computed by applying $F$ to a K-injective
representative and since we've assumed $F$ commutes with
countable products, the lemma follows.
\end{proof}

\noindent
The following lemma is some kind of generalization of
the existence of Cartan-Eilenberg resolutions
(Derived Categories, Section \ref{derived-section-cartan-eilenberg}).

\begin{lemma}
\label{lemma-K-injective-embedding-filtration}
Let $\mathcal{A}$ be a Grothendieck abelian category.
Let $K^\bullet$ be a filtered complex of $\mathcal{A}$, see
Homology, Definition \ref{homology-definition-filtered-complex}.
Then there exists a morphism $j : K^\bullet \to J^\bullet$
of filtered complexes of $\mathcal{A}$ such that
\begin{enumerate}
\item $J^n$, $F^pJ^n$, $J^n/F^pJ^n$ and $F^pJ^n/F^{p'}J^n$ are injective
objects of $\mathcal{A}$,
\item $J^\bullet$, $F^pJ^\bullet$, $J^\bullet/F^pJ^\bullet$, and
$F^pJ^\bullet/F^{p'}J^\bullet$ are K-injective complexes,
\item $j$ induces quasi-isomorphisms
$K^\bullet \to J^\bullet$,
$F^pK^\bullet \to F^pJ^\bullet$,
$K^\bullet/F^pK^\bullet \to J^\bullet/F^pJ^\bullet$, and
$F^pK^\bullet/F^{p'}K^\bullet \to F^pJ^\bullet/F^{p'}J^\bullet$.
\end{enumerate}
\end{lemma}

\begin{proof}
By Theorem \ref{theorem-K-injective-embedding-grothendieck}
we obtain quasi-isomorphisms $i : K^\bullet \to I^\bullet$ and
$i^p : F^pK^\bullet \to I^{p, \bullet}$ as well as commutative diagrams
$$
\vcenter{
\xymatrix{
K^\bullet \ar[d]_i & F^pK^\bullet \ar[l] \ar[d]_{i^p} \\
I^\bullet & I^{p, \bullet} \ar[l]_{\alpha^p}
}
}
\quad\text{and}\quad
\vcenter{
\xymatrix{
F^{p'}K^\bullet \ar[d]_{i^{p'}} &
F^pK^\bullet \ar[l] \ar[d]_{i^p} \\
I^{p', \bullet} &
I^{p, \bullet} \ar[l]_{\alpha^{p p'}}
}
}
\quad\text{for }p' \leq p
$$
such that $\alpha^p \circ \alpha^{p' p} = \alpha^{p'}$
and $\alpha^{p'p''} \circ \alpha^{pp'} = \alpha^{pp''}$.
The problem is that the maps $\alpha^p : I^{p, \bullet} \to I^\bullet$
need not be injective. For each $p$ we choose an injection
$t^p : I^{p, \bullet} \to J^{p, \bullet}$ into an acyclic K-injective
complex $J^{p, \bullet}$ whose terms are injective objects of $\mathcal{A}$
(first map to the cone on the identity and then use the theorem).
Choose a map of complexes $s^p : I^\bullet \to J^{p, \bullet}$
such that the following diagram commutes
$$
\xymatrix{
K^\bullet \ar[d]_i & F^pK^\bullet \ar[l] \ar[d]_{i^p} \\
I^\bullet \ar[rd]_{s^p} & I^{p, \bullet} \ar[d]^{t^p} \\
& J^{p, \bullet}
}
$$
This is possible: the composition $F^pK^\bullet \to J^{p, \bullet}$
is homotopic to zero because $J^{p, \bullet}$ is acyclic and K-injective
(Derived Categories, Lemma \ref{derived-lemma-K-injective}).
Since the objects $J^{p, n - 1}$ are injective and since
$F^pK^n \to K^n \to I^n$ are injective morphisms, we
can lift the maps $F^pK^n \to J^{p, n - 1}$ giving the homotopy
to a map $h^n : I^n \to J^{p, n - 1}$. Then we set $s^p$
equal to $h \circ \text{d} + \text{d} \circ h$.
(Warning: It will not be the case that $t^p = s^p \circ \alpha^p$,
so we have to be careful not to use this below.)

\medskip\noindent
Consider
$$
J^\bullet = I^\bullet \times \prod\nolimits_p J^{p, \bullet}
$$
Because products in $D(\mathcal{A})$ are given by taking
products of K-injective complexes
(Lemma \ref{lemma-derived-products})
and since $J^{p, \bullet}$
is isomorphic to $0$ in $D(\mathcal{A})$ we see that
$J^\bullet \to I^\bullet$ is an isomorphism in $D(\mathcal{A})$.
Consider the map
$$
j = i \times (s^p \circ i)_{p \in \mathbf{Z}} :
K^\bullet
\longrightarrow
I^\bullet \times \prod\nolimits_p J^{p, \bullet} = J^\bullet
$$
By our remarks above this is a quasi-isomorphism. It is also injective.
For $p \in \mathbf{Z}$ we let $F^pJ^\bullet \subset J^\bullet$ be
$$
\Im\left(
\alpha^p \times (t^{p'} \circ \alpha^{pp'})_{p' \leq p} :
I^{p, \bullet}
\to
I^\bullet \times \prod\nolimits_{p' \leq p} J^{p', \bullet}
\right)
\times \prod\nolimits_{p' > p} J^{p', \bullet}
$$
This complex is isomorphic to the complex
$I^{p, \bullet} \times \prod_{p' > p} J^{p, \bullet}$
as $\alpha^{pp} = \text{id}$ and $t^p$ is injective.
Hence $F^pJ^\bullet$ is quasi-isomorphic to $I^{p, \bullet}$ (argue
as above). We have $j(F^pK^\bullet) \subset F^pJ^\bullet$ because
of the commutativity of the diagram above. The corresponding
map of complexes $F^pK^\bullet \to F^pJ^\bullet$ is a quasi-isomorphism
by what we just said. Finally, to see that
$F^{p + 1}J^\bullet \subset F^pJ^\bullet$
use that $\alpha^{p + 1p} \circ \alpha^{pp'} = \alpha^{p + 1p'}$
and the commutativity of the first displayed diagram
in the first paragraph of the proof.

\medskip\noindent
We claim that $j : K^\bullet \to J^\bullet$ is a solution to the
problem posed by the lemma. Namely, $F^pJ^n$ is an injective object
of $\mathcal{A}$ because it is isomorphic to
$I^{p, n} \times \prod_{p' > p} J^{p', n}$ and products of
injectives are injective. Then the injective map $F^pJ^n \to J^n$
splits and hence the quotient $J^n/F^pJ^n$ is injective as well
as a direct summand of the injective object $J^n$.
Similarly for $F^pJ^n/F^{p'}J^n$. This in particular means
that $0 \to F^pJ^\bullet \to J^\bullet \to J^\bullet/F^pJ^\bullet \to 0$
is a termwise split short exact sequence of complexes, hence defines
a distinguished triangle in $K(\mathcal{A})$ by fiat.
Since $J^\bullet$ and $F^pJ^\bullet$ are K-injective complexes
we see that the same is true for $J^\bullet/F^pJ^\bullet$
by Derived Categories, Lemma \ref{derived-lemma-triangle-K-injective}.
A similar argument shows that $F^pJ^\bullet/F^{p'}J^\bullet$
is K-injective. By construction $j : K^\bullet \to J^\bullet$
and the induced maps $F^pK^\bullet \to F^pJ^\bullet$ are
quasi-isomorphisms. Using the long exact cohomology sequences
of the complexes in play we find that the same holds for
$K^\bullet/F^pK^\bullet \to J^\bullet/F^pJ^\bullet$ and
$F^pK^\bullet/F^{p'}K^\bullet \to F^pJ^\bullet/F^{p'}J^\bullet$.
\end{proof}

\begin{remark}
\label{remark-ext-into-filtered-complex}
Let $\mathcal{A}$ be a Grothendieck abelian category.
Let $K^\bullet$ be a filtered complex of $\mathcal{A}$, see
Homology, Definition \ref{homology-definition-filtered-complex}.
For ease of notation denote $K$, $F^pK$, $\text{gr}^pK$ the
object of $D(\mathcal{A})$ represented by $K^\bullet$,
$F^pK^\bullet$, $\text{gr}^pK^\bullet$. Let $M \in D(\mathcal{A})$.
Using Lemma \ref{lemma-K-injective-embedding-filtration}
we can construct a spectral sequence $(E_r, d_r)_{r \geq 1}$
of bigraded objects of $\mathcal{A}$ with $d_r$ of bidgree
$(r, -r + 1)$ and
with
$$
E_1^{p, q} = \Ext^{p + q}(M, \text{gr}^pK)
$$
If for every $n$ we have
$$
\Ext^n(M, F^pK) = 0 \text{ for } p \gg 0
\quad\text{and}\quad
\Ext^n(M, F^pK) = \Ext^n(M, K) \text{ for } p \ll 0
$$
then the spectral sequence is bounded and converges to $\Ext^{p + q}(M, K)$.
Namely, choose any complex $M^\bullet$ representing $M$, choose
$j : K^\bullet \to J^\bullet$ as in the lemma, and consider the complex
$$
\Hom^\bullet(M^\bullet, I^\bullet)
$$
defined exactly as in
More on Algebra, Section \ref{more-algebra-section-hom-complexes}.
Setting $F^p\Hom^\bullet(M^\bullet, I^\bullet) =
\Hom^\bullet(M^\bullet, F^pI^\bullet)$ we obtain a filtered complex.
The spectral sequence of
Homology, Section \ref{homology-section-filtered-complex}
has differentials and terms as described above; details omitted.
The boundedness and convergence follows from
Homology, Lemma \ref{homology-lemma-ss-converges-trivial}.
\end{remark}

\begin{remark}
\label{remark-spectral-sequences-ext}
Let $\mathcal{A}$ be a Grothendieck abelian category.
Let $M, K$ be objects of $D(\mathcal{A})$.
For any choice of complex $K^\bullet$ representing $K$ we
can use the filtration $F^pK^\bullet = \tau_{\leq -p}K^\bullet$
and the discussion in Remark \ref{remark-ext-into-filtered-complex}
to get a spectral sequence with
$$
E_1^{p, q} = \Ext^{2p + q}(M, H^{-p}(K))
$$
This spectral sequence is independent of the choice of
complex $K^\bullet$ representing $K$. After renumbering
$p = -j$ and $q = i + 2j$ we find a spectral sequence
$(E'_r, d'_r)_{r \geq 2}$ with $d'_r$ of bidegree $(r, -r + 1)$, with
$$
(E'_2)^{i, j} = \Ext^i(M, H^j(K))
$$
If $M \in D^-(\mathcal{A})$ and $K \in D^+(\mathcal{A})$ then
both $E_r$ and $E'_r$ are bounded and converge to $\Ext^{p + q}(M, K)$.
If we use the filtration $F^pK^\bullet = \sigma_{\geq p}K^\bullet$
then we get
$$
E_1^{p, q} = \Ext^q(M, K^p)
$$
If $M \in D^-(\mathcal{A})$ and $K^\bullet$ is bounded below, then
this spectral sequence is bounded and converges to $\Ext^{p + q}(M, K)$.
\end{remark}

\begin{remark}
\label{remark-ext-from-filtered-complex}
Let $\mathcal{A}$ be a Grothendieck abelian category. Let
$K \in D(\mathcal{A})$. Let $M^\bullet$ be a filtered complex of
$\mathcal{A}$, see Homology, Definition
\ref{homology-definition-filtered-complex}.
For ease of notation denote $M$, $M/F^pM$, $\text{gr}^pM$ the
object of $D(\mathcal{A})$ represented by $M^\bullet$,
$M^\bullet/F^pM^\bullet$, $\text{gr}^pM^\bullet$.
Dually to Remark \ref{remark-ext-into-filtered-complex}
we can construct a spectral sequence $(E_r, d_r)_{r \geq 1}$
of bigraded objects of $\mathcal{A}$ with $d_r$ of bidgree
$(r, -r + 1)$ and
with
$$
E_1^{p, q} = \Ext^{p + q}(\text{gr}^{-p}M, K)
$$
If for every $n$ we have
$$
\Ext^n(M/F^pM, K) = 0 \text{ for } p \ll 0
\quad\text{and}\quad
\Ext^n(M/F^pM, K) = \Ext^n(M, K) \text{ for } p \gg 0
$$
then the spectral sequence is bounded and converges to $\Ext^{p + q}(M, K)$.
Namely, choose a K-injective complex $I^\bullet$ with injective terms
representing $K$, see Theorem \ref{theorem-K-injective-embedding-grothendieck}.
Consider the complex
$$
\Hom^\bullet(M^\bullet, I^\bullet)
$$
defined exactly as in
More on Algebra, Section \ref{more-algebra-section-hom-complexes}.
Setting
$$
F^p\Hom^\bullet(M^\bullet, I^\bullet) =
\Hom^\bullet(M^\bullet/F^{-p + 1}M^\bullet, I^\bullet)
$$
we obtain a filtered complex (note sign and shift in filtration).
The spectral sequence of
Homology, Section \ref{homology-section-filtered-complex}
has differentials and terms as described above; details omitted.
The boundedness and convergence follows from
Homology, Lemma \ref{homology-lemma-ss-converges-trivial}.
\end{remark}

\begin{remark}
\label{remark-spectral-sequences-ext-variant}
Let $\mathcal{A}$ be a Grothendieck abelian category.
Let $M, K$ be objects of $D(\mathcal{A})$.
For any choice of complex $M^\bullet$ representing $M$ we
can use the filtration $F^pM^\bullet = \tau_{\leq -p}M^\bullet$
and the discussion in Remark \ref{remark-ext-into-filtered-complex}
to get a spectral sequence with
$$
E_1^{p, q} = \Ext^{2p + q}(H^p(M), K)
$$
This spectral sequence is independent of the choice of complex $M^\bullet$
representing $M$. After renumbering $p = -j$ and $q = i + 2j$ we find a
spectral sequence $(E'_r, d'_r)_{r \geq 2}$ with $d'_r$ of bidegree
$(r, -r + 1)$, with
$$
(E'_2)^{i, j} = \Ext^i(H^{-j}(M), K)
$$
If $M \in D^-(\mathcal{A})$ and $K \in D^+(\mathcal{A})$
then $E_r$ and $E'_r$ are bounded and converge to $\Ext^{p + q}(M, K)$.
If we use the filtration $F^pM^\bullet = \sigma_{\geq p}M^\bullet$
then we get
$$
E_1^{p, q} = \Ext^q(M^{-p}, K)
$$
If $K \in D^+(\mathcal{A})$ and $M^\bullet$ is bounded above, then
this spectral sequence is bounded and converges to $\Ext^{p + q}(M, K)$.
\end{remark}

\begin{lemma}
\label{lemma-represent-by-filtered-complex}
Let $\mathcal{A}$ be a Grothendieck abelian category. Suppose given an object
$E \in D(\mathcal{A})$ and an inverse system $\{E^i\}_{i \in \mathbf{Z}}$
of objects of $D(\mathcal{A})$ over $\mathbf{Z}$ together with
a compatible system of maps $E^i \to E$. Picture:
$$
\ldots \to E^{i + 1} \to E^i \to E^{i - 1} \to \ldots \to E
$$
Then there exists a filtered complex $K^\bullet$ of $\mathcal{A}$
(Homology, Definition \ref{homology-definition-filtered-complex})
such that $K^\bullet$ represents $E$
and $F^iK^\bullet$ represents $E^i$ compatibly with the given maps.
\end{lemma}

\begin{proof}
By Theorem \ref{theorem-K-injective-embedding-grothendieck}
we can choose a K-injective complex $I^\bullet$
representing $E$ all of whose terms $I^n$ are injective
objects of $\mathcal{A}$.
Choose a complex $G^{0, \bullet}$ representing $E^0$.
Choose a map of complexes $\varphi^0 : G^{0, \bullet} \to I^\bullet$
representing $E^0 \to E$.
For $i > 0$ we inductively represent $E^i \to E^{i - 1}$
by a map of complexes
$\delta : G^{i, \bullet} \to G^{i - 1, \bullet}$
and we set $\varphi^i = \delta \circ \varphi^{i - 1}$.
For $i < 0$ we inductively represent $E^{i + 1} \to E^i$
by a termwise injective map of complexes
$\delta : G^{i + 1, \bullet} \to G^{i, \bullet}$
(for example you can use
Derived Categories, Lemma \ref{derived-lemma-make-injective}).
Claim: we can find a map of complexes
$\varphi^i : G^{i, \bullet} \to I^\bullet$
representing the map $E^i \to E$ and
fitting into the commutative diagram
$$
\xymatrix{
G^{i + 1, \bullet} \ar[r]_\delta \ar[d]_{\varphi^{i + 1}} &
G^{i, \bullet} \ar[ld]^{\varphi^i} \\
I^\bullet
}
$$
Namely, we first choose any map of complexes
$\varphi : G^{i, \bullet} \to I^\bullet$
representing the map
$E^i \to E$. Then we see that $\varphi \circ \delta$
and $\varphi^{i + 1}$ are homotopic by some homotopy
$h^p : G^{i + 1, p} \to I^{p - 1}$.
Since the terms of
$I^\bullet$ are injective and since $\delta$
is termwise injective, we can lift $h^p$ to
$(h')^p : G^{i, p} \to I^{p - 1}$.
Then we set $\varphi^i = \varphi + h' \circ d + d \circ h'$
and we get what we claimed.

\medskip\noindent
Next, we choose for every $i$ a termwise injective map of complexes
$a^i : G^{i, \bullet} \to J^{i, \bullet}$ with $J^{i, \bullet}$
acyclic, K-injective, with $J^{i, p}$ injective objects of $\mathcal{A}$.
To do this first map $G^{i, \bullet}$ to the cone on the identity
and then apply the theorem cited above.
Arguing as above we can find maps of complexes
$\delta' : J^{i, \bullet} \to J^{i - 1, \bullet}$ such that the diagrams
$$
\xymatrix{
G^{i, \bullet} \ar[r]_\delta \ar[d]_{a^i} &
G^{i - 1, \bullet} \ar[d]^{a^{i - 1}} \\
J^{i, \bullet} \ar[r]^{\delta'} &
J^{i - 1, \bullet}
}
$$
commute. (You could also use the functoriality of cones plus the
functoriality in the theorem to get this.)
Then we consider the maps
$$
\xymatrix{
G^{i + 1, \bullet} \times \prod\nolimits_{p > i + 1} J^{p, \bullet}
\ar[r] \ar[rd] &
G^{i, \bullet} \times \prod\nolimits_{p > i} J^{p, \bullet}
\ar[r] \ar[d] &
G^{i - 1, \bullet} \times \prod\nolimits_{p > i - 1} J^{p, \bullet}
\ar[ld] \\
& I^\bullet \times \prod\nolimits_p J^{p, \bullet}
}
$$
Here the arrows on $J^{p, \bullet}$ are the obvious ones
(identity or zero). On the factor $G^{i, \bullet}$ we use
$\delta : G^{i, \bullet} \to G^{i - 1, \bullet}$, the map
$\varphi^i : G^{i, \bullet} \to I^\bullet$, the zero map
$0 : G^{i, \bullet} \to J^{p, \bullet}$ for $p > i$, the map
$a^i : G^{i, \bullet} \to J^{p, \bullet}$ for $p = i$, and
$(\delta')^{i - p} \circ a^i = a^p \circ \delta^{i - p} :
G^{i, \bullet} \to J^{p, \bullet}$ for $p < i$.
We omit the verification that all the arrows
in the diagram are termwise injective. Thus we obtain a filtered
complex. Because products in $D(\mathcal{A})$ are given by
taking products of K-injective complexes
(Lemma \ref{lemma-derived-products})
and because $J^{p, \bullet}$ is zero in $D(\mathcal{A})$
we conclude this diagram represents the given diagram in the derived
category. This finishes the proof.
\end{proof}

\begin{lemma}
\label{lemma-represent-by-filtered-complex-bis}
In the situation of Lemma \ref{lemma-represent-by-filtered-complex}
assume we have a second inverse system $\{(E')^i\}_{i \in \mathbf{Z}}$
and a compatible system of maps $(E')^i \to E$.
Then there exists a bi-filtered complex $K^\bullet$ of $\mathcal{A}$
such that $K^\bullet$ represents $E$, $F^iK^\bullet$ represents $E^i$,
and $(F')^iK^\bullet$ represents $(E')^i$ compatibly with the given maps.
\end{lemma}

\begin{proof}
Using the lemma we can first choose $K^\bullet$ and $F$.
Then we can choose $(K')^\bullet$ and $F'$ which work for
$\{(E')^i\}_{i \in \mathbf{Z}}$ and the maps $(E')^i \to E$.
Using Lemma \ref{lemma-K-injective-embedding-filtration}
we can assume $K^\bullet$ is a K-injective complex.
Then we can choose a map of complexes
$(K')^\bullet \to K^\bullet$ corresponding to
the given identifications
$(K')^\bullet \cong E \cong K^\bullet$.
We can additionally choose a termwise injective
map $(K')^\bullet \to J^\bullet$ with
$J^\bullet$ acyclic and K-injective.
(To do this first map $(K')^\bullet$ to the cone on the identity
and then apply Theorem \ref{theorem-K-injective-embedding-grothendieck}.)
Then $(K')^\bullet \to K^\bullet \times J^\bullet$ and
$K^\bullet \to K^\bullet \times J^\bullet$
are both termwise injective and quasi-isomorphisms
(as the product represents $E$ by Lemma \ref{lemma-derived-products}).
Then we can simply take the images of the filtrations
on $K^\bullet$ and $(K')^\bullet$ under these maps to conclude.
\end{proof}





\section{The Gabriel-Popescu theorem}
\label{section-gabriel-popescu}

\noindent
In this section we discuss the main theorem of \cite{GP}. The method of
proof follows a write-up by Jacob Lurie and another by Akhil Mathew
who in turn follow the presentation by Kuhn in \cite{Kuhn}.
See also \cite{Takeuchi}.

\medskip\noindent
Let $\mathcal{A}$ be a Grothendieck abelian category and let $U$ be a
generator for $\mathcal{A}$, see
Definition \ref{definition-grothendieck-conditions}.
Let $R = \Hom_\mathcal{A}(U, U)$. Consider the functor
$G : \mathcal{A} \to \text{Mod}_R$ given by
$$
G(A) = \Hom_\mathcal{A}(U, A)
$$
endowed with its canonical right $R$-module structure.

\begin{lemma}
\label{lemma-gabriel-popescu-left-adjoint}
The functor $G$ above has a left adjoint
$F : \text{Mod}_R \to \mathcal{A}$.
\end{lemma}

\begin{proof}
We will give two proofs of this lemma.

\medskip\noindent
The first proof will use the adjoint functor theorem, see
Categories, Theorem \ref{categories-theorem-adjoint-functor}.
Observe that that $G : \mathcal{A} \to \text{Mod}_R$ is left exact and sends
products to products. Hence $G$ commutes with limits. To check the set
theoretical condition in the theorem, suppose that $M$ is an object of
$\text{Mod}_R$. Choose a suitably large cardinal $\kappa$ and denote $E$
a set of objects of $\mathcal{A}$ such that every object $A$ with
$|A| \leq \kappa$ is isomorphic to an element of $E$. This is possible
by Lemma \ref{lemma-set-iso-classes-bounded-size}. Set
$I = \coprod_{A \in E} \Hom_R(M, G(A))$.
We think of an element $i \in I$ as a pair $(A_i, f_i)$.
Finally, let $A$ be an arbitrary object of $\mathcal{A}$
and $f : M \to G(A)$ arbitrary. We are going to think of
elements of $\Im(f) \subset G(A) = \Hom_\mathcal{A}(U, A)$
as maps $u : U \to A$. Set
$$
A' = \Im(\bigoplus\nolimits_{u \in \Im(f)} U \xrightarrow{u} A)
$$
Since $G$ is left exact, we see that $G(A') \subset G(A)$
contains $\Im(f)$ and we get $f' : M \to G(A')$ factoring $f$.
On the other hand, the object $A'$ is
the quotient of a direct sum of at most $|M|$ copies of $U$.
Hence if $\kappa = |\bigoplus_{|M|} U|$, then we see that $(A', f')$
is isomorphic to an element $(A_i, f_i)$ of $E$ and we conclude that $f$
factors as $M \xrightarrow{f_i} G(A_i) \to G(A)$ as desired.

\medskip\noindent
The second proof will give a construction of $F$ which will show
that ``$F(M) = M \otimes_R U$'' in some sense. Namely, for any
$R$-module $M$ we can choose a resolution
$$
\bigoplus\nolimits_{j \in J} R \to
\bigoplus\nolimits_{i \in I} R \to
M \to 0
$$
Then we define $F(M)$ by the corresponding exact sequence
$$
\bigoplus\nolimits_{j \in J} U \to
\bigoplus\nolimits_{i \in I} U \to
F(M) \to 0
$$
This construction is independent of the choice of the resolution
and is functorial; we omit the details.
For any $A$ in $\mathcal{A}$ we obtain an exact sequence
$$
0 \to \Hom_\mathcal{A}(F(M), A) \to
\prod\nolimits_{i \in I} G(A) \to
\prod\nolimits_{j \in J} G(A)
$$
which is isomorphic to the sequence
$$
0 \to \Hom_R(M, G(A)) \to
\Hom_R(\bigoplus\nolimits_{i \in I} R, G(A)) \to
\Hom_R(\bigoplus\nolimits_{j \in J} R, G(A))
$$
which shows that $F$ is the left adjoint to $G$.
\end{proof}

\begin{lemma}
\label{lemma-F-G-monos}
Let $f : M \to G(A)$ be an injective map in $\text{Mod}_R$.
Then the adjoint map $f' : F(M) \to A$ is injective too.
\end{lemma}

\begin{proof}
Choose a map $R^{\oplus n} \to M$ and consider the corresponding map
$U^{\oplus n} \to F(M)$. Consider a map $v : U \to U^{\oplus n}$
such that the composition $U \to U^{\oplus n} \to F(M) \to A$ is $0$.
Then this arrow $v : U \to U^{\oplus n}$ is an element
$v$ of $R^{\oplus n}$ mapping to zero in $G(A)$. Since $f$ is injective,
we conclude that $v$ maps to zero in $M$ which means that
$U \to U^{\oplus n} \to F(M)$ is zero by construction of $F(M)$
in the proof of Lemma \ref{lemma-gabriel-popescu-left-adjoint}.
Since $U$ is a generator we conclude that
$$
\Ker(U^{\oplus n} \to F(M) \to A) = \Ker(U^{\oplus n} \to F(M))
$$
To finish the proof we choose a surjection $\bigoplus_{i \in I} R \to M$
and we consider the corresponding surjection
$$
\pi : \bigoplus\nolimits_{i \in I} U \longrightarrow F(M)
$$
To prove $f'$ is injective it suffices to show that
$\Ker(\pi) = \Ker(f' \circ \pi)$ as subobjects of $\bigoplus_{i \in I} U$.
However, now we can write $\bigoplus_{i \in I} U$ as the filtered colimit
of its subobjects $\bigoplus_{i \in I'} U$ where $I' \subset I$
ranges over the finite subsets. Since filtered colimits are
exact by AB5 for $\mathcal{A}$, we see that
$$
\Ker(\pi) =
\colim_{I' \subset I\text{ finite}}
\left(\bigoplus\nolimits_{i \in I'} U\right)
\bigcap \Ker(\pi)
$$
and
$$
\Ker(f' \circ \pi) =
\colim_{I' \subset I\text{ finite}}
\left(\bigoplus\nolimits_{i \in I'} U\right)
\bigcap \Ker(f' \circ \pi)
$$
and we get equality because the same is true for each $I'$ by
the first displayed equality above.
\end{proof}

\begin{theorem}
\label{theorem-gabriel-popescu}
Let $\mathcal{A}$ be a Grothendieck abelian category. Then there exists
a (noncommutative) ring $R$ and functors $G : \mathcal{A} \to \text{Mod}_R$
and $F : \text{Mod}_R \to \mathcal{A}$ such that
\begin{enumerate}
\item $F$ is the left adjoint to $G$,
\item $G$ is fully faithful, and
\item $F$ is exact.
\end{enumerate}
Moreover, the functors are the ones constructed above.
\end{theorem}

\begin{proof}
We first prove $G$ is fully faithful, or equivalently that
$F \circ G \to \text{id}$ is an isomorphism, see
Categories, Lemma \ref{categories-lemma-adjoint-fully-faithful}.
First, given an object $A$ the map $F(G(A)) \to A$ is surjective,
because every map of $U \to A$ factors through $F(G(A))$ by construction.
On the other hand, the map $F(G(A)) \to A$ is the adjoint of the
map $\text{id} : G(A) \to G(A)$ and hence injective by
Lemma \ref{lemma-F-G-monos}.

\medskip\noindent
The functor $F$ is right exact as it is a left adjoint.
Since $\text{Mod}_R$ has enough projectives, to show that
$F$ is exact, it is enough to show that the first left derived
functor $L_1F$ is zero. To prove $L_1F(M) = 0$ for some $R$-module $M$
choose an exact sequence $0 \to K \to P \to M \to 0$
of $R$-modules with $P$ free. It suffices to show $F(K) \to F(P)$
is injective. Now we can write this sequence as a filtered
colimit of sequences $0 \to K_i \to P_i \to M_i \to 0$
with $P_i$ a finite free $R$-module: just write $P$ in this
manner and set $K_i = K \cap P_i$ and $M_i = \Im(P_i \to M)$.
Because $F$ is a left adjoint it commutes
with colimits and because $\mathcal{A}$ is a Grothendieck
abelian category, we find that $F(K) \to F(P)$
is injective if each $F(K_i) \to F(P_i)$ is injective.
Thus it suffices to check $F(K) \to F(P)$
is injective when $K \subset P = R^{\oplus n}$.
Thus $F(K) \to U^{\oplus n}$ is injective by an application
of Lemma \ref{lemma-F-G-monos}.
\end{proof}

\begin{lemma}
\label{lemma-gabriel-popescu}
\begin{reference}
\cite[Corollary 4.1]{serpe}
\end{reference}
Let $\mathcal{A}$ be a Grothendieck abelian category. Let
$R$, $F$, $G$ be as in the Gabriel-Popescu theorem
(Theorem \ref{theorem-gabriel-popescu}). Then we obtain
derived functors
$$
RG : D(\mathcal{A}) \to D(\text{Mod}_R)
\quad\text{and}\quad
F : D(\text{Mod}_R) \to D(\mathcal{A})
$$
such that $F$ is left adjoint to $RG$, $RG$ is fully faithful,
and $F \circ RG = \text{id}$.
\end{lemma}

\begin{proof}
The existence and adjointness of the functors follows from
Theorems \ref{theorem-gabriel-popescu} and
\ref{theorem-K-injective-embedding-grothendieck}
and
Derived Categories, Lemmas \ref{derived-lemma-K-injective-defined},
\ref{derived-lemma-right-derived-exact-functor}, and
\ref{derived-lemma-derived-adjoint-functors}.
The statement $F \circ RG = \text{id}$ follows because we can
compute $RG$ on an object of $D(\mathcal{A})$ by applying $G$
to a suitable representative complex $I^\bullet$ (for example
a K-injective one) and then $F(G(I^\bullet)) = I^\bullet$
because $F \circ G = \text{id}$. Fully faithfulness of $RG$
follows from this by
Categories, Lemma \ref{categories-lemma-adjoint-fully-faithful}.
\end{proof}





\section{Brown representability and Grothendieck abelian categories}
\label{section-brown}

\noindent
In this section we quickly prove a representability theorem for
derived categories of Grothendieck abelian categories. The reader should
first read the case of compactly generated triangulated categories in
Derived Categories, Section \ref{derived-section-brown}.
After that, instead of reading this section,
it makes sense to consult the literature for more
general results of this nature, for example see
\cite{Franke}, \cite{Neeman}, \cite{Krause}, or take
a look at Derived Categories, Section \ref{derived-section-brown-bis}.

\begin{lemma}
\label{lemma-brown}
Let $\mathcal{A}$ be a Grothendieck abelian category.
Let $H : D(\mathcal{A}) \to \textit{Ab}$ be a contravariant
cohomological functor which transforms direct sums into products.
Then $H$ is representable.
\end{lemma}

\begin{proof}
Let $R, F, G, RG$ be as in Lemma \ref{lemma-gabriel-popescu}
and consider the functor $H \circ F : D(\text{Mod}_R) \to \textit{Ab}$.
Observe that since $F$ is a left adjoint it sends direct sums to
direct sums and hence $H \circ F$ transforms direct sums into products.
On the other hand, the derived category $D(\text{Mod}_R)$ is
generated by a single compact object, namely $R$.
By Derived Categories, Lemma \ref{derived-lemma-brown}
we see that $H \circ F$ is representable, say by $L \in D(\text{Mod}_R)$.
Choose a distinguished triangle
$$
M \to L \to RG(F(L)) \to M[1]
$$
in $D(\text{Mod}_R)$. Then $F(M) = 0$ because $F \circ RG = \text{id}$.
Hence $H(F(M)) = 0$ hence $\Hom(M, L) = 0$.
It follows that $L \to RG(F(L))$ is the inclusion of a direct summand, see
Derived Categories, Lemma \ref{derived-lemma-split}.
For $A$ in $D(\mathcal{A})$ we obtain
\begin{align*}
H(A)
& =
H(F(RG(A)) \\
& =
\Hom(RG(A), L) \\
& \to
\Hom(RG(A), RG(F(L))) \\
& =
\Hom(F(RG(A)), F(L)) \\
& =
\Hom(A, F(L))
\end{align*}
where the arrow has a left inverse functorial in $A$. In other words, we find
that $H$ is the direct summand of a representable functor.
Since $D(\mathcal{A})$ is Karoubian
(Derived Categories, Lemma
\ref{derived-lemma-projectors-have-images-triangulated}) we conclude.
\end{proof}

\begin{proposition}
\label{proposition-brown}
Let $\mathcal{A}$ be a Grothendieck abelian category. Let $\mathcal{D}$
be a triangulated category. Let $F : D(\mathcal{A}) \to \mathcal{D}$ be an
exact functor of triangulated categories which transforms direct sums
into direct sums. Then $F$ has an exact right adjoint.
\end{proposition}

\begin{proof}
For an object $Y$ of $\mathcal{D}$ consider the contravariant functor
$$
D(\mathcal{A}) \to \textit{Ab},\quad
W \mapsto \Hom_\mathcal{D}(F(W), Y)
$$
This is a cohomological functor as $F$ is exact and transforms direct sums
into products as $F$ transforms direct sums into direct sums. Thus by
Lemma \ref{lemma-brown} we find an object $X$ of $D(\mathcal{A})$ such that
$\Hom_{D(\mathcal{A})}(W, X) = \Hom_\mathcal{D}(F(W), Y)$.
The existence of the adjoint follows from
Categories, Lemma \ref{categories-lemma-adjoint-exists}.
Exactness follows from
Derived Categories, Lemma \ref{derived-lemma-adjoint-is-exact}.
\end{proof}











\begin{multicols}{2}[\section{Other chapters}]
\noindent
Preliminaries
\begin{enumerate}
\item \hyperref[introduction-section-phantom]{Introduction}
\item \hyperref[conventions-section-phantom]{Conventions}
\item \hyperref[sets-section-phantom]{Set Theory}
\item \hyperref[categories-section-phantom]{Categories}
\item \hyperref[topology-section-phantom]{Topology}
\item \hyperref[sheaves-section-phantom]{Sheaves on Spaces}
\item \hyperref[sites-section-phantom]{Sites and Sheaves}
\item \hyperref[stacks-section-phantom]{Stacks}
\item \hyperref[fields-section-phantom]{Fields}
\item \hyperref[algebra-section-phantom]{Commutative Algebra}
\item \hyperref[brauer-section-phantom]{Brauer Groups}
\item \hyperref[homology-section-phantom]{Homological Algebra}
\item \hyperref[derived-section-phantom]{Derived Categories}
\item \hyperref[simplicial-section-phantom]{Simplicial Methods}
\item \hyperref[more-algebra-section-phantom]{More on Algebra}
\item \hyperref[smoothing-section-phantom]{Smoothing Ring Maps}
\item \hyperref[modules-section-phantom]{Sheaves of Modules}
\item \hyperref[sites-modules-section-phantom]{Modules on Sites}
\item \hyperref[injectives-section-phantom]{Injectives}
\item \hyperref[cohomology-section-phantom]{Cohomology of Sheaves}
\item \hyperref[sites-cohomology-section-phantom]{Cohomology on Sites}
\item \hyperref[dga-section-phantom]{Differential Graded Algebra}
\item \hyperref[dpa-section-phantom]{Divided Power Algebra}
\item \hyperref[hypercovering-section-phantom]{Hypercoverings}
\end{enumerate}
Schemes
\begin{enumerate}
\setcounter{enumi}{24}
\item \hyperref[schemes-section-phantom]{Schemes}
\item \hyperref[constructions-section-phantom]{Constructions of Schemes}
\item \hyperref[properties-section-phantom]{Properties of Schemes}
\item \hyperref[morphisms-section-phantom]{Morphisms of Schemes}
\item \hyperref[coherent-section-phantom]{Cohomology of Schemes}
\item \hyperref[divisors-section-phantom]{Divisors}
\item \hyperref[limits-section-phantom]{Limits of Schemes}
\item \hyperref[varieties-section-phantom]{Varieties}
\item \hyperref[topologies-section-phantom]{Topologies on Schemes}
\item \hyperref[descent-section-phantom]{Descent}
\item \hyperref[perfect-section-phantom]{Derived Categories of Schemes}
\item \hyperref[more-morphisms-section-phantom]{More on Morphisms}
\item \hyperref[flat-section-phantom]{More on Flatness}
\item \hyperref[groupoids-section-phantom]{Groupoid Schemes}
\item \hyperref[more-groupoids-section-phantom]{More on Groupoid Schemes}
\item \hyperref[etale-section-phantom]{\'Etale Morphisms of Schemes}
\end{enumerate}
Topics in Scheme Theory
\begin{enumerate}
\setcounter{enumi}{40}
\item \hyperref[chow-section-phantom]{Chow Homology}
\item \hyperref[intersection-section-phantom]{Intersection Theory}
\item \hyperref[pic-section-phantom]{Picard Schemes of Curves}
\item \hyperref[adequate-section-phantom]{Adequate Modules}
\item \hyperref[dualizing-section-phantom]{Dualizing Complexes}
\item \hyperref[duality-section-phantom]{Duality for Schemes}
\item \hyperref[discriminant-section-phantom]{Discriminants and Differents}
\item \hyperref[local-cohomology-section-phantom]{Local Cohomology}
\item \hyperref[curves-section-phantom]{Algebraic Curves}
\item \hyperref[resolve-section-phantom]{Resolution of Surfaces}
\item \hyperref[models-section-phantom]{Semistable Reduction}
\item \hyperref[pione-section-phantom]{Fundamental Groups of Schemes}
\item \hyperref[etale-cohomology-section-phantom]{\'Etale Cohomology}
\item \hyperref[ssgroups-section-phantom]{Linear Algebraic Groups}
\item \hyperref[crystalline-section-phantom]{Crystalline Cohomology}
\item \hyperref[proetale-section-phantom]{Pro-\'etale Cohomology}
\end{enumerate}
Algebraic Spaces
\begin{enumerate}
\setcounter{enumi}{56}
\item \hyperref[spaces-section-phantom]{Algebraic Spaces}
\item \hyperref[spaces-properties-section-phantom]{Properties of Algebraic Spaces}
\item \hyperref[spaces-morphisms-section-phantom]{Morphisms of Algebraic Spaces}
\item \hyperref[decent-spaces-section-phantom]{Decent Algebraic Spaces}
\item \hyperref[spaces-cohomology-section-phantom]{Cohomology of Algebraic Spaces}
\item \hyperref[spaces-limits-section-phantom]{Limits of Algebraic Spaces}
\item \hyperref[spaces-divisors-section-phantom]{Divisors on Algebraic Spaces}
\item \hyperref[spaces-over-fields-section-phantom]{Algebraic Spaces over Fields}
\item \hyperref[spaces-topologies-section-phantom]{Topologies on Algebraic Spaces}
\item \hyperref[spaces-descent-section-phantom]{Descent and Algebraic Spaces}
\item \hyperref[spaces-perfect-section-phantom]{Derived Categories of Spaces}
\item \hyperref[spaces-more-morphisms-section-phantom]{More on Morphisms of Spaces}
\item \hyperref[spaces-flat-section-phantom]{Flatness on Algebraic Spaces}
\item \hyperref[spaces-groupoids-section-phantom]{Groupoids in Algebraic Spaces}
\item \hyperref[spaces-more-groupoids-section-phantom]{More on Groupoids in Spaces}
\item \hyperref[bootstrap-section-phantom]{Bootstrap}
\item \hyperref[spaces-pushouts-section-phantom]{Pushouts of Algebraic Spaces}
\end{enumerate}
Topics in Geometry
\begin{enumerate}
\setcounter{enumi}{73}
\item \hyperref[spaces-chow-section-phantom]{Chow Groups of Spaces}
\item \hyperref[groupoids-quotients-section-phantom]{Quotients of Groupoids}
\item \hyperref[spaces-more-cohomology-section-phantom]{More on Cohomology of Spaces}
\item \hyperref[spaces-simplicial-section-phantom]{Simplicial Spaces}
\item \hyperref[spaces-duality-section-phantom]{Duality for Spaces}
\item \hyperref[formal-spaces-section-phantom]{Formal Algebraic Spaces}
\item \hyperref[restricted-section-phantom]{Restricted Power Series}
\item \hyperref[spaces-resolve-section-phantom]{Resolution of Surfaces Revisited}
\end{enumerate}
Deformation Theory
\begin{enumerate}
\setcounter{enumi}{81}
\item \hyperref[formal-defos-section-phantom]{Formal Deformation Theory}
\item \hyperref[defos-section-phantom]{Deformation Theory}
\item \hyperref[cotangent-section-phantom]{The Cotangent Complex}
\item \hyperref[examples-defos-section-phantom]{Deformation Problems}
\end{enumerate}
Algebraic Stacks
\begin{enumerate}
\setcounter{enumi}{85}
\item \hyperref[algebraic-section-phantom]{Algebraic Stacks}
\item \hyperref[examples-stacks-section-phantom]{Examples of Stacks}
\item \hyperref[stacks-sheaves-section-phantom]{Sheaves on Algebraic Stacks}
\item \hyperref[criteria-section-phantom]{Criteria for Representability}
\item \hyperref[artin-section-phantom]{Artin's Axioms}
\item \hyperref[quot-section-phantom]{Quot and Hilbert Spaces}
\item \hyperref[stacks-properties-section-phantom]{Properties of Algebraic Stacks}
\item \hyperref[stacks-morphisms-section-phantom]{Morphisms of Algebraic Stacks}
\item \hyperref[stacks-limits-section-phantom]{Limits of Algebraic Stacks}
\item \hyperref[stacks-cohomology-section-phantom]{Cohomology of Algebraic Stacks}
\item \hyperref[stacks-perfect-section-phantom]{Derived Categories of Stacks}
\item \hyperref[stacks-introduction-section-phantom]{Introducing Algebraic Stacks}
\item \hyperref[stacks-more-morphisms-section-phantom]{More on Morphisms of Stacks}
\item \hyperref[stacks-geometry-section-phantom]{The Geometry of Stacks}
\end{enumerate}
Topics in Moduli Theory
\begin{enumerate}
\setcounter{enumi}{99}
\item \hyperref[moduli-section-phantom]{Moduli Stacks}
\item \hyperref[moduli-curves-section-phantom]{Moduli of Curves}
\end{enumerate}
Miscellany
\begin{enumerate}
\setcounter{enumi}{101}
\item \hyperref[examples-section-phantom]{Examples}
\item \hyperref[exercises-section-phantom]{Exercises}
\item \hyperref[guide-section-phantom]{Guide to Literature}
\item \hyperref[desirables-section-phantom]{Desirables}
\item \hyperref[coding-section-phantom]{Coding Style}
\item \hyperref[obsolete-section-phantom]{Obsolete}
\item \hyperref[fdl-section-phantom]{GNU Free Documentation License}
\item \hyperref[index-section-phantom]{Auto Generated Index}
\end{enumerate}
\end{multicols}



\bibliography{my}
\bibliographystyle{amsalpha}

\end{document}
