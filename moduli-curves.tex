\IfFileExists{stacks-project.cls}{%
\documentclass{stacks-project}
}{%
\documentclass{amsart}
}

% The following AMS packages are automatically loaded with
% the amsart documentclass:
%\usepackage{amsmath}
%\usepackage{amssymb}
%\usepackage{amsthm}

\usepackage{graphicx}

% For dealing with references we use the comment environment
\usepackage{verbatim}
\newenvironment{reference}{\comment}{\endcomment}
%\newenvironment{reference}{}{}
\newenvironment{slogan}{\comment}{\endcomment}
\newenvironment{history}{\comment}{\endcomment}

% For commutative diagrams you can use
% \usepackage{amscd}
\usepackage[all]{xy}

% We use 2cell for 2-commutative diagrams.
\xyoption{2cell}
\UseAllTwocells

% To put source file link in headers.
% Change "template.tex" to "this_filename.tex"
% \usepackage{fancyhdr}
% \pagestyle{fancy}
% \lhead{}
% \chead{}
% \rhead{Source file: \url{template.tex}}
% \lfoot{}
% \cfoot{\thepage}
% \rfoot{}
% \renewcommand{\headrulewidth}{0pt}
% \renewcommand{\footrulewidth}{0pt}
% \renewcommand{\headheight}{12pt}

\usepackage{multicol}

% For cross-file-references
\usepackage{xr-hyper}

% Package for hypertext links:
\usepackage{hyperref}

% For any local file, say "hello.tex" you want to link to please
% use \externaldocument[hello-]{hello}
\externaldocument[introduction-]{introduction}
\externaldocument[conventions-]{conventions}
\externaldocument[sets-]{sets}
\externaldocument[categories-]{categories}
\externaldocument[topology-]{topology}
\externaldocument[sheaves-]{sheaves}
\externaldocument[sites-]{sites}
\externaldocument[stacks-]{stacks}
\externaldocument[fields-]{fields}
\externaldocument[algebra-]{algebra}
\externaldocument[brauer-]{brauer}
\externaldocument[homology-]{homology}
\externaldocument[derived-]{derived}
\externaldocument[simplicial-]{simplicial}
\externaldocument[more-algebra-]{more-algebra}
\externaldocument[smoothing-]{smoothing}
\externaldocument[modules-]{modules}
\externaldocument[sites-modules-]{sites-modules}
\externaldocument[injectives-]{injectives}
\externaldocument[cohomology-]{cohomology}
\externaldocument[sites-cohomology-]{sites-cohomology}
\externaldocument[dga-]{dga}
\externaldocument[dpa-]{dpa}
\externaldocument[hypercovering-]{hypercovering}
\externaldocument[schemes-]{schemes}
\externaldocument[constructions-]{constructions}
\externaldocument[properties-]{properties}
\externaldocument[morphisms-]{morphisms}
\externaldocument[coherent-]{coherent}
\externaldocument[divisors-]{divisors}
\externaldocument[limits-]{limits}
\externaldocument[varieties-]{varieties}
\externaldocument[topologies-]{topologies}
\externaldocument[descent-]{descent}
\externaldocument[perfect-]{perfect}
\externaldocument[more-morphisms-]{more-morphisms}
\externaldocument[flat-]{flat}
\externaldocument[groupoids-]{groupoids}
\externaldocument[more-groupoids-]{more-groupoids}
\externaldocument[etale-]{etale}
\externaldocument[chow-]{chow}
\externaldocument[intersection-]{intersection}
\externaldocument[pic-]{pic}
\externaldocument[adequate-]{adequate}
\externaldocument[dualizing-]{dualizing}
\externaldocument[duality-]{duality}
\externaldocument[discriminant-]{discriminant}
\externaldocument[local-cohomology-]{local-cohomology}
\externaldocument[curves-]{curves}
\externaldocument[resolve-]{resolve}
\externaldocument[models-]{models}
\externaldocument[pione-]{pione}
\externaldocument[etale-cohomology-]{etale-cohomology}
\externaldocument[ssgroups-]{ssgroups}
\externaldocument[proetale-]{proetale}
\externaldocument[crystalline-]{crystalline}
\externaldocument[spaces-]{spaces}
\externaldocument[spaces-properties-]{spaces-properties}
\externaldocument[spaces-morphisms-]{spaces-morphisms}
\externaldocument[decent-spaces-]{decent-spaces}
\externaldocument[spaces-cohomology-]{spaces-cohomology}
\externaldocument[spaces-limits-]{spaces-limits}
\externaldocument[spaces-divisors-]{spaces-divisors}
\externaldocument[spaces-over-fields-]{spaces-over-fields}
\externaldocument[spaces-topologies-]{spaces-topologies}
\externaldocument[spaces-descent-]{spaces-descent}
\externaldocument[spaces-perfect-]{spaces-perfect}
\externaldocument[spaces-more-morphisms-]{spaces-more-morphisms}
\externaldocument[spaces-flat-]{spaces-flat}
\externaldocument[spaces-groupoids-]{spaces-groupoids}
\externaldocument[spaces-more-groupoids-]{spaces-more-groupoids}
\externaldocument[bootstrap-]{bootstrap}
\externaldocument[spaces-pushouts-]{spaces-pushouts}
\externaldocument[spaces-chow-]{spaces-chow}
\externaldocument[groupoids-quotients-]{groupoids-quotients}
\externaldocument[spaces-more-cohomology-]{spaces-more-cohomology}
\externaldocument[spaces-simplicial-]{spaces-simplicial}
\externaldocument[spaces-duality-]{spaces-duality}
\externaldocument[formal-spaces-]{formal-spaces}
\externaldocument[restricted-]{restricted}
\externaldocument[spaces-resolve-]{spaces-resolve}
\externaldocument[formal-defos-]{formal-defos}
\externaldocument[defos-]{defos}
\externaldocument[cotangent-]{cotangent}
\externaldocument[examples-defos-]{examples-defos}
\externaldocument[algebraic-]{algebraic}
\externaldocument[examples-stacks-]{examples-stacks}
\externaldocument[stacks-sheaves-]{stacks-sheaves}
\externaldocument[criteria-]{criteria}
\externaldocument[artin-]{artin}
\externaldocument[quot-]{quot}
\externaldocument[stacks-properties-]{stacks-properties}
\externaldocument[stacks-morphisms-]{stacks-morphisms}
\externaldocument[stacks-limits-]{stacks-limits}
\externaldocument[stacks-cohomology-]{stacks-cohomology}
\externaldocument[stacks-perfect-]{stacks-perfect}
\externaldocument[stacks-introduction-]{stacks-introduction}
\externaldocument[stacks-more-morphisms-]{stacks-more-morphisms}
\externaldocument[stacks-geometry-]{stacks-geometry}
\externaldocument[moduli-]{moduli}
\externaldocument[moduli-curves-]{moduli-curves}
\externaldocument[examples-]{examples}
\externaldocument[exercises-]{exercises}
\externaldocument[guide-]{guide}
\externaldocument[desirables-]{desirables}
\externaldocument[coding-]{coding}
\externaldocument[obsolete-]{obsolete}
\externaldocument[fdl-]{fdl}
\externaldocument[index-]{index}

% Theorem environments.
%
\theoremstyle{plain}
\newtheorem{theorem}[subsection]{Theorem}
\newtheorem{proposition}[subsection]{Proposition}
\newtheorem{lemma}[subsection]{Lemma}

\theoremstyle{definition}
\newtheorem{definition}[subsection]{Definition}
\newtheorem{example}[subsection]{Example}
\newtheorem{exercise}[subsection]{Exercise}
\newtheorem{situation}[subsection]{Situation}

\theoremstyle{remark}
\newtheorem{remark}[subsection]{Remark}
\newtheorem{remarks}[subsection]{Remarks}

\numberwithin{equation}{subsection}

% Macros
%
\def\lim{\mathop{\mathrm{lim}}\nolimits}
\def\colim{\mathop{\mathrm{colim}}\nolimits}
\def\Spec{\mathop{\mathrm{Spec}}}
\def\Hom{\mathop{\mathrm{Hom}}\nolimits}
\def\Ext{\mathop{\mathrm{Ext}}\nolimits}
\def\SheafHom{\mathop{\mathcal{H}\!\mathit{om}}\nolimits}
\def\SheafExt{\mathop{\mathcal{E}\!\mathit{xt}}\nolimits}
\def\Sch{\mathit{Sch}}
\def\Mor{\mathop{Mor}\nolimits}
\def\Ob{\mathop{\mathrm{Ob}}\nolimits}
\def\Sh{\mathop{\mathit{Sh}}\nolimits}
\def\NL{\mathop{N\!L}\nolimits}
\def\proetale{{pro\text{-}\acute{e}tale}}
\def\etale{{\acute{e}tale}}
\def\QCoh{\mathit{QCoh}}
\def\Ker{\mathop{\mathrm{Ker}}}
\def\Im{\mathop{\mathrm{Im}}}
\def\Coker{\mathop{\mathrm{Coker}}}
\def\Coim{\mathop{\mathrm{Coim}}}
\def\id{\mathop{\mathrm{id}}\nolimits}

%
% Macros for linear algebraic groups
%
\def\SL{\mathop{\mathrm{SL}}\nolimits}
\def\GL{\mathop{\mathrm{GL}}\nolimits}
\def\ltimes{{\mathchar"256E}}
\def\rtimes{{\mathchar"256F}}
\def\Rrightarrow{{\mathchar"3456}}

%
% Macros for moduli stacks/spaces
%
\def\QCohstack{\mathcal{QC}\!\mathit{oh}}
\def\Cohstack{\mathcal{C}\!\mathit{oh}}
\def\Spacesstack{\mathcal{S}\!\mathit{paces}}
\def\Quotfunctor{\mathrm{Quot}}
\def\Hilbfunctor{\mathrm{Hilb}}
\def\Curvesstack{\mathcal{C}\!\mathit{urves}}
\def\Polarizedstack{\mathcal{P}\!\mathit{olarized}}
\def\Complexesstack{\mathcal{C}\!\mathit{omplexes}}
% \Pic is the operator that assigns to X its picard group, usage \Pic(X)
% \Picardstack_{X/B} denotes the Picard stack of X over B
% \Picardfunctor_{X/B} denotes the Picard functor of X over B
\def\Pic{\mathop{\mathrm{Pic}}\nolimits}
\def\Picardstack{\mathcal{P}\!\mathit{ic}}
\def\Picardfunctor{\mathrm{Pic}}
\def\Deformationcategory{\mathcal{D}\!\mathit{ef}}


% OK, start here.
%
\begin{document}

\title{Moduli of Curves}

\maketitle

\phantomsection
\label{section-phantom}

\tableofcontents




\section{Introduction}
\label{section-introduction}

\noindent
In this chapter we discuss some of the familiar moduli stacks of curves.
A reference is the celebrated article of Deligne and Mumford, see \cite{DM}.




\section{Conventions and abuse of language}
\label{section-conventions}

\noindent
We continue to use the conventions and the abuse of language
introduced in
Properties of Stacks, Section \ref{stacks-properties-section-conventions}.
Unless otherwise mentioned our base scheme will be $\Spec(\mathbf{Z})$.







\section{The stack of curves}
\label{section-stack-curves}

\noindent
This section is the continuation of Quot, Section \ref{quot-section-curves}.
Let $\Curvesstack$ be the stack whose category of sections over a
scheme $S$ is the category of families of curves over $S$.
It is somewhat important to keep in mind that a
{\it family of curves} is a morphism $f : X \to S$ where $X$
is an algebraic space (!) and $f$ is flat, proper, of finite presentation
and of relative dimension $\leq 1$.
We already know that $\Curvesstack$ is an
algebraic stack over $\mathbf{Z}$, see Quot, Theorem
\ref{quot-theorem-curves-algebraic}.
If we did not allow algebraic spaces in the definition of
our stack, then this theorem would be false.

\medskip\noindent
Often base change is denoted by a subscript, but we cannot use
this notation for $\Curvesstack$ because $\Curvesstack_S$
is our notation for the fibre category over $S$.
This is why in Quot, Remark \ref{quot-remark-curves-base-change}
we used $B\text{-}\Curvesstack$ for the base change
$$
B\text{-}\Curvesstack = \Curvesstack \times B
$$
to the algebraic space $B$. The product on the right is over the
final object, i.e., over $\Spec(\mathbf{Z})$. The object on the left
is the stack classifying families of curves on the category of schemes
over $B$. In particular, if $k$ is a field, then
$$
k\text{-}\Curvesstack = \Curvesstack \times \Spec(k)
$$
is the moduli stack classifying families of curves on the category
of schemes over $k$.
Before we continue, here is a sanity check.

\begin{lemma}
\label{lemma-extend-curves-to-spaces}
Let $T \to B$ be a morphism of algebraic spaces. The category
$$
\Mor_B(T, B\text{-}\Curvesstack) = \Mor(T, \Curvesstack)
$$
is the category of families of curves over $T$.
\end{lemma}

\begin{proof}
A family of curves over $T$ is a morphism $f : X \to T$ of algebraic
spaces, which is flat, proper, of finite presentation, and has
relative dimension $\leq 1$ (Morphisms of Spaces, Definition
\ref{spaces-morphisms-definition-relative-dimension}).
This is exactly the same as the definition in
Quot, Situation \ref{quot-situation-curves}
except that $T$ the base is allowed to be an algebraic space.
Our default base category for algebraic stacks/spaces
is the category of schemes, hence the lemma does not follow
immediately from the definitions. Having said this, we encourage
the reader to skip the proof.

\medskip\noindent
By the product description of $B\text{-}\Curvesstack$ given above,
it suffices to prove the lemma in the absolute case. Choose a scheme
$U$ and a surjective \'etale morphism $p : U \to T$.
Let $R = U \times_T U$ with projections $s, t : R \to U$.

\medskip\noindent
Let $v : T \to \Curvesstack$ be a morphism. Then $v \circ p$
corresponds to a family of curves $X_U \to U$. The canonical
$2$-morphism $v \circ p \circ t \to v \circ p \circ s$
is an isomorphism $\varphi : X_U \times_{U, s} R \to X_U \times_{U, t} R$.
This isomorphism satisfies the cocycle condition on
$R \times_{s, t} R$.
By Bootstrap, Lemma \ref{bootstrap-lemma-descend-algebraic-space}
we obtain a morphism of algebraic spaces $X \to T$
whose pullback to $U$ is equal to $X_U$ compatible with $\varphi$.
Since $\{U \to T\}$ is an \'etale covering, we see that
$X \to T$ is flat, proper, of finite presentation by
Descent on Spaces, Lemmas
\ref{spaces-descent-lemma-descending-property-flat},
\ref{spaces-descent-lemma-descending-property-proper}, and
\ref{spaces-descent-lemma-descending-property-finite-presentation}.
Also $X \to T$ has relative dimension $\leq 1$ because this is
an \'etale local property. Hence $X \to T$ is a family of curves over $T$.

\medskip\noindent
Conversely, let $X \to T$ be a family of curves. Then the
base change $X_U$ determines a morphism $w : U \to \Curvesstack$
and the canonical isomorphism $X_U \times_{U, s} R \to X_U \times_{U, t} R$
determines a $2$-arrow $w \circ s \to w \circ t$ satisfying the
cocycle condition. Thus a morphism $v : T = [U/R] \to \Curvesstack$
by the universal property of the quotient $[U/R]$, see
Groupoids in Spaces, Lemma
\ref{spaces-groupoids-lemma-quotient-stack-2-coequalizer}.
(Actually, it is much easier in this case to go back to before
we introduced our abuse of language and direct construct
the functor $\Sch/T \to \Curvesstack$ which ``is'' the
morphism $T \to \Curvesstack$.)

\medskip\noindent
We omit the verification that the constructions given above
extend to morphisms between objects and are mutually quasi-inverse.
\end{proof}






\section{The stack of polarized curves}
\label{section-polarized-curves}

\noindent
In this section we work out some of the material
discussed in Quot, Remark \ref{quot-remark-alternative-approach-curves}.
Consider the $2$-fibre product
$$
\xymatrix{
\Curvesstack \times_{\Spacesstack'_{fp, flat, proper}}
\Polarizedstack \ar[r] \ar[d] &
\Polarizedstack \ar[d] \\
\Curvesstack \ar[r] &
\Spacesstack'_{fp, flat, proper}
}
$$
We denote this $2$-fibre product by
$$
\textit{PolarizedCurves} =
\Curvesstack
\times_{\Spacesstack'_{fp, flat, proper}}
\Polarizedstack
$$
This fibre product parametrizes polarized curves, i.e., families
of curves endowed with a relatively ample invertible sheaf.
More precisely, an object of
$\textit{PolarizedCurves}$
is a pair $(X \to S, \mathcal{L})$ where
\begin{enumerate}
\item $X \to S$ is a morphism of schemes which is proper, flat,
of finite presentation, and has relative dimension $\leq 1$, and
\item $\mathcal{L}$ is an invertible $\mathcal{O}_X$-module
which is relatively ample on $X/S$.
\end{enumerate}
A morphism $(X' \to S', \mathcal{L}') \to (X \to S, \mathcal{L})$
between objects of
$\textit{PolarizedCurves}$
is given by a triple $(f, g, \varphi)$
where $f : X' \to X$ and $g : S' \to S$
are morphisms of schemes which fit into a commutative diagram
$$
\xymatrix{
X' \ar[d] \ar[r]_f & X \ar[d] \\
S' \ar[r]^g & S
}
$$
inducing an isomorphism $X' \to S' \times_S X$, in other words, the
diagram is cartesian, and $\varphi : f^*\mathcal{L} \to \mathcal{L}'$
is an isomorphism. Composition is defined in the obvious manner.

\begin{lemma}
\label{lemma-polarized-curves-in-polarized}
The morphism
$\textit{PolarizedCurves} \to
\Polarizedstack$ is an open and closed immersion.
\end{lemma}

\begin{proof}
This is true because the $1$-morphism
$\Curvesstack \to \Spacesstack'_{fp, flat, proper}$
is representable by open and closed immersions, see
Quot, Lemma \ref{quot-lemma-curves-open-and-closed-in-spaces}.
\end{proof}

\begin{lemma}
\label{lemma-polarized-curves-over-curves}
The morphism
$\textit{PolarizedCurves} \to \Curvesstack$
is smooth and surjective.
\end{lemma}

\begin{proof}
Surjective. Given a field $k$ and a proper algebraic space
$X$ over $k$ of dimension $\leq 1$, i.e., an object of $\Curvesstack$ over $k$.
By Spaces over Fields, Lemma
\ref{spaces-over-fields-lemma-codim-1-point-in-schematic-locus}
the algebraic space $X$ is a scheme. Hence $X$
is a proper scheme of dimension $\leq 1$ over $k$.
By Varieties, Lemma \ref{varieties-lemma-dim-1-proper-projective}
we see that $X$ is H-projective over $\kappa$.
In particular, there exists an ample invertible $\mathcal{O}_X$-module
$\mathcal{L}$ on $X$. Then $(X, \mathcal{L})$ is an object
of $\textit{PolarizedCurves}$ over
$k$ which maps to $X$.

\medskip\noindent
Smooth. Let $X \to S$ be an object of $\Curvesstack$, i.e., a
morphism $S \to \Curvesstack$. It is clear that
$$
\textit{PolarizedCurves}
\times_{\Curvesstack} S
\subset \Picardstack_{X/S}
$$
is the substack of objects $(T/S, \mathcal{L}/X_T)$ such that
$\mathcal{L}$ is ample on $X_T/T$. This is an open substack by
Descent on Spaces, Lemma \ref{spaces-descent-lemma-ample-in-neighbourhood}.
Since $\Picardstack_{X/S} \to S$ is smooth by
Moduli Stacks, Lemma \ref{moduli-lemma-pic-curves-smooth}
we win.
\end{proof}

\begin{lemma}
\label{lemma-etale-locally-scheme}
Let $X \to S$ be a family of curves.
Then there exists an \'etale covering $\{S_i \to S\}$
such that $X_i = X \times_S S_i$ is a scheme. We may even
assume $X_i$ is H-projective over $S_i$.
\end{lemma}

\begin{proof}
This is an immediate corollary of
Lemma \ref{lemma-polarized-curves-over-curves}.
Namely, unwinding the definitions, this lemma gives there is a
surjective smooth morphism $S' \to S$ such that $X' = X \times_S S'$
comes endowed with an invertible $\mathcal{O}_{X'}$-module
$\mathcal{L}'$ which is ample on $X'/S'$.
Then we can refine the smooth covering $\{S' \to S\}$
by an \'etale covering $\{S_i \to S\}$, see
More on Morphisms, Lemma \ref{more-morphisms-lemma-etale-dominates-smooth}.
After replacing $S_i$ by a suitable open covering we may assume
$X_i \to S_i$ is H-projective, see
Morphisms, Lemmas \ref{morphisms-lemma-proper-ample-locally-projective} and
\ref{morphisms-lemma-characterize-locally-projective}
(this is also discussed in detail in
More on Morphisms, Section \ref{more-morphisms-section-projective}).
\end{proof}






\section{Properties of the stack of curves}
\label{section-properties}

\noindent
The following lemma isn't true for moduli of surfaces, see
Remark \ref{remark-boundedness-aut-does-not-work-surfaces}.

\begin{lemma}
\label{lemma-curves-diagonal-separated-fp}
The diagonal of $\Curvesstack$ is separated
and of finite presentation.
\end{lemma}

\begin{proof}
Recall that $\Curvesstack$ is a limit preserving algebraic stack, see
Quot, Lemma \ref{quot-lemma-curves-limits}.
By Limits of Stacks, Lemma \ref{stacks-limits-lemma-limit-preserving-diagonal}
this implies that
$\Delta : \Polarizedstack \to \Polarizedstack \times \Polarizedstack$
is limit preserving. Hence $\Delta$ is locally of finite presentation
by Limits of Stacks, Proposition
\ref{stacks-limits-proposition-characterize-locally-finite-presentation}.

\medskip\noindent
Let us prove that $\Delta$ is separated. To see this, it suffices to show
that given a scheme $U$ and two objects $Y \to U$ and $X \to U$ of
$\Curvesstack$ over $U$, the algebraic space
$$
\mathit{Isom}_U(Y, X)
$$
is separated. This we have seen in
Moduli Stacks, Lemmas \ref{moduli-lemma-Mor-s-lfp} and
\ref{moduli-lemma-Isom-in-Mor} that the target is
a separated algebraic space.

\medskip\noindent
To finish the proof we show that $\Delta$ is quasi-compact. Since
$\Delta$ is representable by algebraic spaces, it suffices to check
the base change of $\Delta$ by a surjective smooth morphism
$U \to \Curvesstack \times \Curvesstack$ is quasi-compact
(see for example Properties of Stacks, Lemma
\ref{stacks-properties-lemma-check-property-covering}).
We choose $U = \coprod U_i$ to be a disjoint union of affine opens
with a surjective smooth morphism
$$
U \longrightarrow
\textit{PolarizedCurves} \times \textit{PolarizedCurves}
$$
Then $U \to \Curvesstack \times \Curvesstack$ will be surjective
and smooth since $\textit{PolarizedCurves} \to \Curvesstack$
is surjective and smooth by Lemma \ref{lemma-polarized-curves-over-curves}.
Since $\textit{PolarizedCurves}$ is limit preserving
(by Artin's Axioms, Lemma \ref{artin-lemma-fibre-product-limit-preserving}
and Quot, Lemmas \ref{quot-lemma-curves-limits},
\ref{quot-lemma-polarized-limits}, and
\ref{quot-lemma-spaces-limits}), we
see that $\textit{PolarizedCurves} \to \Spec(\mathbf{Z})$ is locally of
finite presentation, hence $U_i \to \Spec(\mathbf{Z})$ is
locally of finite presentation
(Limits of Stacks, Proposition
\ref{stacks-limits-proposition-characterize-locally-finite-presentation}
and Morphisms of Stacks, Lemmas
\ref{stacks-morphisms-lemma-composition-finite-presentation} and
\ref{stacks-morphisms-lemma-smooth-locally-finite-presentation}).
In particular, $U_i$ is Noetherian affine. This reduces us to the
case discussed in the next paragraph.

\medskip\noindent
In this paragraph, given a Noetherian affine scheme $U$ and two objects
$(Y, \mathcal{N})$ and $(X, \mathcal{L})$
of $\textit{PolarizedCurves}$ over $U$, we show the algebraic space
$$
\mathit{Isom}_U(Y, X)
$$
is quasi-compact. Since the connected components of $U$ are open and closed
we may replace $U$ by these. Thus we may and do assume $U$ is connected.
Let $u \in U$ be a point. Let $Q$, $P$ be the Hilbert polynomials
of these families, i.e.,
$$
Q(n) = \chi(Y_u, \mathcal{N}_u^{\otimes n})
\quad\text{and}\quad
P(n) = \chi(X_u, \mathcal{L}_u^{\otimes n})
$$
see Varieties, Lemma \ref{varieties-lemma-numerical-polynomial-from-euler}.
Since $U$ is connected and since
the functions
$u \mapsto \chi(Y_u, \mathcal{N}_u^{\otimes n})$ and
$u \mapsto \chi(X_u, \mathcal{L}_u^{\otimes n})$
are locally constant (see 
Derived Categories of Schemes, Lemma
\ref{perfect-lemma-chi-locally-constant-geometric})
we see that we get the same Hilbert polynomial in every point of $U$.
Set
$$
\mathcal{M} = \text{pr}_1^*\mathcal{N}
\otimes_{\mathcal{O}_{Y \times_U X}} \text{pr}_2^*\mathcal{L}
$$
on $Y \times_U X$. Given $(f, \varphi) \in \mathit{Isom}_U(Y, X)(T)$
for some scheme $T$ over $U$ then for every $t \in T$ we have
\begin{align*}
\chi(Y_t, (\text{id} \times f)^*\mathcal{M}^{\otimes n})
& =
\chi(Y_t,
\mathcal{N}_t^{\otimes n} \otimes_{\mathcal{O}_{Y_t}}
f_t^*\mathcal{L}_t^{\otimes n}) \\
& =
n\deg(\mathcal{N}_t) + n\deg(f_t^*\mathcal{L}_t) +
\chi(Y_t, \mathcal{O}_{Y_t}) \\
& =
Q(n) + n\deg(\mathcal{L}_t) \\
& =
Q(n) + P(n) - P(0)
\end{align*}
by Riemann-Roch for proper curves, more precisely by
Varieties, Definition \ref{varieties-definition-degree-invertible-sheaf} and
Lemma \ref{varieties-lemma-degree-tensor-product}
and the fact that $f_t$ is an isomorphism.
Setting $P'(t) = Q(t) + P(t) - P(0)$ we find
$$
\mathit{Isom}_U(Y, X) =
\mathit{Isom}_U(Y, X) \cap \mathit{Mor}^{P', \mathcal{M}}_U(Y, X)
$$
The intersection is an intersection of open subspaces of
$\mathit{Mor}_U(Y, X)$, see
Moduli Stacks, Lemma \ref{moduli-lemma-Isom-in-Mor} and
Remark \ref{moduli-remark-Mor-numerical}.
Now $\mathit{Mor}^{P', \mathcal{M}}_U(Y, X)$
is a Noetherian algebraic space as it is of finite
presentation over $U$ by
Moduli Stacks, Lemma \ref{moduli-lemma-Mor-qc-over-base}.
Thus the intersection is a Noetherian algebraic space too
and the proof is finished.
\end{proof}

\begin{remark}
\label{remark-boundedness-aut-does-not-work-surfaces}
The boundedness argument in the proof of
Lemma \ref{lemma-curves-diagonal-separated-fp}
does not work for moduli of surfaces and in fact,
the result is wrong, for example because K3 surfaces
over fields can have infinite discrete automorphism groups.
The ``reason'' the argument does not work is that on a
projective surface $S$ over a field,
given ample invertible sheaves $\mathcal{N}$
and $\mathcal{L}$ with Hilbert polynomials $Q$ and $P$,
there is no a priori bound on the Hilbert polynomial
of $\mathcal{N} \otimes_{\mathcal{O}_S} \mathcal{L}$.
In terms of intersection theory, if $H_1$, $H_2$ are ample effective
Cartier divisors on $S$,
then there is no (upper) bound on the intersection number $H_1 \cdot H_2$
in terms of $H_1 \cdot H_1$ and $H_2 \cdot H_2$.
\end{remark}

\begin{lemma}
\label{lemma-curves-qs-lfp}
The morphism $\Curvesstack \to \Spec(\mathbf{Z})$ is quasi-separated and
locally of finite presentation.
\end{lemma}

\begin{proof}
To check $\Curvesstack \to \Spec(\mathbf{Z})$ is quasi-separated we have to
show that its diagonal is quasi-compact and quasi-separated.
This is immediate from Lemma \ref{lemma-curves-diagonal-separated-fp}.
To prove that $\Curvesstack \to \Spec(\mathbf{Z})$ is locally of finite
presentation, it suffices to show that $\Curvesstack$
is limit preserving, see Limits of Stacks, Proposition
\ref{stacks-limits-proposition-characterize-locally-finite-presentation}.
This is Quot, Lemma \ref{quot-lemma-curves-limits}.
\end{proof}






\section{Open substacks of the stack of curves}
\label{section-open}

\noindent
Below we will often characterize an open substack of $\Curvesstack$
by a property $P$ of morphisms of algebraic spaces. To see that $P$
defines an open substack it suffices to check
\begin{enumerate}
\item[(o)] given a family of curves $f : X \to S$ there exists
a largest open subscheme $S' \subset S$ such that
$f|_{f^{-1}(S')} : f^{-1}(S') \to S'$ has $P$ and such that
formation of $S'$ commutes with arbitrary base change.
\end{enumerate}
Namely, suppose (o) holds. Choose a scheme $U$ and a surjective
smooth morphism $m : U \to \Curvesstack$. Let
$R = U \times_{\Curvesstack} U$ and denote $t, s : R \to U$
the projections. Recall that $\Curvesstack = [U/R]$ is a presentation,
see Algebraic Stacks, Lemma \ref{algebraic-lemma-stack-presentation} and
Definition \ref{algebraic-definition-presentation}.
By construction of $\Curvesstack$ as
the stack of curves, the morphism $m$ is the classifying morphism
for a family of curves $C \to U$. The $2$-commutativity
of the diagram
$$
\xymatrix{
R \ar[r]_s \ar[d]_t & U \ar[d] \\
U \ar[r] & \Curvesstack
}
$$
implies that $C \times_{U, s} R  \cong C \times_{U, t} R$
(isomorphism of families of curves over $R$). Let $W \subset U$
be the largest open subscheme such that
$f|_{f^{-1}(W)} : f^{-1}(W) \to W$ has $P$ as in (o).
Since formation of $W$ commutes with base change according to (o)
and by the isomorphism above we find that $s^{-1}(W) = t^{-1}(W)$.
Thus $W \subset U$ corresponds to an open substack
$$
\Curvesstack^P \subset \Curvesstack
$$
according to Properties of Stacks, Lemma
\ref{stacks-properties-lemma-immersion-presentation}.

\medskip\noindent
Continuing with the setup of the previous paragrpah, we claim
the open substack $\Curvesstack^P$ has the following two universal properties:
\begin{enumerate}
\item given a family of curves $X \to S$ the following are equivalent
\begin{enumerate}
\item the classifying morphism $S \to \Curvesstack$ factors through
$\Curvesstack^P$,
\item the morphism $X \to S$ has $P$,
\end{enumerate}
\item given $X$ a proper scheme over a field $k$ of dimension $\leq 1$
the following are equivalent
\begin{enumerate}
\item the classifying morphism $\Spec(k) \to \Curvesstack$ factors
through $\Curvesstack^P$,
\item the morphism $X \to \Spec(k)$ has $P$.
\end{enumerate}
\end{enumerate}
This follows by considering the $2$-fibre product
$$
\xymatrix{
T \ar[r]_p \ar[d]_q & U \ar[d] \\
S \ar[r] & \Curvesstack
}
$$
Observe that $T \to S$ is surjective and smooth as the base
change of $U \to \Curvesstack$. Thus the open $S' \subset S$
given by (o) is determined by its inverse image in $T$.
However, by the invariance under base change of these opens in (o) 
and because $X \times_S T \cong C \times_U T$ by the $2$-commutativity,
we find $q^{-1}(S') = p^{-1}(W)$ as opens of $T$.
This immediately implies (1). Part (2) is a special case of (1).

\medskip\noindent
Given two properties $P$ and $Q$ of morphisms of algebraic spaces,
supposing we already have established $\Curvesstack^Q$ is
an open substack of $\Curvesstack$, then we can use exactly
the same method to prove openness of
$\Curvesstack^{Q, P} \subset \Curvesstack^Q$.
We omit a precise explanation.



\section{Curves with finite reduced automorphism groups}
\label{section-finite-aut}

\noindent
Let $X$ be a proper scheme over a field $k$ of dimension $\leq 1$, i.e.,
an object of $\Curvesstack$ over $k$.
By Lemma \ref{lemma-curves-diagonal-separated-fp}
the automorphism group algebraic space $\mathit{Aut}(X)$
is finite type and separated over $k$.
In particular, $\mathit{Aut}(X)$ is a group scheme, see
More on Groupoids in Spaces, Lemma
\ref{spaces-more-groupoids-lemma-group-space-scheme-locally-finite-type-over-k}.
If the characteristic of $k$ is zero, then $\mathit{Aut}(X)$
is reduced and even smooth over $k$ (Groupoids, Lemma
\ref{groupoids-lemma-group-scheme-characteristic-zero-smooth}).
However, in general $\mathit{Aut}(X)$ is not reduced, even
if $X$ is geometrically reduced.

\begin{example}[Non-reduced automorphism group]
\label{example-non-reduced}
Let $k$ be an algebraically closed field of characteristic $2$.
Set $Y = Z = \mathbf{P}^1_k$. Choose three pairwise distinct $k$-valued points
$a, b, c$ in $\mathbf{A}^1_k$. Thinking of
$\mathbf{A}^1_k \subset \mathbf{P}^1_k = Y = Z$ as an open subschemes,
we get a closed immersion
$$
T =  \Spec(k[t]/(t - a)^2) \amalg \Spec(k[t]/(t - b)^2)
\amalg \Spec(k[t]/(t - c)^2)
\longrightarrow
\mathbf{P}^1_k
$$
Let $X$ be the pushout in the diagram
$$
\xymatrix{
T \ar[r] \ar[d] & Y \ar[d] \\
Z \ar[r] & X
}
$$
Let $U \subset X$ be the affine open part which is the image of
$\mathbf{A}^1_k \amalg \mathbf{A}^1_k$. Then we have an equalizer
diagram
$$
\xymatrix{
\mathcal{O}_X(U) \ar[r] &
k[t] \times k[t] \ar@<1ex>[r] \ar@<-1ex>[r] &
k[t]/(t - a)^2 \times k[t]/(t - b)^2 \times k[t]/(t - c)^2
}
$$
Over the dual numbers $A = k[\epsilon]$ we have a nontrivial automorphism
of this equalizer diagram sending $t$ to $t + \epsilon$. We leave it to the
reader to see that this automorphism extends to an automorphism of $X$
over $A$. On the other hand, the reader easily shows that the
automorphism group of $X$ over $k$ is finite.
Thus $\mathit{Aut}(X)$ must be non-reduced.
\end{example}

\noindent
Let $X$ be a proper scheme over a field $k$ of dimension $\leq 1$, i.e.,
an object of $\Curvesstack$ over $k$. If $\mathit{Aut}(X)$
is geometrically reduced, then it need not be the case that
it has dimension $0$, even if $X$ is smooth and geometrically connected.

\begin{example}[Smooth positive dimensional automorphism group]
\label{example-pos-dim}
Let $k$ be an algebraically closed field. If $X$ is a smooth
genus $0$, resp.\ $1$ curve, then the automorphism group has
dimension $3$, resp.\ $1$. Namely, in the genus $0$ case we have
$X \cong \mathbf{P}^1_k$ by Algebraic Curves, Proposition
\ref{curves-proposition-projective-line}. Since
$$
\mathit{Aut}(\mathbf{P}^1_k) = \text{PGL}_{2, k}
$$
as functors we see that the dimension is $3$. On the other hand,
if the genus of $X$ is $1$, then we see that the map
$X = \underline{\Hilbfunctor}^1_{X/k} \to
\underline{\Picardfunctor}^1_{X/k}$ is an isomorphism, see
Picard Schemes of Curves, Lemma \ref{pic-lemma-picard-pieces}
and
Algebraic Curves, Theorem \ref{curves-theorem-curves-rational-maps}.
Thus $X$ has the structure of an abelian variety
(since $\underline{\Picardfunctor}^1_{X/k} \cong
\underline{\Picardfunctor}^0_{X/k}$).
In particular the (co)tangent bundle of $X$ are trivial
(Groupoids, Lemma \ref{groupoids-lemma-group-scheme-module-differentials}).
We conclude that $\dim_k H^0(X, T_X) = 1$ hence
$\dim \mathit{Aut}(X) \leq 1$. On the other hand, the translations
(viewing $X$ as a group scheme) provide a $1$-dimensional
piece of $\text{Aut}(X)$ and we conlude its dimension is indeed $1$.
\end{example}

\noindent
It turns out that there is an open substack of
$\Curvesstack$ parametrizing curves whose automorphism
group is geometrically reduced and finite.
Here is a precise statement.

\begin{lemma}
\label{lemma-DM-curves}
There exist an open substack $\Curvesstack^{DM} \subset \Curvesstack$
with the following properties
\begin{enumerate}
\item $\Curvesstack^{DM} \subset \Curvesstack$ is the maximal
open substack which is DM,
\item given a family of curves $X \to S$ the following are equivalent
\begin{enumerate}
\item the classifying morphism $S \to \Curvesstack$ factors through
$\Curvesstack^{DM}$,
\item the group algebraic space $\mathit{Aut}_S(X)$ is unramified over $S$,
\end{enumerate}
\item given $X$ a proper scheme over a field $k$ of dimension $\leq 1$
the following are equivalent
\begin{enumerate}
\item the classifying morphism $\Spec(k) \to \Curvesstack$ factors
through $\Curvesstack^{DM}$,
\item $\mathit{Aut}(X)$ is geometrically reduced over $k$ and
has dimension $0$,
\item $\mathit{Aut}(X) \to \Spec(k)$ is unramified.
\end{enumerate}
\end{enumerate}
\end{lemma}

\begin{proof}
The existence of an open substack with property (1) is
Morphisms of Stacks, Lemma \ref{stacks-morphisms-lemma-open-DM-locus}.
The points of this open substack are characterized by (3)(c) by
Morphisms of Stacks, Lemma \ref{stacks-morphisms-lemma-points-DM-locus}.
The equivalence of (3)(b) and (3)(c) is the statement that an
algebraic space $G$ which is locally of finite type, geometrically reduced,
and of dimension $0$ over a field $k$, is unramified over $k$.
First, $G$ is a scheme by Spaces over Fields, Lemma
\ref{spaces-over-fields-lemma-locally-finite-type-dim-zero}.
Then we can take an affine open in $G$ and observe
that it will be proper over $k$ and apply
Varieties, Lemma
\ref{varieties-lemma-proper-geometrically-reduced-global-sections}.
Minor details omitted.

\medskip\noindent
Part (2) is true because (3) holds. Namely, the morphism
$\mathit{Aut}_S(X) \to S$ is locally of finite type. Thus we can check whether
$\mathit{Aut}_S(X) \to S$ is unramified at all points of
$\mathit{Aut}_S(X)$ by checking on fibres at points of the scheme $S$, see
Morphisms of Spaces, Lemma \ref{spaces-morphisms-lemma-where-unramified}.
But after base change to a point of $S$ we fall back into
the equivalence of (3)(a) and (3)(c).
\end{proof}

\begin{lemma}
\label{lemma-in-DM-locus-vector-fields}
Let $X$ be a proper scheme over a field $k$ of dimension $\leq 1$.
Then properties (3)(a), (b), (c) are also equivalent to
$\text{Der}_k(\mathcal{O}_X, \mathcal{O}_X) = 0$.
\end{lemma}

\begin{proof}
In the discussion above we have seen that $G = \mathit{Aut}(X)$
is a group scheme over $\Spec(k)$ which is finite type and separated;
this uses Lemma \ref{lemma-curves-diagonal-separated-fp} and
More on Groupoids in Spaces, Lemma
\ref{spaces-more-groupoids-lemma-group-space-scheme-locally-finite-type-over-k}.
Then $G$ is unramified over $k$ if and only if $\Omega_{G/k} = 0$
(Morphisms, Lemma \ref{morphisms-lemma-unramified-omega-zero}).
By Groupoids, Lemma \ref{groupoids-lemma-group-scheme-module-differentials}
the vanishing holds if $T_{G/k, e} = 0$, where $T_{G/k, e}$ is the tangent
space to $G$ at the identity element $e \in G(k)$, see
Varieties, Definition \ref{varieties-definition-tangent-space}
and the formula in
Varieties, Lemma \ref{varieties-lemma-tangent-space-cotangent-space}.
Since $\kappa(e) = k$ the tangent space is defined in terms of
morphisms $\alpha : \Spec(k[\epsilon]) \to G = \mathit{Aut}(X)$
whose restriction to $\Spec(k)$ is $e$.
It follows that it suffices to show any automorphism
$$
\alpha :
X \times_{\Spec(k)} \Spec(k[\epsilon])
\longrightarrow
X \times_{\Spec(k)} \Spec(k[\epsilon])
$$
over $\Spec(k[\epsilon])$ whose restriction to $\Spec(k)$ is
$\text{id}_X$. Such automorphisms are
called infinitesimal automorphisms.

\medskip\noindent
The infinitesimal automorphisms of $X$ correspond $1$-to-$1$
with derivations of $\mathcal{O}_X$ over $k$. This follows from
More on Morphisms, Lemmas \ref{more-morphisms-lemma-difference-derivation} and
\ref{more-morphisms-lemma-action-by-derivations} (we only need the first one
as we don't care about the reverse direction; also, please look at
More on Morphisms, Remark \ref{more-morphisms-remark-another-special-case}
for an elucidation). For a different argument proving this equality
we refer the reader to
Deformation Problems, Lemma \ref{examples-defos-lemma-schemes-TI}.
\end{proof}





\section{Cohen-Macaulay curves}
\label{section-CM}

\noindent
There is an open substack of $\Curvesstack$ parametrizing
the Cohen-Macaulay ``curves''.

\begin{lemma}
\label{lemma-CM-curves}
There exist an open substack $\Curvesstack^{CM} \subset \Curvesstack$
such that
\begin{enumerate}
\item given a family of curves $X \to S$ the following are equivalent
\begin{enumerate}
\item the classifying morphism $S \to \Curvesstack$ factors
through $\Curvesstack^{CM}$,
\item the morphism $X \to S$ is Cohen-Macaulay,
\end{enumerate}
\item given a scheme $X$ proper over a field $k$ with $\dim(X) \leq 1$
the following are equivalent
\begin{enumerate}
\item the classifying morphism $\Spec(k) \to \Curvesstack$ factors
through $\Curvesstack^{CM}$,
\item $X$ is Cohen-Macaulay.
\end{enumerate}
\end{enumerate}
\end{lemma}

\begin{proof}
Let $f : X \to S$ be a family of curves. By
More on Morphisms of Spaces, Lemma
\ref{spaces-more-morphisms-lemma-flat-finite-presentation-CM-open}
the set
$$
W = \{x \in |X| : f \text{ is Cohen-Macaulay at }x\}
$$
is open in $|X|$ and formation of this open commutes with arbitrary
base change. Since $f$ is proper the subset
$$
S' = S \setminus f(|X| \setminus W)
$$
of $S$ is open and $X \times_S S' \to S'$ is Cohen-Macaulay.
Moreover, formation of $S'$ commutes with arbitrary base
change because this is true for $W$
Thus we get the open substack with the desired properties
by the method discussed in Section \ref{section-open}.
\end{proof}

\begin{lemma}
\label{lemma-CM-1-curves}
There exist an open substack $\Curvesstack^{CM, 1} \subset \Curvesstack$
such that
\begin{enumerate}
\item given a family of curves $X \to S$ the following are equivalent
\begin{enumerate}
\item the classifying morphism $S \to \Curvesstack$ factors
through $\Curvesstack^{CM, 1}$,
\item the morphism $X \to S$ is Cohen-Macaulay and has
relative dimension $1$ (Morphisms of Spaces, Definition
\ref{spaces-morphisms-definition-relative-dimension}),
\end{enumerate}
\item given a scheme $X$ proper over a field $k$ with $\dim(X) \leq 1$
the following are equivalent
\begin{enumerate}
\item the classifying morphism $\Spec(k) \to \Curvesstack$ factors
through $\Curvesstack^{CM, 1}$,
\item $X$ is Cohen-Macaulay and $X$ is equidimensional of
dimension $1$.
\end{enumerate}
\end{enumerate}
\end{lemma}

\begin{proof}
By Lemma \ref{lemma-CM-curves} it is clear that we have
$\Curvesstack^{CM, 1} \subset \Curvesstack^{CM}$
if it exists. Let $f : X \to S$ be a family of curves
such that $f$ is a Cohen-Macaulay morphism. By
More on Morphisms of Spaces, Lemma
\ref{spaces-more-morphisms-lemma-lfp-CM-relative-dimension}
we have a decomposition
$$
X = X_0 \amalg X_1
$$
by open and closed subspaces such that $X_0 \to S$ has relative
dimension $0$ and $X_1 \to S$ has relative dimension $1$.
Since $f$ is proper the subset
$$
S' = S \setminus f(|X_0|)
$$
of $S$ is open and $X \times_S S' \to S'$ is Cohen-Macaulay
and has relative dimension $1$.
Moreover, formation of $S'$ commutes with arbitrary base
change because this is true for the decomposition above
(as relative dimension behaves well with respect to base
change, see Morphisms of Spaces, Lemma
\ref{spaces-morphisms-lemma-dimension-fibre-after-base-change}).
Thus we get the open substack with the desired properties
by the method discussed in Section \ref{section-open}.
\end{proof}







\section{Curves of a given genus}
\label{section-genus}

\noindent
The convention in the Stacks project is that the genus $g$ of a
proper $1$-dimensional scheme $X$ over a field $k$ is defined only
if $H^0(X, \mathcal{O}_X) = k$. In this case
$g = \dim_k H^1(X, \mathcal{O}_X)$.
See Algebraic Curves, Section \ref{curves-section-genus}.
The conditions needed to define the genus define an open substack
which is then a disjoint union of open substacks, one for each genus.

\begin{lemma}
\label{lemma-pre-genus-curves}
There exist an open substack $\Curvesstack^{h0, 1} \subset \Curvesstack$
such that
\begin{enumerate}
\item given a family of curves $f : X \to S$ the following are equivalent
\begin{enumerate}
\item the classifying morphism $S \to \Curvesstack$ factors
through $\Curvesstack^{h0, 1}$,
\item $f_*\mathcal{O}_X = \mathcal{O}_S$, this holds
after arbitrary base change, and the fibres of $f$ have dimension $1$,
\end{enumerate}
\item given a scheme $X$ proper over a field $k$ with $\dim(X) \leq 1$
the following are equivalent
\begin{enumerate}
\item the classifying morphism $\Spec(k) \to \Curvesstack$ factors
through $\Curvesstack^{h0, 1}$,
\item $H^0(X, \mathcal{O}_X) = k$ and $\dim(X) = 1$.
\end{enumerate}
\end{enumerate}
\end{lemma}

\begin{proof}
Given a family of curves $X \to S$ the set of $s \in S$ where
$\kappa(s) = H^0(X_s, \mathcal{O}_{X_s})$
is open in $S$ by Derived Categories of Spaces, Lemma
\ref{spaces-perfect-lemma-jump-loci-geometric}.
Also, the set of points in $S$ where the fibre has
dimension $1$ is open by More on Morphisms of Spaces, Lemma
\ref{spaces-more-morphisms-lemma-dimension-fibres-proper-flat}.
Moreover, if $f : X \to S$ is a family of curves all of whose fibres
have dimension $1$ (and in particular $f$ is surjective), then
condition (1)(b) is equivalent to
$\kappa(s) = H^0(X_s, \mathcal{O}_{X_s})$ for every $s \in S$, see
Derived Categories of Spaces, Lemma \ref{spaces-perfect-lemma-proper-flat-h0}.
Thus we see that the lemma follows from the general discussion in
Section \ref{section-open}.
\end{proof}

\begin{lemma}
\label{lemma-pre-genus-in-CM-1}
We have $\Curvesstack^{h0, 1} \subset \Curvesstack^{CM, 1}$
as open substacks of $\Curvesstack$.
\end{lemma}

\begin{proof}
See Algebraic Curves, Lemma \ref{curves-lemma-automatic} and
Lemmas \ref{lemma-pre-genus-curves} and \ref{lemma-CM-1-curves}.
\end{proof}

\begin{lemma}
\label{lemma-genus}
Let $f : X \to S$ be a family of curves such that
$\kappa(s) = H^0(X_s, \mathcal{O}_{X_s})$ for all $s \in S$, i.e.,
the classifying morphism $S \to \Curvesstack$ factors
through $\Curvesstack^{h0, 1}$ (Lemma \ref{lemma-pre-genus-curves}). Then
\begin{enumerate}
\item $f_*\mathcal{O}_X = \mathcal{O}_S$ and this holds universally,
\item $R^1f_*\mathcal{O}_X$ is a finite locally free $\mathcal{O}_S$-module,
\item for any morphism $h : S' \to S$ if $f' : X' \to S'$ is the base change,
then $h^*(R^1f_*\mathcal{O}_X) = R^1f'_*\mathcal{O}_{X'}$.
\end{enumerate}
\end{lemma}

\begin{proof}
We apply Derived Categories of Spaces, Lemma
\ref{spaces-perfect-lemma-proper-flat-h0}.
This proves part (1). It also implies that locally on $S$
we can write $Rf_*\mathcal{O}_X = \mathcal{O}_S \oplus P$
where $P$ is perfect of tor amplitude in $[1, \infty)$.
Recall that formation of $Rf_*\mathcal{O}_X$ commutes
with arbitrary base change
(Derived Categories of Spaces, Lemma
\ref{spaces-perfect-lemma-flat-proper-perfect-direct-image-general}).
Thus for $s \in S$ we have
$$
H^i(P \otimes_{\mathcal{O}_S}^\mathbf{L} \kappa(s)) =
H^i(X_s, \mathcal{O}_{X_s})
\text{ for }i \geq 1
$$
This is zero unless $i = 1$ since $X_s$ is a $1$-dimensional
Noetherian scheme, see
Cohomology, Proposition \ref{cohomology-proposition-vanishing-Noetherian}.
Then $P = H^1(P)[-1]$ and $H^1(P)$ is finite locally free
for example by More on Algebra, Lemma
\ref{more-algebra-lemma-lift-perfect-from-residue-field}.
Since everything is compatible with base change we
also see that (3) holds.
\end{proof}

\begin{lemma}
\label{lemma-pre-genus-one-piece-per-genus}
There is a decomposition into open and closed substacks
$$
\Curvesstack^{h0, 1} = \coprod\nolimits_{g \geq 0} \Curvesstack_g
$$
where each $\Curvesstack_g$ is characterized as follows:
\begin{enumerate}
\item given a family of curves $f : X \to S$ the following are equivalent
\begin{enumerate}
\item the classifying morphism $S \to \Curvesstack$ factors
through $\Curvesstack_g$,
\item $f_*\mathcal{O}_X = \mathcal{O}_S$, this holds after
arbitrary base change, the fibres of $f$ have dimension $1$, and
$R^1f_*\mathcal{O}_X$ is a locally free $\mathcal{O}_S$-module of rank $g$,
\end{enumerate}
\item given a scheme $X$ proper over a field $k$ with $\dim(X) \leq 1$
the following are equivalent
\begin{enumerate}
\item the classifying morphism $\Spec(k) \to \Curvesstack$ factors
through $\Curvesstack_g$,
\item $\dim(X) = 1$, $k = H^0(X, \mathcal{O}_X)$, and
the genus of $X$ is $g$.
\end{enumerate}
\end{enumerate}
\end{lemma}

\begin{proof}
We already have the existence of $\Curvesstack^{h0, 1}$ as an open
substack of $\Curvesstack$ characterized by the conditions of the
lemma not involving $R^1f_*$ or $H^1$, see Lemma \ref{lemma-pre-genus-curves}.
The existence of the decomposition into open and closed substacks
follows immediately from the discussion in Section \ref{section-open}
and Lemma \ref{lemma-genus}. This proves the characterization in (1).
The characterization in (2) follows from the definition of the
genus in Algebraic Curves, Definition \ref{curves-definition-genus}.
\end{proof}









\section{Geometrically reduced curves}
\label{section-geometrically-reduced}

\noindent
There is an open substack of $\Curvesstack$ parametrizing
the geometrically reduced ``curves''.

\begin{lemma}
\label{lemma-geometrically-reduced-curves}
There exist an open substack $\Curvesstack^{geomred} \subset \Curvesstack$
such that
\begin{enumerate}
\item given a family of curves $X \to S$ the following are equivalent
\begin{enumerate}
\item the classifying morphism $S \to \Curvesstack$ factors
through $\Curvesstack^{geomred}$,
\item the fibres of the morphism $X \to S$ are geometrically reduced
(More on Morphisms of Spaces, Definition
\ref{spaces-more-morphisms-definition-geometrically-reduced-fibre}),
\end{enumerate}
\item given a scheme $X$ proper over a field $k$ with $\dim(X) \leq 1$
the following are equivalent
\begin{enumerate}
\item the classifying morphism $\Spec(k) \to \Curvesstack$ factors
through $\Curvesstack^{geomred}$,
\item $X$ is geometrically reduced over $k$.
\end{enumerate}
\end{enumerate}
\end{lemma}

\begin{proof}
Let $f : X \to S$ be a family of curves. By
More on Morphisms of Spaces, Lemma
\ref{spaces-more-morphisms-lemma-geometrically-reduced-open}
the set
$$
E = \{s \in S : \text{the fibre of }X \to S\text{ at }s
\text{ is geometrically reduced}\}
$$
is open in $S$. Formation of this open commutes with arbitrary
base change by
More on Morphisms of Spaces, Lemma
\ref{spaces-more-morphisms-lemma-base-change-fibres-geometrically-reduced}.
Thus we get the open substack with the desired properties
by the method discussed in Section \ref{section-open}.
\end{proof}

\begin{lemma}
\label{lemma-geomred-in-CM}
We have $\Curvesstack^{geomred} \subset \Curvesstack^{CM}$
as open substacks of $\Curvesstack$.
\end{lemma}

\begin{proof}
This is true because a reduced Noetherian scheme of
dimension $\leq 1$ is Cohen-Macaulay. See
Algebra, Lemma \ref{algebra-lemma-criterion-reduced}.
\end{proof}






\section{Geometrically reduced and connected curves}
\label{section-geometrically-reduced-connected}

\noindent
There is an open substack of $\Curvesstack$ parametrizing
the geometrically reduced and connected ``curves''.
We will get rid of $0$-dimensional objects right away.

\begin{lemma}
\label{lemma-geometrically-reduced-connected-1-curves}
There exist an open substack $\Curvesstack^{grc, 1} \subset \Curvesstack$
such that
\begin{enumerate}
\item given a family of curves $X \to S$ the following are equivalent
\begin{enumerate}
\item the classifying morphism $S \to \Curvesstack$ factors
through $\Curvesstack^{grc, 1}$,
\item the geometric fibres of the morphism $X \to S$ are
reduced, connected, and have dimension $1$,
\end{enumerate}
\item given a scheme $X$ proper over a field $k$ with $\dim(X) \leq 1$
the following are equivalent
\begin{enumerate}
\item the classifying morphism $\Spec(k) \to \Curvesstack$ factors
through $\Curvesstack^{grc, 1}$,
\item $X$ is geometrically reduced, geometrically connected,
and has dimension $1$.
\end{enumerate}
\end{enumerate}
\end{lemma}

\begin{proof}
By Lemmas \ref{lemma-geometrically-reduced-curves},
\ref{lemma-geomred-in-CM}, \ref{lemma-CM-curves}, and \ref{lemma-CM-1-curves}
it is clear that we have
$$
\Curvesstack^{grc, 1}
\subset
\Curvesstack^{geomred} \cap \Curvesstack^{CM, 1}
$$
if it exists. Let $f : X \to S$ be a family of curves such that $f$ is
Cohen-Macaulay, has geometrically reduced fibres, and
has relative dimension $1$. By
More on Morphisms of Spaces, Lemma
\ref{spaces-more-morphisms-lemma-stein-factorization-etale}
in the Stein factorization
$$
X \to T \to S
$$
the morphism $T \to S$ is \'etale. This implies that
there is an open and closed subscheme $S' \subset S$
such that $X \times_S S' \to S'$ has geometrically
connected fibres (in the decomposition of
Morphisms, Lemma \ref{morphisms-lemma-finite-locally-free}
for the finite locally free morphism $T \to S$
this corresponds to $S_1$).
Formation of this open commutes with arbitrary base change
because the number of connected components of geometric
fibres is invariant under base change (it is also true
that the Stein factorization commutes with base change
in our particular case but we don't need this to conclude).
Thus we get the open substack with the desired properties
by the method discussed in Section \ref{section-open}.
\end{proof}

\begin{lemma}
\label{lemma-geomredcon-in-h0-1}
We have $\Curvesstack^{grc, 1} \subset \Curvesstack^{h0, 1}$
as open substacks of $\Curvesstack$. In particular, given
a family of curves $f : X \to S$
whose geometric fibres are reduced, connected and of dimension $1$, then
$R^1f_*\mathcal{O}_X$ is a finite locally free $\mathcal{O}_S$-module
whose formation commutes with arbitrary base change.
\end{lemma}

\begin{proof}
This follows from Varieties, Lemma
\ref{varieties-lemma-proper-geometrically-reduced-global-sections}
and Lemmas \ref{lemma-pre-genus-curves} and
\ref{lemma-geometrically-reduced-connected-1-curves}.
The final statement follows from Lemma \ref{lemma-genus}.
\end{proof}

\begin{lemma}
\label{lemma-one-piece-per-genus}
There is a decomposition into open and closed substacks
$$
\Curvesstack^{grc, 1} = \coprod\nolimits_{g \geq 0} \Curvesstack^{grc, 1}_g
$$
where each $\Curvesstack^{grc, 1}_g$ is characterized as follows:
\begin{enumerate}
\item given a family of curves $f : X \to S$ the following are equivalent
\begin{enumerate}
\item the classifying morphism $S \to \Curvesstack$ factors
through $\Curvesstack^{grc, 1}_g$,
\item the geometric fibres of the morphism $f : X \to S$ are
reduced, connected, of dimension $1$ and
$R^1f_*\mathcal{O}_X$ is a locally free $\mathcal{O}_S$-module
of rank $g$,
\end{enumerate}
\item given a scheme $X$ proper over a field $k$ with $\dim(X) \leq 1$
the following are equivalent
\begin{enumerate}
\item the classifying morphism $\Spec(k) \to \Curvesstack$ factors
through $\Curvesstack^{grc, 1}_g$,
\item $X$ is geometrically reduced, geometrically connected,
has dimension $1$, and has genus $g$.
\end{enumerate}
\end{enumerate}
\end{lemma}

\begin{proof}
First proof: set
$\Curvesstack^{grc, 1}_g = \Curvesstack^{grc, 1} \cap \Curvesstack_g$
and combine Lemmas \ref{lemma-geomredcon-in-h0-1} and
\ref{lemma-pre-genus-one-piece-per-genus}.
Second proof:
The existence of the decomposition into open and closed substacks
follows immediately from the discussion in Section \ref{section-open}
and Lemma \ref{lemma-geomredcon-in-h0-1}.
This proves the characterization in (1).
The characterization in (2) follows as well since
the genus of a geometrically reduced and connected
proper $1$-dimensional scheme $X/k$ is defined
(Algebraic Curves, Definition \ref{curves-definition-genus} and
Varieties, Lemma
\ref{varieties-lemma-proper-geometrically-reduced-global-sections})
and is equal to $\dim_k H^1(X, \mathcal{O}_X)$.
\end{proof}







\section{Gorenstein curves}
\label{section-gorenstein}

\noindent
There is an open substack of $\Curvesstack$ parametrizing
the Gorenstein ``curves''.

\begin{lemma}
\label{lemma-gorenstein-curves}
There exist an open substack $\Curvesstack^{Gorenstein} \subset \Curvesstack$
such that
\begin{enumerate}
\item given a family of curves $X \to S$ the following are equivalent
\begin{enumerate}
\item the classifying morphism $S \to \Curvesstack$ factors
through $\Curvesstack^{Gorenstein}$,
\item the morphism $X \to S$ is Gorenstein,
\end{enumerate}
\item given a scheme $X$ proper over a field $k$ with $\dim(X) \leq 1$
the following are equivalent
\begin{enumerate}
\item the classifying morphism $\Spec(k) \to \Curvesstack$ factors
through $\Curvesstack^{Gorenstein}$,
\item $X$ is Gorenstein.
\end{enumerate}
\end{enumerate}
\end{lemma}

\begin{proof}
Let $f : X \to S$ be a family of curves. By
More on Morphisms of Spaces, Lemma
\ref{spaces-more-morphisms-lemma-flat-finite-presentation-gorenstein-open}
the set
$$
W = \{x \in |X| : f \text{ is Gorenstein at }x\}
$$
is open in $|X|$ and formation of this open commutes with arbitrary
base change. Since $f$ is proper the subset
$$
S' = S \setminus f(|X| \setminus W)
$$
of $S$ is open and $X \times_S S' \to S'$ is Gorenstein.
Moreover, formation of $S'$ commutes with arbitrary base
change because this is true for $W$
Thus we get the open substack with the desired properties
by the method discussed in Section \ref{section-open}.
\end{proof}

\begin{lemma}
\label{lemma-gorenstein-1-curves}
There exist an open substack
$\Curvesstack^{Gorenstein, 1} \subset \Curvesstack$ such that
\begin{enumerate}
\item given a family of curves $X \to S$ the following are equivalent
\begin{enumerate}
\item the classifying morphism $S \to \Curvesstack$ factors
through $\Curvesstack^{Gorenstein, 1}$,
\item the morphism $X \to S$ is Gorenstein and has
relative dimension $1$ (Morphisms of Spaces, Definition
\ref{spaces-morphisms-definition-relative-dimension}),
\end{enumerate}
\item given a scheme $X$ proper over a field $k$ with $\dim(X) \leq 1$
the following are equivalent
\begin{enumerate}
\item the classifying morphism $\Spec(k) \to \Curvesstack$ factors
through $\Curvesstack^{Gorenstein, 1}$,
\item $X$ is Gorenstein and $X$ is equidimensional of
dimension $1$.
\end{enumerate}
\end{enumerate}
\end{lemma}

\begin{proof}
Recall that a Gorenstein scheme is Cohen-Macaulay
(Duality for Schemes, Lemma \ref{duality-lemma-gorenstein-CM})
and that
a Gorenstein morphism is a Cohen-Macaulay morphism
(Duality for Schemes, Lemma \ref{duality-lemma-gorenstein-CM-morphism}.
Thus we can set
$\Curvesstack^{Gorenstein, 1}$ equal to the intersection
of $\Curvesstack^{Gorenstein}$ and $\Curvesstack^{CM, 1}$
inside of $\Curvesstack$ and use
Lemmas \ref{lemma-gorenstein-curves} and \ref{lemma-CM-1-curves}.
\end{proof}






\section{Local complete intersection curves}
\label{section-lci}

\noindent
There is an open substack of $\Curvesstack$ parametrizing
the local complete intersection ``curves''.

\begin{lemma}
\label{lemma-lci-curves}
There exist an open substack $\Curvesstack^{lci} \subset \Curvesstack$
such that
\begin{enumerate}
\item given a family of curves $X \to S$ the following are equivalent
\begin{enumerate}
\item the classifying morphism $S \to \Curvesstack$ factors through
$\Curvesstack^{lci}$,
\item $X \to S$ is a local complete intersection morphism, and
\item $X \to S$ is a syntomic morphism.
\end{enumerate}
\item given $X$ a proper scheme over a field $k$ of dimension $\leq 1$
the following are equivalent
\begin{enumerate}
\item the classifying morphism $\Spec(k) \to \Curvesstack$ factors
through $\Curvesstack^{lci}$,
\item $X$ is a local complete intersection over $k$.
\end{enumerate}
\end{enumerate}
\end{lemma}

\begin{proof}
Recall that being a syntomic morphism is the same as being flat and
a local complete intersection morphism, see
More on Morphisms of Spaces, Lemma \ref{spaces-more-morphisms-lemma-flat-lci}.
Thus (1)(b) is equivalent to (1)(c).
In Section \ref{section-open} we have seen
it suffices to show that given a family of curves
$f : X \to S$, there is an open subscheme $S' \subset S$
such that $S' \times_S X \to S'$ is a local complete intersection
morphism and such that formation of $S'$ commutes with arbitrary base change.
This follows from the more general
More on Morphisms of Spaces, Lemma \ref{spaces-more-morphisms-lemma-where-lci}.
\end{proof}




\section{Curves with isolated singularities}
\label{section-curves-isolated}

\noindent
We can look at the open substack of $\Curvesstack$
parametrizing ``curves'' with only a finite number of singular
points (these may correspond to $0$-dimensional components
in our setup).

\begin{lemma}
\label{lemma-isolated-sings-curves}
There exist an open substack
$\Curvesstack^{+} \subset \Curvesstack$
such that
\begin{enumerate}
\item given a family of curves $X \to S$ the following are equivalent
\begin{enumerate}
\item the classifying morphism $S \to \Curvesstack$ factors through
$\Curvesstack^{+}$,
\item the singular locus of $X \to S$ endowed
with any/some closed subspace structure is finite over $S$.
\end{enumerate}
\item given $X$ a proper scheme over a field $k$ of dimension $\leq 1$
the following are equivalent
\begin{enumerate}
\item the classifying morphism $\Spec(k) \to \Curvesstack$ factors
through $\Curvesstack^{+}$,
\item $X \to \Spec(k)$ is smooth except at finitely many points.
\end{enumerate}
\end{enumerate}
\end{lemma}

\begin{proof}
To prove the lemma it suffices to show that given a family of curves
$f : X \to S$, there is an open subscheme $S' \subset S$
such that the fibre of $S' \times_S X \to S'$ have property (2).
(Formation of the open will automatically commute with base change.)
By definition the locus $T \subset |X|$ of points where $X \to S$
is not smooth is closed. Let $Z \subset X$ be the closed subspace
given by the reduced induced algebraic space structure on $T$
(Properties of Spaces, Definition
\ref{spaces-properties-definition-reduced-induced-space}).
Now if $s \in S$ is a point where $Z_s$ is finite, then there
is an open neighbourhood $U_s \subset S$ of $s$ such that
$Z \cap f^{-1}(U_s) \to U_s$ is finite, see
More on Morphisms of Spaces, Lemma
\ref{spaces-more-morphisms-lemma-proper-finite-fibre-finite-in-neighbourhood}.
This proves the lemma.
\end{proof}




\section{The smooth locus of the stack of curves}
\label{section-smooth}

\noindent
The morphism
$$
\Curvesstack \longrightarrow \Spec(\mathbf{Z})
$$
is smooth over a maximal open substack, see
Morphisms of Stacks, Lemma \ref{stacks-morphisms-lemma-where-smooth}.
We want to give a criterion for when a curve is in this locus.
We will do this using a bit of deformation theory.

\medskip\noindent
Let $k$ be a field. Let $X$ be a proper scheme of dimension $\leq 1$ over $k$.
Choose a Cohen ring $\Lambda$ for $k$, see
Algebra, Lemma \ref{algebra-lemma-cohen-rings-exist}.
Then we are in the situation described in
Deformation Problems, Example \ref{examples-defos-example-schemes} and
Lemma \ref{examples-defos-lemma-schemes-RS}.
Thus we obtain a deformation category $\Deformationcategory_X$
on the category $\mathcal{C}_\Lambda$ of Artinian local
$\Lambda$-algebras with residue field $k$.

\begin{lemma}
\label{lemma-in-smooth-locus}
In the situation above the following are equivalent
\begin{enumerate}
\item the classifying morphism $\Spec(k) \to \Curvesstack$ factors
through the open where $\Curvesstack \to \Spec(\mathbf{Z})$ is smooth,
\item the deformation category $\Deformationcategory_X$ is unobstructed.
\end{enumerate}
\end{lemma}

\begin{proof}
Since $\Curvesstack \longrightarrow \Spec(\mathbf{Z})$ is locally
of finite presentation (Lemma \ref{lemma-curves-qs-lfp})
formation of the open substack where
$\Curvesstack \longrightarrow \Spec(\mathbf{Z})$ is smooth commutes with
flat base change
(Morphisms of Stacks, Lemma \ref{stacks-morphisms-lemma-where-smooth}).
Since the Cohen ring $\Lambda$ is flat over $\mathbf{Z}$,
we may work over $\Lambda$. In other words, we are trying to prove that
$$
\Lambda\text{-}\Curvesstack \longrightarrow \Spec(\Lambda)
$$
is smooth in an open neighbourhood of the point
$x_0 : \Spec(k) \to \Lambda\text{-}\Curvesstack$
defined by $X/k$ if and only if $\Deformationcategory_X$ is unobstructed.

\medskip\noindent
The lemma now follows from
Geometry of Stacks, Lemma \ref{stacks-geometry-lemma-characterize-smoothness}
and the equality
$$
\Deformationcategory_X =
\mathcal{F}_{\Lambda\text{-}\Curvesstack, k, x_0}
$$
This equality is not completely trivial to establish. Namely, on the left
hand side we have the deformation category classifying all flat deformations
$Y \to \Spec(A)$ of $X$ as a scheme over $A \in \Ob(\mathcal{C}_\Lambda)$.
On the right hand side we have the deformation category classifying all
flat morphisms $Y \to \Spec(A)$ with special fibre $X$
where $Y$ is an algebraic space and
$Y \to \Spec(A)$ is proper, of finite presentation, and of
relative dimension $\leq 1$. Since $A$ is Artinian, we find
that $Y$ is a scheme for example by Spaces over Fields, Lemma
\ref{spaces-over-fields-lemma-codim-1-point-in-schematic-locus}.
Thus it remains to show: a flat deformation $Y \to \Spec(A)$ of
$X$ as a scheme over an Artinian local ring $A$ with residue field $k$
is proper, of finite presentation, and of relative dimension $\leq 1$.
Relative dimension is defined in terms of fibres and hence holds
automatically for $Y/A$ since it holds for $X/k$.
The morphism $Y \to \Spec(A)$ is proper and locally of finite presentation
as this is true for $X \to \Spec(k)$, see
More on Morphisms, Lemma \ref{more-morphisms-lemma-deform-property}.
\end{proof}

\noindent
Here is a ``large'' open of the stack of curves which is contained
in the smooth locus.

\begin{lemma}
\label{lemma-big-smooth-part-curves}
The open substack
$$
\Curvesstack^{lci+} =
\Curvesstack^{lci} \cap \Curvesstack^{+}
\subset \Curvesstack
$$
has the following properties
\begin{enumerate}
\item $\Curvesstack^{lci+} \to \Spec(\mathbf{Z})$ is smooth,
\item given a family of curves $X \to S$ the following are equivalent
\begin{enumerate}
\item the classifying morphism $S \to \Curvesstack$ factors through
$\Curvesstack^{lci+}$,
\item $X \to S$ is a local complete intersection morphism and
the singular locus of $X \to S$ endowed with any/some closed subspace
structure is finite over $S$,
\end{enumerate}
\item given $X$ a proper scheme over a field $k$ of dimension $\leq 1$
the following are equivalent
\begin{enumerate}
\item the classifying morphism $\Spec(k) \to \Curvesstack$ factors
through $\Curvesstack^{lci+}$,
\item $X$ is a local complete intersection over $k$ and
$X \to \Spec(k)$ is smooth except at finitely many points.
\end{enumerate}
\end{enumerate}
\end{lemma}

\begin{proof}
If we can show that there is an open substack $\Curvesstack^{lci+}$
whose points are characterized by (2), then we see that
(1) holds by combining Lemma \ref{lemma-in-smooth-locus} with
Deformation Problems, Lemma \ref{examples-defos-lemma-curve-isolated-lci}.
Since
$$
\Curvesstack^{lci+} = \Curvesstack^{lci} \cap \Curvesstack^{+}
$$
inside $\Curvesstack$, we conclude by
Lemmas \ref{lemma-lci-curves} and \ref{lemma-isolated-sings-curves}.
\end{proof}




\section{Smooth curves}
\label{section-smooth-curves}

\noindent
In this section we study open substacks of $\Curvesstack$
parametrizing smooth ``curves''.

\begin{lemma}
\label{lemma-smooth-curves}
There exist an open substacks
$$
\Curvesstack^{smooth, 1} \subset \Curvesstack^{smooth} \subset \Curvesstack
$$
such that
\begin{enumerate}
\item given a family of curves $f : X \to S$ the following are equivalent
\begin{enumerate}
\item the classifying morphism $S \to \Curvesstack$ factors
through $\Curvesstack^{smooth}$, resp.\ $\Curvesstack^{smooth, 1}$,
\item $f$ is smooth, resp.\ smooth of relative dimension $1$,
\end{enumerate}
\item given $X$ a scheme proper over a field $k$ with
$\dim(X) \leq 1$ the following are equivalent
\begin{enumerate}
\item the classifying morphism $\Spec(k) \to \Curvesstack$
factors through $\Curvesstack^{smooth}$, resp.\ $\Curvesstack^{smooth, 1}$,
\item $X$ is smooth over $k$, resp.\ $X$ is smooth over $k$ and
$X$ is equidimensional of dimension $1$.
\end{enumerate}
\end{enumerate}
\end{lemma}

\begin{proof}
To prove the statements regarding $\Curvesstack^{smooth}$
it suffices to show that given a family of curves
$f : X \to S$, there is an open subscheme $S' \subset S$
such that $S' \times_S X \to S'$ is smooth and such that the
formation of this open commutes with base change.
We know that there is a maximal open $U \subset X$ such
that $U \to S$ is smooth and that formation of $U$ commutes
with arbitrary base change, see
Morphisms of Spaces, Lemma \ref{spaces-morphisms-lemma-where-smooth}.
If $T = |X| \setminus |U|$ then $f(T)$ is closed in $S$ as $f$ is proper.
Setting $S' = S \setminus f(T)$ we obtain the desired open.

\medskip\noindent
Let $f : X \to S$ be a family of curves with $f$ smooth.
Then the fibres $X_s$ are smooth over $\kappa(s)$ and hence
Cohen-Macaulay (for example you can see this using
Algebra, Lemmas \ref{algebra-lemma-smooth-over-field} and
\ref{algebra-lemma-lci-CM}). Thus we see that we may set
$$
\Curvesstack^{smooth, 1} = \Curvesstack^{smooth} \cap
\Curvesstack^{CM, 1}
$$
and the desired equivalences follow from what we've already
shown for $\Curvesstack^{smooth}$ and Lemma \ref{lemma-CM-1-curves}.
\end{proof}

\begin{lemma}
\label{lemma-smooth-curves-smooth}
The morphism $\Curvesstack^{smooth} \to \Spec(\mathbf{Z})$ is smooth.
\end{lemma}

\begin{proof}
Follows immediately from the observation that
$\Curvesstack^{smooth} \subset \Curvesstack^{lci+}$
and Lemma \ref{lemma-big-smooth-part-curves}.
\end{proof}

\begin{lemma}
\label{lemma-smooth-curves-h0}
There exist an open substack
$\Curvesstack^{smooth, h0} \subset \Curvesstack$
such that
\begin{enumerate}
\item given a family of curves $f : X \to S$ the following are equivalent
\begin{enumerate}
\item the classifying morphism $S \to \Curvesstack$ factors
through $\Curvesstack^{smooth}$,
\item $f_*\mathcal{O}_X = \mathcal{O}_S$, this holds after any base change,
and $f$ is smooth of relative dimension $1$,
\end{enumerate}
\item given $X$ a scheme proper over a field $k$ with
$\dim(X) \leq 1$ the following are equivalent
\begin{enumerate}
\item the classifying morphism $\Spec(k) \to \Curvesstack$
factors through $\Curvesstack^{smooth, h0}$,
\item $X$ is smooth, $\dim(X) = 1$, and $k = H^0(X, \mathcal{O}_X)$,
\item $X$ is smooth, $\dim(X) = 1$, and $X$ is geometrically connected,
\item $X$ is smooth, $\dim(X) = 1$, and $X$ is geometrically integral, and
\item $X_{\overline{k}}$ is a smooth curve.
\end{enumerate}
\end{enumerate}
\end{lemma}

\begin{proof}
If we set
$$
\Curvesstack^{smooth, h0} = \Curvesstack^{smooth} \cap
\Curvesstack^{h0, 1}
$$
then we see that (1) holds by
Lemmas \ref{lemma-pre-genus-curves} and \ref{lemma-smooth-curves}.
In fact, this also gives the equivalence of (2)(a) and (2)(b).
To finish the proof we have to show that
(2)(b) is equivalent to each of (2)(c), (2)(d), and (2)(e).

\medskip\noindent
A smooth scheme over a field is geometrically normal
(Varieties, Lemma \ref{varieties-lemma-smooth-geometrically-normal}),
smoothness is preserved under base change
(Morphisms, Lemma \ref{morphisms-lemma-base-change-smooth}), and
being smooth is fpqc local on the target
(Descent, Lemma \ref{descent-lemma-descending-property-smooth}).
Keeping this in mind, the equivalence of (2)(b), (2)(c), 2(d), and (2)(e)
follows from Varieties, Lemma \ref{varieties-lemma-geometrically-normal-stein}.
\end{proof}

\begin{definition}
\label{definition-deligne-mumford-smooth}
\begin{reference}
\cite{DM}
\end{reference}
We denote $\mathcal{M}$ and we name it the
{\it moduli stack of smooth proper curves}
the algebraic stack
$\Curvesstack^{smooth, h0}$ parametrizing families of curves
introduced in Lemma \ref{lemma-smooth-curves-h0}.
For $g \geq 0$ we denote $\mathcal{M}_g$ and we name it the
{\it moduli stack of smooth proper curves of genus $g$}
the algebraic stack introduced in
Lemma \ref{lemma-smooth-one-piece-per-genus}.
\end{definition}

\noindent
Here is the obligatory lemma.

\begin{lemma}
\label{lemma-smooth-one-piece-per-genus}
There is a decomposition into open and closed substacks
$$
\mathcal{M} = \coprod\nolimits_{g \geq 0} \mathcal{M}_g
$$
where each $\mathcal{M}_g$ is characterized as follows:
\begin{enumerate}
\item given a family of curves $f : X \to S$ the following are equivalent
\begin{enumerate}
\item the classifying morphism $S \to \Curvesstack$ factors
through $\mathcal{M}_g$,
\item $X \to S$ is smooth, $f_*\mathcal{O}_X = \mathcal{O}_S$,
this holds after any base change, and $R^1f_*\mathcal{O}_X$
is a locally free $\mathcal{O}_S$-module of rank $g$,
\end{enumerate}
\item given $X$ a scheme proper over a field $k$ with
$\dim(X) \leq 1$ the following are equivalent
\begin{enumerate}
\item the classifying morphism $\Spec(k) \to \Curvesstack$
factors through $\mathcal{M}_g$,
\item $X$ is smooth, $\dim(X) = 1$, $k = H^0(X, \mathcal{O}_X)$,
and $X$ has genus $g$,
\item $X$ is smooth, $\dim(X) = 1$, $X$ is geometrically connected, and
$X$ has genus $g$,
\item $X$ is smooth, $\dim(X) = 1$, $X$ is geometrically integral, and
$X$ has genus $g$, and
\item $X_{\overline{k}}$ is a smooth curve of genus $g$.
\end{enumerate}
\end{enumerate}
\end{lemma}

\begin{proof}
Combine Lemmas \ref{lemma-smooth-curves-h0} and
\ref{lemma-pre-genus-one-piece-per-genus}.
You can also use
Lemma \ref{lemma-one-piece-per-genus}
instead.
\end{proof}

\begin{lemma}
\label{lemma-smooth-curves-h0-smooth}
The morphisms $\mathcal{M} \to \Spec(\mathbf{Z})$ and
$\mathcal{M}_g \to \Spec(\mathbf{Z})$
are smooth.
\end{lemma}

\begin{proof}
Since $\mathcal{M}$ is an open substack of
$\Curvesstack^{lci+}$ this follows from
Lemma \ref{lemma-big-smooth-part-curves}.
\end{proof}





\section{Density of smooth curves}
\label{section-smooth-is-dense}

\noindent
The title of this section is misleading as we don't claim
$\Curvesstack^{smooth}$ is dense in $\Curvesstack$.
In fact, this is false as was shown by Mumford in \cite{PathologiesIV}.
However, we will see that the smooth ``curves'' are dense
in a large open.

\begin{lemma}
\label{lemma-smooth-dense}
The inclusion
$$
|\Curvesstack^{smooth}| \subset |\Curvesstack^{lci+}|
$$
is that of an open dense subset.
\end{lemma}

\begin{proof}
By the very construction of the topology on
$|\Curvesstack^{lci+}|$ in
Properties of Stacks, Section \ref{stacks-properties-section-points}
we find that $|\Curvesstack^{smooth}|$
is an open subset. Let $\xi \in |\Curvesstack^{lci+}|$ be a point.
Then there exists a field $k$ and a scheme $X$ over $k$
with $X$ proper over $k$, with $\dim(X) \leq 1$,
with $X$ a local complete intersection over $k$, and
with $X$ is smooth over $k$ except at finitely many points, such
that $\xi$ is the equivalence class of the
classifying morphism $\Spec(k) \to \Curvesstack^{lci+}$ determined by $X$.
See Lemma \ref{lemma-big-smooth-part-curves}.
By Deformation Problems, Lemma
\ref{examples-defos-lemma-smoothing-proper-curve-isolated-lci}
there exists a flat projective morphism $Y \to \Spec(k[[t]])$
whose generic fibre is smooth and whose special fibre is
isomorphic to $X$. Consider the classifying morphism
$$
\Spec(k[[t]]) \longrightarrow \Curvesstack^{lci+}
$$
determined by $Y$. The image of the closed point is $\xi$
and the image of the generic point is in $|\Curvesstack^{smooth}|$.
Since the generic point specializes to the closed point in
$|\Spec(k[[t]])|$ we conclude that $\xi$ is in the closure
of $|\Curvesstack^{smooth}|$ as desired.
\end{proof}






\section{Nodal curves}
\label{section-nodal-curves}

\noindent
In algebraic geometry a special role is played by nodal curves.
We suggest the reader take a brief look at some of the discussion
in Algebraic Curves, Sections \ref{curves-section-nodal} and
\ref{curves-section-families-nodal}
and More on Morphisms of Spaces, Section
\ref{spaces-more-morphisms-section-families-nodal}.

\begin{lemma}
\label{lemma-nodal-curves}
There exist an open substack $\Curvesstack^{nodal} \subset \Curvesstack$
such that
\begin{enumerate}
\item given a family of curves $f : X \to S$ the following are equivalent
\begin{enumerate}
\item the classifying morphism $S \to \Curvesstack$ factors
through $\Curvesstack^{nodal}$,
\item $f$ is at-worst-nodal of relative dimension $1$,
\end{enumerate}
\item given $X$ a scheme proper over a field $k$ with
$\dim(X) \leq 1$ the following are equivalent
\begin{enumerate}
\item the classifying morphism $\Spec(k) \to \Curvesstack$ factors
through $\Curvesstack^{nodal}$,
\item the singularities of $X$ are at-worst-nodal and $X$
is equidimensional of dimension $1$.
\end{enumerate}
\end{enumerate}
\end{lemma}

\begin{proof}
In fact, it suffices to show that given a family of curves
$f : X \to S$, there is an open subscheme $S' \subset S$
such that $S' \times_S X \to S'$ is at-worst-nodal of relative dimension $1$
and such that formation of $S'$ commutes with arbitrary base change.
By More on Morphisms of Spaces, Lemma
\ref{spaces-more-morphisms-lemma-locus-where-nodal}
there is a maximal open subspace $X' \subset X$ such
that $f|_{X'} : X' \to S$ is at-worst-nodal of relative dimension $1$.
Moreover, formation of $X'$ commutes with base change.
Hence we can take
$$
S' = S \setminus |f|(|X| \setminus |X'|)
$$
This is open because a proper morphism is universally closed by
definition.
\end{proof}

\begin{lemma}
\label{lemma-nodal-curves-smooth}
The morphism $\Curvesstack^{nodal} \to \Spec(\mathbf{Z})$ is smooth.
\end{lemma}

\begin{proof}
Follows immediately from the observation that
$\Curvesstack^{nodal} \subset \Curvesstack^{lci+}$
and Lemma \ref{lemma-big-smooth-part-curves}.
\end{proof}




\section{The relative dualizing sheaf}
\label{section-relative-dualizing}

\noindent
This section serves mainly
to introduce notation in the case of families of curves.
Most of the work has already been done in the chapter on duality.

\medskip\noindent
Let $f : X \to S$ be a family of curves. There exists an object
$\omega_{X/S}^\bullet$ in $D_\QCoh(\mathcal{O}_X)$,
called the {\it relative dualizing complex}, having the following
property: for every base change diagram
$$
\xymatrix{
X_U \ar[d]_{f'} \ar[r]_{g'} & X \ar[d]^f \\
U \ar[r]^g & S
}
$$
with $U = \Spec(A)$ affine the complex
$\omega_{X_U/U}^\bullet = L(g')^*\omega_{X/S}^\bullet$
represents the functor
$$
D_\QCoh(\mathcal{O}_{X_U}) \longrightarrow \text{Mod}_A,\quad
K \longmapsto \Hom_U(Rf_*K, \mathcal{O}_U)
$$
More precisely, let $(\omega_{X/S}^\bullet, \tau)$
be the relative dualizing complex of the family as defined in
Duality for Spaces, Definition
\ref{spaces-duality-definition-relative-dualizing-proper-flat}.
Existence is shown in Duality for Spaces, Lemma
\ref{spaces-duality-lemma-existence-relative-dualizing}.
Moreover, formation of $(\omega_{X/S}^\bullet, \tau)$ commutes
with arbitrary base change (essentially by definition; a precise
reference is Duality for Spaces, Lemma
\ref{spaces-duality-lemma-base-change-relative-dualizing}).
From now on we will identify the base change of
$\omega_{X/S}^\bullet$ with the relative dualizing
complex of the base changed family without further mention.

\medskip\noindent
Let $\{S_i \to S\}$ be an \'etale covering with $S_i$ affine such that
$X_i = X \times_S S_i$ is a scheme, see Lemma \ref{lemma-etale-locally-scheme}.
By Duality for Spaces, Lemma \ref{spaces-duality-lemma-compare}
we find that $\omega_{X_i/S_i}^\bullet$ agrees with
the relative dualizing complex for the proper, flat, and
finitely presented morphism $f_i : X_i \to S_i$ of schemes
discussed in Duality for Schemes, Remark
\ref{duality-remark-relative-dualizing-complex}.
Thus to prove a property of $\omega_{X/S}^\bullet$
which is \'etale local, we may assume $X \to S$ is a morphism of schemes
and use the theory developed in the chapter on duality for schemes.
More generally, for any base change of $X$ which is a scheme,
the relative dualizing complex agrees with the
relative dualizing complex of Duality for Schemes, Remark
\ref{duality-remark-relative-dualizing-complex}.
From now on we will use this identification without further mention.

\medskip\noindent
In particular, let $\Spec(k) \to S$ be a morphism where $k$ is a field.
Denote $X_k$ the base change (this is a scheme by
Spaces over Fields, Lemma
\ref{spaces-over-fields-lemma-codim-1-point-in-schematic-locus}).
Then $\omega_{X_k/k}^\bullet$ is isomorphic
to the complex $\omega_{X_k}^\bullet$ of
Algebraic Curves, Lemma \ref{curves-lemma-duality-dim-1}
(both represent the same functor and so we can use the Yoneda lemma,
but really this holds because of the remarks above).
We conclude that the cohomology sheaves
$H^i(\omega_{X_k/k}^\bullet)$
are nonzero only for $i = 0, -1$.
If $X_k$ is Cohen-Macaulay and equidimensional of dimension $1$,
then we only have $H^{-1}$ and if $X_k$ is in addition Gorenstein,
then $H^{-1}(\omega_{X_k/k})$ is invertible, see
Algebraic Curves, Lemmas \ref{curves-lemma-duality-dim-1-CM}
and \ref{curves-lemma-rr}.

\begin{lemma}
\label{lemma-CM-dualizing}
Let $X \to S$ be a family of curves with Cohen-Macaulay fibres
equidimensional of dimension $1$ (Lemma \ref{lemma-CM-1-curves}).
Then $\omega_{X/S}^\bullet = \omega_{X/S}[1]$ where $\omega_{X/S}$
is a pseudo-coherent $\mathcal{O}_X$-module flat over $S$ whose
formation commutes with arbitrary base change.
\end{lemma}

\begin{proof}
We urge the reader to deduce this directly from the discussion above
of what happens after base change to a field. Our proof will
use a somewhat cumbersome reduction to the Noetherian schemes case.

\medskip\noindent
Once we show $\omega_{X/S}^\bullet = \omega_{X/S}[1]$ with
$\omega_{X/S}$ flat over $S$, the statement on base change
will follow as we already know that formation of $\omega_{X/S}^\bullet$
commutes with arbitrary base change. Moreover, the pseudo-coherence
will be automatic as $\omega_{X/S}^\bullet$ is pseudo-coherent
by definition. Vanishing of the other cohomology sheaves and flatness 
may be checked \'etale locally. Thus we may assume $f : X \to S$
is a morphism of schemes with $S$ affine (see discussion above).
Write $S = \lim S_i$ as a cofiltered limit of affine schemes $S_i$
of finite type over $\mathbf{Z}$.
Since $\Curvesstack^{CM, 1}$ is locally of finite presentation over
$\mathbf{Z}$ (as an open substack of $\Curvesstack$, see
Lemmas \ref{lemma-CM-1-curves} and \ref{lemma-curves-qs-lfp}),
we can find an $i$ and a family
of curves $X_i \to S_i$ whose pullback is $X \to S$
(Limits of Stacks, Lemma
\ref{stacks-limits-lemma-representable-by-spaces-limit-preserving}).
After increasing $i$ if necessary we may assume $X_i$ is a scheme,
see Limits of Spaces, Lemma \ref{spaces-limits-lemma-limit-is-scheme}.
Since formation of $\omega_{X/S}^\bullet$ commutes with
arbitrary base change, we may replace $S$ by $S_i$.
Doing so we may and do assume $S_i$ is Noetherian.
Then $f$ is clearly a Cohen-Macaulay morphism
(More on Morphisms, Definition \ref{more-morphisms-definition-CM})
by our assumption on the fibres.
Also then $\omega_{X/S}^\bullet = f^!\mathcal{O}_S$
by the very construction of $f^!$ in
Duality for Schemes, Section \ref{duality-section-upper-shriek}.
Thus the lemma by Duality for Schemes, Lemma
\ref{duality-lemma-affine-flat-Noetherian-CM}.
\end{proof}

\begin{definition}
\label{definition-relative-dualizing-sheaf}
Let $f : X \to S$ be a family of curves with Cohen-Macaulay fibres
equidimensional of dimension $1$ (Lemma \ref{lemma-CM-1-curves}).
Then the $\mathcal{O}_X$-module
$$
\omega_{X/S} = H^{-1}(\omega_{X/S}^\bullet)
$$
studied in Lemma \ref{lemma-CM-dualizing}
is called the {\it relative dualizing sheaf} of $f$.
\end{definition}

\noindent
In the situation of Definition \ref{definition-relative-dualizing-sheaf}
the relative dualizing sheaf $\omega_{X/S}$ has the following property
(which moreover characterizes it locally on $S$):
for every base change diagram
$$
\xymatrix{
X_U \ar[d]_{f'} \ar[r]_{g'} & X \ar[d]^f \\
U \ar[r]^g & S
}
$$
with $U = \Spec(A)$ affine the module $\omega_{X_U/U} = (g')^*\omega_{X/S}$
represents the functor
$$
\QCoh(\mathcal{O}_{X_U}) \longrightarrow \text{Mod}_A,\quad
\mathcal{F} \longmapsto \Hom_A(H^1(X, \mathcal{F}), A)
$$
This follows immediately from the corresponding property of the relative
dualizing complex given above. In particular, if $A = k$ is a field,
then we recover the dualizing module of $X_k$ as introduced and studied in
Algebraic Curves, Lemmas \ref{curves-lemma-duality-dim-1},
\ref{curves-lemma-duality-dim-1-CM}, and \ref{curves-lemma-rr}.

\begin{lemma}
\label{lemma-gorenstein-dualizing}
Let $X \to S$ be a family of curves with Gorenstein fibres
equidimensional of dimension $1$ (Lemma \ref{lemma-gorenstein-1-curves}).
Then the relative dualizing sheaf $\omega_{X/S}$ is an
invertible $\mathcal{O}_X$-module whose
formation commutes with arbitrary base change.
\end{lemma}

\begin{proof}
This is true because the pullback of the relative dualizing module
to a fibre is invertible by the discussion above. Alternatively, you
can argue exactly as in the proof of
Lemma \ref{lemma-CM-dualizing} and deduce the result from
Duality for Schemes, Lemma
\ref{duality-lemma-affine-flat-Noetherian-gorenstein}.
\end{proof}









\section{Prestable curves}
\label{section-prestable-curves}

\noindent
The following definition is equivalent to what appears to be the
generally accepted notion of a prestable family of curves.

\begin{definition}
\label{definition-prestable}
Let $f : X \to S$ be a family of curves. We say $f$ is a
{\it prestable family of curves} if
\begin{enumerate}
\item $f$ is at-worst-nodal of relative dimension $1$, and
\item $f_*\mathcal{O}_X = \mathcal{O}_S$ and this holds after
any base change\footnote{In fact, it suffices to require
$f_*\mathcal{O}_X = \mathcal{O}_S$ because the Stein factorization
of $f$ is \'etale in this case, see
More on Morphisms of Spaces, Lemma
\ref{spaces-more-morphisms-lemma-stein-factorization-etale}.
The condition may also be replaced by asking the geometric
fibres to be connected, see Lemma \ref{lemma-geomredcon-in-h0-1}.}.
\end{enumerate}
\end{definition}

\noindent
Let $X$ be a proper scheme over a field $k$ with $\dim(X) \leq 1$.
Then $X \to \Spec(k)$ is a family of curves and hence we can ask
whether or not it is prestable\footnote{We can't use the term
``prestable curve'' here because curve implies irreducible. See
discussion in Algebraic Curves, Section \ref{curves-section-families-nodal}.}
in the sense of the definition. Unwinding the definitions we see
the following are equivalent
\begin{enumerate}
\item $X$ is prestable,
\item the singularities of $X$ are at-worst-nodal, $\dim(X) = 1$,
and $k = H^0(X, \mathcal{O}_X)$,
\item $X_{\overline{k}}$ is connected and it is smooth over $\overline{k}$
apart from a finite number of nodes
(Algebraic Curves, Definition \ref{curves-definition-multicross}).
\end{enumerate}
This shows that our definition agrees with most definitions one finds
in the literature.

\begin{lemma}
\label{lemma-prestable-curves}
There exist an open substack $\Curvesstack^{prestable} \subset \Curvesstack$
such that
\begin{enumerate}
\item given a family of curves $f : X \to S$ the following are equivalent
\begin{enumerate}
\item the classifying morphism $S \to \Curvesstack$ factors
through $\Curvesstack^{prestable}$,
\item $X \to S$ is a prestable family of curves,
\end{enumerate}
\item given $X$ a scheme proper over a field $k$ with
$\dim(X) \leq 1$ the following are equivalent
\begin{enumerate}
\item the classifying morphism $\Spec(k) \to \Curvesstack$
factors through $\Curvesstack^{prestable}$,
\item the singularities of $X$ are at-worst-nodal, $\dim(X) = 1$,
and $k = H^0(X, \mathcal{O}_X)$.
\end{enumerate}
\end{enumerate}
\end{lemma}

\begin{proof}
Given a family of curves $X \to S$ we see that it is prestable if
and only if the classifying morphism factors both through
$\Curvesstack^{nodal}$ and $\Curvesstack^{h0, 1}$. An alternative
is to use $\Curvesstack^{grc, 1}$ (since a nodal curve is geometrically
reduced hence has $H^0$ equal to the ground field if and only if
it is connected). In a formula
$$
\Curvesstack^{prestable} =
\Curvesstack^{nodal} \cap \Curvesstack^{h0, 1} =
\Curvesstack^{nodal} \cap \Curvesstack^{grc, 1}
$$
Thus the lemma follows from
Lemmas \ref{lemma-pre-genus-curves} and \ref{lemma-nodal-curves}.
\end{proof}

\noindent
For each genus $g \geq 0$ we have the algebraic stack classifying
the prestable curves of genus $g$. In fact, from now on we will say
that $X \to S$ is a {\it prestable family of curves of genus $g$}
if and only if the classifying morphism $S \to \Curvesstack$ factors through
the open substack $\Curvesstack^{prestable}_g$ of
Lemma \ref{lemma-prestable-one-piece-per-genus}.

\begin{lemma}
\label{lemma-prestable-one-piece-per-genus}
There is a decomposition into open and closed substacks
$$
\Curvesstack^{prestable} = \coprod\nolimits_{g \geq 0}
\Curvesstack^{prestable}_g
$$
where each $\Curvesstack^{prestable}_g$ is characterized as follows:
\begin{enumerate}
\item given a family of curves $f : X \to S$ the following are equivalent
\begin{enumerate}
\item the classifying morphism $S \to \Curvesstack$ factors
through $\Curvesstack^{prestable}_g$,
\item $X \to S$ is a prestable family of curves and
$R^1f_*\mathcal{O}_X$ is a locally free $\mathcal{O}_S$-module of rank $g$,
\end{enumerate}
\item given $X$ a scheme proper over a field $k$ with
$\dim(X) \leq 1$ the following are equivalent
\begin{enumerate}
\item the classifying morphism $\Spec(k) \to \Curvesstack$
factors through $\Curvesstack^{prestable}_g$,
\item the singularities of $X$ are at-worst-nodal, $\dim(X) = 1$,
$k = H^0(X, \mathcal{O}_X)$, and the genus of $X$ is $g$.
\end{enumerate}
\end{enumerate}
\end{lemma}

\begin{proof}
Since we have seen that $\Curvesstack^{prestable}$ is contained
in $\Curvesstack^{h0, 1}$, this
follows from Lemmas \ref{lemma-prestable-curves} and
\ref{lemma-pre-genus-one-piece-per-genus}.
\end{proof}

\begin{lemma}
\label{lemma-prestable-curves-smooth}
The morphisms
$\Curvesstack^{prestable} \to \Spec(\mathbf{Z})$ and
$\Curvesstack^{prestable}_g \to \Spec(\mathbf{Z})$ are
smooth.
\end{lemma}

\begin{proof}
Since $\Curvesstack^{prestable}$ is an open substack of
$\Curvesstack^{nodal}$ this follows from
Lemma \ref{lemma-nodal-curves-smooth}.
\end{proof}




\section{Semistable curves}
\label{section-semistable-curves}

\noindent
The following lemma will help us understand families of semistable curves.

\begin{lemma}
\label{lemma-semistable}
Let $f : X \to S$ be a prestable family of curves of genus $g \geq 1$.
Let $s \in S$ be a point of the base scheme. Let $m \geq 2$.
The following are equivalent
\begin{enumerate}
\item $X_s$ does not have a rational tail
(Algebraic Curves, Example \ref{curves-example-rational-tail}), and
\item $f^*f_*\omega_{X/S}^{\otimes m} \to \omega_{X/S}^{\otimes m}$,
is surjective over $f^{-1}(U)$ for some $s \in U \subset S$ open.
\end{enumerate}
\end{lemma}

\begin{proof}
Assume (2). Using the material in Section \ref{section-relative-dualizing}
we conclude that $\omega_{X_s}^{\otimes m}$ is
globally generated. However, if $C \subset X_s$
is a rational tail, then $\deg(\omega_{X_s}|_C) < 0$ by
Algebraic Curves, Lemma \ref{curves-lemma-rational-tail-negative}
hence $H^0(C, \omega_{X_s}|_C) = 0$ by
Varieties, Lemma \ref{varieties-lemma-check-invertible-sheaf-trivial}
which contradicts the fact that it is globally generated.
This proves (1).

\medskip\noindent
Assume (1). First assume that $g \geq 2$. Assumption (1) 
implies $\omega_{X_s}^{\otimes m}$ is globally generated,
see Algebraic Curves, Lemma \ref{curves-lemma-contracting-rational-tails}.
Moreover, we have
$$
\Hom_{\kappa(s)}(H^1(X_s, \omega_{X_s}^{\otimes m}), \kappa(s)) =
H^0(X_s, \omega_{X_s}^{\otimes 1 - m})
$$
by duality, see Algebraic Curves, Lemma \ref{curves-lemma-duality-dim-1-CM}.
Since $\omega_{X_s}^{\otimes m}$ is globally generated we find
that the restriction to each irreducible component has nonegative degree.
Hence the restriction of $\omega_{X_s}^{\otimes 1 - m}$ to each
irreducible component has nonpositive degree. Since
$\deg(\omega_{X_s}^{\otimes 1 - m}) = (1 - m)(2g - 2) < 0$ by Riemann-Roch
(Algebraic Curves, Lemma \ref{curves-lemma-rr}) we conclude that the $H^0$
is zero by Varieties, Lemma \ref{varieties-lemma-no-sections-dual-nef}.
By cohomology and base change we conclude that
$$
E = Rf_*\omega_{X/S}^{\otimes m}
$$
is a perfect complex whose formation commutes with arbitrary base change
(Derived Categories of Spaces, Lemma
\ref{spaces-perfect-lemma-flat-proper-perfect-direct-image-general}).
The vanishing proved above tells us that $E \otimes^\mathbf{L} \kappa(s)$
is equal to $H^0(X_s, \omega_{X_s}^{\otimes m})$ placed in degree $0$.
After shrinking $S$ we find $E = f_*\omega_{X/S}^{\otimes m}$
is a locally free $\mathcal{O}_S$-module placed in degree $0$
(and its formation commutes with arbitrary base change as
we've already said), see Derived Categories of Spaces, Lemma
\ref{spaces-perfect-lemma-open-where-cohomology-in-degree-i-rank-r-geometric}.
The map $f^*f_*\omega_{X/S}^{\otimes m} \to \omega_{X/S}^{\otimes m}$
is surjective after restricting to $X_s$. Thus it is surjective in
an open neighbourhood of $X_s$. Since $f$ is proper, this open
neighbourhood contains $f^{-1}(U)$ for some open neighbourhood
$U$ of $s$ in $S$.

\medskip\noindent
Assume (1) and $g = 1$. By
Algebraic Curves, Lemma \ref{curves-lemma-contracting-rational-tails}
the assumption (1) means that $\omega_{X_s}$ is isomorphic to
$\mathcal{O}_{X_s}$. If we can show that after shrinking $S$
the invertible sheaf $\omega_{X/S}$ because trivial, then
we are done. We may assume $S$ is affine. After shrinking $S$
further, we can write
$$
Rf_*\mathcal{O}_X = (\mathcal{O}_S \xrightarrow{0} \mathcal{O}_S)
$$
sitting in degrees $0$ and $1$
compatibly with further base change, see Lemma \ref{lemma-genus}.
By duality this means that
$$
Rf_*\omega_{X/S} = (\mathcal{O}_S \xrightarrow{0} \mathcal{O}_S)
$$
sitting in degrees $0$ and $1$\footnote{Use that
$Rf_*\omega_{X/S}^\bullet =
Rf_*R\SheafHom_{\mathcal{O}_X}(\mathcal{O}_X. \omega_{X/S}^\bullet) =
R\SheafHom_{\mathcal{O}_S}(Rf_*\mathcal{O}_X, \mathcal{O}_S)$
by Duality for Spaces, Lemma \ref{spaces-duality-lemma-iso-on-RSheafHom} and
Remark \ref{spaces-duality-remark-iso-on-RSheafHom}
and then that $\omega_{X/S}^\bullet = \omega_{X/S}[1]$ by
our definitions in Section \ref{section-relative-dualizing}.}.
In particular we obtain an isomorphism $\mathcal{O}_S \to f_*\omega_{X/S}$
which is compatible with base change since
formation of $Rf_*\omega_{X/S}$ is compatible with base change
(see reference given above).
By adjointness, we get a global section $\sigma \in \Gamma(X, \omega_{X/S})$.
The restriction of this section to the fibre $X_s$
is nonzero (a basis element in fact) and as
$\omega_{X_s}$ is trivial on the fibres,
this section is nonwhere zero on $X_s$.
Thus it nowhere zero in
an open neighbourhood of $X_s$. Since $f$ is proper, this open
neighbourhood contains $f^{-1}(U)$ for some open neighbourhood
$U$ of $s$ in $S$.
\end{proof}

\noindent
Motivated by Lemma \ref{lemma-semistable} we make the following definition.

\begin{definition}
\label{definition-semistable}
Let $f : X \to S$ be a family of curves.
We say $f$ is a {\it semistable family of curves} if
\begin{enumerate}
\item $X \to S$ is a prestable family of curves, and
\item $X_s$ has genus $\geq 1$ and
does not have a rational tail for all $s \in S$.
\end{enumerate}
\end{definition}

\noindent
In particular, a prestable family of curves of genus $0$ is never
semistable.
Let $X$ be a proper scheme over a field $k$ with $\dim(X) \leq 1$.
Then $X \to \Spec(k)$ is a family of curves and hence we can ask
whether or not it is semistable. Unwinding the definitions we see
the following are equivalent
\begin{enumerate}
\item $X$ is semistable,
\item $X$ is prestable, has genus $\geq 1$, and does not have a rational tail,
\item $X_{\overline{k}}$ is connected, is smooth over $\overline{k}$
apart from a finite number of nodes, has genus $\geq 1$, and has no
irreducible component isomorphic to $\mathbf{P}^1_{\overline{k}}$
which meets the rest of $X_{\overline{k}}$ in only one point.
\end{enumerate}
To see the equivalence of (2) and (3) use that $X$ has no rational tails
if and only if $X_{\overline{k}}$ has no rational tails by
Algebraic Curves, Lemma \ref{curves-lemma-contracting-rational-tails}.
This shows that our definition agrees with most definitions one finds
in the literature.

\begin{lemma}
\label{lemma-semistable-curves}
There exist an open substack $\Curvesstack^{semistable} \subset \Curvesstack$
such that
\begin{enumerate}
\item given a family of curves $f : X \to S$ the following are equivalent
\begin{enumerate}
\item the classifying morphism $S \to \Curvesstack$ factors
through $\Curvesstack^{semistable}$,
\item $X \to S$ is a semistable family of curves,
\end{enumerate}
\item given $X$ a scheme proper over a field $k$ with
$\dim(X) \leq 1$ the following are equivalent
\begin{enumerate}
\item the classifying morphism $\Spec(k) \to \Curvesstack$
factors through $\Curvesstack^{semistable}$,
\item the singularities of $X$ are at-worst-nodal, $\dim(X) = 1$,
$k = H^0(X, \mathcal{O}_X)$, the genus of $X$ is $\geq 1$, and
$X$ has no rational tails,
\item the singularities of $X$ are at-worst-nodal, $\dim(X) = 1$,
$k = H^0(X, \mathcal{O}_X)$, and $\omega_{X_s}^{\otimes m}$ is
globally generated for $m \geq 2$.
\end{enumerate}
\end{enumerate}
\end{lemma}

\begin{proof}
The equivalence of (2)(b) and (2)(c) is
Algebraic Curves, Lemma \ref{curves-lemma-contracting-rational-tails}.
In the rest of the proof we will work with (2)(b)
in accordance with Definition \ref{definition-semistable}.

\medskip\noindent
By the discussion in Section \ref{section-open}
it suffices to look at families $f : X \to S$ of
prestable curves. By Lemma \ref{lemma-semistable}
we obtain the desired openness of the locus in question.
Formation of this open commutes with arbitrary base change,
because the (non)existence of rational tails is insensitive
to ground field extensions by
Algebraic Curves, Lemma \ref{curves-lemma-contracting-rational-tails}.
\end{proof}

\begin{lemma}
\label{lemma-semistable-one-piece-per-genus}
There is a decomposition into open and closed substacks
$$
\Curvesstack^{semistable} = \coprod\nolimits_{g \geq 1}
\Curvesstack^{semistable}_g
$$
where each $\Curvesstack^{semistable}_g$ is characterized as follows:
\begin{enumerate}
\item given a family of curves $f : X \to S$ the following are equivalent
\begin{enumerate}
\item the classifying morphism $S \to \Curvesstack$ factors
through $\Curvesstack^{semistable}_g$,
\item $X \to S$ is a semistable family of curves and
$R^1f_*\mathcal{O}_X$ is a locally free $\mathcal{O}_S$-module of rank $g$,
\end{enumerate}
\item given $X$ a scheme proper over a field $k$ with
$\dim(X) \leq 1$ the following are equivalent
\begin{enumerate}
\item the classifying morphism $\Spec(k) \to \Curvesstack$
factors through $\Curvesstack^{semistable}_g$,
\item the singularities of $X$ are at-worst-nodal, $\dim(X) = 1$,
$k = H^0(X, \mathcal{O}_X)$, the genus of $X$ is $g$, and $X$
has no rational tail,
\item the singularities of $X$ are at-worst-nodal, $\dim(X) = 1$,
$k = H^0(X, \mathcal{O}_X)$, the genus of $X$ is $g$, and
$\omega_{X_s}^{\otimes m}$ is globally generated for $m \geq 2$.
\end{enumerate}
\end{enumerate}
\end{lemma}

\begin{proof}
Combine Lemmas \ref{lemma-semistable-curves} and
\ref{lemma-prestable-one-piece-per-genus}.
\end{proof}

\begin{lemma}
\label{lemma-semistable-curves-smooth}
The morphisms
$\Curvesstack^{semistable} \to \Spec(\mathbf{Z})$ and
$\Curvesstack^{semistable}_g \to \Spec(\mathbf{Z})$
are smooth.
\end{lemma}

\begin{proof}
Since $\Curvesstack^{semistable}$ is an open substack of
$\Curvesstack^{nodal}$ this follows from
Lemma \ref{lemma-nodal-curves-smooth}.
\end{proof}







\section{Stable curves}
\label{section-stable-curves}

\noindent
The following lemma will help us understand families of stable curves.

\begin{lemma}
\label{lemma-stable}
Let $f : X \to S$ be a prestable family of curves of genus $g \geq 2$.
Let $s \in S$ be a point of the base scheme.
The following are equivalent
\begin{enumerate}
\item $X_s$ does not have a rational tail and does not have a
rational bridge
(Algebraic Curves, Examples
\ref{curves-example-rational-tail} and
\ref{curves-example-rational-bridge}), and
\item $\omega_{X/S}$ is ample on $f^{-1}(U)$ for some $s \in U \subset S$ open.
\end{enumerate}
\end{lemma}

\begin{proof}
Assume (2). Then $\omega_{X_s}$ is ample on $X_s$.
By Algebraic Curves, Lemmas \ref{curves-lemma-rational-tail-negative} and
\ref{curves-lemma-rational-bridge-zero}
we conclude that (1) holds (we also
use the characterization of ample invertible sheaves
in Varieties, Lemma
\ref{varieties-lemma-ampleness-in-terms-of-degrees-components}).

\medskip\noindent
Assume (1). Then $\omega_{X_s}$ is ample on $X_s$ by
Algebraic Curves, Lemmas \ref{curves-lemma-contracting-rational-bridges}.
We conclude by Descent on Spaces, Lemma
\ref{spaces-descent-lemma-ample-in-neighbourhood}.
\end{proof}

\noindent
Motivated by Lemma \ref{lemma-stable} we make the following definition.

\begin{definition}
\label{definition-stable}
Let $f : X \to S$ be a family of curves.
We say $f$ is a {\it stable family of curves} if
\begin{enumerate}
\item $X \to S$ is a prestable family of curves, and
\item $X_s$ has genus $\geq 2$ and does not have a rational tails
or bridges for all $s \in S$.
\end{enumerate}
\end{definition}

\noindent
In particular, a prestable family of curves of genus $0$ or $1$ is never
stable.
Let $X$ be a proper scheme over a field $k$ with $\dim(X) \leq 1$.
Then $X \to \Spec(k)$ is a family of curves and hence we can ask
whether or not it is stable. Unwinding the definitions we see
the following are equivalent
\begin{enumerate}
\item $X$ is stable,
\item $X$ is prestable, has genus $\geq 2$, does not have a rational tail,
and does not have a rational bridge,
\item $X$ is geometrically connected, is smooth over $k$
apart from a finite number of nodes, and $\omega_X$ is ample.
\end{enumerate}
To see the equivalence of (2) and (3) use
Lemma \ref{lemma-stable} above.
This shows that our definition agrees with most definitions one finds
in the literature.

\begin{lemma}
\label{lemma-stable-curves}
There exist an open substack $\Curvesstack^{stable} \subset \Curvesstack$
such that
\begin{enumerate}
\item given a family of curves $f : X \to S$ the following are equivalent
\begin{enumerate}
\item the classifying morphism $S \to \Curvesstack$ factors
through $\Curvesstack^{stable}$,
\item $X \to S$ is a stable family of curves,
\end{enumerate}
\item given $X$ a scheme proper over a field $k$ with
$\dim(X) \leq 1$ the following are equivalent
\begin{enumerate}
\item the classifying morphism $\Spec(k) \to \Curvesstack$
factors through $\Curvesstack^{stable}$,
\item the singularities of $X$ are at-worst-nodal, $\dim(X) = 1$,
$k = H^0(X, \mathcal{O}_X)$, the genus of $X$ is $\geq 2$, and
$X$ has no rational tails or bridges,
\item the singularities of $X$ are at-worst-nodal, $\dim(X) = 1$,
$k = H^0(X, \mathcal{O}_X)$, and $\omega_{X_s}$ is ample.
\end{enumerate}
\end{enumerate}
\end{lemma}

\begin{proof}
By the discussion in Section \ref{section-open}
it suffices to look at families $f : X \to S$ of
prestable curves. By Lemma \ref{lemma-stable}
we obtain the desired openness of the locus in question.
Formation of this open commutes with arbitrary base change,
either because the (non)existence of rational tails or bridges
is insensitive to ground field extensions by
Algebraic Curves, Lemmas
\ref{curves-lemma-contracting-rational-tails} and
\ref{curves-lemma-contracting-rational-bridges}
or because ampleness is insensitive to base field extensions by
Descent, Lemma \ref{descent-lemma-descending-property-ample}.
\end{proof}

\begin{definition}
\label{definition-deligne-mumford}
\begin{reference}
\cite{DM}
\end{reference}
We denote $\overline{\mathcal{M}}$ and we name the
{\it moduli stack of stable curves} the algebraic stack
$\Curvesstack^{stable}$ parametrizing stable families of curves
introduced in Lemma \ref{lemma-stable-curves}.
For $g \geq 2$ we denote $\overline{\mathcal{M}}_g$ and we name the
{\it moduli stack of stable curves of genus $g$}
the algebraic stack introduced in Lemma \ref{lemma-stable-one-piece-per-genus}.
\end{definition}

\noindent
Here is the obligatory lemma.

\begin{lemma}
\label{lemma-stable-one-piece-per-genus}
There is a decomposition into open and closed substacks
$$
\overline{\mathcal{M}} = \coprod\nolimits_{g \geq 2} \overline{\mathcal{M}}_g
$$
where each $\overline{\mathcal{M}}_g$ is characterized as follows:
\begin{enumerate}
\item given a family of curves $f : X \to S$ the following are equivalent
\begin{enumerate}
\item the classifying morphism $S \to \Curvesstack$ factors
through $\overline{\mathcal{M}}_g$,
\item $X \to S$ is a stable family of curves and
$R^1f_*\mathcal{O}_X$ is a locally free $\mathcal{O}_S$-module of rank $g$,
\end{enumerate}
\item given $X$ a scheme proper over a field $k$ with
$\dim(X) \leq 1$ the following are equivalent
\begin{enumerate}
\item the classifying morphism $\Spec(k) \to \Curvesstack$
factors through $\overline{\mathcal{M}}_g$,
\item the singularities of $X$ are at-worst-nodal, $\dim(X) = 1$,
$k = H^0(X, \mathcal{O}_X)$, the genus of $X$ is $g$, and $X$
has no rational tails or bridges.
\item the singularities of $X$ are at-worst-nodal, $\dim(X) = 1$,
$k = H^0(X, \mathcal{O}_X)$, the genus of $X$ is $g$, and
$\omega_{X_s}$ is ample.
\end{enumerate}
\end{enumerate}
\end{lemma}

\begin{proof}
Combine Lemmas \ref{lemma-stable-curves} and
\ref{lemma-prestable-one-piece-per-genus}.
\end{proof}

\begin{lemma}
\label{lemma-stable-curves-smooth}
The morphisms
$\overline{\mathcal{M}} \to \Spec(\mathbf{Z})$ and
$\overline{\mathcal{M}}_g \to \Spec(\mathbf{Z})$
are smooth.
\end{lemma}

\begin{proof}
Since $\overline{\mathcal{M}}$ is an open substack of
$\Curvesstack^{nodal}$ this follows from
Lemma \ref{lemma-nodal-curves-smooth}.
\end{proof}

\begin{lemma}
\label{lemma-stable-curves-deligne-mumford}
The stacks $\overline{\mathcal{M}}$ and
$\overline{\mathcal{M}}_g$
are open substacks of $\Curvesstack^{DM}$.
In particular, $\overline{\mathcal{M}}$ and
$\overline{\mathcal{M}}_g$ are DM
(Morphisms of Stacks, Definition
\ref{stacks-morphisms-definition-absolute-separated})
as well as Deligne-Mumford stacks
(Algebraic Stacks, Definition \ref{algebraic-definition-deligne-mumford}).
\end{lemma}

\begin{proof}
Proof of the first assertion.
Let $X$ be a scheme proper over a field $k$ whose singularities
are at-worst-nodal, $\dim(X) = 1$, $k = H^0(X, \mathcal{O}_X)$,
the genus of $X$ is $\geq 2$, and $X$ has no rational tails or bridges.
We have to show that the classifying morphism
$\Spec(k) \to \overline{\mathcal{M}} \to \Curvesstack$
factors through $\Curvesstack^{DM}$.
We may first replace $k$ by the algebraic closure
(since we already know the relevant stacks are open
substacks of the algebraic stack $\Curvesstack$).
By Lemmas \ref{lemma-stable-curves}, \ref{lemma-DM-curves}, and
\ref{lemma-in-DM-locus-vector-fields} it suffices to show that
$\text{Der}_k(\mathcal{O}_X, \mathcal{O}_X) = 0$.
This is proven in
Algebraic Curves, Lemma \ref{curves-lemma-stable-vector-fields}.

\medskip\noindent
Since $\Curvesstack^{DM}$ is the maximal open substack of
$\Curvesstack$ which is DM, we see this is true also for the
open substack $\overline{\mathcal{M}}$ of $\Curvesstack^{DM}$.
Finally, a DM algebraic stack is Deligne-Mumford by
Morphisms of Stacks, Theorem \ref{stacks-morphisms-theorem-DM}.
\end{proof}

\begin{lemma}
\label{lemma-smooth-dense-in-stable}
Let $g \geq 2$. The inclusion
$$
|\mathcal{M}_g| \subset |\overline{\mathcal{M}}_g|
$$
is that of an open dense subset.
\end{lemma}

\begin{proof}
Since $\overline{\mathcal{M}}_g \subset \Curvesstack^{lci+}$
is open and since
$\Curvesstack^{smooth} \cap \overline{\mathcal{M}}_g = \mathcal{M}_g$
this follows immediately from
Lemma \ref{lemma-smooth-dense}.
\end{proof}




\section{Contraction morphisms}
\label{section-contracting}

\noindent
We urge the reader to familiarize themselves with
Algebraic Curves, Sections
\ref{curves-section-contracting-rational-tails},
\ref{curves-section-contracting-rational-bridges}, and
\ref{curves-section-contracting-to-stable}
before continuing here. The main result of this section
is the existence of a ``stabilization'' morphism
$$
\Curvesstack^{prestable}_g
\longrightarrow
\overline{\mathcal{M}}_g
$$
See Lemma \ref{lemma-stabilization-morphism}.
Loosely speaking, this morphism sends the moduli point of
a nodal genus $g$ curve to the moduli point of the associated stable
curve constructed in
Algebraic Curves, Lemma \ref{curves-lemma-characterize-contraction-to-stable}.

\begin{lemma}
\label{lemma-contract}
Let $S$ be a scheme and $s \in S$ a point.
Let $f : X \to S$ and $g : Y \to S$ be families of curves.
Let $c : X \to Y$ be a morphism over $S$. If
$c_{s, *}\mathcal{O}_{X_s} = \mathcal{O}_{Y_s}$ and
$R^1c_{s, *}\mathcal{O}_{X_s} = 0$, then
after replacing $S$ by an open neighbourhood of $s$
we have $\mathcal{O}_Y = c_*\mathcal{O}_X$ and $R^1c_*\mathcal{O}_X = 0$
and this remains true after base change by any morphism $S' \to S$.
\end{lemma}

\begin{proof}
Let $(U, u) \to (S, s)$ be an \'etale neighbourhood such that
$\mathcal{O}_{Y_U} = (X_U \to Y_U)_*\mathcal{O}_{X_U}$ and
$R^1(X_U \to Y_U)_*\mathcal{O}_{X_U} = 0$ and the same is true
after base change by $U' \to U$. Then we replace $S$ by the
open image of $U \to S$. Given $S' \to S$ we set $U' = U \times_S S'$
and we obtain \'etale coverings $\{U' \to S'\}$ and
$\{Y_{U'} \to Y_{S'}\}$. Thus the truth of the statement for
the base change of $c$ by $S' \to S$ follows from the truth
of the statement for the base change of $X_U \to Y_U$ by
$U' \to U$. In other words, the question is local in the
\'etale topology on $S$.
Thus by Lemma \ref{lemma-etale-locally-scheme} we may assume
$X$ and $Y$ are schemes. By
More on Morphisms, Lemma \ref{more-morphisms-lemma-h1-fibre-zero-isom}
there exists an open subscheme $V \subset Y$ containing $Y_s$
such that $c_*\mathcal{O}_X|_V = \mathcal{O}_V$ and
$R^1c_*\mathcal{O}_X|_V = 0$ and such that this remains true after
any base change by $S' \to S$. Since $g : Y \to S$ is proper, we can
find an open neighbourhood $U \subset S$ of $s$ such that
$g^{-1}(U) \subset V$. Then $U$ works.
\end{proof}

\begin{lemma}
\label{lemma-contract-basic-uniqueness}
Let $S$ be a scheme and $s \in S$ a point.
Let $f : X \to S$ and $g_i : Y_i \to S$, $i = 1, 2$ be families of curves.
Let $c_i : X \to Y_i$ be morphisms over $S$.
Assume there is an isomorphism $Y_{1, s} \cong Y_{2, s}$
of fibres compatible with $c_{1, s}$ and $c_{2, s}$.
If $c_{1, s, *}\mathcal{O}_{X_s} = \mathcal{O}_{Y_{1, s}}$ and
$R^1c_{1, s, *}\mathcal{O}_{X_s} = 0$, then there exist an
open neighbourhood $U$ of $s$ and an isomorphism
$Y_{1, U} \cong Y_{2, U}$ of families of curves over $U$
compatible with the given isomorphism of fibres and with
$c_1$ and $c_2$.
\end{lemma}

\begin{proof}
Recall that $\mathcal{O}_{S, s} = \colim \mathcal{O}_S(U)$ where
the colimit is over the system of affine neighbourhoods $U$ of $s$.
Thus the category of algebraic spaces of finite presentation over
the local ring is the colimit of the categories of algebraic spaces
of finite presentation over the affine neighbourhoods of $s$.
See Limits of Spaces, Lemma
\ref{spaces-limits-lemma-descend-finite-presentation}.
In this way we reduce to the case where $S$ is the spectrum of
a local ring and $s$ is the closed point.

\medskip\noindent
Assume $S = \Spec(A)$ where $A$ is a local ring and $s$ is the closed point.
Write $A = \colim A_j$ with $A_j$ local Noetherian (say essentially
of finite type over $\mathbf{Z}$) and local transition homomorphisms.
Set $S_j = \Spec(A_j)$ with closed point $s_j$. We can find a
$j$ and families of curves $X_j \to S_j$, $Y_{j, i} \to S_j$,
see Lemma \ref{lemma-curves-qs-lfp} and
Limits of Stacks, Lemma
\ref{stacks-limits-lemma-representable-by-spaces-limit-preserving}.
After possibly increasing $j$ we can find morphisms
$c_{j, i} : X_j \to Y_{j, i}$ whose base change to $s$ is $c_i$, see
Limits of Spaces, Lemma
\ref{spaces-limits-lemma-descend-finite-presentation}.
Since $\kappa(s) = \colim \kappa(s_j)$ we can similarly
assume there is an isomorphism $Y_{j, 1, s_j} \cong Y_{j, 2, s_j}$
compatible with $c_{j, 1, s_j}$ and $c_{j, 2, s_j}$.
Finally, the assumptions
$c_{1, s, *}\mathcal{O}_{X_s} = \mathcal{O}_{Y_{1, s}}$ and
$R^1c_{1, s, *}\mathcal{O}_{X_s} = 0$
are inherited by $c_{j, 1, s_j}$ because
$\{s_j \to s\}$ is an fpqc covering and
$c_{1, s}$ is the base of $c_{j, 1, s_j}$ by this covering (details omitted).
In this way we reduce the lemma to the case discussed in the next paragraph.

\medskip\noindent
Assume $S$ is the spectrum of a Noetherian local ring $\Lambda$ and
$s$ is the closed point. Consider the scheme theoretic image $Z$ of
$$
(c_1, c_2) : X \longrightarrow Y_1 \times_S Y_2
$$
The statement of the lemma is equivalent to the assertion that $Z$ maps
isomorphically to $Y_1$ and $Y_2$ via the projection morphisms.
Since taking the scheme theoretic image of this morphism commutes
with flat base change (Morphisms of Spaces, Lemma
\ref{spaces-morphisms-lemma-flat-base-change-scheme-theoretic-image},
we may replace $\Lambda$ by its completion
(More on Algebra, Section \ref{more-algebra-section-permanence-completion}).

\medskip\noindent
Assume $S$ is the spectrum of a complete Noetherian local ring $\Lambda$.
Observe that $X$, $Y_1$, $Y_2$ are schemes in this case
(More on Morphisms of Spaces, Lemma
\ref{spaces-more-morphisms-lemma-projective-over-complete}).
Denote $X_n$, $Y_{1, n}$, $Y_{2, n}$ the base changes of
$X$, $Y_1$, $Y_2$ to $\Spec(\Lambda/\mathfrak m^{n + 1})$.
Recall that the arrow
$$
\Deformationcategory_{X_s \to Y_{2, s}} \cong
\Deformationcategory_{X_s \to Y_{1, s}} \longrightarrow
\Deformationcategory_{X_s}
$$
is an equivalence, see Deformation Problems, Lemma
\ref{examples-defos-lemma-schemes-morphisms-smooth-to-base}.
Thus there is an isomorphism of formal objects
$(X_n \to Y_{1, n}) \cong (X_n \to Y_{2, n})$
of $\Deformationcategory_{X_s \to Y_{1, s}}$.
Finally, by Grothendieck's algebraization theorem
(Cohomology of Schemes, Lemma \ref{coherent-lemma-algebraize-morphism})
this produces an isomorphism $Y_1 \to Y_2$ compatible with $c_1$ and $c_2$.
\end{proof}

\begin{lemma}
\label{lemma-contract-basic}
Let $f : X \to S$ be a family of curves. Let $s \in S$ be a point.
Let $h_0 : X_s \to Y_0$ be a morphism to a proper scheme $Y_0$ over $\kappa(s)$
such that $h_{0, *}\mathcal{O}_{X_s} = \mathcal{O}_{Y_0}$ and
$R^1h_{0, *}\mathcal{O}_{X_s} = 0$. Then there exist an elementary
\'etale neighbourhood $(U, u) \to (S, s)$, a family of curves $Y \to U$,
and a morphism $h : X_U \to Y$ over $U$ whose fibre in $u$
is isomorphic to $h_0$.
\end{lemma}

\begin{proof}
We first do some reductions; we urge the reader to skip ahead.
The question is local on $S$, hence we may assume $S$ is affine.
Write $S = \lim S_i$ as a cofiltered limit of affine schemes $S_i$
of finite type over $\mathbf{Z}$.
For some $i$ we can find a family of curves $X_i \to S_i$
whose base change is $X \to S$. This follows from
Lemma \ref{lemma-curves-qs-lfp} and
Limits of Stacks, Lemma
\ref{stacks-limits-lemma-representable-by-spaces-limit-preserving}.
Let $s_i \in S_i$ be the image of $s$. Observe that
$\kappa(s) = \colim \kappa(s_i)$ and that $X_s$ is a scheme
(Spaces over Fields, Lemma
\ref{spaces-over-fields-lemma-codim-1-point-in-schematic-locus}).
After increasing $i$ we may assume there exists a morphism
$h_{i, 0} : X_{i, s_i} \to Y_i$
of finite type schemes over $\kappa(s_i)$ whose base change to
$\kappa(s)$ is $h_0$, see
Limits, Lemma \ref{limits-lemma-descend-finite-presentation}.
After increasing $i$ we may assume $Y_i$ is proper over $\kappa(s_i)$, see
Limits, Lemma \ref{limits-lemma-eventually-proper}.
Let $g_{i, 0} : Y_0 \to Y_{i, 0}$ be the projection. Observe that
this is a faithfully flat morphism as the base change of
$\Spec(\kappa(s)) \to \Spec(\kappa(s_i))$.
By flat base change we have
$$
h_{0, *}\mathcal{O}_{X_s} = g_{i, 0}^*h_{i, 0, *}\mathcal{O}_{X_{i, s_i}}
\quad\text{and}\quad
R^1h_{0, *}\mathcal{O}_{X_s} = g_{i, 0}^*Rh_{i, 0, *}\mathcal{O}_{X_{i, s_i}}
$$
see Cohomology of Schemes, Lemma
\ref{coherent-lemma-flat-base-change-cohomology}.
By faithful flatness we see that $X_i \to S_i$, $s_i \in S_i$, and
$X_{i, s_i} \to Y_i$ satisfies all the assumptions of the lemma.
This reduces us to the case discussed in the next paragraph.

\medskip\noindent
Assume $S$ is affine of finite type over $\mathbf{Z}$.
Let $\mathcal{O}_{S, s}^h$ be the henselization of the local
ring of $S$ at $s$. Observe that $\mathcal{O}_{S, s}^h$ is a G-ring by
More on Algebra, Lemma \ref{more-algebra-lemma-henselization-G-ring} and
Proposition \ref{more-algebra-proposition-ubiquity-G-ring}.
Suppose we can construct a family of curves
$Y' \to \Spec(\mathcal{O}_{S, s}^h)$ and a morphism
$$
h' : X \times_S \Spec(\mathcal{O}_{S, s}^h) \longrightarrow Y'
$$
over $\Spec(\mathcal{O}_{S, s}^h)$ whose base change to the closed
point is $h_0$. This will be enough. Namely, first we use that
$$
\mathcal{O}_{S, s}^h = \colim_{(U, u)} \mathcal{O}_U(U)
$$
where the colimit is over the filtered category of 
elementary \'etale neighbourhoods (More on Morphisms, Lemma
\ref{more-morphisms-lemma-describe-henselization}).
Next, we use again that given $Y'$ we can descend it to
$Y \to U$ for some $U$ (see references given above).
Then we use
Limits, Lemma \ref{limits-lemma-descend-finite-presentation}
to descend $h'$ to some $h$. This reduces us to the case
discussed in the next paragraph.

\medskip\noindent
Assume $S = \Spec(\Lambda)$ where $(\Lambda, \mathfrak m, \kappa)$
is a henselian Noetherian local G-ring and $s$ is the closed point of $S$.
Recall that the map
$$
\Deformationcategory_{X_s \to Y_0} \to \Deformationcategory_{X_s}
$$
is an equivalence, see Deformation Problems, Lemma
\ref{examples-defos-lemma-schemes-morphisms-smooth-to-base}.
(This is the only important step in the proof; everything else
is technique.) Denote $\Lambda^\wedge$ the $\mathfrak m$-adic completion.
The pullbacks $X_n$ of $X$ to $\Lambda/\mathfrak m^{n + 1}$ define
a formal object $\xi$ of $\Deformationcategory_{X_s}$ over $\Lambda^\wedge$.
From the equivalence we obtain a formal object
$\xi'$ of $\Deformationcategory_{X_s \to Y_0}$ over $\Lambda^\wedge$.
Thus we obtain a huge commutative diagram
$$
\xymatrix{
\ldots \ar[r] &
X_n \ar[r] \ar[d] &
X_{n - 1} \ar[r] \ar[d] &
\ldots \ar[r] &
X_s \ar[d] \\
\ldots \ar[r] &
Y_n \ar[r] \ar[d] &
Y_{n - 1} \ar[r] \ar[d] &
\ldots \ar[r] &
Y_0 \ar[d] \\
\ldots \ar[r] &
\Spec(\Lambda/\mathfrak m^{n + 1}) \ar[r] &
\Spec(\Lambda/\mathfrak m^n) \ar[r] &
\ldots \ar[r] &
\Spec(\kappa)
}
$$
The formal object $(Y_n)$ comes from a family of curves
$Y' \to \Spec(\Lambda^\wedge)$ by
Quot, Lemma \ref{quot-lemma-curves-existence}.
By More on Morphisms of Spaces, Lemma
\ref{spaces-more-morphisms-lemma-algebraize-morphism}
we get a morphism $h' : X_{\Lambda^\wedge} \to Y'$ inducing
the given morphisms $X_n \to Y_n$ for all $n$
and in particular the given morphism $X_s \to Y_0$.

\medskip\noindent
To finish we do a standard algebraization/approximation argument.
First, we observe that we can find a finitely generated $\Lambda$-subalgebra
$\Lambda \subset A \subset \Lambda^\wedge$, a family of curves
$Y'' \to \Spec(A)$ and a morphism $h'' : X_A \to Y''$ over $A$
whose base change to $\Lambda^\wedge$ is $h'$.
This is true because $\Lambda^\wedge$ is the filtered colimit
of these rings $A$ and we can argue as before using
that $\Curvesstack$ is locally of finite presentation
(which gives us $Y''$ over $A$ by
Limits of Stacks, Lemma
\ref{stacks-limits-lemma-representable-by-spaces-limit-preserving})
and using 
Limits of Spaces, Lemma \ref{spaces-limits-lemma-descend-finite-presentation}
to descend $h'$ to some $h''$.
Then we can apply the approximation property for G-rings
(in the form of Smoothing Ring Maps, Theorem
\ref{smoothing-theorem-approximation-property})
to find a map $A \to \Lambda$ which induces the same map
$A \to \kappa$ as we obtain from $A \to \Lambda^\wedge$.
Base changing $h''$ to $\Lambda$ the proof is complete.
\end{proof}

\begin{lemma}
\label{lemma-contract-prestable-to-stable}
Let $f : X \to S$ be a prestable family of curves of genus $g \geq 2$.
There is a factorization $X \to Y \to S$ of $f$ where $g : Y \to S$
is a stable family of curves and $c : X \to Y$ has the following
properties
\begin{enumerate}
\item $\mathcal{O}_Y = c_*\mathcal{O}_X$ and $R^1c_*\mathcal{O}_X = 0$
and this remains true after base change by any morphism $S' \to S$, and
\item for any $s \in S$ the morphism $c_s : X_s \to Y_s$ is the
contraction of rational tails and bridges discussed in
Algebraic Curves, Section \ref{curves-section-contracting-to-stable}.
\end{enumerate}
Moreover $c : X \to Y$ is unique up to unique isomorphism.
\end{lemma}

\begin{proof}
Let $s \in S$. Let $c_0 : X_s \to Y_0$ be the contraction of
Algebraic Curves, Section \ref{curves-section-contracting-to-stable}
(more precisely Algebraic Curves, Lemma
\ref{curves-lemma-characterize-contraction-to-stable}).
By Lemma \ref{lemma-contract-basic}
there exists an elementary \'etale neighbourhood
$(U, u)$ and a morphism $c : X_U \to Y$
of families of curves over $U$ which recovers
$c_0$ as the fibre at $u$.
Since $\omega_{Y_0}$ is ample, after possibly shrinking $U$,
we see that $Y \to U$ is a stable family of genus $g$
by the openness inherent in
Lemmas \ref{lemma-stable-curves} and \ref{lemma-stable-one-piece-per-genus}.
After possibly shrinking $U$ once more, assertion (1) of the lemma for
$c : X_U \to Y$ follows from Lemma \ref{lemma-contract}.
Moreover, part (2) holds by the uniqueness in Algebraic Curves, Lemma
\ref{curves-lemma-characterize-contraction-to-stable}.
We conclude that a morphism $c$ as in the lemma exists \'etale locally
on $S$. More precisely, there exists an \'etale covering
$\{U_i \to S\}$ and morphisms $c_i : X_{U_i} \to Y_i$ over $U_i$
where $Y_i \to U_i$ is a stable family of curves
having properties (1) and (2) stated in the lemma.

\medskip\noindent
To finish the proof it suffices to prove uniqueness of $c : X \to Y$
(up to unique isomorphism). Namely, once this is done, then we
obtain isomorphisms
$$
\varphi_{ij} :
Y_i \times_{U_i} (U_i \times_S U_j)
\longrightarrow
Y_i \times_{U_j} (U_i \times_S U_j)
$$
satisfying the cocycle condition (by uniqueness) over
$U_i \times U_j \times U_k$. Since $\overline{\mathcal{M}_g}$
is an algebraic stack, we have effectiveness of descent data
and we obtain $Y \to S$. The morphisms $c_i$ descend to a morphism
$c : X \to Y$ over $S$. Finally, properties (1) and (2) for $c$
are immediate from properties (1) and (2) for $c_i$.

\medskip\noindent
Finally, if $c_1 : X \to Y_i$, $i = 1, 2$ are two morphisms towards
stably families of curves over $S$ satisfying (1) and (2), then
we obtain a morphism $Y_1 \to Y_2$ compatible with $c_1$ and $c_2$
at least locally on $S$ by Lemma \ref{lemma-contract-basic-uniqueness}.
We omit the verification that these morphisms are unique
(hint: this follows from the fact that the scheme theoretic image
of $c_1$ is $Y_1$). Hence these locally given morphisms glue
and the proof is complete.
\end{proof}

\begin{lemma}
\label{lemma-stabilization-morphism}
Let $g \geq 2$. There is a morphism of algebraic stacks over $\mathbf{Z}$
$$
stabilization :
\Curvesstack^{prestable}_g
\longrightarrow
\overline{\mathcal{M}}_g
$$
which sends a prestable family of curves $X \to S$ of genus $g$
to the stable family $Y \to S$ associated to it in
Lemma \ref{lemma-contract-prestable-to-stable}.
\end{lemma}

\begin{proof}
To see this is true, it suffices to check that the construction of
Lemma \ref{lemma-contract-prestable-to-stable}
is compatible with base change (and isomorphisms but that's immediate),
see the (abuse of) language for algebraic stacks introduced
in Properties of Stacks, Section \ref{stacks-properties-section-conventions}.
To see this it suffices to check properties (1) and (2) of
Lemma \ref{lemma-contract-prestable-to-stable} are stable
under base change. This is immediately clear for (1).
For (2) this follows either from the fact that the contractions of
Algebraic Curves, Lemmas \ref{curves-lemma-contracting-rational-tails} and
\ref{curves-lemma-contracting-rational-bridges}
are stable under ground field extensions, or
because the conditions characterizing the morphisms on fibres in
Algebraic Curves, Lemma \ref{curves-lemma-characterize-contraction-to-stable}
are preserved under ground field extensions.
\end{proof}





\section{Stable reduction theorem}
\label{section-stable-reduction}

\noindent
In the chapter on semistable reduction we have proved the celebrated
theorem on semistable reduction of curves. Let $K$ be the fraction
field of a discrete valuation ring $R$. Let $C$ be a projective smooth
curve over $K$ with $K = H^0(C, \mathcal{O}_C)$. According to
Semistable Reduction, Definition \ref{models-definition-semistable}
we say $C$ has {\it semistable reduction} if either
there is a prestable family of curves over $R$ with generic fibre $C$, or
some (equivalently any) minimal regular model of $C$ over $R$ is prestable.
In this section we show that for curves of genus $g \geq 2$
this is also equivalent to stable reduction.

\begin{lemma}
\label{lemma-stable-reduction}
Let $R$ be a discrete valuation ring with fraction field $K$.
Let $C$ be a smooth projective curve over $K$ with
$K = H^0(C, \mathcal{O}_C)$ having genus $g \geq 2$.
The following are equivalent
\begin{enumerate}
\item $C$ has semistable reduction
(Semistable Reduction, Definition \ref{models-definition-semistable}), or
\item there is a stable family of curves over $R$ with generic fibre $C$.
\end{enumerate}
\end{lemma}

\begin{proof}
Since a stable family of curves is also prestable, it is immediate that
(2) implies (1). Conversely, given a prestable family of curves over $R$
with generic fibre $C$, we can contract it to a stable family of curves
by Lemma \ref{lemma-contract-prestable-to-stable}. Since the generic
fibre already is stable, it does not get changed by this procedure and
the proof is complete.
\end{proof}

\noindent
The following lemma tells us the stable family of curves over $R$
promised in Lemma \ref{lemma-stable-reduction}
is unique up to unique isomorphism.

\begin{lemma}
\label{lemma-unique-stable-model}
Let $R$ be a discrete valuation ring with fraction field $K$.
Let $C$ be a smooth proper curve over $K$
with $K = H^0(C, \mathcal{O}_C)$ and genus $g$.
If $X$ and $X'$ are models of $C$
(Semistable Reduction, Section \ref{models-section-models})
and $X$ and $X'$ are stable families of genus $g$ curves over $R$,
then there exists a unique isomorphism $X \to X'$ of models.
\end{lemma}

\begin{proof}
Let $Y$ be the minimal model for $C$. Recall that $Y$ exists, is
unique, and is at-worst-nodal of relative dimension $1$ over $R$, see
Semistable Reduction,
Proposition \ref{models-proposition-exists-minimal-model} and
Lemmas \ref{models-lemma-minimal-model-unique} and
\ref{models-lemma-semistable} (applies because we have $X$).
There is a contraction morphism
$$
Y \longrightarrow Z
$$
such that $Z$ is a stable family of curves of genus $g$ over $R$
(Lemma \ref{lemma-contract-prestable-to-stable}). We claim
there is a unique isomorphism of models $X \to Z$.
By symmetry the same is true for $X'$ and this will finish the proof.

\medskip\noindent
By Semistable Reduction, Lemma \ref{models-lemma-blowup-at-worst-nodal}
there exists a sequence
$$
X_m \to \ldots \to X_1 \to X_0 = X
$$
such that $X_{i + 1} \to X_i$ is the blowing up of a closed point
$x_i$ where $X_i$ is singular, $X_i \to \Spec(R)$ is at-worst-nodal
of relative dimension $1$, and $X_m$ is regular.
By Semistable Reduction, Lemma \ref{models-lemma-pre-exists-minimal-model}
there is a sequence
$$
X_m = Y_n \to Y_{n - 1} \to \ldots \to Y_1 \to Y_0 = Y
$$
of proper regular models of $C$, such that each morphism is a
contraction of an exceptional curve of the first kind\footnote{In fact
we have $X_m = Y$, i.e., $X_m$ does not contain any exceptional curves
of the first kind. We encourage the reader to think this through
as it simplifies the proof somewhat.}.
By Semistable Reduction, Lemma \ref{models-lemma-blowdown-at-worst-nodal}
each $Y_i$ is at-worst-nodal of relative dimension $1$ over $R$.
To prove the claim it suffices to show that there is an isomorphism
$X \to Z$ compatible with the morphisms $X_m \to X$
and $X_m = Y_n \to Y \to Z$. Let $s \in \Spec(R)$ be the closed point.
By either
Lemma \ref{lemma-contract-basic-uniqueness} or
Lemma \ref{lemma-contract-prestable-to-stable}
we reduce to proving that the morphisms
$X_{m, s} \to X_s$ and
$X_{m, s} \to Z_s$
are both equal to the canonical morphism of
Algebraic Curves, Lemma \ref{curves-lemma-characterize-contraction-to-stable}.

\medskip\noindent
For a morphism $c : U \to V$ of schemes over $\kappa(s)$
we say $c$ has property (*) if $\dim(U_v) \leq 1$ for $v \in V$,
$\mathcal{O}_V = c_*\mathcal{O}_U$, and $R^1c_*\mathcal{O}_U = 0$.
This property is stable under composition.
Since both $X_s$ and $Z_s$ are stable genus $g$ curves over $\kappa(s)$,
it suffices to show that each of the morphisms $Y_s \to Z_s$,
$X_{i + 1, s} \to X_{i, s}$, and $Y_{i + 1, s} \to Y_{i, s}$,
satisfy property (*), see
Algebraic Curves, Lemma \ref{curves-lemma-characterize-contraction-to-stable}.

\medskip\noindent
Property (*) holds for $Y_s \to Z_s$ by construction.

\medskip\noindent
The morphisms $c : X_{i + 1, s} \to X_{i, s}$ are constructed and studied
in the proof of
Semistable Reduction, Lemma \ref{models-lemma-blowup-at-worst-nodal}.
It suffices to check (*) \'etale locally on $X_{i, s}$.
Hence it suffices to check (*) for the base change of the morphism
``$X_1 \to X_0$'' in Semistable Reduction, Example \ref{models-example-blowup}
to $R/\pi R$.
We leave the explicit calculation to the reader.

\medskip\noindent
The morphism $c : Y_{i + 1, s} \to Y_{i, s}$ is the restriction
of the blow down of an exceptional curve $E \subset Y_{i + 1}$
of the first kind, i.e., $b : Y_{i + 1} \to Y_i$ is a contraction of $E$,
i.e., $b$ is a blowing up of a regular point on the surface $Y_i$
(Resolution of Surfaces, Section \ref{resolve-section-minus-one}).
Then $\mathcal{O}_{Y_i} = b_*\mathcal{O}_{Y_{i + 1}}$ and
$R^1b_*\mathcal{O}_{Y_{i + 1}} = 0$, see for example
Resolution of Surfaces, Lemma
\ref{resolve-lemma-cohomology-of-blowup}.
We conclude that $\mathcal{O}_{Y_{i, s}} = c_*\mathcal{O}_{Y_{i + 1, s}}$
and $R^1c_*\mathcal{O}_{Y_{i + 1, s}} = 0$ by
More on Morphisms, Lemmas \ref{more-morphisms-lemma-check-h1-fibre-zero},
\ref{more-morphisms-lemma-h1-fibre-zero}, and
\ref{more-morphisms-lemma-h1-fibre-zero-check-h0-kappa}
(only gives surjectivity of
$\mathcal{O}_{Y_{i, s}} \to c_*\mathcal{O}_{Y_{i + 1, s}}$ but
injectivity follows easily from the fact that $Y_{i, s}$ is reduced
and $c$ changes things only over one closed point).
This finishes the proof.
\end{proof}

\noindent
From Lemma \ref{lemma-stable-reduction} and
Semistable Reduction, Theorem \ref{models-theorem-semistable-reduction}
we immediately deduce the stable reduction theorem.

\begin{theorem}
\label{theorem-stable-reduction}
\begin{reference}
\cite[Corollary 2.7]{DM}
\end{reference}
Let $R$ be a discrete valuation ring with fraction field $K$. Let $C$ be a
smooth projective curve over $K$ with $H^0(C, \mathcal{O}_C) = K$
and genus $g \geq 2$. Then
\begin{enumerate}
\item there exists an extension of discrete valuation rings $R \subset R'$
inducing a finite separable extension of fraction fields $K'/K$ and
a stable family of curves $Y \to \Spec(R')$ of genus $g$ with
$Y_{K'} \cong C_{K'}$ over $K'$, and
\item there exists a finite separable extension $L/K$ and a stable
family of curves $Y \to \Spec(A)$ of genus $g$ where $A \subset L$
is the integral closure of $R$ in $L$ such that
$Y_L \cong C_L$ over $L$.
\end{enumerate}
\end{theorem}

\begin{proof}
Part (1) is an immediate consequence of Lemma \ref{lemma-stable-reduction} and
Semistable Reduction, Theorem \ref{models-theorem-semistable-reduction}.

\medskip\noindent
Proof of (2). Let $L/K$ be the finite separable extension found in part (3) of
Semistable Reduction, Theorem \ref{models-theorem-semistable-reduction}.
Let $A \subset L$ be the integral closure of $R$.
Recall that $A$ is a Dedekind domain finite over $R$ with
finitely many maximal ideals $\mathfrak m_1, \ldots, \mathfrak m_n$, see
More on Algebra, Remark \ref{more-algebra-remark-finite-separable-extension}.
Set $S = \Spec(A)$, $S_i = \Spec(A_{\mathfrak m_i})$,
$U = \Spec(L)$, and $U_i = S_i \setminus \{\mathfrak m_i\}$.
Observe that $U \cong U_i$ for $i = 1, \ldots, n$.
Set $X = C_L$ viewed as a scheme over the open subscheme $U$ of $S$.
By our choice of $L$ and $A$ and Lemma \ref{lemma-stable-reduction}
we have stable families of curves $X_i \to S_i$ and isomorphisms
$X \times_U U_i \cong X_i \times_{S_i} U_i$.
By Limits of Spaces, Lemma
\ref{spaces-limits-lemma-glueing-near-multiple-closed-points}
we can find a finitely presented morphism $Y \to S$
whose base change to $S_i$ is isomorphic to $X_i$ for $i = 1, \ldots, n$.
Alternatively, you can use that $S = \bigcup_{i = 1, \ldots, n} S_i$
is an open covering of $S$ and $S_i \cap S_j = U$ for $i \not = j$
and use $n - 1$ applications of
Limits of Spaces, Lemma \ref{spaces-limits-lemma-relative-glueing}
to get $Y \to S$ whose
base change to $S_i$ is isomorphic to $X_i$ for $i = 1, \ldots, n$.
Clearly $Y \to S$ is the stable family of curves we were looking for.
\end{proof}






\section{Properties of the stack of stable curves}
\label{section-properties-stable}

\noindent
In this section we prove the basic structure result for
$\overline{\mathcal{M}}_g$ for $g \geq 2$.

\begin{lemma}
\label{lemma-stable-separated}
Let $g \geq 2$. The stack $\overline{\mathcal{M}}_g$ is separated.
\end{lemma}

\begin{proof}
The statement means that the morphism
$\overline{\mathcal{M}}_g \to \Spec(\mathbf{Z})$ is separated.
We will prove this using the refined Noetherian valuative criterion
as stated in 
More on Morphisms of Stacks, Lemma
\ref{stacks-more-morphisms-lemma-refined-valuative-criterion-separated}

\medskip\noindent
Since $\overline{\mathcal{M}}_g$ is an open substack of
$\Curvesstack$, we see $\overline{\mathcal{M}}_g \to \Spec(\mathbf{Z})$
is quasi-separated and
locally of finite presentation by Lemma \ref{lemma-curves-qs-lfp}.
In particular the stack $\overline{\mathcal{M}}_g$ is locally
Noetherian (Morphisms of Stacks, Lemma
\ref{stacks-morphisms-lemma-locally-finite-type-locally-noetherian}).
By Lemma \ref{lemma-smooth-dense-in-stable} the open immersion
$\mathcal{M}_g \to \overline{\mathcal{M}}_g$
has dense image. Also, $\mathcal{M}_g \to \overline{\mathcal{M}}_g$
is quasi-compact (Morphisms of Stacks, Lemma
\ref{stacks-morphisms-lemma-locally-closed-in-noetherian}),
hence of finite type. Thus all the preliminary assumptions
of More on Morphisms of Stacks, Lemma
\ref{stacks-more-morphisms-lemma-refined-valuative-criterion-separated}
are satisfied for the morphisms
$$
\mathcal{M}_g \to \overline{\mathcal{M}}_g
\quad\text{and}\quad
\overline{\mathcal{M}}_g \to \Spec(\mathbf{Z})
$$
and it suffices to check the following: given any $2$-commutative diagram
$$
\xymatrix{
\Spec(K) \ar[r] \ar[d] &
\mathcal{M}_g \ar[r] &
\overline{\mathcal{M}}_g \ar[d] \\
\Spec(R) \ar[rr] \ar@{..>}[rru] & & \Spec(\mathbf{Z})
}
$$
where $R$ is a discrete valuation ring with field of fractions $K$
the category of dotted arrows is either empty or a setoid with exactly
one isomorphism class. (Observe that we don't need to worry about
$2$-arrows too much, see Morphisms of Stacks, Lemma
\ref{stacks-morphisms-lemma-cat-dotted-arrows-independent}).
Unwinding what this means using that
$\mathcal{M}_g$, resp.\ $\overline{\mathcal{M}}_g$ are
the algebraic stacks parametrizing smooth, resp.\ stable families
of genus $g$ curves, we find that what we have to prove
is exactly the uniqueness result stated and proved in
Lemma \ref{lemma-unique-stable-model}.
\end{proof}

\begin{lemma}
\label{lemma-stable-quasi-compact}
Let $g \geq 2$. The stack $\overline{\mathcal{M}}_g$ is quasi-compact.
\end{lemma}

\begin{proof}
We will use the notation from Section \ref{section-polarized-curves}.
Consider the subset
$$
T \subset |\textit{PolarizedCurves}|
$$
of points $\xi$ such that there exists a field $k$ and a pair
$(X, \mathcal{L})$ over $k$ representing $\xi$
with the following two properties
\begin{enumerate}
\item $X$ is a stable genus $g$ curve, and
\item $\mathcal{L} = \omega_X^{\otimes 3}$.
\end{enumerate}
Clearly, under the continuous map
$$
|\textit{PolarizedCurves}|
\longrightarrow
|\Curvesstack|
$$
the image of the set $T$ is exactly the open subset
$$
|\overline{\mathcal{M}}_g| \subset |\Curvesstack|
$$
Thus it suffices to show that $T$ is quasi-compact.
By Lemma \ref{lemma-polarized-curves-in-polarized} we see that
$$
|\textit{PolarizedCurves}| \subset |\Polarizedstack|
$$
is an open and closed immersion. Thus it suffices to
prove quasi-compactness of $T$ as a subset of
$|\Polarizedstack|$. For this we use the criterion of
Moduli Stacks, Lemma \ref{moduli-lemma-bounded-polarized}.
First, we observe that for $(X, \mathcal{L})$
as above the Hilbert polynomial $P$ is the function
$P(t) = (6g - 6)t + (1 - g)$ by Riemann-Roch, see
Algebraic Curves, Lemma \ref{curves-lemma-rr}.
Next, we observe that $H^1(X, \mathcal{L}) = 0$
and $\mathcal{L}$ is very ample by
Algebraic Curves, Lemma \ref{curves-lemma-tricanonical}.
This means exactly that with $n = P(3) - 1$
there is a closed immersion
$$
i : X \longrightarrow \mathbf{P}^n_k
$$
such that $\mathcal{L} = i^*\mathcal{O}_{\mathbf{P}^1_k}(1)$
as desired.
\end{proof}

\noindent
Here is the main theorem of this section.

\begin{theorem}
\label{theorem-stable-smooth-proper}
Let $g \geq 2$. The algebraic stack $\overline{\mathcal{M}}_g$ is a
Deligne-Mumford stack, proper and smooth over $\Spec(\mathbf{Z})$.
Moreover, the locus $\mathcal{M}_g$ parametrizing smooth curves
is a dense open substack.
\end{theorem}

\begin{proof}
Most of the properties mentioned in the statement have already been shown.
Smoothness is Lemma \ref{lemma-stable-curves-smooth}.
Deligne-Mumford is Lemma \ref{lemma-stable-curves-deligne-mumford}.
Openness of $\mathcal{M}_g$ is Lemma \ref{lemma-smooth-dense-in-stable}.
We know that $\overline{\mathcal{M}}_g \to \Spec(\mathbf{Z})$
is separated by Lemma \ref{lemma-stable-separated} and we know that
$\overline{\mathcal{M}}_g$ is quasi-compact by
Lemma \ref{lemma-stable-quasi-compact}.
Thus, to show that $\overline{\mathcal{M}}_g \to \Spec(\mathbf{Z})$
is proper and finish the proof, we may apply
More on Morphisms of Stacks, Lemma
\ref{stacks-more-morphisms-lemma-refined-valuative-criterion-proper}
to the morphisms $\mathcal{M}_g \to \overline{\mathcal{M}}_g$ and
$\overline{\mathcal{M}}_g \to \Spec(\mathbf{Z})$.
Thus it suffices to check the following: given any $2$-commutative diagram
$$
\xymatrix{
\Spec(K) \ar[r] \ar[d]_j &
\mathcal{M}_g \ar[r] &
\overline{\mathcal{M}}_g \ar[d] \\
\Spec(A) \ar[rr] & & \Spec(\mathbf{Z})
}
$$
where $A$ is a discrete valuation ring with field of fractions $K$, there
exist an extension $K'/K$ of fields, a valuation ring $A' \subset K'$
dominating $A$ such that the category of dotted arrows for the
induced diagram
$$
\xymatrix{
\Spec(K') \ar[r] \ar[d]_{j'} & \overline{\mathcal{M}}_g \ar[d] \\
\Spec(A') \ar[r] \ar@{..>}[ru] & \Spec(\mathbf{Z})
}
$$
is nonempty (Morphisms of Stacks, Definition
\ref{stacks-morphisms-definition-fill-in-diagram}).
(Observe that we don't need to worry about
$2$-arrows too much, see Morphisms of Stacks, Lemma
\ref{stacks-morphisms-lemma-cat-dotted-arrows-independent}).
Unwinding what this means using that
$\mathcal{M}_g$, resp.\ $\overline{\mathcal{M}}_g$ are the algebraic
stacks parametrizing smooth, resp.\ stable families of genus $g$ curves,
we find that what we have to prove is exactly the result contained
in the stable reduction theorem, i.e., Theorem
\ref{theorem-stable-reduction}.
\end{proof}











\begin{multicols}{2}[\section{Other chapters}]
\noindent
Preliminaries
\begin{enumerate}
\item \hyperref[introduction-section-phantom]{Introduction}
\item \hyperref[conventions-section-phantom]{Conventions}
\item \hyperref[sets-section-phantom]{Set Theory}
\item \hyperref[categories-section-phantom]{Categories}
\item \hyperref[topology-section-phantom]{Topology}
\item \hyperref[sheaves-section-phantom]{Sheaves on Spaces}
\item \hyperref[sites-section-phantom]{Sites and Sheaves}
\item \hyperref[stacks-section-phantom]{Stacks}
\item \hyperref[fields-section-phantom]{Fields}
\item \hyperref[algebra-section-phantom]{Commutative Algebra}
\item \hyperref[brauer-section-phantom]{Brauer Groups}
\item \hyperref[homology-section-phantom]{Homological Algebra}
\item \hyperref[derived-section-phantom]{Derived Categories}
\item \hyperref[simplicial-section-phantom]{Simplicial Methods}
\item \hyperref[more-algebra-section-phantom]{More on Algebra}
\item \hyperref[smoothing-section-phantom]{Smoothing Ring Maps}
\item \hyperref[modules-section-phantom]{Sheaves of Modules}
\item \hyperref[sites-modules-section-phantom]{Modules on Sites}
\item \hyperref[injectives-section-phantom]{Injectives}
\item \hyperref[cohomology-section-phantom]{Cohomology of Sheaves}
\item \hyperref[sites-cohomology-section-phantom]{Cohomology on Sites}
\item \hyperref[dga-section-phantom]{Differential Graded Algebra}
\item \hyperref[dpa-section-phantom]{Divided Power Algebra}
\item \hyperref[hypercovering-section-phantom]{Hypercoverings}
\end{enumerate}
Schemes
\begin{enumerate}
\setcounter{enumi}{24}
\item \hyperref[schemes-section-phantom]{Schemes}
\item \hyperref[constructions-section-phantom]{Constructions of Schemes}
\item \hyperref[properties-section-phantom]{Properties of Schemes}
\item \hyperref[morphisms-section-phantom]{Morphisms of Schemes}
\item \hyperref[coherent-section-phantom]{Cohomology of Schemes}
\item \hyperref[divisors-section-phantom]{Divisors}
\item \hyperref[limits-section-phantom]{Limits of Schemes}
\item \hyperref[varieties-section-phantom]{Varieties}
\item \hyperref[topologies-section-phantom]{Topologies on Schemes}
\item \hyperref[descent-section-phantom]{Descent}
\item \hyperref[perfect-section-phantom]{Derived Categories of Schemes}
\item \hyperref[more-morphisms-section-phantom]{More on Morphisms}
\item \hyperref[flat-section-phantom]{More on Flatness}
\item \hyperref[groupoids-section-phantom]{Groupoid Schemes}
\item \hyperref[more-groupoids-section-phantom]{More on Groupoid Schemes}
\item \hyperref[etale-section-phantom]{\'Etale Morphisms of Schemes}
\end{enumerate}
Topics in Scheme Theory
\begin{enumerate}
\setcounter{enumi}{40}
\item \hyperref[chow-section-phantom]{Chow Homology}
\item \hyperref[intersection-section-phantom]{Intersection Theory}
\item \hyperref[pic-section-phantom]{Picard Schemes of Curves}
\item \hyperref[adequate-section-phantom]{Adequate Modules}
\item \hyperref[dualizing-section-phantom]{Dualizing Complexes}
\item \hyperref[duality-section-phantom]{Duality for Schemes}
\item \hyperref[discriminant-section-phantom]{Discriminants and Differents}
\item \hyperref[local-cohomology-section-phantom]{Local Cohomology}
\item \hyperref[curves-section-phantom]{Algebraic Curves}
\item \hyperref[resolve-section-phantom]{Resolution of Surfaces}
\item \hyperref[models-section-phantom]{Semistable Reduction}
\item \hyperref[pione-section-phantom]{Fundamental Groups of Schemes}
\item \hyperref[etale-cohomology-section-phantom]{\'Etale Cohomology}
\item \hyperref[ssgroups-section-phantom]{Linear Algebraic Groups}
\item \hyperref[crystalline-section-phantom]{Crystalline Cohomology}
\item \hyperref[proetale-section-phantom]{Pro-\'etale Cohomology}
\end{enumerate}
Algebraic Spaces
\begin{enumerate}
\setcounter{enumi}{56}
\item \hyperref[spaces-section-phantom]{Algebraic Spaces}
\item \hyperref[spaces-properties-section-phantom]{Properties of Algebraic Spaces}
\item \hyperref[spaces-morphisms-section-phantom]{Morphisms of Algebraic Spaces}
\item \hyperref[decent-spaces-section-phantom]{Decent Algebraic Spaces}
\item \hyperref[spaces-cohomology-section-phantom]{Cohomology of Algebraic Spaces}
\item \hyperref[spaces-limits-section-phantom]{Limits of Algebraic Spaces}
\item \hyperref[spaces-divisors-section-phantom]{Divisors on Algebraic Spaces}
\item \hyperref[spaces-over-fields-section-phantom]{Algebraic Spaces over Fields}
\item \hyperref[spaces-topologies-section-phantom]{Topologies on Algebraic Spaces}
\item \hyperref[spaces-descent-section-phantom]{Descent and Algebraic Spaces}
\item \hyperref[spaces-perfect-section-phantom]{Derived Categories of Spaces}
\item \hyperref[spaces-more-morphisms-section-phantom]{More on Morphisms of Spaces}
\item \hyperref[spaces-flat-section-phantom]{Flatness on Algebraic Spaces}
\item \hyperref[spaces-groupoids-section-phantom]{Groupoids in Algebraic Spaces}
\item \hyperref[spaces-more-groupoids-section-phantom]{More on Groupoids in Spaces}
\item \hyperref[bootstrap-section-phantom]{Bootstrap}
\item \hyperref[spaces-pushouts-section-phantom]{Pushouts of Algebraic Spaces}
\end{enumerate}
Topics in Geometry
\begin{enumerate}
\setcounter{enumi}{73}
\item \hyperref[spaces-chow-section-phantom]{Chow Groups of Spaces}
\item \hyperref[groupoids-quotients-section-phantom]{Quotients of Groupoids}
\item \hyperref[spaces-more-cohomology-section-phantom]{More on Cohomology of Spaces}
\item \hyperref[spaces-simplicial-section-phantom]{Simplicial Spaces}
\item \hyperref[spaces-duality-section-phantom]{Duality for Spaces}
\item \hyperref[formal-spaces-section-phantom]{Formal Algebraic Spaces}
\item \hyperref[restricted-section-phantom]{Restricted Power Series}
\item \hyperref[spaces-resolve-section-phantom]{Resolution of Surfaces Revisited}
\end{enumerate}
Deformation Theory
\begin{enumerate}
\setcounter{enumi}{81}
\item \hyperref[formal-defos-section-phantom]{Formal Deformation Theory}
\item \hyperref[defos-section-phantom]{Deformation Theory}
\item \hyperref[cotangent-section-phantom]{The Cotangent Complex}
\item \hyperref[examples-defos-section-phantom]{Deformation Problems}
\end{enumerate}
Algebraic Stacks
\begin{enumerate}
\setcounter{enumi}{85}
\item \hyperref[algebraic-section-phantom]{Algebraic Stacks}
\item \hyperref[examples-stacks-section-phantom]{Examples of Stacks}
\item \hyperref[stacks-sheaves-section-phantom]{Sheaves on Algebraic Stacks}
\item \hyperref[criteria-section-phantom]{Criteria for Representability}
\item \hyperref[artin-section-phantom]{Artin's Axioms}
\item \hyperref[quot-section-phantom]{Quot and Hilbert Spaces}
\item \hyperref[stacks-properties-section-phantom]{Properties of Algebraic Stacks}
\item \hyperref[stacks-morphisms-section-phantom]{Morphisms of Algebraic Stacks}
\item \hyperref[stacks-limits-section-phantom]{Limits of Algebraic Stacks}
\item \hyperref[stacks-cohomology-section-phantom]{Cohomology of Algebraic Stacks}
\item \hyperref[stacks-perfect-section-phantom]{Derived Categories of Stacks}
\item \hyperref[stacks-introduction-section-phantom]{Introducing Algebraic Stacks}
\item \hyperref[stacks-more-morphisms-section-phantom]{More on Morphisms of Stacks}
\item \hyperref[stacks-geometry-section-phantom]{The Geometry of Stacks}
\end{enumerate}
Topics in Moduli Theory
\begin{enumerate}
\setcounter{enumi}{99}
\item \hyperref[moduli-section-phantom]{Moduli Stacks}
\item \hyperref[moduli-curves-section-phantom]{Moduli of Curves}
\end{enumerate}
Miscellany
\begin{enumerate}
\setcounter{enumi}{101}
\item \hyperref[examples-section-phantom]{Examples}
\item \hyperref[exercises-section-phantom]{Exercises}
\item \hyperref[guide-section-phantom]{Guide to Literature}
\item \hyperref[desirables-section-phantom]{Desirables}
\item \hyperref[coding-section-phantom]{Coding Style}
\item \hyperref[obsolete-section-phantom]{Obsolete}
\item \hyperref[fdl-section-phantom]{GNU Free Documentation License}
\item \hyperref[index-section-phantom]{Auto Generated Index}
\end{enumerate}
\end{multicols}


\bibliography{my}
\bibliographystyle{amsalpha}

\end{document}
