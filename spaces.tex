\IfFileExists{stacks-project.cls}{%
\documentclass{stacks-project}
}{%
\documentclass{amsart}
}

% The following AMS packages are automatically loaded with
% the amsart documentclass:
%\usepackage{amsmath}
%\usepackage{amssymb}
%\usepackage{amsthm}

\usepackage{graphicx}

% For dealing with references we use the comment environment
\usepackage{verbatim}
\newenvironment{reference}{\comment}{\endcomment}
%\newenvironment{reference}{}{}
\newenvironment{slogan}{\comment}{\endcomment}
\newenvironment{history}{\comment}{\endcomment}

% For commutative diagrams you can use
% \usepackage{amscd}
\usepackage[all]{xy}

% We use 2cell for 2-commutative diagrams.
\xyoption{2cell}
\UseAllTwocells

% To put source file link in headers.
% Change "template.tex" to "this_filename.tex"
% \usepackage{fancyhdr}
% \pagestyle{fancy}
% \lhead{}
% \chead{}
% \rhead{Source file: \url{template.tex}}
% \lfoot{}
% \cfoot{\thepage}
% \rfoot{}
% \renewcommand{\headrulewidth}{0pt}
% \renewcommand{\footrulewidth}{0pt}
% \renewcommand{\headheight}{12pt}

\usepackage{multicol}

% For cross-file-references
\usepackage{xr-hyper}

% Package for hypertext links:
\usepackage{hyperref}

% For any local file, say "hello.tex" you want to link to please
% use \externaldocument[hello-]{hello}
\externaldocument[introduction-]{introduction}
\externaldocument[conventions-]{conventions}
\externaldocument[sets-]{sets}
\externaldocument[categories-]{categories}
\externaldocument[topology-]{topology}
\externaldocument[sheaves-]{sheaves}
\externaldocument[sites-]{sites}
\externaldocument[stacks-]{stacks}
\externaldocument[fields-]{fields}
\externaldocument[algebra-]{algebra}
\externaldocument[brauer-]{brauer}
\externaldocument[homology-]{homology}
\externaldocument[derived-]{derived}
\externaldocument[simplicial-]{simplicial}
\externaldocument[more-algebra-]{more-algebra}
\externaldocument[smoothing-]{smoothing}
\externaldocument[modules-]{modules}
\externaldocument[sites-modules-]{sites-modules}
\externaldocument[injectives-]{injectives}
\externaldocument[cohomology-]{cohomology}
\externaldocument[sites-cohomology-]{sites-cohomology}
\externaldocument[dga-]{dga}
\externaldocument[dpa-]{dpa}
\externaldocument[hypercovering-]{hypercovering}
\externaldocument[schemes-]{schemes}
\externaldocument[constructions-]{constructions}
\externaldocument[properties-]{properties}
\externaldocument[morphisms-]{morphisms}
\externaldocument[coherent-]{coherent}
\externaldocument[divisors-]{divisors}
\externaldocument[limits-]{limits}
\externaldocument[varieties-]{varieties}
\externaldocument[topologies-]{topologies}
\externaldocument[descent-]{descent}
\externaldocument[perfect-]{perfect}
\externaldocument[more-morphisms-]{more-morphisms}
\externaldocument[flat-]{flat}
\externaldocument[groupoids-]{groupoids}
\externaldocument[more-groupoids-]{more-groupoids}
\externaldocument[etale-]{etale}
\externaldocument[chow-]{chow}
\externaldocument[intersection-]{intersection}
\externaldocument[pic-]{pic}
\externaldocument[adequate-]{adequate}
\externaldocument[dualizing-]{dualizing}
\externaldocument[duality-]{duality}
\externaldocument[discriminant-]{discriminant}
\externaldocument[local-cohomology-]{local-cohomology}
\externaldocument[curves-]{curves}
\externaldocument[resolve-]{resolve}
\externaldocument[models-]{models}
\externaldocument[pione-]{pione}
\externaldocument[etale-cohomology-]{etale-cohomology}
\externaldocument[ssgroups-]{ssgroups}
\externaldocument[proetale-]{proetale}
\externaldocument[crystalline-]{crystalline}
\externaldocument[spaces-]{spaces}
\externaldocument[spaces-properties-]{spaces-properties}
\externaldocument[spaces-morphisms-]{spaces-morphisms}
\externaldocument[decent-spaces-]{decent-spaces}
\externaldocument[spaces-cohomology-]{spaces-cohomology}
\externaldocument[spaces-limits-]{spaces-limits}
\externaldocument[spaces-divisors-]{spaces-divisors}
\externaldocument[spaces-over-fields-]{spaces-over-fields}
\externaldocument[spaces-topologies-]{spaces-topologies}
\externaldocument[spaces-descent-]{spaces-descent}
\externaldocument[spaces-perfect-]{spaces-perfect}
\externaldocument[spaces-more-morphisms-]{spaces-more-morphisms}
\externaldocument[spaces-flat-]{spaces-flat}
\externaldocument[spaces-groupoids-]{spaces-groupoids}
\externaldocument[spaces-more-groupoids-]{spaces-more-groupoids}
\externaldocument[bootstrap-]{bootstrap}
\externaldocument[spaces-pushouts-]{spaces-pushouts}
\externaldocument[spaces-chow-]{spaces-chow}
\externaldocument[groupoids-quotients-]{groupoids-quotients}
\externaldocument[spaces-more-cohomology-]{spaces-more-cohomology}
\externaldocument[spaces-simplicial-]{spaces-simplicial}
\externaldocument[spaces-duality-]{spaces-duality}
\externaldocument[formal-spaces-]{formal-spaces}
\externaldocument[restricted-]{restricted}
\externaldocument[spaces-resolve-]{spaces-resolve}
\externaldocument[formal-defos-]{formal-defos}
\externaldocument[defos-]{defos}
\externaldocument[cotangent-]{cotangent}
\externaldocument[examples-defos-]{examples-defos}
\externaldocument[algebraic-]{algebraic}
\externaldocument[examples-stacks-]{examples-stacks}
\externaldocument[stacks-sheaves-]{stacks-sheaves}
\externaldocument[criteria-]{criteria}
\externaldocument[artin-]{artin}
\externaldocument[quot-]{quot}
\externaldocument[stacks-properties-]{stacks-properties}
\externaldocument[stacks-morphisms-]{stacks-morphisms}
\externaldocument[stacks-limits-]{stacks-limits}
\externaldocument[stacks-cohomology-]{stacks-cohomology}
\externaldocument[stacks-perfect-]{stacks-perfect}
\externaldocument[stacks-introduction-]{stacks-introduction}
\externaldocument[stacks-more-morphisms-]{stacks-more-morphisms}
\externaldocument[stacks-geometry-]{stacks-geometry}
\externaldocument[moduli-]{moduli}
\externaldocument[moduli-curves-]{moduli-curves}
\externaldocument[examples-]{examples}
\externaldocument[exercises-]{exercises}
\externaldocument[guide-]{guide}
\externaldocument[desirables-]{desirables}
\externaldocument[coding-]{coding}
\externaldocument[obsolete-]{obsolete}
\externaldocument[fdl-]{fdl}
\externaldocument[index-]{index}

% Theorem environments.
%
\theoremstyle{plain}
\newtheorem{theorem}[subsection]{Theorem}
\newtheorem{proposition}[subsection]{Proposition}
\newtheorem{lemma}[subsection]{Lemma}

\theoremstyle{definition}
\newtheorem{definition}[subsection]{Definition}
\newtheorem{example}[subsection]{Example}
\newtheorem{exercise}[subsection]{Exercise}
\newtheorem{situation}[subsection]{Situation}

\theoremstyle{remark}
\newtheorem{remark}[subsection]{Remark}
\newtheorem{remarks}[subsection]{Remarks}

\numberwithin{equation}{subsection}

% Macros
%
\def\lim{\mathop{\mathrm{lim}}\nolimits}
\def\colim{\mathop{\mathrm{colim}}\nolimits}
\def\Spec{\mathop{\mathrm{Spec}}}
\def\Hom{\mathop{\mathrm{Hom}}\nolimits}
\def\Ext{\mathop{\mathrm{Ext}}\nolimits}
\def\SheafHom{\mathop{\mathcal{H}\!\mathit{om}}\nolimits}
\def\SheafExt{\mathop{\mathcal{E}\!\mathit{xt}}\nolimits}
\def\Sch{\mathit{Sch}}
\def\Mor{\mathop{Mor}\nolimits}
\def\Ob{\mathop{\mathrm{Ob}}\nolimits}
\def\Sh{\mathop{\mathit{Sh}}\nolimits}
\def\NL{\mathop{N\!L}\nolimits}
\def\proetale{{pro\text{-}\acute{e}tale}}
\def\etale{{\acute{e}tale}}
\def\QCoh{\mathit{QCoh}}
\def\Ker{\mathop{\mathrm{Ker}}}
\def\Im{\mathop{\mathrm{Im}}}
\def\Coker{\mathop{\mathrm{Coker}}}
\def\Coim{\mathop{\mathrm{Coim}}}
\def\id{\mathop{\mathrm{id}}\nolimits}

%
% Macros for linear algebraic groups
%
\def\SL{\mathop{\mathrm{SL}}\nolimits}
\def\GL{\mathop{\mathrm{GL}}\nolimits}
\def\ltimes{{\mathchar"256E}}
\def\rtimes{{\mathchar"256F}}
\def\Rrightarrow{{\mathchar"3456}}

%
% Macros for moduli stacks/spaces
%
\def\QCohstack{\mathcal{QC}\!\mathit{oh}}
\def\Cohstack{\mathcal{C}\!\mathit{oh}}
\def\Spacesstack{\mathcal{S}\!\mathit{paces}}
\def\Quotfunctor{\mathrm{Quot}}
\def\Hilbfunctor{\mathrm{Hilb}}
\def\Curvesstack{\mathcal{C}\!\mathit{urves}}
\def\Polarizedstack{\mathcal{P}\!\mathit{olarized}}
\def\Complexesstack{\mathcal{C}\!\mathit{omplexes}}
% \Pic is the operator that assigns to X its picard group, usage \Pic(X)
% \Picardstack_{X/B} denotes the Picard stack of X over B
% \Picardfunctor_{X/B} denotes the Picard functor of X over B
\def\Pic{\mathop{\mathrm{Pic}}\nolimits}
\def\Picardstack{\mathcal{P}\!\mathit{ic}}
\def\Picardfunctor{\mathrm{Pic}}
\def\Deformationcategory{\mathcal{D}\!\mathit{ef}}


% OK, start here.
%
\begin{document}

\title{Algebraic Spaces}


\maketitle

\phantomsection
\label{section-phantom}

\tableofcontents

\section{Introduction}
\label{section-introduction}

\noindent
Algebraic spaces were first introduced by Michael Artin,
see \cite{ArtinI}, \cite{ArtinII},
\cite{Artin-Theorem-Representability},
\cite{Artin-Construction-Techniques},
\cite{Artin-Algebraic-Spaces},
\cite{Artin-Algebraic-Approximation},
\cite{Artin-Implicit-Function},
and \cite{ArtinVersal}.
Some of the foundational material was developed jointly with
Knutson, who produced the book \cite{Kn}.
Artin defined (see \cite[Definition 1.3]{Artin-Implicit-Function})
an algebraic space as a sheaf for the \'etale topology
which is locally in the \'etale topology representable.
In most of Artin's work the categories of schemes
considered are schemes locally of finite type over a fixed
excellent Noetherian base.

\medskip\noindent
Our definition is slightly different from Artin's original definition.
Namely, our algebraic spaces are sheaves for the fppf topology
whose diagonal is representable and which have an \'etale ``cover''
by a scheme. Working with the fppf topology instead of the \'etale
topology is just a technical point and scarcely makes
any difference; we will show in
Bootstrap, Section \ref{bootstrap-section-spaces-etale}
that we would have gotten the same category of algebraic spaces
if we had worked with the \'etale topology. In that same chapter
we will prove that the condition on the diagonal
can in some sense be removed, see
Bootstrap, Section \ref{bootstrap-section-bootstrap}.

\medskip\noindent
After defining algebraic spaces we make some foundational observations.
The main result in this chapter is that with our definitions
an algebraic space is the same thing as an \'etale equivalence relation,
see the discussion in Section \ref{section-presentations} and
Theorem \ref{theorem-presentation}. The analogue of this theorem in
Artin's setting is \cite[Theorem 1.5]{Artin-Implicit-Function}, or
\cite[Proposition II.1.7]{Kn}. In other words, the sheaf
defined by an \'etale equivalence relation has a representable diagonal.
It follows that our definition agrees with Artin's original definition
in a broad sense. It also means that one can give examples of algebraic
spaces by simply writing down an \'etale equivalence relation.

\medskip\noindent
In Section \ref{section-separation} we introduce various separation
axioms on algebraic spaces that we have found in the literature.
Finally in Section \ref{section-examples}
we give some weird and not so weird examples of algebraic spaces.






\section{General remarks}
\label{section-general}

\noindent
We work in a suitable big fppf site $\Sch_{fppf}$
as in Topologies, Definition \ref{topologies-definition-big-fppf-site}.
So, if not explicitly stated otherwise all schemes will be objects
of $\Sch_{fppf}$.
In Section \ref{section-change-big-site} we discuss what
changes if you change the big fppf site.

\medskip\noindent
We will always work relative to a base $S$ contained in $\Sch_{fppf}$.
And we will then work with the big fppf site $(\Sch/S)_{fppf}$,
see Topologies, Definition \ref{topologies-definition-big-small-fppf}.
The absolute case can be recovered by taking
$S = \Spec(\mathbf{Z})$.

\medskip\noindent
If $U, T$ are schemes over $S$, then we denote
$U(T)$ for the set of $T$-valued points {\it over} $S$.
In a formula: $U(T) = \Mor_S(T, U)$.

\medskip\noindent
Note that any fpqc covering is a universal effective epimorphism, see
Descent, Lemma \ref{descent-lemma-fpqc-universal-effective-epimorphisms}.
Hence the topology on $\Sch_{fppf}$
is weaker than the canonical topology and all representable presheaves
are sheaves.







\section{Representable morphisms of presheaves}
\label{section-representable}

\noindent
Let $S$ be a scheme contained in $\Sch_{fppf}$.
Let $F, G : (\Sch/S)_{fppf}^{opp} \to \textit{Sets}$.
Let $a : F \to G$ be a representable transformation of functors, see
Categories,
Definition \ref{categories-definition-representable-map-presheaves}.
This means that for every
$U \in \Ob((\Sch/S)_{fppf})$ and
any $\xi \in G(U)$ the fiber product $h_U \times_{\xi, G} F$ is representable.
Choose a representing object $V_\xi$ and an isomorphism
$h_{V_\xi} \to h_U \times_G F$.
By the Yoneda lemma, see Categories, Lemma \ref{categories-lemma-yoneda},
the projection $h_{V_\xi} \to h_U \times_G F \to h_U$ comes from a unique
morphism of schemes $a_\xi : V_\xi \to U$.
Suggestively we could represent this by the diagram
$$
\xymatrix{
V_\xi \ar@{~>}[r] \ar[d]_{a_\xi} & h_{V_\xi} \ar[d] \ar[r] & F \ar[d]^a \\
U \ar@{~>}[r] & h_U \ar[r]^\xi & G
}
$$
where the squiggly arrows represent the Yoneda embedding.
Here are some lemmas about this notion that work in great generality.

\begin{lemma}
\label{lemma-morphism-schemes-gives-representable-transformation}
Let $S$ be a scheme contained in $\Sch_{fppf}$ and let
$X$, $Y$ be objects of $(\Sch/S)_{fppf}$.
Let $f : X \to Y$ be a morphism of schemes.
Then
$$
h_f : h_X \longrightarrow h_Y
$$
is a representable transformation of functors.
\end{lemma}

\begin{proof}
This is formal and relies only on the fact that
the category $(\Sch/S)_{fppf}$ has fibre products.
\end{proof}

\begin{lemma}
\label{lemma-composition-representable-transformations}
Let $S$ be a scheme contained in $\Sch_{fppf}$.
Let $F, G, H : (\Sch/S)_{fppf}^{opp} \to \textit{Sets}$.
Let $a : F \to G$, $b : G \to H$ be representable transformations of functors.
Then
$$
b \circ a : F \longrightarrow H
$$
is a representable transformation of functors.
\end{lemma}

\begin{proof}
This is entirely formal and works in any category.
\end{proof}

\begin{lemma}
\label{lemma-base-change-representable-transformations}
Let $S$ be a scheme contained in $\Sch_{fppf}$.
Let $F, G, H : (\Sch/S)_{fppf}^{opp} \to \textit{Sets}$.
Let $a : F \to G$ be a representable transformation of functors.
Let $b : H \to G$ be any transformation of functors.
Consider the fibre product diagram
$$
\xymatrix{
H \times_{b, G, a} F \ar[r]_-{b'} \ar[d]_{a'} & F \ar[d]^a \\
H \ar[r]^b & G
}
$$
Then the base change $a'$ is a representable transformation of functors.
\end{lemma}

\begin{proof}
This is entirely formal and works in any category.
\end{proof}

\begin{lemma}
\label{lemma-product-representable-transformations}
Let $S$ be a scheme contained in $\Sch_{fppf}$.
Let $F_i, G_i : (\Sch/S)_{fppf}^{opp} \to \textit{Sets}$, $i = 1, 2$.
Let $a_i : F_i \to G_i$, $i = 1, 2$
be representable transformations of functors.
Then
$$
a_1 \times a_2 : F_1 \times F_2 \longrightarrow G_1 \times G_2
$$
is a representable transformation of functors.
\end{lemma}

\begin{proof}
Write $a_1 \times a_2$ as the composition
$F_1 \times F_2 \to G_1 \times F_2 \to G_1 \times G_2$.
The first arrow is the base change of $a_1$ by the map
$G_1 \times F_2 \to G_1$, and the second arrow
is the base change of $a_2$ by the map
$G_1 \times G_2 \to G_2$. Hence this lemma is a formal
consequence of Lemmas \ref{lemma-composition-representable-transformations}
and \ref{lemma-base-change-representable-transformations}.
\end{proof}

\begin{lemma}
\label{lemma-representable-transformation-to-sheaf}
Let $S$ be a scheme contained in $\Sch_{fppf}$.
Let $F, G : (\Sch/S)_{fppf}^{opp} \to \textit{Sets}$.
Let $a : F \to G$ be a representable transformation of functors.
If $G$ is a sheaf, then so is $F$.
\end{lemma}

\begin{proof}
Let $\{\varphi_i : T_i \to T\}$ be a covering of the site
$(\Sch/S)_{fppf}$.
Let $s_i \in F(T_i)$ which satisfy the sheaf condition.
Then $\sigma_i = a(s_i) \in G(T_i)$ satisfy the sheaf condition
also. Hence there exists a unique $\sigma \in G(T)$ such
that $\sigma_i = \sigma|_{T_i}$. By assumption
$F' = h_T \times_{\sigma, G, a} F$ is a representable presheaf
and hence (see remarks in Section \ref{section-general}) a sheaf.
Note that $(\varphi_i, s_i) \in F'(T_i)$ satisfy the
sheaf condition also, and hence come from some unique
$(\text{id}_T, s) \in F'(T)$. Clearly $s$ is the section of
$F$ we are looking for.
\end{proof}

\begin{lemma}
\label{lemma-representable-transformation-diagonal}
Let $S$ be a scheme contained in $\Sch_{fppf}$.
Let $F, G : (\Sch/S)_{fppf}^{opp} \to \textit{Sets}$.
Let $a : F \to G$ be a representable transformation of functors.
Then $\Delta_{F/G} : F \to F \times_G F$ is representable.
\end{lemma}

\begin{proof}
Let $U \in \Ob((\Sch/S)_{fppf})$. Let
$\xi = (\xi_1, \xi_2) \in (F \times_G F)(U)$.
Set $\xi' = a(\xi_1) = a(\xi_2) \in G(U)$.
By assumption there exist a scheme $V$ and a morphism $V \to U$
representing the fibre product $h_U \times_{\xi', G} F$.
In particular, the elements $\xi_1, \xi_2$ give morphisms
$f_1, f_2 : U \to V$ over $U$. Because $V$ represents the
fibre product $h_U \times_{\xi', G} F$ and because
$\xi' = a \circ \xi_1 = a \circ \xi_2$
we see that if $g : U' \to U$ is a morphism then
$$
g^*\xi_1 = g^*\xi_2
\Leftrightarrow
f_1 \circ g = f_2 \circ g.
$$
In other words, we see that $h_U \times_{\xi, F \times_G F} F$
is represented by $V \times_{\Delta, V \times V, (f_1, f_2)} U$
which is a scheme.
\end{proof}










\section{Lists of useful properties of morphisms of schemes}
\label{section-lists}

\noindent
For ease of reference we list in the following remarks the
properties of morphisms which possess some of the properties
required of them in later results.

\begin{remark}
\label{remark-list-properties-stable-base-change}
Here is a list of properties/types of morphisms
which are {\it stable under arbitrary base change}:
\begin{enumerate}
\item closed, open, and locally closed immersions, see
Schemes, Lemma \ref{schemes-lemma-base-change-immersion},
\item quasi-compact, see
Schemes, Lemma \ref{schemes-lemma-quasi-compact-preserved-base-change},
\item universally closed, see
Schemes, Definition \ref{schemes-definition-universally-closed},
\item (quasi-)separated, see
Schemes, Lemma \ref{schemes-lemma-separated-permanence},
\item monomorphism, see
Schemes, Lemma \ref{schemes-lemma-base-change-monomorphism}
\item surjective, see
Morphisms, Lemma \ref{morphisms-lemma-base-change-surjective},
\item universally injective, see
Morphisms, Lemma \ref{morphisms-lemma-universally-injective},
\item affine, see
Morphisms, Lemma \ref{morphisms-lemma-base-change-affine},
\item quasi-affine, see
Morphisms, Lemma \ref{morphisms-lemma-base-change-quasi-affine},
\item (locally) of finite type, see
Morphisms, Lemma \ref{morphisms-lemma-base-change-finite-type},
\item (locally) quasi-finite, see
Morphisms, Lemma \ref{morphisms-lemma-base-change-quasi-finite},
\item (locally) of finite presentation, see
Morphisms, Lemma \ref{morphisms-lemma-base-change-finite-presentation},
\item locally of finite type of relative dimension $d$, see
Morphisms, Lemma \ref{morphisms-lemma-base-change-relative-dimension-d},
\item universally open, see
Morphisms, Definition \ref{morphisms-definition-open},
\item flat, see
Morphisms, Lemma \ref{morphisms-lemma-base-change-flat},
\item syntomic, see
Morphisms, Lemma \ref{morphisms-lemma-base-change-syntomic},
\item smooth, see
Morphisms, Lemma \ref{morphisms-lemma-base-change-smooth},
\item unramified (resp.\ G-unramified), see
Morphisms, Lemma \ref{morphisms-lemma-base-change-unramified},
\item \'etale, see
Morphisms, Lemma \ref{morphisms-lemma-base-change-etale},
\item proper, see
Morphisms, Lemma \ref{morphisms-lemma-base-change-proper},
\item H-projective, see
Morphisms, Lemma \ref{morphisms-lemma-H-projective-base-change},
\item (locally) projective, see
Morphisms, Lemma \ref{morphisms-lemma-base-change-projective},
\item finite or integral, see
Morphisms, Lemma \ref{morphisms-lemma-base-change-finite},
\item finite locally free, see
Morphisms, Lemma \ref{morphisms-lemma-base-change-finite-locally-free},
\item universally submersive, see
Morphisms, Lemma \ref{morphisms-lemma-base-change-universally-submersive},
\item universal homeomorphism, see
Morphisms, Lemma \ref{morphisms-lemma-base-change-universal-homeomorphism}.
\end{enumerate}
Add more as needed.
\end{remark}

\begin{remark}
\label{remark-list-properties-stable-composition}
Of the properties of morphisms which are stable under base change
(as listed in
Remark \ref{remark-list-properties-stable-base-change})
the following are also {\it stable under compositions}:
\begin{enumerate}
\item closed, open and locally closed immersions, see
Schemes, Lemma \ref{schemes-lemma-composition-immersion},
\item quasi-compact, see
Schemes, Lemma \ref{schemes-lemma-composition-quasi-compact},
\item universally closed, see
Morphisms, Lemma \ref{morphisms-lemma-composition-proper},
\item (quasi-)separated, see
Schemes, Lemma \ref{schemes-lemma-separated-permanence},
\item monomorphism, see
Schemes, Lemma \ref{schemes-lemma-composition-monomorphism},
\item surjective, see
Morphisms, Lemma \ref{morphisms-lemma-composition-surjective},
\item universally injective, see
Morphisms, Lemma \ref{morphisms-lemma-composition-universally-injective},
\item affine, see
Morphisms, Lemma \ref{morphisms-lemma-composition-affine},
\item quasi-affine, see
Morphisms, Lemma \ref{morphisms-lemma-composition-quasi-affine},
\item (locally) of finite type, see
Morphisms, Lemma \ref{morphisms-lemma-composition-finite-type},
\item (locally) quasi-finite, see
Morphisms, Lemma \ref{morphisms-lemma-composition-quasi-finite},
\item (locally) of finite presentation, see
Morphisms, Lemma \ref{morphisms-lemma-composition-finite-presentation},
\item universally open, see
Morphisms, Lemma \ref{morphisms-lemma-composition-open},
\item flat, see
Morphisms, Lemma \ref{morphisms-lemma-composition-flat},
\item syntomic, see
Morphisms, Lemma \ref{morphisms-lemma-composition-syntomic},
\item smooth, see
Morphisms, Lemma \ref{morphisms-lemma-composition-smooth},
\item unramified (resp.\ G-unramified), see
Morphisms, Lemma \ref{morphisms-lemma-composition-unramified},
\item \'etale, see
Morphisms, Lemma \ref{morphisms-lemma-composition-etale},
\item proper, see
Morphisms, Lemma \ref{morphisms-lemma-composition-proper},
\item H-projective, see
Morphisms, Lemma \ref{morphisms-lemma-H-projective-composition},
\item finite or integral, see
Morphisms, Lemma \ref{morphisms-lemma-composition-finite},
\item finite locally free, see
Morphisms, Lemma \ref{morphisms-lemma-composition-finite-locally-free},
\item universally submersive, see
Morphisms, Lemma \ref{morphisms-lemma-composition-universally-submersive},
\item universal homeomorphism, see
Morphisms, Lemma \ref{morphisms-lemma-composition-universal-homeomorphism}.
\end{enumerate}
Add more as needed.
\end{remark}

\begin{remark}
\label{remark-list-properties-fpqc-local-base}
Of the properties mentioned which are stable under base change
(as listed in Remark \ref{remark-list-properties-stable-base-change})
the following are also {\it fpqc local on the base}
(and a fortiori fppf local on the base):
\begin{enumerate}
\item for immersions we have this for
\begin{enumerate}
\item closed immersions, see
Descent, Lemma \ref{descent-lemma-descending-property-closed-immersion},
\item open immersions, see
Descent, Lemma \ref{descent-lemma-descending-property-open-immersion}, and
\item quasi-compact immersions, see
Descent,
Lemma \ref{descent-lemma-descending-property-quasi-compact-immersion},
\end{enumerate}
\item quasi-compact, see
Descent, Lemma \ref{descent-lemma-descending-property-quasi-compact},
\item universally closed, see
Descent, Lemma
\ref{descent-lemma-descending-property-universally-closed},
\item (quasi-)separated, see
Descent, Lemmas
\ref{descent-lemma-descending-property-quasi-separated}, and
\ref{descent-lemma-descending-property-separated},
\item monomorphism, see
Descent, Lemma \ref{descent-lemma-descending-property-monomorphism},
\item surjective, see
Descent, Lemma \ref{descent-lemma-descending-property-surjective},
\item universally injective, see
Descent, Lemma \ref{descent-lemma-descending-property-universally-injective},
\item affine, see
Descent, Lemma \ref{descent-lemma-descending-property-affine},
\item quasi-affine, see
Descent, Lemma \ref{descent-lemma-descending-property-quasi-affine},
\item (locally) of finite type, see
Descent,
Lemmas \ref{descent-lemma-descending-property-locally-finite-type}, and
\ref{descent-lemma-descending-property-finite-type},
\item (locally) quasi-finite, see
Descent, Lemma \ref{descent-lemma-descending-property-quasi-finite},
\item (locally) of finite presentation, see
Descent, Lemmas
\ref{descent-lemma-descending-property-locally-finite-presentation}, and
\ref{descent-lemma-descending-property-finite-presentation},
\item locally of finite type of relative dimension $d$, see
Descent,
Lemma \ref{descent-lemma-descending-property-relative-dimension-d},
\item universally open, see
Descent, Lemma \ref{descent-lemma-descending-property-universally-open},
\item flat, see
Descent, Lemma \ref{descent-lemma-descending-property-flat},
\item syntomic, see
Descent, Lemma \ref{descent-lemma-descending-property-syntomic},
\item smooth, see
Descent, Lemma \ref{descent-lemma-descending-property-smooth},
\item unramified (resp.\ G-unramified), see
Descent, Lemma \ref{descent-lemma-descending-property-unramified},
\item \'etale, see
Descent, Lemma \ref{descent-lemma-descending-property-etale},
\item proper, see
Descent, Lemma \ref{descent-lemma-descending-property-proper},
\item finite or integral, see
Descent, Lemma \ref{descent-lemma-descending-property-finite},
\item finite locally free, see
Descent, Lemma \ref{descent-lemma-descending-property-finite-locally-free},
\item universally submersive, see
Descent, Lemma \ref{descent-lemma-descending-property-universally-submersive},
\item universal homeomorphism, see
Descent, Lemma \ref{descent-lemma-descending-property-universal-homeomorphism}.
\end{enumerate}
Note that the property of being an ``immersion'' may not be fpqc local
on the base, but in
Descent, Lemma \ref{descent-lemma-descending-fppf-property-immersion}
we proved that it is fppf local on the base.
\end{remark}








\section{Properties of representable morphisms of presheaves}
\label{section-representable-properties}

\noindent
Here is the definition that makes this work.

\begin{definition}
\label{definition-relative-representable-property}
With $S$, and $a : F \to G$ representable as above.
Let $\mathcal{P}$ be a property of morphisms of schemes which
\begin{enumerate}
\item is preserved under any base change,
see Schemes, Definition \ref{schemes-definition-preserved-by-base-change},
and
\item is fppf local on the base, see
Descent, Definition \ref{descent-definition-property-morphisms-local}.
\end{enumerate}
In this case we say that $a$ has {\it property $\mathcal{P}$} if for every
$U \in \Ob((\Sch/S)_{fppf})$ and
any $\xi \in G(U)$ the resulting morphism of schemes
$V_\xi \to U$ has property $\mathcal{P}$.
\end{definition}

\noindent
It is important to note that we will only use this definition for
properties of morphisms that are stable under base change, and
local in the fppf topology on the base. This is
not because the definition doesn't make sense otherwise; rather it
is because we may want to give a different definition which is
better suited to the property we have in mind.

\begin{remark}
\label{remark-warning}
Consider the property $\mathcal{P}=$``surjective''.
In this case there could be some ambiguity if we say
``let $F \to G$ be a surjective map''.
Namely, we could mean the notion defined
in Definition \ref{definition-relative-representable-property}
above, or we could mean a surjective map of presheaves, see
Sites, Definition \ref{sites-definition-presheaves-injective-surjective},
or, if both $F$ and $G$ are sheaves,
we could mean a surjective map of sheaves, see
Sites, Definition \ref{sites-definition-sheaves-injective-surjective}.
If not mentioned otherwise when discussing morphisms of algebraic spaces
we will always mean the first. See
Lemma \ref{lemma-surjective-flat-locally-finite-presentation}
for a case where surjectivity implies surjectivity as a map of sheaves.
\end{remark}

\noindent
Here is a sanity check.

\begin{lemma}
\label{lemma-morphism-schemes-gives-representable-transformation-property}
Let $S$, $X$, $Y$ be objects of $\Sch_{fppf}$.
Let $f : X \to Y$ be a morphism of schemes.
Let $\mathcal{P}$ be as in
Definition \ref{definition-relative-representable-property}.
Then $h_X \longrightarrow h_Y$ has property $\mathcal{P}$ if
and only if $f$ has property $\mathcal{P}$.
\end{lemma}

\begin{proof}
Note that the lemma makes sense by
Lemma \ref{lemma-morphism-schemes-gives-representable-transformation}.
Proof omitted.
\end{proof}

\begin{lemma}
\label{lemma-composition-representable-transformations-property}
Let $S$ be a scheme contained in $\Sch_{fppf}$.
Let $F, G, H : (\Sch/S)_{fppf}^{opp} \to \textit{Sets}$.
Let $\mathcal{P}$ be a property as in
Definition \ref{definition-relative-representable-property}
which is stable under composition.
Let $a : F \to G$, $b : G \to H$ be representable transformations of functors.
If $a$ and $b$ have property $\mathcal{P}$ so does
$b \circ a : F \longrightarrow H$.
\end{lemma}

\begin{proof}
Note that the lemma makes sense by
Lemma \ref{lemma-composition-representable-transformations}.
Proof omitted.
\end{proof}

\begin{lemma}
\label{lemma-base-change-representable-transformations-property}
Let $S$ be a scheme contained in $\Sch_{fppf}$.
Let $F, G, H : (\Sch/S)_{fppf}^{opp} \to \textit{Sets}$.
Let $\mathcal{P}$ be a property as in
Definition \ref{definition-relative-representable-property}.
Let $a : F \to G$ be a representable transformation of functors.
Let $b : H \to G$ be any transformation of functors.
Consider the fibre product diagram
$$
\xymatrix{
H \times_{b, G, a} F \ar[r]_-{b'} \ar[d]_{a'} & F \ar[d]^a \\
H \ar[r]^b & G
}
$$
If $a$ has property $\mathcal{P}$ then also the base change $a'$
has property $\mathcal{P}$.
\end{lemma}

\begin{proof}
Note that the lemma makes sense by
Lemma \ref{lemma-base-change-representable-transformations}.
Proof omitted.
\end{proof}

\begin{lemma}
\label{lemma-descent-representable-transformations-property}
Let $S$ be a scheme contained in $\Sch_{fppf}$.
Let $F, G, H : (\Sch/S)_{fppf}^{opp} \to \textit{Sets}$.
Let $\mathcal{P}$ be a property as in
Definition \ref{definition-relative-representable-property}.
Let $a : F \to G$ be a representable transformation of functors.
Let $b : H \to G$ be any transformation of functors.
Consider the fibre product diagram
$$
\xymatrix{
H \times_{b, G, a} F \ar[r]_-{b'} \ar[d]_{a'} & F \ar[d]^a \\
H \ar[r]^b & G
}
$$
Assume that $b$ induces a surjective map of fppf sheaves $H^\# \to G^\#$.
In this case, if $a'$ has property $\mathcal{P}$, then also $a$
has property $\mathcal{P}$.
\end{lemma}

\begin{proof}
First we remark that by
Lemma \ref{lemma-base-change-representable-transformations}
the transformation $a'$ is representable.
Let $U \in \Ob((\Sch/S)_{fppf})$, and let
$\xi \in G(U)$. By assumption there exists an fppf covering
$\{U_i \to U\}_{i \in I}$ and elements $\xi_i \in H(U_i)$ mapping
to $\xi|_U$ via $b$. From general category theory it follows that for
each $i$ we have a fibre product diagram
$$
\xymatrix{
U_i \times_{\xi_i, H, a'} (H \times_{b, G, a} F) \ar[r] \ar[d] &
U \times_{\xi, G, a} F \ar[d] \\
U_i \ar[r] & U
}
$$
By assumption the left vertical arrow is a morphism of schemes which
has property $\mathcal{P}$. Since $\mathcal{P}$ is local in the fppf
topology this implies that also the right vertical arrow has property
$\mathcal{P}$ as desired.
\end{proof}

\begin{lemma}
\label{lemma-product-representable-transformations-property}
Let $S$ be a scheme contained in $\Sch_{fppf}$.
Let $F_i, G_i : (\Sch/S)_{fppf}^{opp} \to \textit{Sets}$,
$i = 1, 2$.
Let $a_i : F_i \to G_i$, $i = 1, 2$ be representable transformations
of functors.
Let $\mathcal{P}$ be a property as in
Definition \ref{definition-relative-representable-property}
which is stable under composition.
If $a_1$ and $a_2$ have property $\mathcal{P}$ so does
$a_1 \times a_2 : F_1 \times F_2 \longrightarrow G_1 \times G_2$.
\end{lemma}

\begin{proof}
Note that the lemma makes sense by
Lemma \ref{lemma-product-representable-transformations}.
Proof omitted.
\end{proof}

\begin{lemma}
\label{lemma-representable-transformations-property-implication}
Let $S$ be a scheme contained in $\Sch_{fppf}$.
Let $F, G : (\Sch/S)_{fppf}^{opp} \to \textit{Sets}$.
Let $a : F \to G$ be a representable transformation of functors.
Let $\mathcal{P}$, $\mathcal{P}'$ be properties as in
Definition \ref{definition-relative-representable-property}.
Suppose that for any morphism of schemes $f : X \to Y$
we have $\mathcal{P}(f) \Rightarrow \mathcal{P}'(f)$.
If $a$ has property $\mathcal{P}$ then
$a$ has property $\mathcal{P}'$.
\end{lemma}

\begin{proof}
Formal.
\end{proof}

\begin{lemma}
\label{lemma-surjective-flat-locally-finite-presentation}
Let $S$ be a scheme.
Let $F, G : (\Sch/S)_{fppf}^{opp} \to \textit{Sets}$ be sheaves.
Let $a : F \to G$ be representable, flat,
locally of finite presentation, and surjective.
Then $a : F \to G$ is surjective as a map of sheaves.
\end{lemma}

\begin{proof}
Let $T$ be a scheme over $S$ and let $g : T \to G$ be a $T$-valued point of
$G$. By assumption $T' = F \times_G T$ is (representable by) a scheme and
the morphism $T' \to T$ is a flat, locally of finite presentation, and
surjective. Hence $\{T' \to T\}$ is an fppf covering such
that $g|_{T'} \in G(T')$ comes from an element of $F(T')$, namely
the map $T' \to F$. This proves the map is surjective as
a map of sheaves, see
Sites, Definition \ref{sites-definition-sheaves-injective-surjective}.
\end{proof}

\noindent
Here is a characterization of those functors for which the
diagonal is representable.

\begin{lemma}
\label{lemma-representable-diagonal}
Let $S$ be a scheme contained in $\Sch_{fppf}$.
Let $F$ be a presheaf of sets on $(\Sch/S)_{fppf}$.
The following are equivalent:
\begin{enumerate}
\item the diagonal $F \to F \times F$ is representable,
\item for $U \in \Ob((\Sch/S)_{fppf})$ and any $a \in F(U)$
the map $a : h_U \to F$ is representable,
\item for every pair $U, V \in \Ob((\Sch/S)_{fppf})$
and any $a \in F(U)$, $b \in F(V)$ the fibre product
$h_U \times_{a, F, b} h_V$ is representable.
\end{enumerate}
\end{lemma}

\begin{proof}
This is completely formal, see
Categories, Lemma \ref{categories-lemma-representable-diagonal}.
It depends only on the fact that the category $(\Sch/S)_{fppf}$
has products of pairs of objects and fibre products, see
Topologies, Lemma \ref{topologies-lemma-fibre-products-fppf}.
\end{proof}

\noindent
In the situation of the lemma, for any morphism
$\xi : h_U \to F$ as in the lemma, it makes sense
to say that $\xi$ has property $\mathcal{P}$, for any property
as in Definition \ref{definition-relative-representable-property}.
In particular this holds for $\mathcal{P} = $ ``surjective''
and $\mathcal{P} = $ ``\'etale'', see
Remark \ref{remark-list-properties-fpqc-local-base}
above. We will use this remark in the definition
of algebraic spaces below.

\begin{lemma}
\label{lemma-transformation-diagonal-properties}
Let $S$ be a scheme contained in $\Sch_{fppf}$.
Let $F$ be a presheaf of sets on $(\Sch/S)_{fppf}$.
Let $\mathcal{P}$ be a property as in
Definition \ref{definition-relative-representable-property}.
If for every $U, V \in \Ob((\Sch/S)_{fppf})$ and $a \in F(U)$,
$b \in F(V)$ we have
\begin{enumerate}
\item $h_U \times_{a, F, b} h_V$ is representable, say by the scheme $W$, and
\item the morphism $W \to U \times_S V$ corresponding to
$h_U \times_{a, F, b} h_V \to h_U \times h_V$ has property $\mathcal{P}$,
\end{enumerate}
then $\Delta : F \to F \times F$ is representable and has
property $\mathcal{P}$.
\end{lemma}

\begin{proof}
Observe that $\Delta$ is representable by
Lemma \ref{lemma-representable-diagonal}.
We can formulate condition (2) as saying that
the transformation $h_U \times_{a, F, b} h_V \to h_{U \times_S V}$
has property $\mathcal{P}$, see Lemma
\ref{lemma-morphism-schemes-gives-representable-transformation-property}.
Consider $T \in \Ob((\Sch/S)_{fppf})$ and $(a, b) \in (F \times F)(T)$.
Observe that we have the commutative diagram
$$
\xymatrix{
F \times_{\Delta, F \times F, (a, b)} h_T \ar[d] \ar[r] &
h_T \ar[d]^{\Delta_{T/S}} \\
h_T \times_{a, F, b} h_T \ar[r] \ar[d] &
h_{T \times_S T} \ar[d]^{(a, b)} \\
F \ar[r]^\Delta & F \times F
}
$$
both of whose squares are cartesian. In this way we see that
the morphism $F \times_{F \times F} h_T \to h_T$ is the base
change of a morphism having property $\mathcal{P}$ by
$\Delta_{T/S}$. Since $\mathcal{P}$ is preserved under base
change this finishes the proof.
\end{proof}


















\section{Algebraic spaces}
\label{section-algebraic-spaces}

\noindent
Here is the definition.

\begin{definition}
\label{definition-algebraic-space}
Let $S$ be a scheme contained in $\Sch_{fppf}$.
An {\it algebraic space over $S$} is a presheaf
$$
F : (\Sch/S)^{opp}_{fppf} \longrightarrow \textit{Sets}
$$
with the following properties
\begin{enumerate}
\item The presheaf $F$ is a sheaf.
\item The diagonal morphism $F  \to F \times F$ is representable.
\item There exists a scheme $U \in \Ob((\Sch/S)_{fppf})$
and a map $h_U \to F$ which is surjective, and \'etale.
\end{enumerate}
\end{definition}

\noindent
There are two differences with the ``usual'' definition, for example the
definition in Knutson's book \cite{Kn}.

\medskip\noindent
The first is that we require $F$ to be a sheaf in the fppf topology.
One reason for doing this is that many natural examples
of algebraic spaces satisfy the sheaf condition for the fppf coverings
(and even for fpqc coverings). Also, one of the reasons that algebraic
spaces have been so useful is via Michael Artin's results on algebraic spaces.
Built into his method is a condition which guarantees the result is
locally of finite presentation over $S$.
Combined it somehow seems to us that the fppf topology
is the natural topology to work with. In the end the category
of algebraic spaces ends up being the same. See
Bootstrap, Section \ref{bootstrap-section-spaces-etale}.

\medskip\noindent
The second is that we only require the diagonal map for $F$ to be
representable, whereas in \cite{Kn} it is required that it also
be quasi-compact. If $F = h_U$ for some scheme $U$ over $S$
this corresponds to the condition that $U$ be quasi-separated.
Our point of view is to try to prove a certain
number of the results that follow only assuming that the diagonal
of $F$ be representable, and simply add an additional hypothesis wherever
this is necessary. In any case it has the pleasing consequence that
the following lemma is true.

\begin{lemma}
\label{lemma-scheme-is-space}
A scheme is an algebraic space. More precisely,
given a scheme $T \in \Ob((\Sch/S)_{fppf})$
the representable functor $h_T$ is an algebraic space.
\end{lemma}

\begin{proof}
The functor $h_T$ is a sheaf by our remarks in Section \ref{section-general}.
The diagonal $h_T \to h_T \times h_T = h_{T \times T}$ is
representable because $(\Sch/S)_{fppf}$ has fibre products.
The identity map $h_T \to h_T$ is surjective \'etale.
\end{proof}

\begin{definition}
\label{definition-morphism-algebraic-spaces}
Let $F$, $F'$ be algebraic spaces over $S$.
A {\it morphism $f : F \to F'$ of algebraic spaces over $S$}
is a transformation of functors from $F$ to $F'$.
\end{definition}

\noindent
The category of algebraic spaces over $S$ contains the category
$(\Sch/S)_{fppf}$ as a full subcategory via the
Yoneda embedding $T/S \mapsto h_T$. From now on we no longer distinguish
between a scheme $T/S$ and the algebraic space it represents.
Thus when we say ``Let $f : T \to F$ be a morphism from the scheme
$T$ to the algebraic space $F$'', we mean that
$T \in \Ob((\Sch/S)_{fppf})$, that $F$ is an
algebraic space over $S$, and that $f : h_T \to F$ is a morphism
of algebraic spaces over $S$.






\section{Fibre products of algebraic spaces}
\label{section-fibre-products}

\noindent
The category of algebraic spaces over $S$ has both products and
fibre products.

\begin{lemma}
\label{lemma-product-spaces}
Let $S$ be a scheme contained in $\Sch_{fppf}$.
Let $F, G$ be algebraic spaces over $S$.
Then $F \times G$ is an algebraic space, and is a product
in the category of algebraic spaces over $S$.
\end{lemma}

\begin{proof}
It is clear that $H = F \times G$ is a sheaf.
The diagonal of $H$ is simply the product of the
diagonals of $F$ and $G$. Hence it is representable by
Lemma \ref{lemma-product-representable-transformations}.
Finally, if $U \to F$ and $V \to G$ are surjective
\'etale morphisms, with $U, V \in \Ob((\Sch/S)_{fppf})$,
then $U \times V \to F \times G$ is surjective \'etale
by Lemma \ref{lemma-product-representable-transformations-property}.
\end{proof}

\begin{lemma}
\label{lemma-fibre-product-spaces-over-sheaf-with-representable-diagonal}
Let $S$ be a scheme contained in $\Sch_{fppf}$.
Let $H$ be a sheaf on $(\Sch/S)_{fppf}$ whose diagonal
is representable. Let $F, G$ be algebraic spaces over $S$.
Let $F \to H$, $G \to H$ be maps of sheaves.
Then $F \times_H G$ is an algebraic space.
\end{lemma}

\begin{proof}
We check the 3 conditions of
Definition \ref{definition-algebraic-space}.
A fibre product of sheaves is a sheaf, hence $F \times_H G$ is a sheaf.
The diagonal of $F \times_H G$ is the left vertical arrow in
$$
\xymatrix{
F \times_H G \ar[r] \ar[d]_\Delta &
F \times G \ar[d]^{\Delta_F \times \Delta_G} \\
(F \times F) \times_{(H \times H)} (G \times G) \ar[r] &
(F \times F) \times (G \times G)
}
$$
which is cartesian. Hence $\Delta$ is representable as the base change
of the morphism on the right which is representable, see
Lemmas \ref{lemma-product-representable-transformations} and
\ref{lemma-base-change-representable-transformations}.
Finally, let $U, V \in \Ob((\Sch/S)_{fppf})$
and $a : U \to F$, $b : V \to G$ be surjective and \'etale.
As $\Delta_H$ is representable, we see that $U \times_H V$ is a scheme.
The morphism
$$
U \times_H V \longrightarrow F \times_H G
$$
is surjective and \'etale as a composition of the base changes
$U \times_H V \to U \times_H G$ and $U \times_H G \to F \times_H G$
of the \'etale surjective morphisms $U \to F$ and $V \to G$, see
Lemmas \ref{lemma-composition-representable-transformations} and
\ref{lemma-base-change-representable-transformations}.
This proves the last condition of
Definition \ref{definition-algebraic-space}
holds and we conclude that $F \times_H G$ is an algebraic space.
\end{proof}

\begin{lemma}
\label{lemma-fibre-product-spaces}
Let $S$ be a scheme contained in $\Sch_{fppf}$.
Let $F \to H$, $G \to H$ be morphisms of algebraic spaces over $S$.
Then $F \times_H G$ is an algebraic space, and is a fibre product
in the category of algebraic spaces over $S$.
\end{lemma}

\begin{proof}
It follows from the stronger
Lemma \ref{lemma-fibre-product-spaces-over-sheaf-with-representable-diagonal}
that $F \times_H G$ is an algebraic space.
It is clear that $F \times_H G$
is a fibre product in the category of algebraic spaces over $S$
since that is a full subcategory of the category
of (pre)sheaves of sets on $(\Sch/S)_{fppf}$.
\end{proof}






\section{Glueing algebraic spaces}
\label{section-glueing-algebraic-spaces}

\noindent
In this section we really start abusing notation and not
distinguish between schemes and the spaces they represent.

\begin{lemma}
\label{lemma-coproduct-sheaves-open-and-closed}
Let $S \in \Ob(\Sch_{fppf})$. Let $F$ and $G$ be sheaves on
$(\Sch/S)_{fppf}^{opp}$ and denote $F \amalg G$ the coproduct
in the category of sheaves. The map $F \to F \amalg G$ is representable by
open and closed immersions.
\end{lemma}

\begin{proof}
Let $U$ be a scheme and let $\xi \in (F \amalg G)(U)$. Recall the
coproduct in the category of sheaves is the sheafification of
the coproduct presheaf (Sites, Lemma \ref{sites-lemma-colimit-sheaves}).
Thus there exists an fppf covering $\{g_i : U_i \to U\}_{i \in I}$
and a disjoint union decomposition $I = I' \amalg I''$ such that
$U_i \to U \to F \amalg G$ factors through $F$, resp.\ $G$
if and only if $i \in I'$, resp.\ $i \in I''$. Since $F$ and
$G$ have empty intersection in $F \amalg G$ we conclude that
$U_i \times_U U_j$ is empty if $i \in I'$ and $j \in I''$.
Hence $U' = \bigcup_{i \in I'} g_i(U_i)$ and
$U'' = \bigcup_{i \in I''} g_i(U_i)$ are disjoint open
(Morphisms, Lemma \ref{morphisms-lemma-fppf-open}) subschemes of $U$
with $U = U' \amalg U''$.
We omit the verification that $U' = U \times_{F \amalg G} F$.
\end{proof}

\begin{lemma}
\label{lemma-representable-sheaf-coproduct-sheaves}
Let $S \in \Ob(\Sch_{fppf})$.
Let $U \in \Ob((\Sch/S)_{fppf})$.
Given a set $I$ and sheaves $F_i$ on $\Ob((\Sch/S)_{fppf})$,
if $U \cong \coprod_{i\in I} F_i$
as sheaves, then each $F_i$ is representable by an open and closed
subscheme $U_i$ and $U \cong \coprod U_i$ as schemes.
\end{lemma}

\begin{proof}
By Lemma \ref{lemma-coproduct-sheaves-open-and-closed}
the map $F_i \to U$ is representable by open and closed immersions.
Hence $F_i$ is representable by an open and closed subscheme $U_i$ of $U$.
We have $U = \coprod U_i$ because we have $U \cong \coprod F_i$
as sheaves and we can test the equality on points.
\end{proof}

\begin{lemma}
\label{lemma-algebraic-space-coproduct-sheaves}
Let $S \in \Ob(\Sch_{fppf})$.
Let $F$ be an algebraic space over $S$.
Given a set $I$ and sheaves $F_i$ on
$\Ob((\Sch/S)_{fppf})$,
if $F \cong \coprod_{i\in I} F_i$ as sheaves,
then each $F_i$ is an algebraic space over $S$.
\end{lemma}

\begin{proof}
The representability of $F \to F \times F$ implies that each diagonal morphism
$F_i \to F_i \times F_i$ is representable (immediate from the definitions
and the fact that $F \times_{(F \times F)} (F_i \times F_i) = F_i$).
Choose a scheme $U$ in $(\Sch/S)_{fppf}$ and a surjective
\'etale morphism $U \to F$ (this exist by hypothesis).
The base change $U \times_F F_i \to F_i$ is surjective and \'etale
by Lemma \ref{lemma-base-change-representable-transformations-property}.
On the other hand, $U \times_F F_i$ is a scheme by
Lemma \ref{lemma-coproduct-sheaves-open-and-closed}.
Thus we have verified all the conditions in
Definition \ref{definition-algebraic-space}
and $F_i$ is an algebraic space.
\end{proof}

\noindent
The condition on the size of $I$ and the $F_i$ in the
following lemma may be ignored by those not worried about
set theoretic questions.

\begin{lemma}
\label{lemma-coproduct-algebraic-spaces}
Let $S \in \Ob(\Sch_{fppf})$.
Suppose given a set $I$ and algebraic spaces $F_i$, $i \in I$.
Then $F = \coprod_{i \in I} F_i$ is an algebraic space
provided $I$, and the $F_i$ are not too ``large'': for example if we
can choose surjective \'etale morphisms $U_i \to F_i$ such that
$\coprod_{i \in I} U_i$ is isomorphic to an object of
$(\Sch/S)_{fppf}$, then $F$ is an algebraic space.
\end{lemma}

\begin{proof}
By construction $F$ is a sheaf. We omit the verification that the
diagonal morphism of $F$ is representable. Finally, if $U$ is an
object of $(\Sch/S)_{fppf}$ isomorphic to $\coprod_{i \in I} U_i$
then it is straightforward to verify that the resulting map
$U \to \coprod F_i$ is surjective and \'etale.
\end{proof}

\noindent
Here is the analogue of Schemes, Lemma \ref{schemes-lemma-glue-functors}.

\begin{lemma}
\label{lemma-glueing-algebraic-spaces}
Let $S \in \Ob(\Sch_{fppf})$.
Let $F$ be a presheaf of sets on $(\Sch/S)_{fppf}$.
Assume
\begin{enumerate}
\item $F$ is a sheaf,
\item there exists an index set $I$
and subfunctors $F_i \subset F$ such that
\begin{enumerate}
\item each $F_i$ is an algebraic space,
\item each $F_i \to F$ is representable,
\item each $F_i \to F$ is an open immersion (see
Definition \ref{definition-relative-representable-property}),
\item the map $\coprod F_i \to F$ is surjective as a map of sheaves, and
\item $\coprod F_i$ is an algebraic space (set theoretic condition, see
Lemma \ref{lemma-coproduct-algebraic-spaces}).
\end{enumerate}
\end{enumerate}
Then $F$ is an algebraic space.
\end{lemma}

\begin{proof}
Let $T$ be an object of $(\Sch/S)_{fppf}$. Let $T \to F$ be a morphism.
By assumption (2)(b) and (2)(c) the fibre product $F_i \times_F T$
is representable by an open subscheme $V_i \subset T$. It follows that
$(\coprod F_i) \times_F T$ is represented by the scheme $\coprod V_i$ over $T$.
By assumption (2)(d) there exists an fppf covering $\{T_j \to T\}_{j \in J}$
such that $T_j \to T \to F$ factors through $F_i$, $i = i(j)$.
Hence $T_j \to T$ factors through the open subscheme $V_{i(j)} \subset T$.
Since $\{T_j \to T\}$ is jointly surjective, it follows that
$T = \bigcup V_i$ is an open covering. In particular, the transformation
of functors $\coprod F_i \to F$ is representable
and surjective in the sense of
Definition \ref{definition-relative-representable-property}
(see Remark \ref{remark-warning} for a discussion).

\medskip\noindent
Next, let $T' \to F$ be a second morphism from an object in $(\Sch/S)_{fppf}$.
Write as above $T' = \bigcup V'_i$ with $V'_i = T' \times_F F_i$.
To show that the diagonal $F \to F \times F$ is representable
we have to show that $G = T \times_F T'$ is representable, see
Lemma \ref{lemma-representable-diagonal}.
Consider the subfunctors $G_i = G \times_F F_i$.
Note that $G_i = V_i \times_{F_i} V'_i$, and hence is representable
as $F_i$ is an algebraic space.
By the above the $G_i$ form a Zariski covering of $G$.
Hence by Schemes, Lemma \ref{schemes-lemma-glue-functors}
we see $G$ is representable.

\medskip\noindent
Choose a scheme $U \in \Ob((\Sch/S)_{fppf})$ and a surjective
\'etale morphism $U \to \coprod F_i$ (this exists by hypothesis).
We may write $U = \coprod U_i$ with $U_i$ the inverse image of $F_i$,
see Lemma \ref{lemma-representable-sheaf-coproduct-sheaves}.
We claim that $U \to F$ is surjective and \'etale. Surjectivity follows
as $\coprod F_i \to F$ is surjective (see first paragraph of the proof)
by applying
Lemma \ref{lemma-composition-representable-transformations-property}.
Consider the fibre product $U \times_F T$ where $T \to F$ is as
above. We have to show that $U \times_F T \to T$ is \'etale.
Since $U \times_F T = \coprod U_i \times_F T$ it suffices to show
each $U_i \times_F T \to T$ is \'etale. Since
$U_i \times_F T = U_i \times_{F_i} V_i$ this follows from the
fact that $U_i \to F_i$ is \'etale and $V_i \to T$ is an open immersion
(and Morphisms, Lemmas \ref{morphisms-lemma-open-immersion-etale}
and \ref{morphisms-lemma-composition-etale}).
\end{proof}














\section{Presentations of algebraic spaces}
\label{section-presentations}

\noindent
Given an algebraic space we can find a ``presentation'' of it.


\begin{lemma}
\label{lemma-space-presentation}
Let $F$ be an algebraic space over $S$. Let $f : U \to F$ be a
surjective \'etale morphism from a scheme to $F$. Set $R = U \times_F U$.
Then
\begin{enumerate}
\item $j : R \to U \times_S U$ defines an equivalence relation on
$U$ over $S$ (see
Groupoids, Definition \ref{groupoids-definition-equivalence-relation}).
\item the morphisms $s, t : R \to U$ are \'etale, and
\item the diagram
$$
\xymatrix{
R \ar@<1ex>[r] \ar@<-1ex>[r] &
U \ar[r] &
F
}
$$
is a coequalizer diagram in $\Sh((\Sch/S)_{fppf})$.
\end{enumerate}
\end{lemma}

\begin{proof}
Let $T/S$ be an object of $(\Sch/S)_{fppf}$.
Then $R(T) = \{(a, b) \in U(T) \times U(T) \mid f \circ a = f \circ b\}$
which defines an equivalence relation on $U(T)$.
The morphisms $s, t : R \to U$ are \'etale because the morphism
$U \to F$ is \'etale.

\medskip\noindent
To prove (3) we first show that
$U \to F$ is a surjection of sheaves, see
Sites, Definition \ref{sites-definition-sheaves-injective-surjective}.
Let $\xi \in F(T)$ with $T$ as above. Let $V = T \times_{\xi, F, f}U$.
By assumption $V$ is a scheme and $V \to T$ is surjective \'etale.
Hence $\{V \to T\}$ is a covering for the fppf topology.
Since $\xi|_V$ factors through $U$ by construction we
conclude $U \to F$ is surjective. Surjectivity implies that
$F$ is the coequalizer of the diagram by
Sites, Lemma \ref{sites-lemma-coequalizer-surjection}.
\end{proof}

\noindent
This lemma suggests the following definitions.

\begin{definition}
\label{definition-etale-equivalence-relation}
Let $S$ be a scheme. Let $U$ be a scheme over $S$.
An {\it \'etale equivalence relation} on $U$ over $S$
is an equivalence relation $j : R \to U \times_S U$
such that $s, t : R \to U$ are \'etale morphisms of schemes.
\end{definition}

\begin{definition}
\label{definition-presentation}
Let $F$ be an algebraic space over $S$.
A {\it presentation} of $F$ is given by a scheme
$U$ over $S$ and an \'etale equivalence relation $R$ on $U$ over $S$, and
a surjective \'etale morphism $U \to F$ such that $R = U \times_F U$.
\end{definition}

\noindent
Equivalently we could ask for the existence of an isomorphism
$$
U/R \cong F
$$
where the quotient $U/R$ is as defined in
Groupoids, Section \ref{groupoids-section-quotient-sheaves}.
To construct algebraic spaces we will study the converse question, namely,
for which equivalence relations the quotient sheaf $U/R$ is an algebraic space.
It will finally turn out this is always the case if $R$ is an \'etale
equivalence relation on $U$ over $S$, see Theorem \ref{theorem-presentation}.





























\section{Algebraic spaces and equivalence relations}
\label{section-spaces-from-equivalence-relations}

\noindent
Suppose given a scheme $U$ over $S$
and an \'etale equivalence relation $R$ on $U$ over $S$.
We would like to show this defines an algebraic space.
We will produce a series of lemmas that prove the quotient sheaf $U/R$
(see Groupoids, Definition \ref{groupoids-definition-quotient-sheaf})
has all the properties required
of it in Definition \ref{definition-algebraic-space}.

\begin{lemma}
\label{lemma-pullback-etale-equivalence-relation}
Let $S$ be a scheme. Let $U$ be a scheme over $S$.
Let $j = (s, t) : R \to U \times_S U$
be an \'etale equivalence relation on $U$ over $S$.
Let $U' \to U$ be an \'etale morphism.
Let $R'$ be the restriction of $R$ to $U'$, see
Groupoids, Definition \ref{groupoids-definition-restrict-relation}.
Then $j' : R' \to U' \times_S U'$ is an \'etale equivalence
relation also.
\end{lemma}

\begin{proof}
It is clear from the description of $s', t'$ in
Groupoids, Lemma \ref{groupoids-lemma-restrict-groupoid}
that $s' , t' : R' \to U'$ are \'etale
as compositions of base changes of \'etale morphisms
(see Morphisms, Lemma \ref{morphisms-lemma-base-change-etale}
and \ref{morphisms-lemma-composition-etale}).
\end{proof}

\noindent
We will often use the following lemma to find open subspaces of algebraic
spaces. A slight improvement (with more general hypotheses) of this lemma is
Bootstrap, Lemma \ref{bootstrap-lemma-better-finding-opens}.

\begin{lemma}
\label{lemma-finding-opens}
Let $S$ be a scheme.
Let $U$ be a scheme over $S$.
Let $j = (s, t) : R \to U \times_S U$ be a pre-relation.
Let $g : U' \to U$ be a morphism.
Assume
\begin{enumerate}
\item $j$ is an equivalence relation,
\item $s, t : R \to U$ are surjective, flat and
locally of finite presentation,
\item $g$ is flat and locally of finite presentation.
\end{enumerate}
Let $R' = R|_{U'}$ be the restriction of $R$ to $U'$. Then
$U'/R' \to U/R$ is representable, and is an open immersion.
\end{lemma}

\begin{proof}
By Groupoids, Lemma \ref{groupoids-lemma-restrict-relation}
the morphism $j' = (s', t') : R' \to U' \times_S U'$
defines an equivalence relation. Since $g$ is flat and locally of
finite presentation we see that $g$ is universally open as well
(Morphisms, Lemma \ref{morphisms-lemma-fppf-open}).
For the same reason $s, t$ are universally open as well.
Let $W^1 = g(U') \subset U$, and let $W = t(s^{-1}(W^1))$.
Then $W^1$ and $W$ are open in $U$. Moreover, as $j$ is an
equivalence relation we have $t(s^{-1}(W)) = W$ (see
Groupoids, Lemma \ref{groupoids-lemma-constructing-invariant-opens}
for example).

\medskip\noindent
By
Groupoids,
Lemma \ref{groupoids-lemma-quotient-pre-equivalence-relation-restrict}
the map of sheaves $F' = U'/R' \to F = U/R$ is injective.
Let $a : T \to F$ be a morphism from a scheme into $U/R$.
We have to show that $T \times_F F'$ is representable
by an open subscheme of $T$.

\medskip\noindent
The morphism $a$ is given by the following data:
an fppf covering $\{\varphi_j : T_j \to T\}_{j \in J}$ of $T$ and
morphisms $a_j : T_j \to U$ such that the maps
$$
a_j \times a_{j'} :
T_j \times_T T_{j'}
\longrightarrow
U \times_S U
$$
factor through $j : R \to U \times_S U$ via some (unique) maps
$r_{jj'} : T_j \times_T T_{j'} \to R$. The system
$(a_j)$ corresponds to $a$ in the sense that the diagrams
$$
\xymatrix{
T_j \ar[r]_{a_j} \ar[d] & U \ar[d] \\
T \ar[r]^a & F
}
$$
commute.

\medskip\noindent
Consider the open subsets $W_j = a_j^{-1}(W) \subset T_j$.
Since $t(s^{-1}(W)) = W$ we see that
$$
W_j \times_T T_{j'} =
r_{jj'}^{-1}(t^{-1}(W)) = r_{jj'}^{-1}(s^{-1}(W)) =
T_j \times_T W_{j'}.
$$
By
Descent, Lemma \ref{descent-lemma-open-fpqc-covering}
this means there exists an open
$W_T \subset T$ such that $\varphi_j^{-1}(W_T) = W_j$ for all $j \in J$.
We claim that $W_T \to T$ represents $T \times_F F' \to T$.

\medskip\noindent
First, let us show that $W_T \to T \to F$ is an element of
$F'(W_T)$. Since $\{W_j \to W_T\}_{j \in J}$ is an
fppf covering of $W_T$, it is enough to show that
each $W_j \to U \to F$ is an element of $F'(W_j)$ (as $F'$ is a sheaf
for the fppf topology). Consider the commutative diagram
$$
\xymatrix{
W'_j \ar[rr] \ar[dd] \ar[rd] & & U' \ar[d]^g \\
& s^{-1}(W^1) \ar[r]_s \ar[d]^t & W^1 \ar[d] \\
W_j \ar[r]^{a_j|_{W_j}} & W \ar[r] & F
}
$$
where $W'_j = W_j \times_W s^{-1}(W^1) \times_{W^1} U'$.
Since $t$ and $g$ are surjective, flat and locally of finite
presentation, so is $W'_j \to W_j$. Hence the restriction of
the element $W_j \to U \to F$ to $W'_j$ is an element of $F'$
as desired.

\medskip\noindent
Suppose that $f : T' \to T$ is a morphism of schemes
such that $a|_{T'} \in F'(T')$. We have to show that
$f$ factors through the open $W_T$. Since
$\{T' \times_T T_j \to T'\}$ is an fppf covering of $T'$
it is enough to show each $T' \times_T T_j \to T$
factors through $W_T$. Hence we may assume $f$ factors
as $\varphi_j \circ f_j : T' \to T_j \to T$ for some $j$.
In this case the condition $a|_{T'} \in F'(T')$ means that there exists
some fppf covering $\{\psi_i : T'_i \to T'\}_{i \in I}$ and some
morphisms $b_i : T'_i \to U'$ such that
$$
\xymatrix{
T'_i \ar[r]_{b_i} \ar[d]_{f_j \circ \psi_i} & U' \ar[r]_g & U \ar[d] \\
T_j \ar[r]^{a_j} & U \ar[r] & F
}
$$
is commutative. This commutativity means that there exists a
morphism $r'_i : T'_i \to R$ such that
$t \circ r'_i = a_j \circ f_j \circ \psi_i$, and
$s \circ r'_i = g \circ b_i$. This implies that
$\Im(f_j \circ \psi_i) \subset W_j$ and we win.
\end{proof}

\noindent
The following lemma is not completely trivial although it looks
like it should be trivial.

\begin{lemma}
\label{lemma-when-it-works-it-works}
Let $S$ be a scheme. Let $U$ be a scheme over $S$.
Let $j = (s, t) : R \to U \times_S U$
be an \'etale equivalence relation on $U$ over $S$.
If the quotient $U/R$ is an algebraic space, then
$U \to U/R$ is \'etale and surjective. Hence
$(U, R, U \to U/R)$ is a presentation of the algebraic
space $U/R$.
\end{lemma}

\begin{proof}
Denote $c : U \to U/R$ the morphism in question.
Let $T$ be a scheme and let $a : T \to U/R$ be a morphism.
We have to show that the morphism (of schemes)
$\pi : T \times_{a, U/R, c} U \to T$ is \'etale and surjective.
The morphism $a$ corresponds to an fppf covering
$\{\varphi_i : T_i \to T\}$ and morphisms $a_i : T_i \to U$ such
that
$a_i \times a_{i'} : T_i \times_T T_{i'} \to U \times_S U$
factors through $R$, and such that $c \circ a_i = a \circ \varphi_i$.
Hence
$$
T_i \times_{\varphi_i, T} T \times_{a, U/R, c} U =
T_i \times_{c \circ a_i, U/R, c} U =
T_i \times_{a_i, U} U \times_{c, U/R, c} U = T_i \times_{a_i, U, t} R.
$$
Since $t$ is \'etale and surjective we conclude that
the base change of $\pi$ to $T_i$ is surjective and \'etale.
Since the property of being surjective and \'etale is local
on the base in the fpqc topology (see
Remark \ref{remark-list-properties-fpqc-local-base})
we win.
\end{proof}

\begin{lemma}
\label{lemma-presentation-quasi-compact}
Let $S$ be a scheme.
Let $U$ be a scheme over $S$.
Let $j = (s, t) : R \to U \times_S U$
be an \'etale equivalence relation on $U$ over $S$.
Assume that $U$ is affine. Then the quotient $F = U/R$
is an algebraic space, and $U \to F$ is \'etale and surjective.
\end{lemma}

\begin{proof}
Since $j : R \to U \times_S U$ is a monomorphism we see that $j$ is separated
(see Schemes, Lemma \ref{schemes-lemma-monomorphism-separated}).
Since $U$ is affine we see that $U \times_S U$
(which comes equipped with a monomorphism into the affine scheme
$U \times U$) is separated. Hence we see that $R$ is separated.
In particular the morphisms $s, t$ are separated as well as \'etale.

\medskip\noindent
Since the composition $R \to U \times_S U \to U$ is
locally of finite type we conclude that
$j$ is locally of finite type (see
Morphisms, Lemma \ref{morphisms-lemma-permanence-finite-type}).
As $j$ is also a monomorphism it has finite fibres and
we see that $j$ is locally quasi-finite by
Morphisms, Lemma \ref{morphisms-lemma-finite-fibre}.
Altogether we see that $j$ is separated and locally quasi-finite.

\medskip\noindent
Our first step is to show that the quotient map
$c : U \to F$ is representable.
Consider a scheme $T$ and a morphism $a : T \to F$.
We have to show that the sheaf $G = T \times_{a, F, c} U$
is representable.
As seen in the proofs of Lemmas \ref{lemma-finding-opens} and
\ref{lemma-when-it-works-it-works} there exists an fppf covering
$\{\varphi_i : T_i \to T\}_{i \in I}$ and morphisms $a_i : T_i \to U$
such that $a_i \times a_{i'} : T_i \times_T T_{i'} \to U \times_S U$
factors through $R$, and such that $c \circ a_i = a \circ \varphi_i$.
As in the proof of Lemma \ref{lemma-when-it-works-it-works} we see that
\begin{eqnarray*}
T_i \times_{\varphi_i, T} G & = &
T_i \times_{\varphi_i, T} T \times_{a, U/R, c} U \\
& = & T_i \times_{c \circ a_i, U/R, c} U \\
& = & T_i \times_{a_i, U} U \times_{c, U/R, c} U \\
& = & T_i \times_{a_i, U, t} R
\end{eqnarray*}
Since $t$ is separated and \'etale, and in particular
separated and locally quasi-finite (by Morphisms, Lemmas
\ref{morphisms-lemma-unramified-quasi-finite} and
\ref{morphisms-lemma-flat-unramified-etale})
we see that the restriction
of $G$ to each $T_i$ is representable by a morphism of schemes
$X_i \to T_i$ which is separated and locally quasi-finite. By
Descent, Lemma \ref{descent-lemma-descent-data-sheaves}
we obtain a descent datum $(X_i, \varphi_{ii'})$ relative
to the fppf-covering $\{T_i \to T\}$. Since each
$X_i \to T_i$ is separated and locally quasi-finite we see by
More on Morphisms, Lemma
\ref{more-morphisms-lemma-separated-locally-quasi-finite-morphisms-fppf-descend}
that this descent datum is effective.
Hence by
Descent, Lemma \ref{descent-lemma-descent-data-sheaves} (2)
we conclude that $G$ is representable as desired.

\medskip\noindent
The second step of the proof is to show that $U \to F$ is surjective and
\'etale. This is clear from the above since in the first step above we
saw that $G = T \times_{a, F, c} U$ is a scheme over $T$ which base changes
to schemes $X_i \to T_i$ which are surjective and \'etale. Thus $G \to T$
is surjective and \'etale (see
Remark \ref{remark-list-properties-fpqc-local-base}).
Alternatively one can reread the proof of
Lemma \ref{lemma-when-it-works-it-works} in the current
situation.

\medskip\noindent
The third and final step is to show that the diagonal map $F \to F \times F$
is representable. We first observe that the diagram
$$
\xymatrix{
R \ar[r] \ar[d]_j & F \ar[d]^\Delta \\
U \times_S U \ar[r] & F \times F
}
$$
is a fibre product square. By
Lemma \ref{lemma-product-representable-transformations} the morphism
$U \times_S U \to F \times F$ is representable (note that
$h_U \times h_U = h_{U \times_S U}$). Moreover, by
Lemma \ref{lemma-product-representable-transformations-property}
the morphism $U \times_S U \to F \times F$ is surjective
and \'etale (note also that \'etale and surjective occur in the lists of
Remarks \ref{remark-list-properties-fpqc-local-base}
and \ref{remark-list-properties-stable-composition}).
It follows either from
Lemma \ref{lemma-base-change-representable-transformations}
and the diagram above, or by writing $R \to F$ as $R \to U \to F$ and
Lemmas
\ref{lemma-morphism-schemes-gives-representable-transformation} and
\ref{lemma-composition-representable-transformations} that
$R \to F$ is representable as well. Let $T$ be a scheme and let
$a : T \to F \times F$ be a morphism. We have to show that
$G = T \times_{a, F \times F, \Delta} F$ is representable.
By what was said above the morphism (of schemes)
$$
T' = (U \times_S U) \times_{F \times F, a} T \longrightarrow T
$$
is surjective and \'etale. Hence $\{T' \to T\}$ is an \'etale
covering of $T$. Note also that
$$
T' \times_T G = T' \times_{U \times_S U, j} R
$$
as can be seen contemplating the following cube
$$
\xymatrix{
& R \ar[rr] \ar[dd] & & F \ar[dd] \\
T' \times_T G \ar[rr] \ar[dd] \ar[ru] & & G \ar[dd] \ar[ru] & \\
& U \times_S U \ar'[r][rr] & & F \times F \\
T' \ar[rr] \ar[ru] & & T \ar[ru]
}
$$
Hence we see that the restriction of $G$ to $T'$ is representable
by a scheme $X$, and moreover that the morphism $X \to T'$ is
a base change of the morphism $j$. Hence $X \to T'$ is
separated and locally quasi-finite (see second paragraph of the proof).
By Descent, Lemma \ref{descent-lemma-descent-data-sheaves}
we obtain a descent datum $(X, \varphi)$ relative
to the fppf-covering $\{T' \to T\}$. Since
$X \to T'$ is separated and locally quasi-finite we see by
More on Morphisms, Lemma
\ref{more-morphisms-lemma-separated-locally-quasi-finite-morphisms-fppf-descend}
that this descent datum is effective.
Hence by
Descent, Lemma \ref{descent-lemma-descent-data-sheaves} (2)
we conclude that $G$ is representable as desired.
\end{proof}

\begin{theorem}
\label{theorem-presentation}
Let $S$ be a scheme. Let $U$ be a scheme over $S$.
Let $j = (s, t) : R \to U \times_S U$
be an \'etale equivalence relation on $U$ over $S$.
Then the quotient $U/R$ is an algebraic space,
and $U \to U/R$ is \'etale and surjective, in other words
$(U, R, U \to U/R)$ is a presentation of $U/R$.
\end{theorem}

\begin{proof}
By Lemma \ref{lemma-when-it-works-it-works}
it suffices to prove that $U/R$ is an algebraic space.
Let $U' \to U$ be a surjective, \'etale morphism.
Then $\{U' \to U\}$ is in particular an fppf covering.
Let $R'$ be the restriction of $R$ to $U'$, see
Groupoids, Definition \ref{groupoids-definition-restrict-relation}.
According to
Groupoids, Lemma \ref{groupoids-lemma-quotient-groupoid-restrict}
we see that $U/R \cong U'/R'$.
By Lemma \ref{lemma-pullback-etale-equivalence-relation} $R'$ is an
\'etale equivalence relation on $U'$. Thus we may replace $U$ by $U'$.

\medskip\noindent
We apply the previous remark to $U' = \coprod U_i$, where
$U = \bigcup U_i$ is an affine open covering of $U$. Hence we
may and do assume that $U = \coprod U_i$ where
each $U_i$ is an affine scheme.

\medskip\noindent
Consider the restriction $R_i$ of $R$ to $U_i$.
By Lemma \ref{lemma-pullback-etale-equivalence-relation}
this is an \'etale equivalence relation.
Set $F_i = U_i/R_i$ and $F = U/R$.
It is clear that $\coprod F_i \to F$ is surjective.
By Lemma \ref{lemma-finding-opens} each $F_i \to F$
is representable, and an open immersion.
By Lemma \ref{lemma-presentation-quasi-compact}
applied to $(U_i, R_i)$ we see that $F_i$ is an algebraic space.
Then by Lemma \ref{lemma-when-it-works-it-works} we see that
$U_i \to F_i$ is \'etale and surjective.
From Lemma \ref{lemma-coproduct-algebraic-spaces}
it follows that $\coprod F_i$ is an algebraic space.
Finally, we have verified all
hypotheses of Lemma \ref{lemma-glueing-algebraic-spaces}
and it follows that $F = U/R$ is an algebraic space.
\end{proof}










\section{Algebraic spaces, retrofitted}
\label{section-algebraic-spaces-retrofitted}

\noindent
We start building our arsenal of lemmas dealing with algebraic spaces.
The first result says that in Definition \ref{definition-algebraic-space}
we can weaken the condition on the diagonal as follows.

\begin{lemma}
\label{lemma-etale-locally-representable-gives-space}
Let $S$ be a scheme contained in $\Sch_{fppf}$.
Let $F$ be a sheaf on $(\Sch/S)_{fppf}$
such that there exists $U \in \Ob((\Sch/S)_{fppf})$ and a map
$U \to F$ which is representable, surjective, and \'etale.
Then $F$ is an algebraic space.
\end{lemma}

\begin{proof}
Set $R = U \times_F U$. This is a scheme as $U \to F$ is assumed representable.
The projections $s, t : R \to U$ are \'etale as $U \to F$ is assumed \'etale.
The map $j = (t, s) : R \to U \times_S U$ is a monomorphism and an equivalence
relation as $R = U \times_F U$. By Theorem \ref{theorem-presentation}
the quotient sheaf $F' = U/R$ is an algebraic space and $U \to F'$
is surjective and \'etale. Again since $R = U \times_F U$ we obtain
a canonical factorization $U \to F' \to F$ and $F' \to F$ is an injective
map of sheaves. On the other hand, $U \to F$ is surjective as a map
of sheaves by Lemma \ref{lemma-surjective-flat-locally-finite-presentation}.
Thus $F' \to F$ is also surjective and we conclude $F' = F$ is an
algebraic space.
\end{proof}

\begin{lemma}
\label{lemma-etale-locally-representable-by-space-gives-space}
Let $S$ be a scheme contained in $\Sch_{fppf}$. Let $G$ be an algebraic
space over $S$, let $F$ be a sheaf on $(\Sch/S)_{fppf}$, and let
$G \to F$ be a representable transformation of functors which is
surjective and \'etale. Then $F$ is an algebraic space.
\end{lemma}

\begin{proof}
Pick a scheme $U$ and a surjective \'etale morphism $U \to G$.
Since $G$ is an algebraic space $U \to G$ is representable.
Hence the composition $U \to G \to F$ is representable,
surjective, and \'etale. See Lemmas
\ref{lemma-composition-representable-transformations} and
\ref{lemma-composition-representable-transformations-property}.
Thus $F$ is an algebraic space by
Lemma \ref{lemma-etale-locally-representable-gives-space}.
\end{proof}

\begin{lemma}
\label{lemma-representable-over-space}
\begin{slogan}
A functor that admits a representable morphism to an algebraic space is
an algebraic space.
\end{slogan}
Let $S$ be a scheme contained in $\Sch_{fppf}$.
Let $F$ be an algebraic space over $S$.
Let $G \to F$ be a representable transformation of functors.
Then $G$ is an algebraic space.
\end{lemma}

\begin{proof}
By Lemma \ref{lemma-representable-transformation-to-sheaf}
we see that $G$ is a sheaf. The diagram
$$
\xymatrix{
G \times_F G \ar[r] \ar[d] & F \ar[d]^{\Delta_F} \\
G \times G \ar[r] & F \times F
}
$$
is cartesian. Hence we see that $G \times_F G \to G \times G$
is representable by
Lemma \ref{lemma-base-change-representable-transformations}.
By
Lemma \ref{lemma-representable-transformation-diagonal}
we see that $G \to G \times_F G$ is representable.
Hence $\Delta_G : G \to G \times G$ is representable as a composition
of representable transformations, see
Lemma \ref{lemma-composition-representable-transformations}.
Finally, let $U$ be an object of $(\Sch/S)_{fppf}$
and let $U \to F$ be surjective and \'etale. By assumption
$U \times_F G$ is representable by a scheme $U'$. By
Lemma \ref{lemma-base-change-representable-transformations-property}
the morphism $U' \to G$ is surjective and \'etale. This verifies
the final condition of Definition \ref{definition-algebraic-space} and we win.
\end{proof}

\begin{lemma}
\label{lemma-representable-morphisms-spaces-property}
Let $S$ be a scheme contained in $\Sch_{fppf}$.
Let $F$, $G$ be algebraic spaces over $S$.
Let $G \to F$ be a representable morphism.
Let $U \in \Ob((\Sch/S)_{fppf})$, and $q : U \to F$
surjective and \'etale. Set $V = G \times_F U$.
Finally, let $\mathcal{P}$ be a property of morphisms
of schemes as in Definition \ref{definition-relative-representable-property}.
Then $G \to F$ has property $\mathcal{P}$ if and only if
$V \to U$ has property $\mathcal{P}$.
\end{lemma}

\begin{proof}
(This lemma follows from
Lemmas \ref{lemma-base-change-representable-transformations-property} and
\ref{lemma-descent-representable-transformations-property},
but we give a direct proof here also.)
It is clear from the definitions that if $G \to F$ has property
$\mathcal{P}$, then $V \to U$ has property $\mathcal{P}$.
Conversely, assume $V \to U$ has property $\mathcal{P}$.
Let $T \to F$ be a morphism from a scheme to $F$.
Let $T' = T \times_F G$ which is a scheme since $G \to F$ is
representable. We have to show that $T' \to T$ has property $\mathcal{P}$.
Consider the commutative diagram of schemes
$$
\xymatrix{
V \ar[d] & T \times_F V \ar[d] \ar[l] \ar[r] &
T \times_F G \ar[d] \ar@{=}[r] & T' \\
U & T \times_F U \ar[l] \ar[r] & T
}
$$
where both squares are fibre product squares. Hence we conclude
the middle arrow has property $\mathcal{P}$ as a base change
of $V \to U$. Finally, $\{T \times_F U \to T\}$ is a fppf covering
as it is surjective \'etale, and hence we conclude that
$T' \to T$ has property $\mathcal{P}$ as it is local on the
base in the fppf topology.
\end{proof}

\begin{lemma}
\label{lemma-morphism-sheaves-with-P-effective-descent-etale}
Let $S$ be a scheme contained in $\Sch_{fppf}$.
Let $G \to F$ be a transformation of presheaves on $(\Sch/S)_{fppf}$.
Let $\mathcal{P}$ be a property of morphisms of schemes.
Assume
\begin{enumerate}
\item $\mathcal{P}$ is preserved under any base change, fppf local on the
base, and morphisms of type $\mathcal{P}$ satisfy descent for fppf coverings,
see Descent, Definition \ref{descent-definition-descending-types-morphisms},
\item $G$ is a sheaf,
\item $F$ is an algebraic space,
\item there exists a $U \in \Ob((\Sch/S)_{fppf})$
and a surjective \'etale morphism $U \to F$ such that
$V = G \times_F U$ is representable, and
\item $V \to U$ has $\mathcal{P}$.
\end{enumerate}
Then $G$ is an algebraic space, $G \to F$ is representable and has property
$\mathcal{P}$.
\end{lemma}

\begin{proof}
Let $T$ be a scheme and let $T \to F$ be a morphism. Then
$U \times_F T \to T$ is surjective \'etale, hence $\{U \times_F T \to T\}$ is
a covering for the \'etale topology. Consider
$$
W = G \times_F (U \times_F T) = V \times_F T = V \times_U (U \times_F T).
$$
It is a scheme since $F$ is an algebraic space. The morphism
$W \to U \times_F T$ has property $\mathcal{P}$ since it is a
base change of $V \to U$. There is an isomorphism
\begin{align*}
W \times_T (U \times_F T) & =
(G \times_F (U \times_F T)) \times_T (U \times_F T) \\
& = (U \times_F T) \times_T (G \times_F (U \times_F T)) \\
& = (U \times_F T) \times_T W
\end{align*}
over $(U \times_F T) \times_T (U \times_F T)$. The middle equality maps
$((g, (u_1, t)), (u_2, t))$ to $((u_1, t), (g, (u_2, t)))$.
This defines a descent datum for $W/U \times_F T/T$, see
Descent, Definition \ref{descent-definition-descent-datum}.
This follows from
Descent, Lemma \ref{descent-lemma-descent-data-sheaves}.
Namely we have a sheaf $G \times_F T$, whose
base change to $U \times_F T$ is represented by $W$ and the isomorphism
above is the one from the proof of
Descent, Lemma \ref{descent-lemma-descent-data-sheaves}.
By assumption on $\mathcal{P}$ the descent datum above is representable.
Hence by the last statement of
Descent, Lemma \ref{descent-lemma-descent-data-sheaves}
we see that $G \times_F T$ is representable. This proves that
$G \to F$ is a representable transformation of functors.

\medskip\noindent
As $G \to F$ is representable, we see that $G$ is an algebraic space by
Lemma \ref{lemma-representable-over-space}. The fact that $G \to F$ has
property $\mathcal{P}$ now follows from
Lemma \ref{lemma-representable-morphisms-spaces-property}.
\end{proof}

\begin{lemma}
\label{lemma-lift-morphism-presentations}
Let $S$ be a scheme contained in $\Sch_{fppf}$.
Let $F, G$ be algebraic spaces over $S$.
Let $a : F \to G$ be a morphism.
Given any $V \in \Ob((\Sch/S)_{fppf})$
and a surjective \'etale morphism $q : V \to G$ there exists
a $U \in \Ob((\Sch/S)_{fppf})$
and a commutative diagram
$$
\xymatrix{
U \ar[d]_p \ar[r]_\alpha &
V \ar[d]^q \\
F \ar[r]^a & G
}
$$
with $p$ surjective and \'etale.
\end{lemma}

\begin{proof}
First choose $W \in \Ob((\Sch/S)_{fppf})$
with surjective \'etale morphism $W \to F$.
Next, put $U = W \times_G V$. Since $G$ is an algebraic space
we see that $U$ is isomorphic to an object of $(\Sch/S)_{fppf}$.
As $q$ is surjective \'etale, we see that $U \to W$ is surjective
\'etale (see
Lemma \ref{lemma-base-change-representable-transformations-property}).
Thus $U \to F$ is surjective \'etale as a composition of surjective
\'etale morphisms (see
Lemma \ref{lemma-composition-representable-transformations-property}).
\end{proof}



























\section{Immersions and Zariski coverings of algebraic spaces}
\label{section-Zariski}

\noindent
At this point an interesting phenomenon occurs. We have already defined
the notion of an open immersion of algebraic spaces (through
Definition \ref{definition-relative-representable-property})
but we have yet to define the notion of a {\it point}\footnote{We
will associate a topological space to an algebraic space in
Properties of Spaces, Section
\ref{spaces-properties-section-points},
and its opens will correspond exactly to the open subspaces defined below.}.
Thus the {\it Zariski topology} of an algebraic space
has already been defined, but there is no space yet!

\medskip\noindent
Perhaps superfluously we formally introduce immersions as follows.

\begin{definition}
\label{definition-immersion}
Let $S \in \Ob(\Sch_{fppf})$ be a scheme.
Let $F$ be an algebraic space over $S$.
\begin{enumerate}
\item A morphism of algebraic spaces over $S$
is called an {\it open immersion} if it is representable, and an open immersion
in the sense of Definition \ref{definition-relative-representable-property}.
\item An {\it open subspace} of $F$ is a subfunctor $F' \subset F$
such that $F'$ is an algebraic space and $F' \to F$ is an
open immersion.
\item A morphism of algebraic spaces over $S$
is called a {\it closed immersion} if it is representable, and a closed
immersion in the sense of
Definition \ref{definition-relative-representable-property}.
\item A {\it closed subspace} of $F$ is a subfunctor $F' \subset F$
such that $F'$ is an algebraic space and $F' \to F$ is a
closed immersion.
\item A morphism of algebraic spaces over $S$
is called an {\it immersion} if it is representable, and an immersion
in the sense of Definition \ref{definition-relative-representable-property}.
\item A {\it locally closed subspace} of $F$ is a subfunctor $F' \subset F$
such that $F'$ is an algebraic space and $F' \to F$ is an
immersion.
\end{enumerate}
\end{definition}

\noindent
We note that these definitions make sense since an immersion
is in particular a monomorphism (see
Schemes, Lemma \ref{schemes-lemma-immersions-monomorphisms}
and Lemma \ref{lemma-representable-transformations-property-implication}),
and hence the image of an
immersion $G \to F$ of algebraic spaces is a subfunctor $F' \subset F$
which is (canonically) isomorphic to $G$. Thus some of the discussion
of Schemes, Section \ref{schemes-section-immersions} carries over to the
setting of algebraic spaces.

\begin{lemma}
\label{lemma-composition-immersions}
Let $S \in \Ob(\Sch_{fppf})$ be a scheme.
A composition of (closed, resp.\ open) immersions of
algebraic spaces over $S$ is a (closed, resp.\ open)
immersion of algebraic spaces over $S$.
\end{lemma}

\begin{proof}
See Lemma \ref{lemma-composition-representable-transformations-property} and
Remarks \ref{remark-list-properties-fpqc-local-base} (see very last line of
that remark) and \ref{remark-list-properties-stable-composition}.
\end{proof}

\begin{lemma}
\label{lemma-base-change-immersions}
Let $S \in \Ob(\Sch_{fppf})$ be a scheme.
A base change of a (closed, resp.\ open) immersion
of algebraic spaces over $S$ is a (closed, resp.\ open)
immersion of algebraic spaces over $S$.
\end{lemma}

\begin{proof}
See Lemma \ref{lemma-base-change-representable-transformations-property} and
Remark \ref{remark-list-properties-fpqc-local-base} (see very last line of
that remark).
\end{proof}

\begin{lemma}
\label{lemma-sub-subspaces}
Let $S \in \Ob(\Sch_{fppf})$ be a scheme.
Let $F$ be an algebraic space over $S$. Let $F_1$, $F_2$ be
locally closed subspaces of $F$. If $F_1 \subset F_2$ as subfunctors
of $F$, then $F_1$ is a locally closed subspace of $F_2$.
Similarly for closed and open subspaces.
\end{lemma}

\begin{proof}
Let $T \to F_2$ be a morphism with $T$ a scheme.
Since $F_2 \to F$ is a monomorphism, we see that
$T \times_{F_2} F_1 = T \times_F F_1$. The lemma follows
formally from this.
\end{proof}

\noindent
Let us formally define the notion of a Zariski open covering of
algebraic spaces. Note that in Lemma \ref{lemma-glueing-algebraic-spaces}
we have already encountered such open coverings as a method for
constructing algebraic spaces.

\begin{definition}
\label{definition-Zariski-open-covering}
Let $S \in \Ob(\Sch_{fppf})$ be a scheme.
Let $F$ be an algebraic space over $S$.
A {\it Zariski covering} $\{F_i \subset F\}_{i \in I}$ of $F$
is given by a set $I$ and a collection of open subspaces
$F_i \subset F$ such that $\coprod F_i \to F$ is a surjective
map of sheaves.
\end{definition}

\noindent
Note that if $T$ is a schemes,
and $a : T \to F$ is a morphism, then each of the fibre products
$T \times_F F_i$ is identified with an open subscheme
$T_i \subset T$. The final condition of the definition signifies
exactly that $T = \bigcup_{i \in I} T_i$.

\medskip\noindent
It is clear that the collection $F_{Zar}$ of open subspaces of
$F$ is a set (as $(\Sch/S)_{fppf}$ is a site, hence a set).
Moreover, we can turn $F_{Zar}$ into a category by letting the
morphisms be inclusions of subfunctors (which are automatically open
immersions by Lemma \ref{lemma-sub-subspaces}). Finally,
Definition \ref{definition-Zariski-open-covering}
provides the notion of a Zariski covering $\{F_i \to F'\}_{i \in I}$
in the category $F_{Zar}$. Hence, just as in the case of a topological
space (see Sites, Example \ref{sites-example-site-topological})
by suitably choosing a set of coverings
we may obtain a Zariski site of the algebraic space $F$.

\begin{definition}
\label{definition-small-Zariski-site}
Let $S \in \Ob(\Sch_{fppf})$ be a scheme. Let $F$ be an algebraic space over
$S$. A {\it small Zariski site $F_{Zar}$} of an algebraic space $F$ is one
of the sites described above.
\end{definition}

\noindent
Hence this gives a notion of what it means for something to be true
Zariski locally on an algebraic space, which is how we will use this
notion. In general the Zariski topology is not fine enough for our
purposes. For example we can consider the category of Zariski sheaves
on an algebraic space. It will turn out that this is not the
correct thing to consider, even for quasi-coherent sheaves.
One only gets the desired result when using the \'etale or fppf site of
$F$ to define quasi-coherent sheaves.











\section{Separation conditions on algebraic spaces}
\label{section-separation}

\noindent
A separation condition on an algebraic space $F$ is a condition
on the diagonal morphism $F \to F \times F$. Let us first
list the properties the diagonal has automatically.
Since the diagonal is representable by definition the following lemma
makes sense (through
Definition \ref{definition-relative-representable-property}).

\begin{lemma}
\label{lemma-properties-diagonal}
Let $S$ be a scheme contained in $\Sch_{fppf}$.
Let $F$ be an algebraic space over $S$.
Let $\Delta : F \to F \times F$ be the diagonal morphism.
Then
\begin{enumerate}
\item $\Delta$ is locally of finite type,
\item $\Delta$ is a monomorphism,
\item $\Delta$ is separated, and
\item $\Delta$ is locally quasi-finite.
\end{enumerate}
\end{lemma}

\begin{proof}
Let $F = U/R$ be a presentation of $F$.
As in the proof of Lemma \ref{lemma-presentation-quasi-compact} the diagram
$$
\xymatrix{
R \ar[r] \ar[d]_j & F \ar[d]^\Delta \\
U \times_S U \ar[r] & F \times F
}
$$
is cartesian. Hence according to
Lemma \ref{lemma-representable-morphisms-spaces-property}
it suffices to show that $j$ has the properties listed in the lemma.
(Note that each of the properties (1) -- (4) occur in the lists
of Remarks \ref{remark-list-properties-stable-base-change}
and \ref{remark-list-properties-fpqc-local-base}.)
Since $j$ is an equivalence relation it is a monomorphism.
Hence it is separated by
Schemes, Lemma \ref{schemes-lemma-monomorphism-separated}.
As $R$ is an \'etale equivalence relation we see that
$s, t : R \to U$ are \'etale. Hence $s, t$ are locally of finite
type. Then it follows from
Morphisms, Lemma \ref{morphisms-lemma-permanence-finite-type} that
$j$ is locally of finite type. Finally, as it is a monomorphism
its fibres are finite. Thus we conclude that it is locally quasi-finite by
Morphisms, Lemma \ref{morphisms-lemma-finite-fibre}.
\end{proof}

\noindent
Here are some common types of separation conditions, relative to the base
scheme $S$. There is also an absolute notion of these conditions which we
will discuss in
Properties of Spaces, Section \ref{spaces-properties-section-separation}.
Moreover, we will discuss separation conditions for a morphism of
algebraic spaces in
Morphisms of Spaces, Section \ref{spaces-morphisms-section-separation-axioms}.

\begin{definition}
\label{definition-separated}
Let $S$ be a scheme contained in $\Sch_{fppf}$.
Let $F$ be an algebraic space over $S$.
Let $\Delta : F \to F \times F$ be the diagonal morphism.
\begin{enumerate}
\item We say $F$ is {\it separated over $S$} if $\Delta$ is a closed immersion.
\item We say $F$ is {\it locally separated over $S$}\footnote{In the
literature this often refers to quasi-separated and
locally separated algebraic spaces.} if $\Delta$ is an
immersion.
\item We say $F$ is {\it quasi-separated over $S$} if $\Delta$ is quasi-compact.
\item We say $F$ is {\it Zariski locally quasi-separated over $S$}\footnote{This
definition was suggested by B.\ Conrad.} if there
exists a Zariski covering $F = \bigcup_{i \in I} F_i$ such that
each $F_i$ is quasi-separated.
\end{enumerate}
\end{definition}

\noindent
Note that if the diagonal is quasi-compact (when $F$ is separated or
quasi-separated) then the diagonal is actually
quasi-finite and separated, hence quasi-affine (by More on Morphisms,
Lemma \ref{more-morphisms-lemma-quasi-finite-separated-quasi-affine}).








\section{Examples of algebraic spaces}
\label{section-examples}

\noindent
In this section we construct some examples of algebraic spaces.
Some of these were suggested by B.\ Conrad.
Since we do not yet have a lot of theory at our disposal the
discussion is a bit awkward in some places.

\begin{example}
\label{example-affine-line-involution}
Let $k$ be a field of characteristic $\not = 2$. Let $U = \mathbf{A}^1_k$. Set
$$
j : R = \Delta \amalg \Gamma \longrightarrow U \times_k U
$$
where $\Delta = \{(x, x) \mid x \in \mathbf{A}^1_k\}$ and
$\Gamma = \{(x, -x) \mid x \in \mathbf{A}^1_k, x \not = 0\}$.
It is clear that $s, t : R \to U$ are \'etale, and hence
$j$ is an \'etale equivalence relation.
The quotient $X = U/R$ is an algebraic space by
Theorem \ref{theorem-presentation}.
Since $R$ is quasi-compact we see that $X$ is quasi-separated.
On the other hand, $X$ is not locally separated because
the morphism $j$ is not an immersion.
\end{example}

\begin{example}
\label{example-non-representable-descent}
Let $k$ be a field. Let $k'/k$ be a degree $2$ Galois extension
with $\text{Gal}(k'/k) = \{1, \sigma\}$. Let $S = \Spec(k[x])$
and $U = \Spec(k'[x])$. Note that
$$
U \times_S U =
\Spec((k' \otimes_k k')[x]) =
\Delta(U) \amalg \Delta'(U)
$$
where $\Delta' = (1, \sigma) : U \to U \times_S U$. Take
$$
R = \Delta(U) \amalg \Delta'(U \setminus \{0_U\})
$$
where $0_U \in U$ denotes the $k'$-rational point whose $x$-coordinate is zero.
It is easy to see that $R$ is an \'etale equivalence relation on $U$ over $S$
and hence $X = U/R$ is an algebraic space by
Theorem \ref{theorem-presentation}. Here are some properties of $X$ (some
of which will not make sense until later):
\begin{enumerate}
\item $X \to S$ is an isomorphism over $S \setminus \{0_S\}$,
\item the morphism $X \to S$ is \'etale (see
Properties of Spaces,
Definition \ref{spaces-properties-definition-etale})
\item the fibre $0_X$ of $X \to S$ over $0_S$ is isomorphic to
$\Spec(k') = 0_U$,
\item $X$ is not a scheme because if it were, then $\mathcal{O}_{X, 0_X}$
would be a local domain $(\mathcal{O}, \mathfrak m, \kappa)$ with
fraction field $k(x)$, with $x \in \mathfrak m$ and residue field
$\kappa = k'$ which is impossible,
\item $X$ is not separated, but it is
locally separated and quasi-separated,
\item there exists a surjective, finite, \'etale morphism $S' \to S$
such that the base change $X' = S' \times_S X$ is a scheme (namely, if
we base change to $S' = \Spec(k'[x])$ then $U$ splits into
two copies of $S'$ and $X'$ becomes isomorphic to the affine line with
$0$ doubled, see
Schemes, Example \ref{schemes-example-affine-space-zero-doubled}), and
\item if we think of $X$ as a finite type algebraic space over
$\Spec(k)$, then similarly the base change $X_{k'}$ is a scheme
but $X$ is not a scheme.
\end{enumerate}
In particular, this gives an example of a descent datum for schemes
relative to the covering $\{\Spec(k') \to \Spec(k)\}$
which is not effective.
\end{example}

\noindent
See also Examples, 
Lemma \ref{examples-lemma-non-effective-descent-projective},
which shows that descent data need not be effective
even for a projective morphism of schemes. That example
gives a smooth separated algebraic space of dimension 3
over ${\mathbf C}$ which is not a scheme.

\medskip\noindent
We will use the following lemma as a convenient way to construct
algebraic spaces as quotients of schemes by free group actions.

\begin{lemma}
\label{lemma-quotient}
Let $U \to S$ be a morphism of $\Sch_{fppf}$.
Let $G$ be an abstract group. Let $G \to \text{Aut}_S(U)$
be a group homomorphism. Assume
\begin{itemize}
\item[(*)] if $u \in U$ is a point, and $g(u) = u$
for some non-identity element $g \in G$, then $g$
induces a nontrivial automorphism of $\kappa(u)$.
\end{itemize}
Then
$$
j :
R = \coprod\nolimits_{g \in G} U
\longrightarrow
U \times_S U,
\quad
(g, x) \longmapsto (g(x), x)
$$
is an \'etale equivalence relation and hence
$$
F = U/R
$$
is an algebraic space by Theorem \ref{theorem-presentation}.
\end{lemma}

\begin{proof}
In the statement of the lemma the symbol $\text{Aut}_S(U)$ denotes
the group of automorphisms of $U$ over $S$.
Assume $(*)$ holds. Let us show that
$$
j :
R = \coprod\nolimits_{g \in G} U
\longrightarrow
U \times_S U,
\quad
(g, x) \longmapsto (g(x), x)
$$
is a monomorphism. This signifies that if $T$ is a nonempty
scheme, and $h : T \to U$ is a $T$-valued point such that
$g \circ h = g' \circ h$ then $g = g'$. Suppose
$T \not = \emptyset$, $h : T \to U$ and $g \circ h = g' \circ h$.
Let $t \in T$. Consider the composition
$\Spec(\kappa(t)) \to \Spec(\kappa(h(t))) \to U$.
Then we conclude that $g^{-1} \circ g'$ fixes $u = h(t)$ and
acts as the identity on its residue field. Hence $g = g'$ by $(*)$.

\medskip\noindent
Thus if $(*)$ holds we see that $j$ is a relation (see
Groupoids, Definition \ref{groupoids-definition-equivalence-relation}).
Moreover, it is an equivalence relation since on $T$-valued points
for a connected scheme $T$ we see that
$R(T) = G \times U(T) \to U(T) \times U(T)$ (recall that we always
work over $S$). Moreover, the morphisms $s, t : R \to U$ are \'etale
since $R$ is a disjoint product of copies of $U$.
This proves that $j : R \to U \times_S U$ is an \'etale equivalence relation.
\end{proof}

\noindent
Given a scheme $U$ and an action of a group $G$ on $U$ we say the action
of $G$ on $U$ is {\it free} if condition $(*)$ of Lemma \ref{lemma-quotient}
holds. This is equivalent to the notion of a free action of the constant
group scheme $G_S$ on $U$ as defined in
Groupoids, Definition \ref{groupoids-definition-free-action}.
The lemma can be interpreted as saying that quotients of schemes by
free actions of groups exist in the category of algebraic spaces.

\begin{definition}
\label{definition-quotient}
Notation $U \to S$, $G$, $R$ as in Lemma \ref{lemma-quotient}.
If the action of $G$ on $U$ satisfies $(*)$ we say $G$ {\it acts freely}
on the scheme $U$. In this case the algebraic space $U/R$ is denoted
$U/G$ and is called the {\it quotient of $U$ by $G$}.
\end{definition}

\noindent
This notation is consistent with the notation $U/G$ introduced in
Groupoids, Definition \ref{groupoids-definition-quotient-sheaf}.
We will later make sense of the quotient as an algebraic stack without
any assumptions on the action whatsoever; when we do this we will use the
notation $[U/G]$. Before we discuss the examples we prove
some more lemmas to facilitate the discussion. Here is a lemma discussing the
various separation conditions for this quotient when $G$ is finite.

\begin{lemma}
\label{lemma-quotient-finite-separated}
Notation and assumptions as in Lemma \ref{lemma-quotient}.
Assume $G$ is finite. Then
\begin{enumerate}
\item if $U \to S$ is quasi-separated, then $U/G$ is quasi-separated
over $S$, and
\item if $U \to S$ is separated, then $U/G$ is separated over $S$.
\end{enumerate}
\end{lemma}

\begin{proof}
In the proof of Lemma \ref{lemma-properties-diagonal}
we saw that it suffices to prove the
corresponding properties for the morphism $j : R \to U \times_S U$.
If $U \to S$ is quasi-separated, then for every affine open $V \subset U$
which maps into an affine of $S$
the opens $g(V) \cap V$ are quasi-compact. It follows that $j$ is
quasi-compact.
If $U \to S$ is separated, the diagonal $\Delta_{U/S}$ is a closed
immersion. Hence $j : R \to U \times_S U$ is a finite coproduct
of closed immersions with disjoint images. Hence $j$ is a closed immersion.
\end{proof}

\begin{lemma}
\label{lemma-quotient-field-map}
Notation and assumptions as in Lemma \ref{lemma-quotient}.
If $\Spec(k) \to U/G$ is a morphism, then there exist
\begin{enumerate}
\item a finite Galois extension $k'/k$,
\item a finite subgroup $H \subset G$,
\item an isomorphism $H \to \text{Gal}(k'/k)$, and
\item an $H$-equivariant morphism $\Spec(k') \to U$.
\end{enumerate}
Conversely, such data determine a morphism $\Spec(k) \to U/G$.
\end{lemma}

\begin{proof}
Consider the fibre product $V = \Spec(k) \times_{U/G} U$.
Here is a diagram
$$
\xymatrix{
V \ar[r] \ar[d] & U \ar[d] \\
\Spec(k) \ar[r] & U/G
}
$$
Then $V$ is a nonempty scheme \'etale over $\Spec(k)$ and hence is a
disjoint union $V = \coprod_{i \in I} \Spec(k_i)$
of spectra of fields $k_i$ finite separable over $k$
(Morphisms, Lemma \ref{morphisms-lemma-etale-over-field}).
We have
\begin{align*}
V \times_{\Spec(k)} V
& =
(\Spec(k) \times_{U/G} U) \times_{\Spec(k)}(\Spec(k) \times_{U/G} U) \\
& = 
\Spec(k) \times_{U/G} U \times_{U/G} U \\
& =
\Spec(k) \times_{U/G} U \times G \\
& =
V \times G
\end{align*}
The action of $G$ on $U$ induces an action of $a : G \times V \to V$.
The displayed equality means that
$G \times V \to V \times_{\Spec(k)} V$, $(g, v) \mapsto (a(g, v), v)$
is an isomorphism. In particular we see that for every $i$ we have
an isomorphism $H_i \times \Spec(k_i) \to \Spec(k_i \otimes_k k_i)$
where $H_i \subset G$ is the subgroup of elements fixing $i \in I$.
Thus $H_i$ is finite and is the Galois group of $k_i/k$.
We omit the converse construction.
\end{proof}

\noindent
It follows from this lemma for example that
if $k'/k$ is a finite Galois extension, then
$\Spec(k')/\text{Gal}(k'/k) \cong \Spec(k)$.
What happens if the extension is infinite? Here is an example.

\begin{example}
\label{example-Qbar}
Let $S = \Spec(\mathbf{Q})$.
Let $U = \Spec(\overline{\mathbf{Q}})$.
Let $G = \text{Gal}(\overline{\mathbf{Q}}/\mathbf{Q})$ with obvious
action on $U$. Then by construction property $(*)$ of
Lemma \ref{lemma-quotient} holds and we obtain an algebraic space
$$
X = \Spec(\overline{\mathbf{Q}})/G
\longrightarrow
S = \Spec(\mathbf{Q}).
$$
Of course this is totally ridiculous as an approximation of $S$!
Namely, by the Artin-Schreier theorem,
see \cite[Theorem 17, page 316]{JacobsonIII},
the only finite subgroups of $\text{Gal}(\overline{\mathbf{Q}}/\mathbf{Q})$
are $\{1\}$ and the conjugates of the order two group
$\text{Gal}(\overline{\mathbf{Q}}/\overline{\mathbf{Q}} \cap \mathbf{R})$.
Hence, if
$\Spec(k) \to X$ is a morphism with $k$ algebraic over $\mathbf{Q}$,
then it follows from Lemma \ref{lemma-quotient-field-map} and the theorem
just mentioned that either $k$ is $\overline{\mathbf{Q}}$ or isomorphic to
$\overline{\mathbf{Q}} \cap \mathbf{R}$.
\end{example}

\noindent
What is wrong with the example above is that
the Galois group comes equipped with a topology,
and this should somehow be part of any construction
of a quotient of $\Spec(\overline{\mathbf{Q}})$.
The following example is much more reasonable in my opinion
and may actually occur in ``nature''.

\begin{example}
\label{example-affine-line-translation}
Let $k$ be a field of characteristic zero.
Let $U = \mathbf{A}^1_k$ and let $G = \mathbf{Z}$.
As action we take $n(x) = x + n$, i.e., the action of
$\mathbf{Z}$ on the affine line by translation.
The only fixed point is the generic point and it
is clearly the case that $\mathbf{Z}$ injects into
the automorphism group of the field $k(x)$. (This is
where we use the characteristic zero assumption.)
Consider the morphism
$$
\gamma : \Spec(k(x)) \longrightarrow X = \mathbf{A}^1_k/\mathbf{Z}
$$
of the generic point of the affine line into the quotient.
We claim that this morphism does not factor through any
monomorphism $\Spec(L) \to X$ of the spectrum of
a field to $X$. (Contrary to what happens for schemes, see
Schemes, Section \ref{schemes-section-points}.) In fact, since
$\mathbf{Z}$ does not have any nontrivial finite subgroups we see from
Lemma \ref{lemma-quotient-field-map} that for any such
factorization $k(x) = L$. Finally, $\gamma$ is not a monomorphism
since
$$
\Spec(k(x)) \times_{\gamma, X, \gamma} \Spec(k(x))
\cong
\Spec(k(x)) \times \mathbf{Z}.
$$
\end{example}

\noindent
This example suggests that in order to define points of an algebraic space
$X$ we should consider equivalence classes of morphisms from spectra
of fields into $X$ and not the set of monomorphisms from spectra of fields.

\medskip\noindent
We finish with a truly awful example.

\begin{example}
\label{example-infinite-product}
Let $k$ be a field.
Let $A = \prod_{n \in \mathbf{N}} k$ be the infinite product.
Set $U = \Spec(A)$ seen as a scheme over $S = \Spec(k)$.
Note that the projection maps $\text{pr}_n : A \to k$ define open
and closed immersions $f_n : S \to U$. Set
$$
R = U \amalg \coprod\nolimits_{(n, m) \in \mathbf{N}^2, \ n \not = m} S
$$
with morphism $j$ equal to $\Delta_{U/S}$ on the component $U$
and $j = (f_n, f_m)$ on the component $S$ corresponding to $(n, m)$.
It is clear from the remark above that $s, t$ are \'etale.
It is also clear that $j$ is an equivalence relation. Hence we
obtain an algebraic space
$$
X = U/R.
$$
To see what this means we specialize to the case where
the field $k$ is finite with $q$ elements. Let us first
discuss the topological space $|U|$ associated to the scheme $U$
a little bit. All elements of $A$ satisfy $x^q = x$.
Hence every residue field of $A$ is isomorphic to $k$, and
all points of $U$ are closed. But the topology on $U$ isn't
the discrete topology. Let $u_n \in |U|$ be the point corresponding
to $f_n$. As mentioned above the points $u_n$ are
the open points (and hence isolated). This implies there have
to be other points since we know $U$ is quasi-compact, see
Algebra, Lemma \ref{algebra-lemma-quasi-compact}
(hence not equal to an infinite discrete set).
Another way to see this is because the (proper) ideal
$$
I =
\{x = (x_n) \in A \mid \text{all but a finite number of }x_n\text{ are zero}\}
$$
is contained in a maximal ideal. Note also that every element
$x$ of $A$ is of the form $x = ue$ where $u$ is a unit and $e$ is an
idempotent. Hence a basis for the topology of $A$ consists of open and
closed subsets (see Algebra, Lemma \ref{algebra-lemma-idempotent-spec}.)
So the topology on $|U|$ is totally disconnected, but nontrivial.
Finally, note that $\{u_n\}$ is dense in $|U|$.

\medskip\noindent
We will later define a topological space $|X|$ associated to $X$, see
Properties of Spaces, Section \ref{spaces-properties-section-points}.
What can we say about $|X|$?
It turns out that the map $|U| \to |X|$ is surjective and continuous.
All the points $u_n$ map to the same point $x_0$ of $|X|$, and none of
the other points get identified. Since $\{u_n\}$ is dense in $|U|$ we
conclude that the closure of $x_0$ in $|X|$ is $|X|$. In other words
$|X|$ is irreducible and $x_0$ is a generic point of $|X|$. This seems
bizarre since also $x_0$ is the image of a section
$S \to X$ of the structure morphism $X \to S$ (and in the case of
schemes this would imply it was a closed point, see
Morphisms, Lemma
\ref{morphisms-lemma-algebraic-residue-field-extension-closed-point-fibre}).
\end{example}

\noindent
Whatever you think is actually going on in this example, it certainly
shows that some care has to be exercised when defining irreducible
components, connectedness, etc of algebraic spaces.






\section{Change of big site}
\label{section-change-big-site}

\noindent
In this section we briefly discuss what happens when we change big sites.
The upshot is that we can always enlarge the big site at will, hence we
may assume any set of schemes we want to consider is contained in the big
fppf site over which we consider our algebraic space. Here is a precise
statement of the result.

\begin{lemma}
\label{lemma-change-big-site}
Suppose given big sites $\Sch_{fppf}$ and $\Sch'_{fppf}$.
Assume that $\Sch_{fppf}$ is contained in $\Sch'_{fppf}$,
see Topologies, Section \ref{topologies-section-change-alpha}.
Let $S$ be an object of $\Sch_{fppf}$. Let
\begin{align*}
g : \Sh((\Sch/S)_{fppf})
\longrightarrow
\Sh((\Sch'/S)_{fppf}), \\
f : \Sh((\Sch'/S)_{fppf})
\longrightarrow
\Sh((\Sch/S)_{fppf})
\end{align*}
be the morphisms of topoi of
Topologies, Lemma \ref{topologies-lemma-change-alpha}.
Let $F$ be a sheaf of sets on $(\Sch/S)_{fppf}$. Then
\begin{enumerate}
\item if $F$ is representable by a scheme
$X \in \Ob((\Sch/S)_{fppf})$ over $S$,
then $f^{-1}F$ is representable too, in fact it is representable by the
same scheme $X$, now viewed as an object of $(\Sch'/S)_{fppf}$, and
\item if $F$ is an algebraic space over $S$, then $f^{-1}F$ is an algebraic
space over $S$ also.
\end{enumerate}
\end{lemma}

\begin{proof}
Let $X \in \Ob((\Sch/S)_{fppf})$. Let us write $h_X$ for the
representable sheaf on $(\Sch/S)_{fppf}$ associated to $X$, and
$h'_X$ for the representable sheaf on $(\Sch'/S)_{fppf}$ associated to
$X$. By the description of $f^{-1}$ in
Topologies, Section \ref{topologies-section-change-alpha}
we see that $f^{-1}h_X = h'_X$. This proves (1).

\medskip\noindent
Next, suppose that $F$
is an algebraic space over $S$. By Lemma \ref{lemma-space-presentation}
this means that $F = h_U/h_R$ for some \'etale equivalence relation
$R \to U \times_S U$ in $(\Sch/S)_{fppf}$. Since $f^{-1}$ is an
exact functor we conclude that $f^{-1}F = h'_U/h'_R$. Hence
$f^{-1}F$ is an algebraic space over $S$ by Theorem \ref{theorem-presentation}.
\end{proof}

\noindent
Note that this lemma is purely set theoretical and has virtually no content.
Moreover, it is not true (in general) that the restriction of an algebraic
space over the bigger site is an algebraic space over the smaller site (simply
by reasons of cardinality). Hence we can only ever use a simple lemma of this
kind to enlarge the base category and never to shrink it.

\begin{lemma}
\label{lemma-fully-faithful}
Suppose $\Sch_{fppf}$ is contained in $\Sch'_{fppf}$.
Let $S$ be an object of $\Sch_{fppf}$. Denote
$\textit{Spaces}/S$ the category of algebraic spaces over $S$
defined using $\Sch_{fppf}$. Similarly, denote
$\textit{Spaces}'/S$ the category of algebraic spaces over $S$
defined using $\Sch'_{fppf}$. The construction of
Lemma \ref{lemma-change-big-site}
defines a fully faithful functor
$$
\textit{Spaces}/S \longrightarrow \textit{Spaces}'/S
$$
whose essential image consists of those $X' \in \Ob(\textit{Spaces}'/S)$
such that there exist $U, R \in \Ob((\Sch/S)_{fppf})$\footnote{Requiring the
existence of $R$ is necessary because of our choice of the function $Bound$ in
Sets, Equation (\ref{sets-equation-bound}). The size of the fibre product
$U \times_{X'} U$ can grow faster than $Bound$ in terms of the size of $U$. We
can illustrate this by setting $S = \Spec(A)$, $U = \Spec(A[x_i, i \in I])$ and
$R = \coprod_{(\lambda_i) \in A^I} \Spec(A[x_i, y_i]/(x_i - \lambda_i y_i))$.
In this case the size of $R$ grows like $\kappa^\kappa$ where $\kappa$ is the
size of $U$.} and morphisms
$$
U \longrightarrow X'
\quad\text{and}\quad
R \longrightarrow U \times_{X'} U
$$
in $\Sh((\Sch'/S)_{fppf})$ which are surjective as maps of sheaves
(for example if the displayed morphisms are surjective and \'etale).
\end{lemma}

\begin{proof}
In Sites, Lemma \ref{sites-lemma-bigger-site} we have seen that the functor
$f^{-1} : \Sh((\Sch/S)_{fppf}) \to \Sh((\Sch'/S)_{fppf})$
is fully faithful (see discussion in
Topologies, Section \ref{topologies-section-change-alpha}).
Hence we see that the displayed functor of the lemma is fully faithful.

\medskip\noindent
Suppose that $X' \in \Ob(\textit{Spaces}'/S)$ such that there exists
$U \in \Ob((\Sch/S)_{fppf})$ and a map $U \to X'$ in
$\Sh((\Sch'/S)_{fppf})$ which is surjective as a map of sheaves.
Let $U' \to X'$ be a surjective \'etale morphism with
$U' \in \Ob((\Sch'/S)_{fppf})$. Let $\kappa = \text{size}(U)$, see
Sets, Section \ref{sets-section-categories-schemes}.
Then $U$ has an affine open covering $U = \bigcup_{i \in I} U_i$
with $|I| \leq \kappa$. Observe that $U' \times_{X'} U \to U$ is \'etale
and surjective. For each $i$ we can pick a quasi-compact
open $U'_i \subset U'$ such that $U'_i \times_{X'} U_i \to U_i$
is surjective (because the scheme $U' \times_{X'} U_i$ is the
union of the Zariski opens $W \times_{X'} U_i$ for $W \subset U'$
affine and because $U' \times_{X'} U_i \to U_i$ is \'etale hence open).
Then $\coprod_{i \in I} U'_i \to X$ is surjective \'etale
because of our assumption that $U \to X$ and hence $\coprod U_i \to X$
is a surjection of sheaves (details omitted).
Because $U'_i \times_{X'} U \to U'_i$ is a surjection of sheaves
and because $U'_i$ is quasi-compact,
we can find a quasi-compact open $W_i \subset U'_i \times_{X'} U$
such that $W_i \to U'_i$ is surjective as a map of sheaves
(details omitted). Then $W_i \to U$ is \'etale and we conclude
that $\text{size}(W_i) \leq \text{size}(U)$, see
Sets, Lemma \ref{sets-lemma-bound-finite-type}. By
Sets, Lemma \ref{sets-lemma-bound-by-covering} we conclude
that $\text{size}(U'_i) \leq \text{size}(U)$.
Hence $\coprod_{i \in I} U'_i$ is isomorphic to an object of
$(\Sch/S)_{fppf}$ by Sets, Lemma \ref{sets-lemma-bound-size}.

\medskip\noindent
Now let $X'$, $U \to X'$ and $R \to U \times_{X'} U$ be as in the
statement of the lemma. In the previous paragraph we have seen that
we can find $U' \in \Ob((\Sch/S)_{fppf})$ and a surjective \'etale morphism
$U' \to X'$ in $\Sh((\Sch'/S)_{fppf})$. Then
$U' \times_{X'} U \to U'$ is a surjection of sheaves, i.e., we can find an fppf
covering $\{U'_i \to U'\}$ such that $U'_i \to U'$ factors through
$U' \times_{X'} U \to U'$.
By Sets, Lemma \ref{sets-lemma-bound-fppf-covering}
we can find
$\tilde U \to U'$
which is surjective, flat, and locally of finite presentation,
with $\text{size}(\tilde U) \leq \text{size}(U')$,
such that $\tilde U \to U'$ factors through
$U' \times_{X'} U \to U'$. Then we consider
$$
\xymatrix{
U' \times_{X'} U' \ar[d] &
\tilde U \times_{X'} \tilde U  \ar[l] \ar[d] \ar[r] &
U \times_{X'} U \ar[d] \\
U' \times_S U' & \tilde U \times_S \tilde U \ar[l] \ar[r] & U \times_S U
}
$$
The squares are cartesian. We know the objects of the bottom row
are represented by objects of $(\Sch/S)_{fppf}$. By the result of the
argument of the previous paragraph, the same is true for $U \times_{X'} U$
(as we have the surjection of sheaves $R \to U \times_{X'} U$
by assumption). Since $(\Sch/S)_{fppf}$ is closed under fibre
products (by construction), we see that $\tilde U \times_{X'} \tilde U$
is represented by an object of $(\Sch/S)_{fppf}$. Finally, the
map $\tilde U \times_{X'} \tilde U \to U' \times_{X'} U'$ is
a surjection of fppf sheaves as $\tilde U \to U'$ is so.
Thus we can once more apply the result of the previous paragraph
to conclude that $R' = U' \times_{X'} U'$ is represented by an object
of $(\Sch/S)_{fppf}$. At this point
Lemma \ref{lemma-space-presentation} and
Theorem \ref{theorem-presentation} imply that $X = h_{U'}/h_{R'}$
is an object of $\textit{Spaces}/S$ such that $f^{-1}X \cong X'$
as desired.
\end{proof}



\section{Change of base scheme}
\label{section-change-base-scheme}

\noindent
In this section we briefly discuss what happens when we change base schemes.
The upshot is that given a morphism $S \to S'$ of base schemes, any algebraic
space over $S$ can be viewed as an algebraic space over $S'$. And, given an
algebraic space $F'$ over $S'$ there is a base change $F'_S$ which is
an algebraic space over $S$.
We explain only what happens in case $S \to S'$ is a morphism of the
big fppf site under consideration, if only $S$ or $S'$ is contained in the
big site, then one first enlarges the big site as in
Section \ref{section-change-big-site}.

\begin{lemma}
\label{lemma-change-base-scheme}
Suppose given a big site $\Sch_{fppf}$.
Let $g : S \to S'$ be morphism of $\Sch_{fppf}$.
Let $j : (\Sch/S)_{fppf} \to (\Sch/S')_{fppf}$ be
the corresponding localization functor.
Let $F$ be a sheaf of sets on $(\Sch/S)_{fppf}$.
Then
\begin{enumerate}
\item for a scheme $T'$ over $S'$ we have
$j_!F(T'/S') =
\coprod\nolimits_{\varphi : T' \to S} F(T' \xrightarrow{\varphi} S),$
\item if $F$ is representable by a scheme
$X \in \Ob((\Sch/S)_{fppf})$,
then $j_!F$ is representable by $j(X)$ which is
$X$ viewed as a scheme over $S'$, and
\item if $F$ is an algebraic space over $S$, then $j_!F$ is an algebraic
space over $S'$, and if $F = U/R$ is a presentation, then
$j_!F = j(U)/j(R)$ is a presentation.
\end{enumerate}
Let $F'$ be a sheaf of sets on $(\Sch/S')_{fppf}$. Then
\begin{enumerate}
\item[(4)] for a scheme $T$ over $S$ we have $j^{-1}F'(T/S) = F'(T/S')$,
\item[(5)] if $F'$ is representable by a scheme
$X' \in \Ob((\Sch/S')_{fppf})$, then
$j^{-1}F'$ is representable, namely by $X'_S = S \times_{S'} X'$, and
\item[(6)] if $F'$ is an algebraic space, then
$j^{-1}F'$ is an algebraic space, and if $F' = U'/R'$ is a presentation,
then $j^{-1}F' = U'_S/R'_S$ is a presentation.
\end{enumerate}
\end{lemma}

\begin{proof}
The functors $j_!$, $j_*$ and $j^{-1}$ are defined in
Sites, Lemma \ref{sites-lemma-relocalize}
where it is also shown that $j = j_{S/S'}$ is the localization
of $(\Sch/S')_{fppf}$ at the object $S/S'$. Hence
all of the material on localization functors is available for $j$.
The formula in (1) is
Sites, Lemma \ref{sites-lemma-describe-j-shriek-good-site}.
By definition $j_!$ is the left adjoint to restriction $j^{-1}$,
hence $j_!$ is right exact. By
Sites, Lemma \ref{sites-lemma-j-shriek-commutes-equalizers-fibre-products}
it also commutes with fibre products and equalizers.
By
Sites, Lemma \ref{sites-lemma-describe-j-shriek-representable}
we see that $j_!h_X = h_{j(X)}$ hence (2) holds.
If $F$ is an algebraic space over $S$, then we can write $F = U/R$
(Lemma \ref{lemma-space-presentation})
and we get
$$
j_!F = j(U)/j(R)
$$
because $j_!$ being right exact commutes with coequalizers, and moreover
$j(R) = j(U) \times_{j_!F} j(U)$ as $j_!$ commutes with fibre products.
Since the morphisms $j(s), j(t) : j(R) \to j(U)$ are simply the morphisms
$s, t : R \to U$ (but viewed as morphisms of schemes over $S'$), they
are still \'etale. Thus $(j(U), j(R), s, t)$ is an \'etale equivalence relation.
Hence by
Theorem \ref{theorem-presentation}
we conclude that $j_!F$ is an algebraic space.

\medskip\noindent
Proof of (4), (5), and (6). The description of $j^{-1}$ is in
Sites, Section \ref{sites-section-localize}.
The restriction of the representable sheaf associated to $X'/S'$
is the representable sheaf associated to
$X'_S = S \times_{S'} Y'$ by
Sites, Lemma \ref{sites-lemma-localize-given-products}.
The restriction functor $j^{-1}$ is exact, hence $j^{-1}F' = U'_S/R'_S$.
Again by exactness the sheaf $R'_S$ is still an equivalence relation on
$U'_S$. Finally the two maps $R'_S \to U'_S$ are \'etale as base changes
of the \'etale morphisms $R' \to U'$. Hence $j^{-1}F' = U'_S/R'_S$ is
an algebraic space by
Theorem \ref{theorem-presentation}
and we win.
\end{proof}

\noindent
Note how the presentation $j_!F = j(U)/j(R)$ is just the presentation
of $F$ but viewed as a presentation by schemes over $S'$. Hence the
following definition makes sense.

\begin{definition}
\label{definition-base-change}
Let $\Sch_{fppf}$ be a big fppf site.
Let $S \to S'$ be a morphism of this site.
\begin{enumerate}
\item If $F'$ is an algebraic space over $S'$, then the
{\it base change of $F'$ to $S$} is the
algebraic space $j^{-1}F'$ described in
Lemma \ref{lemma-change-base-scheme}. We denote it $F'_S$.
\item If $F$ is an algebraic space over $S$, then $F$
{\it viewed as an algebraic space over $S'$}
is the algebraic space $j_!F$ over $S'$ described in
Lemma \ref{lemma-change-base-scheme}. We often simply denote this
$F$; if not then we will write $j_!F$.
\end{enumerate}
\end{definition}

\noindent
The algebraic space $j_!F$ comes equipped with a canonical morphism
$j_!F \to S$ of algebraic spaces over $S'$. This is true simply
because the sheaf $j_!F$ maps to $h_S$ (see for example the
explicit description in
Lemma \ref{lemma-change-base-scheme}).
In fact, in
Sites, Lemma \ref{sites-lemma-essential-image-j-shriek}
we have seen that the category of sheaves on $(\Sch/S)_{fppf}$
is equivalent to the category of pairs $(\mathcal{F}', \mathcal{F}' \to h_S)$
consisting of a sheaf on $(\Sch/S')_{fppf}$ and a map of sheaves
$\mathcal{F}' \to h_S$. The equivalence assigns to the sheaf $\mathcal{F}$
the pair $(j_!\mathcal{F}, j_!\mathcal{F} \to h_S)$.
This, combined with the above, leads to the following
result for categories of algebraic spaces.

\begin{lemma}
\label{lemma-category-of-spaces-over-smaller-base-scheme}
Let $\Sch_{fppf}$ be a big fppf site.
Let $S \to S'$ be a morphism of this site.
The construction above give an equivalence of
categories
$$
\left\{
\begin{matrix}
\text{category of algebraic}\\
\text{spaces over }S
\end{matrix}
\right\}
\leftrightarrow
\left\{
\begin{matrix}
\text{category of pairs }(F', F' \to S)\text{ consisting}\\
\text{of an algebraic space }F'\text{ over }S'\text{ and a}\\
\text{morphism }F' \to S\text{ of algebraic spaces over }S'
\end{matrix}
\right\}
$$
\end{lemma}

\begin{proof}
Let $F$ be an algebraic space over $S$. The functor from left to right
assigns the pair $(j_!F, j_!F \to S)$ to $F$
which is an object of the right hand side by
Lemma \ref{lemma-change-base-scheme}.
Since this defines an equivalence of categories of sheaves by
Sites, Lemma \ref{sites-lemma-essential-image-j-shriek}
to finish the proof it suffices to show:
if $F$ is a sheaf and $j_!F$ is an algebraic space, then $F$
is an algebraic space. To do this, write
$j_!F = U'/R'$ as in
Lemma \ref{lemma-space-presentation}
with $U', R' \in \Ob((\Sch/S')_{fppf})$.
Then the compositions $U' \to j_!F \to S$ and $R' \to j_!F \to S$
are morphisms of schemes over $S'$. Denote $U, R$ the corresponding
objects of $(\Sch/S)_{fppf}$. The two morphisms
$R' \to U'$ are morphisms over $S$ and hence correspond to
morphisms $R \to U$. Since these are simply the same
morphisms (but viewed over $S$) we see that we get an \'etale
equivalence relation over $S$. As $j_!$ defines an equivalence of
categories of sheaves (see reference above) we see that
$F = U/R$ and by
Theorem \ref{theorem-presentation}
we see that $F$ is an algebraic space.
\end{proof}

\noindent
The following lemma is a slight rephrasing of the above.

\begin{lemma}
\label{lemma-rephrase}
Let $\Sch_{fppf}$ be a big fppf site.
Let $S \to S'$ be a morphism of this site.
Let $F'$ be a sheaf on $(\Sch/S')_{fppf}$.
The following are equivalent:
\begin{enumerate}
\item The restriction $F'|_{(\Sch/S)_{fppf}}$
is an algebraic space over $S$, and
\item the sheaf $h_S \times F'$ is an algebraic space over $S'$.
\end{enumerate}
\end{lemma}

\begin{proof}
The restriction and the product match under
the equivalence of categories of
Sites, Lemma \ref{sites-lemma-essential-image-j-shriek}
so that
Lemma \ref{lemma-category-of-spaces-over-smaller-base-scheme}
above gives the result.
\end{proof}

\noindent
We finish this section with a lemma on a compatibility.

\begin{lemma}
\label{lemma-viewed-as-properties}
Let $\Sch_{fppf}$ be a big fppf site.
Let $S \to S'$ be a morphism of this site.
Let $F$ be an algebraic space over $S$.
Let $T$ be a scheme over $S$ and let $f : T \to F$ be
a morphism over $S$.
Let $f' : T' \to F'$ be the morphism over $S'$ we get from
$f$ by applying the equivalence of categories described in
Lemma \ref{lemma-category-of-spaces-over-smaller-base-scheme}.
For any property $\mathcal{P}$ as in
Definition \ref{definition-relative-representable-property}
we have $\mathcal{P}(f') \Leftrightarrow \mathcal{P}(f)$.
\end{lemma}

\begin{proof}
Suppose that $U$ is a scheme over $S$, and $U \to F$ is a surjective \'etale
morphism. Denote $U'$ the scheme $U$ viewed as a scheme over $S'$. In
Lemma \ref{lemma-change-base-scheme}
we have seen that $U' \to F'$ is surjective \'etale. Since
$$
j(T \times_{f, F} U) = T' \times_{f', F'} U'
$$
the morphism of schemes $T \times_{f, F} U \to U$
is identified with the morphism of schemes
$T' \times_{f', F'} U' \to U'$.
It is the same morphism, just viewed over different base schemes.
Hence the lemma follows from
Lemma \ref{lemma-representable-morphisms-spaces-property}.
\end{proof}










\begin{multicols}{2}[\section{Other chapters}]
\noindent
Preliminaries
\begin{enumerate}
\item \hyperref[introduction-section-phantom]{Introduction}
\item \hyperref[conventions-section-phantom]{Conventions}
\item \hyperref[sets-section-phantom]{Set Theory}
\item \hyperref[categories-section-phantom]{Categories}
\item \hyperref[topology-section-phantom]{Topology}
\item \hyperref[sheaves-section-phantom]{Sheaves on Spaces}
\item \hyperref[sites-section-phantom]{Sites and Sheaves}
\item \hyperref[stacks-section-phantom]{Stacks}
\item \hyperref[fields-section-phantom]{Fields}
\item \hyperref[algebra-section-phantom]{Commutative Algebra}
\item \hyperref[brauer-section-phantom]{Brauer Groups}
\item \hyperref[homology-section-phantom]{Homological Algebra}
\item \hyperref[derived-section-phantom]{Derived Categories}
\item \hyperref[simplicial-section-phantom]{Simplicial Methods}
\item \hyperref[more-algebra-section-phantom]{More on Algebra}
\item \hyperref[smoothing-section-phantom]{Smoothing Ring Maps}
\item \hyperref[modules-section-phantom]{Sheaves of Modules}
\item \hyperref[sites-modules-section-phantom]{Modules on Sites}
\item \hyperref[injectives-section-phantom]{Injectives}
\item \hyperref[cohomology-section-phantom]{Cohomology of Sheaves}
\item \hyperref[sites-cohomology-section-phantom]{Cohomology on Sites}
\item \hyperref[dga-section-phantom]{Differential Graded Algebra}
\item \hyperref[dpa-section-phantom]{Divided Power Algebra}
\item \hyperref[hypercovering-section-phantom]{Hypercoverings}
\end{enumerate}
Schemes
\begin{enumerate}
\setcounter{enumi}{24}
\item \hyperref[schemes-section-phantom]{Schemes}
\item \hyperref[constructions-section-phantom]{Constructions of Schemes}
\item \hyperref[properties-section-phantom]{Properties of Schemes}
\item \hyperref[morphisms-section-phantom]{Morphisms of Schemes}
\item \hyperref[coherent-section-phantom]{Cohomology of Schemes}
\item \hyperref[divisors-section-phantom]{Divisors}
\item \hyperref[limits-section-phantom]{Limits of Schemes}
\item \hyperref[varieties-section-phantom]{Varieties}
\item \hyperref[topologies-section-phantom]{Topologies on Schemes}
\item \hyperref[descent-section-phantom]{Descent}
\item \hyperref[perfect-section-phantom]{Derived Categories of Schemes}
\item \hyperref[more-morphisms-section-phantom]{More on Morphisms}
\item \hyperref[flat-section-phantom]{More on Flatness}
\item \hyperref[groupoids-section-phantom]{Groupoid Schemes}
\item \hyperref[more-groupoids-section-phantom]{More on Groupoid Schemes}
\item \hyperref[etale-section-phantom]{\'Etale Morphisms of Schemes}
\end{enumerate}
Topics in Scheme Theory
\begin{enumerate}
\setcounter{enumi}{40}
\item \hyperref[chow-section-phantom]{Chow Homology}
\item \hyperref[intersection-section-phantom]{Intersection Theory}
\item \hyperref[pic-section-phantom]{Picard Schemes of Curves}
\item \hyperref[adequate-section-phantom]{Adequate Modules}
\item \hyperref[dualizing-section-phantom]{Dualizing Complexes}
\item \hyperref[duality-section-phantom]{Duality for Schemes}
\item \hyperref[discriminant-section-phantom]{Discriminants and Differents}
\item \hyperref[local-cohomology-section-phantom]{Local Cohomology}
\item \hyperref[curves-section-phantom]{Algebraic Curves}
\item \hyperref[resolve-section-phantom]{Resolution of Surfaces}
\item \hyperref[models-section-phantom]{Semistable Reduction}
\item \hyperref[pione-section-phantom]{Fundamental Groups of Schemes}
\item \hyperref[etale-cohomology-section-phantom]{\'Etale Cohomology}
\item \hyperref[ssgroups-section-phantom]{Linear Algebraic Groups}
\item \hyperref[crystalline-section-phantom]{Crystalline Cohomology}
\item \hyperref[proetale-section-phantom]{Pro-\'etale Cohomology}
\end{enumerate}
Algebraic Spaces
\begin{enumerate}
\setcounter{enumi}{56}
\item \hyperref[spaces-section-phantom]{Algebraic Spaces}
\item \hyperref[spaces-properties-section-phantom]{Properties of Algebraic Spaces}
\item \hyperref[spaces-morphisms-section-phantom]{Morphisms of Algebraic Spaces}
\item \hyperref[decent-spaces-section-phantom]{Decent Algebraic Spaces}
\item \hyperref[spaces-cohomology-section-phantom]{Cohomology of Algebraic Spaces}
\item \hyperref[spaces-limits-section-phantom]{Limits of Algebraic Spaces}
\item \hyperref[spaces-divisors-section-phantom]{Divisors on Algebraic Spaces}
\item \hyperref[spaces-over-fields-section-phantom]{Algebraic Spaces over Fields}
\item \hyperref[spaces-topologies-section-phantom]{Topologies on Algebraic Spaces}
\item \hyperref[spaces-descent-section-phantom]{Descent and Algebraic Spaces}
\item \hyperref[spaces-perfect-section-phantom]{Derived Categories of Spaces}
\item \hyperref[spaces-more-morphisms-section-phantom]{More on Morphisms of Spaces}
\item \hyperref[spaces-flat-section-phantom]{Flatness on Algebraic Spaces}
\item \hyperref[spaces-groupoids-section-phantom]{Groupoids in Algebraic Spaces}
\item \hyperref[spaces-more-groupoids-section-phantom]{More on Groupoids in Spaces}
\item \hyperref[bootstrap-section-phantom]{Bootstrap}
\item \hyperref[spaces-pushouts-section-phantom]{Pushouts of Algebraic Spaces}
\end{enumerate}
Topics in Geometry
\begin{enumerate}
\setcounter{enumi}{73}
\item \hyperref[spaces-chow-section-phantom]{Chow Groups of Spaces}
\item \hyperref[groupoids-quotients-section-phantom]{Quotients of Groupoids}
\item \hyperref[spaces-more-cohomology-section-phantom]{More on Cohomology of Spaces}
\item \hyperref[spaces-simplicial-section-phantom]{Simplicial Spaces}
\item \hyperref[spaces-duality-section-phantom]{Duality for Spaces}
\item \hyperref[formal-spaces-section-phantom]{Formal Algebraic Spaces}
\item \hyperref[restricted-section-phantom]{Restricted Power Series}
\item \hyperref[spaces-resolve-section-phantom]{Resolution of Surfaces Revisited}
\end{enumerate}
Deformation Theory
\begin{enumerate}
\setcounter{enumi}{81}
\item \hyperref[formal-defos-section-phantom]{Formal Deformation Theory}
\item \hyperref[defos-section-phantom]{Deformation Theory}
\item \hyperref[cotangent-section-phantom]{The Cotangent Complex}
\item \hyperref[examples-defos-section-phantom]{Deformation Problems}
\end{enumerate}
Algebraic Stacks
\begin{enumerate}
\setcounter{enumi}{85}
\item \hyperref[algebraic-section-phantom]{Algebraic Stacks}
\item \hyperref[examples-stacks-section-phantom]{Examples of Stacks}
\item \hyperref[stacks-sheaves-section-phantom]{Sheaves on Algebraic Stacks}
\item \hyperref[criteria-section-phantom]{Criteria for Representability}
\item \hyperref[artin-section-phantom]{Artin's Axioms}
\item \hyperref[quot-section-phantom]{Quot and Hilbert Spaces}
\item \hyperref[stacks-properties-section-phantom]{Properties of Algebraic Stacks}
\item \hyperref[stacks-morphisms-section-phantom]{Morphisms of Algebraic Stacks}
\item \hyperref[stacks-limits-section-phantom]{Limits of Algebraic Stacks}
\item \hyperref[stacks-cohomology-section-phantom]{Cohomology of Algebraic Stacks}
\item \hyperref[stacks-perfect-section-phantom]{Derived Categories of Stacks}
\item \hyperref[stacks-introduction-section-phantom]{Introducing Algebraic Stacks}
\item \hyperref[stacks-more-morphisms-section-phantom]{More on Morphisms of Stacks}
\item \hyperref[stacks-geometry-section-phantom]{The Geometry of Stacks}
\end{enumerate}
Topics in Moduli Theory
\begin{enumerate}
\setcounter{enumi}{99}
\item \hyperref[moduli-section-phantom]{Moduli Stacks}
\item \hyperref[moduli-curves-section-phantom]{Moduli of Curves}
\end{enumerate}
Miscellany
\begin{enumerate}
\setcounter{enumi}{101}
\item \hyperref[examples-section-phantom]{Examples}
\item \hyperref[exercises-section-phantom]{Exercises}
\item \hyperref[guide-section-phantom]{Guide to Literature}
\item \hyperref[desirables-section-phantom]{Desirables}
\item \hyperref[coding-section-phantom]{Coding Style}
\item \hyperref[obsolete-section-phantom]{Obsolete}
\item \hyperref[fdl-section-phantom]{GNU Free Documentation License}
\item \hyperref[index-section-phantom]{Auto Generated Index}
\end{enumerate}
\end{multicols}


\bibliography{my}
\bibliographystyle{amsalpha}

\end{document}
