\documentclass[10pt]{amsart}
\usepackage[leqno]{amsmath}
\usepackage{amssymb}
   \topmargin=0in
   \oddsidemargin=0in
   \evensidemargin=0in
   \textwidth=6.5in
   \textheight=8.5in


\def\Spec{\mathop{\rm Spec}}
\def\Hom{\mathop{\rm Hom}\nolimits}
\def\Coker{\mathop{\rm Coker}}


\begin{document}
\centerline{\sc Algebraic Groups I. Homework 1}
\medskip\medskip

\medskip\noindent
1.  This exercise studies the endomorphism rings
of the $k$-groups ${\mathbf{G}}_m$ and ${\mathbf{G}}_a$, with $k$ any commutative ring. 

(i)  Prove that ${\rm{End}}_k({\mathbf{G}}_a)$ consists of $f \in k[t]$ such that
$f(x+y) = f(x) + f(y)$ in $k[x,y]$, and that
${\rm{End}}_k({\mathbf{G}}_m)$ consists of $f \in k[t,t^{-1}]$ such that $f(xy) = f(x)f(y)$
in $k[x,y,x^{-1},y^{-1}]$ and $f$ has no zeros on any geometric fibers over
$\Spec k$.

(ii) Deduce that if $k$ is a $\mathbf{Q}$-algebra then naturally ${\rm{End}}_k({\mathbf{G}}_a) = k$,
and that if $k$ is a field with characteristic $p > 0$ then it consists of
$f = \sum c_j t^{p^j}$ ($c_j \in k$).  What if $k = \mathbf{Z}/(p^2)$?

(iii) Prove that ${\rm{End}}_k({\mathbf{G}}_m) = \mathbf{Z}$ when $k$ is a field, and deduce
the same when $k$ is an artin local ring via induction on the length of $k$.
(Hint: reduce to the case when $f$ vanishes on the special fiber.) 

(iv) Prove that ${\rm{End}}_k({\mathbf{G}}_m) = \mathbf{Z}$ for $k$ {\em any} local ring by
using (iii) to settle the case of a complete local noetherian ring, then any local noetherian
ring, and finally any local ring (by using local noetherian subrings of $k$).
Deduce that if $k$ is any ring whatsoever, an endomorphism of the $k$-group
${\mathbf{G}}_m$ is $t \mapsto t^n$  for a locally constant function $n:\Spec k \rightarrow \mathbf{Z}$.

\medskip\noindent
2. Let $V$ be a finite-dimensional vector space over a field $k$.  This exercise develops
coordinate-free versions of ${\rm{GL}}_n$, ${\rm{SL}}_n$, and
${\rm{Sp}}_{2n}$ attached to $V$.

(i) Elements of the graded symmetric algebra
${\rm{Sym}}(V^{\ast})$ are called {\em polynomial functions on $V$}.  Justify the name
(even for finite $k$!)  by identifying them with {\em functorial maps} of sets $V_R \rightarrow R$
given by polynomial expressions relative to some (equivalently, any) basis of $V$,
with $R$ a varying $k$-algebra.
In particular, show that $\det$ is a polynomial function on ${\rm{End}}(V)$. 

(ii) For any $k$-algebra $R$, define the functors
$\underline{\rm{End}}(V)$ and $\underline{\rm{Aut}}(V)$ on $k$-algebras $R$ by
$R \rightsquigarrow {\rm{End}}(V_R)$, $R \rightsquigarrow {\rm{Aut}}_R(V_R)$.  
Using the  identification ${\rm{End}}(V_R,V_R) = {\rm{End}}(V)_R$,
prove that $\underline{\rm{End}}(V)$ is represented by
${\rm{Sym}}({\rm{End}}(V)^{\ast})$.

(iii) Define $\det \in
{\rm{Sym}}({\rm{End}}(V)^{\ast})$ and prove its non-vanishing locus 
$${\rm{GL}}(V) := \Spec({\rm{Sym}}({\rm{End}}(V)^{\ast})[1/\det])$$ represents $\underline{\rm{Aut}}(V)$
as subfunctor of $\underline{\rm{End}}(V)$.
Also discuss ${\rm{SL}}(V)$ as a closed $k$-subgroup of
${\rm{GL}}(V)$.

(iv) Let $B:V \times V \rightarrow k$ be a bilinear form.
Prove that the subfunctor $\underline{\rm{Aut}}(V,B)$ of points of
$\underline{\rm{Aut}}(V)$ preserving $B$
is represented by a closed $k$-subgroup of ${\rm{GL}}(V)$.  (You can use coordinates
in the proof!)  This is pretty bad unless $B$ is non-degenerate.
(In the alternating non-degenerate case it is denoted ${\rm{Sp}}(B)$.) 

Assuming non-degeneracy, a linear automorphism $T$ of $V_R$ is 
a {\em $B$-similitude} if $B_R(Tv,Tw) = \mu(T)B(v,w)$ for all
$v, w \in V_R$ and some $\mu(T) \in R^{\times}$.  Prove $\mu(T)$ is then unique,
and show that the functor of $B$-similitudes is represented by a closed
$k$-subgroup of ${\rm{GL}}(V) \times {\mathbf{G}}_m$. (In the alternating case it is
denoted
${\rm{GSp}}(B)$.)  

\medskip\noindent
3. (i) Prove that if a connected scheme $X$ of finite type over a field $k$ has a $k$-rational point,
then $X_{k'} = X \otimes_k k'$ is connected for every finite extension $k'/k$
(hint: $X_{k'} \rightarrow X$ is open and closed; look at fiber over
$X(k)$).  Deduce that $X_{k'}$ is connected
for {\em every} extension $k'/k$ (i.e., $X$ is {\em geometrically connected} over $k$).

(ii) Prove that if $X$ and $Y$ are geometrically connected of finite type over $k$, so
is $X \times Y$; give a counterexample over $k = \mathbf{Q}$ if ``geometrically'' is removed.   
Deduce that if $G$ is a $k$-group then the identity component $G^0$
is a $k$-subgroup whose formation commutes with any extension on $k$. 

\medskip\noindent
4.  Let $G$ be a group of finite type over a field $k$.

(i) Prove that $(G_{\overline{k}})_{\rm{red}}$ is a closed $\overline{k}$-subgroup of
$G_{\overline{k}}$, and prove it is {\em smooth}. 
 Deduce that $G^0$ is {\em geometrically irreducible}.  
 
 (ii) Over any imperfect field $k$, one can make a non-reduced $k$-group $G$ such that
 $G_{\rm{red}}$ is {\em not} a $k$-subgroup.  Where does an attempted proof to the contrary get stuck?

(iii) Assume $k$ is imperfect, ${\rm{char}}(k) = p > 0$, and choose
$a \in k - k^p$.  Prove $x_0^p + a x_1^p + \dots + a^{p-1} x_{p-1}^p = 1$
defines a reduced $k$-group (think of ${\rm{N}}_{k(a^{1/p})/k}$)
that is non-reduced over $\overline{k}$ and hence not smooth!

(iv) Prove that the condition $t^n = 1$ defines a finite closed $k$-subgroup
$\mu_n \subseteq {\mathbf{G}}_m$, and show its preimage $G$ under $\det:\text{GL}_N \rightarrow {\mathbf{G}}_m$
is a $k$-subgroup of $\text{GL}_N$.  Accepting that
${\rm{SL}}_N$ is connected, prove $G^0 = {\rm{SL}}_N$ if ${\rm{char}}(k)\nmid n$.  For $k = \mathbf{Q}$ and $n = 5$,
prove that $G - G^0$ is connected but over $\overline{k}$ has 4 connected components. 

\newpage

\centerline{\sc Algebraic Groups I. Homework 2}
\medskip\medskip

\medskip\noindent
1.  Let $k$ be a perfect 
field, and $G$ a 1-dimensional connected linear algebraic $k$-group (so $G$ is geometrically
integral over $k$).   Assume $G$ is in the additive case.  This exercise proves
$G$ is $k$-isomorphic to $\mathbf{G}_a$.

(i)  Let $X$ denote its regular compactification over $k$.  
Prove that $X_{\overline{k}}$
is regular, so $X$ is smooth (hint: $\overline{k}$ is a direct limit of finite separable
extensions of $k$, and unit discriminant is a sufficient test for integral closures in the
Dedekind setting).  Deduce that $X - G$ consists of a single physical point, say
$\Spec k'$.

(ii) Prove that $k' \otimes_k \overline{k}$ is reduced
and in fact equal to $\overline{k}$.  Deduce $k' = k$, and prove that $X \simeq
{\mathbf{P}}^1_k$.  Show that $G \simeq \mathbf{G}_a$ as $k$-groups.

\medskip\noindent
2. Let $T$ be a torus of dimension $r \ge 1$
over a field $k$ (e.g., a 1-dimensional connected linear algebraic group in
the multiplicative case).  This exercise proves that $T_{k'} \simeq \mathbf{G}_m^r$ 
for some finite separable extension $k'/k$.

(i) Prove that it suffices to treat the case $k = k_s$.

(ii) Assume $k = k_s$.  We constructed an isomorphism
$f:T_{k'} \simeq {\mathbf{G}}_m^r$ as $k'$-groups for some finite extension $k'/k$.
Let $k'' = k' \otimes_k k'$, and let $p_1, p_2:\Spec k'' \rightrightarrows \Spec k'$ be the
projections.  Prove that $k''$ is an artin local ring {\em with residue field $k'$},
and deduce that the $k''$-isomorphisms $p_i^{\ast}(f):T_{k''} \simeq {\mathbf{G}}_m^r$
coincide by comparing them with $f$ on the special fiber!

(iii) For any $k$-vector space $V$, prove that the only elements of
$k' \otimes_k V$ with equal images under both maps to $k'' \otimes_k V$ are the elements
of $V$ (hint: reduce to the case $V = k$ and replace $k'$ with any $k$-vector space
$W$, and $k''$ with $W \otimes_k W$).  Deduce that $f$
uniquely descends to a $k$-isomorphism.

\medskip\noindent
3.    Let $X$ and $Y$ be schemes over a field $k$, $K/k$ an extension field,
and $f, g:X \rightrightarrows Y$ 
two $k$-morphisms. 

(i) Prove $f_K = g_K$ if and only if $f = g$.  (Use surjectivity of
$X_K \rightarrow X$ to aid in reducing to the affine case.)  Likewise
prove that if $Z, Z' \subseteq X$ are closed subschemes such that $Z_K = Z'_K$ inside
of $X_K$ then $Z = Z'$, 

(ii) If $f_K$ is an isomorphism and $X$ and $Y$ are affine, prove $f$ is an isomorphism.
Then do the same without affineness (may be really hard without Serre's
cohomological criterion for affineness). 

(iii) Assume $K/k$ is Galois, $\Gamma = {\rm{Gal}}(K/k)$.  Prove that if
a map $F:X_K \rightarrow Y_K$ satisfies $\gamma^{\ast}(F) = F$ for all
$\gamma \in \Gamma$, then $F = f_K$ for a unique $k$-map $f:X \rightarrow Y$.
Likewise, if $Z' \subseteq X_K$ is a closed subscheme
and $\gamma^{\ast}(Z') = Z'$ for all $\gamma \in \Gamma$ then prove
$Z' = Z_K$ for a unique closed subscheme $Z \subseteq X$. Do the same for open subschemes. 

\medskip\noindent
4. Let $q:V \rightarrow k$ be a quadratic form on a finite-dimensional vector space
$V$ of dimension $d \ge 2$, and let 
$B_q:V \times V \rightarrow k$ be the corresponding symmetric bilinear form.
Let $V^{\perp} = \{v \in V\,|\,B_q(v,\cdot)= 0\}$; we call $\delta_q := \dim V^{\perp}$ the {\em defect}
of $q$.

(i) Prove that $B_q$ uniquely factors through a non-degenerate symmetric bilinear form
on $V/V^{\perp}$, and $B_q$ is non-degenerate precisely when the defect is 0. 
Prove that if ${\rm{char}}(k) = 2$ then $B_q$ is alternating, and deduce
that $\delta_q \equiv \dim V \bmod 2$ for such $k$ (so $\delta_q \ge 1$ if $\dim V$ is odd). 

(ii) Prove that if $\delta_q = 0$ then $q_{\overline{k}}$ 
admits one of the following ``standard forms'':
$\sum_{i=1}^n x_i x_{i+n}$ if $\dim V = 2n$ ($n \ge 1$), and 
$x_0^2 + \sum_{i=1}^n x_i x_{i+n}$ if $\dim V = 2n+1$ ($n \ge 1$). 
Do the same if ${\rm{char}}(k) = 2$ and $\delta_q = 1$. (Distinguish whether or
not $q|_{V^{\perp}} \ne 0$.) 
How about the converse?  

(iii) If ${\rm{char}}(k) \ne 2$, prove $\delta_q = 0$  if and only if  $q \ne 0$ and 
$(q = 0) \subseteq {\mathbf{P}}^{d-1}$ is smooth.  
If ${\rm{char}}(k) = 2$ then prove $\delta_q \le 1$ with $q|_{V^{\perp}} \ne 0$
when $\delta_q = 1$ if and only if $q \ne 0$ and 
the $(q=0)$ is smooth.  (Hint: use (ii) to simplify calculations.) 
We say $q$ is {\em non-degenerate} when $q \ne 0$ and $(q = 0)$ is smooth in 
${\mathbf{P}}^{d-1}$. 


\medskip\noindent
5.  Learn about separability and $\Omega^1$ by reading in Matsumura's CRT:
\S25 up to before 25.3 (this is better than AG15.1--15.8 in Borel's book), and read 
 \S26 up through and including Theorem 26.3. 
 
 (i) Do Exercises 25.3, 25.4 in Matsumura, and read AG17.1 in Borel's book (noting he
 requires $V$ to be geometrically reduced over $k$!).  
 
 (ii) Use 26.2 in Matsumura to prove that 
 a finite type reduced $k$-scheme $X$ is smooth on
 a dense open if and only if all functions fields of $X$ (at its generic points)
 are {\em separable} over $k$.
 
 (iii) Using separating transcendence bases, the primitive element theorem,
  and ``denominator chasing'',
 prove that if $X$ is smooth on a dense open then $X(k_s)$ is Zariski-dense in
 $X_{k_s}$.  (Hint: it suffices to prove $X(k_s)$ is non-empty!)

\newpage

\centerline{\sc Algebraic Groups I. Homework 3}

\noindent
1.  Let $k[x_{ij}]$ be the polynomial ring in 
variables $x_{ij}$ with $1 \le i, j \le n$.
Observe that the localization $k[x_{ij}]_{\det}$ has a natural $\mathbf{Z}$-grading, since
$\det \in k[x_{ij}]$ is homogeneous.  Let $k[x_{ij}]_{(\det)}$ denote the degree-0 part
(i.e., fractions $f/\det^e$ with $f$ homogenous of degree $e\deg(\det) = e n$, for $e \ge 0$).

(i) Define ${\rm{PGL}}_n = \Spec(k[x_{ij}]_{(\det)})$.
Identify this with the open affine $\{\det \ne 0\}$ in $\mathbf{P}^{n^2-1}$, and 
construct an injective map of sets 
${\rm{GL}}_n(R)/R^{\times} \rightarrow
{\rm{PGL}}_n(R) := \Hom_k(\Spec R, {\rm{PGL}}_n)$
naturally in $k$-algebras $R$.

(ii) For any $R$ and any $m \in {\rm{PGL}}_n(R)$, show that there is an affine open covering
$\{\Spec R_i\}$ of $\Spec R$ such that $m|_{R_i} \in {\rm{GL}}_n(R_i)/R_i^{\times}$.
Deduce that ${\rm{PGL}}_n(R)$ is the {\em sheafification} of the presheaf
$U \mapsto {\rm{GL}}_n(U)/{\rm{GL}}_1(U)$ on $\Spec U$, and that
${\rm{PGL}}_n$ has a unique $k$-group structure such that
${\rm{GL}}_n \rightarrow {\rm{PGL}}_n$ is a $k$-homomorphism. 

(iii) Prove that if $R$ is {\em local} then ${\rm{GL}}_n(R)/R^{\times} = 
{\rm{PGL}}_n(R)$, and construct a {\em counterexample} with $n = 2$
for any Dedekind domain $R$ whose class group has nontrivial 2-torsion.
(Hint: $I \oplus I \simeq R^2$ when $I$ is 2-torsion.) 

(iv) Write out the effect of multiplication and inversion on ${\rm{PGL}}_n$ at the level of coordinate rings.

\medskip\noindent
2.  The {\em scheme-theoretic kernel} of a $k$-homomorphism
$f:G' \rightarrow G$ between $k$-group schemes is the scheme-theoretic fiber $f^{-1}(e)$
(with $e:\Spec k \rightarrow G$ the identity). It is denoted $\ker f$.

(i) Prove that if $R$ is any $k$-algebra then $(\ker f)(R) = \ker(G'(R) \rightarrow G(R))$
as subgroups of $G'(R)$; deduce that $\ker f$ is a normal $k$-subgroup of $G'$. 

(ii) Prove that the homomorphism ${\rm{GL}}_n \rightarrow {\rm{PGL}}_n$
constructed in Exercise 1 is surjective with scheme-theoretic kernel equal to the $k$-subgroup
$D \simeq {\rm{GL}}_1$ of scalar diagonal matrices.

(iii) Let $\mu_n = \ker(t^n:\mathbf{G}_m \rightarrow \mathbf{G}_m) = 
\Spec(k[t,1/t]/(t^n - 1))$.  Identify $\mu_n(R)$ with the group of $n$th roots of unity
in $R^{\times}$ naturally in any $k$-algebra $R$, and 
prove that the homomorphism ${\rm{SL}}_n \rightarrow {\rm{PGL}}_n$
obtained by restriction of the map in (ii) to ${\rm{SL}}_n$ is surjective,
with kernel $\mu_n$.

\medskip\noindent
3.  Let $G$ be a $k$-group of finite type equipped with an action on 
$k$-scheme $V$ of finite type.   Let $W, W' \subseteq V$ be closed
subschemes.   Define the {\em functorial centralizer}
$\underline{Z}_G(W)$ and {\em functorial transporter}
$\underline{\rm{Tran}}_G(W,W')$ as follows:  for any $k$-scheme $S$,
$\underline{Z}_G(W)(S)$ is the subgroup of points $g \in G(S)$
such that the $g$-action on $V_S$ is trivial, 
and $\underline{\rm{Tran}}_G(W,W')(S)$ is the subset of points $g \in G(S)$
such that $g.(W_S) \subseteq W'_S$ (as closed subschemes of
$V_S$).   The {\em functorial normalizer}
$\underline{N}_G(W)$ is $\underline{\rm{Tran}}_G(W,W)$.

These are of most interest when $W$ is a smooth closed $k$-subgroup of $V = G$
equipped with the left translation action.  Below, assume $W$ is {\em geometrically reduced} and {\em separated} over $k$.

(i) Prove $W$ is smooth on a dense open, so $W(k_s)$ is Zariski-dense in $W_{k_s}$
(by Exercise 5(iii), HW2).  Hint: if $k = k_s$ then $W_{\overline{k}} \rightarrow W$ is a homeomorphism,
and in general use Galois descent (as in Exercise 3(iii), HW2).

(ii) For each $w \in W(k)$, let $\alpha_w:G \rightarrow W$ be the orbit map $g \mapsto g.w$.
Define $Z_G(w) = \alpha_w^{-1}(w)$. 
Prove that $Z_G(w)(S)$ is the subgroup of points $g \in G(S)$
such that $g.w_S = w_S$ in $W(S)$.

(iii) If $k = k_s$ prove $\cap_{w \in W(k)} Z_G(w)$ represents
$\underline{Z}_G(W)$.  (You need to use separatedness.) For general $k$ apply Galois descent 
to $Z_{G_{k_s}}(W_{k_s})$; the representing scheme is denoted $Z_G(W)$. 

(iv) If $k = k_s$, prove that 
$\cap_{w \in W(k)} \alpha_w^{-1}(W')$ represents $\underline{{\rm{Tran}}}_G(W,W')$.  Then use Galois
descent to prove representability by a closed subscheme ${\rm{Tran}}_G(W,W')$ for any $k$.
The representing scheme is denoted ${\rm{Tran}}_G(W,W)$,
so $N_G(W) := {\rm{Tran}}_G(W,W)$ represents $\underline{N}_G(W)$.  

(v) Prove that for any $k$-algebra $R$ and $g \in N_G(W)(R)$,
the $g$-action $V_R \simeq V_R$ carries $W_R$ {\em isomorphically} onto itself,
and deduce that $N_G(W)$ is a $k$-subgroup of $G$. (Hint: reduce to artin local $R$
and $k = \overline{k}$.) 

\medskip\noindent
4.  Let $G$ be a $k$-group of finite type.  This exercise builds on the previous one.  
Note $G$ is separated:  $\Delta_{G/k}$ is a base change of $e:\Spec k \rightarrow G$!
If $G$ is smooth then the {\em scheme-theoretic center} of $G$ is $Z_G := Z_G(G)$.

(i) Let $G$ be ${\rm{SL}}_n$ or ${\rm{GL}}_n$ or ${\rm{PGL}}_n$, and let
$T$ be the diagonal $k$-torus in each case.
Prove that $Z_G(T) = T$ (as subschemes of $G$, not just at the level of geometric points!). 
Hint: to deduce the ${\rm{PGL}}_n$-case from the ${\rm{GL}}_n$-case, prove that
the diagonal $k$-torus in ${\rm{GL}}_n$ is the scheme-theoretic preimage of the
one in ${\rm{PGL}}_n$. 

(ii) Using (i), prove $Z_{{\rm{SL}}_n} = \mu_n$, $Z_{{\rm{PGL}}_n} = 1$, and 
$Z_{{\rm{GL}}_n}$ is the $k$-subgroup of scalar diagonal matrices.

(iii) Prove that for a smooth closed subscheme $V$ in $G$, the formation of $Z_G(V)$ 
and $N_G(V)$ commutes with any extension of the ground field.  (Hint: use the functorial characterizations,
not the explicit constructions.)  This applies to $Z_G$ when $G$ is smooth.  

\newpage

\centerline{\sc Algebraic Groups I. Homework 4}
\medskip\medskip

\medskip\noindent
1.  Let $T \subset {\rm{Sp}}_{2n}$ be the
points $\left(\begin{smallmatrix} t & 0 \\ 0 & t^{-1} \end{smallmatrix}\right)$
for diagonal $t \in {\rm{GL}}_n$.  Prove $Z_G(T) = T$ (so $T$ is a maximal torus!);
deduce $Z_{{\rm{Sp}}_{2n}} = \mu_2$.
The Appendix ``Properties of orthogonal groups'' computes $Z_{{\rm{SO}}(q)}$ (see Theorem 1.7). 


\medskip\noindent
2.  Prove that ${\rm{PGL}}_n$ is smooth using the infinitesimal criterion, and prove that it
is connected by a suitable ``action'' argument.
The Appendix ``Properties of orthogonal groups'' treats the harder analogue for ${\rm{SO}}(q)$.



\medskip\noindent
3.  Let $X$ be a scheme over a field $k$, and $x \in X(k)$.
Recall that ${\rm{Tan}}_x(X)$ is identified as a set with the fiber of
$X(k[\epsilon]) \rightarrow X(k)$ over $x$. 
Let $k[\epsilon,\epsilon'] = k[t,t']/(t,t')^2$, so this is 3-dimensional with basis $\{1, \epsilon, \epsilon'\}$.

(i)  For $c \in k$, consider the $k$-algebra endomorphism of $k[\epsilon]$
defined by $\epsilon \mapsto c \epsilon$.  Show that the resulting endomorphism of
$X(k[\epsilon])$ over $X(k)$ restricts to scalar multiplication by $c$ on the fiber
${\rm{Tan}}_x(X)$. 

(ii) Using the two natural quotient maps $k[\epsilon, \epsilon'] \twoheadrightarrow k[\epsilon]$, 
define a natural map 
$$X(k[\epsilon, \epsilon']) \rightarrow
X(k[\epsilon]) \times_{X(k)} X(k[\epsilon])$$
and prove it is bijective.  Using the natural quotient map $k[\epsilon, \epsilon'] \twoheadrightarrow
k[\epsilon]$, show that the resulting map
$$X(k[\epsilon]) \times_{X(k)} X(k[\epsilon]) \stackrel{\simeq}{\leftarrow} 
X(k[\epsilon, \epsilon']) \rightarrow X(k[\epsilon])$$
induces addition on ${\rm{Tan}}_x(X)$: 
the $k$-linear structure on ${\rm{Tan}}_x(X)$ is encoded by the functor of $X$!

(iii) For $(X,x) = (G,e)$ with a $k$-group $G$, relate addition on ${\rm{Tan}}_x(X)$ to the group 
law on $G$:  for $m:G \times G \rightarrow G$, show that
${\rm{Tan}}_e(G) \times {\rm{Tan}}_e(G) =
{\rm{Tan}}_{(e,e)}(G \times G) \rightarrow {\rm{Tan}}_e(G)$ 
is addition. 

\medskip\noindent
4.  Let $A$ be a finite-dimensional associative algebra over a field $k$.
Define the ring functor 
$\underline{A}$ on $k$-algebras by 
$\underline{A}(R) = A \otimes_k R$ and the group functor
$\underline{A}^{\times}$ by $\underline{A}^{\times}(R) = (A \otimes_k R)^{\times}$.  

(i) Prove that $\underline{A}$ is represented by an affine space over $k$.
Using the $k$-scheme map ${\rm{N}}_{A/k}:
\underline{A} \rightarrow \mathbf{A}^1_k$ defined functorially by
$u \mapsto \det(m_u)$, where $m_u:A \otimes_k R \rightarrow A \otimes_k R$ is left multiplication
by $u \in \underline{A}(R)$, prove that
$\underline{A}^{\times}$ is represented by the open {\em affine} subscheme 
${\rm{N}}_{A/k}^{-1}(\mathbf{G}_m)$.  (This is often called
``$A^{\times}$ viewed as a $k$-group'', a phrase that
is, strictly speaking, meaningless,  since $A^{\times}$ does not encode the $k$-algebra $A$.) 

(ii) For $A = {\rm{Mat}}_n(k)$ show that
$\underline{A}^{\times} = {\rm{GL}}_n$, and for $k = \mathbf{Q}$ and $A = \mathbf{Q}(\sqrt{d})$ identify it with
an explicit $\mathbf{Q}$-subgroup of ${\rm{GL}}_2$ (depending on $d$).

(iii)  How does the kernel of ${\rm{N}}_{A/k}:\underline{A}^{\times} \rightarrow
\mathbf{G}_m$ (the {\em group of norm-$1$ units}) relate to Exercise 4(iii) in HW1 as a special case?
For $A = {\rm{Mat}}_n(k)$, show that this homomorphism is the $n$th power (!)
of the determinant.  

\medskip\noindent
5.  This exercise develops a very important special case of Exercise 4.
Let $A$ be a finite-dimensional central simple algebra over $k$.  By general theory, this is exactly
the condition that $A_{\overline{k}} \simeq {\rm{Mat}}_n(\overline{k})$ 
as $\overline{k}$-algebras (for some $n \ge 1$), and such an isomorphism is unique up to 
conjugation by a unit (Skolem-Noether theorem).  

(i) By a clever application of the Skolem-Noether theorem (see Exercise 30, Chapter 3 of the
book by Farb/Dennis on non-commutative algebra), it is a classical fact that
the linear derivations of a matrix algebra over a field are  precisely the inner derivations
(i.e., $x \mapsto yx - xy$ for some $y$).  Combining this with length-induction on artin local rings, prove
the Skolem-Noether theorem for ${\rm{Mat}}_n(R)$ for any artin local ring $R$
(i.e., all $R$-algebra automorphisms are conjugation by a unit).  

(ii) Construct an affine $k$-scheme $I$ of finite type such that naturally 
$I(R) = {\rm{Isom}}_R(A_R, {\rm{Mat}}_n(R))$, the set of $R$-algebra isomorphisms.
Note that $I(\overline{k})$ is non-empty!  
Prove $I$ is smooth by checking the infinitesimal criterion for $I_{\overline{k}}$ with the help of (i).
Deduce that $A_K \simeq {\rm{Mat}}_n(K)$ for a finite {\em separable} extension $K/k$.  

(iii) By (ii), we can choose a finite Galois extension $K/k$ and a $K$-algebra isomorphism
$\theta:A_K \simeq {\rm{Mat}}_n(K)$, and by Skolem-Noether this is unique up to conjugation by a unit.
Prove that for any choice of $\theta$, the determinant map
transfers to a multiplicative map $\underline{A}_K \rightarrow \mathbf{A}^1_K$
which is independent of $\theta$.  Deduce that it is ${\rm{Gal}}(K/k)$-equivariant,
and so descends to a multiplicative map ${\rm{Nrd}}_{A/k}:\underline{A} \rightarrow \mathbf{A}^1_k$
which ``becomes'' the determinant over {\em any} extension $F/k$ for which $A_F \simeq
{\rm{Mat}}_n(F)$.  Prove that ${\rm{Nrd}}_{A/k}^n = {\rm{N}}_{A/k}$ 
(explaining the name {\em reduced norm}
for ${\rm{Nrd}}_{A/k}$), and conclude that $\underline{A}^{\times} = 
{\rm{Nrd}}_{A/k}^{-1}(\mathbf{G}_m)$.  

(iv)  Let ${\rm{SL}}(A)$ denote the scheme-theoretic kernel of 
${\rm{Nrd}}_{A/k}:\underline{A}^{\times} \rightarrow \mathbf{G}_m$.  Prove that its formation
commutes with any extension of the ground field, and that it 
becomes isomorphic to ${\rm{SL}}_n$ over $\overline{k}$.  In particular,
${\rm{SL}}(A)$ is {\em smooth} and {\em connected}; it is a ``twisted form'' of
${\rm{SL}}_n$.  (This is false for $\ker {\rm{N}}_{A/k}$
whenever ${\rm{char}}(k)|n$!)  

\newpage

\centerline{\sc Algebraic Groups I. Homework 5}

1.  Let $k$ be a field, $U_n$ the standard strictly upper-triangular unipotent $k$-subgroup of
${\rm{GL}}_n$.   Prove that no nontrivial $k$-group scheme is isomorphic to closed $k$-subgroups of
$\mathbf{G}_a$ and ${\mathbf{G}}_m$.  (If ${\rm{char}}(k) = p > 0$, the key is to prove that $\mu_p$ is not a $k$-subgroup of
$\mathbf{G}_a$.)
Deduce that $T \cap U_n = 1$ for any $k$-torus $T$ in ${\rm{GL}}_n$.

\medskip\noindent
2. Let a smooth finite type $k$-group $G$ act
linearly on a finite-dimensional $V$.  Let $\underline{V}$ denote the affine space
whose $A$-points are $V_A$.  Define
$\underline{V}^G(A)$ to be the set of
$v \in V_A$ on which $G_A$ acts trivially. 

(i) Prove that $\underline{V}^G$ is represented by the closed subscheme 
associated to a $k$-subspace of $V$ (denoted of course as $V^G$).  Hint:  use Galois descent to reduce
to the case $k = k_s$, and then show $V^{G(k)}$ works.

(ii) For an extension field $K/k$, prove that $(V_K)^{G_K} = (V^G)_K$ inside of $V_K$. 

\medskip\noindent
3.  This exercise develops the important concept of {\em Weil restriction of scalars} in the affine
case.  It is an analogue
of viewing a complex manifold as a real manifold with twice the dimension (and ``complex points''
become ``real points'').  Let $k$ be a field, $k'$ a finite commutative
$k$-algebra ({\em not} necessarily a field!), and $X'$ an affine $k'$-scheme of finite
type.  Consider the functor ${\rm{R}}_{k'/k}(X'):A \rightsquigarrow X'(k' \otimes_k A)$ on $k$-algebras. 

(i) By considering $X' = \mathbf{A}^n_{k'}$ and then any $X'$ via a closed
immersion into an affine space, prove that this functor is represented by an affine
$k$-scheme of finite type, again denoted ${\rm{R}}_{k'/k}(X')$.  Prove its formation
naturally commutes with products in $X'$, and compute
${\rm{R}}_{k'/k}(\mathbf{G}_m)$ inside ${\rm{R}}_{k'/k}(\mathbf{A}^1_{k'})$.  What if $k' = 0$?  


(ii) Prove ${\rm{R}}_{k'/k}(\Spec k') = \Spec k$, and explain why ${\rm{R}}_{k'/k}(X')$ is naturally
a $k$-group when $X'$ is a $k'$-group.  

(iii) For an extension field $K/k$, prove that ${\rm{R}}_{k'/k}(X')_K \simeq
{\rm{R}}_{K'/K}(X'_{K'})$ for $K' = k' \otimes_k K$.  Taking $K = \overline{k}$, use the infinitesimal
criterion to prove that if $k'$ is a field then ${\rm{R}}_{k'/k}(X')$ is $k$-smooth when $X'$ is $k'$-smooth.  
(Can you see it directly from the construction?)   Warning: if $k'/k$ is not separable then
${\rm{R}}_{k'/k}(X')$ can be empty (resp.\:disconnected) when $X'$ is non-empty (resp.\:geometrically
integral)!

(iv) If $k'/k$ is a separable extension field, prove 
${\rm{R}}_{k'/k}(X')_{k_s} \simeq \prod_{\sigma} \sigma^{\ast}(X')$ with $\sigma$ varying through
${\rm{Hom}}_k(k', k_s)$.   Transfer the natural
${\rm{Gal}}(k_s/k)$-action on the left over to the right and describe it. 

\medskip\noindent
4.  Let $\Gamma = {\rm{Gal}}(k_s/k)$.
For any $k$-torus $T$, define the {\em character group}
${\rm{X}}(T) = {\rm{Hom}}_{k_s}(T_{k_s}, \mathbf{G}_m)$.
A {\em $\Gamma$-lattice} is a finite
free $\mathbf{Z}$-module equipped with a $\Gamma$-action making an open subgroup act trivially. 

(i) Prove ${\rm{X}}(T)$ is a finite free $\mathbf{Z}$-module of rank $\dim T$.  Describe a natural
$\Gamma$-lattice structure on ${\rm{X}}(T)$. 

(ii) For a $\Gamma$-lattice $\Lambda$, prove
$R \rightsquigarrow {\rm{Hom}}(\Lambda, R_{k_s}^{\times})^{\Gamma}$
is represented by a $k$-torus ${\rm{D}}_k(\Lambda)$, the
{\em dual} of $\Lambda$.   (Hint: use finite Galois descent to reduce
to $\Lambda$ with trivial $\Gamma$-action.) Prove
$\Lambda \simeq {\rm{X}}({\rm{D}}_k(\Lambda))$ naturally as $\Gamma$-lattices.

(iii) Prove $T \simeq {\rm{D}}_k({\rm{X}}(T))$ naturally as $k$-tori, so 
the category of $k$-tori is anti-equivalent to the category of $\Gamma$-lattices. Describe
scalar extension in such terms, and prove $T$ is $k$-split if and only if
${\rm{X}}(T) = {\rm{X}}(T)^{\Gamma}$. 

(iv) Prove a map of
$k$-tori $T' \rightarrow T$ is surjective if and only if
${\rm{X}}(T) \rightarrow {\rm{X}}(T')$ is injective. 
Prove $\ker(T' \rightarrow T)$ is a $k$-torus (resp.\:finite, resp.\:0) if and only if $\Coker(X(T) \rightarrow
X(T'))$ is torsion-free (resp.\:finite, resp.\:0).  Inducting on $\dim T$, prove
smooth {\em connected} $k$-subgroups $M$ of $T$ are $k$-tori. (Hint: prove
$M(\overline{k})$ is divisible.) 

(v) If $k'/k$ is a finite separable subextension of $k_s$, prove that
${\rm{R}}_{k'/k}(T')$ is a $k$-torus if $T'$ is a $k'$-torus. (For $T' = \mathbf{G}_m$, this is
``${k'}^{\times}$ viewed as a $k$-group''.)  
By functorial considerations,  
prove ${\rm{X}}({\rm{R}}_{k'/k}(T')) = {\rm{Ind}}_{\Gamma'}^{\Gamma}({\rm{X}}(T))$
with $\Gamma'$ the open subgroup corresponding to $k'$.
For every $k$-torus $T$, construct a surjective $k$-homomorphism
$\prod_i {\rm{Res}}_{k'_i/k}({\mathbf{G}}_m) \twoheadrightarrow T$ 
for finite separable extensions $k'_i/k$.
Conclude that $k$-tori are {\em unirational} over $k$. 

(vi) (optional) For a finite extension field $k'/k$, define a {\em norm} map
${\rm{N}}_{k'/k}:{\rm{R}}_{k'/k}(\mathbf{G}_m) \rightarrow {\mathbf{G}}_m$. Prove
its kernel is a torus when $k'/k$ is separable (e.g., $k = \mathbf{R}$!),
and relate to HW1, Exercise 4(iii) for imperfect $k$. 

\medskip\noindent
5.  Consider a $k$-torus $T \subset {\rm{GL}}(V)$, with $k$ infinite. 
Let $A_T \subset {\rm{End}}(V)$ be the commutative $k$-subalgebra generated by $T(k)$
(Zariski-dense in $T$ since $k$ is infinite, due to unirationality from Exercise 4(iv)). 

(i) Using Jordan decomposition, prove that all elements of $T(\overline{k})$ are semisimple in 
${\rm{End}}(V_{\overline{k}})$. 

(ii) Assume $k = k_s$.  Prove $A_T$ is a product of copies of $k$, and 
 $T(k) = A_T^{\times}$ when $T$ is maximal.

(iii) Using Galois descent and the end of  4(v), prove
$(A_T)_{k_s} = A_{T_{k_s}}$, and deduce 
$T(k) = A_T^{\times}$ for maximal $T$. Show naturally
$T \simeq {\rm{Res}}_{A_T/k}(\mathbf{G}_m)$, and that 
maximal $k$-subtori in ${\rm{GL}}(V)$ and maximal \'etale commutative $k$-subalgebras
of ${\rm{End}}(V)$ are in bijective correspondence.  Generalize to {\em finite} $k$ with another definition of 
$A_T$, and to central simple algebras  in place of ${\rm{End}}(V)$ (hint: use HW4 Exercise 5(ii) and Galois descent).

(iv) For any (possibly finite) $k$, prove a smooth connected {\em commutative} $k$-group is a torus if and only if 
its $\overline{k}$-points are semisimple.  (Use the end of Exercise 4(iv).) 

\newpage

\centerline{\sc Algebraic Groups I. Homework 6}

\medskip\noindent
1.  Use the method of proof of Proposition 4.10, Chapter I, to prove the following scheme-theoretic version:
if $k$ is a field and a smooth unipotent affine $k$-group $G$ is equipped with a left
action on a quasi-affine $k$-scheme $V$ of finite type then for any $v \in V(k)$
the smooth locally closed image of the orbit map $G \rightarrow V$ defined by $g \mapsto gv$ is
actually closed in $V$.  

(Hint: to begin, let $k[V]$ denote the $k$-algebra of global functions on $V$
and prove that $R \otimes_k k[V]$ is the $R$-algebra of global functions on $V_R$ for any $k$-algebra $R$.
Use this to construct a functorial $k$-linear representation of $G$ on $k[V]$ respecting the $k$-algebra structure.
Borel's $K$ should be replaced with $k$ after passing to the case
$k = \overline{k}$.  Note that it is not necessary to assume Borel's $F$ is non-empty; the argument directly proves $J$ meets
$k^{\times}$, so $J = (1)$ and hence $F$ is empty.)

\medskip\noindent
2.  A $k$-homomorphism $f:G' \rightarrow G$ between $k$-groups of finite type is an {\em isogeny} if
it is surjective and flat with finite kernel.

(i) Prove that a surjective homomorphism between smooth finite type $k$-groups of the same dimension is an isogeny.
(The Miracle Flatness Theorem will be useful here.)

(ii) Prove that a map $f:T' \rightarrow T$ between $k$-tori is an isogeny if and only if the corresponding map
${\rm{X}}(T) \rightarrow {\rm{X}}(T')$ between Galois lattices is injective with finite cokernel.

(iii) Prove the following are equivalent for a $k$-torus $T$: (a) it contains ${\mathbf{G}}_m$ as a $k$-subgroup, 
(b) there exists a surjective $k$-homomorphism $T \twoheadrightarrow \mathbf{G}_m$, and (c) ${\rm{X}}(T)_{\mathbf{Q}}$ has a nonzero
${\rm{Gal}}(k_s/k)$-invariant vector.   Such $T$ are called {\em $k$-isotropic}; otherwise we say $T$ is
{\em $k$-anisotropic}.  In general, a smooth affine $k$-group is called {\em $k$-isotropic} if it contains
${\mathbf{G}}_m$ as a $k$-subgroup, and {\em $k$-anisotropic} otherwise. 

(iv)  Let $T$ be a $k$-torus.  Prove the existence of a $k$-split $k$-subtorus $T_s$ that contains all others, as well as
a $k$-anisotropic $k$-subtorus $T_a$ that contains all others.  Also prove that $T_s \times T_a \rightarrow T$ is an isogeny.
Compute $T_s$ and $T_a$ for $T = {\rm{R}}_{k'/k}(\mathbf{G}_m)$ for a finite separable extension $k'/k$.

\medskip\noindent
3. (i) For a $k$-torus $T$,
prove the existence of an \'etale $k$-group ${\rm{Aut}}_{T/k}$
representing the automorphism functor $S \rightsquigarrow
 {\rm{Aut}}_S(T_S)$.  (Hint:  if $T$ is $k$-split then
show that the constant $k$-group associated to ${\rm{Aut}}({\rm{X}}(T)) \simeq {\rm{GL}}_r(\mathbf{Z})$ does the job.
In general let $k'/k$ be finite Galois such that $T_{k'}$ is $k'$-split, and use Galois descent.)

(ii) Using the existence of the \'etale $k$-group ${\rm{Aut}}_{T/k}$, prove that if a connected $k$-group scheme
$G$ is equipped with an action on $T$ then the action must be trivial.  Deduce that if $T$ is a normal $k$-subgroup of
a connected finite type $k$-group $G$ then it is a central $k$-subgroup.  
Give an example of a smooth connected $k$-group containing $\mathbf{G}_a$ as a {\em non-central} normal $k$-subgroup.
(Hint: look inside ${\rm{SL}}_2$.) 

\medskip\noindent
4. Let $T$ be a $k$-torus in a $k$-group $G$ of finite type.  This exercise uses ${\rm{Aut}}_{T/k}$ from Exercise 3. 

(i) Construct a $k$-morphism $N_G(T) \rightarrow {\rm{Aut}}_{T/k}$ with kernel $Z_G(T)$.
Prove $W(G,T) := 
N_G(T)(\overline{k})/Z_G(T)(\overline{k})$ is naturally a {\em finite} subgroup of
${\rm{Aut}}_{\mathbf{Z}}({\rm{X}}(T))$.  If $f:G' \rightarrow G$ is surjective with finite kernel and $T'$ is a $k$-torus in 
$G'$ containing $\ker f$ with $f(T') = T$ then prove 
$W(G',T') \rightarrow W(G,T)$ is an isomorphism. 

(ii) For $G = {\rm{GL}}_n, {\rm{PGL}}_n, {\rm{SL}}_n, {\rm{Sp}}_{2n}$ and $T$ the $k$-split diagonal maximal $k$-torus
(so $Z_G(T) = T$), respectively identify
${\rm{X}}(T)$ with $\mathbf{Z}^n$, $\mathbf{Z}^n/{\rm{diag}}$, $\{m \in \mathbf{Z}^n\,|\,\sum m_j = 0\}$,
and $\mathbf{Z}^n$.  Prove $N_G(T)(k)/Z_G(T)(k) \subset
{\rm{Aut}}_{\mathbf{Q}}({\rm{X}}(T)_{\mathbf{Q}})$ is $S_n$ for the first three, and
$S_n \ltimes \langle -1 \rangle^n$ for ${\rm{Sp}}_{2n}$, all with natural action. 
(Hint: to control $N_G(T)$, via 
$G \hookrightarrow {\rm{GL}}(V)$ decompose $V$ as a direct sum of
$T$-stable lines with {\em distinct} eigencharacters.)

\medskip\noindent
5. Let $(V,q)$ be a non-degenerate quadratic space over a field $k$ with $\dim V \ge 2$. This exercise proves 
${\rm{SO}}(q)$ contains ${\mathbf{G}}_m$ (i.e., it is $k$-isotropic in the sense of Exercise 2(iii)) if and only if $q = 0$ has a solution in $V - \{0\}$. 

(i) If $q = 0$ has a nonzero solution $v$ in $V$, prove that $v$ lies in a hyperbolic plane $H$ with $H \oplus H^{\perp} = V$.
(If ${\rm{char}}(k) = 2$ and $\dim V$ is odd, work over $\overline{k}$ to show $v \not\in V^{\perp}$.)  
Use this to construct a $\mathbf{G}_m$ inside of ${\rm{SO}}(q)$.

(ii) If ${\rm{SO}}(q)$ contains ${\mathbf{G}}_m$ as a $k$-subgroup $S$, prove that $q = 0$ has a nonzero solution in $V$.
(Hint: apply Exercise 5(iii) in HW5 to the 2-dimensional $k$-split $k$-torus $T$ generated in ${\rm{GL}}(V)$ by $S$ 
and the central ${\mathbf{G}}_m$.  If $A \simeq k \times k$ is the corresponding ``$k$-split'' commutative $k$-subalgebra of
${\rm{End}}(V)$, prove the resulting inclusion $\mathbf{G}_m = S \hookrightarrow T = {\rm{R}}_{A/k}(\mathbf{G}_m) = 
\mathbf{G}_m \times \mathbf{G}_m$ is $t \mapsto (t^n, t^{n'})$ with $n \ne n'$.  Use the $A$-module structure on $V$
to find a $k$-basis $\{e_i\}$ that identifies $S$ with ${\rm{diag}}(t^{n_1}, \dots, t^{n_d})$ for $n_1 \le \dots \le n_d$ with $\sum n_i = 0$.
Prove $n_1 < 0 < n_d$, and if 
$q = \sum_{i \le j} a_{ij} x_i x_j$ in these coordinates then prove $n_i + n_j = 0$ when $a_{ij} \ne 0$.
Deduce $q(v) = 0$ for any $v$ in the span of the $e_i$ for which $n_i < 0$, or for which $n_i > 0$.) 

\newpage

\centerline{\sc Algebraic Groups I. Homework 7}

\noindent
0. (optional)  Read the 
proof (p. 101 in Mumford's ``Abelian Varieties'') of 
{\em Cartier's theorem}:  group schemes $G$ locally of finite type over a field of characteristic 0 are smooth!
(This uses the left-invariant derivations.)

\medskip\noindent
1. (i) Prove that $\partial_x$ is an invariant vector field on ${\mathbf{G}}_a$,
and $t^{-1} \partial_t$ is an invariant vector field on $\mathbf{G}_m$.

(ii) Let $A$ be a finite-dimensional associative $k$-algebra,
and $\underline{A}^{\times}$ the associated $k$-group of units. 
Prove ${\rm{Tan}}_e(\underline{A}^{\times}) = A$ naturally, and 
that the Lie algebra structure is then $[a,a'] = aa' - a'a$.  Using 
$A = {\rm{End}}(V)$, compute $\mathfrak{gl}(V)$.  Use this to compute the Lie algebras
$\mathfrak{sl}(V), \mathfrak{pgl}(V), \mathfrak{sp}(B), \mathfrak{gsp}(B), \mathfrak{so}(q)$.

(iii) Read Corollary A.7.6 and Lemma A.7.13 (and the paragraph preceding it)
in the book {\em Pseudo-reductive groups}.  Compute the $p$-Lie algebra structure on
${\rm{Lie}}(\underline{A}^{\times})$, ${\rm{Lie}}(\mathbf{G}_m)$,
and ${\rm{Lie}}(\mathbf{G}_a)$ if ${\rm{char}}(k) = p > 0$. 

\medskip\noindent
2. Let $G$ be a smooth group of dimension $d > 0$ over $k$.

(i) Define the concept of {\em left-invariant} differential $i$-form for $i \ge 0$,
and prove the space $\Omega^{i,\ell}_G(G)$ of such form has dimension $d\choose{i}$.  
Compute the 1-dimensional $\Omega^{d,\ell}_G(G)$ for ${\rm{GL}}(V)$, ${\rm{SL}}(V)$, and
${\rm{PGL}}(V)$.

(ii)  Using right-translation, construct a linear representation of
$G$ on $\Omega^{d,\ell}_G(G)$;
the associated character $\chi_G:G \rightarrow \mathbf{G}_m$ 
is the {\em modulus
character}.  Prove $\chi_G|_{Z_G} = 1$ and deduce that $\chi_G = 1$ if $G/Z_G = 
\mathcal{D}(G/Z_G)$.

(iii) (optional) If $k$ is local (allow $\mathbf{R}$, $\mathbf{C}$) and $X$ is smooth, 
use the $k$-analytic inverse function theorem to equip $X(k)$ with a functorial
$k$-analytic manifold structure, and use $k$-analytic
Change of Variables to assign a measure on $X(k)$ to a nowhere-vanishing 
$\omega \in \Omega^{\dim X}_X(X)$.  (Serre's ``Lie groups and Lie algebras''
does $k$-analytic foundations.) Relate with Haar measures, and prove
$\chi_G^{\pm 1}|_{G(k)}$ {\em is} the classical modulus character.

\medskip\noindent
3.  Let $K/k$ be a degree-2 finite \'etale algebra (i.e., a separable quadratic field extension or
$k \times k$), and let $\sigma$ be the unique non-trivial $k$-automorphism of $K$; note that
$K^{\sigma} = k$.
A {\em $\sigma$-hermitian space} is a pair $(V,h)$ consisting of a finite free $K$-module
equipped with a perfect $\sigma$-semilinear form $h:V \times V \rightarrow K$
(i.e., $h(cv,v') = c h(v,v')$, $h(v,cv') = \sigma h(v,v')$, and $h(v',v) = \sigma(h(v,v'))$). 
Note $v \mapsto h(v,v)$ is a quadratic form $q_h:V \rightarrow k$ over $k$ satisfying
$q_h(cv) = {\rm{N}}_{K/k}(c) q_h(v)$ for $c \in K$, $v \in V$, and $\dim_k V$ is even
(${\rm{char}}(k) = 2$ ok!). 

The {\em unitary group} ${\rm{U}}(h)$ over $k$  is the subgroup of
${\rm{R}}_{K/k}({\rm{GL}}(V))$ preserving $h$. 
Using ${\rm{R}}_{K/k}({\rm{SL}}(V))$ gives the {\em special unitary group}
${\rm{SU}}(h)$.  Example: $V = F$ finite \'etale over $K$ 
with an involution $\sigma'$ lifting $\sigma$, 
and $h(v,v') := {\rm{Tr}}_{F/K}(v\sigma'(v'))$; e.g., $F$ and $K$ CM fields, $k$ totally real,
and complex conjugations $\sigma'$ and $\sigma$.

(i) If $K = k \times k$, prove $V \simeq V_0 \times V_0^{\vee}$ with $h((v, \ell),(v',\ell')) = 
(\ell'(v),\ell(v'))$ for a $k$-vector space $V_0$.
Identify ${\rm{U}}(h)$ with ${\rm{GL}}(V_0)$ carrying ${\rm{SU}}(h)$ to ${\rm{SL}}(V_0)$.
Compute $q_h$ and prove non-degeneracy. 

(ii) In the non-split case prove that ${\rm{U}}(h)_K \simeq {\rm{GL}}_n$ carrying
${\rm{SU}}(h)$ to ${\rm{SL}}_n$ ($n = \dim_K V$). 
Prove ${\rm{U}}(h)$ is smooth and connected
with derived group ${\rm{SU}}(h)$ and center $\mathbf{G}_m$, and $q_h$ is non-degenerate.
Compute $\mathfrak{su}(h)$.

(iii) Identify ${\rm{U}}(h)$ with a $k$-subgroup of ${\rm{SO}}(q_h)$.  Discuss the split case, and 
all cases with $k = \mathbf{R}$. 

\medskip\noindent
4. Let a smooth $k$-group $H$ act on a separated $k$-scheme $Y$.
For a $k$-scheme $S$, let 
$Y^H(S)$ be the set of $y \in Y(S)$ invariant by the $H_S$-action on $Y_S$
(i.e., $y_{S'}$ is $H(S')$-invariant for all $S$-schemes $S'$).

(i) If $k = k_s$, prove $Y^H$ is represented by 
the closed subscheme $\cap_{h \in H(k)} Y^h$
where $Y^h = \alpha_h^{-1}(\Delta_{Y/k})$ for $\alpha_h:Y \rightarrow Y \times Y$ the map $y \mapsto (y, h.y)$.
Then prove representability by a closed subscheme of $Y$ for general $k$ by Galois descent.
Relate this to Exercise 2 in HW5. 

(ii) For $y \in Y^H(k)$ explain why $H$ acts on ${\rm{Tan}}_y(Y)$ and prove
${\rm{Tan}}_y(Y^H) = {\rm{Tan}}_y(Y)^H$. 

(iii) Assume $H$ is a closed subgroup of a $k$-group $G$ of finite type, 
$\mathfrak{g} := {\rm{Lie}}(G)$ and $\mathfrak{h} := {\rm{Lie}}(H)$. 
Prove ${\rm{Tan}}_e(Z_G(H)) = \mathfrak{g}^H$ via adjoint action.  Also prove 
${\rm{Tan}}_e(N_G(H)) = \cap_{h \in H(k)} ({\rm{Ad}}_G(h) - 1)^{-1}(\mathfrak{h})$ when $k = k_s$.

\medskip\noindent
5. A diagram $1 \rightarrow G' \stackrel{j}{\rightarrow} G \stackrel{\pi}{\rightarrow} G'' \rightarrow 1$ of
finite type $k$-groups is {\em exact} if $\pi$ is faithfully flat and $G' = \ker \pi$.

(i)  For any such diagram, 
prove $G'' = G/G'$ via $\pi$.  Prove a diagram of $k$-tori
$1 \rightarrow T' \rightarrow T \rightarrow T'' \rightarrow$ is exact if and only if
$0 \rightarrow {\rm{X}}(T'') \rightarrow {\rm{X}}(T) \rightarrow {\rm{X}}(T') \rightarrow 0$ is exact
(as $\mathbf{Z}$-modules). 

(ii) If $G'$ is finite then $\pi$ is an {\em isogeny}.  Prove that isogenies are
{\em finite flat} with constant degree, and that $\pi_n:{\rm{SL}}_n \rightarrow
{\rm{PGL}}_n$ is an isogeny of degree $n$.  Compute ${\rm{Lie}}(\pi_n)$; when is it surjective?

(iii) Prove that a short exact sequence of finite type $k$-groups induces a left-exact sequence
of Lie algebras, short 
exact if $G$ and $G'$ are smooth.  (Smoothness of $G$ can be dropped.)

(iv) Read \S{A}.3 through Example A.3.4 in {\em Pseudo-reductive groups}, 
and prove $F_{X/k}:X \rightarrow X^{(p)}$
is  finite flat of degree $p^{\dim X}$ for $k$-smooth $X$.  
Prove ${\rm{Lie}}(F_{G/k})=0$, and compute $F_{G/k}$ for ${\rm{GL}}(V)$
and ${\rm{O}}(q)$.

\newpage

\centerline{\sc Algebraic Groups I. Homework 8}

\medskip\noindent
1. Let $A$ be a central simple algebra over a field $k$, $T$ a $k$-torus in $\underline{A}^{\times}$.

(i) Adapt Exercise 5 in HW5 to make an \'etale commutative $k$-subalgebra $A_T \subseteq A$
such that $(A_T)_{k_s}$ is generated by $T(k_s)$, and establish a bijection 
between the sets of maximal $k$-tori in $\underline{A}^{\times}$ and maximal \'etale commutative $k$-subalgebras of $A$.
Deduce that ${\rm{SL}}(A)$ is $k$-anisotropic if and only if $A$ is a division algebra.

(ii) For an \'etale commutative $k$-subalgebra $C \subseteq A$, prove $Z_A(C)$ is a semisimple $k$-algebra
with center $C$. 

(iv) If $T$ is {\em maximal} as a $k$-split subtorus of $\underline{A}^{\times}$ prove $T$ is the $k$-group of units in $A_T$
and that the (central!) simple factors $B_i$ of $B_T := Z_A(A_T)$ are {\em division 
algebras}.  

(v) Fix $A \simeq {\rm{End}}_D(V)$ for a right module $V$ over a central division algebra $D$, so $V$ is
a left $A$-module and $V = \prod V_i$ with {\em nonzero} left $B_i$-modules $V_i$.
If $T$ is maximal as a $k$-split torus in $\underline{A}^{\times}$, 
prove $V_i$ has rank 1 over $B_i$ and $D$, so $B_i \simeq D$.
Using $D$-bases, deduce that 
{\em all maximal $k$-split tori in $\underline{A}^{\times}$ are $\underline{A}^{\times}(k)$-conjugate}. 

\medskip\noindent
2. For a torus $T$ over a local field $k$ (allow $\mathbf{R}$, $\mathbf{C}$), 
prove $T$ is $k$-anisotropic if and only if $T(k)$ is compact.

\medskip\noindent
3.  Let $Y$ be a smooth separated $k$-scheme locally of finite type, and 
$T$ a $k$-torus with a left action on $Y$.  This exercise proves that
$Y^T$ is {\em smooth}.  

(i) Reduce to the case $k = \overline{k}$.
Fix a finite local $k$-algebra $R$
with residue field $k$, and an ideal $J$ in $R$ with $J \mathfrak{m}_R = 0$.
Choose $\overline{y} \in Y^T(R/J)$, and for $R$-algebras $A$ let $E(A)$ be
the fiber of $Y(A) \twoheadrightarrow Y(A/JA)$ over $\overline{y}_{A/JA}$.
Let $y_0 = \overline{y} \bmod \mathfrak{m}_R \in Y^T(k)$ and $A_0 = A/\mathfrak{m}_R A$.
Prove $E(A) \ne \emptyset$ and make it a torsor over the $A_0$-module
$F(A) := JA \otimes_k {\rm{Tan}}_{y_0}(Y) = JA \otimes_{A_0} (A_0 \otimes_k {\rm{Tan}}_{y_0}(Y))$
naturally in $A$ (denoted $v+y$). 

(ii) Define an $A_0$-linear $T(A_0)$-action on $F(A)$ (hence a $T_R$-action on $F$),
and prove that $E(A)$ is $T(A)$-stable in 
$Y(A)$ with $t.(v+y) = t_0.v + t.y$ for $y \in E(A)$, $t \in T(A)$, $v \in F(A)$, 
and $t_0 = t \bmod \mathfrak{m}_R$.

(iii) Choose $\xi \in E(R)$ and define a map of functors
$h:T_R \rightarrow F$ by $t.\xi = h(t) + \xi$ for 
points $t$ of $T_R$; check
it is a 1-cocycle, and is a 1-coboundary if and only if $E^{T_R}(R) \ne \emptyset$.
For $V_0 = J \otimes_k {\rm{Tan}}_{y_0}(Y)$
use $h$ to define a 1-cocycle $h_0:T \rightarrow \underline{V}_0$, and prove
$t.(v,c) := (t.v + c h_0(t), c)$ is a $k$-linear representation of $T$ on $V_0 \oplus k$.
Use a $T$-equivariant splitting (!) to prove $h_0$ (and then $h$) is a 1-coboundary;
deduce 
$Y^T$ is smooth!

\medskip\noindent
4. Let $G$ be a smooth $k$-group of finite type, and $T$ a $k$-torus equipped with 
a left action on $G$ (an interesting case being $T$ a $k$-subgroup acting by conjugation, in
which case $G^T = Z_G(T)$).

(i) Use Exercise 3 to show $Z_G(T)$ is smooth, and 
by computing its tangent space at the identity prove for {\em connected} $G$ that
$T \subset Z_G$ if and only if $T$ acts trivially on $\mathfrak{g} = {\rm{Lie}}(G)$.

(ii) Assume $T$ is a $k$-subgroup of $G$ acting by conjugation. Using Exercise 4(iii) of
HW7 and the semisimplicity of the restriction to $T$ of ${\rm{Ad}}_G:G \rightarrow 
{\rm{GL}}(\mathfrak{g})$, prove that $N_G(T)$ and $Z_G(T)$ have the same tangent space
at the identity.  Via (i), deduce that $Z_G(T)$ is an {\em open subscheme} of $N_G(T)$,
so $N_G(T)$ is {\em smooth} and $N_G(T)/Z_G(T)$ is finite \'etale over $k$. 

(iii) Assumptions as in (ii), the 
{\em Weyl group} $W = W(G,T)$ is $N_G(T)/Z_G(T)$.
If $T$ is $k$-split, use the equality ${\rm{End}}_k(T) = {\rm{End}}_{k_s}(T_{k_s})$
to prove that $W(k) = W(k_s)$ and deduce that $W$ is a constant $k$-group.
But show $N_G(T)(k)$ does {\em not} map onto $W(k)$ if 
$k$ is infinite and $K$ is a separable quadratic extension of $k$
such that $-1 \not\in {\rm{N}}_{K/k}(K^{\times})$ (e.g., $k$ totally real and $K$ a CM extension,
or $k = \mathbf{Q}$ and $K = \mathbf{Q}(\sqrt{3})$) with $G = {\rm{SL}}(K) \simeq {\rm{SL}}_2$
and $T$ the {\em non-split} maximal $k$-torus corresponding the norm-1 part of $K \subset {\rm{End}}_k(K)$.

(iv) Prove that $N_G(T)(k) \rightarrow W(k) = W(\overline{k})$ is surjective for the cases 
in HW6, Exercise 4(ii).  

\medskip\noindent
5.  (i) For any field $k$, affine $k$-scheme $X$ of finite type, and nonzero 
finite $k$-algebra $k'$, define a natural map $j_{X,k'/k}:X \rightarrow {\rm{Res}}_{k'/k}(X_{k'})$ by
$X(R) \rightarrow X(k' \otimes_k R) = X_{k'}(k' \otimes_k R)$ for $k$-algebras $R$.
Prove $j_{X,k'/k}$ is a closed immersion and that its formation commutes with fiber products in $X$.

(ii) Let $G$ be an affine $k$-group of finite type. Prove that $j_{G,k'/k}$ is a $k$-homomorphism.

(iii) A {\em vector group} over $k$ is a $k$-group $G$ admitting an isomorphism
$G \simeq \mathbf{G}_a^n$, and a {\em linear structure} on $G$ is the resulting $\mathbf{G}_m$-action.
A {\em linear homomorphism}
$G' \rightarrow G$ between vector groups equipped with linear structures is a $k$-homomorphism
which respects the linear structures.  For example, 
$(x,y) \mapsto (x, y + x^p)$ is a {\em non-linear} automorphism of
$\mathbf{G}_a^2$ (with its usual linear structure) when ${\rm{char}}(k) = p > 0$.

For any $k$, prove $\mathbf{G}_a$ admits a unique linear structure and its linear endomorphism
ring is $k$.  Giving $\mathbf{G}_a^n$ and $\mathbf{G}_a^m$  their usual linear structures, prove
the linear $k$-homomorphisms $\mathbf{G}_a^n \rightarrow \mathbf{G}_a^m$ correspond
to ${\rm{Mat}}_{m \times n}(k)$.  Are there non-linear homomorphisms if
${\rm{char}}(k) = 0$? 

\newpage

\centerline{\sc Algebraic Groups I. Homework 9}

\medskip\noindent
1. Read Appendix B in the book {\em Pseudo-reductive groups} to learn Tits' structure theory for
smooth connected unipotent groups over arbitrary fields
$k$ with positive characteristic, and how $k$-tori act on such groups.  Especially noteworthy
are the results labelled B.1.13, B.2.7, B.3.4, and B.4.3.

\medskip\noindent
2. Let $U$ be a smooth connected commutative affine $k$-group, 
and assume $U$ is $p$-torsion if ${\rm{char}}(k) = p > 0$.

(i) If ${\rm{char}}(k) > 0$ and $U$ is $k$-split, use B.1.12 in {\em Pseudo-reductive groups} to prove 
$U$ is a vector group.  

(ii) Assume ${\rm{char}}(k) = 0$. Prove that any short exact sequence $0 \rightarrow
\mathbf{G}_a \rightarrow G \rightarrow \mathbf{G}_a \rightarrow 0$
is split.  (Hint: $\log(u)$ is an ``algebraic'' function on the unipotent points of
${\rm{Mat}}_n$.) Deduce that $U \simeq \mathbf{G}_a^N$, and prove that any action on
$U$ by a $k$-split torus $T$ respects this linear structure.

\medskip\noindent
3. Let $k'/k$ be a degree-$p$ purely inseparable extension of a field $k$ of characteristic $p > 0$.

(i) Prove that $U= {\rm{R}}_{k'/k}(\mathbf{G}_m)/\mathbf{G}_m$ is smooth and connected 
of dimension $p-1$, and is $p$-torsion.  Deduce it is unipotent.   

(ii) In the Appendix ``Quotient formalism'' it is proved that any commutative extension of
$\mathbf{G}_a$ by $\mathbf{G}_m$ over any field is uniquely split over that field.
Prove that ${\rm{R}}_{k'/k}(\mathbf{G}_m)(k_s)[p] = 1$, and 
deduce that $U$ in (i) does not contain $\mathbf{G}_a$ as a $k$-subgroup!  (For a salvage, see Lemma B.1.10
in {\em Pseudo-reductive groups}:  a $p$-torsion smooth connected commutative affine group over
any field of characteristic $p > 0$ admits an \'etale isogeny onto a vector group.) 

\medskip\noindent
4. Let $G$ be a smooth group of finite type over a field $k$, and $N$ a commutative normal $k$-subgroup scheme. 

(i) Prove that the left $G$-action on $N$ via conjugation factors uniquely through 
an action of $G/N$ on $N$, and if $N$ is central in $G$ then prove that the action of $G$ on itself
via conjugation uniquely factors through an action of $G/N$ on $G$.  Describe this explicitly for $G = {\rm{SL}}_n$
and $N = \mu_n$ over any field $k$, accounting for the fact that ${\rm{SL}}_n(k) \rightarrow {\rm{PGL}}_n(k)$ is 
generally {\em not} surjective. 

(ii) Prove the commutator map $G \times G \rightarrow G$ uniquely factors through
a $k$-morphism $(G/Z_G) \times (G/Z_G) \rightarrow \mathcal{D}(G)$.

\medskip\noindent
5.  Let $B$ be a smooth connected solvable group over a field $k$.

(i) If $B = \mathbf{G}_m \rtimes \mathbf{G}_a$ with the standard semi-direct product
structure, prove that $Z_B(t,0)$ is the left factor for any $t \in k^{\times} - \{1\}$. 

(ii) Deduce by inductive arguments resting on (i) that if $k = \overline{k}$
and $S \subset B(k)$ is a commutative subgroup of semisimple
elements then $S \subset T(k)$ for some maximal torus $T \subset B$.

(iii) Assume ${\rm{char}}(k) \ne 2$ with $k = \overline{k}$, and let $G = {\rm{SO}}_n$ with $n \ge 3$. 
Let $\mu \simeq \mu_2^{n-1}$ be the ``diagonal'' $k$-subgroup 
$\{(\zeta_i) \in \mu_2^n\,|\,\prod \zeta_i = 1\}$. Prove that the disconnected $\mu$ is
maximal as a solvable smooth $k$-subgroup of $G$ and is not contained in
any maximal $k$-torus of $G$ (hint:  it has too much 2-torsion), so in particular is not 
contained in any
Borel $k$-subgroup (by (ii))!

\medskip\noindent
6. Let $G$ be a quasi-split smooth connected affine $k$-group, and $B \subset G$ a Borel $k$-subgroup.  Let 
$T$ be a maximal $k$-torus in $B$.  

(i) Using conjugacy of maximal tori in $G_{\overline{k}}$, prove $g \mapsto gBg^{-1}$ is a bijection from
$N_G(T)(\overline{k})/Z_G(T)(\overline{k})$ onto the set of Borel $\overline{k}$-subgroups containing $T_{\overline{k}}$.
In particular, this set is {\em finite}.  

(ii) Using HW8 Exercise 4, prove that $N_G(T)(k_s)/Z_G(T)(k_s)
\rightarrow N_G(\overline{k})/Z_G(T)(\overline{k})$ is bijective, and deduce that every Borel subgroup of
$G_{\overline{k}}$ containing $T_{\overline{k}}$ is defined over $k_s$!

(iii) Assume that $T$ is $k$-split and $Z_G(T) = T$. Using Hilbert 90 and HW8 Exercise 4, prove that
$N_G(T)(k)/T(k) \rightarrow N_G(T)(k_s)/Z_G(T)(k_s)$ is bijective.  Deduce that 
every Borel subgroup of $G_{\overline{k}}$ containing $T_{\overline{k}}$ is defined over $k$!
In each of the classical cases
(${\rm{GL}}_n$, ${\rm{SL}}_n$, ${\rm{PGL}}_n$, ${\rm{Sp}}_{2n}$, and ${\rm{SO}}_n$),
find all $B$ containing the $k$-split maximal``diagonal'' $T$.  How many parabolic $k$-subgroups
can you find containing one such $B$?  (At least for ${\rm{GL}}_n$, ${\rm{SL}}_n$,
and ${\rm{PGL}}_n$, prove you have found all such parabolics.)

(iv) Prove that each maximal smooth unipotent subgroup of $G_{\overline{k}}$ admits a conjugate
contained in $B_{\overline{k}}$, and deduce that if $B \cap B' = T$ for another Borel $B'$ containing $T$
then $G$ is reductive.  Use this with (iii) to prove reductivity for
${\rm{GL}}_n$ ($n \ge 1$), ${\rm{SL}}_n$ ($n \ge 2$), ${\rm{PGL}}_n$ ($n \ge 2$), ${\rm{Sp}}_{2n}$ ($n \ge 1$), and
${\rm{SO}}_n$ ($n \ge 2$). 

\newpage

\centerline{\sc Algebraic Groups I. Homework 10}

\medskip\noindent
1.  Let $G$ be a smooth connected affine group over a field $k$.

(i) For a maximal $k$-torus $T$ in $G$ and a smooth connected $k$-subgroup $N$ in $G$
that is normalized by $T$, 
prove that $T \cap N$ is a maximal $k$-torus in $N$ (e.g., smooth and connected!).
Show by example that $S \cap N$ can be disconnected for a non-maximal $k$-torus $S$.
Hint: first analyze $Z_G(T) \cap N$ using $T \ltimes N$ to reduce to the case
when $T$ is central in $G$, and then pass to $G/T$. 

(ii) Let $H$ be a smooth connected normal $k$-subgroup of $G$, and
$P$ a parabolic $k$-subgroup.  If $k = \overline{k}$ then 
prove $(P \cap H)_{\rm{red}}^0$ is a parabolic
$k$-subgroup of $H$, and use Chevalley's theorem on parabolics being their
own normalizers on geometric points 
(applied to $H$) to prove $P \cap H$ is connected (hint: work over $\overline{k}$). 

(iii) Granting $Q = N_H(Q)$ scheme-theoretically for parabolic $Q$ in $H$
 (Prop. 3.5.7 in {\em Pseudo-reductive
groups}, rests on structure theory of reductive groups), prove $P \cap H$ in (ii) is smooth.
(Hint: prove $(P \cap H)_{\rm{red}}^0$ is normal in $P$, hence in $P \cap H$!) 
In particular, $B \cap H$ is a Borel $k$-subgroup of $H$ for all Borels $B$ of $G$.

\medskip\noindent
2.  Let $k$ be a field, and $G \in \{{\rm{SL}}_2, {\rm{PGL}}_2\}$.  

(i) Define a unique ${\rm{PGL}}_2$-action on ${\rm{SL}}_2$ lifting conjugation. Prove 
a $k$-automorphism of $G$ preserving the standard Borel $k$-subgroup and the
diagonal $k$-torus is induced by the action of a diagonal $k$-point of 
${\rm{PGL}}_2$.

(ii) Prove that the homomorphism ${\rm{PGL}}_2(k) \rightarrow
{\rm{Aut}}_k(G)$ is an isomorphism.  In particular,
every $k$-automorphism of ${\rm{PGL}}_2$ is inner. Show that
${\rm{SL}}_2$ admits non-inner $k$-automorphisms if and only if $k^{\times} \ne (k^{\times})^2$.

\medskip\noindent
3. Let $\lambda:\mathbf{G}_m \rightarrow G$ be a 1-parameter $k$-subgroup of a smooth 
affine $k$-group $G$. 
 Define  $\mu:U_G(\lambda^{-1}) \times P_G(\lambda) \rightarrow G$ 
to be multiplication.  We seek to prove it is an open immersion. 
 Let $\mathfrak{g} = {\rm{Lie}}(G)$.
 
 (i) For $n \in \mathbf{Z}$ define
$\mathfrak{g}_n$ to be the $n$-weight space for $\lambda$
(i.e., ${\rm{ad}}(\lambda(t)).X = t^n X$). Define $\mathfrak{g}_{\lambda \ge 0} = \oplus_{n \ge 0}
 \mathfrak{g}_n$, and similarly for $\mathfrak{g}_{\lambda > 0}$.
 Prove ${\rm{Lie}}(P_G(\lambda)) = \mathfrak{g}_{\lambda \ge 0}$,
 ${\rm{Lie}}(U_G(\lambda)) = \mathfrak{g}_{\lambda > 0}$, and 
${\rm{Tan}}_{(e,e)}(\mu)$ is an isomorphism.
 
   (ii) If $G = {\rm{GL}}(V)$ and 
 the $\mathbf{G}_m$-action on $V$ has weights $e_1 > \dots > e_m$, justify the block-matrix
descriptions of $U_G(\lambda^{\pm 1})$, $Z_G(\lambda)$, and $P_G(\lambda)$.
Deduce $U_G(\lambda^{-1})$ and $P_G(\lambda)$ are smooth and have trivial
intersection.  

 (iii) Working over $\overline{k}$ and using suitable left and right translations by geometric points,
 prove that ${\rm{d}}\mu(\xi)$ is an isomorphism for all $\overline{k}$-points
 $\xi$ of $U_G(\lambda^{-1}) \times P_G(\lambda)$.  Deduce
 that if $U_G(\lambda^{-1})$ and $P_G(\lambda)$ are smooth (OK for ${\rm{GL}}(V)$ by (ii)) 
 then $\mu$ induces an isomorphism between complete
 local rings at all $\overline{k}$-points, and conclude that
  $\mu$ is flat and quasi-finite.  Hence, $\mu$ has open image in such cases.
  
  (iv) Using valuative criterion for properness, prove 
  a flat quasi-finite separated map $f:X \rightarrow Y$ between noetherian schemes is proper if all 
  fibers $X_y$ have the same rank. (Hint: base change to $Y$ the 
  spectrum of a dvr.)  By Zariski's Main Theorem, proper quasi-finite maps
  are finite.  Deduce $\mu$ is an open immersion if $U_G(\lambda^{-1})$ and 
$P_G(\lambda)$ are smooth with trivial intersection. 
(Hint: finite flat of fiber-degree 1 is isomorphism.)  

This settles
${\rm{GL}}(V)$; the Appendix ``Dynamic approach to algebraic groups'' 
then yields the general case! 
  
\medskip\noindent
4.  Let $\lambda:\mathbf{G}_m \rightarrow G$ be a 1-parameter $k$-subgroup of a smooth affine $k$-group.  For any integer $n \ge 1$, prove that $P_G(\lambda^n) = P_G(\lambda)$,
$U_G(\lambda^n) = U_G(\lambda)$, and $Z_G(\lambda^n) = Z_G(\lambda)$.

\medskip\noindent
5. Let $G$ be a reductive group over a field $k$, and $N$ a smooth closed normal $k$-subgroup.
Prove $N$ is reductive.  In particular, $\mathcal{D}(G)$ is reductive. 

\medskip\noindent
6.  Prove that $\mu_n[d] = \mu_d$ for $d|n$, and that 
$\mathbf{Z}/n\mathbf{Z} \rightarrow {\rm{End}}(\mu_n)$ is an isomorphism. 
  
  \medskip\noindent
  7.  Prove that a rational homomorphism (defined in evident manner: respecting multiplication
  as rational map) between smooth connected groups over a field $k$ extends uniquely to
  a $k$-homomorphism. (Hint: pass to the case $k = k_s$ by Galois descent, and then use suitable
  $k$-point translations.)
  
\medskip\noindent
8. (optional) Let $G$ be a smooth connected affine group over an algebraically closed
field $k$, ${\rm{char}}(k)=0$. 

(i) If all finite-dimensional linear representations of $G$ are completely reducible, then prove
that $G$ is reductive.  (Hint: use Lie-Kolchin, and behavior of semisimplicity under restriction
to a normal subgroup.  This will not use characteristic 0.)

(ii) Conversely, assume that $G$ is reductive. The structure theory of reductive groups implies that 
$\mathfrak{g}$ is a semisimple Lie algebra, and a subspace of a finite-dimensional linear
representation space for $G$ is $G$-stable if and only if it is $\mathfrak{g}$-stable under the induced action
$\mathfrak{g} \rightarrow {\rm{End}}(V)$ since ${\rm{char}}(k) = 0$.  
Prove that all finite-dimensional linear representations
of $G$ are completely reducible.







\end{document}
