\IfFileExists{stacks-project.cls}{%
\documentclass{stacks-project}
}{%
\documentclass{amsart}
}

% The following AMS packages are automatically loaded with
% the amsart documentclass:
%\usepackage{amsmath}
%\usepackage{amssymb}
%\usepackage{amsthm}

\usepackage{graphicx}

% For dealing with references we use the comment environment
\usepackage{verbatim}
\newenvironment{reference}{\comment}{\endcomment}
%\newenvironment{reference}{}{}
\newenvironment{slogan}{\comment}{\endcomment}
\newenvironment{history}{\comment}{\endcomment}

% For commutative diagrams you can use
% \usepackage{amscd}
\usepackage[all]{xy}

% We use 2cell for 2-commutative diagrams.
\xyoption{2cell}
\UseAllTwocells

% To put source file link in headers.
% Change "template.tex" to "this_filename.tex"
% \usepackage{fancyhdr}
% \pagestyle{fancy}
% \lhead{}
% \chead{}
% \rhead{Source file: \url{template.tex}}
% \lfoot{}
% \cfoot{\thepage}
% \rfoot{}
% \renewcommand{\headrulewidth}{0pt}
% \renewcommand{\footrulewidth}{0pt}
% \renewcommand{\headheight}{12pt}

\usepackage{multicol}

% For cross-file-references
\usepackage{xr-hyper}

% Package for hypertext links:
\usepackage{hyperref}

% For any local file, say "hello.tex" you want to link to please
% use \externaldocument[hello-]{hello}
\externaldocument[introduction-]{introduction}
\externaldocument[conventions-]{conventions}
\externaldocument[sets-]{sets}
\externaldocument[categories-]{categories}
\externaldocument[topology-]{topology}
\externaldocument[sheaves-]{sheaves}
\externaldocument[sites-]{sites}
\externaldocument[stacks-]{stacks}
\externaldocument[fields-]{fields}
\externaldocument[algebra-]{algebra}
\externaldocument[brauer-]{brauer}
\externaldocument[homology-]{homology}
\externaldocument[derived-]{derived}
\externaldocument[simplicial-]{simplicial}
\externaldocument[more-algebra-]{more-algebra}
\externaldocument[smoothing-]{smoothing}
\externaldocument[modules-]{modules}
\externaldocument[sites-modules-]{sites-modules}
\externaldocument[injectives-]{injectives}
\externaldocument[cohomology-]{cohomology}
\externaldocument[sites-cohomology-]{sites-cohomology}
\externaldocument[dga-]{dga}
\externaldocument[dpa-]{dpa}
\externaldocument[hypercovering-]{hypercovering}
\externaldocument[schemes-]{schemes}
\externaldocument[constructions-]{constructions}
\externaldocument[properties-]{properties}
\externaldocument[morphisms-]{morphisms}
\externaldocument[coherent-]{coherent}
\externaldocument[divisors-]{divisors}
\externaldocument[limits-]{limits}
\externaldocument[varieties-]{varieties}
\externaldocument[topologies-]{topologies}
\externaldocument[descent-]{descent}
\externaldocument[perfect-]{perfect}
\externaldocument[more-morphisms-]{more-morphisms}
\externaldocument[flat-]{flat}
\externaldocument[groupoids-]{groupoids}
\externaldocument[more-groupoids-]{more-groupoids}
\externaldocument[etale-]{etale}
\externaldocument[chow-]{chow}
\externaldocument[intersection-]{intersection}
\externaldocument[pic-]{pic}
\externaldocument[adequate-]{adequate}
\externaldocument[dualizing-]{dualizing}
\externaldocument[duality-]{duality}
\externaldocument[discriminant-]{discriminant}
\externaldocument[local-cohomology-]{local-cohomology}
\externaldocument[curves-]{curves}
\externaldocument[resolve-]{resolve}
\externaldocument[models-]{models}
\externaldocument[pione-]{pione}
\externaldocument[etale-cohomology-]{etale-cohomology}
\externaldocument[ssgroups-]{ssgroups}
\externaldocument[proetale-]{proetale}
\externaldocument[crystalline-]{crystalline}
\externaldocument[spaces-]{spaces}
\externaldocument[spaces-properties-]{spaces-properties}
\externaldocument[spaces-morphisms-]{spaces-morphisms}
\externaldocument[decent-spaces-]{decent-spaces}
\externaldocument[spaces-cohomology-]{spaces-cohomology}
\externaldocument[spaces-limits-]{spaces-limits}
\externaldocument[spaces-divisors-]{spaces-divisors}
\externaldocument[spaces-over-fields-]{spaces-over-fields}
\externaldocument[spaces-topologies-]{spaces-topologies}
\externaldocument[spaces-descent-]{spaces-descent}
\externaldocument[spaces-perfect-]{spaces-perfect}
\externaldocument[spaces-more-morphisms-]{spaces-more-morphisms}
\externaldocument[spaces-flat-]{spaces-flat}
\externaldocument[spaces-groupoids-]{spaces-groupoids}
\externaldocument[spaces-more-groupoids-]{spaces-more-groupoids}
\externaldocument[bootstrap-]{bootstrap}
\externaldocument[spaces-pushouts-]{spaces-pushouts}
\externaldocument[spaces-chow-]{spaces-chow}
\externaldocument[groupoids-quotients-]{groupoids-quotients}
\externaldocument[spaces-more-cohomology-]{spaces-more-cohomology}
\externaldocument[spaces-simplicial-]{spaces-simplicial}
\externaldocument[spaces-duality-]{spaces-duality}
\externaldocument[formal-spaces-]{formal-spaces}
\externaldocument[restricted-]{restricted}
\externaldocument[spaces-resolve-]{spaces-resolve}
\externaldocument[formal-defos-]{formal-defos}
\externaldocument[defos-]{defos}
\externaldocument[cotangent-]{cotangent}
\externaldocument[examples-defos-]{examples-defos}
\externaldocument[algebraic-]{algebraic}
\externaldocument[examples-stacks-]{examples-stacks}
\externaldocument[stacks-sheaves-]{stacks-sheaves}
\externaldocument[criteria-]{criteria}
\externaldocument[artin-]{artin}
\externaldocument[quot-]{quot}
\externaldocument[stacks-properties-]{stacks-properties}
\externaldocument[stacks-morphisms-]{stacks-morphisms}
\externaldocument[stacks-limits-]{stacks-limits}
\externaldocument[stacks-cohomology-]{stacks-cohomology}
\externaldocument[stacks-perfect-]{stacks-perfect}
\externaldocument[stacks-introduction-]{stacks-introduction}
\externaldocument[stacks-more-morphisms-]{stacks-more-morphisms}
\externaldocument[stacks-geometry-]{stacks-geometry}
\externaldocument[moduli-]{moduli}
\externaldocument[moduli-curves-]{moduli-curves}
\externaldocument[examples-]{examples}
\externaldocument[exercises-]{exercises}
\externaldocument[guide-]{guide}
\externaldocument[desirables-]{desirables}
\externaldocument[coding-]{coding}
\externaldocument[obsolete-]{obsolete}
\externaldocument[fdl-]{fdl}
\externaldocument[index-]{index}

% Theorem environments.
%
\theoremstyle{plain}
\newtheorem{theorem}[subsection]{Theorem}
\newtheorem{proposition}[subsection]{Proposition}
\newtheorem{lemma}[subsection]{Lemma}

\theoremstyle{definition}
\newtheorem{definition}[subsection]{Definition}
\newtheorem{example}[subsection]{Example}
\newtheorem{exercise}[subsection]{Exercise}
\newtheorem{situation}[subsection]{Situation}

\theoremstyle{remark}
\newtheorem{remark}[subsection]{Remark}
\newtheorem{remarks}[subsection]{Remarks}

\numberwithin{equation}{subsection}

% Macros
%
\def\lim{\mathop{\mathrm{lim}}\nolimits}
\def\colim{\mathop{\mathrm{colim}}\nolimits}
\def\Spec{\mathop{\mathrm{Spec}}}
\def\Hom{\mathop{\mathrm{Hom}}\nolimits}
\def\Ext{\mathop{\mathrm{Ext}}\nolimits}
\def\SheafHom{\mathop{\mathcal{H}\!\mathit{om}}\nolimits}
\def\SheafExt{\mathop{\mathcal{E}\!\mathit{xt}}\nolimits}
\def\Sch{\mathit{Sch}}
\def\Mor{\mathop{Mor}\nolimits}
\def\Ob{\mathop{\mathrm{Ob}}\nolimits}
\def\Sh{\mathop{\mathit{Sh}}\nolimits}
\def\NL{\mathop{N\!L}\nolimits}
\def\proetale{{pro\text{-}\acute{e}tale}}
\def\etale{{\acute{e}tale}}
\def\QCoh{\mathit{QCoh}}
\def\Ker{\mathop{\mathrm{Ker}}}
\def\Im{\mathop{\mathrm{Im}}}
\def\Coker{\mathop{\mathrm{Coker}}}
\def\Coim{\mathop{\mathrm{Coim}}}
\def\id{\mathop{\mathrm{id}}\nolimits}

%
% Macros for linear algebraic groups
%
\def\SL{\mathop{\mathrm{SL}}\nolimits}
\def\GL{\mathop{\mathrm{GL}}\nolimits}
\def\ltimes{{\mathchar"256E}}
\def\rtimes{{\mathchar"256F}}
\def\Rrightarrow{{\mathchar"3456}}

%
% Macros for moduli stacks/spaces
%
\def\QCohstack{\mathcal{QC}\!\mathit{oh}}
\def\Cohstack{\mathcal{C}\!\mathit{oh}}
\def\Spacesstack{\mathcal{S}\!\mathit{paces}}
\def\Quotfunctor{\mathrm{Quot}}
\def\Hilbfunctor{\mathrm{Hilb}}
\def\Curvesstack{\mathcal{C}\!\mathit{urves}}
\def\Polarizedstack{\mathcal{P}\!\mathit{olarized}}
\def\Complexesstack{\mathcal{C}\!\mathit{omplexes}}
% \Pic is the operator that assigns to X its picard group, usage \Pic(X)
% \Picardstack_{X/B} denotes the Picard stack of X over B
% \Picardfunctor_{X/B} denotes the Picard functor of X over B
\def\Pic{\mathop{\mathrm{Pic}}\nolimits}
\def\Picardstack{\mathcal{P}\!\mathit{ic}}
\def\Picardfunctor{\mathrm{Pic}}
\def\Deformationcategory{\mathcal{D}\!\mathit{ef}}


% OK, start here.
%
\begin{document}

\title{Relative Cycles}


\maketitle

\phantomsection
\label{section-phantom}

\tableofcontents



\section{Introduction}
\label{section-introduction}

\noindent
In this chapter we briefly explain how to get a theory of relative cycles
very similar to the one in \cite{SV}; the precise relationship between the
definitions will be explained in Remark \ref{remark-compare-SV}.



\section{Cycles relative to fields}
\label{section-relative-fields}

\noindent
Let $k$ be a field. Let $X$ be a locally algebraic scheme over $k$.
Let $r \geq 0$ be an integer. In this setting we have the group
$Z_r(X)$ of $r$-cycles on $X$, see
Chow Homology, Section \ref{chow-section-cycles}.

\medskip\noindent
{\bf Base change.} For any field extension $k'/k$ there is a base change
map $Z_r(X) \to Z_r(X_{k'})$, see
Chow Homology, Section \ref{chow-section-change-base}.
Namely, given an integral closed subscheme $Z \subset X$
of dimension $r$ we send $[Z] \in Z_r(X)$ to the $r$-cycle
$[Z_{k'}]_r \in Z_r(X_{k'})$ associated to the closed subscheme
$Z_{k'} \subset X_{k'}$ (of course in general $Z_{k'}$
is neither irreducible nor reduced). The base change map
$Z_r(X) \to Z_r(X_{k'})$ is always injective.

\begin{lemma}
\label{lemma-multiplicities-field-extension}
Let $K/k$ be a field extension. Let $Z$ be an integral locally algebraic
scheme over $k$. The multiplicity $m_{Z', Z_K}$ of an irreducible
component $Z' \subset Z_K$ is $1$ or a power of the characteristic of $k$.
\end{lemma}

\begin{proof}
If the characteristic of $k$ is zero, then $k$ is perfect and
the multiplicity is always $1$ since $X_K$ is reduced by
Varieties, Lemma \ref{varieties-lemma-geometrically-reduced}.
Assume the characteristic of $k$ is $p > 0$.
Let $L$ be the function field of $Z$. Since $Z$ is locally algebraic
over $k$, the field extension $L/k$ is finitely generated.
The ring $K \otimes_k L$ is Noetherian
(Algebra, Lemma \ref{algebra-lemma-Noetherian-field-extension}) and we
have to show that the length of $(K \otimes_k L)_\mathfrak q$
is a power of $p$ for every minimal prime ideal $\mathfrak q$.

\medskip\noindent
Let $L'/L$ be a finite purely inseparable extension, say of degree
$p^n$. Then $K \otimes_k L \subset K \otimes_k L'$ is a finite
free ring map of degree $p^n$ which induces a homeomorphism on
spectra and purely inseparable residue field extensions.
Hence for every minimal prime $\mathfrak q$ as above
there is a unique minimal prime
$\mathfrak q' \subset K \otimes_k L'$ lying over it and
$$
p^n \text{length}((K \otimes_k L)_\mathfrak q) =
[\kappa(\mathfrak q') : \kappa(\mathfrak q)]
\text{length}((K \otimes_k L')_{\mathfrak q'})
$$
by Algebra, Lemma \ref{algebra-lemma-pushdown-module} applied
to $M = (K \otimes_k L')_{\mathfrak q'} \cong
(K \otimes_k L)_{\mathfrak q}^{\oplus p^n}$.
Since $[\kappa(\mathfrak q') : \kappa(\mathfrak q)]$ is a power
of $p$ we conclude that it suffices to prove the
statement for $L'$ and $\mathfrak q'$.

\medskip\noindent
By the previous paragraph and Algebra, Lemma \ref{algebra-lemma-make-separable}
we may assume that we have a subfield $L/k'/k$
such that $L/k'$ is separable and $k'/k$ is finite
purely inseparable. In this case $\mathfrak q$ lies
over the unique prime ideal $\mathfrak p$ of $K \otimes_k k'$
and the ring map $K \otimes_k k' \to (K \otimes_k L)_\mathfrak q$
is formally smooth. Hence
$\text{length}((K \otimes_k L)_\mathfrak q) =
\text{length}(K \otimes_k k')$.
Since $K \otimes_k k'$ is finite free over $K$ of
rank a power of $p$ the same argument as before shows
that $\text{length}(K \otimes_k k')$ is a power of $p$.
\end{proof}

\begin{lemma}
\label{lemma-how-different}
Let $k$ be a field of characteristic $p > 0$ with perfect closure $k^{perf}$.
Let $X$ be an algebraic scheme over $k$. Let $r \geq 0$ be an integer.
The cokernel of the injective map $Z_r(X) \to Z_r(X_{k^{perf}})$ is a
$p$-power torsion module (More on Algebra, Definition
\ref{more-algebra-definition-f-power-torsion}).
\end{lemma}

\begin{proof}
Since $X$ is quasi-compact, the abelian group $Z_r(X)$ is free with basis
given by the integral closed subschemes of dimension $r$. Similarly for
$Z_r(X_{k^{perf}}$.
Since $X_{k^{perf}} \to X$ is a homeomorphism, it follows
that $Z_r(X) \to Z_r(X_{k^{perf}})$ is injective with torsion cokernel.
Every element in the cokernel is $p$-power torsion by
Lemma \ref{lemma-multiplicities-field-extension}.
\end{proof}





\section{Specialization of cycles}
\label{section-specialization}

\noindent
Let $R$ be a valuation ring with fraction field $K$ and residue field $\kappa$.
Let $X$ be a scheme locally of finite type over $R$. Let $r \geq 0$.
There is a specialization map
$$
sp_{X/R} : Z_r(X_K) \longrightarrow Z_r(X_\kappa)
$$
defined as follows. For an integral closed subscheme $Z \subset X_K$
of dimension $r$ we denote $\overline{Z}$ the scheme theoretic image
of $Z \to X$. Then we let $sp_{X/R}$ be the unique $\mathbf{Z}$-linear
map such that
$$
sp_{X/R}([Z]) = [\overline{Z}_\kappa]_r
$$
We briefly discuss why this is well defined. First, observe that the
morphism $X_K \to X$ is quasi-compact and hence the morphism $Z \to X$
is quasi-compact. Thus taking the scheme theoretic image of $Z \to X$
commutes with flat base change by
Morphisms, Lemma \ref{morphisms-lemma-flat-base-change-scheme-theoretic-image}.
In particular, base changing back to $X_K$ we see that $Z = \overline{Z}_K$.
Since $Z$ is integral, of course $\overline{Z}$ is integral too and
in fact is equal to the unique integral closed subscheme whose generic
point is the (image of the) generic point of $Z$. It follows from
Varieties, Lemma \ref{varieties-lemma-dominate-valuation-ring-dimension-fibres}
that $Z_\kappa$ is equidimensional of dimension $r$.

\begin{lemma}
\label{lemma-specialization-module}
Let $R$ be a valuation ring with fraction field $K$ and residue field $\kappa$.
Let $X$ be a scheme locally of finite type over $R$. Let $r \geq 0$.
Let $\mathcal{F}$ be a finite type, quasi-coherent $\mathcal{O}_X$-module
flat over $R$. Assume $\dim(\text{Supp}(\mathcal{F}_K)) \leq r$.
Then $\dim(\text{Supp}(\mathcal{F}_\kappa)) \leq r$ and
$$
sp_{X/R}([\mathcal{F}_K]_r) = [\mathcal{F}_\kappa]_r
$$
\end{lemma}

\begin{proof}
The statement on dimension follows from More on Morphisms, Lemma
\ref{more-morphisms-lemma-relative-dimension-support-flat}.
Let $x$ be a generic point of an irreducible component
of $\text{Supp}(\mathcal{F}_\kappa)$. Let $Z \subset X_\kappa$
be the integral closed subscheme with generic point $x$.
Assume $\dim(Z) = r$. Let $A = \mathcal{O}_{X, x}$ and $N = \mathcal{F}_x$.
For an $R$-module $M$ we denote $\overline{M} = M/\mathfrak m M$
where $\mathfrak m \subset R$ is the maximal ideal.
Then $N$ is a finite $A$-module flat over $R$ such that
the support of $\overline{N}$ in $\Spec(\overline{A})$ is
consists of the closed point (or is empty).
The finite module $N_K$ over the Noetherian ring $A_K$
has finite support and hence has a finite filtration
$$
0 = N_{K, 0} \subset N_{K, 1} \subset \ldots \subset N_{K, n} = N_K
$$
by submodules such that $N_{K, i}/N_{K, i - i} \cong \kappa(y_i)$
where the $y_i$ are the generic points of the irreducible components
of $\text{Supp}(\mathcal{F}_K)$ of dimension $r$ which specialize to $x$.
Let $y_i$ correspond to the prime ideal $\mathfrak p_i \subset A$.
Unwinding the definitions we have to show
$$
\sum \text{length}_{\overline{A}}(\overline{A/\mathfrak p_i})
=
\text{length}_{\overline{A}}(\overline{N})
$$
Since $R$ is a valuation ring, we can find a filtration
$$
0 = N_0 \subset N_1 \subset \ldots \subset N_n = N
$$
by submodules such that $N_i/N_{i - 1}$ is $R$-flat for all $i$
and such that $N_{K, i} = (N_i)_K$ for all $i$.
Observe that each $N/N_i$ is a finite $A$-module flat over the
valuation ring $R$ and hence of finite presentation over $A$
(More on Flatness, Lemma \ref{}). Thus $N_i/N_{i - 1}$ is a
finite $A$-module too (Algebra, Lemma \ref{}). By flatness we see that
$\overline{N}$ has a filtration whose subquotients are
$\overline{N_i/N_{i - 1}}$. Thus it suffices to show that given
an $R$-flat finite $A$-module $M$ with $M_K = \kappa(\mathfrak q_i)$
we have
$$
A/\mathfrak q_i \cong M
$$
Namely, by Lemma \ref{} we can find $x \in M$ which does not map
to zero in $\overline{M}$. This gives an injective map
$A/\mathfrak q_i \to M$ whose cokernel is flat (Lemma \ref{})
and hence zero (take $\otimes K$).
\end{proof}

\begin{lemma}
\label{lemma-specialization-closed}
Let $R$ be a valuation ring with fraction field $K$ and residue field $\kappa$.
Let $X$ be a scheme locally of finite type over $R$. Let $r \geq 0$.
Let $W \subset X$ be a closed subscheme flat over $R$. Assume
$\dim(W_K) \leq r$. Then $\dim(W_\kappa) \leq r$ and
$$
sp_{X/R}([W_K]_r) = [W_\kappa]_r
$$
\end{lemma}

\begin{proof}
Taking $\mathcal{F} = \mathcal{O}_W$ this is a special case of
Lemma \ref{lemma-specialization-module}. See
Chow Homology, Lemma \ref{chow-lemma-cycle-closed-coherent}.
\end{proof}

\begin{lemma}
\label{lemma-specialization-extension}
Let $R'/R$ be an extension of valuation rings inducing fraction field
extension $K'/K$ and residue field extension $\kappa'/\kappa$
(More on Algebra, Definition
\ref{more-algebra-definition-extension-valuation-rings}).
Let $X$ be locally of finite type over $R$. Denote $X' = X_{R'}$.
Then the diagram
$$
\xymatrix{
Z_r(X'_{K'}) \ar[rr]_{sp_{X'/R'}} & & Z_r(X'_{\kappa'}) \\
Z_r(X_K) \ar[rr]^{sp_{X/R}} \ar[u] & & Z_r(X_\kappa) \ar[u]
}
$$
commutes where $r \geq 0$ and the vertical arrows are base change maps.
\end{lemma}

\begin{proof}
Observe that $X'_{K'} = X_{K'} = X_K \times_{\Spec(K)} \Spec(K')$
and similarly for closed fibres, so that the vertical arrows indeed
make sense (see Section \ref{section-relative-fields}).
Now if $Z \subset X_K$ is an integral closed subscheme with
scheme theoretic image $\overline{Z} \subset X$, then we see that
$Z_{K'} \subset X_{K'}$ is a closed subscheme with scheme theoretic
image $\overline{Z}_{R'} \subset X_{R'}$. The base change of
$[Z]$ is $[Z_{K'}]_r = [\overline{Z}_{K'}]_r$ by definition. We have
$$
sp_{X/R}([Z]) = [\overline{Z}_\kappa]_r
\quad\text{and}\quad
sp_{X'/R'}([\overline{Z}_{K'}]_r) = [(\overline{Z}_{R'})_{\kappa'}]_r
$$
by Lemma \ref{lemma-specialization-module}. Since
$(\overline{Z}_{R'})_{\kappa'} = (\overline{Z}_\kappa)_{\kappa'}$
we conclude.
\end{proof}

\begin{lemma}
\label{lemma-specialization-flat-pullback}
Let $R$ be a valuation ring with fraction field $K$ and residue field $\kappa$.
Let $X$ be a scheme locally of finite type over $R$.
Let $f : X' \to X$ be a morphism which is locally of finite type, flat,
and of relative dimension $e$. 
Then the diagram
$$
\xymatrix{
Z_{r + e}(X'_K) \ar[rr]_{sp_{X'/R}} & & Z_{r + e}(X'_\kappa) \\
Z_r(X_K) \ar[rr]^{sp_{X/R}} \ar[u] & & Z_r(X_\kappa) \ar[u]
}
$$
commutes where $r \geq 0$ and the vertical arrows are given
by flat pullback.
\end{lemma}

\begin{proof}
Let $Z \subset X$ be an integral closed subscheme dominating $R$.
By the construction of $sp_{X/R}$ we see that $sp_{X/R}([Z_K]) = [Z_\kappa]_r$
and that this characterizes the specialization map.
Set $Z' = f^{-1}(Z) = X' \times_X Z$.
Since $R$ is a valuation ring, $Z$ is flat over $R$.
Hence $Z'$ is flat over $R$ and
$sp_{X'/R}([Z'_K]_{r + e}) = [Z'_\kappa]_{r + e}$
by Lemma \ref{lemma-specialization-closed}.
Since by Chow Homology, Lemma \ref{chow-lemma-pullback-coherent}
we have $f_K^*[Z_K] = [Z'_K]_{r + e}$ and
$f_\kappa^*[Z_\kappa]_r = [Z'_\kappa]_{r + e}$ we win.
\end{proof}










\section{Families of cycles on fibres}
\label{section-cycles-fibres}

\noindent
Let $f : X \to S$ be a morphism of schemes which is locally of finite type.
Let $r \geq 0$ be an integer. A
{\it family $\alpha$ of $r$-cycles on fibres of $X/S$} is a family
$$
\alpha = (\alpha_s)_{s \in S}
$$
indexed by the points $s$ of the scheme $S$ where
$\alpha_s \in Z_r(X_s)$.
is an $r$ cycle on the scheme theoretic fibre $X_s$ of $f$ at $s$.
There are various constructions we can perform on families of
$r$-cycles on fibres.

\medskip\noindent
{\bf Base change.} Let
$$
\xymatrix{
X' \ar[r] \ar[d] & X \ar[d] \\
S' \ar[r]^g & S
}
$$
be a catesian square of morphisms of schemes with $f$ locally of finite type.
Let $r \geq 0$ be an integer. Given a family $\alpha$ of $r$-cycles on
fibres of $X/S$ we define the {\it base change} $g^*\alpha$ of $\alpha$
to be the family
$$
g^*\alpha = (\alpha'_{s'})_{s' \in S'}
$$
where $\alpha'_{s'} \in Z_r(X'_{s'})$ is the base change
of the cycle $\alpha_s$ with $s' = g(s)$ as in
Section \ref{section-relative-fields} via the identitification
$X'_{s'} = X_x \times_{\Spec(\kappa(s))} \Spec(\kappa(s'))$
of scheme theoretic fibres.

\medskip\noindent
{\bf Restriction.} Let $f : X \to S$ be a morphism of schemes which is locally
of finite type. Let $r \geq 0$ be an integer. Let $U \subset X$ be an open
subscheme. Given a family $\alpha$ of $r$-cycles on fibres of $X/S$
we can define the {\it restriction} $\alpha|_U$ of $\alpha$ to be the family
$$
\alpha|_U = (\alpha_s|_{U_s})_{s \in S}
$$
of restrictions to scheme theoretic fibres.

\medskip\noindent
{\bf Flat pullback.} Let $f : X \to S$ be a morphism of schemes which is locally
of finite type. Let $r \geq 0$ be an integer. Let $h : X' \to X$ be a
flat morphism, locally of finite type, and of relative dimension $e$.
Let $r \geq 0$ be an integer. Given a family $\alpha$ of $r$-cycles
on fibres of $X/S$ we define the {\it flat pullback} $h^*\alpha$ of $\alpha$
to be the family of $(r + e)$-cycles on fibres
$$
h^*\alpha = (\alpha'_s)_{s \in S}
$$
where $\alpha'_s \in Z_{r + e}(X'_s)$ is the flat pullback
of the cycle $\alpha_s$ in $Z_r(X_s)$ by the flat morphism
$f_s : X'_s \to X_s$
of scheme theoretic fibres.

\begin{example}
\label{example-family-associated-module}
Let $f : X \to S$ be a morphism of schemes which is locally of finite type.
Let $r \geq 0$ be an integer. Let $\mathcal{F}$ be a quasi-coherent
$\mathcal{O}_X$-module of finite type. For $s \in S$ denote $\mathcal{F}_s$
the pullback of $\mathcal{F}$ to $X_s$.
Assume $\dim(\text{Supp}(\mathcal{F}_s)) \leq r$ for all $s \in S$.
Then we can associate to $\mathcal{F}$ the family $[\mathcal{F}/X/S]_r$ of
$r$-cycles on fibres of $X/S$ defined by the formula
$$
[\mathcal{F}/X/S]_r = ([\mathcal{F}_s]_r)_{s \in S}
$$
where $[\mathcal{F}_s]_r$ is given by Chow Homology, Definition
\ref{chow-definition-cycle-associated-to-coherent-sheaf}.
\end{example}

\begin{lemma}
\label{lemma-family-associated-module}
The construction in Example \ref{example-family-associated-module}
is compatible with base change, restriction,
and flat pullback.
\end{lemma}

\begin{proof}
Omitted.
\end{proof}

\begin{example}
\label{example-family-associated-closed}
Let $f : X \to S$ be a morphism of schemes which is locally of finite type.
Let $r \geq 0$ be an integer. Let $Z \subset X$ be a closed subscheme.
For $s \in S$ denote $Z_s$ the inverse image of $Z$ in $X_s$
or equivalently the scheme theoretic fibre of $Z$ at $s$ viewed
as a closed subscheme of $X_s$.
Assume $\dim(Z_s) \leq r$ for all $s \in S$.
Then we can associate to $Z$ the family $[Z/X/S]_r$
of $r$-cycles on fibres of $X/S$ defined by the formula
$$
[Z/X/S]_r = ([Z_s]_r)_{s \in S}
$$
where $[Z_s]_r$ is given by
Chow Homology, Definition
\ref{chow-definition-cycle-associated-to-closed-subscheme}.
\end{example}

\begin{lemma}
\label{lemma-family-associated-closed}
The construction in Example \ref{example-family-associated-closed}
is compatible with base change, restriction,
and flat pullback.
\end{lemma}

\begin{proof}
Taking $\mathcal{F} = (Z \to X)_*\mathcal{O}_Z$ this is a special case of
Lemma \ref{lemma-family-associated-module}. See
Chow Homology, Lemma \ref{chow-lemma-cycle-closed-coherent}.
\end{proof}

\begin{remark}[Support]
\label{remark-supports-family}
Let $f : X \to S$ be a morphism of schemes which is locally of finite type.
Let $r \geq 0$ be an integer. Let $\alpha$ be a family of $r$-cycles
on fibres of $X/S$. We define the {\it support} of $\alpha$ to be
$$
\text{Supp}(\alpha) =
\bigcup\nolimits_{s \in S} \text{Supp}(\alpha_s) \subset X
$$
Here $\text{Supp}(\alpha_s) \subset X_s$ is the
support of the cycle $\alpha_s$
(Chow Homology, Definition \ref{chow-definition-support-cycle}).
The support $\text{Supp}(\alpha)$
is rarely a closed subset of $X$.
\end{remark}

\begin{lemma}
\label{lemma-support-family}
Taking the support as in Remark \ref{remark-supports-family}
is compatible with base change, restriction, and flat pullback.
\end{lemma}

\begin{proof}
Omitted.
\end{proof}

\begin{lemma}
\label{lemma-descend-family}
Let $f : X \to S$ be a morphism of schemes which is locally of finite type.
Let $r \geq 0$ be an integer. Let $g : S' \to S$ be a morphism of
schemes and $X' = S' \times_S X$. Assume that for every $s \in S$ there
exists a point $s' \in S'$ with $g(s') = s$ and such that
$\kappa(s')/\kappa(s)$ is a separable extension of fields. Then
\begin{enumerate}
\item For families $\alpha_1$ and $\alpha_2$ of $r$-cycles on fibres of $X/S$
if $g^*\alpha_1 = g^*\alpha_2$, then $\alpha_1 = \alpha_2$.
\item Given a family $\alpha'$ of $r$-cycles on fibres of $X'/S'$ if
$\text{pr}_1^*\alpha' = \text{pr}_2^*\alpha'$ as families of
$r$-cycles on fibres of $(S' \times_S S') \times_S X / (S' \times_S S')$,
then there is a unique family $\alpha$ of $r$-cycles on fibres of $X/S$
such that $g^*\alpha = \alpha'$.
\end{enumerate}
\end{lemma}

\begin{proof}
Part (1) follows from the injectivity of the base change map discussed
in Section \ref{section-relative-fields}. (This argument works as
long as $S' \to S$ is surjective.)

\medskip\noindent
Let $\alpha'$ be as in (2). Let $s \in S$ be a point. To prove (2)
we have to find an $\alpha_s$ in $Z_r(X_s)$
whose base change to $X_{s'}$ is equal to $\alpha'_{s'}$
for all $s' \in S'$ mapping to $s$. Thus we may replace $S$ by
$\Spec(\kappa(s))$ and $S'$ by $\Spec(\kappa(s)) \times_S S'$
to reduce to the case discussed in the next paragraph.

\medskip\noindent
Assume $S = \Spec(k)$ where $k$ is a field.
Let $W' \subset X'$ be the support of $\alpha'$.
By Lemma \ref{lemma-support-family}
and our assumption on $\alpha'$ we see that
$$
p_1^{-1}(W') = p_2^{-1}(W')
$$
as subsets of $(S' \times_S S') \times_S X$ where $p_1$ and $p_2$
are the morphisms to $X' = S' \times_S X$ induced by the projections
$\text{pr}_1$ and $\text{pr}_2$ from $S' \times_S S'$ to $S'$.
By Descent, Lemma \ref{descent-lemma-equiv-fibre-product}
and the fact that $X' \times_X X' = (S' \times_S S') \times_S X$,
this means there exists a subset $W \subset X$ whose inverse image in
$X'$ is $W'$. Let $s' \in S'$ be a point.
The morphism $s' \to S = \Spec(k)$ is universally open
(Morphisms, Lemma \ref{}), the subset
$\text{Supp}(\alpha'_{s'}) \subset X'_{s'}$ is closed and
is the inverse image of $W$. Hence we conclude $W \subset X$ is closed.

\medskip\noindent
Let $Z \subset W$ be an irreducible component viewed as an integral
closed subscheme of $X$. Let $s' \in S'$ be a point such that
$\kappa(s')/k$ is a separable field extension. Then the
base change $Z_{s'} \subset X_{s'}$ is a reduced closed subscheme,
which is locally a finite union of irreducible components of
$\text{Supp}(\alpha'_{s'})$. Let $Y \subset Z_{s'}$ be one
of the irreducible components. Let $n_Y$ be the coefficient
of $Y$ in $\alpha'_{s'}$. Set $\alpha = \sum n_Y [Z]$.
In the next paragraph we show that
$\alpha' = g^*\alpha$ which finishes the proof.

\medskip\noindent
Let $s'' \in S'$ be a second point (could be equal to $s'$).
Let $Y' \subset Z_{s''}$ be an irreducible component.
To finish the proof, it suffices to show that the coefficient
of $Y'$ in $\alpha'_{s''}$ is equal to the coefficient of $Y'$
in the base change of $\alpha$ by $\kappa(s'')/k$.
Since $Y \to Z$ and $Y' \to Z$ are dominant, we can find a point
$\xi$ of $Y \times_Z Y'$ which maps to the generic point of both
$Y$ and $Y'$. In fact, then $\xi$ is contained in $V \times_Z Z_{s''}$
where $V \subset Y$ is the open which is the complement of the
other irreducible components of $Z_{s'}$.
Then $V \subset Z_{s'}$ is an open subscheme and
$V \times_Z Z_{s''}$ is an open in the source of the morphism
$$
\Spec(\kappa(s') \otimes_k \kappa(s''))
\times_{\Spec(k)} Z
\longrightarrow
\Spec(\kappa(s') \otimes_k \kappa(s''))
$$
Thus we may also assume that $\xi$ is a generic point
of a fibre of this morphism (by replacing $\xi$ with a generalization
in its fibre if necessary). Let $t \in S' \times_S S'$ be the
image of $\xi$. This means that in the diagram
$$
\xymatrix{
X_t \ar[r] \ar[d] & X_{s''} \ar[d] \\
X_{s'} \ar[r] & X_s
}
$$
there exists an integral closed subscheme $Y'' \subset X_t$
whose image in $X_{s'}$ is $Y$ and whose image in $X_{s''}$
is $Y'$. Now the coefficient of $Y''$ in the pullback of
$\alpha$ by base change is independent of how we go
around the diagram. The coefficient of $Y$ in $\alpha'_{s'}$
is equal to $n_Y$ and hence to the coefficient of the pullback
of $\alpha$ by $\kappa(s')/k$. On the other hand, the base
change of $\alpha'_{s'}$ and $\alpha'_{s''}$ to $X_t$ agree
by assumption. Thus we obtain the desired equality.
FIXME.
\end{proof}

\begin{lemma}
\label{lemma-pullback-universally-bijective}
Let $g : S' \to S$ be a bijective morphism of schemes
which induces isomorphisms of residue fields.
Let $f : X \to S$ be locally of finite type. Set $X' = S' \times_S X$.
Let $r \geq 0$. Then base change by $g$ determines a bijection
between the group of families of $r$-cycles on fibres of $X/S$ and
the group of families of $r$-cycles on fibres of $X'/S'$.
\end{lemma}

\begin{proof}
Omitted.
\end{proof}

\begin{remark}
\label{remark-disjoint-decomposition-base}
Let $f : X \to S$ be a morphism of schemes which is locally of
finite type. Assume $S = S' \amalg S''$ is a disjoint union of
two schemes. Set $X' = S' \times_S X$ and $X'' = S'' \times_S X$.
Then the group of families of $r$-cycles on fibres of $X/S$
is the direct sum of the group of families of $r$-cycles on fibres of $X'/S'$
and the group of families of $r$-cycles on fibres of $X''/S''$.
Moreover, the constructions in
Examples \ref{example-family-associated-module} and
\ref{example-family-associated-closed} are compatible with this.
For example, suppose that $\mathcal{F}'$ is a quasi-coherent
$\mathcal{O}_{X'}$-module of finite type such that
$\dim(\text{Supp}(\mathcal{F}'_{s'})) \leq r$ for all $s' \in S'$.
Then we let $\mathcal{F}$ be the quasi-coherent
$\mathcal{O}_X$-module of finite type which restricts to
$\mathcal{F}'$ on $X'$ and $0$ on $X''$.
With this choice we see that $[\mathcal{F}/X/S]_r$
is equal to the sum of $[\mathcal{F}'/X'/S']_r$ and $0$.
Similarly, if $Z' \subset X'$ is a closed subscheme of relative
dimension $\leq r$ over $S'$, then we may view $Z'$ as a closed subscheme
$Z$ of $X$ of relative dimension $\leq r$ over $S$. Of course
then $[Z/X/S]_r$ is equal to the sum of $[Z'/X'/S']_r$ and $0$.
\end{remark}







\section{Families of cycles on fibres: specializations}
\label{section-families-specialization}

\noindent
Let $f : X \to S$ be a morphism of schemes which is locally of finite type.
Let $r \geq 0$ be an integer. Give a family $\alpha$ of $r$-cycles on fibres
of $X/S$ we say $\alpha$ is {\it compatible with specializations} if
for every morphism $g : \Spec(R) \to S$ where $R$ is a valuation ring we have
$$
sp_{X/R}(\alpha_\eta) = \alpha_0
$$
where $sp_{X/R}$ is as in Section \ref{}
and $\alpha_\eta$ (resp.\ $\alpha_0$) is the value of the base change
$g^*\alpha$ of $\alpha$ at the generic (resp.\ closed) point of $\Spec(R)$.

\begin{lemma}
\label{lemma-families-specialization}
In the situation above, if $\alpha$ is compatible with specializations,
then any restriction, base change, or flat pullback of $\alpha$ is compatible
with specializations.
\end{lemma}

\begin{proof}
For flat pullback use Lemma \ref{lemma-specialization-flat-pullback}.
Restriction is a special case of flat pullback. To see it holds for
base change use that base change is transitive.
\end{proof}

\begin{lemma}
\label{lemma-families-specialization-V-descent}
Let $f : X \to S$ be a morphism of schemes which is locally of finite type.
Let $r \geq 0$ be an integer. Let $\alpha$ be a family of $r$-cycles on fibres
of $X/S$. Let $\{g_i : S_i \to S\}$ be a V covering (Toppologies, Definition
\ref{topologies-definition-V-covering}). Then $\alpha$ is compatible
with specializations if and only if each base change $g_i^*\alpha$
is compatible with specializations.
\end{lemma}

\begin{proof}
Omitted.
\end{proof}

\begin{lemma}
\label{lemma-families-specialization-fppf-descent}
Let $f : X \to S$ be a morphism of schemes which is locally of finite type.
Let $r, e \geq 0$ be integers. Let $\alpha$ be a family of $r$-cycles on fibres
of $X/S$. Let $\{f_i : X_i \to X\}$ be a jointly surjective family
of flat morphisms, locally of finite type, and of relative dimension $e$.
Then $\alpha$ is compatible with specializations if and only if each flat
pullback $f_i^*\alpha$ is compatible with specializations.
\end{lemma}

\begin{proof}
Omitted.
\end{proof}

\begin{lemma}
\label{lemma-uniqueness-extension}
Let $f : X \to S$ be a morphism of schemes which is locally of finite type.
Let $r \geq 0$.
Let $\alpha$ and $\beta$ be families of $r$-cycles on fibres of $X/S$.
Let $g : S' \to S$ be a morphism of schemes. Assume
\begin{enumerate}
\item $g$ is quasi-compact and $g(S')$ is dense in $S$,
\item the base changes of $\alpha$ and $\beta$ to $S'$ are equal,
\item $\alpha$ and $\beta$ are compatible with specializations.
\end{enumerate}
Then $\alpha = \beta$.
\end{lemma}

\begin{proof}
Omitted.
\end{proof}

\begin{lemma}
\label{lemma-family-associated-module-specialization}
In the situation of Example \ref{example-family-associated-module}
assume $\mathcal{F}$ is flat over $S$ in dimensions $\geq r$
(More on Flatness, Definition \ref{flat-definition-flat-dimension-n}).
Then $[\mathcal{F}/X/S]_r$ is compatible with specializations.
\end{lemma}

\begin{proof}
By More on Flatness, Lemma \ref{flat-lemma-pre-flat-dimension-n}
the hypothesis on $\mathcal{F}$ is preserved by any base change.
Also, formation of $[\mathcal{F}/X/S]_r$ is compatible with any
base change by Lemma \ref{lemma-family-associated-module}.
Since the condition of being compatible with specializations
is checked after base change to the spectrum of a valuation ring,
this reduces us to the case where $S$ is the spectrum of a valuation ring.
In this case the set
$U = \{x \in X \mid \mathcal{F}\text{ flat at }x\text{ over }S\}$
is open in $X$ by
More on Flatness, Lemma \ref{flat-lemma-finite-type-flat}.
Since the complement of $U$ in $X$ has fibres of dimension $< r$ over
$S$ by assumption, we see that restriction along the inclusion
$U \subset X$ induces an isomorphism on the groups of relative $r$-cycles
after any base change, compatible with specialization maps and with
formation of the relative cycle associated to $\mathcal{F}$.
Thus it suffices to show compability with
specializations for $[\mathcal{F}|_U / U /S]_r$.
Since $\mathcal{F}|_U$ is flat over $S$, this follows from
Lemma \ref{lemma-specialization-module} and the definitions.
\end{proof}

\begin{lemma}
\label{lemma-family-associated-closed-specialization}
In the situation of Example \ref{example-family-associated-closed}
assume $Z$ is flat over $S$ in dimensions $\geq r$.
Then $[Z/X/S]_r$ is compatible with specialization.
\end{lemma}

\begin{proof}
The assumption means that $\mathcal{O}_Z$ is flat over $S$ in
dimensions $\geq r$. Thus applying
Lemma \ref{lemma-family-associated-module-specialization}
with $\mathcal{F} = (Z \to X)_*\mathcal{O}_Z$ we conclude.
\end{proof}

\begin{lemma}
\label{lemma-get-cycles}
Let $f : X \to S$ be a finite type morphism of schemes with $S$ Noetherian.
Let $r \geq 0$. Let $\alpha$ be a family of $r$-cycles on fibres of $X/S$
compatible with specializations. Then there exists an envelope
(Chow Homology, Definition \ref{chow-definition-envelope})
$g : T \to S$, an integer $t$, integers
$n_1, \ldots, n_t \in \mathbf{Z}$, and closed subschemes
$Z_i \subset X_T$ flat and of relative dimension $\leq r$ over $T$ such that
$$
g^*\alpha = \sum n_i [Z_i/X_T/T]
$$
\end{lemma}

\begin{proof}
Suppose we have proper morphisms $g_j : T_j \to S$, $j = 1, \ldots, m$
such that $T = \coprod T_j \to S$ is an envelope and
such that $g_j^*\alpha$ has an expression as in the lemma.
Then the lemma holds, see
Remark \ref{remark-disjoint-decomposition-base}.

\medskip\noindent
By Noetherian induction, we may assume the result holds for
the pullback of $\alpha$ by any closed immersion $S' \to S$
which is not an isomorphism.

\medskip\noindent
Let $S = \bigcup S_j$ be the decomposition of $S$ into its irreducible
components. We may and do view $S_j$ as an integral closed subscheme of $S$.
Observe that $\coprod S_j \to S$ is an envelope.
By the discussion in the first paragraph, it suffices to prove
the lemma for the pullback of $\alpha$ by $S_j \to S$.
By the previous paragraph, this is true unless $S$
is integral.

\medskip\noindent
Assume $S$ is an integral scheme. Let $\eta \in S$ be the generic point
and let $K = \kappa(\eta)$ be the function field of $S$.
Then $\alpha_\eta$ is an $r$-cycle on $X_K$.
Write $\alpha_\eta = \sum n_i[Y_i]$.
Taking the closure of $Y_i$ we obtain integral closed subschemes
$Y'_i \subset X$ whose base change to $\eta$ is $Y_i$.
By generic flatness (for example Morphisms,
Proposition \ref{morphisms-proposition-generic-flatness}),
we see that $Y'_i$ is flat over an open $U$ of $S$ for each $i$.
Applying More on Flatness, Lemma \ref{flat-lemma-flat-after-blowing-up}
we can find a $U$-admissible blowing up $T \to S$
such that the strict transform $Z_i \subset X_T$ of $Y_i$ is flat over $T$.
Set $\beta = \sum n_i[Z_i/X_T/T]$. By
Lemma \ref{lemma-family-associated-closed-specialization}
we see that $\beta$ is compatible with specializations.
It is clear that the pullbacks of $\beta$ and $\alpha$
to $\Spec(K)$ are the same. It follows that $\beta$ and the pullback of
$\alpha$ by $T \to S$ are the same by Lemma \ref{lemma-uniqueness-extension}.

\medskip\noindent
Let $S' = S \setminus U$ with the induced reduced closed subscheme structure.
Set $\alpha' = h^*\alpha$ where $h : S' \to S$ is the inclusion morphism.
By Noetherian induction (see above), the lemma holds for
$\alpha'$ on $X \times_S S'$ over $S'$.
Thus we may find an envelope $g' : T' \to S'$ such that
$(g')^*\alpha' = (h \circ g')^*\alpha$ has an expression as in
the lemma. Since $T \amalg T' \to S$ is an enevelope, we conclude
by the remarks in the first paragraph of the proof.
\end{proof}














\begin{multicols}{2}[\section{Other chapters}]
\noindent
Preliminaries
\begin{enumerate}
\item \hyperref[introduction-section-phantom]{Introduction}
\item \hyperref[conventions-section-phantom]{Conventions}
\item \hyperref[sets-section-phantom]{Set Theory}
\item \hyperref[categories-section-phantom]{Categories}
\item \hyperref[topology-section-phantom]{Topology}
\item \hyperref[sheaves-section-phantom]{Sheaves on Spaces}
\item \hyperref[sites-section-phantom]{Sites and Sheaves}
\item \hyperref[stacks-section-phantom]{Stacks}
\item \hyperref[fields-section-phantom]{Fields}
\item \hyperref[algebra-section-phantom]{Commutative Algebra}
\item \hyperref[brauer-section-phantom]{Brauer Groups}
\item \hyperref[homology-section-phantom]{Homological Algebra}
\item \hyperref[derived-section-phantom]{Derived Categories}
\item \hyperref[simplicial-section-phantom]{Simplicial Methods}
\item \hyperref[more-algebra-section-phantom]{More on Algebra}
\item \hyperref[smoothing-section-phantom]{Smoothing Ring Maps}
\item \hyperref[modules-section-phantom]{Sheaves of Modules}
\item \hyperref[sites-modules-section-phantom]{Modules on Sites}
\item \hyperref[injectives-section-phantom]{Injectives}
\item \hyperref[cohomology-section-phantom]{Cohomology of Sheaves}
\item \hyperref[sites-cohomology-section-phantom]{Cohomology on Sites}
\item \hyperref[dga-section-phantom]{Differential Graded Algebra}
\item \hyperref[dpa-section-phantom]{Divided Power Algebra}
\item \hyperref[hypercovering-section-phantom]{Hypercoverings}
\end{enumerate}
Schemes
\begin{enumerate}
\setcounter{enumi}{24}
\item \hyperref[schemes-section-phantom]{Schemes}
\item \hyperref[constructions-section-phantom]{Constructions of Schemes}
\item \hyperref[properties-section-phantom]{Properties of Schemes}
\item \hyperref[morphisms-section-phantom]{Morphisms of Schemes}
\item \hyperref[coherent-section-phantom]{Cohomology of Schemes}
\item \hyperref[divisors-section-phantom]{Divisors}
\item \hyperref[limits-section-phantom]{Limits of Schemes}
\item \hyperref[varieties-section-phantom]{Varieties}
\item \hyperref[topologies-section-phantom]{Topologies on Schemes}
\item \hyperref[descent-section-phantom]{Descent}
\item \hyperref[perfect-section-phantom]{Derived Categories of Schemes}
\item \hyperref[more-morphisms-section-phantom]{More on Morphisms}
\item \hyperref[flat-section-phantom]{More on Flatness}
\item \hyperref[groupoids-section-phantom]{Groupoid Schemes}
\item \hyperref[more-groupoids-section-phantom]{More on Groupoid Schemes}
\item \hyperref[etale-section-phantom]{\'Etale Morphisms of Schemes}
\end{enumerate}
Topics in Scheme Theory
\begin{enumerate}
\setcounter{enumi}{40}
\item \hyperref[chow-section-phantom]{Chow Homology}
\item \hyperref[intersection-section-phantom]{Intersection Theory}
\item \hyperref[pic-section-phantom]{Picard Schemes of Curves}
\item \hyperref[adequate-section-phantom]{Adequate Modules}
\item \hyperref[dualizing-section-phantom]{Dualizing Complexes}
\item \hyperref[duality-section-phantom]{Duality for Schemes}
\item \hyperref[discriminant-section-phantom]{Discriminants and Differents}
\item \hyperref[local-cohomology-section-phantom]{Local Cohomology}
\item \hyperref[curves-section-phantom]{Algebraic Curves}
\item \hyperref[resolve-section-phantom]{Resolution of Surfaces}
\item \hyperref[models-section-phantom]{Semistable Reduction}
\item \hyperref[pione-section-phantom]{Fundamental Groups of Schemes}
\item \hyperref[etale-cohomology-section-phantom]{\'Etale Cohomology}
\item \hyperref[ssgroups-section-phantom]{Linear Algebraic Groups}
\item \hyperref[crystalline-section-phantom]{Crystalline Cohomology}
\item \hyperref[proetale-section-phantom]{Pro-\'etale Cohomology}
\end{enumerate}
Algebraic Spaces
\begin{enumerate}
\setcounter{enumi}{56}
\item \hyperref[spaces-section-phantom]{Algebraic Spaces}
\item \hyperref[spaces-properties-section-phantom]{Properties of Algebraic Spaces}
\item \hyperref[spaces-morphisms-section-phantom]{Morphisms of Algebraic Spaces}
\item \hyperref[decent-spaces-section-phantom]{Decent Algebraic Spaces}
\item \hyperref[spaces-cohomology-section-phantom]{Cohomology of Algebraic Spaces}
\item \hyperref[spaces-limits-section-phantom]{Limits of Algebraic Spaces}
\item \hyperref[spaces-divisors-section-phantom]{Divisors on Algebraic Spaces}
\item \hyperref[spaces-over-fields-section-phantom]{Algebraic Spaces over Fields}
\item \hyperref[spaces-topologies-section-phantom]{Topologies on Algebraic Spaces}
\item \hyperref[spaces-descent-section-phantom]{Descent and Algebraic Spaces}
\item \hyperref[spaces-perfect-section-phantom]{Derived Categories of Spaces}
\item \hyperref[spaces-more-morphisms-section-phantom]{More on Morphisms of Spaces}
\item \hyperref[spaces-flat-section-phantom]{Flatness on Algebraic Spaces}
\item \hyperref[spaces-groupoids-section-phantom]{Groupoids in Algebraic Spaces}
\item \hyperref[spaces-more-groupoids-section-phantom]{More on Groupoids in Spaces}
\item \hyperref[bootstrap-section-phantom]{Bootstrap}
\item \hyperref[spaces-pushouts-section-phantom]{Pushouts of Algebraic Spaces}
\end{enumerate}
Topics in Geometry
\begin{enumerate}
\setcounter{enumi}{73}
\item \hyperref[spaces-chow-section-phantom]{Chow Groups of Spaces}
\item \hyperref[groupoids-quotients-section-phantom]{Quotients of Groupoids}
\item \hyperref[spaces-more-cohomology-section-phantom]{More on Cohomology of Spaces}
\item \hyperref[spaces-simplicial-section-phantom]{Simplicial Spaces}
\item \hyperref[spaces-duality-section-phantom]{Duality for Spaces}
\item \hyperref[formal-spaces-section-phantom]{Formal Algebraic Spaces}
\item \hyperref[restricted-section-phantom]{Restricted Power Series}
\item \hyperref[spaces-resolve-section-phantom]{Resolution of Surfaces Revisited}
\end{enumerate}
Deformation Theory
\begin{enumerate}
\setcounter{enumi}{81}
\item \hyperref[formal-defos-section-phantom]{Formal Deformation Theory}
\item \hyperref[defos-section-phantom]{Deformation Theory}
\item \hyperref[cotangent-section-phantom]{The Cotangent Complex}
\item \hyperref[examples-defos-section-phantom]{Deformation Problems}
\end{enumerate}
Algebraic Stacks
\begin{enumerate}
\setcounter{enumi}{85}
\item \hyperref[algebraic-section-phantom]{Algebraic Stacks}
\item \hyperref[examples-stacks-section-phantom]{Examples of Stacks}
\item \hyperref[stacks-sheaves-section-phantom]{Sheaves on Algebraic Stacks}
\item \hyperref[criteria-section-phantom]{Criteria for Representability}
\item \hyperref[artin-section-phantom]{Artin's Axioms}
\item \hyperref[quot-section-phantom]{Quot and Hilbert Spaces}
\item \hyperref[stacks-properties-section-phantom]{Properties of Algebraic Stacks}
\item \hyperref[stacks-morphisms-section-phantom]{Morphisms of Algebraic Stacks}
\item \hyperref[stacks-limits-section-phantom]{Limits of Algebraic Stacks}
\item \hyperref[stacks-cohomology-section-phantom]{Cohomology of Algebraic Stacks}
\item \hyperref[stacks-perfect-section-phantom]{Derived Categories of Stacks}
\item \hyperref[stacks-introduction-section-phantom]{Introducing Algebraic Stacks}
\item \hyperref[stacks-more-morphisms-section-phantom]{More on Morphisms of Stacks}
\item \hyperref[stacks-geometry-section-phantom]{The Geometry of Stacks}
\end{enumerate}
Topics in Moduli Theory
\begin{enumerate}
\setcounter{enumi}{99}
\item \hyperref[moduli-section-phantom]{Moduli Stacks}
\item \hyperref[moduli-curves-section-phantom]{Moduli of Curves}
\end{enumerate}
Miscellany
\begin{enumerate}
\setcounter{enumi}{101}
\item \hyperref[examples-section-phantom]{Examples}
\item \hyperref[exercises-section-phantom]{Exercises}
\item \hyperref[guide-section-phantom]{Guide to Literature}
\item \hyperref[desirables-section-phantom]{Desirables}
\item \hyperref[coding-section-phantom]{Coding Style}
\item \hyperref[obsolete-section-phantom]{Obsolete}
\item \hyperref[fdl-section-phantom]{GNU Free Documentation License}
\item \hyperref[index-section-phantom]{Auto Generated Index}
\end{enumerate}
\end{multicols}


\bibliography{my}
\bibliographystyle{amsalpha}

\end{document}
