\IfFileExists{stacks-project.cls}{%
\documentclass{stacks-project}
}{%
\documentclass{amsart}
}

% The following AMS packages are automatically loaded with
% the amsart documentclass:
%\usepackage{amsmath}
%\usepackage{amssymb}
%\usepackage{amsthm}

\usepackage{graphicx}

% For dealing with references we use the comment environment
\usepackage{verbatim}
\newenvironment{reference}{\comment}{\endcomment}
%\newenvironment{reference}{}{}
\newenvironment{slogan}{\comment}{\endcomment}
\newenvironment{history}{\comment}{\endcomment}

% For commutative diagrams you can use
% \usepackage{amscd}
\usepackage[all]{xy}

% We use 2cell for 2-commutative diagrams.
\xyoption{2cell}
\UseAllTwocells

% To put source file link in headers.
% Change "template.tex" to "this_filename.tex"
% \usepackage{fancyhdr}
% \pagestyle{fancy}
% \lhead{}
% \chead{}
% \rhead{Source file: \url{template.tex}}
% \lfoot{}
% \cfoot{\thepage}
% \rfoot{}
% \renewcommand{\headrulewidth}{0pt}
% \renewcommand{\footrulewidth}{0pt}
% \renewcommand{\headheight}{12pt}

\usepackage{multicol}

% For cross-file-references
\usepackage{xr-hyper}

% Package for hypertext links:
\usepackage{hyperref}

% For any local file, say "hello.tex" you want to link to please
% use \externaldocument[hello-]{hello}
\externaldocument[introduction-]{introduction}
\externaldocument[conventions-]{conventions}
\externaldocument[sets-]{sets}
\externaldocument[categories-]{categories}
\externaldocument[topology-]{topology}
\externaldocument[sheaves-]{sheaves}
\externaldocument[sites-]{sites}
\externaldocument[stacks-]{stacks}
\externaldocument[fields-]{fields}
\externaldocument[algebra-]{algebra}
\externaldocument[brauer-]{brauer}
\externaldocument[homology-]{homology}
\externaldocument[derived-]{derived}
\externaldocument[simplicial-]{simplicial}
\externaldocument[more-algebra-]{more-algebra}
\externaldocument[smoothing-]{smoothing}
\externaldocument[modules-]{modules}
\externaldocument[sites-modules-]{sites-modules}
\externaldocument[injectives-]{injectives}
\externaldocument[cohomology-]{cohomology}
\externaldocument[sites-cohomology-]{sites-cohomology}
\externaldocument[dga-]{dga}
\externaldocument[dpa-]{dpa}
\externaldocument[hypercovering-]{hypercovering}
\externaldocument[schemes-]{schemes}
\externaldocument[constructions-]{constructions}
\externaldocument[properties-]{properties}
\externaldocument[morphisms-]{morphisms}
\externaldocument[coherent-]{coherent}
\externaldocument[divisors-]{divisors}
\externaldocument[limits-]{limits}
\externaldocument[varieties-]{varieties}
\externaldocument[topologies-]{topologies}
\externaldocument[descent-]{descent}
\externaldocument[perfect-]{perfect}
\externaldocument[more-morphisms-]{more-morphisms}
\externaldocument[flat-]{flat}
\externaldocument[groupoids-]{groupoids}
\externaldocument[more-groupoids-]{more-groupoids}
\externaldocument[etale-]{etale}
\externaldocument[chow-]{chow}
\externaldocument[intersection-]{intersection}
\externaldocument[pic-]{pic}
\externaldocument[adequate-]{adequate}
\externaldocument[dualizing-]{dualizing}
\externaldocument[duality-]{duality}
\externaldocument[discriminant-]{discriminant}
\externaldocument[local-cohomology-]{local-cohomology}
\externaldocument[curves-]{curves}
\externaldocument[resolve-]{resolve}
\externaldocument[models-]{models}
\externaldocument[pione-]{pione}
\externaldocument[etale-cohomology-]{etale-cohomology}
\externaldocument[ssgroups-]{ssgroups}
\externaldocument[proetale-]{proetale}
\externaldocument[crystalline-]{crystalline}
\externaldocument[spaces-]{spaces}
\externaldocument[spaces-properties-]{spaces-properties}
\externaldocument[spaces-morphisms-]{spaces-morphisms}
\externaldocument[decent-spaces-]{decent-spaces}
\externaldocument[spaces-cohomology-]{spaces-cohomology}
\externaldocument[spaces-limits-]{spaces-limits}
\externaldocument[spaces-divisors-]{spaces-divisors}
\externaldocument[spaces-over-fields-]{spaces-over-fields}
\externaldocument[spaces-topologies-]{spaces-topologies}
\externaldocument[spaces-descent-]{spaces-descent}
\externaldocument[spaces-perfect-]{spaces-perfect}
\externaldocument[spaces-more-morphisms-]{spaces-more-morphisms}
\externaldocument[spaces-flat-]{spaces-flat}
\externaldocument[spaces-groupoids-]{spaces-groupoids}
\externaldocument[spaces-more-groupoids-]{spaces-more-groupoids}
\externaldocument[bootstrap-]{bootstrap}
\externaldocument[spaces-pushouts-]{spaces-pushouts}
\externaldocument[spaces-chow-]{spaces-chow}
\externaldocument[groupoids-quotients-]{groupoids-quotients}
\externaldocument[spaces-more-cohomology-]{spaces-more-cohomology}
\externaldocument[spaces-simplicial-]{spaces-simplicial}
\externaldocument[spaces-duality-]{spaces-duality}
\externaldocument[formal-spaces-]{formal-spaces}
\externaldocument[restricted-]{restricted}
\externaldocument[spaces-resolve-]{spaces-resolve}
\externaldocument[formal-defos-]{formal-defos}
\externaldocument[defos-]{defos}
\externaldocument[cotangent-]{cotangent}
\externaldocument[examples-defos-]{examples-defos}
\externaldocument[algebraic-]{algebraic}
\externaldocument[examples-stacks-]{examples-stacks}
\externaldocument[stacks-sheaves-]{stacks-sheaves}
\externaldocument[criteria-]{criteria}
\externaldocument[artin-]{artin}
\externaldocument[quot-]{quot}
\externaldocument[stacks-properties-]{stacks-properties}
\externaldocument[stacks-morphisms-]{stacks-morphisms}
\externaldocument[stacks-limits-]{stacks-limits}
\externaldocument[stacks-cohomology-]{stacks-cohomology}
\externaldocument[stacks-perfect-]{stacks-perfect}
\externaldocument[stacks-introduction-]{stacks-introduction}
\externaldocument[stacks-more-morphisms-]{stacks-more-morphisms}
\externaldocument[stacks-geometry-]{stacks-geometry}
\externaldocument[moduli-]{moduli}
\externaldocument[moduli-curves-]{moduli-curves}
\externaldocument[examples-]{examples}
\externaldocument[exercises-]{exercises}
\externaldocument[guide-]{guide}
\externaldocument[desirables-]{desirables}
\externaldocument[coding-]{coding}
\externaldocument[obsolete-]{obsolete}
\externaldocument[fdl-]{fdl}
\externaldocument[index-]{index}

% Theorem environments.
%
\theoremstyle{plain}
\newtheorem{theorem}[subsection]{Theorem}
\newtheorem{proposition}[subsection]{Proposition}
\newtheorem{lemma}[subsection]{Lemma}

\theoremstyle{definition}
\newtheorem{definition}[subsection]{Definition}
\newtheorem{example}[subsection]{Example}
\newtheorem{exercise}[subsection]{Exercise}
\newtheorem{situation}[subsection]{Situation}

\theoremstyle{remark}
\newtheorem{remark}[subsection]{Remark}
\newtheorem{remarks}[subsection]{Remarks}

\numberwithin{equation}{subsection}

% Macros
%
\def\lim{\mathop{\mathrm{lim}}\nolimits}
\def\colim{\mathop{\mathrm{colim}}\nolimits}
\def\Spec{\mathop{\mathrm{Spec}}}
\def\Hom{\mathop{\mathrm{Hom}}\nolimits}
\def\Ext{\mathop{\mathrm{Ext}}\nolimits}
\def\SheafHom{\mathop{\mathcal{H}\!\mathit{om}}\nolimits}
\def\SheafExt{\mathop{\mathcal{E}\!\mathit{xt}}\nolimits}
\def\Sch{\mathit{Sch}}
\def\Mor{\mathop{Mor}\nolimits}
\def\Ob{\mathop{\mathrm{Ob}}\nolimits}
\def\Sh{\mathop{\mathit{Sh}}\nolimits}
\def\NL{\mathop{N\!L}\nolimits}
\def\proetale{{pro\text{-}\acute{e}tale}}
\def\etale{{\acute{e}tale}}
\def\QCoh{\mathit{QCoh}}
\def\Ker{\mathop{\mathrm{Ker}}}
\def\Im{\mathop{\mathrm{Im}}}
\def\Coker{\mathop{\mathrm{Coker}}}
\def\Coim{\mathop{\mathrm{Coim}}}
\def\id{\mathop{\mathrm{id}}\nolimits}

%
% Macros for linear algebraic groups
%
\def\SL{\mathop{\mathrm{SL}}\nolimits}
\def\GL{\mathop{\mathrm{GL}}\nolimits}
\def\ltimes{{\mathchar"256E}}
\def\rtimes{{\mathchar"256F}}
\def\Rrightarrow{{\mathchar"3456}}

%
% Macros for moduli stacks/spaces
%
\def\QCohstack{\mathcal{QC}\!\mathit{oh}}
\def\Cohstack{\mathcal{C}\!\mathit{oh}}
\def\Spacesstack{\mathcal{S}\!\mathit{paces}}
\def\Quotfunctor{\mathrm{Quot}}
\def\Hilbfunctor{\mathrm{Hilb}}
\def\Curvesstack{\mathcal{C}\!\mathit{urves}}
\def\Polarizedstack{\mathcal{P}\!\mathit{olarized}}
\def\Complexesstack{\mathcal{C}\!\mathit{omplexes}}
% \Pic is the operator that assigns to X its picard group, usage \Pic(X)
% \Picardstack_{X/B} denotes the Picard stack of X over B
% \Picardfunctor_{X/B} denotes the Picard functor of X over B
\def\Pic{\mathop{\mathrm{Pic}}\nolimits}
\def\Picardstack{\mathcal{P}\!\mathit{ic}}
\def\Picardfunctor{\mathrm{Pic}}
\def\Deformationcategory{\mathcal{D}\!\mathit{ef}}


% OK, start here.
%
\begin{document}

\title{Groupoids in Algebraic Spaces}


\maketitle

\phantomsection
\label{section-phantom}

\tableofcontents

\section{Introduction}
\label{section-introduction}

\noindent
This chapter is devoted to generalities concerning groupoids in algebraic
spaces. We recommend reading the beautiful paper \cite{K-M} by Keel and Mori.

\medskip\noindent
A lot of what we say here is a repeat of what we said in the chapter
on groupoid schemes, see
Groupoids, Section \ref{groupoids-section-introduction}.
The discussion of quotient stacks is new here.


\section{Conventions}
\label{section-conventions}

\noindent
The standing assumption is that all schemes are contained in
a big fppf site $\Sch_{fppf}$. And all rings $A$ considered
have the property that $\Spec(A)$ is (isomorphic) to an
object of this big site.

\medskip\noindent
Let $S$ be a scheme and let $X$ be an algebraic space over $S$.
In this chapter and the following we will write $X \times_S X$
for the product of $X$ with itself (in the category of algebraic
spaces over $S$), instead of $X \times X$.

\medskip\noindent
We continue our convention to label projection maps starting with
index $0$, so we have $\text{pr}_0 : X \times_S Y \to X$ and
$\text{pr}_1 : X \times_S Y \to Y$.




\section{Notation}
\label{section-notation}

\noindent
Let $S$ be a scheme; this will be our base scheme and all algebraic spaces
will be over $S$. Let $B$ be an algebraic space over $S$; this will be our
base algebraic space, and often other algebraic spaces, and schemes will
be over $B$. If we say that $X$ is an algebraic space over $B$, then we
mean that $X$ is an algebraic space over $S$ which comes equipped with
structure morphism $X \to B$. Moreover, we try to reserve the letter $T$
to denote a ``test'' scheme over $B$. In other words $T$ is a scheme
which comes equipped with a structure morphism $T \to B$.
In this situation we denote $X(T)$ for the set of $T$-valued points of
$X$ {\it over} $B$. In a formula:
$$
X(T) = \Mor_B(T, X).
$$
Similarly, given a second algebraic space $Y$ over $B$ we set
$$
X(Y) = \Mor_B(Y, X).
$$
Suppose we are given algebraic spaces $X$, $Y$ over $B$ as above and a
morphism $f : X \to Y$ over $B$. For any scheme $T$ over $B$ we get an
induced map of sets
$$
f : X(T) \longrightarrow Y(T)
$$
which is functorial in the scheme $T$ over $B$.
As $f$ is a map of sheaves on $(\Sch/S)_{fppf}$ over the
sheaf $B$ it is clear that $f$ determines and is determined by this rule.
More generally, we use the same notation for maps between fibre
products. For example, if
$X$, $Y$, $Z$ are algebraic spaces over $B$, and if
$m : X \times_B Y \to Z \times_B Z$ is
a morphism of algebraic spaces over $B$, then we think of $m$ as corresponding
to a collection of maps between $T$-valued points
$$
X(T) \times Y(T) \longrightarrow Z(T) \times Z(T).
$$
And so on and so forth.

\medskip\noindent
Finally, given two maps
$f, g : X \to Y$ of algebraic spaces over $B$, if the induced maps
$f, g : X(T) \to Y(T)$ are equal for every scheme $T$ over $B$, then
$f = g$, and hence also $f, g : X(Z) \to Y(Z)$ are equal for
every third algebraic space $Z$ over $B$. Hence, for example,
to check the axioms for an group algebraic space $G$ over $B$, it
suffices to check commutativity of diagram on $T$-valued points
where $T$ is a scheme over $B$ as we do in
Definition \ref{definition-group-space}
below.




\section{Equivalence relations}
\label{section-equivalence-relations}

\noindent
Please refer to
Groupoids, Section \ref{groupoids-section-equivalence-relations}
for notation.

\begin{definition}
\label{definition-equivalence-relation}
Let $B \to S$ as in Section \ref{section-notation}.
Let $U$ be an algebraic space over $B$.
\begin{enumerate}
\item A {\it pre-relation} on $U$ over $B$ is any morphism
$j : R \to U \times_B U$ of algebraic spaces over $B$.
In this case we set
$t = \text{pr}_0 \circ j$ and $s = \text{pr}_1 \circ j$, so
that $j = (t, s)$.
\item A {\it relation} on $U$ over $B$ is a monomorphism
$j : R \to U \times_B U$ of algebraic spaces over $B$.
\item A {\it pre-equivalence relation} is a pre-relation
$j : R \to U \times_B U$ such that the image of
$j : R(T) \to U(T) \times U(T)$ is an equivalence relation for
all schemes $T$ over $B$.
\item We say a morphism $R \to U \times_B U$ of algebraic spaces over $B$
is an {\it equivalence relation on $U$ over $B$}
if and only if for every $T$ over $B$ the $T$-valued
points of $R$ define an equivalence relation
on the set of $T$-valued points of $U$.
\end{enumerate}
\end{definition}

\noindent
In other words, an equivalence relation is a pre-equivalence relation
such that $j$ is a relation.

\begin{lemma}
\label{lemma-restrict-relation}
Let $B \to S$ as in Section \ref{section-notation}.
Let $U$ be an algebraic space over $B$.
Let $j : R \to U \times_B U$ be a pre-relation.
Let $g : U' \to U$ be a morphism of algebraic spaces over $B$.
Finally, set
$$
R' = (U' \times_B U')\times_{U \times_B U} R
\xrightarrow{j'}
U' \times_B U'
$$
Then $j'$ is a pre-relation on $U'$ over $B$.
If $j$ is a relation, then $j'$ is a relation.
If $j$ is a pre-equivalence relation, then $j'$ is a pre-equivalence relation.
If $j$ is an equivalence relation, then $j'$ is an equivalence relation.
\end{lemma}

\begin{proof}
Omitted.
\end{proof}

\begin{definition}
\label{definition-restrict-relation}
Let $B \to S$ as in Section \ref{section-notation}.
Let $U$ be an algebraic space over $B$.
Let $j : R \to U \times_B U$ be a pre-relation.
Let $g : U' \to U$ be a morphism of algebraic spaces over $B$.
The pre-relation $j' : R' \to U' \times_B U'$ of
Lemma \ref{lemma-restrict-relation} is called
the {\it restriction}, or {\it pullback} of the pre-relation $j$ to $U'$.
In this situation we sometimes write $R' = R|_{U'}$.
\end{definition}

\begin{lemma}
\label{lemma-pre-equivalence-equivalence-relation-points}
Let $B \to S$ as in Section \ref{section-notation}.
Let $j : R \to U \times_B U$ be a pre-relation of algebraic spaces over $B$.
Consider the relation on $|U|$ defined by the rule
$$
x \sim y
\Leftrightarrow
\exists\ r \in |R| :
t(r) = x,
s(r) = y.
$$
If $j$ is a pre-equivalence relation then this is an equivalence relation.
\end{lemma}

\begin{proof}
Suppose that $x \sim y$ and $y \sim z$.
Pick $r \in |R|$ with $t(r) = x$, $s(r) = y$ and
pick $r' \in |R|$ with $t(r') = y$, $s(r') = z$.
We may pick a field $K$ such that $r$ and $r'$ can be
represented by morphisms $r, r' : \Spec(K) \to R$
with $s \circ r = t \circ r'$.
Denote $x = t \circ r$, $y = s \circ r = t \circ r'$,
and $z = s \circ r'$, so $x, y, z : \Spec(K) \to U$.
By construction $(x, y) \in j(R(K))$ and
$(y, z) \in j(R(K))$. Since $j$ is a pre-equivalence relation
we see that also $(x, z) \in j(R(K))$.
This clearly implies that $x \sim z$.

\medskip\noindent
The proof that $\sim$ is reflexive and symmetric is omitted.
\end{proof}















\section{Group algebraic spaces}
\label{section-group-spaces}

\noindent
Please refer to
Groupoids, Section \ref{groupoids-section-group-schemes}
for notation.

\begin{definition}
\label{definition-group-space}
Let $B \to S$ as in Section \ref{section-notation}.
\begin{enumerate}
\item A {\it group algebraic space over $B$} is a pair $(G, m)$, where
$G$ is an algebraic space over $B$ and $m : G \times_B G \to G$ is
a morphism of algebraic spaces over $B$ with the following property:
For every scheme $T$ over $B$ the pair $(G(T), m)$ is a group.
\item A {\it morphism $\psi : (G, m) \to (G', m')$ of
group algebraic spaces over $B$}
is a morphism $\psi : G \to G'$ of algebraic spaces over $B$ such that for
every $T/B$ the induced map $\psi : G(T) \to G'(T)$ is a homomorphism
of groups.
\end{enumerate}
\end{definition}

\noindent
Let $(G, m)$ be a group algebraic space over the algebraic space $B$.
By the discussion in
Groupoids, Section \ref{groupoids-section-group-schemes}
we obtain morphisms of algebraic spaces over $B$
(identity) $e : B \to G$ and (inverse) $i : G \to G$ such that
for every $T$ the quadruple $(G(T), m, e, i)$ satisfies the
axioms of a group.

\medskip\noindent
Let $(G, m)$, $(G', m')$ be group algebraic spaces over $B$.
Let $f : G \to G'$ be a morphism of algebraic spaces over $B$.
It follows from the definition that $f$ is a morphism
of group algebraic spaces over $B$ if and only if the following diagram
is commutative:
$$
\xymatrix{
G \times_B G \ar[r]_-{f \times f} \ar[d]_m &
G' \times_B G' \ar[d]^m \\
G \ar[r]^f & G'
}
$$

\begin{lemma}
\label{lemma-base-change-group-space}
Let $B \to S$ as in Section \ref{section-notation}.
Let $(G, m)$ be a group algebraic space over $B$.
Let $B' \to B$ be a morphism of algebraic spaces.
The pullback $(G_{B'}, m_{B'})$ is a group algebraic space over $B'$.
\end{lemma}

\begin{proof}
Omitted.
\end{proof}








\section{Properties of group algebraic spaces}
\label{section-properties-group-spaces}

\noindent
In this section we collect some simple properties of group algebraic spaces
which hold over any base.

\begin{lemma}
\label{lemma-group-scheme-separated}
Let $S$ be a scheme. Let $B$ be an algebraic space over $S$.
Let $G$ be a group algebraic space over $B$.
Then $G \to B$ is separated (resp.\ quasi-separated, resp.\ locally separated)
if and only if the identity morphism $e : B \to G$ is a closed immersion
(resp.\ quasi-compact, resp.\ an immersion).
\end{lemma}

\begin{proof}
We recall that by
Morphisms of Spaces, Lemma \ref{spaces-morphisms-lemma-section-immersion}
we have that $e$ is a closed immersion (resp.\ quasi-compact, resp.\ an
immersion) if $G \to B$ is separated (resp.\ quasi-separated, resp.\ locally
separated).
For the converse, consider the diagram
$$
\xymatrix{
G \ar[r]_-{\Delta_{G/B}} \ar[d] &
G \times_B G \ar[d]^{(g, g') \mapsto m(i(g), g')} \\
B \ar[r]^e & G
}
$$
It is an exercise in the functorial point of view in algebraic geometry
to show that this diagram is cartesian. In other words, we see that
$\Delta_{G/B}$ is a base change of $e$. Hence if $e$ is a
closed immersion (resp.\ quasi-compact, resp.\ an immersion) so is
$\Delta_{G/B}$, see
Spaces, Lemma \ref{spaces-lemma-base-change-immersions}
(resp.\ Morphisms of Spaces, Lemma
\ref{spaces-morphisms-lemma-base-change-quasi-compact},
resp.\ Spaces, Lemma \ref{spaces-lemma-base-change-immersions}).
\end{proof}

\begin{lemma}
\label{lemma-group-scheme-unramified-or-lqf}
Let $S$ be a scheme. Let $B$ be an algebraic space over $S$.
Let $G$ be a group algebraic space over $B$. Assume $G \to B$
is locally of finite type. Then
$G \to B$ is unramified (resp.\ locally quasi-finite)
if and only if $G \to B$ is unramified (resp.\ quasi-finite)
at $e(b)$ for all $b \in |B|$.
\end{lemma}

\begin{proof}
By Morphisms of Spaces, Lemma \ref{spaces-morphisms-lemma-where-unramified}
(resp.\ Morphisms of Spaces, Lemma
\ref{spaces-morphisms-lemma-base-change-quasi-finite-locus})
there is a maximal open subspace $U \subset G$ such that $U \to B$ is
unramified (resp.\ locally quasi-finite) and formation of $U$
commutes with base change. Thus we reduce to the case where
$B = \Spec(k)$ is the spectrum of a field.
Let $g \in G(K)$ be a point with values in an extension $K/k$.
Then to check whether or not $g$ is in $U$, we may base
change to $K$. Hence it suffices to show
$$
G \to \Spec(k)\text{ is unramified at }e
\Leftrightarrow
G \to \Spec(k)\text{ is unramified at }g
$$
for a $k$-rational point $g$ (resp.\ similarly for
quasi-finite at $g$ and $e$). Since translation by $g$
is an automorphism of $G$ over $k$ this is clear.
\end{proof}

\begin{lemma}
\label{lemma-open-over-which-unramified-or-lqf}
Let $S$ be a scheme. Let $B$ be an algebraic space over $S$.
Let $G$ be a group algebraic space over $B$. Assume $G \to B$
is locally of finite type.
\begin{enumerate}
\item There exists a maximal open subspace $U \subset B$
such that $G_U \to U$ is unramified and formation of $U$
commutes with base change.
\item There exists a maximal open subspace $U \subset B$
such that $G_U \to U$ is locally quasi-finite and formation of $U$
commutes with base change.
\end{enumerate}
\end{lemma}

\begin{proof}
By Morphisms of Spaces, Lemma \ref{spaces-morphisms-lemma-where-unramified}
(resp.\ Morphisms of Spaces, Lemma
\ref{spaces-morphisms-lemma-base-change-quasi-finite-locus})
there is a maximal open subspace $W \subset G$ such that $W \to B$ is
unramified (resp.\ locally quasi-finite). Moreover formation of $W$
commutes with base change.
By Lemma \ref{lemma-group-scheme-unramified-or-lqf}
we see that $U = e^{-1}(W)$ in either case.
\end{proof}












\section{Examples of group algebraic spaces}
\label{section-examples-group-spaces}

\noindent
If $G \to S$ is a group scheme over the base scheme $S$, then the base
change $G_B$ to any algebraic space $B$ over $S$ is an group algebraic
space over $B$ by
Lemma \ref{lemma-base-change-group-space}.
We will frequently use this in the examples below.

\begin{example}[Multiplicative group algebraic space]
\label{example-multiplicative-group}
Let $B \to S$ as in Section \ref{section-notation}.
Consider the functor which associates to any scheme $T$ over
$B$ the group $\Gamma(T, \mathcal{O}_T^*)$
of units in the global sections of the structure sheaf.
This is representable by the group algebraic space
$$
\mathbf{G}_{m, B} = B \times_S \mathbf{G}_{m, S}
$$
over $B$. Here $\mathbf{G}_{m, S}$ is the multiplicative group scheme
over $S$, see
Groupoids, Example \ref{groupoids-example-multiplicative-group}.
\end{example}

\begin{example}[Roots of unity as a group algebraic space]
\label{example-roots-of-unity}
Let $B \to S$ as in Section \ref{section-notation}.
Let $n \in \mathbf{N}$. Consider the functor which associates
to any scheme $T$ over $B$ the subgroup of $\Gamma(T, \mathcal{O}_T^*)$
consisting of $n$th roots of unity.
This is representable by the group algebraic space
$$
\mu_{n, B} = B \times_S \mu_{n, S}
$$
over $B$. Here $\mu_{n, S}$ is the group scheme of $n$th roots of
unity over $S$, see
Groupoids, Example \ref{groupoids-example-roots-of-unity}.
\end{example}

\begin{example}[Additive group algebraic space]
\label{example-additive-group}
Let $B \to S$ as in Section \ref{section-notation}.
Consider the functor which associates
to any scheme $T$ over $B$ the group $\Gamma(T, \mathcal{O}_T)$
of global sections of the structure sheaf.
This is representable by the group algebraic space
$$
\mathbf{G}_{a, B} = B \times_S \mathbf{G}_{a, S}
$$
over $B$. Here $\mathbf{G}_{a, S}$ is the additive group scheme
over $S$, see
Groupoids, Example \ref{groupoids-example-additive-group}.
\end{example}

\begin{example}[General linear group algebraic space]
\label{example-general-linear-group}
Let $B \to S$ as in Section \ref{section-notation}.
Let $n \geq 1$.
Consider the functor which associates
to any scheme $T$ over $B$ the group
$$
\GL_n(\Gamma(T, \mathcal{O}_T))
$$
of invertible $n \times n$ matrices over
the global sections of the structure sheaf.
This is representable by the group algebraic space
$$
\GL_{n, B} = B \times_S \text{GL}_{n, S}
$$
over $B$. Here $\mathbf{G}_{m, S}$ is the general linear group scheme
over $S$, see
Groupoids, Example \ref{groupoids-example-general-linear-group}.
\end{example}

\begin{example}
\label{example-determinant}
Let $B \to S$ as in Section \ref{section-notation}.
Let $n \geq 1$.
The determinant defines a morphism of group algebraic spaces
$$
\det : \GL_{n, B} \longrightarrow \mathbf{G}_{m, B}
$$
over $B$. It is the base change of the determinant morphism
over $S$ from
Groupoids, Example \ref{groupoids-example-determinant}.
\end{example}

\begin{example}[Constant group algebraic space]
\label{example-constant-group}
Let $B \to S$ as in Section \ref{section-notation}.
Let $G$ be an abstract group. Consider the functor
which associates to any scheme $T$ over $B$ the group
of locally constant maps $T \to G$ (where $T$ has the Zariski topology
and $G$ the discrete topology). This is representable by the
group algebraic space
$$
G_B = B \times_S G_S
$$
over $B$. Here $G_S$ is the constant group scheme introduced in
Groupoids, Example \ref{groupoids-example-constant-group}.
\end{example}





\section{Actions of group algebraic spaces}
\label{section-action-group-space}

\noindent
Please refer to
Groupoids, Section \ref{groupoids-section-action-group-scheme}
for notation.

\begin{definition}
\label{definition-action-group-space}
Let $B \to S$ as in Section \ref{section-notation}.
Let $(G, m)$ be a group algebraic space over $B$.
Let $X$ be an algebraic space over $B$.
\begin{enumerate}
\item An {\it action of $G$ on the algebraic space $X/B$} is
a morphism $a : G \times_B X \to X$ over $B$ such that
for every scheme $T$ over $B$ the map $a : G(T) \times X(T) \to X(T)$
defines the structure of a $G(T)$-set on $X(T)$.
\item Suppose that $X$, $Y$ are algebraic spaces over $B$ each endowed
with an action of $G$. An {\it equivariant} or more precisely
a {\it $G$-equivariant} morphism $\psi : X \to Y$
is a morphism of algebraic spaces over $B$ such
that for every $T$ over $B$ the map $\psi : X(T) \to Y(T)$ is
a morphism of $G(T)$-sets.
\end{enumerate}
\end{definition}

\noindent
In situation (1) this means that the diagrams
\begin{equation}
\label{equation-action}
\xymatrix{
G \times_B G \times_B X \ar[r]_-{1_G \times a} \ar[d]_{m \times 1_X} &
G \times_B X \ar[d]^a \\
G \times_B X \ar[r]^a & X
}
\quad
\xymatrix{
G \times_B X \ar[r]_-a & X \\
X\ar[u]^{e \times 1_X} \ar[ru]_{1_X}
}
\end{equation}
are commutative. In situation (2) this just means that the diagram
$$
\xymatrix{
G \times_B X \ar[r]_-{\text{id} \times f} \ar[d]_a &
G \times_B Y \ar[d]^a \\
X \ar[r]^f & Y
}
$$
commutes.

\begin{definition}
\label{definition-free-action}
Let $B \to S$, $G \to B$, and $X \to B$ as in
Definition \ref{definition-action-group-space}.
Let $a : G \times_B X \to X$ be an action of $G$ on $X/B$.
We say the action is {\it free} if for every scheme $T$ over $B$
the action $a : G(T) \times X(T) \to X(T)$ is a free action of
the group $G(T)$ on the set $X(T)$.
\end{definition}

\begin{lemma}
\label{lemma-free-action}
Situation as in
Definition \ref{definition-free-action},
The action $a$ is free if and only if
$$
G \times_B X \to X \times_B X, \quad (g, x) \mapsto (a(g, x), x)
$$
is a monomorphism of algebraic spaces.
\end{lemma}

\begin{proof}
Immediate from the definitions.
\end{proof}









\section{Principal homogeneous spaces}
\label{section-principal-homogeneous}

\noindent
This section is the analogue of
Groupoids, Section \ref{groupoids-section-principal-homogeneous}.
We suggest reading that section first.

\begin{definition}
\label{definition-pseudo-torsor}
Let $S$ be a scheme. Let $B$ be an algebraic space over $S$.
Let $(G, m)$ be a group algebraic space over $B$.
Let $X$ be an algebraic space over $B$, and let
$a : G \times_B X \to X$ be an action of $G$ on $X$.
\begin{enumerate}
\item We say $X$ is a {\it pseudo $G$-torsor} or that $X$ is
{\it formally principally homogeneous under $G$} if the induced
morphism $G \times_B X \to X \times_B X$,
$(g, x) \mapsto (a(g, x), x)$ is an isomorphism.
\item A pseudo $G$-torsor $X$ is called {\it trivial} if there exists
an $G$-equivariant isomorphism $G \to X$ over $B$ where $G$ acts on
$G$ by left multiplication.
\end{enumerate}
\end{definition}

\noindent
It is clear that if $B' \to B$ is a morphism of algebraic spaces then
the pullback $X_{B'}$ of a pseudo $G$-torsor over $B$ is a
pseudo $G_{B'}$-torsor over $B'$.

\begin{lemma}
\label{lemma-characterize-trivial-pseudo-torsors}
In the situation of
Definition \ref{definition-pseudo-torsor}.
\begin{enumerate}
\item The algebraic space $X$ is a pseudo $G$-torsor if and only if for
every scheme $T$ over $B$ the set $X(T)$ is either empty or the action
of the group $G(T)$ on $X(T)$ is simply transitive.
\item A pseudo $G$-torsor $X$ is trivial if and only if the morphism
$X \to B$ has a section.
\end{enumerate}
\end{lemma}

\begin{proof}
Omitted.
\end{proof}

\begin{definition}
\label{definition-principal-homogeneous-space}
Let $S$ be a scheme.
Let $B$ be an algebraic space over $S$.
Let $(G, m)$ be a group algebraic space over $B$.
Let $X$ be a pseudo $G$-torsor over $B$.
\begin{enumerate}
\item We say $X$ is a
{\it principal homogeneous space}, or more precisely a
{\it principal homogeneous $G$-space over $B$}
if there exists a fpqc covering\footnote{The default type of torsor in
Groupoids, Definition \ref{groupoids-definition-principal-homogeneous-space}
is a pseudo torsor which is trivial on an fpqc covering.
Since $G$, as an algebraic space, can be seen a sheaf of groups
there already is a notion of a $G$-torsor which corresponds
to fppf-torsor, see
Lemma \ref{lemma-torsor}.
Hence we use ``principal homogeneous space'' for a pseudo torsor which
is fpqc locally trivial, and we try to avoid using the word torsor in
this situation.}
$\{B_i \to B\}_{i \in I}$ such that each
$X_{B_i} \to B_i$ has a section (i.e., is a trivial pseudo $G_{B_i}$-torsor).
\item Let $\tau \in \{Zariski, \etale, smooth, syntomic, fppf\}$.
We say $X$ is a {\it $G$-torsor in the $\tau$ topology}, or a
{\it $\tau$ $G$-torsor}, or simply a {\it $\tau$ torsor}
if there exists a $\tau$ covering $\{B_i \to B\}_{i \in I}$
such that each $X_{B_i} \to B_i$ has a section.
\item If $X$ is a principal homogeneous $G$-space over $B$,
then we say that it is
{\it quasi-isotrivial} if it is a torsor for the \'etale topology.
\item If $X$ is a principal homogeneous $G$-space over $B$,
then we say that it is
{\it locally trivial} if it is a torsor for the Zariski topology.
\end{enumerate}
\end{definition}

\noindent
We sometimes say ``let $X$ be a $G$-principal homogeneous space over $B$''
to indicate that $X$ is an algebraic space over $B$ equipped with an
action of $G$ which turns it into a principal homogeneous space over $B$.
Next we show that this agrees with the notation introduced earlier
when both apply.

\begin{lemma}
\label{lemma-torsor}
Let $S$ be a scheme.
Let $(G, m)$ be a group algebraic space over $S$.
Let $X$ be an algebraic space over $S$, and let
$a : G \times_S X \to X$ be an action of $G$ on $X$.
Then
$X$ is a $G$-torsor in the $fppf$-topology in the sense of
Definition \ref{definition-principal-homogeneous-space}
if and only if
$X$ is a $G$-torsor on $(\Sch/S)_{fppf}$
in the sense of
Cohomology on Sites, Definition \ref{sites-cohomology-definition-torsor}.
\end{lemma}

\begin{proof}
Omitted.
\end{proof}

\begin{lemma}
\label{lemma-pseudo-torsor-implications}
Let $S$ be a scheme. Let $B$ be an algebraic space over $S$.
Let $G$ be a group algebraic space over $B$.
Let $X$ be a pseudo $G$-torsor over $B$.
Assume $G$ and $X$ locally of finite type over $B$.
\begin{enumerate}
\item If $G \to B$ is unramified, then $X \to B$ is unramified.
\item If $G \to B$ is locally quasi-finite, then $X \to B$ is
locally quasi-finite.
\end{enumerate}
\end{lemma}

\begin{proof}
Proof of (1). By Morphisms of Spaces, Lemma
\ref{spaces-morphisms-lemma-where-unramified}
we reduce to the case where $B$ is the spectrum of a field.
If $X$ is empty, then the result holds. If $X$ is nonempty,
then after increasing the field, we may assume $X$ has a point.
Then $G \cong X$ and the result holds.

\medskip\noindent
The proof of (2) works in exactly the same way using
Morphisms of Spaces, Lemma
\ref{spaces-morphisms-lemma-base-change-quasi-finite-locus}.
\end{proof}
















\section{Equivariant quasi-coherent sheaves}
\label{section-equivariant}

\noindent
Please compare with
Groupoids, Section \ref{groupoids-section-equivariant}.

\begin{definition}
\label{definition-equivariant-module}
Let $B \to S$ as in Section \ref{section-notation}.
Let $(G, m)$ be a group algebraic space over $B$, and
let $a : G \times_B X \to X$ be an action of $G$
on the algebraic space $X$ over $B$.
An {\it $G$-equivariant quasi-coherent $\mathcal{O}_X$-module},
or simply a {\it equivariant quasi-coherent $\mathcal{O}_X$-module},
is a pair $(\mathcal{F}, \alpha)$, where $\mathcal{F}$ is a quasi-coherent
$\mathcal{O}_X$-module, and $\alpha$ is a $\mathcal{O}_{G \times_B X}$-module
map
$$
\alpha : a^*\mathcal{F} \longrightarrow \text{pr}_1^*\mathcal{F}
$$
where $\text{pr}_1 : G \times_B X \to X$ is the projection
such that
\begin{enumerate}
\item the diagram
$$
\xymatrix{
(1_G \times a)^*\text{pr}_2^*\mathcal{F} \ar[r]_-{\text{pr}_{12}^*\alpha} &
\text{pr}_2^*\mathcal{F} \\
(1_G \times a)^*a^*\mathcal{F} \ar[u]^{(1_G \times a)^*\alpha} \ar@{=}[r] &
(m \times 1_X)^*a^*\mathcal{F} \ar[u]_{(m \times 1_X)^*\alpha}
}
$$
is a commutative in the category of
$\mathcal{O}_{G \times_B G \times_B X}$-modules, and
\item the pullback
$$
(e \times 1_X)^*\alpha : \mathcal{F} \longrightarrow \mathcal{F}
$$
is the identity map.
\end{enumerate}
For explanation compare with the relevant diagrams of
Equation (\ref{equation-action}).
\end{definition}

\noindent
Note that the commutativity of the first diagram guarantees that
$(e \times 1_X)^*\alpha$ is an idempotent operator on $\mathcal{F}$,
and hence condition (2) is just the condition that it is an isomorphism.

\begin{lemma}
\label{lemma-pullback-equivariant}
Let $B \to S$ as in Section \ref{section-notation}.
Let $G$ be a group algebraic space over $B$.
Let $f : X \to Y$ be a $G$-equivariant morphism between
algebraic spaces over $B$ endowed with $G$-actions.
Then pullback $f^*$ given by
$(\mathcal{F}, \alpha) \mapsto (f^*\mathcal{F}, (1_G \times f)^*\alpha)$
defines a functor from the category of $G$-equivariant sheaves on
$X$ to the category of quasi-coherent $G$-equivariant sheaves on $Y$.
\end{lemma}

\begin{proof}
Omitted.
\end{proof}





\section{Groupoids in algebraic spaces}
\label{section-groupoids}

\noindent
Please refer to
Groupoids, Section \ref{groupoids-section-groupoids}
for notation.

\begin{definition}
\label{definition-groupoid}
Let $B \to S$ as in Section \ref{section-notation}.
\begin{enumerate}
\item A {\it groupoid in algebraic spaces over $B$} is a
quintuple $(U, R, s, t, c)$ where
$U$ and $R$ are algebraic spaces over $B$, and
$s, t : R \to U$ and $c : R \times_{s, U, t} R \to R$
are morphisms of algebraic spaces over $B$ with the
following property: For any scheme $T$ over $B$ the quintuple
$$
(U(T), R(T), s, t, c)
$$
is a groupoid category.
\item A {\it morphism
$f : (U, R, s, t, c) \to (U', R', s', t', c')$
of groupoids in algebraic spaces over $B$} is given by morphisms
of algebraic spaces $f : U \to U'$ and $f : R \to R'$ over $B$
with the following property:  For any scheme
$T$ over $B$ the maps $f$ define a functor from the
groupoid category $(U(T), R(T), s, t, c)$ to the
groupoid category $(U'(T), R'(T), s', t', c')$.
\end{enumerate}
\end{definition}

\noindent
Let $(U, R, s, t, c)$ be a groupoid in algebraic spaces over $B$.
Note that there are unique morphisms of algebraic spaces
$e : U \to R$ and $i : R \to R$ over $B$ such that for every scheme $T$
over $B$ the induced map $e : U(T) \to R(T)$ is the identity, and
$i : R(T) \to R(T)$ is the inverse of the groupoid category.
The septuple $(U, R, s, t, c, e, i)$ satisfies commutative diagrams
corresponding to each of the axioms (1), (2)(a), (2)(b), (3)(a) and (3)(b) of
Groupoids, Section \ref{groupoids-section-groupoids}.
Conversely given a septuple with this property the quintuple $(U, R, s, t, c)$
is a groupoid in algebraic spaces over $B$. Note that $i$ is an isomorphism,
and $e$ is a section of both $s$ and $t$.
Moreover, given a groupoid in algebraic spaces over $B$ we denote
$$
j = (t, s) : R \longrightarrow U \times_B U
$$
which is compatible with our conventions in
Section \ref{section-equivalence-relations}
above. We sometimes say ``let $(U, R, s, t, c, e, i)$ be a
groupoid in algebraic spaces over $B$'' to stress the existence of identity and
inverse.

\begin{lemma}
\label{lemma-groupoid-pre-equivalence}
Let $B \to S$ as in Section \ref{section-notation}.
Given a groupoid in algebraic spaces $(U, R, s, t, c)$ over $B$
the morphism $j : R \to U \times_B U$ is a pre-equivalence
relation.
\end{lemma}

\begin{proof}
Omitted.
This is a nice exercise in the definitions.
\end{proof}

\begin{lemma}
\label{lemma-equivalence-groupoid}
Let $B \to S$ as in Section \ref{section-notation}.
Given an equivalence relation $j : R \to U \times_B U$ over $B$
there is a unique way to extend it to a groupoid in algebraic spaces
$(U, R, s, t, c)$ over $B$.
\end{lemma}

\begin{proof}
Omitted.
This is a nice exercise in the definitions.
\end{proof}

\begin{lemma}
\label{lemma-diagram}
Let $B \to S$ as in Section \ref{section-notation}.
Let $(U, R, s, t, c)$ be a groupoid in algebraic spaces over $B$.
In the commutative diagram
$$
\xymatrix{
& U & \\
R \ar[d]_s \ar[ru]^t &
R \times_{s, U, t} R
\ar[l]^-{\text{pr}_0} \ar[d]^{\text{pr}_1} \ar[r]_-c &
R \ar[d]^s \ar[lu]_t \\
U & R \ar[l]_t \ar[r]^s & U
}
$$
the two lower squares are fibre product squares.
Moreover, the triangle on top (which is really a square)
is also cartesian.
\end{lemma}

\begin{proof}
Omitted.
Exercise in the definitions and the functorial point of
view in algebraic geometry.
\end{proof}

\begin{lemma}
\label{lemma-diagram-pull}
Let $B \to S$ be as in Section \ref{section-notation}.
Let $(U, R, s, t, c, e, i)$ be a groupoid in algebraic spaces over $B$.
The diagram
\begin{equation}
\label{equation-pull}
\xymatrix{
R \times_{t, U, t} R
\ar@<1ex>[r]^-{\text{pr}_1} \ar@<-1ex>[r]_-{\text{pr}_0}
\ar[d]_{\text{pr}_0 \times c \circ (i, 1)} &
R \ar[r]^t \ar[d]^{\text{id}_R} &
U \ar[d]^{\text{id}_U} \\
R \times_{s, U, t} R
\ar@<1ex>[r]^-c \ar@<-1ex>[r]_-{\text{pr}_0} \ar[d]_{\text{pr}_1} &
R \ar[r]^t \ar[d]^s &
U \\
R \ar@<1ex>[r]^s \ar@<-1ex>[r]_t &
U
}
\end{equation}
is commutative. The two top rows are isomorphic via the vertical maps given.
The two lower left squares are cartesian.
\end{lemma}

\begin{proof}
The commutativity of the diagram follows from the axioms of a groupoid.
Note that, in terms of groupoids, the top left vertical arrow assigns to
a pair of morphisms $(\alpha, \beta)$ with the same target, the pair
of morphisms $(\alpha, \alpha^{-1} \circ \beta)$. In any groupoid
this defines a bijection between
$\text{Arrows} \times_{t, \text{Ob}, t} \text{Arrows}$
and
$\text{Arrows} \times_{s, \text{Ob}, t} \text{Arrows}$. Hence the second
assertion of the lemma.
The last assertion follows from Lemma \ref{lemma-diagram}.
\end{proof}

\begin{lemma}
\label{lemma-base-change-groupoid}
Let $B \to S$ be as in Section \ref{section-notation}.
Let $(U, R, s, t, c)$ be a groupoid in algebraic spaces over $B$.
Let $B' \to B$ be a morphism of algebraic spaces.
Then the base changes $U' = B' \times_B U$,
$R' = B' \times_B R$ endowed with the base changes $s'$, $t'$, $c'$
of the morphisms $s, t, c$ form a groupoid in algebraic spaces
$(U', R', s', t', c')$ over $B'$ and the projections
determine a morphism
$(U', R', s', t', c') \to (U, R, s, t, c)$
of groupoids in algebraic spaces over $B$.
\end{lemma}

\begin{proof}
Omitted. Hint:
$R' \times_{s', U', t'} R' = B' \times_B (R \times_{s, U, t} R)$.
\end{proof}





\section{Quasi-coherent sheaves on groupoids}
\label{section-groupoids-quasi-coherent}

\noindent
Please compare with
Groupoids, Section \ref{groupoids-section-groupoids-quasi-coherent}.

\begin{definition}
\label{definition-groupoid-module}
Let $B \to S$ as in Section \ref{section-notation}.
Let $(U, R, s, t, c)$ be a groupoid in algebraic spaces over $B$.
A {\it quasi-coherent module on $(U, R, s, t, c)$}
is a pair $(\mathcal{F}, \alpha)$, where $\mathcal{F}$ is a quasi-coherent
$\mathcal{O}_U$-module, and $\alpha$ is a $\mathcal{O}_R$-module
map
$$
\alpha : t^*\mathcal{F} \longrightarrow s^*\mathcal{F}
$$
such that
\begin{enumerate}
\item the diagram
$$
\xymatrix{
& \text{pr}_1^*t^*\mathcal{F} \ar[r]_-{\text{pr}_1^*\alpha} &
\text{pr}_1^*s^*\mathcal{F} \ar@{=}[rd] & \\
\text{pr}_0^*s^*\mathcal{F} \ar@{=}[ru] & & & c^*s^*\mathcal{F} \\
& \text{pr}_0^*t^*\mathcal{F} \ar[lu]^{\text{pr}_0^*\alpha} \ar@{=}[r] &
c^*t^*\mathcal{F} \ar[ru]_{c^*\alpha}
}
$$
is a commutative in the category of
$\mathcal{O}_{R \times_{s, U, t} R}$-modules, and
\item the pullback
$$
e^*\alpha : \mathcal{F} \longrightarrow \mathcal{F}
$$
is the identity map.
\end{enumerate}
Compare with the commutative diagrams of Lemma \ref{lemma-diagram}.
\end{definition}

\noindent
The commutativity of the first diagram forces the operator $e^*\alpha$
to be idempotent. Hence the second condition can be reformulated as saying
that $e^*\alpha$ is an isomorphism. In fact, the condition implies that
$\alpha$ is an isomorphism.

\begin{lemma}
\label{lemma-isomorphism}
Let $S$ be a scheme, let $(U, R, s, t, c)$ be a groupoid scheme over $S$.
If $(\mathcal{F}, \alpha)$ is a quasi-coherent module on $(U, R, s, t, c)$
then $\alpha$ is an isomorphism.
\end{lemma}

\begin{proof}
Pull back the commutative diagram of
Definition \ref{definition-groupoid-module}
by the morphism $(i, 1) : R \to R \times_{s, U, t} R$.
Then we see that $i^*\alpha \circ \alpha = s^*e^*\alpha$.
Pulling back by the morphism $(1, i)$ we obtain the relation
$\alpha \circ i^*\alpha = t^*e^*\alpha$. By the second assumption 
these morphisms are the identity. Hence $i^*\alpha$ is an inverse of
$\alpha$.
\end{proof}

\begin{lemma}
\label{lemma-pullback}
Let $B \to S$ as in Section \ref{section-notation}.
Consider a morphism
$f : (U, R, s, t, c) \to (U', R', s', t', c')$
of groupoid in algebraic spaces over $B$. Then pullback $f^*$ given by
$$
(\mathcal{F}, \alpha) \mapsto (f^*\mathcal{F}, f^*\alpha)
$$
defines a functor from the category of quasi-coherent sheaves on
$(U', R', s', t', c')$ to the category of quasi-coherent sheaves on
$(U, R, s, t, c)$.
\end{lemma}

\begin{proof}
Omitted.
\end{proof}

\begin{lemma}
\label{lemma-colimits}
Let $B \to S$ be as in Section \ref{section-notation}.
Let $(U, R, s, t, c)$ be a groupoid in algebraic spaces over $B$.
The category of quasi-coherent modules on $(U, R, s, t, c)$ has colimits.
\end{lemma}

\begin{proof}
Let $i \mapsto (\mathcal{F}_i, \alpha_i)$ be a diagram over the index
category $\mathcal{I}$. We can form the colimit
$\mathcal{F} = \colim \mathcal{F}_i$
which is a quasi-coherent sheaf on $U$, see
Properties of Spaces, Lemma
\ref{spaces-properties-lemma-properties-quasi-coherent}.
Since colimits commute with pullback we see that
$s^*\mathcal{F} = \colim s^*\mathcal{F}_i$ and similarly
$t^*\mathcal{F} = \colim t^*\mathcal{F}_i$. Hence we can set
$\alpha = \colim \alpha_i$. We omit the proof that $(\mathcal{F}, \alpha)$
is the colimit of the diagram in the category of quasi-coherent modules
on $(U, R, s, t, c)$.
\end{proof}

\begin{lemma}
\label{lemma-abelian}
Let $B \to S$ as in Section \ref{section-notation}.
Let $(U, R, s, t, c)$ be a groupoid in algebraic spaces over $B$.
If $s$, $t$ are flat, then the category of quasi-coherent modules on
$(U, R, s, t, c)$ is abelian.
\end{lemma}

\begin{proof}
Let $\varphi : (\mathcal{F}, \alpha) \to (\mathcal{G}, \beta)$ be a
homomorphism of quasi-coherent modules on $(U, R, s, t, c)$. Since $s$ is flat
we see that
$$
0 \to s^*\Ker(\varphi)
\to s^*\mathcal{F} \to s^*\mathcal{G} \to s^*\Coker(\varphi) \to 0
$$
is exact and similarly for pullback by $t$. Hence $\alpha$ and $\beta$
induce isomorphisms
$\kappa : t^*\Ker(\varphi) \to s^*\Ker(\varphi)$ and
$\lambda : t^*\Coker(\varphi) \to s^*\Coker(\varphi)$
which satisfy the cocycle condition. Then it is straightforward to
verify that $(\Ker(\varphi), \kappa)$ and
$(\Coker(\varphi), \lambda)$ are a kernel and cokernel in the
category of quasi-coherent modules on $(U, R, s, t, c)$. Moreover,
the condition $\Coim(\varphi) = \Im(\varphi)$ follows
because it holds over $U$.
\end{proof}





\section{Crystals in quasi-coherent sheaves}
\label{section-crystals}

\noindent
Let $(I, \Phi, j)$ be a pair consisting of a set $I$
and a pre-relation $j : \Phi \to I \times I$. Assume given for every
$i \in I$ a scheme $X_i$ and for every $\phi \in \Phi$
a morphism of schemes $f_\phi : X_{i'} \to X_i$ where
$j(\phi) = (i, i')$. Set
$X = (\{X_i\}_{i \in I}, \{f_\phi\}_{\phi \in \Phi})$.
Define a {\it crystal in quasi-coherent modules on $X$}
as a rule which associates to every
$i \in \Ob(\mathcal{I})$ a quasi-coherent sheaf $\mathcal{F}_i$ on $X_i$
and for every $\phi \in \Phi$ with $j(\phi) = (i, i')$ an isomorphism
$$
\alpha_\phi : f_\phi^*\mathcal{F}_i \longrightarrow \mathcal{F}_{i'}
$$
of quasi-coherent sheaves on $X_{i'}$. These crystals in quasi-coherent
modules form an additive category $\textit{CQC}(X)$\footnote{We could single
out a set of triples $\phi, \phi', \phi'' \in \Phi$ with
$j(\phi) = (i, i')$, $j(\phi') = (i', i'')$, and $j(\phi'') = (i, i'')$
such that $f_{\phi''} = f_\phi \circ f_{\phi'}$ and require that
$\alpha_{\phi'} \circ f_{\phi'}^*\alpha_\phi = \alpha_{\phi''}$
for these triples. This would define an additive subcategory.
For example the data $(I, \Phi)$ could be the set of objects and
arrows of an index category and $X$ could be a diagram of schemes over
this index category. The result of
Lemma \ref{lemma-crystals-in-quasi-coherent-modules}
immediately gives the corresponding result in the subcategory.}.
This category has colimits
(proof is the same as the proof of Lemma \ref{lemma-colimits}).
If all the morphisms $f_\phi$ are flat, then $\textit{CQC}(X)$ is
abelian (proof is the same as the proof of Lemma \ref{lemma-abelian}).
Let $\kappa$ be a cardinal. We say that a crystal in quasi-coherent modules
$\mathcal{F}$ on $X$ is {\it $\kappa$-generated} if each $\mathcal{F}_i$ is
$\kappa$-generated (see
Properties, Definition \ref{properties-definition-kappa-generated}).

\begin{lemma}
\label{lemma-crystals-in-quasi-coherent-modules}
In the situation above, if all the morphisms $f_\phi$ are flat, then there
exists a cardinal $\kappa$ such that every object
$(\{\mathcal{F}_i\}_{i \in I}, \{\alpha_\phi\}_{\phi \in \Phi})$
of $\textit{CQC}(X)$ is the directed colimit of its
$\kappa$-generated submodules.
\end{lemma}

\begin{proof}
In the lemma and in this proof a {\it submodule} of
$(\{\mathcal{F}_i\}_{i \in I}, \{\alpha_\phi\}_{\phi \in \Phi})$
means the data of a quasi-coherent submodule
$\mathcal{G}_i \subset \mathcal{F}_i$ for all $i$ such that
$\alpha_\phi(f_\phi^*\mathcal{G}_i) = \mathcal{G}_{i'}$
as subsheaves of $\mathcal{F}_{i'}$ for all $\phi \in \Phi$.
This makes sense because since $f_\phi$ is flat the
pullback $f^*_\phi$ is exact, i.e., preserves subsheaves.
The proof will be a variant to the proof of
Properties, Lemma \ref{properties-lemma-colimit-kappa}.
We urge the reader to read that proof first.

\medskip\noindent
We claim that it suffices to prove the lemma in case all the schemes
$X_i$ are affine. To see this let
$$
J = \coprod\nolimits_{i \in I} \{U \subset X_i\text{ affine open}\}
$$
and let
\begin{align*}
\Psi = & \coprod\nolimits_{\phi \in \Phi}
\{
(U, V) \mid
U \subset X_i, V \subset X_{i'}\text{ affine open with } f_\phi(U) \subset V
\} \\
&
\amalg \coprod\nolimits_{i \in I}
\{
(U, U') \mid
U, U' \subset X_i\text{ affine open with } U \subset U'
\}
\end{align*}
endowed with the obvious map $\Psi \to J \times J$. Then our
$(\mathcal{F}, \alpha)$ induces a crystal in quasi-coherent sheaves
$(\{\mathcal{H}_j\}_{j \in J}, \{\beta_\psi\}_{\psi \in \Psi})$
on $Y = (J, \Psi)$ by setting $\mathcal{H}_{(i, U)} = \mathcal{F}_i|_U$
for $(i, U) \in J$ and setting $\beta_\psi$ for $\psi \in \Psi$
equal to the restriction of $\alpha_\phi$ to $U$
if $\psi = (\phi, U, V)$ and
equal to $\text{id} : (\mathcal{F}_i|_{U'})|_U \to \mathcal{F}_i|_U$
when $\psi = (i, U, U')$. Moreover, submodules of
$(\{\mathcal{H}_j\}_{j \in J}, \{\beta_\psi\}_{\psi \in \Psi})$
correspond $1$-to-$1$ with submodules of
$(\{\mathcal{F}_i\}_{i \in I}, \{\alpha_\phi\}_{\phi \in \Phi})$.
We omit the proof (hint: use
Sheaves, Section \ref{sheaves-section-bases}).
Moreover, it is clear that if $\kappa$ works for $Y$, then
the same $\kappa$ works for $X$ (by the definition of $\kappa$-generated
modules). Hence it suffices to proof the lemma for crystals in
quasi-coherent sheaves on $Y$.

\medskip\noindent
Assume that all the schemes $X_i$ are affine. Let $\kappa$ be an infinite
cardinal larger than the cardinality of $I$ or $\Phi$. Let
$(\{\mathcal{F}_i\}_{i \in I}, \{\alpha_\phi\}_{\phi \in \Phi})$
be an object of $\textit{CQC}(X)$. For each $i$ write
$X_i = \Spec(A_i)$ and $M_i = \Gamma(X_i, \mathcal{F}_i)$.
For every $\phi \in \Phi$ with $j(\phi) = (i, i')$ the map
$\alpha_\phi$ translates into an $A_{i'}$-module isomorphism
$$
\alpha_\phi : M_i \otimes_{A_i} A_{i'} \longrightarrow M_{i'}
$$
Using the axiom of choice choose a rule
$$
(\phi, m) \longmapsto S(\phi, m')
$$
where the source is the collection of pairs $(\phi, m')$ such that
$\phi \in \Phi$ with $j(\phi) = (i, i')$ and $m' \in M_{i'}$ and
where the output is a finite subset $S(\phi, m') \subset M_i$ so that
$$
m' = \alpha_\phi\left(\sum\nolimits_{m \in S(\phi, m')} m \otimes a'_m\right)
$$
for some $a'_m \in A_{i'}$.

\medskip\noindent
Having made these choices we claim that any section of any $\mathcal{F}_i$
over any $X_i$ is in a $\kappa$-generated submodule. To see this suppose
that we are given a collection $\mathcal{S} = \{S_i\}_{i \in I}$ of subsets
$S_i \subset M_i$ each with cardinality at most $\kappa$. Then we define
a new collection $\mathcal{S}' = \{S'_i\}_{i \in I}$ with
$$
S'_i = S_i \cup
\bigcup\nolimits_{(\phi, m'),\ j(\phi) = (i, i'),\ m' \in S_{i'}} S(\phi, m')
$$
Note that each $S'_i$ still has cardinality at most $\kappa$.
Set $\mathcal{S}^{(0)} = \mathcal{S}$,
$\mathcal{S}^{(1)} = \mathcal{S}'$ and by induction
$\mathcal{S}^{(n + 1)} = (\mathcal{S}^{(n)})'$. Then set
$S_i^{(\infty)} = \bigcup_{n \geq 0} S_i^{(n)}$ and
$\mathcal{S}^{(\infty)} = \{S_i^{(\infty)}\}_{i \in I}$.
By construction, for every $\phi \in \Phi$ with $j(\phi) = (i, i')$
and every $m' \in S^{(\infty)}_{i'}$ we can write $m'$
as a finite linear combination of images $\alpha_\phi(m \otimes 1)$
with $m \in S_i^{(\infty)}$. Thus we see that setting $N_i$ equal
to the $A_i$-submodule of $M_i$ generated by $S_i^{(\infty)}$
the corresponding quasi-coherent submodules
$\widetilde{N_i} \subset \mathcal{F}_i$ form a $\kappa$-generated submodule.
This finishes the proof.
\end{proof}

\begin{lemma}
\label{lemma-set-generators}
Let $B \to S$ as in Section \ref{section-notation}.
Let $(U, R, s, t, c)$ be a groupoid in algebraic spaces over $B$.
If $s$, $t$ are flat, then there exists a set $T$ and a family of objects
$(\mathcal{F}_t, \alpha_t)_{t \in T}$ of $\QCoh(U, R, s, t, c)$
such that every object $(\mathcal{F}, \alpha)$ is the directed colimit
of its submodules isomorphic to one of the objects $(\mathcal{F}_t, \alpha_t)$.
\end{lemma}

\begin{proof}
This lemma is a generalization of
Groupoids, Lemma \ref{groupoids-lemma-colimit-kappa}
which deals with the case of a groupoid in schemes.
We can't quite use the same argument, so we use the
material on ``crystals of quasi-coherent sheaves'' we developed above.

\medskip\noindent
Choose a scheme $W$ and a surjective \'etale morphism $W \to U$.
Choose a scheme $V$ and a surjective \'etale morphism
$V \to W \times_{U, s} R$. Choose a scheme $V'$
and a surjective \'etale morphism $V' \to R \times_{t, U} W$.
Consider the collection of schemes
$$
I = \{W, W \times_U W, V, V', V \times_R V'\}
$$
and the set of morphisms of schemes
$$
\Phi = \{\text{pr}_i : W \times_U W \to W, V \to W, V' \to W,
V \times_R V' \to V, V \times_R V' \to V'\}
$$
Set $X = (I, \Phi)$. Recall that we have defined a category $\textit{CQC}(X)$
of crystals of quasi-coherent sheaves on $X$. There is a functor
$$
\QCoh(U, R, s, t, c) \longrightarrow \textit{CQC}(X)
$$
which assigns to $(\mathcal{F}, \alpha)$ the sheaf $\mathcal{F}|_W$ on $W$,
the sheaf $\mathcal{F}|_{W \times_U W}$ on $W \times_U W$, the
pullback of $\mathcal{F}$ via $V \to W \times_{U, s} R \to W \to U$
on $V$, the pullback of $\mathcal{F}$ via
$V' \to R \times_{t, U} W \to W \to U$ on $V'$, and finally the
pullback of $\mathcal{F}$ via
$V \times_R V' \to V \to W \times_{U, s} R \to W \to U$ on $V \times_R V'$.
As comparison maps $\{\alpha_\phi\}_{\phi \in \Phi}$
we use the obvious ones (coming from associativity of pullbacks)
except for the map
$\phi = \text{pr}_{V'} : V \times_R V' \to V'$ we use the pullback
of $\alpha : t^*\mathcal{F} \to s^*\mathcal{F}$ to $V \times_R V'$.
This makes sense because of the following commutative diagram
$$
\xymatrix{
& V \times_R V' \ar[ld] \ar[rd] \\
V \ar[rd] \ar[dd] & & V' \ar[ld] \ar[dd] \\
& R \ar@<-1ex>[dd]_s \ar@<1ex>[dd]^t \\
W \ar[rd] & & W \ar[ld] \\
& U
}
$$
The functor displayed above isn't an equivalence of categories.
However, since $W \to U$ is surjective \'etale it is faithful\footnote{In
fact the functor is fully faithful, but we won't need this.}.
Since all the morphisms in the diagram above are flat we see that
it is an exact functor of abelian categories. Moreover, we
claim that given $(\mathcal{F}, \alpha)$ with image
$(\{\mathcal{F}_i\}_{i \in I}, \{\alpha_\phi\}_{\phi \in \Phi})$
there is a $1$-to-$1$ correspondence between quasi-coherent submodules
of $(\mathcal{F}, \alpha)$ and
$(\{\mathcal{F}_i\}_{i \in I}, \{\alpha_\phi\}_{\phi \in \Phi})$.
Namely, given a submodule of
$(\{\mathcal{F}_i\}_{i \in I}, \{\alpha_\phi\}_{\phi \in \Phi})$
compatibility of the submodule over $W$ with the projection maps
$W \times_U W \to W$ will guarantee the submodule comes from a
quasi-coherent submodule of $\mathcal{F}$ (by
Properties of Spaces, Proposition
\ref{spaces-properties-proposition-quasi-coherent})
and compatibility with $\alpha_{\text{pr}_{V'}}$ will insure this
subsheaf is compatible with $\alpha$ (details omitted).

\medskip\noindent
Choose a cardinal $\kappa$ as in
Lemma \ref{lemma-crystals-in-quasi-coherent-modules}
for the system $X = (I, \Phi)$. It is clear from
Properties, Lemma \ref{properties-lemma-set-of-iso-classes}
that there is a set of isomorphism classes of $\kappa$-generated
crystals in quasi-coherent sheaves on $X$.
Hence the result is clear.
\end{proof}









\section{Groupoids and group spaces}
\label{section-groupoids-group-spaces}

\noindent
Please compare with
Groupoids, Section \ref{groupoids-section-groupoids-group-schemes}.

\begin{lemma}
\label{lemma-groupoid-from-action}
Let $B \to S$ as in Section \ref{section-notation}.
Let $(G, m)$ be a group algebraic space over $B$ with
identity $e_G$ and inverse $i_G$.
Let $X$ be an algebraic space over $B$ and let $a : G \times_B X \to X$
be an action of $G$ on $X$ over $B$.
Then we get a groupoid in algebraic spaces $(U, R, s, t, c, e, i)$ over $B$
in the following manner:
\begin{enumerate}
\item We set $U = X$, and $R = G \times_B X$.
\item We set $s : R \to U$ equal to $(g, x) \mapsto x$.
\item We set $t : R \to U$ equal to $(g, x) \mapsto a(g, x)$.
\item We set $c : R \times_{s, U, t} R \to R$ equal to
$((g, x), (g', x')) \mapsto (m(g, g'), x')$.
\item We set $e : U \to R$ equal to $x \mapsto (e_G(x), x)$.
\item We set $i : R \to R$ equal to $(g, x) \mapsto (i_G(g), a(g, x))$.
\end{enumerate}
\end{lemma}

\begin{proof}
Omitted. Hint: It is enough to show that this works on the set
level. For this use the description above the lemma describing
$g$ as an arrow from $v$ to $a(g, v)$.
\end{proof}

\begin{lemma}
\label{lemma-action-groupoid-modules}
Let $B \to S$ as in Section \ref{section-notation}.
Let $(G, m)$ be a group algebraic space over $B$.
Let $X$ be an algebraic space over $B$ and let $a : G \times_B X \to X$
be an action of $G$ on $X$ over $B$. Let $(U, R, s, t, c)$ be
the groupoid in algebraic spaces constructed in
Lemma \ref{lemma-groupoid-from-action}.
The rule
$(\mathcal{F}, \alpha) \mapsto (\mathcal{F}, \alpha)$ defines
an equivalence of categories between $G$-equivariant
$\mathcal{O}_X$-modules and the category of quasi-coherent
modules on $(U, R, s, t, c)$.
\end{lemma}

\begin{proof}
The assertion makes sense because $t = a$ and $s = \text{pr}_1$
as morphisms $R = G \times_B X \to X$, see
Definitions \ref{definition-equivariant-module} and
\ref{definition-groupoid-module}.
Using the translation in Lemma \ref{lemma-groupoid-from-action}
the commutativity requirements
of the two definitions match up exactly.
\end{proof}





\section{The stabilizer group algebraic space}
\label{section-stabilizer}

\noindent
Please compare with
Groupoids, Section \ref{groupoids-section-stabilizer}.
Given a groupoid in algebraic spaces we get a group algebraic space as follows.

\begin{lemma}
\label{lemma-groupoid-stabilizer}
Let $B \to S$ as in Section \ref{section-notation}.
Let $(U, R, s, t, c)$ be a groupoid in algebraic spaces over $B$.
The algebraic space $G$ defined by the cartesian square
$$
\xymatrix{
G \ar[r] \ar[d] & R \ar[d]^{j = (t, s)} \\
U \ar[r]^-{\Delta} & U \times_B U
}
$$
is a group algebraic space over $U$ with composition law
$m$ induced by the composition law $c$.
\end{lemma}

\begin{proof}
This is true because in a groupoid category the
set of self maps of any object forms a group.
\end{proof}

\noindent
Since $\Delta$ is a monomorphism we see that $G = j^{-1}(\Delta_{U/B})$ is a
subsheaf of $R$. Thinking of it in this way, the structure morphism
$G = j^{-1}(\Delta_{U/B}) \to U$ is induced by either $s$ or $t$
(it is the same), and $m$ is induced by $c$.

\begin{definition}
\label{definition-stabilizer-groupoid}
Let $B \to S$ as in Section \ref{section-notation}.
Let $(U, R, s, t, c)$ be a groupoid in algebraic spaces over $B$.
The group algebraic space $j^{-1}(\Delta_{U/B}) \to U$ is called the
{\it stabilizer of the groupoid in algebraic spaces $(U, R, s, t, c)$}.
\end{definition}

\noindent
In the literature the stabilizer group algebraic space is often denoted $S$
(because the word stabilizer starts with an ``s'' presumably);
we cannot do this since we have already used $S$ for the base scheme.

\begin{lemma}
\label{lemma-groupoid-action-stabilizer}
Let $B \to S$ as in Section \ref{section-notation}.
Let $(U, R, s, t, c)$ be a groupoid in algebraic spaces over $B$, and let
$G/U$ be its stabilizer. Denote $R_t/U$ the algebraic space $R$ seen as an
algebraic space over $U$ via the morphism $t : R \to U$. There is a
canonical left action
$$
a : G \times_U R_t \longrightarrow R_t
$$
induced by the composition law $c$.
\end{lemma}

\begin{proof}
In terms of points over $T/B$ we define $a(g, r) = c(g, r)$.
\end{proof}








\section{Restricting groupoids}
\label{section-restrict-groupoid}

\noindent
Please refer to
Groupoids, Section \ref{groupoids-section-restrict-groupoid}
for notation.

\begin{lemma}
\label{lemma-restrict-groupoid}
Let $B \to S$ as in Section \ref{section-notation}.
Let $(U, R, s, t, c)$ be a groupoid in algebraic spaces over $B$.
Let $g : U' \to U$ be a morphism of algebraic spaces.
Consider the following diagram
$$
\xymatrix{
R' \ar[d] \ar[r] \ar@/_3pc/[dd]_{t'} \ar@/^1pc/[rr]^{s'}&
R \times_{s, U} U' \ar[r] \ar[d] &
U' \ar[d]^g \\
U' \times_{U, t} R \ar[d] \ar[r] &
R \ar[r]^s \ar[d]_t &
U \\
U' \ar[r]^g &
U
}
$$
where all the squares are fibre product squares. Then there is a
canonical composition law $c' : R' \times_{s', U', t'} R' \to R'$
such that $(U', R', s', t', c')$ is a groupoid in algebraic spaces over
$B$ and such that $U' \to U$, $R' \to R$ defines a morphism
$(U', R', s', t', c') \to (U, R, s, t, c)$ of groupoids in algebraic spaces
over $B$. Moreover, for any scheme $T$ over $B$ the functor of groupoids
$$
(U'(T), R'(T), s', t', c') \to (U(T), R(T), s, t, c)
$$
is the restriction (see
Groupoids, Section \ref{groupoids-section-restrict-groupoid})
of $(U(T), R(T), s, t, c)$ via the map $U'(T) \to U(T)$.
\end{lemma}

\begin{proof}
Omitted.
\end{proof}

\begin{definition}
\label{definition-restrict-groupoid}
Let $B \to S$ as in Section \ref{section-notation}.
Let $(U, R, s, t, c)$ be a groupoid in algebraic spaces over $B$.
Let $g : U' \to U$ be a morphism of algebraic spaces over $B$.
The morphism of groupoids in algebraic spaces
$(U', R', s', t', c') \to (U, R, s, t, c)$
constructed in Lemma \ref{lemma-restrict-groupoid} is called
the {\it restriction of $(U, R, s, t, c)$ to $U'$}.
We sometime use the notation $R' = R|_{U'}$ in this case.
\end{definition}

\begin{lemma}
\label{lemma-restrict-groupoid-relation}
The notions of restricting groupoids and
(pre-)equivalence relations defined in Definitions
\ref{definition-restrict-groupoid} and \ref{definition-restrict-relation}
agree via the constructions of
Lemmas \ref{lemma-groupoid-pre-equivalence} and
\ref{lemma-equivalence-groupoid}.
\end{lemma}

\begin{proof}
What we are saying here is that $R'$ of
Lemma \ref{lemma-restrict-groupoid} is also
equal to
$$
R' = (U' \times_B U')\times_{U \times_B U} R
\longrightarrow
U' \times_B U'
$$
In fact this might have been a clearer way to state that lemma.
\end{proof}





\section{Invariant subspaces}
\label{section-invariant}

\noindent
In this section we discuss briefly the notion of an invariant subspace.

\begin{definition}
\label{definition-invariant-open}
Let $B \to S$ as in Section \ref{section-notation}.
Let $(U, R, s, t, c)$ be a groupoid in algebraic spaces over the base $B$.
\begin{enumerate}
\item We say an open subspace $W \subset U$ is {\it $R$-invariant} if
$t(s^{-1}(W)) \subset W$.
\item A locally closed subspace $Z \subset U$ is called {\it $R$-invariant}
if $t^{-1}(Z) = s^{-1}(Z)$ as locally closed subspaces of $R$.
\item A monomorphism of algebraic spaces $T \to U$ is {\it $R$-invariant}
if $T \times_{U, t} R = R \times_{s, U} T$ as algebraic spaces over $R$.
\end{enumerate}
\end{definition}

\noindent
For an open subspace $W \subset U$ the $R$-invariance
is also equivalent to requiring that $s^{-1}(W) = t^{-1}(W)$.
If $W \subset U$ is $R$-invariant then the restriction of $R$ to
$W$ is just $R_W = s^{-1}(W) = t^{-1}(W)$. Similarly, if $Z \subset U$
is an $R$-invariant locally closed subspace, then the restriction of $R$
to $Z$ is just $R_Z = s^{-1}(Z) = t^{-1}(Z)$.

\begin{lemma}
\label{lemma-constructing-invariant-opens}
Let $B \to S$ as in Section \ref{section-notation}.
Let $(U, R, s, t, c)$ be a groupoid in algebraic spaces over $B$.
\begin{enumerate}
\item If $s$ and $t$ are open, then for every open $W \subset U$
the open $s(t^{-1}(W))$ is $R$-invariant.
\item If $s$ and $t$ are open and quasi-compact, then $U$ has an open
covering consisting of $R$-invariant quasi-compact open subspaces.
\end{enumerate}
\end{lemma}

\begin{proof}
Assume $s$ and $t$ open and $W \subset U$ open.
Since $s$ is open we see that $W' = s(t^{-1}(W))$ is an open subspace of $U$.
Now it is quite easy to using the functorial point of view
that this is an $R$-invariant open subset of $U$, but we are going to argue
this directly by some diagrams, since we think it is instructive.
Note that $t^{-1}(W')$ is the image of the morphism
$$
A := t^{-1}(W) \times_{s|_{t^{-1}(W)}, U, t} R
\xrightarrow{\text{pr}_1} R
$$
and that $s^{-1}(W')$ is the image of the morphism
$$
B := R \times_{s, U, s|_{t^{-1}(W)}} t^{-1}(W)
\xrightarrow{\text{pr}_0} R.
$$
The algebraic spaces $A$, $B$
on the left of the arrows above are open subspaces of
$R \times_{s, U, t} R$ and $R \times_{s, U, s} R$ respectively.
By Lemma \ref{lemma-diagram} the diagram
$$
\xymatrix{
R \times_{s, U, t} R \ar[rd]_{\text{pr}_1} \ar[rr]_{(\text{pr}_1, c)} & &
R \times_{s, U, s} R \ar[ld]^{\text{pr}_0} \\
& R &
}
$$
is commutative, and the horizontal arrow is an isomorphism. Moreover, it is
clear that $(\text{pr}_1, c)(A) = B$. Hence we conclude
$s^{-1}(W') = t^{-1}(W')$, and $W'$ is $R$-invariant. This proves (1).

\medskip\noindent
Assume now that $s$, $t$ are both open and quasi-compact.
Then, if $W \subset U$ is a quasi-compact open, then also
$W' = s(t^{-1}(W))$ is a quasi-compact open, and invariant by the
discussion above. Letting $W$ range over images of affines \'etale over $U$
we see (2).
\end{proof}





\section{Quotient sheaves}
\label{section-quotient-sheaves}

\noindent
Let $S$ be a scheme, and let $B$ be an algebraic space over $S$.
Let $j : R \to U \times_B U$ be a pre-relation over $B$.
For each scheme $S'$ over $S$ we can take the equivalence relation
$\sim_{S'}$ generated by the image of $j(S') : R(S') \to U(S') \times U(S')$.
Hence we get a presheaf
\begin{equation}
\label{equation-quotient-presheaf}
\begin{matrix}
(\Sch/S)^{opp}_{fppf} &
\longrightarrow &
\textit{Sets}, \\
S' &
\longmapsto  &
U(S')/\sim_{S'}
\end{matrix}
\end{equation}
Note that since $j$ is a morphism of algebraic spaces over $B$
and into $U \times_B U$ there is a canonical transformation of
presheaves from the presheaf (\ref{equation-quotient-presheaf}) to $B$.

\begin{definition}
\label{definition-quotient-sheaf}
Let $B \to S$ and the pre-relation $j : R \to U \times_B U$ be as above.
In this setting the {\it quotient sheaf $U/R$} associated
to $j$ is the sheafification of the presheaf
(\ref{equation-quotient-presheaf}) on $(\Sch/S)_{fppf}$.
If $j : R \to U \times_B U$ comes from the action of a
group algebraic space $G$ over $B$ on $U$ as in
Lemma \ref{lemma-groupoid-from-action}
then we denote the quotient sheaf $U/G$.
\end{definition}

\noindent
This means exactly that the diagram
$$
\xymatrix{
R \ar@<1ex>[r] \ar@<-1ex>[r] &
U \ar[r] &
U/R
}
$$
is a coequalizer diagram in the category of sheaves of sets
on $(\Sch/S)_{fppf}$. Again there is a canonical map
of sheaves $U/R \to B$ as $j$ is a morphism of algebraic spaces over
$B$ into $U \times_B U$.

\begin{remark}
\label{remark-quotient-variant}
A variant of the construction above would have been to sheafify
the functor
$$
\begin{matrix}
(\textit{Spaces}/B)^{opp}_{fppf} &
\longrightarrow &
\textit{Sets}, \\
X &
\longmapsto  &
U(X)/\sim_X
\end{matrix}
$$
where now $\sim_X \subset U(X) \times U(X)$ is the equivalence relation
generated by the image of $j : R(X) \to U(X) \times U(X)$.
Here of course $U(X) = \Mor_B(X, U)$ and $R(X) = \Mor_B(X, R)$.
In fact, the result would have been the same, via the identifications
of (insert future reference in Topologies of Spaces here).
\end{remark}

\begin{definition}
\label{definition-representable-quotient}
In the situation of Definition \ref{definition-quotient-sheaf}.
We say that the pre-relation $j$ has a
{\it quotient representable by an algebraic space}
if the sheaf $U/R$ is an algebraic space.
We say that the pre-relation $j$ has a
{\it representable quotient}
if the sheaf $U/R$ is representable by a scheme.
We will say a groupoid in algebraic spaces $(U, R, s, t, c)$ over $B$ has a
{\it representable quotient}
(resp.\ {\it quotient representable by an algebraic space}
if the quotient $U/R$ with $j = (t, s)$ is representable (resp.\ an
algebraic space).
\end{definition}

\noindent
If the quotient $U/R$ is representable by $M$ (either a scheme or an algebraic
space over $S$), then it comes equipped with a canonical structure morphism
$M \to B$ as we've seen above.

\medskip\noindent
The following lemma characterizes $M$ representing the quotient.
It applies for example if $U \to M$ is flat,
of finite presentation and surjective, and $R \cong U \times_M U$.

\begin{lemma}
\label{lemma-criterion-quotient-representable}
In the situation of Definition \ref{definition-quotient-sheaf}.
Assume there is an algebraic space $M$ over $S$,
and a morphism $U \to M$ such that
\begin{enumerate}
\item the morphism $U \to M$ equalizes $s, t$,
\item the map $U \to M$ is a surjection of sheaves, and
\item the induced map $(t, s) : R \to U \times_M U$ is a
surjection of sheaves.
\end{enumerate}
In this case $M$ represents the quotient sheaf $U/R$.
\end{lemma}

\begin{proof}
Condition (1) says that $U \to M$ factors through $U/R$.
Condition (2) says that $U/R \to M$ is surjective as a map of sheaves.
Condition (3) says that $U/R \to M$ is injective as a map of sheaves.
Hence the lemma follows.
\end{proof}

\noindent
The following lemma is wrong if we do not require $j$ to be a
pre-equivalence relation (but just a pre-relation say).

\begin{lemma}
\label{lemma-quotient-pre-equivalence}
Let $S$ be a scheme. Let $B$ be an algebraic space over $S$.
Let $j : R \to U \times_B U$ be a pre-equivalence relation over $B$.
For a scheme $S'$ over $S$ and $a, b \in U(S')$ the following are equivalent:
\begin{enumerate}
\item $a$ and $b$ map to the same element of $(U/R)(S')$, and
\item there exists an fppf covering $\{f_i : S_i \to S'\}$ of $S'$
and morphisms $r_i : S_i \to R$ such that
$a \circ f_i = s \circ r_i$ and $b \circ f_i = t \circ r_i$.
\end{enumerate}
In other words, in this case the map of sheaves
$$
R \longrightarrow U \times_{U/R} U
$$
is surjective.
\end{lemma}

\begin{proof}
Omitted. Hint: The reason this works is that the presheaf
(\ref{equation-quotient-presheaf}) in this case is really given
by $T \mapsto U(T)/j(R(T))$ as $j(R(T)) \subset U(T) \times U(T)$
is an equivalence relation, see
Definition \ref{definition-equivalence-relation}.
\end{proof}

\begin{lemma}
\label{lemma-quotient-pre-equivalence-relation-restrict}
Let $S$ be a scheme. Let $B$ be an algebraic space over $S$.
Let $j : R \to U \times_B U$ be a pre-relation over $B$
and $g : U' \to U$ a morphism of algebraic spaces over $B$.
Let $j' : R' \to U' \times_B U'$ be the restriction of $j$ to $U'$.
The map of quotient sheaves
$$
U'/R' \longrightarrow U/R
$$
is injective. If $U' \to U$ is surjective as a map of sheaves, for
example if $\{g : U' \to U\}$ is an fppf covering (see
Topologies on Spaces,
Definition \ref{spaces-topologies-definition-fppf-covering}),
then $U'/R' \to U/R$ is an isomorphism of sheaves.
\end{lemma}

\begin{proof}
Suppose $\xi, \xi' \in (U'/R')(S')$ are sections which
map to the same section of $U/R$.
Then we can find an fppf covering $\mathcal{S} = \{S_i \to S'\}$ of $S'$
such that $\xi|_{S_i}, \xi'|_{S_i}$ are given by $a_i, a_i' \in U'(S_i)$. By
Lemma \ref{lemma-quotient-pre-equivalence}
and the axioms of a site we may after refining
$\mathcal{T}$ assume there exist morphisms $r_i : S_i \to R$
such that $g \circ a_i = s \circ r_i$, $g \circ a_i' = t \circ r_i$.
Since by construction
$R' = R \times_{U \times_S U} (U' \times_S U')$
we see that $(r_i, (a_i, a_i')) \in R'(S_i)$ and this
shows that $a_i$ and $a_i'$ define the same section
of $U'/R'$ over $S_i$. By the sheaf condition this implies
$\xi = \xi'$.

\medskip\noindent
If $U' \to U$ is a surjective map of sheaves, then $U'/R' \to U/R$ is
surjective also. Finally, if $\{g : U' \to U\}$ is a fppf covering, then
the map of sheaves $U' \to U$ is surjective, see
Topologies on Spaces,
Lemma \ref{spaces-topologies-lemma-fppf-covering-surjective}.
\end{proof}

\begin{lemma}
\label{lemma-quotient-groupoid-restrict}
Let $S$ be a scheme. Let $B$ be an algebraic space over $S$.
Let $(U, R, s, t, c)$ be a groupoid in algebraic spaces over $B$.
Let $g : U' \to U$ a morphism of algebraic spaces over $B$.
Let $(U', R', s', t', c')$ be the restriction of $(U, R, s, t, c)$ to $U'$.
The map of quotient sheaves
$$
U'/R' \longrightarrow U/R
$$
is injective. If the composition
$$
\xymatrix{
U' \times_{g, U, t} R \ar[r]_-{\text{pr}_1} \ar@/^3ex/[rr]^h
& R \ar[r]_s & U
}
$$
is a surjection of fppf sheaves then the map is bijective.
This holds for example if $\{h : U' \times_{g, U, t} R \to U\}$ is an
$fppf$-covering, or if $U' \to U$ is a surjection of sheaves, or if
$\{g : U' \to U\}$ is a covering in the fppf topology.
\end{lemma}

\begin{proof}
Injectivity follows on combining
Lemmas \ref{lemma-groupoid-pre-equivalence} and
\ref{lemma-quotient-pre-equivalence-relation-restrict}.
To see surjectivity (see
Sites, Section \ref{sites-section-sheaves-injective}
for a characterization of surjective maps of sheaves) we argue as follows.
Suppose that $T$ is a scheme and $\sigma \in U/R(T)$.
There exists a covering $\{T_i \to T\}$ such that $\sigma|_{T_i}$
is the image of some element $f_i \in U(T_i)$. Hence we
may assume that $\sigma$ if the image of $f \in U(T)$.
By the assumption that $h$ is a surjection of sheaves, we
can find an fppf covering $\{\varphi_i : T_i \to T\}$ and morphisms
$f_i : T_i \to U' \times_{g, U, t} R$ such that
$f \circ \varphi_i = h \circ f_i$. Denote
$f'_i = \text{pr}_0 \circ f_i : T_i \to U'$. Then we see that
$f'_i \in U'(T_i)$ maps to $g \circ f'_i \in U(T_i)$ and
that $g \circ f'_i \sim_{T_i} h \circ f_i = f \circ \varphi_i$
notation as in (\ref{equation-quotient-presheaf}). Namely, the
element of $R(T_i)$ giving the relation is $\text{pr}_1 \circ f_i$.
This means that the restriction
of $\sigma$ to $T_i$ is in the image of $U'/R'(T_i) \to U/R(T_i)$
as desired.

\medskip\noindent
If $\{h\}$ is an fppf covering, then it induces a surjection of sheaves, see
Topologies on Spaces,
Lemma \ref{spaces-topologies-lemma-fppf-covering-surjective}.
If $U' \to U$ is surjective, then also $h$ is surjective as $s$ has a section
(namely the neutral element $e$ of the groupoid scheme).
\end{proof}





\section{Quotient stacks}
\label{section-stacks}

\noindent
In this section and the next few sections we describe a kind of
generalization of
Section \ref{section-quotient-sheaves}
above and
Groupoids, Section \ref{groupoids-section-quotient-sheaves}.
It is different in the following way:
We are going to take quotient stacks instead of quotient sheaves.

\medskip\noindent
Let us assume we have a scheme $S$, and algebraic space $B$ over $S$
and a groupoid in algebraic spaces $(U, R, s, t, c)$ over $B$. Given these data
we consider the functor
\begin{equation}
\label{equation-quotient-stack}
\begin{matrix}
(\Sch/S)_{fppf}^{opp} &
\longrightarrow &
\textit{Groupoids} \\
S' &
\longmapsto &
(U(S'), R(S'), s, t, c)
\end{matrix}
\end{equation}
By
Categories, Example \ref{categories-example-functor-groupoids}
this ``presheaf in groupoids'' corresponds to a category fibred in groupoids
over $(\Sch/S)_{fppf}$. In this chapter we will denote
this
$$
[U/_{\!p}R] \to (\Sch/S)_{fppf}
$$
where the subscript ${}_p$ is there to distinguish from the quotient stack.

\begin{definition}
\label{definition-quotient-stack}
Quotient stacks. Let $B \to S$ be as above.
\begin{enumerate}
\item Let $(U, R, s, t, c)$ be a groupoid in algebraic spaces over $B$.
The {\it quotient stack}
$$
p : [U/R] \longrightarrow (\Sch/S)_{fppf}
$$
of $(U, R, s, t, c)$ is the stackification (see
Stacks, Lemma \ref{stacks-lemma-stackify-groupoids})
of the category fibred in groupoids $[U/_{\!p}R]$ over
$(\Sch/S)_{fppf}$ associated to
(\ref{equation-quotient-stack}).
\item Let $(G, m)$ be a group algebraic space over $B$.
Let $a : G \times_B X \to X$ be an action of $G$ on an algebraic space
over $B$. The {\it quotient stack}
$$
p : [X/G] \longrightarrow (\Sch/S)_{fppf}
$$
is the quotient stack associated to the groupoid in algebraic spaces
$(X, G \times_B X, s, t, c)$ over $B$ of
Lemma \ref{lemma-groupoid-from-action}.
\end{enumerate}
\end{definition}

\noindent
Thus $[U/R]$ and $[X/G]$ are stacks in groupoids over
$(\Sch/S)_{fppf}$. These stacks will be very important later
on and hence it makes sense to give a detailed description.
Recall that given an algebraic space $X$ over $S$ we use the
notation $\mathcal{S}_X \to (\Sch/S)_{fppf}$ to denote
the stack in sets associated to the sheaf $X$, see
Categories, Lemma \ref{categories-lemma-2-category-fibred-sets}
and
Stacks, Lemma \ref{stacks-lemma-when-stack-in-sets}.

\begin{lemma}
\label{lemma-quotient-stack-arrows}
Assume $B \to S$ and $(U, R, s, t, c)$ as in
Definition \ref{definition-quotient-stack} (1).
There are canonical $1$-morphisms
$\pi : \mathcal{S}_U \to [U/R]$, and $[U/R] \to \mathcal{S}_B$
of stacks in groupoids over $(\Sch/S)_{fppf}$.
The composition $\mathcal{S}_U \to \mathcal{S}_B$ is the $1$-morphism
associated to the structure morphism $U \to B$.
\end{lemma}

\begin{proof}
During this proof let us denote $[U/_{\!p}R]$ the category fibred in
groupoids associated to the presheaf in groupoids
(\ref{equation-quotient-stack}). By construction of the stackification
there is a $1$-morphism $[U/_{\!p}R] \to [U/R]$.
The $1$-morphism $\mathcal{S}_U \to [U/R]$ is simply the composition
$\mathcal{S}_U \to [U/_{\!p}R] \to [U/R]$, where the first arrow
associates to the scheme $S'/S$ and morphism $x : S' \to U$ over $S$
the object $x \in U(S')$ of the fibre category of $[U/_{\!p}R]$
over $S'$.

\medskip\noindent
To construct the $1$-morphism $[U/R] \to \mathcal{S}_B$ it is enough to
construct the $1$-morphism $[U/_{\!p}R] \to \mathcal{S}_B$, see
Stacks, Lemma \ref{stacks-lemma-stackify-groupoids-universal-property}.
On objects over $S'/S$ we just use the map
$$
U(S') \longrightarrow B(S')
$$
coming from the structure morphism $U \to B$.
And clearly, if $a \in R(S')$ is an ``arrow'' with source
$s(a) \in U(S')$ and target $t(a) \in U(S')$, then since
$s$ and $t$ are morphisms {\it over} $B$ these both
map to the same element $\overline{a}$ of $B(S')$. Hence we can map an arrow
$a \in R(S')$ to the identity morphism of $\overline{a}$. (This is
good because the fibre category $(\mathcal{S}_B)_{S'}$ only contains
identities.) We omit the verification that this rule is compatible with
pullback on these split fibred categories, and hence defines a
$1$-morphism $[U/_{\!p}R] \to \mathcal{S}_B$ as desired.

\medskip\noindent
We omit the verification of the last statement.
\end{proof}

\begin{lemma}
\label{lemma-quotient-stack-2-arrow}
Assumptions and notation as in Lemma \ref{lemma-quotient-stack-arrows}.
There exists a canonical $2$-morphism
$\alpha : \pi \circ s \to \pi \circ t$ making the diagram
$$
\xymatrix{
\mathcal{S}_R \ar[r]_s \ar[d]_t & \mathcal{S}_U \ar[d]^\pi \\
\mathcal{S}_U \ar[r]^-\pi & [U/R]
}
$$
$2$-commutative.
\end{lemma}

\begin{proof}
Let $S'$ be a scheme over $S$. Let $r : S' \to R$ be a morphism over $S$.
Then $r \in R(S')$ is an isomorphism between the objects
$s \circ r, t \circ r \in U(S')$. Moreover, this construction is
compatible with pullbacks. This gives a canonical
$2$-morphism $\alpha_p : \pi_p \circ s \to \pi_p \circ t$
where $\pi_p : \mathcal{S}_U \to [U/_{\!p}R]$ is as in the
proof of
Lemma \ref{lemma-quotient-stack-arrows}. Thus even the diagram
$$
\xymatrix{
\mathcal{S}_R \ar[r]_s \ar[d]^t & \mathcal{S}_U \ar[d]^{\pi_p} \\
\mathcal{S}_U \ar[r]^-{\pi_p} & [U/_{\!p}R]
}
$$
is $2$-commutative. Thus a fortiori the diagram
of the lemma is $2$-commutative.
\end{proof}

\begin{remark}
\label{remark-fundamental-square}
In future chapters we will use the ambiguous notation where
instead of writing $\mathcal{S}_X$ for the stack in sets associated
to $X$ we simply write $X$. Using this notation the diagram of
Lemma \ref{lemma-quotient-stack-2-arrow}
becomes the familiar diagram
$$
\xymatrix{
R \ar[r]_s \ar[d]_t & U \ar[d]^\pi \\
U \ar[r]^-\pi & [U/R]
}
$$
In the following sections we will show that this diagram has
many good properties. In particular we will show that it is
a $2$-fibre product
(Section \ref{section-quotient-stack-2-cartesian})
and that it is close to being a $2$-coequalizer of $s$ and $t$
(Section \ref{section-quotient-stacks-2-coequalize}).
\end{remark}




\section{Functoriality of quotient stacks}
\label{section-functoriality-quotient-stacks}

\noindent
A morphism of groupoids in algebraic spaces gives an associated morphism
of quotient stacks.

\begin{lemma}
\label{lemma-quotient-stack-functorial}
Let $S$ be a scheme. Let $B$ be an algebraic space over $S$.
Let $f : (U, R, s, t, c) \to (U', R', s', t', c')$ be a morphism of
groupoids in algebraic spaces over $B$.
Then $f$ induces a canonical $1$-morphism of quotient stacks
$$
[f] : [U/R] \longrightarrow [U'/R'].
$$
\end{lemma}

\begin{proof}
Denote $[U/_{\!p}R]$ and $[U'/_{\!p}R']$ the categories fibred
in groupoids over the base site $(\Sch/S)_{fppf}$ associated to the
functors (\ref{equation-quotient-stack}). It is clear that $f$ defines
a $1$-morphism $[U/_{\!p}R] \to [U'/_{\!p}R']$ which we can compose
with the stackyfication functor for $[U'/R']$ to get $[U/_{\!p}R] \to [U'/R']$.
Then, by the universal property of the stackyfication functor
$[U/_{\!p}R] \to [U/R]$, see
Stacks, Lemma \ref{stacks-lemma-stackify-groupoids-universal-property}
we get $[U/R] \to [U'/R']$.
\end{proof}

\noindent
Let $B \to S$ and $f : (U, R, s, t, c) \to (U', R', s', t', c')$ be as in
Lemma \ref{lemma-quotient-stack-functorial}.
In this situation, we define a third groupoid in algebraic spaces over
$B$ as follows, using the language of $T$-valued points where $T$
is a (varying) scheme over $B$:
\begin{enumerate}
\item $U'' = U \times_{f, U', t'} R'$ so that a $T$-valued point is
a pair $(u, r')$ with $f(u) = t'(r')$,
\item $R'' = R \times_{f \circ s, U', t'} R'$ so that a $T$-valued point is
a pair $(r, r')$ with $f(s(r)) = t'(r')$,
\item $s'' : R'' \to U''$ is given by $s''(r, r') = (s(r), r')$,
\item $t'' : R'' \to U''$ is given by $t''(r, r') = (t(r), c'(f(r), r'))$,
\item $c'' : R'' \times_{s'', U'', t''} R'' \to R''$ is given by
$c''((r_1, r'_1), (r_2, r'_2)) = (c(r_1, r_2), r'_2)$.
\end{enumerate}
The formula for $c''$ makes sense as $s''(r_1, r'_1) = t''(r_2, r'_2)$.
It is clear that $c''$ is associative. The identity $e''$ is given by
$e''(u, r) = (e(u), r)$. The inverse of $(r, r')$ is given by
$(i(r), c'(f(r), r'))$. Thus we do indeed get a groupoid
in algebraic spaces over $B$.

\medskip\noindent
Clearly the maps $U'' \to U$ and $R'' \to R$ define a morphism
$g : (U'', R'', s'', t'', c'') \to (U, R, s, t, c)$
of groupoids in algebraic spaces over $B$. Moreover, the maps
$U'' \to U'$, $(u, r') \mapsto s'(r')$ and
$R'' \to U'$, $(r, r') \mapsto s'(r')$ show that in fact
$(U'', R'', s'', t'', c'')$ is a groupoid in algebraic spaces over $U'$.

\begin{lemma}
\label{lemma-cartesian-square-of-morphism}
Notation and assumption as in
Lemma \ref{lemma-quotient-stack-functorial}.
Let $(U'', R'', s'', t'', c'')$ be the groupoid in algebraic spaces over $B$
constructed above.
There is a $2$-commutative square
$$
\xymatrix{
[U''/R''] \ar[d] \ar[r]_{[g]} & [U/R] \ar[d]^{[f]} \\
\mathcal{S}_{U'} \ar[r] & [U'/R']
}
$$
which identifies $[U''/R'']$ with the $2$-fibre product.
\end{lemma}

\begin{proof}
The maps $[f]$ and $[g]$ come from an application of
Lemma \ref{lemma-quotient-stack-functorial}
and the other two maps come from
Lemma \ref{lemma-quotient-stack-arrows}
(and the fact that $(U'', R'', s'', t'', c'')$ lives over $U'$).
To show the $2$-fibre product property, it suffices to prove the lemma
for the diagram
$$
\xymatrix{
[U''/_{\!p}R''] \ar[d] \ar[r]_{[g]} & [U/_{\!p}R] \ar[d]^{[f]} \\
\mathcal{S}_{U'} \ar[r] & [U'/_{\!p}R']
}
$$
of categories fibred in groupoids, see
Stacks, Lemma
\ref{stacks-lemma-stackification-fibre-product-categories-fibred-in-groupoids}.
In other words, it suffices to show that an object of the $2$-fibre product
$\mathcal{S}_U \times_{[U'/_{\!p}R']} [U/_{\!p}R]$ over $T$ corresponds
to a $T$-valued point of $U''$ and similarly for morphisms. And of course
this is exactly how we constructed $U''$ and $R''$ in the first place.

\medskip\noindent
In detail, an object of $\mathcal{S}_U \times_{[U'/_{\!p}R']} [U/_{\!p}R]$
over $T$ is a triple $(u', u, r')$ where $u'$ is a $T$-valued point of $U'$,
$u$ is a $T$-valued point of $U$, and $r'$ is a morphism from $u'$ to
$f(u)$ in $[U'/R']_T$, i.e., $r'$ is a $T$-valued point of $R$ with
$s'(r') = u'$ and $t'(r') = f(u)$. Clearly we can forget about $u'$
without losing information and we see that these objects are in one-to-one
correspondence with $T$-valued points of $R''$.

\medskip\noindent
Similarly for morphisms: Let $(u'_1, u_1, r'_1)$ and $(u'_2, u_2, r'_2)$ be
two objects of the fibre product over $T$. Then a morphism from
$(u'_2, u_2, r'_2)$ to $(u'_1, u_1, r'_1)$
is given by $(1, r)$ where $1 : u'_1 \to u'_2$ means simply $u'_1 = u'_2$
(this is so because $\mathcal{S}_U$ is fibred in sets), and $r$ is a
$T$-valued point of $R$ with $s(r) = u_2$, $t(r) = u_1$ and moreover
$c'(f(r), r'_2) = r'_1$. Hence the arrow
$$
(1, r) : (u'_2, u_2, r'_2) \to (u'_1, u_1, r'_1)
$$
is completely determined by knowing the pair $(r, r'_2)$. Thus the functor
of arrows is represented by $R''$, and moreover the morphisms $s''$, $t''$,
and $c''$ clearly correspond to source, target and composition in
the $2$-fibre product $\mathcal{S}_U \times_{[U'/_{\!p}R']} [U/_{\!p}R]$.
\end{proof}







\section{The 2-cartesian square of a quotient stack}
\label{section-quotient-stack-2-cartesian}

\noindent
In this section we compute the $\mathit{Isom}$-sheaves for a quotient stack
and we deduce that the defining diagram of a quotient stack is a
$2$-fibre product.

\begin{lemma}
\label{lemma-quotient-stack-morphisms}
Assume $B \to S$, $(U, R, s, t, c)$ and $\pi : \mathcal{S}_U \to [U/R]$
are as in
Lemma \ref{lemma-quotient-stack-arrows}.
Let $S'$ be a scheme over $S$.
Let $x, y \in \Ob([U/R]_{S'})$ be objects of the
quotient stack over $S'$. If $x = \pi(x')$ and $y = \pi(y')$ for
some morphisms $x', y' : S' \to U$, then
$$
\mathit{Isom}(x, y) = S' \times_{(y', x'), U \times_S U} R
$$
as sheaves over $S'$.
\end{lemma}

\begin{proof}
Let $[U/_{\!p}R]$ be the category fibred in groupoids associated to
the presheaf in groupoids (\ref{equation-quotient-stack}) as in the proof of
Lemma \ref{lemma-quotient-stack-arrows}.
By construction the sheaf $\mathit{Isom}(x, y)$ is the sheaf associated
to the presheaf $\mathit{Isom}(x', y')$. On the other hand, by definition
of morphisms in $[U/_{\!p}R]$ we have
$$
\mathit{Isom}(x', y') = S' \times_{(y', x'), U \times_S U} R
$$
and the right hand side is an algebraic space, therefore a sheaf.
\end{proof}

\begin{lemma}
\label{lemma-quotient-stack-2-cartesian}
Assume $B \to S$, $(U, R, s, t, c)$, and $\pi : \mathcal{S}_U \to [U/R]$
are as in
Lemma \ref{lemma-quotient-stack-arrows}.
The $2$-commutative square
$$
\xymatrix{
\mathcal{S}_R \ar[r]_s \ar[d]_t & \mathcal{S}_U \ar[d]^\pi \\
\mathcal{S}_U \ar[r]^-\pi & [U/R]
}
$$
of
Lemma \ref{lemma-quotient-stack-2-arrow}
is a $2$-fibre product of stacks in groupoids of $(\Sch/S)_{fppf}$.
\end{lemma}

\begin{proof}
According to
Stacks, Lemma \ref{stacks-lemma-2-product-stacks-in-groupoids}
the lemma makes sense. It also tells us that we have to show that
the functor
$$
\mathcal{S}_R \longrightarrow \mathcal{S}_U \times_{[U/R]} \mathcal{S}_U
$$
which maps $r : T \to R$ to $(T, t(r), s(r), \alpha(r))$ is an equivalence,
where the right hand side is the $2$-fibre product as described in
Categories, Lemma \ref{categories-lemma-2-product-categories-over-C}.
This is, after spelling out the definitions, exactly the content of
Lemma \ref{lemma-quotient-stack-morphisms}. (Alternative proof: Work out
the meaning of
Lemma \ref{lemma-cartesian-square-of-morphism}
in this situation will give you the result also.)
\end{proof}

\begin{lemma}
\label{lemma-quotient-stack-isom}
Assume $B \to S$ and $(U, R, s, t, c)$ are as in
Definition \ref{definition-quotient-stack} (1).
For any scheme $T$ over $S$ and objects $x, y$ of $[U/R]$ over $T$
the sheaf $\mathit{Isom}(x, y)$ on $(\Sch/T)_{fppf}$ has
the following property: There exists a fppf covering
$\{T_i \to T\}_{i \in I}$ such that
$\mathit{Isom}(x, y)|_{(\Sch/T_i)_{fppf}}$
is representable by an algebraic space.
\end{lemma}

\begin{proof}
Follows immediately from
Lemma \ref{lemma-quotient-stack-morphisms}
and the fact that both $x$ and $y$ locally in the fppf
topology come from objects of $\mathcal{S}_U$ by construction
of the quotient stack.
\end{proof}














\section{The 2-coequalizer property of a quotient stack}
\label{section-quotient-stacks-2-coequalize}

\noindent
On a groupoid we have the composition, which leads to a cocycle
condition for the canonical $2$-morphism of the lemma above.
To give the precise formulation we will use the notation introduced in
Categories, Sections \ref{categories-section-formal-cat-cat} and
\ref{categories-section-2-categories}.

\begin{lemma}
\label{lemma-quotient-stack-cocycle}
Assumptions and notation as in
Lemmas \ref{lemma-quotient-stack-arrows} and
\ref{lemma-quotient-stack-2-arrow}.
The vertical composition of
$$
\xymatrix@C=15pc{
\mathcal{S}_{R \times_{s, U, t} R}
\ruppertwocell^{\pi \circ s \circ \text{pr}_1 = \pi \circ s \circ c}{\ \ \ \ \ \ \alpha \star \text{id}_{\text{pr}_1}}
\ar[r]_(.3){\pi \circ t \circ \text{pr}_1 = \pi \circ s \circ \text{pr}_0}
\rlowertwocell_{\pi \circ t \circ \text{pr}_0 = \pi \circ t \circ c}{\ \ \ \ \ \ \alpha \star \text{id}_{\text{pr}_0}}
&
[U/R]
}
$$
is the $2$-morphism $\alpha \star \text{id}_c$. In a formula
$\alpha \star \text{id}_c =
(\alpha \star \text{id}_{\text{pr}_0})
\circ
(\alpha \star \text{id}_{\text{pr}_1})
$.
\end{lemma}

\begin{proof}
We make two remarks:
\begin{enumerate}
\item The formula
$\alpha \star \text{id}_c = (\alpha \star \text{id}_{\text{pr}_0}) \circ
(\alpha \star \text{id}_{\text{pr}_1})$ only makes sense if you realize
the {\it equalities} $\pi \circ s \circ \text{pr}_1 = \pi \circ s \circ c$,
$\pi \circ t \circ \text{pr}_1 = \pi \circ s \circ \text{pr}_0$, and
$\pi \circ t \circ \text{pr}_0 = \pi \circ t \circ c$. Namely, the second
one implies the vertical composition $\circ$ makes sense, and the other
two guarantee the two sides of the formula are $2$-morphisms with the
same source and target.
\item The reason the lemma holds is that composition in the
category fibred in groupoids $[U/_{\!p}R]$ associated to the presheaf
in groupoids (\ref{equation-quotient-stack}) comes from the composition
law $c : R \times_{s, U, t} R \to R$.
\end{enumerate}
We omit the proof of the lemma.
\end{proof}

\noindent
Note that, in the situation of the lemma, we actually have the equalities
$s \circ \text{pr}_1 = s \circ c$,
$t \circ \text{pr}_1 = s \circ \text{pr}_0$, and
$t \circ \text{pr}_0 = t \circ c$ before composing with $\pi$.
Hence the formula in the lemma below makes sense in exactly the same way
that the formula in the lemma above makes sense.

\begin{lemma}
\label{lemma-quotient-stack-2-coequalizer}
Assumptions and notation as in
Lemmas \ref{lemma-quotient-stack-arrows} and
\ref{lemma-quotient-stack-2-arrow}.
The $2$-commutative diagram of Lemma \ref{lemma-quotient-stack-2-arrow}
is a $2$-coequalizer in the following sense:
Given
\begin{enumerate}
\item a stack in groupoids $\mathcal{X}$ over $(\Sch/S)_{fppf}$,
\item a $1$-morphism $f : \mathcal{S}_U \to \mathcal{X}$, and
\item a $2$-arrow $\beta : f \circ s \to f \circ t$
\end{enumerate}
such that
$$
\beta \star \text{id}_c
=
(\beta \star \text{id}_{\text{pr}_0})
\circ
(\beta \star \text{id}_{\text{pr}_1})
$$
then there exists a $1$-morphism $[U/R] \to \mathcal{X}$ which makes the
diagram
$$
\xymatrix{
\mathcal{S}_R \ar[r]_s \ar[d]^t & \mathcal{S}_U \ar[d] \ar[ddr]^f \\
\mathcal{S}_U \ar[r] \ar[rrd]_f & [U/R] \ar[rd] \\
& & \mathcal{X}
}
$$
$2$-commute.
\end{lemma}

\begin{proof}
Suppose given $\mathcal{X}$, $f$ and $\beta$ as in the lemma. By
Stacks, Lemma \ref{stacks-lemma-stackify-groupoids-universal-property}
it suffices to construct a $1$-morphism $g : [U/_{\!p}R] \to \mathcal{X}$.
First we note that the $1$-morphism
$\mathcal{S}_U \to [U/_{\!p}R]$ is bijective on objects.
Hence on objects we can set $g(x) = f(x)$ for
$x \in \Ob(\mathcal{S}_U) = \Ob([U/_{\!p}R])$.
A morphism $\varphi : x \to y$ of $[U/_{\!p}R]$ arises from a
commutative diagram
$$
\xymatrix{
S_2 \ar[dd]_h \ar[r]_x \ar[dr]_\varphi & U \\
& R \ar[u]_s \ar[d]^t \\
S_1 \ar[r]^y & U.
}
$$
Thus we can set $g(\varphi)$ equal to the composition
$$
\xymatrix{
f(x) \ar@{=}[r] \ar[rrrrrd] &
f(s \circ \varphi) \ar@{=}[r] &
(f \circ s)(\varphi) \ar[r]^\beta &
(f \circ t)(\varphi) \ar@{=}[r] &
f(t \circ \varphi) \ar@{=}[r] &
f(y \circ h) \ar[d] \\
& & & & & f(y).
}
$$
The vertical arrow is the result of applying the functor $f$ to the
canonical morphism $y \circ h \to y$ in $\mathcal{S}_U$ (namely, the
strongly cartesian morphism lifting $h$ with target $y$).
Let us verify that $f$ so defined is compatible with composition, at least
on fibre categories. So let $S'$ be a scheme over $S$, and let
$a : S' \to R \times_{s, U, t} R$ be a morphism. In this situation
we set $x = s \circ \text{pr}_1 \circ a = s \circ c \circ a$,
$y = t \circ \text{pr}_1 \circ a = s \circ \text{pr}_0 \circ a$, and
$z = t \circ \text{pr}_0 \circ a = t \circ \text{pr}_0 \circ c$ to
get a commutative diagram
$$
\xymatrix{
x \ar[rr]_{c \circ a} \ar[rd]_{\text{pr}_1 \circ a} & & z \\
& y \ar[ru]_{\text{pr}_0 \circ a}
}
$$
in the fibre category $[U/_{\!p}R]_{S'}$. Moreover, any commutative
triangle in this fibre category has this form. Then we see by our definitions
above that $f$ maps this to a commutative diagram if and only if
the diagram
$$
\xymatrix{
& (f \circ s)(c \circ a) \ar[r]_-{\beta} &
(f \circ t)(c \circ a) \ar@{=}[rd] & \\
(f \circ s)(\text{pr}_1 \circ a) \ar[rd]^\beta \ar@{=}[ru] & & &
(f \circ t)(\text{pr}_0 \circ a) \\
& (f \circ t)(\text{pr}_1 \circ a) \ar@{=}[r] &
(f \circ s)(\text{pr}_0 \circ a) \ar[ru]^\beta
}
$$
is commutative which is exactly the condition expressed by the formula
in the lemma. We omit the
verification that $f$ maps identities to identities and is compatible
with composition for arbitrary morphisms.
\end{proof}












\section{Explicit description of quotient stacks}
\label{section-explicit-quotient-stacks}

\noindent
In order to formulate the result we need to introduce some
notation. Assume $B \to S$ and $(U, R, s, t, c)$ are as in
Definition \ref{definition-quotient-stack} (1).
Let $T$ be a scheme over $S$. Let $\mathcal{T} = \{T_i \to T\}_{i \in I}$
be an fppf covering. A {\it $[U/R]$-descent datum} relative
to $\mathcal{T}$ is given by a system $(u_i, r_{ij})$ where
\begin{enumerate}
\item for each $i$ a morphism $u_i : T_i \to U$, and
\item for each $i, j$ a morphism $r_{ij} : T_i \times_T T_j \to R$
\end{enumerate}
such that
\begin{enumerate}
\item[(a)] as morphisms $T_i \times_T T_j \to U$ we have
$$
s \circ r_{ij} = u_i \circ \text{pr}_0
\quad\text{and}\quad
t \circ r_{ij} = u_j \circ \text{pr}_1,
$$
\item[(b)] as morphisms $T_i \times_T T_j \times_T T_k \to R$ we have
$$
c \circ (r_{jk} \circ \text{pr}_{12}, r_{ij} \circ \text{pr}_{01})
= r_{ik} \circ \text{pr}_{02}.
$$
\end{enumerate}
A {\it morphism $(u_i, r_{ij}) \to (u'_i, r'_{ij})$ between two
$[U/R]$-descent data} over the same covering $\mathcal{T}$
is a collection $(r_i : T_i \to R)$ such that
\begin{enumerate}
\item[$(\alpha)$] as morphisms $T_i \to U$ we have
$$
u_i = s \circ r_i
\quad\text{and}\quad
u'_i = t \circ r_i
$$
\item[$(\beta)$] as morphisms $T_i \times_T T_j \to R$ we have
$$
c \circ (r'_{ij}, r_i \circ \text{pr}_0)
=
c \circ (r_j \circ \text{pr}_1, r_{ij}).
$$
\end{enumerate}
There is a natural composition law on morphisms of descent data relative
to a fixed covering and we obtain a category of descent data.
This category is a groupoid. Finally, if
$\mathcal{T}' = \{T'_j \to T\}_{j \in J}$
is a second fppf covering which refines $\mathcal{T}$ then there
is a notion of pullback of descent data. This is particularly easy to
describe explicitly in this case. Namely, if $\alpha : J \to I$ and
$\varphi_j : T'_j \to T_{\alpha(i)}$ is the morphism of coverings, then
the pullback of the descent datum $(u_i, r_{ii'})$ is simply
$$
(u_{\alpha(i)} \circ \varphi_j,
r_{\alpha(j)\alpha(j')} \circ \varphi_j \times \varphi_{j'}).
$$
Pullback defined in this manner defines a functor from the category of
descent data over $\mathcal{T}$ to the category of descend data over
$\mathcal{T}'$.

\begin{lemma}
\label{lemma-quotient-stack-objects}
Assume $B \to S$ and $(U, R, s, t, c)$ are as in
Definition \ref{definition-quotient-stack} (1).
Let $\pi : \mathcal{S}_U \to [U/R]$ be as in
Lemma \ref{lemma-quotient-stack-arrows}.
Let $T$ be a scheme over $S$.
\begin{enumerate}
\item for every object $x$ of the fibre category $[U/R]_T$
there exists an fppf covering $\{f_i : T_i \to T\}_{i \in I}$ such that
$f_i^*x \cong \pi(u_i)$ for some $u_i \in U(T_i)$,
\item the composition of the isomorphisms
$$
\pi(u_i \circ \text{pr}_0)
=
\text{pr}_0^*\pi(u_i)
\cong
\text{pr}_0^*f_i^*x
\cong
\text{pr}_1^*f_j^*x
\cong
\text{pr}_1^*\pi(u_j)
=
\pi(u_j \circ \text{pr}_1)
$$
are of the form $\pi(r_{ij})$ for certain morphisms
$r_{ij} : T_i \times_T T_j \to R$,
\item the system $(u_i, r_{ij})$ forms
a $[U/R]$-descent datum as defined above,
\item any $[U/R]$-descent datum $(u_i, r_{ij})$ arises in this manner,
\item if $x$ corresponds to $(u_i, r_{ij})$ as above, and
$y \in \Ob([U/R]_T)$ corresponds to $(u'_i, r'_{ij})$
then there is a canonical bijection
$$
\Mor_{[U/R]_T}(x, y)
\longleftrightarrow
\left\{
\begin{matrix}
\text{morphisms }(u_i, r_{ij}) \to (u'_i, r'_{ij})\\
\text{of }[U/R]\text{-descent data}
\end{matrix}
\right\}
$$
\item this correspondence is compatible with refinements of fppf coverings.
\end{enumerate}
\end{lemma}

\begin{proof}
Statement (1) is part of the construction of the stackyfication.
Part (2) follows from
Lemma \ref{lemma-quotient-stack-morphisms}.
We omit the verification of (3).
Part (4) is a translation of the fact that in a stack all descent data
are effective.
We omit the verifications of (5) and (6).
\end{proof}








\section{Restriction and quotient stacks}
\label{section-quotient-stack-restrict}

\noindent
In this section we study what happens to the quotient stack
when taking a restriction.

\begin{lemma}
\label{lemma-quotient-stack-restrict}
Notation and assumption as in
Lemma \ref{lemma-quotient-stack-functorial}.
The morphism of quotient stacks
$$
[f] : [U/R] \longrightarrow [U'/R']
$$
is fully faithful if and only if $R'$ is the restriction of
$R$ via the morphism $f : U \to U'$.
\end{lemma}

\begin{proof}
Let $x, y$ be objects of $[U/R]$ over a scheme $T/S$.
Let $x', y'$ be the images of $x, y$ in the category $[U'/'R]_T$.
The functor $[f]$ is fully faithful if and only if the map of sheaves
$$
\mathit{Isom}(x, y) \longrightarrow \mathit{Isom}(x', y')
$$
is an isomorphism for every $T, x, y$. We may test this locally on $T$
(in the fppf topology). Hence, by
Lemma \ref{lemma-quotient-stack-objects}
we may assume that $x, y$ come from $a, b \in U(T)$.
In that case we see that $x', y'$ correspond to $f \circ a, f \circ b$. By
Lemma \ref{lemma-quotient-stack-morphisms}
the displayed map of sheaves in this case becomes
$$
T \times_{(a, b), U \times_B U} R
\longrightarrow
T \times_{f \circ a, f \circ b, U' \times_B U'} R'.
$$
This is an isomorphism if $R$ is the restriction, because in that case
$R = (U \times_B U) \times_{U' \times_B U'} R'$, see
Lemma \ref{lemma-restrict-groupoid-relation}
and its proof. Conversely, if the last displayed map is an isomorphism
for all $T, a, b$, then it follows that
$R = (U \times_B U) \times_{U' \times_B U'} R'$, i.e.,
$R$ is the restriction of $R'$.
\end{proof}

\begin{lemma}
\label{lemma-quotient-stack-restrict-equivalence}
Notation and assumption as in
Lemma \ref{lemma-quotient-stack-functorial}.
The morphism of quotient stacks
$$
[f] : [U/R] \longrightarrow [U'/R']
$$
is an equivalence if and only if
\begin{enumerate}
\item $(U, R, s, t, c)$ is the restriction of $(U', R', s', t', c')$
via $f : U \to U'$, and
\item the map
$$
\xymatrix{
U \times_{f, U', t'} R' \ar[r]_-{\text{pr}_1} \ar@/^3ex/[rr]^h
& R' \ar[r]_{s'} & U'
}
$$
is a surjection of sheaves.
\end{enumerate}
Part (2) holds for example if $\{h : U \times_{f, U', t'} R' \to U'\}$
is an fppf covering, or if $f : U \to U'$ is a surjection of sheaves, or if
$\{f : U \to U'\}$ is an fppf covering.
\end{lemma}

\begin{proof}
We already know that part (1) is equivalent to
fully faithfulness by
Lemma \ref{lemma-quotient-stack-restrict}.
Hence we may assume that (1) holds and that $[f]$ is fully faithful.
Our goal is to show, under these assumptions, that $[f]$ is an equivalence
if and only if (2) holds. We may use
Stacks, Lemma \ref{stacks-lemma-characterize-essentially-surjective-when-ff}
which characterizes equivalences.

\medskip\noindent
Assume (2). We will use
Stacks, Lemma \ref{stacks-lemma-characterize-essentially-surjective-when-ff}
to prove $[f]$ is an equivalence.
Suppose that $T$ is a scheme and $x' \in \Ob([U'/R']_T)$.
There exists a covering $\{g_i : T_i \to T\}$ such that $g_i^*x'$
is the image of some element $a'_i \in U'(T_i)$, see
Lemma \ref{lemma-quotient-stack-objects}.
Hence we may assume that $x'$ is the image of $a' \in U'(T)$.
By the assumption that $h$ is a surjection of sheaves, we
can find an fppf covering $\{\varphi_i : T_i \to T\}$ and morphisms
$b_i : T_i \to U \times_{g, U', t'} R'$ such that
$a' \circ \varphi_i = h \circ b_i$. Denote
$a_i = \text{pr}_0 \circ b_i : T_i \to U$. Then we see that
$a_i \in U(T_i)$ maps to $f \circ a_i \in U'(T_i)$ and
that $f \circ a_i \cong_{T_i} h \circ b_i = a' \circ \varphi_i$,
where $\cong_{T_i}$ denotes isomorphism in the fibre category
$[U'/R']_{T_i}$. Namely, the element of $R'(T_i)$ giving the isomorphism
is $\text{pr}_1 \circ b_i$. This means that the restriction
of $x$ to $T_i$ is in the essential image of the functor
$[U/R]_{T_i} \to [U'/R']_{T_i}$ as desired.

\medskip\noindent
Assume $[f]$ is an equivalence. Let $\xi' \in [U'/R']_{U'}$ denote the
object corresponding to the identity morphism of $U'$. Applying
Stacks, Lemma \ref{stacks-lemma-characterize-essentially-surjective-when-ff}
we see there exists an fppf covering $\mathcal{U}' = \{g'_i : U'_i \to U'\}$
such that $(g'_i)^*\xi' \cong [f](\xi_i)$ for some $\xi_i$ in $[U/R]_{U'_i}$.
After refining the covering $\mathcal{U}'$ (using
Lemma \ref{lemma-quotient-stack-objects})
we may assume $\xi_i$ comes from a morphism $a_i : U'_i \to U$.
The fact that $[f](\xi_i) \cong (g'_i)^*\xi'$ means that, after possibly
refining the covering $\mathcal{U}'$ once more, there exist morphisms
$r'_i : U'_i \to R'$ with $t' \circ r'_i = f \circ a_i$ and
$s' \circ r'_i = \text{id}_{U'} \circ g'_i$. Picture
$$
\xymatrix{
U \ar[d]^f & & U'_i \ar[ll]^{a_i} \ar[ld]_{r'_i} \ar[d]^{g'_i} \\
U' & R' \ar[l]_{t'} \ar[r]^{s'} & U'
}
$$
Thus $(a_i, r'_i) : U'_i \to U \times_{g, U', t'} R'$ are morphisms
such that $h \circ (a_i, r'_i) = g'_i$ and we conclude that
$\{h : U \times_{g, U', t'} R' \to U'\}$ can be refined by the
fppf covering $\mathcal{U}'$ which means that $h$ induces a surjection
of sheaves, see
Topologies on Spaces, Lemma
\ref{spaces-topologies-lemma-fppf-covering-surjective}.

\medskip\noindent
If $\{h\}$ is an fppf covering, then it induces a surjection of sheaves, see
Topologies on Spaces,
Lemma \ref{spaces-topologies-lemma-fppf-covering-surjective}.
If $U' \to U$ is surjective, then also $h$ is surjective as $s$ has a section
(namely the neutral element $e$ of the groupoid in algebraic spaces).
\end{proof}

\begin{lemma}
\label{lemma-criterion-fibre-product}
Notation and assumption as in
Lemma \ref{lemma-quotient-stack-functorial}.
Assume that
$$
\xymatrix{
R \ar[d]_s \ar[r]_f & R' \ar[d]^{s'} \\
U \ar[r]^f & U'
}
$$
is cartesian. Then
$$
\xymatrix{
\mathcal{S}_U \ar[d] \ar[r] & [U/R] \ar[d]^{[f]} \\
\mathcal{S}_{U'} \ar[r] & [U'/R']
}
$$
is a $2$-fibre product square.
\end{lemma}

\begin{proof}
Applying the inverse isomorphisms $i : R \to R$ and $i' : R' \to R'$
to the (first) cartesian diagram of the statement of the lemma we see
that
$$
\xymatrix{
R \ar[d]_t \ar[r]_f & R' \ar[d]^{t'} \\
U \ar[r]^f & U'
}
$$
is cartesian as well. By
Lemma \ref{lemma-cartesian-square-of-morphism}
we have a $2$-fibre square
$$
\xymatrix{
[U''/R''] \ar[d] \ar[r] & [U/R] \ar[d] \\
\mathcal{S}_{U'} \ar[r] & [U'/R']
}
$$
where $U'' = U \times_{f, U', t'} R'$ and
$R'' = R \times_{f \circ s, U', t'} R'$. By the above
we see that $(t, f) : R \to U''$ is an isomorphism, and that
$$
R'' =
R \times_{f \circ s, U', t'} R' =
R \times_{s, U} U \times_{f, U', t'} R' =
R \times_{s, U, t} \times R.
$$
Explicitly the isomorphism $R \times_{s, U, t} R \to R''$ is given by
the rule $(r_0, r_1) \mapsto (r_0, f(r_1))$.
Moreover, $s'', t'', c''$ translate into the maps
$$
R \times_{s, U, t} R \to R,
\quad
s''(r_0, r_1) = r_1, \quad t''(r_0, r_1) = c(r_0, r_1)
$$
and
$$
\begin{matrix}
c'' : &
(R \times_{s, U, t} R) \times_{s'', R, t''} (R \times_{s, U, t} R)
&
\longrightarrow
&
R \times_{s, U, t} R, \\
&
((r_0, r_1), (r_2, r_3)) &
\longmapsto & (c(r_0, r_2), r_3).
\end{matrix}
$$
Precomposing with the isomorphism
$$
R \times_{s, U, s} R \longrightarrow R \times_{s, U, t} R,
\quad
(r_0, r_1) \longmapsto (c(r_0, i(r_1)), r_1)
$$
we see that $t''$ and $s''$ turn into
$\text{pr}_0$ and $\text{pr}_1$ and that $c''$ turns into
$\text{pr}_{02} :
R \times_{s, U, s} R \times_{s, U, s} R \to R \times_{s, U, s} R$.
Hence we see that there is an isomorphism
$[U''/R''] \cong [R/R \times_{s, U, s} R]$ where as a groupoid
in algebraic spaces $(R, R \times_{s, U, s} R, s'', t'', c'')$
is the restriction of the trivial groupoid
$(U, U, \text{id}, \text{id}, \text{id})$ via $s : R \to U$.
Since $s : R \to U$ is a surjection of fppf sheaves (as it has a
right inverse) the morphism
$$
[U''/R''] \cong [R/R \times_{s, U, s} R]
\longrightarrow
[U/U] = \mathcal{S}_U
$$
is an equivalence by
Lemma \ref{lemma-quotient-stack-restrict-equivalence}.
This proves the lemma.
\end{proof}





\section{Inertia and quotient stacks}
\label{section-inertia}

\noindent
The (relative) inertia stack of a stack in groupoids is defined in
Stacks, Section \ref{stacks-section-the-inertia-stack}.
The actual construction, in the setting of fibred categories, and some
of its properties is in
Categories, Section \ref{categories-section-inertia}.

\begin{lemma}
\label{lemma-presentation-inertia}
Assume $B \to S$ and $(U, R, s, t, c)$ as in
Definition \ref{definition-quotient-stack} (1).
Let $G/U$ be the stabilizer group algebraic space of the groupoid
$(U, R, s, t, c, e, i)$, see
Definition \ref{definition-stabilizer-groupoid}.
Set $R' = R \times_{s, U} G$ and set
\begin{enumerate}
\item $s' : R' \to G$, $(r, g) \mapsto g$,
\item $t' : R' \to G$, $(r, g) \mapsto c(r, c(g, i(r)))$,
\item $c' : R' \times_{s', G, t'} R' \to R'$,
$((r_1, g_1), (r_2, g_2) \mapsto (c(r_1, r_2), g_1)$.
\end{enumerate}
Then $(G, R', s', t', c')$ is a groupoid in algebraic spaces over $B$
and
$$
\mathcal{I}_{[U/R]} = [G/ R'].
$$
i.e., the associated quotient stack is the inertia stack of $[U/R]$.
\end{lemma}

\begin{proof}
By
Stacks, Lemma \ref{stacks-lemma-stackification-inertia}
it suffices to prove that $\mathcal{I}_{[U/_{\!p}R]} = [G/_{\!p} R']$.
Let $T$ be a scheme over $S$. Recall that an object of the inertia fibred
category of $[U/_{\!p}R]$ over $T$ is given by a pair
$(x, g)$ where $x$ is an object of $[U/_{\!\!p}R]$ over $T$
and $g$ is an automorphism of $x$ in its fibre category over $T$.
In other words, $x : T \to U$ and $g : T \to R$ such that
$x = s \circ g = t \circ g$. This means exactly that
$g : T \to G$. A morphism in the inertia fibred category
from $(x, g) \to (y, h)$ over $T$ is given by
$r : T \to R$ such that $s(r) = x$, $t(r) = y$
and $c(r, g) = c(h, r)$, see the commutative diagram in
Categories, Lemma \ref{categories-lemma-inertia-fibred-category}.
In a formula
$$
h = c(r, c(g, i(r))) = c(c(r, g), i(r)).
$$
The notation $s(r)$, etc is a short hand for $s \circ r$, etc.
The composition of $r_1 : (x_2, g_2) \to (x_1, g_1)$
and $r_2 : (x_1, g_1) \to (x_2, g_2)$ is
$c(r_1, r_2) : (x_1, g_1) \to (x_3, g_3)$.

\medskip\noindent
Note that in the above we could have written $g$ in stead of $(x, g)$
for an object of $\mathcal{I}_{[U/_{\!p}R]}$ over $T$ as $x$ is the
image of $g$ under the structure morphism $G \to U$. Then the morphisms
$g \to h$ in $\mathcal{I}_{[U/_{\!p}R]}$ over $T$ correspond exactly
to morphisms $r' : T \to R'$ with $s'(r') = g$ and $t'(r') = h$.
Moreover, the composition corresponds to the rule explained in (3).
Thus the lemma is proved.
\end{proof}

\begin{lemma}
\label{lemma-2-cartesian-inertia}
Assume $B \to S$ and $(U, R, s, t, c)$ as in
Definition \ref{definition-quotient-stack} (1).
Let $G/U$ be the stabilizer group algebraic space of the groupoid
$(U, R, s, t, c, e, i)$, see
Definition \ref{definition-stabilizer-groupoid}.
There is a canonical $2$-cartesian diagram
$$
\xymatrix{
\mathcal{S}_G \ar[r] \ar[d] & \mathcal{S}_U \ar[d] \\
\mathcal{I}_{[U/R]} \ar[r] & [U/R]
}
$$
of stacks in groupoids of $(\Sch/S)_{fppf}$.
\end{lemma}

\begin{proof}
By
Lemma \ref{lemma-criterion-fibre-product}
it suffices to prove that the morphism
$s' : R' \to G$ of
Lemma \ref{lemma-presentation-inertia}
isomorphic to the base change of $s$ by the structure
morphism $G \to U$. This base change property is
clear from the construction of $s'$.
\end{proof}







\section{Gerbes and quotient stacks}
\label{section-gerbes}

\noindent
In this section we relate quotient stacks to the discussion
Stacks, Section \ref{stacks-section-gerbes}
and especially gerbes as defined in
Stacks, Definition \ref{stacks-definition-gerbe-over-stack-in-groupoids}.
The stacks in groupoids occurring in this section
are generally speaking not algebraic stacks!

\begin{lemma}
\label{lemma-when-gerbe}
Notation and assumption as in
Lemma \ref{lemma-quotient-stack-functorial}.
The morphism of quotient stacks
$$
[f] : [U/R] \longrightarrow [U'/R']
$$
turns $[U/R]$ into a gerbe over $[U'/R']$ if $f : U \to U'$ and
$R \to R'|_U$ are surjective maps of fppf sheaves. Here $R'|_U$ is
the restriction of $R'$ to $U$ via $f : U \to U'$.
\end{lemma}

\begin{proof}
We will verify that
Stacks, Lemma \ref{stacks-lemma-when-gerbe}
properties (2) (a) and (2) (b) hold. Property (2)(a) holds because $U \to U'$
is a surjective map of sheaves (use
Lemma \ref{lemma-quotient-stack-objects}
to see that objects in $[U'/R']$ locally come from $U'$).
To prove (2)(b) let $x, y$ be objects of $[U/R]$ over a scheme $T/S$.
Let $x', y'$ be the images of $x, y$ in the category $[U'/'R]_T$.
Condition (2)(b) requires us to check the map of sheaves
$$
\mathit{Isom}(x, y) \longrightarrow \mathit{Isom}(x', y')
$$
on $(\Sch/T)_{fppf}$ is surjective. To see this we may work
fppf locally on $T$ and assume that come from $a, b \in U(T)$.
In that case we see that $x', y'$ correspond to $f \circ a, f \circ b$. By
Lemma \ref{lemma-quotient-stack-morphisms}
the displayed map of sheaves in this case becomes
$$
T \times_{(a, b), U \times_B U} R
\longrightarrow
T \times_{f \circ a, f \circ b, U' \times_B U'} R' =
T \times_{(a, b), U \times_B U} R'|_U.
$$
Hence the assumption that $R \to R'|_U$ is a surjective map of fppf sheaves
on $(\Sch/S)_{fppf}$ implies the desired surjectivity.
\end{proof}

\begin{lemma}
\label{lemma-group-quotient-gerbe}
Let $S$ be a scheme. Let $B$ be an algebraic space over $S$. Let
$G$ be a group algebraic space over $B$. Endow $B$ with the trivial
action of $G$. The morphism
$$
[B/G] \longrightarrow \mathcal{S}_B
$$
(Lemma \ref{lemma-quotient-stack-arrows})
turns $[B/G]$ into a gerbe over $B$.
\end{lemma}

\begin{proof}
Immediate from
Lemma \ref{lemma-when-gerbe}
as the morphisms $B \to B$ and $B \times_B G \to B$ are surjective
as morphisms of sheaves.
\end{proof}








\section{Quotient stacks and change of big site}
\label{section-bigger-site}

\noindent
We suggest skipping this section on a first reading.
Pullbacks of stacks are defined in
Stacks, Section \ref{stacks-section-inverse-image}.

\begin{lemma}
\label{lemma-quotient-stack-change-big-site}
Suppose given big sites $\Sch_{fppf}$ and $\Sch'_{fppf}$.
Assume that $\Sch_{fppf}$ is contained in $\Sch'_{fppf}$,
see Topologies, Section \ref{topologies-section-change-alpha}.
Let $S \in \Ob(\Sch_{fppf})$.
Let $B, U, R \in \Sh((\Sch/S)_{fppf})$ be algebraic spaces,
and let $(U, R, s, t, c)$ be a groupoid in algebraic spaces over $B$.
Let $f : (\Sch'/S)_{fppf} \to (\Sch/S)_{fppf}$ the morphism
of sites corresponding to the inclusion functor
$u : \Sch_{fppf} \to \Sch'_{fppf}$.
Then we have a canonical equivalence
$$
[f^{-1}U/f^{-1}R]
\longrightarrow
f^{-1}[U/R]
$$
of stacks in groupoids over $(\Sch'/S)_{fppf}$.
\end{lemma}

\begin{proof}
Note that $f^{-1}B, f^{-1}U, f^{-1}R \in \Sh((\Sch'/S)_{fppf})$
are algebraic spaces by
Spaces, Lemma \ref{spaces-lemma-change-big-site}
and hence $(f^{-1}U, f^{-1}R, f^{-1}s, f^{-1}t, f^{-1}c)$
is a groupoid in algebraic spaces over $f^{-1}B$. Thus the statement makes
sense.

\medskip\noindent
The category $u_p[U/_{\!p}R]$ is the localization of the category
$u_{pp}[U/_{\!p}R]$ at right multiplicative system $I$ of morphisms.
An object of $u_{pp}[U/_{\!p}R]$ is a triple
$$
(T', \phi : T' \to T, x)
$$
where
$T' \in \Ob((\Sch'/S)_{fppf})$,
$T \in \Ob((\Sch/S)_{fppf})$, $\phi$ is a morphism of schemes
over $S$, and $x : T \to U$ is a morphism of sheaves on
$(\Sch/S)_{fppf}$. Note that the morphism of
schemes $\phi : T' \to T$ is the same thing as a morphism
$\phi : T' \to u(T)$, and since $u(T)$ represents $f^{-1}T$ it is the
same thing as a morphism $T' \to f^{-1}T$. Moreover, as $f^{-1}$ on
algebraic spaces is fully faithful, see
Spaces, Lemma \ref{spaces-lemma-fully-faithful},
we may think of $x$ as a morphism $x : f^{-1}T \to f^{-1}U$ as well.
From now on we will make such identifications without further mention.
A morphism
$$
(a, a', \alpha) :
(T'_1, \phi_1 : T'_1 \to T_1, x_1)
\longrightarrow
(T'_2, \phi_2 : T'_2 \to T_2, x_2)
$$
of $u_{pp}[U/_{\!p}R]$ is a commutative diagram
$$
\xymatrix{
& & U \\
T'_1 \ar[d]_{a'} \ar[r]_{\phi_1} &
T_1 \ar[d]_a \ar[ru]^{x_1} \ar[r]_\alpha &
R \ar[d]^t \ar[u]_s \\
T'_2 \ar[r]^{\phi_2} &
T_2 \ar[r]^{x_2} &
U
}
$$
and such a morphism is an element of $I$ if and only if
$T'_1 = T'_2$ and $a' = \text{id}$.
We define a functor
$$
u_{pp}[U/_{\!p}R] \longrightarrow [f^{-1}U/_{\!p}f^{-1}R]
$$
by the rules
$$
(T', \phi : T' \to T, x) \longmapsto (x \circ \phi : T' \to f^{-1}U)
$$
on objects and
$$
(a, a', \alpha) \longmapsto (\alpha \circ \phi_1 : T'_1 \to f^{-1}R)
$$
on morphisms as above. It is clear that elements of $I$ are transformed
into isomorphisms as $(f^{-1}U, f^{-1}R, f^{-1}s, f^{-1}t, f^{-1}c)$
is a groupoid in algebraic spaces over $f^{-1}B$. Hence this functor
factors in a canonical way through a functor
$$
u_p[U/_{\!p}R] \longrightarrow [f^{-1}U/_{\!p}f^{-1}R]
$$
Applying stackification we obtain a functor of stacks
$$
f^{-1}[U/R] \longrightarrow [f^{-1}U/f^{-1}R]
$$
over $(\Sch'/S)_{fppf}$, as by
Stacks, Lemma \ref{stacks-lemma-technical-up}
the stack $f^{-1}[U/R]$ is the stackification of $u_p[U/_{\!p}R]$.

\medskip\noindent
At this point we have a morphism of stacks, and to verify that it is an
equivalence it suffices to show that it is fully faithful and that
objects are locally in the essential image, see
Stacks, Lemmas \ref{stacks-lemma-characterize-ff} and
\ref{stacks-lemma-characterize-essentially-surjective-when-ff}.
The statement on objects holds as $f^{-1}R$ admits a surjective \'etale
morphism $f^{-1}W \to f^{-1}R$ for some object $W$ of
$(\Sch/S)_{fppf}$. To show that the functor is ``full'', it
suffices to show that morphisms are locally in the image
of the functor which holds as $f^{-1}U$ admits a surjective \'etale morphism
$f^{-1}W \to f^{-1}U$ for some object $W$ of $(\Sch/S)_{fppf}$.
We omit the proof that the functor is faithful.
\end{proof}










\section{Separation conditions}
\label{section-separation}

\noindent
This really means conditions on the morphism $j : R \to U \times_B U$
when given a groupoid in algebraic spaces $(U, R, s, t, c)$ over $B$.
As in the previous section we first formulate the corresponding diagram.

\begin{lemma}
\label{lemma-diagram-diagonal}
Let $B \to S$ be as in Section \ref{section-notation}.
Let $(U, R, s, t, c)$ be a groupoid in algebraic spaces over $B$.
Let $G \to U$ be the stabilizer group algebraic space.
The commutative diagram
$$
\xymatrix{
R \ar[d]^{\Delta_{R/U \times_B U}} \ar[rrr]_{f \mapsto (f, s(f))} & & &
R \times_{s, U} U \ar[d] \ar[r] & U \ar[d] \\
R \times_{(U \times_B U)} R \ar[rrr]^{(f, g) \mapsto (f, f^{-1} \circ g)} & & &
R \times_{s, U} G \ar[r] & G
}
$$
the two left horizontal arrows are isomorphisms
and the right square is a fibre product square.
\end{lemma}

\begin{proof}
Omitted.
Exercise in the definitions and the functorial point of
view in algebraic geometry.
\end{proof}

\begin{lemma}
\label{lemma-diagonal}
Let $B \to S$ be as in Section \ref{section-notation}.
Let $(U, R, s, t, c)$ be a groupoid in algebraic spaces over $B$.
Let $G \to U$ be the stabilizer group algebraic space.
\begin{enumerate}
\item The following are equivalent
\begin{enumerate}
\item $j : R \to U \times_B U$ is separated,
\item $G \to U$ is separated, and
\item $e : U \to G$ is a closed immersion.
\end{enumerate}
\item The following are equivalent
\begin{enumerate}
\item $j : R \to U \times_B U$ is locally separated,
\item $G \to U$ is locally separated, and
\item $e : U \to G$ is an immersion.
\end{enumerate}
\item The following are equivalent
\begin{enumerate}
\item $j : R \to U \times_B U$ is quasi-separated,
\item $G \to U$ is quasi-separated, and
\item $e : U \to G$ is quasi-compact.
\end{enumerate}
\end{enumerate}
\end{lemma}

\begin{proof}
The group algebraic space $G \to U$ is the base change of $R \to U \times_B U$
by the diagonal morphism $U \to U \times_B U$, see
Lemma \ref{lemma-groupoid-stabilizer}. Hence if
$j$ is separated (resp.\ locally separated, resp.\ quasi-separated),
then $G \to U$ is separated (resp.\ locally separated, resp.\ quasi-separated).
See
Morphisms of Spaces, Lemma
\ref{spaces-morphisms-lemma-base-change-separated}.
Thus (a) $\Rightarrow$ (b) in (1), (2), and (3).

\medskip\noindent
Conversely, if $G \to U$ is separated
(resp.\ locally separated, resp.\ quasi-separated), then the morphism
$e : U \to G$, as a section of the structure morphism $G \to U$ is a closed
immersion (resp.\ an immersion, resp.\ quasi-compact), see
Morphisms of Spaces, Lemma \ref{spaces-morphisms-lemma-section-immersion}.
Thus (b) $\Rightarrow$ (c) in (1), (2), and (3).

\medskip\noindent
If $e$ is a closed immersion (resp.\ an immersion, resp.\ quasi-compact)
then by the result of
Lemma \ref{lemma-diagram-diagonal}
(and
Spaces, Lemma \ref{spaces-lemma-base-change-immersions}, and
Morphisms of Spaces,
Lemma \ref{spaces-morphisms-lemma-base-change-quasi-compact})
we see that $\Delta_{R/U \times_B U}$ is a closed
immersion (resp.\ an immersion, resp.\ quasi-compact).
Thus (c) $\Rightarrow$ (a) in (1), (2), and (3).
\end{proof}















\begin{multicols}{2}[\section{Other chapters}]
\noindent
Preliminaries
\begin{enumerate}
\item \hyperref[introduction-section-phantom]{Introduction}
\item \hyperref[conventions-section-phantom]{Conventions}
\item \hyperref[sets-section-phantom]{Set Theory}
\item \hyperref[categories-section-phantom]{Categories}
\item \hyperref[topology-section-phantom]{Topology}
\item \hyperref[sheaves-section-phantom]{Sheaves on Spaces}
\item \hyperref[sites-section-phantom]{Sites and Sheaves}
\item \hyperref[stacks-section-phantom]{Stacks}
\item \hyperref[fields-section-phantom]{Fields}
\item \hyperref[algebra-section-phantom]{Commutative Algebra}
\item \hyperref[brauer-section-phantom]{Brauer Groups}
\item \hyperref[homology-section-phantom]{Homological Algebra}
\item \hyperref[derived-section-phantom]{Derived Categories}
\item \hyperref[simplicial-section-phantom]{Simplicial Methods}
\item \hyperref[more-algebra-section-phantom]{More on Algebra}
\item \hyperref[smoothing-section-phantom]{Smoothing Ring Maps}
\item \hyperref[modules-section-phantom]{Sheaves of Modules}
\item \hyperref[sites-modules-section-phantom]{Modules on Sites}
\item \hyperref[injectives-section-phantom]{Injectives}
\item \hyperref[cohomology-section-phantom]{Cohomology of Sheaves}
\item \hyperref[sites-cohomology-section-phantom]{Cohomology on Sites}
\item \hyperref[dga-section-phantom]{Differential Graded Algebra}
\item \hyperref[dpa-section-phantom]{Divided Power Algebra}
\item \hyperref[hypercovering-section-phantom]{Hypercoverings}
\end{enumerate}
Schemes
\begin{enumerate}
\setcounter{enumi}{24}
\item \hyperref[schemes-section-phantom]{Schemes}
\item \hyperref[constructions-section-phantom]{Constructions of Schemes}
\item \hyperref[properties-section-phantom]{Properties of Schemes}
\item \hyperref[morphisms-section-phantom]{Morphisms of Schemes}
\item \hyperref[coherent-section-phantom]{Cohomology of Schemes}
\item \hyperref[divisors-section-phantom]{Divisors}
\item \hyperref[limits-section-phantom]{Limits of Schemes}
\item \hyperref[varieties-section-phantom]{Varieties}
\item \hyperref[topologies-section-phantom]{Topologies on Schemes}
\item \hyperref[descent-section-phantom]{Descent}
\item \hyperref[perfect-section-phantom]{Derived Categories of Schemes}
\item \hyperref[more-morphisms-section-phantom]{More on Morphisms}
\item \hyperref[flat-section-phantom]{More on Flatness}
\item \hyperref[groupoids-section-phantom]{Groupoid Schemes}
\item \hyperref[more-groupoids-section-phantom]{More on Groupoid Schemes}
\item \hyperref[etale-section-phantom]{\'Etale Morphisms of Schemes}
\end{enumerate}
Topics in Scheme Theory
\begin{enumerate}
\setcounter{enumi}{40}
\item \hyperref[chow-section-phantom]{Chow Homology}
\item \hyperref[intersection-section-phantom]{Intersection Theory}
\item \hyperref[pic-section-phantom]{Picard Schemes of Curves}
\item \hyperref[adequate-section-phantom]{Adequate Modules}
\item \hyperref[dualizing-section-phantom]{Dualizing Complexes}
\item \hyperref[duality-section-phantom]{Duality for Schemes}
\item \hyperref[discriminant-section-phantom]{Discriminants and Differents}
\item \hyperref[local-cohomology-section-phantom]{Local Cohomology}
\item \hyperref[curves-section-phantom]{Algebraic Curves}
\item \hyperref[resolve-section-phantom]{Resolution of Surfaces}
\item \hyperref[models-section-phantom]{Semistable Reduction}
\item \hyperref[pione-section-phantom]{Fundamental Groups of Schemes}
\item \hyperref[etale-cohomology-section-phantom]{\'Etale Cohomology}
\item \hyperref[ssgroups-section-phantom]{Linear Algebraic Groups}
\item \hyperref[crystalline-section-phantom]{Crystalline Cohomology}
\item \hyperref[proetale-section-phantom]{Pro-\'etale Cohomology}
\end{enumerate}
Algebraic Spaces
\begin{enumerate}
\setcounter{enumi}{56}
\item \hyperref[spaces-section-phantom]{Algebraic Spaces}
\item \hyperref[spaces-properties-section-phantom]{Properties of Algebraic Spaces}
\item \hyperref[spaces-morphisms-section-phantom]{Morphisms of Algebraic Spaces}
\item \hyperref[decent-spaces-section-phantom]{Decent Algebraic Spaces}
\item \hyperref[spaces-cohomology-section-phantom]{Cohomology of Algebraic Spaces}
\item \hyperref[spaces-limits-section-phantom]{Limits of Algebraic Spaces}
\item \hyperref[spaces-divisors-section-phantom]{Divisors on Algebraic Spaces}
\item \hyperref[spaces-over-fields-section-phantom]{Algebraic Spaces over Fields}
\item \hyperref[spaces-topologies-section-phantom]{Topologies on Algebraic Spaces}
\item \hyperref[spaces-descent-section-phantom]{Descent and Algebraic Spaces}
\item \hyperref[spaces-perfect-section-phantom]{Derived Categories of Spaces}
\item \hyperref[spaces-more-morphisms-section-phantom]{More on Morphisms of Spaces}
\item \hyperref[spaces-flat-section-phantom]{Flatness on Algebraic Spaces}
\item \hyperref[spaces-groupoids-section-phantom]{Groupoids in Algebraic Spaces}
\item \hyperref[spaces-more-groupoids-section-phantom]{More on Groupoids in Spaces}
\item \hyperref[bootstrap-section-phantom]{Bootstrap}
\item \hyperref[spaces-pushouts-section-phantom]{Pushouts of Algebraic Spaces}
\end{enumerate}
Topics in Geometry
\begin{enumerate}
\setcounter{enumi}{73}
\item \hyperref[spaces-chow-section-phantom]{Chow Groups of Spaces}
\item \hyperref[groupoids-quotients-section-phantom]{Quotients of Groupoids}
\item \hyperref[spaces-more-cohomology-section-phantom]{More on Cohomology of Spaces}
\item \hyperref[spaces-simplicial-section-phantom]{Simplicial Spaces}
\item \hyperref[spaces-duality-section-phantom]{Duality for Spaces}
\item \hyperref[formal-spaces-section-phantom]{Formal Algebraic Spaces}
\item \hyperref[restricted-section-phantom]{Restricted Power Series}
\item \hyperref[spaces-resolve-section-phantom]{Resolution of Surfaces Revisited}
\end{enumerate}
Deformation Theory
\begin{enumerate}
\setcounter{enumi}{81}
\item \hyperref[formal-defos-section-phantom]{Formal Deformation Theory}
\item \hyperref[defos-section-phantom]{Deformation Theory}
\item \hyperref[cotangent-section-phantom]{The Cotangent Complex}
\item \hyperref[examples-defos-section-phantom]{Deformation Problems}
\end{enumerate}
Algebraic Stacks
\begin{enumerate}
\setcounter{enumi}{85}
\item \hyperref[algebraic-section-phantom]{Algebraic Stacks}
\item \hyperref[examples-stacks-section-phantom]{Examples of Stacks}
\item \hyperref[stacks-sheaves-section-phantom]{Sheaves on Algebraic Stacks}
\item \hyperref[criteria-section-phantom]{Criteria for Representability}
\item \hyperref[artin-section-phantom]{Artin's Axioms}
\item \hyperref[quot-section-phantom]{Quot and Hilbert Spaces}
\item \hyperref[stacks-properties-section-phantom]{Properties of Algebraic Stacks}
\item \hyperref[stacks-morphisms-section-phantom]{Morphisms of Algebraic Stacks}
\item \hyperref[stacks-limits-section-phantom]{Limits of Algebraic Stacks}
\item \hyperref[stacks-cohomology-section-phantom]{Cohomology of Algebraic Stacks}
\item \hyperref[stacks-perfect-section-phantom]{Derived Categories of Stacks}
\item \hyperref[stacks-introduction-section-phantom]{Introducing Algebraic Stacks}
\item \hyperref[stacks-more-morphisms-section-phantom]{More on Morphisms of Stacks}
\item \hyperref[stacks-geometry-section-phantom]{The Geometry of Stacks}
\end{enumerate}
Topics in Moduli Theory
\begin{enumerate}
\setcounter{enumi}{99}
\item \hyperref[moduli-section-phantom]{Moduli Stacks}
\item \hyperref[moduli-curves-section-phantom]{Moduli of Curves}
\end{enumerate}
Miscellany
\begin{enumerate}
\setcounter{enumi}{101}
\item \hyperref[examples-section-phantom]{Examples}
\item \hyperref[exercises-section-phantom]{Exercises}
\item \hyperref[guide-section-phantom]{Guide to Literature}
\item \hyperref[desirables-section-phantom]{Desirables}
\item \hyperref[coding-section-phantom]{Coding Style}
\item \hyperref[obsolete-section-phantom]{Obsolete}
\item \hyperref[fdl-section-phantom]{GNU Free Documentation License}
\item \hyperref[index-section-phantom]{Auto Generated Index}
\end{enumerate}
\end{multicols}


\bibliography{my}
\bibliographystyle{amsalpha}

\end{document}
