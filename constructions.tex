\IfFileExists{stacks-project.cls}{%
\documentclass{stacks-project}
}{%
\documentclass{amsart}
}

% The following AMS packages are automatically loaded with
% the amsart documentclass:
%\usepackage{amsmath}
%\usepackage{amssymb}
%\usepackage{amsthm}

\usepackage{graphicx}

% For dealing with references we use the comment environment
\usepackage{verbatim}
\newenvironment{reference}{\comment}{\endcomment}
%\newenvironment{reference}{}{}
\newenvironment{slogan}{\comment}{\endcomment}
\newenvironment{history}{\comment}{\endcomment}

% For commutative diagrams you can use
% \usepackage{amscd}
\usepackage[all]{xy}

% We use 2cell for 2-commutative diagrams.
\xyoption{2cell}
\UseAllTwocells

% To put source file link in headers.
% Change "template.tex" to "this_filename.tex"
% \usepackage{fancyhdr}
% \pagestyle{fancy}
% \lhead{}
% \chead{}
% \rhead{Source file: \url{template.tex}}
% \lfoot{}
% \cfoot{\thepage}
% \rfoot{}
% \renewcommand{\headrulewidth}{0pt}
% \renewcommand{\footrulewidth}{0pt}
% \renewcommand{\headheight}{12pt}

\usepackage{multicol}

% For cross-file-references
\usepackage{xr-hyper}

% Package for hypertext links:
\usepackage{hyperref}

% For any local file, say "hello.tex" you want to link to please
% use \externaldocument[hello-]{hello}
\externaldocument[introduction-]{introduction}
\externaldocument[conventions-]{conventions}
\externaldocument[sets-]{sets}
\externaldocument[categories-]{categories}
\externaldocument[topology-]{topology}
\externaldocument[sheaves-]{sheaves}
\externaldocument[sites-]{sites}
\externaldocument[stacks-]{stacks}
\externaldocument[fields-]{fields}
\externaldocument[algebra-]{algebra}
\externaldocument[brauer-]{brauer}
\externaldocument[homology-]{homology}
\externaldocument[derived-]{derived}
\externaldocument[simplicial-]{simplicial}
\externaldocument[more-algebra-]{more-algebra}
\externaldocument[smoothing-]{smoothing}
\externaldocument[modules-]{modules}
\externaldocument[sites-modules-]{sites-modules}
\externaldocument[injectives-]{injectives}
\externaldocument[cohomology-]{cohomology}
\externaldocument[sites-cohomology-]{sites-cohomology}
\externaldocument[dga-]{dga}
\externaldocument[dpa-]{dpa}
\externaldocument[hypercovering-]{hypercovering}
\externaldocument[schemes-]{schemes}
\externaldocument[constructions-]{constructions}
\externaldocument[properties-]{properties}
\externaldocument[morphisms-]{morphisms}
\externaldocument[coherent-]{coherent}
\externaldocument[divisors-]{divisors}
\externaldocument[limits-]{limits}
\externaldocument[varieties-]{varieties}
\externaldocument[topologies-]{topologies}
\externaldocument[descent-]{descent}
\externaldocument[perfect-]{perfect}
\externaldocument[more-morphisms-]{more-morphisms}
\externaldocument[flat-]{flat}
\externaldocument[groupoids-]{groupoids}
\externaldocument[more-groupoids-]{more-groupoids}
\externaldocument[etale-]{etale}
\externaldocument[chow-]{chow}
\externaldocument[intersection-]{intersection}
\externaldocument[pic-]{pic}
\externaldocument[adequate-]{adequate}
\externaldocument[dualizing-]{dualizing}
\externaldocument[duality-]{duality}
\externaldocument[discriminant-]{discriminant}
\externaldocument[local-cohomology-]{local-cohomology}
\externaldocument[curves-]{curves}
\externaldocument[resolve-]{resolve}
\externaldocument[models-]{models}
\externaldocument[pione-]{pione}
\externaldocument[etale-cohomology-]{etale-cohomology}
\externaldocument[ssgroups-]{ssgroups}
\externaldocument[proetale-]{proetale}
\externaldocument[crystalline-]{crystalline}
\externaldocument[spaces-]{spaces}
\externaldocument[spaces-properties-]{spaces-properties}
\externaldocument[spaces-morphisms-]{spaces-morphisms}
\externaldocument[decent-spaces-]{decent-spaces}
\externaldocument[spaces-cohomology-]{spaces-cohomology}
\externaldocument[spaces-limits-]{spaces-limits}
\externaldocument[spaces-divisors-]{spaces-divisors}
\externaldocument[spaces-over-fields-]{spaces-over-fields}
\externaldocument[spaces-topologies-]{spaces-topologies}
\externaldocument[spaces-descent-]{spaces-descent}
\externaldocument[spaces-perfect-]{spaces-perfect}
\externaldocument[spaces-more-morphisms-]{spaces-more-morphisms}
\externaldocument[spaces-flat-]{spaces-flat}
\externaldocument[spaces-groupoids-]{spaces-groupoids}
\externaldocument[spaces-more-groupoids-]{spaces-more-groupoids}
\externaldocument[bootstrap-]{bootstrap}
\externaldocument[spaces-pushouts-]{spaces-pushouts}
\externaldocument[spaces-chow-]{spaces-chow}
\externaldocument[groupoids-quotients-]{groupoids-quotients}
\externaldocument[spaces-more-cohomology-]{spaces-more-cohomology}
\externaldocument[spaces-simplicial-]{spaces-simplicial}
\externaldocument[spaces-duality-]{spaces-duality}
\externaldocument[formal-spaces-]{formal-spaces}
\externaldocument[restricted-]{restricted}
\externaldocument[spaces-resolve-]{spaces-resolve}
\externaldocument[formal-defos-]{formal-defos}
\externaldocument[defos-]{defos}
\externaldocument[cotangent-]{cotangent}
\externaldocument[examples-defos-]{examples-defos}
\externaldocument[algebraic-]{algebraic}
\externaldocument[examples-stacks-]{examples-stacks}
\externaldocument[stacks-sheaves-]{stacks-sheaves}
\externaldocument[criteria-]{criteria}
\externaldocument[artin-]{artin}
\externaldocument[quot-]{quot}
\externaldocument[stacks-properties-]{stacks-properties}
\externaldocument[stacks-morphisms-]{stacks-morphisms}
\externaldocument[stacks-limits-]{stacks-limits}
\externaldocument[stacks-cohomology-]{stacks-cohomology}
\externaldocument[stacks-perfect-]{stacks-perfect}
\externaldocument[stacks-introduction-]{stacks-introduction}
\externaldocument[stacks-more-morphisms-]{stacks-more-morphisms}
\externaldocument[stacks-geometry-]{stacks-geometry}
\externaldocument[moduli-]{moduli}
\externaldocument[moduli-curves-]{moduli-curves}
\externaldocument[examples-]{examples}
\externaldocument[exercises-]{exercises}
\externaldocument[guide-]{guide}
\externaldocument[desirables-]{desirables}
\externaldocument[coding-]{coding}
\externaldocument[obsolete-]{obsolete}
\externaldocument[fdl-]{fdl}
\externaldocument[index-]{index}

% Theorem environments.
%
\theoremstyle{plain}
\newtheorem{theorem}[subsection]{Theorem}
\newtheorem{proposition}[subsection]{Proposition}
\newtheorem{lemma}[subsection]{Lemma}

\theoremstyle{definition}
\newtheorem{definition}[subsection]{Definition}
\newtheorem{example}[subsection]{Example}
\newtheorem{exercise}[subsection]{Exercise}
\newtheorem{situation}[subsection]{Situation}

\theoremstyle{remark}
\newtheorem{remark}[subsection]{Remark}
\newtheorem{remarks}[subsection]{Remarks}

\numberwithin{equation}{subsection}

% Macros
%
\def\lim{\mathop{\mathrm{lim}}\nolimits}
\def\colim{\mathop{\mathrm{colim}}\nolimits}
\def\Spec{\mathop{\mathrm{Spec}}}
\def\Hom{\mathop{\mathrm{Hom}}\nolimits}
\def\Ext{\mathop{\mathrm{Ext}}\nolimits}
\def\SheafHom{\mathop{\mathcal{H}\!\mathit{om}}\nolimits}
\def\SheafExt{\mathop{\mathcal{E}\!\mathit{xt}}\nolimits}
\def\Sch{\mathit{Sch}}
\def\Mor{\mathop{Mor}\nolimits}
\def\Ob{\mathop{\mathrm{Ob}}\nolimits}
\def\Sh{\mathop{\mathit{Sh}}\nolimits}
\def\NL{\mathop{N\!L}\nolimits}
\def\proetale{{pro\text{-}\acute{e}tale}}
\def\etale{{\acute{e}tale}}
\def\QCoh{\mathit{QCoh}}
\def\Ker{\mathop{\mathrm{Ker}}}
\def\Im{\mathop{\mathrm{Im}}}
\def\Coker{\mathop{\mathrm{Coker}}}
\def\Coim{\mathop{\mathrm{Coim}}}
\def\id{\mathop{\mathrm{id}}\nolimits}

%
% Macros for linear algebraic groups
%
\def\SL{\mathop{\mathrm{SL}}\nolimits}
\def\GL{\mathop{\mathrm{GL}}\nolimits}
\def\ltimes{{\mathchar"256E}}
\def\rtimes{{\mathchar"256F}}
\def\Rrightarrow{{\mathchar"3456}}

%
% Macros for moduli stacks/spaces
%
\def\QCohstack{\mathcal{QC}\!\mathit{oh}}
\def\Cohstack{\mathcal{C}\!\mathit{oh}}
\def\Spacesstack{\mathcal{S}\!\mathit{paces}}
\def\Quotfunctor{\mathrm{Quot}}
\def\Hilbfunctor{\mathrm{Hilb}}
\def\Curvesstack{\mathcal{C}\!\mathit{urves}}
\def\Polarizedstack{\mathcal{P}\!\mathit{olarized}}
\def\Complexesstack{\mathcal{C}\!\mathit{omplexes}}
% \Pic is the operator that assigns to X its picard group, usage \Pic(X)
% \Picardstack_{X/B} denotes the Picard stack of X over B
% \Picardfunctor_{X/B} denotes the Picard functor of X over B
\def\Pic{\mathop{\mathrm{Pic}}\nolimits}
\def\Picardstack{\mathcal{P}\!\mathit{ic}}
\def\Picardfunctor{\mathrm{Pic}}
\def\Deformationcategory{\mathcal{D}\!\mathit{ef}}


% OK, start here.
%
\begin{document}

\title{Constructions of Schemes}


\maketitle

\phantomsection
\label{section-phantom}

\tableofcontents

\section{Introduction}
\label{section-introduction}

\noindent
In this chapter we introduce ways of constructing schemes out of others.
A basic reference is \cite{EGA}.





\section{Relative glueing}
\label{section-relative-glueing}

\noindent
The following lemma is relevant in case we are trying to construct a
scheme $X$ over $S$, and we already know how to construct the restriction
of $X$ to the affine opens of $S$. The actual result is completely general
and works in the setting of (locally) ringed spaces, although our proof
is written in the language of schemes.

\begin{lemma}
\label{lemma-relative-glueing}
Let $S$ be a scheme.
Let $\mathcal{B}$ be a basis for the topology of $S$.
Suppose given the following data:
\begin{enumerate}
\item For every $U \in \mathcal{B}$ a scheme $f_U : X_U \to U$ over $U$.
\item For $U, V \in \mathcal{B}$ with $V \subset U$ a morphism
$\rho^U_V : X_V \to X_U$ over $U$.
\end{enumerate}
Assume that
\begin{enumerate}
\item[(a)] each $\rho^U_V$ induces an isomorphism
$X_V \to f_U^{-1}(V)$ of schemes over $V$,
\item[(b)] whenever $W, V, U \in \mathcal{B}$, with
$W \subset V \subset U$ we have $\rho^U_W = \rho^U_V \circ \rho ^V_W$.
\end{enumerate}
Then there exists a morphism $f : X \to S$ of schemes
and isomorphisms $i_U : f^{-1}(U) \to X_U$ over $U \in \mathcal{B}$
such that for $V, U \in \mathcal{B}$ with $V \subset U$ the composition
$$
\xymatrix{
X_V \ar[r]^{i_V^{-1}} &
f^{-1}(V) \ar[rr]^{inclusion} & &
f^{-1}(U) \ar[r]^{i_U} &
X_U
}
$$
is the morphism $\rho^U_V$. Moreover $X$ is unique up to
unique isomorphism over $S$.
\end{lemma}

\begin{proof}
To prove this we will use Schemes, Lemma \ref{schemes-lemma-glue-functors}.
First we define a contravariant functor $F$ from the category of schemes
to the category of sets. Namely, for a scheme $T$ we set
$$
F(T) =
\left\{
\begin{matrix}
(g, \{h_U\}_{U \in \mathcal{B}}),
\ g : T \to S, \ h_U : g^{-1}(U) \to X_U, \\
f_U \circ h_U = g|_{g^{-1}(U)},
\ h_U|_{g^{-1}(V)} = \rho^U_V \circ h_V
\ \forall\ V, U \in \mathcal{B}, V \subset U
\end{matrix}
\right\}.
$$
The restriction mapping $F(T) \to F(T')$ given a morphism
$T' \to T$ is just gotten by composition.
For any $W \in \mathcal{B}$ we consider the subfunctor
$F_W \subset F$ consisting of those systems $(g, \{h_U\})$
such that $g(T) \subset W$.

\medskip\noindent
First we show $F$ satisfies the sheaf property for the Zariski topology.
Suppose that $T$ is a scheme, $T = \bigcup V_i$ is an open covering,
and $\xi_i \in F(V_i)$ is an element such that
$\xi_i|_{V_i \cap V_j} = \xi_j|_{V_i \cap V_j}$.
Say $\xi_i = (g_i, \{h_{i, U}\})$. Then we immediately see that
the morphisms $g_i$ glue to a unique global morphism
$g : T \to S$. Moreover, it is clear that
$g^{-1}(U) = \bigcup g_i^{-1}(U)$. Hence the morphisms
$h_{i, U} : g_i^{-1}(U) \to X_U$ glue to a unique morphism
$h_U : g^{-1}(U) \to X_U$. It is easy to verify that the system
$(g, \{h_U\})$ is an element of $F(T)$. Hence $F$ satisfies the
sheaf property for the Zariski topology.

\medskip\noindent
Next we verify that each $F_W$, $W \in \mathcal{B}$ is representable.
Namely, we claim that the transformation of functors
$$
F_W \longrightarrow \Mor(-, X_W), \ (g, \{h_U\}) \longmapsto h_W
$$
is an isomorphism. To see this suppose that $T$ is a scheme and
$\alpha : T \to X_W$ is a morphism. Set $g = f_W \circ \alpha$.
For any $U \in \mathcal{B}$ such that $U \subset W$ we can
define $h_U : g^{-1}(U) \to X_U$ be the composition
$(\rho^W_U)^{-1} \circ \alpha|_{g^{-1}(U)}$. This works because
the image $\alpha(g^{-1}(U))$ is contained in $f_W^{-1}(U)$ and
condition (a) of the lemma. It is clear that
$f_U \circ h_U = g|_{g^{-1}(U)}$ for such a $U$.
Moreover, if also $V \in \mathcal{B}$ and $V \subset U \subset W$,
then $\rho^U_V \circ h_V = h_U|_{g^{-1}(V)}$ by property (b)
of the lemma. We still have to define $h_U$ for an arbitrary
element $U \in \mathcal{B}$. Since $\mathcal{B}$ is a basis for
the topology on $S$ we can find an open covering
$U \cap W = \bigcup U_i$ with $U_i \in \mathcal{B}$. Since $g$ maps into $W$
we have
$g^{-1}(U) = g^{-1}(U \cap W) = \bigcup g^{-1}(U_i)$.
Consider the morphisms
$h_i = \rho^U_{U_i} \circ h_{U_i} : g^{-1}(U_i) \to X_U$.
It is a simple matter to use condition (b) of the lemma
to prove that
$h_i|_{g^{-1}(U_i) \cap g^{-1}(U_j)} = h_j|_{g^{-1}(U_i) \cap g^{-1}(U_j)}$.
Hence these morphisms glue to give the desired morphism
$h_U : g^{-1}(U) \to X_U$. We omit the (easy) verification that
the system $(g, \{h_U\})$ is an element of $F_W(T)$ which
maps to $\alpha$ under the displayed arrow above.

\medskip\noindent
Next, we verify each $F_W \subset F$ is representable by open immersions.
This is clear from the definitions.

\medskip\noindent
Finally we have to verify
the collection $(F_W)_{W \in \mathcal{B}}$ covers $F$.
This is clear by construction and the fact that $\mathcal{B}$ is
a basis for the topology of $S$.

\medskip\noindent
Let $X$ be a scheme representing the functor $F$.
Let $(f, \{i_U\}) \in F(X)$ be a ``universal family''.
Since each $F_W$ is representable by $X_W$ (via the morphism of functors
displayed above) we see that $i_W : f^{-1}(W) \to X_W$
is an isomorphism as desired. The lemma is proved.
\end{proof}

\begin{lemma}
\label{lemma-relative-glueing-sheaves}
Let $S$ be a scheme.
Let $\mathcal{B}$ be a basis for the topology of $S$.
Suppose given the following data:
\begin{enumerate}
\item For every $U \in \mathcal{B}$ a scheme $f_U : X_U \to U$ over $U$.
\item For every $U \in \mathcal{B}$ a quasi-coherent sheaf $\mathcal{F}_U$
over $X_U$.
\item For every pair $U, V \in \mathcal{B}$ such that
$V \subset U$ a morphism $\rho^U_V : X_V \to X_U$.
\item  For every pair $U, V \in \mathcal{B}$ such that
$V \subset U$ a morphism
$\theta^U_V : (\rho^U_V)^*\mathcal{F}_U \to \mathcal{F}_V$.
\end{enumerate}
Assume that
\begin{enumerate}
\item[(a)] each $\rho^U_V$ induces an isomorphism
$X_V \to f_U^{-1}(V)$ of schemes over $V$,
\item[(b)] each $\theta^U_V$ is an isomorphism,
\item[(c)] whenever $W, V, U \in \mathcal{B}$, with
$W \subset V \subset U$ we have $\rho^U_W = \rho^U_V \circ \rho ^V_W$,
\item[(d)] whenever $W, V, U \in \mathcal{B}$, with
$W \subset V \subset U$ we have
$\theta^U_W = \theta^V_W \circ (\rho^V_W)^*\theta^U_V$.
\end{enumerate}
Then there exists a morphism of schemes $f : X \to S$
together with a quasi-coherent sheaf $\mathcal{F}$ on $X$
and isomorphisms $i_U : f^{-1}(U) \to X_U$ and
$\theta_U : i_U^*\mathcal{F}_U \to \mathcal{F}|_{f^{-1}(U)}$
over $U \in \mathcal{B}$ such that
for $V, U \in \mathcal{B}$ with $V \subset U$ the composition
$$
\xymatrix{
X_V \ar[r]^{i_V^{-1}} &
f^{-1}(V) \ar[rr]^{inclusion} & &
f^{-1}(U) \ar[r]^{i_U} &
X_U
}
$$
is the morphism $\rho^U_V$, and the composition
\begin{equation}
\label{equation-glue}
(\rho^U_V)^*\mathcal{F}_U
=
(i_V^{-1})^*((i_U^*\mathcal{F}_U)|_{f^{-1}(V)})
\xrightarrow{\theta_U|_{f^{-1}(V)}}
(i_V^{-1})^*(\mathcal{F}|_{f^{-1}(V)})
\xrightarrow{\theta_V^{-1}}
\mathcal{F}_V
\end{equation}
is equal to $\theta^U_V$. Moreover $(X, \mathcal{F})$ is unique
up to unique isomorphism over $S$.
\end{lemma}

\begin{proof}
By Lemma \ref{lemma-relative-glueing} we get the scheme $X$ over $S$
and the isomorphisms $i_U$.
Set $\mathcal{F}'_U = i_U^*\mathcal{F}_U$ for $U \in \mathcal{B}$.
This is a quasi-coherent $\mathcal{O}_{f^{-1}(U)}$-module.
The maps
$$
\mathcal{F}'_U|_{f^{-1}(V)} =
i_U^*\mathcal{F}_U|_{f^{-1}(V)} =
i_V^*(\rho^U_V)^*\mathcal{F}_U \xrightarrow{i_V^*\theta^U_V}
i_V^*\mathcal{F}_V = \mathcal{F}'_V
$$
define isomorphisms
$(\theta')^U_V : \mathcal{F}'_U|_{f^{-1}(V)} \to \mathcal{F}'_V$
whenever $V \subset U$ are elements of $\mathcal{B}$.
Condition (d) says exactly that this is compatible in case
we have a triple of elements $W \subset V \subset U$ of $\mathcal{B}$.
This allows us to get well defined isomorphisms
$$
\varphi_{12} :
\mathcal{F}'_{U_1}|_{f^{-1}(U_1 \cap U_2)}
\longrightarrow
\mathcal{F}'_{U_2}|_{f^{-1}(U_1 \cap U_2)}
$$
whenever $U_1, U_2 \in \mathcal{B}$ by covering the intersection
$U_1 \cap U_2 = \bigcup V_j$ by elements $V_j$ of $\mathcal{B}$
and taking
$$
\varphi_{12}|_{V_j} =
\left((\theta')^{U_2}_{V_j}\right)^{-1}
\circ
(\theta')^{U_1}_{V_j}.
$$
We omit the verification that these maps do indeed glue to a
$\varphi_{12}$ and we omit the verification of the
cocycle condition of a glueing datum for sheaves
(as in Sheaves, Section \ref{sheaves-section-glueing-sheaves}).
By Sheaves, Lemma \ref{sheaves-lemma-glue-sheaves}
we get our $\mathcal{F}$ on $X$. We omit the verification
of (\ref{equation-glue}).
\end{proof}

\begin{remark}
\label{remark-relative-glueing-functorial}
There is a functoriality property for the constructions explained
in Lemmas \ref{lemma-relative-glueing} and
\ref{lemma-relative-glueing-sheaves}. Namely, suppose given
two collections of data $(f_U : X_U \to U, \rho^U_V)$ and
$(g_U : Y_U \to U, \sigma^U_V)$ as in Lemma \ref{lemma-relative-glueing}.
Suppose for every $U \in \mathcal{B}$ given
a morphism $h_U : X_U \to Y_U$ over $U$ compatible with
the restrictions $\rho^U_V$ and $\sigma^U_V$. Functoriality
means that this gives rise to a morphism of schemes
$h : X \to Y$ over $S$ restricting back to the morphisms $h_U$,
where $f : X \to S$ is obtained from
the datum $(f_U : X_U \to U, \rho^U_V)$ and $g : Y \to S$
is obtained from the datum $(g_U : Y_U \to U, \sigma^U_V)$.

\medskip\noindent
Similarly, suppose given
two collections of data
$(f_U : X_U \to U, \mathcal{F}_U, \rho^U_V, \theta^U_V)$ and
$(g_U : Y_U \to U, \mathcal{G}_U, \sigma^U_V, \eta^U_V)$
as in Lemma \ref{lemma-relative-glueing-sheaves}.
Suppose for every $U \in \mathcal{B}$ given
a morphism $h_U : X_U \to Y_U$ over $U$ compatible with
the restrictions $\rho^U_V$ and $\sigma^U_V$, and a morphism
$\tau_U : h_U^*\mathcal{G}_U \to \mathcal{F}_U$ compatible with
the maps $\theta^U_V$ and $\eta^U_V$. Functoriality
means that these give rise to a morphism of schemes
$h : X \to Y$ over $S$ restricting back to the morphisms $h_U$,
and a morphism $h^*\mathcal{G} \to \mathcal{F}$ restricting back
to the maps $h_U$
where $(f : X \to S, \mathcal{F})$ is obtained from the datum
$(f_U : X_U \to U, \mathcal{F}_U, \rho^U_V, \theta^U_V)$ and
where $(g : Y \to S, \mathcal{G})$ is obtained from the datum
$(g_U : Y_U \to U, \mathcal{G}_U, \sigma^U_V, \eta^U_V)$.

\medskip\noindent
We omit the verifications and we omit a suitable formulation of
``equivalence of categories'' between relative glueing data
and relative objects.
\end{remark}















\section{Relative spectrum via glueing}
\label{section-spec-via-glueing}

\begin{situation}
\label{situation-relative-spec}
Here $S$ is a scheme, and $\mathcal{A}$ is a quasi-coherent
$\mathcal{O}_S$-algebra. This means that $\mathcal{A}$ is a
sheaf of $\mathcal{O}_S$-algebras which is quasi-coherent as an
$\mathcal{O}_S$-module.
\end{situation}

\noindent
In this section we outline how to construct a morphism
of schemes
$$
\underline{\Spec}_S(\mathcal{A}) \longrightarrow S
$$
by glueing the spectra $\Spec(\Gamma(U, \mathcal{A}))$
where $U$ ranges over the affine opens of $S$. We first show that the
spectra of the values of $\mathcal{A}$ over affines form a
suitable collection of schemes, as in Lemma \ref{lemma-relative-glueing}.

\begin{lemma}
\label{lemma-spec-inclusion}
In Situation \ref{situation-relative-spec}.
Suppose $U \subset U' \subset S$ are affine opens.
Let $A = \mathcal{A}(U)$ and $A' = \mathcal{A}(U')$.
The map of rings $A' \to A$ induces a morphism
$\Spec(A) \to \Spec(A')$, and the diagram
$$
\xymatrix{
\Spec(A) \ar[r] \ar[d] &
\Spec(A') \ar[d] \\
U \ar[r] &
U'
}
$$
is cartesian.
\end{lemma}

\begin{proof}
Let $R = \mathcal{O}_S(U)$ and $R' = \mathcal{O}_S(U')$.
Note that the map $R \otimes_{R'} A' \to A$ is an isomorphism as
$\mathcal{A}$ is quasi-coherent
(see Schemes, Lemma \ref{schemes-lemma-widetilde-pullback} for example).
The result follows from the description of the fibre product of
affine schemes in
Schemes, Lemma \ref{schemes-lemma-fibre-product-affine-schemes}.
\end{proof}

\noindent
In particular the morphism $\Spec(A) \to \Spec(A')$
of the lemma is an open immersion.

\begin{lemma}
\label{lemma-transitive-spec}
In Situation \ref{situation-relative-spec}.
Suppose $U \subset U' \subset U'' \subset S$ are affine opens.
Let $A = \mathcal{A}(U)$, $A' = \mathcal{A}(U')$ and $A'' = \mathcal{A}(U'')$.
The composition of the morphisms
$\Spec(A) \to \Spec(A')$, and
$\Spec(A') \to \Spec(A'')$ of
Lemma \ref{lemma-spec-inclusion} gives the
morphism $\Spec(A) \to \Spec(A'')$
of Lemma \ref{lemma-spec-inclusion}.
\end{lemma}

\begin{proof}
This follows as the map $A'' \to A$ is the composition of $A'' \to A'$ and
$A' \to A$ (because $\mathcal{A}$ is a sheaf).
\end{proof}

\begin{lemma}
\label{lemma-glue-relative-spec}
In Situation \ref{situation-relative-spec}.
There exists a morphism of schemes
$$
\pi : \underline{\Spec}_S(\mathcal{A}) \longrightarrow S
$$
with the following properties:
\begin{enumerate}
\item for every affine open $U \subset S$ there exists an isomorphism
$i_U : \pi^{-1}(U) \to \Spec(\mathcal{A}(U))$ over $U$, and
\item for $U \subset U' \subset S$ affine open the composition
$$
\xymatrix{
\Spec(\mathcal{A}(U)) \ar[r]^{i_U^{-1}} &
\pi^{-1}(U) \ar[rr]^{inclusion} & &
\pi^{-1}(U') \ar[r]^{i_{U'}} &
\Spec(\mathcal{A}(U'))
}
$$
is the open immersion of Lemma \ref{lemma-spec-inclusion} above.
\end{enumerate}
Moreover, $\underline{\Spec}_S(\mathcal{A})$
is unique up to unique isomorphism over $S$.
\end{lemma}

\begin{proof}
Follows immediately from
Lemmas \ref{lemma-relative-glueing},
\ref{lemma-spec-inclusion}, and
\ref{lemma-transitive-spec}.
Uniqueness is stated in the last sentence of
Lemma \ref{lemma-relative-glueing}.
\end{proof}












\section{Relative spectrum as a functor}
\label{section-spec}

\noindent
We place ourselves in Situation \ref{situation-relative-spec}, i.e.,
$S$ is a scheme and $\mathcal{A}$ is a quasi-coherent sheaf of
$\mathcal{O}_S$-algebras.

\medskip\noindent
For any $f : T \to S$ the pullback
$f^*\mathcal{A}$ is a quasi-coherent sheaf of $\mathcal{O}_T$-algebras.
We are going to consider pairs $(f : T \to S, \varphi)$ where
$f$ is a morphism of schemes and $\varphi : f^*\mathcal{A} \to \mathcal{O}_T$
is a morphism of $\mathcal{O}_T$-algebras. Note that this is the
same as giving a $f^{-1}\mathcal{O}_S$-algebra homomorphism
$\varphi : f^{-1}\mathcal{A} \to \mathcal{O}_T$, see
Sheaves, Lemma \ref{sheaves-lemma-adjointness-tensor-restrict}.
This is also the same as giving an $\mathcal{O}_S$-algebra map
$\varphi : \mathcal{A} \to f_*\mathcal{O}_T$, see
Sheaves, Lemma \ref{sheaves-lemma-adjoint-push-pull-modules}.
We will use all three ways of thinking about $\varphi$,
without further mention.

\medskip\noindent
Given such a
pair $(f : T \to S, \varphi)$ and a morphism $a : T' \to T$ we get
a second pair $(f' = f \circ a, \varphi' = a^*\varphi)$ which we
call the pullback of $(f, \varphi)$. One way to describe
$\varphi' = a^*\varphi$ is as the composition
$\mathcal{A} \to f_*\mathcal{O}_T \to f'_*\mathcal{O}_{T'}$
where the second map is $f_*a^\sharp$ with
$a^\sharp : \mathcal{O}_T \to a_*\mathcal{O}_{T'}$.
In this way we have defined a functor
\begin{eqnarray}
\label{equation-spec}
F : \Sch^{opp} & \longrightarrow & \textit{Sets} \\
T & \longmapsto & F(T) = \{\text{pairs }(f, \varphi) \text{ as above}\}
\nonumber
\end{eqnarray}

\begin{lemma}
\label{lemma-spec-base-change}
In Situation \ref{situation-relative-spec}.
Let $F$ be the functor
associated to $(S, \mathcal{A})$ above.
Let $g : S' \to S$ be a morphism of schemes.
Set $\mathcal{A}' = g^*\mathcal{A}$. Let $F'$ be the
functor associated to $(S', \mathcal{A}')$ above.
Then there is a canonical isomorphism
$$
F' \cong h_{S'} \times_{h_S} F
$$
of functors.
\end{lemma}

\begin{proof}
A pair $(f' : T \to S', \varphi' : (f')^*\mathcal{A}' \to \mathcal{O}_T)$
is the same as a pair $(f, \varphi : f^*\mathcal{A} \to \mathcal{O}_T)$
together with a factorization of $f$ as $f = g \circ f'$. Namely with
this notation we have
$(f')^* \mathcal{A}' = (f')^*g^*\mathcal{A} = f^*\mathcal{A}$.
Hence the lemma.
\end{proof}

\begin{lemma}
\label{lemma-spec-affine}
In Situation \ref{situation-relative-spec}.
Let $F$ be the functor associated to $(S, \mathcal{A})$ above.
If $S$ is affine, then $F$ is representable by the
affine scheme $\Spec(\Gamma(S, \mathcal{A}))$.
\end{lemma}

\begin{proof}
Write $S = \Spec(R)$ and $A = \Gamma(S, \mathcal{A})$.
Then $A$ is an $R$-algebra and $\mathcal{A} = \widetilde A$.
The ring map $R \to A$ gives rise to a canonical map
$$
f_{univ} : \Spec(A)
\longrightarrow
S = \Spec(R).
$$
We have
$f_{univ}^*\mathcal{A} =  \widetilde{A \otimes_R A}$
by Schemes, Lemma \ref{schemes-lemma-widetilde-pullback}.
Hence there is a canonical map
$$
\varphi_{univ} :
f_{univ}^*\mathcal{A} = \widetilde{A \otimes_R A}
\longrightarrow
\widetilde A = \mathcal{O}_{\Spec(A)}
$$
coming from the $A$-module map $A \otimes_R A \to A$,
$a \otimes a' \mapsto aa'$. We claim that the pair
$(f_{univ}, \varphi_{univ})$ represents $F$ in this case.
In other words we claim that for any scheme $T$ the map
$$
\Mor(T, \Spec(A)) \longrightarrow \{\text{pairs } (f, \varphi)\},\quad
a \longmapsto (f_{univ} \circ a, a^*\varphi_{univ})
$$
is bijective.

\medskip\noindent
Let us construct the inverse map.
For any pair $(f : T \to S, \varphi)$ we get the induced
ring map
$$
\xymatrix{
A = \Gamma(S, \mathcal{A}) \ar[r]^{f^*} &
\Gamma(T, f^*\mathcal{A}) \ar[r]^{\varphi} &
\Gamma(T, \mathcal{O}_T)
}
$$
This induces a morphism of schemes $T \to \Spec(A)$
by Schemes, Lemma \ref{schemes-lemma-morphism-into-affine}.

\medskip\noindent
The verification that this map is inverse to the map
displayed above is omitted.
\end{proof}

\begin{lemma}
\label{lemma-spec}
In Situation \ref{situation-relative-spec}.
The functor $F$ is representable by a scheme.
\end{lemma}

\begin{proof}
We are going to use Schemes, Lemma \ref{schemes-lemma-glue-functors}.

\medskip\noindent
First we check that $F$ satisfies the sheaf property for the
Zariski topology. Namely, suppose that $T$ is a scheme,
that $T = \bigcup_{i \in I} U_i$ is an open covering,
and that $(f_i, \varphi_i) \in F(U_i)$ such that
$(f_i, \varphi_i)|_{U_i \cap U_j} = (f_j, \varphi_j)|_{U_i \cap U_j}$.
This implies that the morphisms $f_i : U_i \to S$
glue to a morphism of schemes $f : T \to S$ such that
$f|_{U_i} = f_i$, see Schemes, Section \ref{schemes-section-glueing-schemes}.
Thus $f_i^*\mathcal{A} = f^*\mathcal{A}|_{U_i}$ and by assumption the
morphisms $\varphi_i$ agree on $U_i \cap U_j$. Hence by Sheaves,
Section \ref{sheaves-section-glueing-sheaves} these glue to a
morphism of $\mathcal{O}_T$-algebras $f^*\mathcal{A} \to \mathcal{O}_T$.
This proves that $F$ satisfies the sheaf condition with respect to
the Zariski topology.

\medskip\noindent
Let $S = \bigcup_{i \in I} U_i$ be an affine open covering.
Let $F_i \subset F$ be the subfunctor consisting of
those pairs $(f : T \to S, \varphi)$ such that
$f(T) \subset U_i$.

\medskip\noindent
We have to show each $F_i$ is representable.
This is the case because $F_i$ is identified with
the functor associated to $U_i$ equipped with
the quasi-coherent $\mathcal{O}_{U_i}$-algebra $\mathcal{A}|_{U_i}$,
by Lemma \ref{lemma-spec-base-change}.
Thus the result follows from Lemma \ref{lemma-spec-affine}.

\medskip\noindent
Next we show that $F_i \subset F$ is representable by open immersions.
Let $(f : T \to S, \varphi) \in F(T)$. Consider $V_i = f^{-1}(U_i)$.
It follows from the definition of $F_i$ that given $a : T' \to T$
we gave $a^*(f, \varphi) \in F_i(T')$ if and only if $a(T') \subset V_i$.
This is what we were required to show.

\medskip\noindent
Finally, we have to show that the collection $(F_i)_{i \in I}$
covers $F$. Let $(f : T \to S, \varphi) \in F(T)$.
Consider $V_i = f^{-1}(U_i)$. Since $S = \bigcup_{i \in I} U_i$
is an open covering of $S$ we see that $T = \bigcup_{i \in I} V_i$
is an open covering of $T$. Moreover $(f, \varphi)|_{V_i} \in F_i(V_i)$.
This finishes the proof of the lemma.
\end{proof}

\begin{lemma}
\label{lemma-glueing-gives-functor-spec}
In Situation \ref{situation-relative-spec}.
The scheme $\pi : \underline{\Spec}_S(\mathcal{A}) \to S$
constructed in Lemma \ref{lemma-glue-relative-spec}
and the scheme representing the functor $F$ are
canonically isomorphic as schemes over $S$.
\end{lemma}

\begin{proof}
Let $X \to S$ be the scheme representing the functor $F$.
Consider the sheaf of $\mathcal{O}_S$-algebras
$\mathcal{R} = \pi_*\mathcal{O}_{\underline{\Spec}_S(\mathcal{A})}$.
By construction of $\underline{\Spec}_S(\mathcal{A})$
we have isomorphisms $\mathcal{A}(U) \to \mathcal{R}(U)$
for every affine open $U \subset S$; this follows from
Lemma \ref{lemma-glue-relative-spec} part (1).
For $U \subset U' \subset S$ open these isomorphisms are
compatible with the restriction mappings; this follows from
Lemma \ref{lemma-glue-relative-spec} part (2).
Hence by Sheaves, Lemma \ref{sheaves-lemma-restrict-basis-equivalence-modules}
these isomorphisms result from an isomorphism of $\mathcal{O}_S$-algebras
$\varphi : \mathcal{A} \to \mathcal{R}$. Hence this gives an element
$(\pi, \varphi) \in F(\underline{\Spec}_S(\mathcal{A}))$.
Since $X$ represents the functor $F$ we get a corresponding
morphism of schemes $can : \underline{\Spec}_S(\mathcal{A}) \to X$
over $S$.

\medskip\noindent
Let $U \subset S$ be any affine open. Let $F_U \subset F$ be
the subfunctor of $F$ corresponding to pairs $(f, \varphi)$ over
schemes $T$ with $f(T) \subset U$. Clearly the base change
$X_U$ represents $F_U$. Moreover, $F_U$ is represented by
$\Spec(\mathcal{A}(U)) = \pi^{-1}(U)$ according to
Lemma \ref{lemma-spec-affine}. In other words $X_U \cong \pi^{-1}(U)$.
We omit the verification that this identification is brought about
by the base change of the morphism $can$ to $U$.
\end{proof}

\begin{definition}
\label{definition-relative-spec}
Let $S$ be a scheme. Let $\mathcal{A}$ be a quasi-coherent sheaf of
$\mathcal{O}_S$-algebras. The {\it relative spectrum of $\mathcal{A}$ over
$S$}, or simply the {\it spectrum of $\mathcal{A}$ over $S$} is the scheme
constructed in Lemma \ref{lemma-glue-relative-spec} which represents the
functor $F$ (\ref{equation-spec}), see
Lemma \ref{lemma-glueing-gives-functor-spec}.
We denote it $\pi : \underline{\Spec}_S(\mathcal{A}) \to S$.
The ``universal family'' is a morphism of $\mathcal{O}_S$-algebras
$$
\mathcal{A}
\longrightarrow
\pi_*\mathcal{O}_{\underline{\Spec}_S(\mathcal{A})}
$$
\end{definition}

\noindent
The following lemma says among other things that forming the
relative spectrum commutes with base change.

\begin{lemma}
\label{lemma-spec-properties}
Let $S$ be a scheme. Let $\mathcal{A}$ be a quasi-coherent
sheaf of $\mathcal{O}_S$-algebras. Let
$\pi : \underline{\Spec}_S(\mathcal{A}) \to S$
be the relative spectrum of $\mathcal{A}$ over $S$.
\begin{enumerate}
\item For every affine open $U \subset S$ the inverse image
$\pi^{-1}(U)$ is affine.
\item For every morphism $g : S' \to S$ we have
$S' \times_S \underline{\Spec}_S(\mathcal{A}) =
\underline{\Spec}_{S'}(g^*\mathcal{A})$.
\item
The universal map
$$
\mathcal{A}
\longrightarrow
\pi_*\mathcal{O}_{\underline{\Spec}_S(\mathcal{A})}
$$
is an isomorphism of $\mathcal{O}_S$-algebras.
\end{enumerate}
\end{lemma}

\begin{proof}
Part (1) comes from the description of the relative spectrum
by glueing, see Lemma \ref{lemma-glue-relative-spec}.
Part (2) follows immediately from Lemma \ref{lemma-spec-base-change}.
Part (3) follows because it is local on $S$ and it is clear in case $S$
is affine by Lemma \ref{lemma-spec-affine} for example.
\end{proof}

\begin{lemma}
\label{lemma-canonical-morphism}
Let $f : X \to S$ be a quasi-compact and quasi-separated morphism
of schemes. By Schemes, Lemma \ref{schemes-lemma-push-forward-quasi-coherent}
the sheaf $f_*\mathcal{O}_X$ is a quasi-coherent sheaf of
$\mathcal{O}_S$-algebras. There is a canonical morphism
$$
can : X \longrightarrow \underline{\Spec}_S(f_*\mathcal{O}_X)
$$
of schemes over $S$.
For any affine open $U \subset S$ the restriction $can|_{f^{-1}(U)}$
is identified with the canonical morphism
$$
f^{-1}(U) \longrightarrow \Spec(\Gamma(f^{-1}(U), \mathcal{O}_X))
$$
coming from Schemes, Lemma \ref{schemes-lemma-morphism-into-affine}.
\end{lemma}

\begin{proof}
The morphism comes, via the definition of $\underline{\Spec}$
as the scheme representing the functor $F$, from the canonical map
$\varphi : f^*f_*\mathcal{O}_X \to \mathcal{O}_X$ (which by adjointness of
push and pull corresponds to
$\text{id} : f_*\mathcal{O}_X \to f_*\mathcal{O}_X$).
The statement on the restriction to $f^{-1}(U)$
follows from the description of the relative spectrum over
affines, see Lemma \ref{lemma-spec-affine}.
\end{proof}











\section{Affine n-space}
\label{section-affine-n-space}

\noindent
As an application of the relative spectrum
we define affine $n$-space over a base scheme
$S$ as follows. For any integer $n \geq 0$ we can consider the
quasi-coherent sheaf of $\mathcal{O}_S$-algebras
$\mathcal{O}_S[T_1, \ldots, T_n]$. It is quasi-coherent because
as a sheaf of $\mathcal{O}_S$-modules it is just the direct sum
of copies of $\mathcal{O}_S$ indexed by multi-indices.

\begin{definition}
\label{definition-affine-n-space}
Let $S$ be a scheme and $n \geq 0$.
The scheme
$$
\mathbf{A}^n_S =
\underline{\Spec}_S(\mathcal{O}_S[T_1, \ldots, T_n])
$$
over $S$ is called {\it affine $n$-space over $S$}.
If $S = \Spec(R)$ is affine then we also call this
{\it affine $n$-space over $R$} and we denote it $\mathbf{A}^n_R$.
\end{definition}

\noindent
Note that $\mathbf{A}^n_R = \Spec(R[T_1, \ldots, T_n])$.
For any morphism $g : S' \to S$ of schemes we have
$g^*\mathcal{O}_S[T_1, \ldots, T_n] = \mathcal{O}_{S'}[T_1, \ldots, T_n]$
and hence $\mathbf{A}^n_{S'} = S' \times_S \mathbf{A}^n_S$ is the base
change. Therefore an alternative definition of affine $n$-space
is the formula
$$
\mathbf{A}^n_S = S \times_{\Spec(\mathbf{Z})} \mathbf{A}^n_{\mathbf{Z}}.
$$
Also, a morphism from an $S$-scheme $f : X \to S$
to $\mathbf{A}^n_S$ is given by a homomorphism of
$\mathcal{O}_S$-algebras
$\mathcal{O}_S[T_1, \ldots, T_n] \to f_*\mathcal{O}_X$.
This is clearly the same thing as giving the images of the $T_i$.
In other words, a morphism from $X$ to $\mathbf{A}^n_S$ over $S$
is the same as giving $n$ elements
$h_1, \ldots, h_n \in \Gamma(X, \mathcal{O}_X)$.

















\section{Vector bundles}
\label{section-vector-bundle}

\noindent
Let $S$ be a scheme.
Let $\mathcal{E}$ be a quasi-coherent sheaf of $\mathcal{O}_S$-modules.
By Modules, Lemma \ref{modules-lemma-whole-tensor-algebra-permanence}
the symmetric algebra $\text{Sym}(\mathcal{E})$ of
$\mathcal{E}$ over $\mathcal{O}_S$
is a quasi-coherent sheaf of $\mathcal{O}_S$-algebras.
Hence it makes sense to apply the relative spectrum.

\begin{definition}
\label{definition-vector-bundle}
Let $S$ be a scheme. Let $\mathcal{E}$ be a quasi-coherent
$\mathcal{O}_S$-module\footnote{The reader may expect here
the condition that $\mathcal{E}$ is finite locally free. We do not
do so in order to be consistent with \cite[II, Definition 1.7.8]{EGA}.}.
The {\it vector bundle associated to $\mathcal{E}$} is
$$
\mathbf{V}(\mathcal{E}) = \underline{\Spec}_S(\text{Sym}(\mathcal{E})).
$$
\end{definition}

\noindent
The vector bundle associated to $\mathcal{E}$ comes with a bit
of extra structure. Namely, we have a grading
$$
\pi_*\mathcal{O}_{\mathbf{V}(\mathcal{E})} =
\bigoplus\nolimits_{n \geq 0} \text{Sym}^n(\mathcal{E}).
$$
which turns $\pi_*\mathcal{O}_{\mathbf{V}(\mathcal{E})}$
into a graded $\mathcal{O}_S$-algebra. Conversely, we can recover
$\mathcal{E}$ from the degree $1$ part of this.
Thus we define an abstract vector bundle as follows.

\begin{definition}
\label{definition-abstract-vector-bundle}
Let $S$ be a scheme. A {\it vector bundle $\pi : V \to S$ over $S$} is an
affine morphism of schemes such that $\pi_*\mathcal{O}_V$ is endowed with
the structure of a graded $\mathcal{O}_S$-algebra
$\pi_*\mathcal{O}_V = \bigoplus\nolimits_{n \geq 0} \mathcal{E}_n$
such that $\mathcal{E}_0 = \mathcal{O}_S$ and such that the maps
$$
\text{Sym}^n(\mathcal{E}_1) \longrightarrow \mathcal{E}_n
$$
are isomorphisms for all $n \geq 0$. A {\it morphism of vector bundles
over $S$} is a morphism $f : V \to V'$ such that the induced map
$$
f^* : \pi'_*\mathcal{O}_{V'} \longrightarrow \pi_*\mathcal{O}_V
$$
is compatible with the given gradings.
\end{definition}

\noindent
An example of a vector bundle over $S$ is affine $n$-space
$\mathbf{A}^n_S$ over $S$, see Definition \ref{definition-affine-n-space}.
This is true because
$\mathcal{O}_S[T_1, \ldots, T_n] = \text{Sym}(\mathcal{O}_S^{\oplus n})$.

\begin{lemma}
\label{lemma-category-vector-bundles}
The category of vector bundles over a scheme $S$ is
anti-equivalent to the category of quasi-coherent $\mathcal{O}_S$-modules.
\end{lemma}

\begin{proof}
Omitted. Hint: In one direction one uses the functor
$\underline{\Spec}_S(\text{Sym}^*_{\mathcal{O}_S}(-))$
and in the other the functor
$(\pi : V \to S) \leadsto (\pi_*\mathcal{O}_V)_1$ where the subscript
indicates we take the degree $1$ part.
\end{proof}




\section{Cones}
\label{section-cone}

\noindent
In algebraic geometry cones correspond to graded algebras. By our conventions
a graded ring or algebra $A$ comes with a grading
$A = \bigoplus_{d \geq 0} A_d$ by the nonnegative integers, see
Algebra, Section \ref{algebra-section-graded}.

\begin{definition}
\label{definition-cone}
Let $S$ be a scheme. Let $\mathcal{A}$ be a quasi-coherent
graded $\mathcal{O}_S$-algebra. Assume that $\mathcal{O}_S \to \mathcal{A}_0$
is an isomorphism\footnote{Often one imposes the assumption that
$\mathcal{A}$ is generated by $\mathcal{A}_1$ over $\mathcal{O}_S$. We do not
assume this in order to be consistent with \cite[II, (8.3.1)]{EGA}.}.
The {\it cone associated to $\mathcal{A}$} or the
{\it affine cone associated to $\mathcal{A}$}
is
$$
C(\mathcal{A}) = \underline{\Spec}_S(\mathcal{A}).
$$
\end{definition}

\noindent
The cone associated to a graded sheaf of $\mathcal{O}_S$-algebras
comes with a bit of extra structure. Namely, we obtain a grading
$$
\pi_*\mathcal{O}_{C(\mathcal{A})} =
\bigoplus\nolimits_{n \geq 0} \mathcal{A}_n
$$
Thus we can define an abstract cone as follows.

\begin{definition}
\label{definition-abstract-cone}
Let $S$ be a scheme. A {\it cone $\pi : C \to S$ over $S$} is an
affine morphism of schemes such that $\pi_*\mathcal{O}_C$ is endowed with
the structure of a graded $\mathcal{O}_S$-algebra
$\pi_*\mathcal{O}_C = \bigoplus\nolimits_{n \geq 0} \mathcal{A}_n$
such that $\mathcal{A}_0 = \mathcal{O}_S$. A {\it morphism of cones}
from $\pi : C \to S$ to $\pi' : C' \to S$
is a morphism $f : C \to C'$ such that the induced map
$$
f^* : \pi'_*\mathcal{O}_{C'} \longrightarrow \pi_*\mathcal{O}_C
$$
is compatible with the given gradings.
\end{definition}

\noindent
Any vector bundle is an example of a cone. In fact the category of
vector bundles over $S$ is a full subcategory of the category of cones
over $S$.










\section{Proj of a graded ring}
\label{section-proj}

\noindent
In this section we construct Proj of a graded ring
following \cite[II, Section 2]{EGA}.

\medskip\noindent
Let $S$ be a graded ring. Consider the topological space $\text{Proj}(S)$
associated to $S$, see Algebra, Section \ref{algebra-section-proj}.
We will endow this space with a sheaf of rings $\mathcal{O}_{\text{Proj}(S)}$
such that the resulting pair $(\text{Proj}(S), \mathcal{O}_{\text{Proj}(S)})$
will be a scheme.

\medskip\noindent
Recall that $\text{Proj}(S)$ has a basis of open sets $D_{+}(f)$,
$f \in S_d$, $d \geq 1$ which we call {\it standard opens}, see Algebra,
Section \ref{algebra-section-proj}. This terminology will always
imply that $f$ is homogeneous of positive degree even if we forget to
mention it. In addition, the intersection of two standard opens is another:
$D_{+}(f) \cap D_{+}(g) = D_{+}(fg)$, for $f, g \in S$ homogeneous of positive
degree.

\begin{lemma}
\label{lemma-standard-open}
Let $S$ be a graded ring. Let $f \in S$ homogeneous of positive degree.
\begin{enumerate}
\item If $g\in S$ homogeneous of positive degree
and $D_{+}(g) \subset D_{+}(f)$, then
\begin{enumerate}
\item $f$ is invertible in $S_g$, and
$f^{\deg(g)}/g^{\deg(f)}$ is invertible in $S_{(g)}$,
\item $g^e = af$ for some $e \geq 1$ and $a \in S$ homogeneous,
\item there is a canonical $S$-algebra map $S_f \to S_g$,
\item there is a canonical $S_0$-algebra map $S_{(f)} \to S_{(g)}$
compatible with the map $S_f \to S_g$,
\item the map $S_{(f)} \to S_{(g)}$ induces an isomorphism
$$
(S_{(f)})_{g^{\deg(f)}/f^{\deg(g)}} \cong S_{(g)},
$$
\item these maps induce a commutative diagram of
topological spaces
$$
\xymatrix{
D_{+}(g) \ar[d] &
\{\mathbf{Z}\text{-graded primes of }S_g\} \ar[l] \ar[r] \ar[d] &
\Spec(S_{(g)}) \ar[d] \\
D_{+}(f) &
\{\mathbf{Z}\text{-graded primes of }S_f\} \ar[l] \ar[r] &
\Spec(S_{(f)})
}
$$
where the horizontal maps are homeomorphisms and the vertical maps
are open immersions,
\item there are compatible canonical $S_f$ and $S_{(f)}$-module
maps $M_f \to M_g$ and $M_{(f)} \to M_{(g)}$ for any graded $S$-module $M$,
and
\item the map $M_{(f)} \to M_{(g)}$ induces an isomorphism
$$
(M_{(f)})_{g^{\deg(f)}/f^{\deg(g)}} \cong M_{(g)}.
$$
\end{enumerate}
\item Any open covering of $D_{+}(f)$ can be refined to a finite
open covering of the form $D_{+}(f) = \bigcup_{i = 1}^n D_{+}(g_i)$.
\item Let $g_1, \ldots, g_n \in S$ be homogeneous of positive degree.
Then $D_{+}(f) \subset \bigcup D_{+}(g_i)$
if and only if
$g_1^{\deg(f)}/f^{\deg(g_1)}, \ldots, g_n^{\deg(f)}/f^{\deg(g_n)}$
generate the unit ideal in $S_{(f)}$.
\end{enumerate}
\end{lemma}

\begin{proof}
Recall that $D_{+}(g) = \Spec(S_{(g)})$ with identification
given by the ring maps $S \to S_g \leftarrow S_{(g)}$, see
Algebra, Lemma \ref{algebra-lemma-topology-proj}.
Thus $f^{\deg(g)}/g^{\deg(f)}$ is an element of $S_{(g)}$ which is not
contained in any prime ideal, and hence invertible,
see Algebra, Lemma \ref{algebra-lemma-Zariski-topology}.
We conclude that (a) holds.
Write the inverse of $f$ in $S_g$ as $a/g^d$.
We may replace $a$ by its homogeneous part of degree $d\deg(g) - \deg(f)$.
This means $g^d - af$ is annihilated by a power of $g$, whence
$g^e = af$ for some $a \in S$ homogeneous of degree $e\deg(g) - \deg(f)$.
This proves (b).
For (c), the map $S_f \to S_g$ exists by (a) from the universal property
of localization, or we can define it by mapping $b/f^n$
to $a^nb/g^{ne}$. This clearly induces a map of the subrings
$S_{(f)} \to S_{(g)}$ of degree zero elements as well.
We can similarly define $M_f \to M_g$ and $M_{(f)} \to M_{(g)}$ by mapping
$x/f^n$ to $a^nx/g^{ne}$. The statements writing $S_{(g)}$
resp.\ $M_{(g)}$ as principal localizations of $S_{(f)}$ resp.\ $M_{(f)}$
are clear from the formulas above. The maps in the commutative diagram
of topological spaces correspond to the ring maps given above. The
horizontal arrows are homeomorphisms by
Algebra, Lemma \ref{algebra-lemma-topology-proj}.
The vertical arrows are open immersions since the left
one is the inclusion of an open subset.

\medskip\noindent
The open $D_{+}(f)$ is quasi-compact because it is homeomorphic
to $\Spec(S_{(f)})$, see  Algebra, Lemma \ref{algebra-lemma-quasi-compact}.
Hence the second statement follows directly
from the fact that the standard opens form
a basis for the topology.

\medskip\noindent
The third statement follows directly from
Algebra, Lemma \ref{algebra-lemma-Zariski-topology}.
\end{proof}

\noindent
In Sheaves, Section \ref{sheaves-section-bases} we defined
the notion of a sheaf on a basis, and we showed that it is
essentially equivalent to the notion of a sheaf on the space,
see Sheaves, Lemmas \ref{sheaves-lemma-extend-off-basis} and
\ref{sheaves-lemma-extend-off-basis-structures}. Moreover,
we showed in
Sheaves, Lemma \ref{sheaves-lemma-cofinal-systems-coverings-standard-case}
that it is sufficient to check the sheaf
condition on a cofinal system of open coverings for each
standard open. By the lemma above it suffices to check
on the finite coverings by standard opens.

\begin{definition}
\label{definition-standard-covering}
Let $S$ be a graded ring.
Suppose that $D_{+}(f) \subset \text{Proj}(S)$ is a standard
open. A {\it standard open covering} of $D_{+}(f)$
is a covering $D_{+}(f) = \bigcup_{i = 1}^n D_{+}(g_i)$,
where $g_1, \ldots, g_n \in S$ are homogeneous of positive degree.
\end{definition}

\noindent
Let $S$ be a graded ring. Let $M$ be a graded $S$-module. We will define
a presheaf $\widetilde M$ on the basis of standard opens.
Suppose that $U \subset \text{Proj}(S)$ is a standard open.
If $f, g \in S$ are homogeneous of positive degree
such that $D_{+}(f) = D_{+}(g)$, then
by Lemma \ref{lemma-standard-open} above there are canonical
maps $M_{(f)} \to M_{(g)}$ and $M_{(g)} \to M_{(f)}$ which are
mutually inverse. Hence we may choose any $f$ such that $U = D_{+}(f)$
and define
$$
\widetilde M(U) = M_{(f)}.
$$
Note that if $D_{+}(g) \subset D_{+}(f)$, then by
Lemma \ref{lemma-standard-open} above we have
a canonical map
$$
\widetilde M(D_{+}(f)) = M_{(f)} \longrightarrow
M_{(g)} = \widetilde M(D_{+}(g)).
$$
Clearly, this defines a presheaf of abelian groups on the basis
of standard opens. If $M = S$, then $\widetilde S$ is a presheaf
of rings on the basis of standard opens. And for general $M$ we
see that $\widetilde M$ is a presheaf of $\widetilde S$-modules
on the basis of standard opens.

\medskip\noindent
Let us compute the stalk of $\widetilde M$ at a point
$x \in \text{Proj}(S)$.
Suppose that $x$ corresponds to the homogeneous prime
ideal $\mathfrak p \subset S$.
By definition of the stalk we see that
$$
\widetilde M_x
=
\colim_{f\in S_d, d > 0, f\not\in \mathfrak p} M_{(f)}
$$
Here the set $\{f \in S_d, d > 0, f \not \in \mathfrak p\}$ is preordered by
the rule $f \geq f' \Leftrightarrow D_{+}(f) \subset D_{+}(f')$.
If $f_1, f_2 \in S \setminus \mathfrak p$ are homogeneous of positive
degree, then we have
$f_1f_2 \geq f_1$ in this ordering. In
Algebra, Section \ref{algebra-section-proj}
we defined $M_{(\mathfrak p)}$ as the module whose elements are fractions
$x/f$ with $x, f$ homogeneous, $\deg(x) = \deg(f)$, $f \not \in \mathfrak p$.
Since $\mathfrak p \in \text{Proj}(S)$ there exists at least
one $f_0 \in S$ homogeneous of positive degree with $f_0 \not\in \mathfrak p$.
Hence $x/f = f_0x/ff_0$ and we see that we may always assume
the denominator of an element in $M_{(\mathfrak p)}$ has positive degree.
From these remarks it follows easily that
$$
\widetilde M_x = M_{(\mathfrak p)}.
$$

\medskip\noindent
Next, we check the sheaf condition for the standard open coverings.
If $D_{+}(f) = \bigcup_{i = 1}^n D_{+}(g_i)$, then the sheaf condition
for this covering is equivalent with the exactness of the
sequence
$$
0 \to M_{(f)} \to \bigoplus M_{(g_i)} \to \bigoplus M_{(g_ig_j)}.
$$
Note that $D_{+}(g_i) = D_{+}(fg_i)$, and hence we can rewrite this
sequence as the sequence
$$
0 \to M_{(f)} \to \bigoplus M_{(fg_i)} \to \bigoplus M_{(fg_ig_j)}.
$$
By Lemma \ref{lemma-standard-open} we see that
$g_1^{\deg(f)}/f^{\deg(g_1)}, \ldots, g_n^{\deg(f)}/f^{\deg(g_n)}$
generate the unit ideal in $S_{(f)}$, and that the modules
$M_{(fg_i)}$, $M_{(fg_ig_j)}$ are the principal localizations
of the $S_{(f)}$-module $M_{(f)}$ at these elements and their products.
Thus we may apply Algebra, Lemma \ref{algebra-lemma-cover-module}
to the module $M_{(f)}$ over $S_{(f)}$ and the elements
$g_1^{\deg(f)}/f^{\deg(g_1)}, \ldots, g_n^{\deg(f)}/f^{\deg(g_n)}$.
We conclude that the sequence is exact. By the remarks
made above, we see that $\widetilde M$ is a sheaf
on the basis of standard opens.

\medskip\noindent
Thus we conclude from the material in
Sheaves, Section \ref{sheaves-section-bases}
that there exists a
unique sheaf of rings $\mathcal{O}_{\text{Proj}(S)}$
which agrees with $\widetilde S$ on the standard opens.
Note that by our computation of stalks above and
Algebra, Lemma \ref{algebra-lemma-proj-prime} the
stalks of this sheaf of rings are all local rings.

\medskip\noindent
Similarly, for any graded $S$-module $M$ there exists
a unique sheaf of $\mathcal{O}_{\text{Proj}(S)}$-modules
$\mathcal{F}$ which agrees with $\widetilde M$ on the
standard opens, see
Sheaves, Lemma \ref{sheaves-lemma-extend-off-basis-module}.

\begin{definition}
\label{definition-structure-sheaf}
Let $S$ be a graded ring.
\begin{enumerate}
\item The {\it structure sheaf $\mathcal{O}_{\text{Proj}(S)}$ of the
homogeneous spectrum of $S$} is the unique sheaf of rings
$\mathcal{O}_{\text{Proj}(S)}$
which agrees with $\widetilde S$ on the basis of standard opens.
\item The locally ringed space
$(\text{Proj}(S), \mathcal{O}_{\text{Proj}(S)})$ is called
the {\it homogeneous spectrum} of $S$ and denoted $\text{Proj}(S)$.
\item The sheaf of $\mathcal{O}_{\text{Proj}(S)}$-modules
extending $\widetilde M$ to all opens of $\text{Proj}(S)$
is called the sheaf of $\mathcal{O}_{\text{Proj}(S)}$-modules
associated to $M$. This sheaf is denoted $\widetilde M$ as
well.
\end{enumerate}
\end{definition}

\noindent
We summarize the results obtained so far.

\begin{lemma}
\label{lemma-proj-sheaves}
Let $S$ be a graded ring. Let $M$ be a graded $S$-module.
Let $\widetilde M$ be the sheaf of $\mathcal{O}_{\text{Proj}(S)}$-modules
associated to $M$.
\begin{enumerate}
\item For every $f \in S$ homogeneous of positive degree we have
$$
\Gamma(D_{+}(f), \mathcal{O}_{\text{Proj}(S)}) = S_{(f)}.
$$
\item For every $f\in S$ homogeneous of positive degree
we have $\Gamma(D_{+}(f), \widetilde M) = M_{(f)}$
as an $S_{(f)}$-module.
\item Whenever $D_{+}(g) \subset D_{+}(f)$ the restriction mappings
on $\mathcal{O}_{\text{Proj}(S)}$ and $\widetilde M$
are the maps
$S_{(f)} \to S_{(g)}$ and $M_{(f)} \to M_{(g)}$ from Lemma
\ref{lemma-standard-open}.
\item Let $\mathfrak p$ be a homogeneous prime of $S$ not containing
$S_{+}$, and let $x \in \text{Proj}(S)$
be the corresponding point. We have
$\mathcal{O}_{\text{Proj}(S), x} = S_{(\mathfrak p)}$.
\item Let $\mathfrak p$ be a homogeneous prime of $S$ not containing
$S_{+}$, and let $x \in \text{Proj}(S)$
be the corresponding point. We have $\mathcal{F}_x = M_{(\mathfrak p)}$
as an $S_{(\mathfrak p)}$-module.
\item
\label{item-map}
There is a canonical ring map
$
S_0 \longrightarrow \Gamma(\text{Proj}(S), \widetilde S)
$
and a canonical $S_0$-module map
$
M_0 \longrightarrow \Gamma(\text{Proj}(S), \widetilde M)
$
compatible with the descriptions of sections over standard opens
and stalks above.
\end{enumerate}
Moreover, all these identifications are functorial in the graded
$S$-module $M$. In particular, the functor $M \mapsto \widetilde M$
is an exact functor from the category of graded $S$-modules
to the category of $\mathcal{O}_{\text{Proj}(S)}$-modules.
\end{lemma}

\begin{proof}
Assertions (1) - (5) are clear from the discussion above.
We see (6) since there are canonical maps $M_0 \to M_{(f)}$,
$x \mapsto x/1$ compatible with the restriction maps
described in (3). The exactness of the functor $M \mapsto \widetilde M$
follows from the fact that the functor $M \mapsto M_{(\mathfrak p)}$
is exact (see Algebra, Lemma \ref{algebra-lemma-proj-prime})
and the fact that exactness of short exact sequences
may be checked on stalks, see
Modules, Lemma \ref{modules-lemma-abelian}.
\end{proof}

\begin{remark}
\label{remark-global-sections-not-isomorphism}
The map from $M_0$ to the global sections of $\widetilde M$
is generally far from being an isomorphism. A trivial
example is to take $S = k[x, y, z]$ with $1 = \deg(x) = \deg(y) = \deg(z)$
(or any number of variables) and to take $M = S/(x^{100}, y^{100}, z^{100})$.
It is easy to see that $\widetilde M = 0$, but $M_0 = k$.
\end{remark}

\begin{lemma}
\label{lemma-standard-open-proj}
Let $S$ be a graded ring. Let $f \in S$ be homogeneous of positive degree.
Suppose that $D(g) \subset \Spec(S_{(f)})$ is a standard open.
Then there exists an $h \in S$ homogeneous of positive degree such that
$D(g)$ corresponds to $D_{+}(h) \subset D_{+}(f)$ via the homeomorphism
of Algebra, Lemma \ref{algebra-lemma-topology-proj}. In fact we can
take $h$ such that $g = h/f^n$ for some $n$.
\end{lemma}

\begin{proof}
Write $g = h/f^n$ for some $h$ homogeneous of positive degree
and some $n \geq 1$. If $D_{+}(h)$ is not contained in
$D_{+}(f)$ then we replace $h$ by $hf$ and $n$ by $n + 1$.
Then $h$ has the required shape and $D_{+}(h) \subset D_{+}(f)$
corresponds to $D(g) \subset \Spec(S_{(f)})$.
\end{proof}

\begin{lemma}
\label{lemma-proj-scheme}
Let $S$ be a graded ring.
The locally ringed space $\text{Proj}(S)$ is a scheme.
The standard opens $D_{+}(f)$ are affine opens.
For any graded $S$-module $M$ the sheaf
$\widetilde M$ is a quasi-coherent sheaf of
$\mathcal{O}_{\text{Proj}(S)}$-modules.
\end{lemma}

\begin{proof}
Consider a standard open $D_{+}(f) \subset \text{Proj}(S)$.
By Lemmas \ref{lemma-standard-open} and \ref{lemma-proj-sheaves}
we have $\Gamma(D_{+}(f), \mathcal{O}_{\text{Proj}(S)}) = S_{(f)}$, and
we have a homeomorphism $\varphi : D_{+}(f) \to \Spec(S_{(f)})$.
For any standard open $D(g) \subset \Spec(S_{(f)})$ we may
pick an $h \in S_{+}$ as in Lemma \ref{lemma-standard-open-proj}.
Then $\varphi^{-1}(D(g)) = D_{+}(h)$, and by
Lemmas \ref{lemma-proj-sheaves} and \ref{lemma-standard-open} we see
$$
\Gamma(D_{+}(h), \mathcal{O}_{\text{Proj}(S)})
=
S_{(h)}
=
(S_{(f)})_{h^{\deg(f)}/f^{\deg(h)}}
=
(S_{(f)})_g
=
\Gamma(D(g), \mathcal{O}_{\Spec(S_{(f)})}).
$$
Thus the restriction of $\mathcal{O}_{\text{Proj}(S)}$ to
$D_{+}(f)$ corresponds via the homeomorphism $\varphi$
exactly to the sheaf $\mathcal{O}_{\Spec(S_{(f)})}$
as defined in Schemes, Section \ref{schemes-section-affine-schemes}.
We conclude that $D_{+}(f)$ is an affine scheme isomorphic to
$\Spec(S_{(f)})$ via $\varphi$ and
hence that $\text{Proj}(S)$ is a scheme.

\medskip\noindent
In exactly the same way we show that $\widetilde M$ is a
quasi-coherent sheaf of $\mathcal{O}_{\text{Proj}(S)}$-modules.
Namely, the argument above will show that
$$
\widetilde M|_{D_{+}(f)} \cong \varphi^*\left(\widetilde{M_{(f)}}\right)
$$
which shows that $\widetilde M$ is quasi-coherent.
\end{proof}

\begin{lemma}
\label{lemma-proj-separated}
Let $S$ be a graded ring.
The scheme $\text{Proj}(S)$ is separated.
\end{lemma}

\begin{proof}
We have to show that the canonical morphism
$\text{Proj}(S) \to \Spec(\mathbf{Z})$
is separated.
We will use Schemes, Lemma \ref{schemes-lemma-characterize-separated}.
Thus it suffices to show given any pair of standard opens
$D_{+}(f)$ and $D_{+}(g)$ that $D_{+}(f) \cap D_{+}(g) = D_{+}(fg)$
is affine (clear) and that the ring map
$$
S_{(f)} \otimes_{\mathbf{Z}} S_{(g)} \longrightarrow S_{(fg)}
$$
is surjective. Any element $s$ in $S_{(fg)}$ is of
the form $s = h/(f^ng^m)$ with $h \in S$ homogeneous of degree
$n\deg(f) + m\deg(g)$. We may multiply $h$ by a suitable
monomial $f^ig^j$ and assume that $n = n' \deg(g)$, and
$m = m' \deg(f)$. Then we can rewrite $s$ as
$s = h/f^{(n' + m')\deg(g)} \cdot f^{m'\deg(g)}/g^{m'\deg(f)}$.
So $s$ is indeed in the image of the displayed arrow.
\end{proof}

\begin{lemma}
\label{lemma-proj-quasi-compact}
Let $S$ be a graded ring.
The scheme $\text{Proj}(S)$ is quasi-compact if and only
if there exist finitely many homogeneous elements
$f_1, \ldots, f_n \in S_{+}$ such that
$S_{+} \subset \sqrt{(f_1, \ldots, f_n)}$. In this case
$\text{Proj}(S) = D_+(f_1) \cup \ldots \cup D_+(f_n)$.
\end{lemma}

\begin{proof}
Given such a collection of elements the standard affine opens
$D_{+}(f_i)$ cover $\text{Proj}(S)$ by
Algebra, Lemma \ref{algebra-lemma-topology-proj}.
Conversely, if $\text{Proj}(S)$ is quasi-compact, then we
may cover it by finitely many standard opens
$D_{+}(f_i)$, $i = 1, \ldots, n$ and we see that
$S_{+} \subset \sqrt{(f_1, \ldots, f_n)}$ by the
lemma referenced above.
\end{proof}

\begin{lemma}
\label{lemma-structure-morphism-proj}
Let $S$ be a graded ring. The scheme $\text{Proj}(S)$ has a canonical morphism
towards the affine scheme $\Spec(S_0)$, agreeing with the map on
topological spaces coming from
Algebra, Definition \ref{algebra-definition-proj}.
\end{lemma}

\begin{proof}
We saw above that our construction of $\widetilde S$,
resp.\ $\widetilde M$ gives a sheaf of $S_0$-algebras, resp.\ $S_0$-modules.
Hence we get a morphism by
Schemes, Lemma \ref{schemes-lemma-morphism-into-affine}.
This morphism, when restricted to $D_{+}(f)$ comes from the
canonical ring map $S_0 \to S_{(f)}$. The maps
$S \to S_f$, $S_{(f)} \to S_f$ are $S_0$-algebra maps, see
Lemma \ref{lemma-standard-open}.
Hence if the homogeneous prime $\mathfrak p \subset S$
corresponds to the $\mathbf{Z}$-graded prime $\mathfrak p' \subset S_f$
and the (usual) prime $\mathfrak p'' \subset S_{(f)}$, then
each of these has the same inverse image in $S_0$.
\end{proof}

\begin{lemma}
\label{lemma-proj-valuative-criterion}
Let $S$ be a graded ring. If $S$ is finitely generated as
an algebra over $S_0$, then
the morphism $\text{Proj}(S) \to \Spec(S_0)$ satisfies
the existence and uniqueness parts of the valuative criterion,
see Schemes, Definition \ref{schemes-definition-valuative-criterion}.
\end{lemma}

\begin{proof}
The uniqueness part follows from the fact that $\text{Proj}(S)$ is
separated (Lemma \ref{lemma-proj-separated} and
Schemes, Lemma \ref{schemes-lemma-separated-implies-valuative}).
Choose $x_i \in S_{+}$ homogeneous, $i = 1, \ldots, n$
which generate $S$ over $S_0$. Let $d_i = \deg(x_i)$ and
set $d = \text{lcm}\{d_i\}$. Suppose we are given a diagram
$$
\xymatrix{
\Spec(K) \ar[r] \ar[d] & \text{Proj}(S) \ar[d] \\
\Spec(A) \ar[r] & \Spec(S_0)
}
$$
as in Schemes, Definition \ref{schemes-definition-valuative-criterion}.
Denote $v : K^* \to \Gamma$ the valuation of $A$, see
Algebra, Definition \ref{algebra-definition-value-group}.
We may choose an $f \in S_{+}$ homogeneous such that
$\Spec(K)$ maps into $D_{+}(f)$. Then we get a commutative
diagram of ring maps
$$
\xymatrix{
K & S_{(f)} \ar[l]^{\varphi} \\
A \ar[u] & S_0 \ar[l] \ar[u]
}
$$
After renumbering we may assume that $\varphi(x_i^{\deg(f)}/f^{d_i})$
is nonzero for $i = 1, \ldots, r$ and zero for $i = r + 1, \ldots, n$.
Since the open sets $D_{+}(x_i)$ cover $\text{Proj}(S)$ we see that $r \geq 1$.
Let $i_0 \in \{1, \ldots, r\}$ be an index minimizing
$\gamma_i = (d/d_i)v(\varphi(x_i^{\deg(f)}/f^{d_i}))$ in $\Gamma$.
For convenience set $x_0 = x_{i_0}$ and $d_0 = d_{i_0}$.
The ring map $\varphi$ factors though a map $\varphi' : S_{(fx_0)} \to K$
which gives a ring map $S_{(x_0)} \to S_{(fx_0)} \to K$.
The algebra $S_{(x_0)}$ is generated over $S_0$ by the elements
$x_1^{e_1} \ldots x_n^{e_n}/x_0^{e_0}$, where $\sum e_i d_i = e_0 d_0$.
If $e_i > 0$ for some $i > r$, then
$\varphi'(x_1^{e_1} \ldots x_n^{e_n}/x_0^{e_0}) = 0$.
If $e_i = 0$ for $i > r$, then we have
\begin{align*}
d \deg(f) v(\varphi'(x_1^{e_1} \ldots x_r^{e_r}/x_0^{e_0}))
& =
d v(\varphi'(x_1^{e_1 \deg(f)} \ldots x_r^{e_r \deg(f)}/x_0^{e_0 \deg(f)})) \\
& =
d \sum e_i v(\varphi'(x_i^{\deg(f)}/f^{d_i}))
- e_0 v(\varphi'(x_0^{\deg(f)}/f^{d_0})) \\
& =
\sum e_i d_i \gamma_i - e_0 d_0 \gamma_0 \\
& \geq
\sum e_i d_i \gamma_0 - e_0 d_0 \gamma_0 = 0
\end{align*}
because $\gamma_0$ is minimal among the $\gamma_i$.
This implies that $S_{(x_0)}$ maps into $A$ via $\varphi'$.
The corresponding morphism of schemes
$\Spec(A) \to \Spec(S_{(x_0)}) = D_{+}(x_0)
\subset \text{Proj}(S)$ provides the morphism fitting into
the first commutative diagram of this proof.
\end{proof}

\noindent
We saw in the proof of Lemma \ref{lemma-proj-valuative-criterion}
that, under the hypotheses of that lemma, the morphism
$\text{Proj}(S) \to \Spec(S_0)$ is quasi-compact as well. Hence (by
Schemes, Proposition \ref{schemes-proposition-characterize-universally-closed})
we see that $\text{Proj}(S) \to \Spec(S_0)$ is universally closed in
the situation of the lemma. We give several examples showing these results
do not hold without some assumption on the graded ring $S$.

\begin{example}
\label{example-not-existence-valuative-big-proj}
Let $\mathbf{C}[X_1, X_2, X_3, \ldots]$ be the graded $\mathbf{C}$-algebra
with each $X_i$ in degree $1$. Consider the ring map
$$
\mathbf{C}[X_1, X_2, X_3, \ldots]
\longrightarrow
\mathbf{C}[t^\alpha ; \alpha \in \mathbf{Q}_{\geq 0}]
$$
which maps $X_i$ to $t^{1/i}$. The right hand side becomes a valuation ring
$A$ upon localization at the ideal $\mathfrak m = (t^\alpha ; \alpha > 0)$.
Let $K$ be the fraction field of $A$. The above gives a morphism
$\Spec(K) \to \text{Proj}(\mathbf{C}[X_1, X_2, X_3, \ldots])$ which does not
extend to a morphism defined on all of $\Spec(A)$.
The reason is that the image of $\Spec(A)$ would be contained
in one of the $D_{+}(X_i)$ but then $X_{i + 1}/X_i$ would map
to an element of $A$ which it doesn't since it maps to
$t^{1/(i + 1) - 1/i}$.
\end{example}

\begin{example}
\label{example-not-existence-valuative-small-proj}
Let $R = \mathbf{C}[t]$ and
$$
S = R[X_1, X_2, X_3, \ldots]/(X_i^2 - tX_{i + 1}).
$$
The grading is such that $R = S_0$ and $\deg(X_i) = 2^{i - 1}$.
Note that if $\mathfrak p \in \text{Proj}(S)$ then
$t \not \in \mathfrak p$ (otherwise $\mathfrak p$ has to contain
all of the $X_i$ which is not allowed for an element of
the homogeneous spectrum). Thus we see that
$D_{+}(X_i) = D_{+}(X_{i + 1})$ for all $i$. Hence
$\text{Proj}(S)$ is quasi-compact; in fact it is affine
since it is equal to $D_{+}(X_1)$. It is easy to see that
the image of $\text{Proj}(S) \to \Spec(R)$ is
$D(t)$. Hence the morphism $\text{Proj}(S) \to \Spec(R)$
is not closed. Thus the valuative criterion cannot apply because
it would imply that the morphism is closed (see
Schemes, Proposition \ref{schemes-proposition-characterize-universally-closed}
).
\end{example}

\begin{example}
\label{example-trivial-proj}
Let $A$ be a ring.
Let $S = A[T]$ as a graded $A$ algebra with $T$ in degree $1$.
Then the canonical morphism $\text{Proj}(S) \to \Spec(A)$
(see Lemma \ref{lemma-structure-morphism-proj})
is an isomorphism.
\end{example}

\begin{example}
\label{example-open-subset-proj}
Let $X = \Spec(A)$ be an affine scheme, and let $U \subset X$
be an open subscheme.  Grade $A[T]$ by setting $\deg T = 1$.  Define $S$
to be the subring of $A[T]$ generated by $A$ and all $fT^i$, where $i \ge 0$
and where $f \in A$ is such that $D(f) \subset U$.  We claim that $S$
is a graded ring with $S_0 = A$ such that $\text{Proj}(S) \cong U$,
and this isomorphism identifies the canonical morphism
$\text{Proj}(S) \to \Spec(A)$ of Lemma \ref{lemma-structure-morphism-proj}
with the inclusion $U \subset X$.

\medskip\noindent
Suppose $\mathfrak p \in \text{Proj}(S)$ is such that every $fT \in S_1$
is in $\mathfrak p$.  Then every generator $fT^i$ with $i \ge 1$
is in $\mathfrak p$ because $(fT^i)^2 = (fT)(fT^{2i-1}) \in \mathfrak p$
and $\mathfrak p$ is radical.  But then $\mathfrak p \supset S_+$, which
is impossible. Consequently $\text{Proj}(S)$ is covered by the standard
open affine subsets $\{D_+(fT)\}_{fT \in S_1}$.

\medskip\noindent
Observe that, if $fT \in S_1$, then the inclusion $S \subset A[T]$
induces a graded isomorphism of $S[(fT)^{-1}]$ with $A[T, T^{-1}, f^{-1}]$.
Hence the standard open subset $D_+(fT) \cong \Spec(S_{(fT)})$
is isomorphic to $\Spec(A[T, T^{-1}, f^{-1}]_0) = \Spec(A[f^{-1}])$.
It is clear that this isomorphism is a restriction of the canonical morphism
$\text{Proj}(S) \to \Spec(A)$.  If in addition $gT \in S_1$, then
$S[(fT)^{-1}, (gT)^{-1}] \cong A[T, T^{-1}, f^{-1}, g^{-1}]$
as graded rings, so $D_+(fT) \cap D_+(gT) \cong \Spec(A[f^{-1}, g^{-1}])$.
Therefore $\text{Proj}(S)$ is the union of open subschemes $D_+(fT)$
which are isomorphic to the open subschemes $D(f) \subset X$
under the canonical morphism, and these open subschemes intersect
in $\text{Proj}(S)$ in the same way they do in $X$.
We conclude that the canonical morphism is an isomorphism of
$\text{Proj}(S)$ with the union of all $D(f) \subset U$, which is $U$.
\end{example}





















\section{Quasi-coherent sheaves on Proj}
\label{section-quasi-coherent-proj}

\noindent
Let $S$ be a graded ring. Let $M$ be a graded $S$-module.
We saw in Lemma \ref{lemma-proj-sheaves} how to construct a quasi-coherent
sheaf of modules $\widetilde{M}$ on $\text{Proj}(S)$ and a map
\begin{equation}
\label{equation-map-global-sections}
M_0 \longrightarrow \Gamma(\text{Proj}(S), \widetilde{M})
\end{equation}
of the degree $0$ part of $M$ to the global sections of $\widetilde{M}$.
The degree $0$ part of the $n$th twist $M(n)$ of the graded module $M$ (see
Algebra, Section \ref{algebra-section-graded})
is equal to $M_n$. Hence we can get maps
\begin{equation}
\label{equation-map-global-sections-degree-n}
M_n \longrightarrow \Gamma(\text{Proj}(S), \widetilde{M(n)}).
\end{equation}
We would like to be able to perform this operation for any quasi-coherent
sheaf $\mathcal{F}$ on $\text{Proj}(S)$. We will do this by tensoring
with the $n$th twist of the structure sheaf, see
Definition \ref{definition-twist}. In order to relate the two notions
we will use the following lemma.

\begin{lemma}
\label{lemma-widetilde-tensor}
Let $S$ be a graded ring.
Let $(X, \mathcal{O}_X) = (\text{Proj}(S), \mathcal{O}_{\text{Proj}(S)})$
be the scheme of Lemma \ref{lemma-proj-scheme}.
Let $f \in S_{+}$ be homogeneous. Let $x \in X$ be a point
corresponding to the homogeneous prime $\mathfrak p \subset S$.
Let $M$, $N$ be graded $S$-modules.
There is a canonical map of $\mathcal{O}_{\text{Proj}(S)}$-modules
$$
\widetilde M \otimes_{\mathcal{O}_X} \widetilde N
\longrightarrow
\widetilde{M \otimes_S N}
$$
which induces the canonical map
$
M_{(f)} \otimes_{S_{(f)}} N_{(f)}
\to
(M \otimes_S N)_{(f)}
$
on sections over $D_{+}(f)$ and the canonical map
$
M_{(\mathfrak p)} \otimes_{S_{(\mathfrak p)}} N_{(\mathfrak p)}
\to
(M \otimes_S N)_{(\mathfrak p)}
$
on stalks at $x$. Moreover, the following diagram
$$
\xymatrix{
M_0 \otimes_{S_0} N_0 \ar[r] \ar[d] &
(M \otimes_S N)_0 \ar[d] \\
\Gamma(X, \widetilde M \otimes_{\mathcal{O}_X} \widetilde N) \ar[r] &
\Gamma(X, \widetilde{M \otimes_S N})
}
$$
is commutative where the vertical maps are given by
(\ref{equation-map-global-sections}).
\end{lemma}

\begin{proof}
To construct a morphism as displayed is the same as constructing
a $\mathcal{O}_X$-bilinear map
$$
\widetilde M \times \widetilde N
\longrightarrow
\widetilde{M \otimes_S N}
$$
see Modules, Section \ref{modules-section-tensor-product}.
It suffices to define this on sections over the opens $D_{+}(f)$
compatible with restriction mappings. On $D_{+}(f)$ we use the
$S_{(f)}$-bilinear map
$M_{(f)} \times N_{(f)} \to (M \otimes_S N)_{(f)}$,
$(x/f^n, y/f^m) \mapsto (x \otimes y)/f^{n + m}$. Details omitted.
\end{proof}

\begin{remark}
\label{remark-not-isomorphism}
In general the map constructed in Lemma \ref{lemma-widetilde-tensor}
above is not an isomorphism. Here is an example. Let $k$
be a field. Let $S = k[x, y, z]$ with $k$ in degree $0$ and
$\deg(x) = 1$, $\deg(y) = 2$, $\deg(z) = 3$.
Let $M = S(1)$ and $N = S(2)$, see
Algebra, Section \ref{algebra-section-graded}
for notation. Then $M \otimes_S N = S(3)$.
Note that
\begin{eqnarray*}
S_z
& = &
k[x, y, z, 1/z] \\
S_{(z)}
& = &
k[x^3/z, xy/z, y^3/z^2]
\cong
k[u, v, w]/(uw - v^3) \\
M_{(z)} & = & S_{(z)} \cdot x + S_{(z)} \cdot y^2/z \subset S_z \\
N_{(z)} & = & S_{(z)} \cdot y + S_{(z)} \cdot x^2 \subset S_z \\
S(3)_{(z)} & = & S_{(z)} \cdot z \subset S_z
\end{eqnarray*}
Consider the maximal ideal $\mathfrak m = (u, v, w) \subset S_{(z)}$.
It is not hard to see that both $M_{(z)}/\mathfrak mM_{(z)}$
and $N_{(z)}/\mathfrak mN_{(z)}$ have dimension $2$ over
$\kappa(\mathfrak m)$. But
$S(3)_{(z)}/\mathfrak mS(3)_{(z)}$ has dimension $1$.
Thus the map $M_{(z)} \otimes N_{(z)} \to S(3)_{(z)}$ is not
an isomorphism.
\end{remark}









\section{Invertible sheaves on Proj}
\label{section-invertible-on-proj}

\noindent
Recall from Algebra, Section \ref{algebra-section-graded}
the construction of the twisted module $M(n)$ associated
to a graded module over a graded ring.

\begin{definition}
\label{definition-twist}
Let $S$ be a graded ring. Let $X = \text{Proj}(S)$.
\begin{enumerate}
\item We define $\mathcal{O}_X(n) = \widetilde{S(n)}$.
This is called the $n$th
{\it twist of the structure sheaf of $\text{Proj}(S)$}.
\item For any sheaf of $\mathcal{O}_X$-modules $\mathcal{F}$ we set
$\mathcal{F}(n) = \mathcal{F} \otimes_{\mathcal{O}_X} \mathcal{O}_X(n)$.
\end{enumerate}
\end{definition}

\noindent
We are going to use Lemma \ref{lemma-widetilde-tensor}
to construct some canonical maps.
Since $S(n) \otimes_S S(m) = S(n + m)$ we see that there
are canonical maps
\begin{equation}
\label{equation-multiply}
\mathcal{O}_X(n) \otimes_{\mathcal{O}_X} \mathcal{O}_X(m)
\longrightarrow
\mathcal{O}_X(n + m).
\end{equation}
These maps are not isomorphisms in general, see the example in
Remark \ref{remark-not-isomorphism}. The same example shows
that $\mathcal{O}_X(n)$ is {\it not} an invertible sheaf on $X$ in
general. Tensoring with an arbitrary $\mathcal{O}_X$-module $\mathcal{F}$
we get maps
\begin{equation}
\label{equation-multiply-on-sheaf}
\mathcal{O}_X(n) \otimes_{\mathcal{O}_X} \mathcal{F}(m)
\longrightarrow
\mathcal{F}(n + m).
\end{equation}
The maps (\ref{equation-map-global-sections-degree-n})
on global sections give a map of graded rings
\begin{equation}
\label{equation-global-sections}
S \longrightarrow \bigoplus\nolimits_{n \geq 0} \Gamma(X, \mathcal{O}_X(n)).
\end{equation}
And for an arbitrary $\mathcal{O}_X$-module $\mathcal{F}$ the maps
(\ref{equation-multiply-on-sheaf}) give a graded module structure
\begin{equation}
\label{equation-global-sections-module}
\bigoplus\nolimits_{n \geq 0} \Gamma(X, \mathcal{O}_X(n))
\times
\bigoplus\nolimits_{m \in \mathbf{Z}} \Gamma(X, \mathcal{F}(m))
\longrightarrow
\bigoplus\nolimits_{m \in \mathbf{Z}} \Gamma(X, \mathcal{F}(m))
\end{equation}
and via (\ref{equation-global-sections}) also a $S$-module structure.
More generally, given any graded $S$-module
$M$ we have $M(n) = M \otimes_S S(n)$. Hence we get maps
\begin{equation}
\label{equation-multiply-more-generally}
\widetilde M(n)
=
\widetilde M
\otimes_{\mathcal{O}_X}
\mathcal{O}_X(n)
\longrightarrow
\widetilde{M(n)}.
\end{equation}
On global sections (\ref{equation-map-global-sections-degree-n})
defines a map of graded $S$-modules
\begin{equation}
\label{equation-global-sections-more-generally}
M \longrightarrow
\bigoplus\nolimits_{n \in \mathbf{Z}} \Gamma(X, \widetilde{M(n)}).
\end{equation}
Here is an important fact which follows basically immediately from the
definitions.

\begin{lemma}
\label{lemma-when-invertible}
Let $S$ be a graded ring. Set $X = \text{Proj}(S)$.
Let $f \in S$ be homogeneous of degree $d > 0$.
The sheaves $\mathcal{O}_X(nd)|_{D_{+}(f)}$ are invertible,
and in fact trivial for all $n \in \mathbf{Z}$
(see Modules, Definition \ref{modules-definition-invertible}).
The maps (\ref{equation-multiply}) restricted to $D_{+}(f)$
$$
\mathcal{O}_X(nd)|_{D_{+}(f)} \otimes_{\mathcal{O}_{D_{+}(f)}}
\mathcal{O}_X(m)|_{D_{+}(f)}
\longrightarrow
\mathcal{O}_X(nd + m)|_{D_{+}(f)},
$$
the maps (\ref{equation-multiply-on-sheaf}) restricted to $D_+(f)$
$$
\mathcal{O}_X(nd)|_{D_{+}(f)} \otimes_{\mathcal{O}_{D_{+}(f)}}
\mathcal{F}(m)|_{D_{+}(f)}
\longrightarrow
\mathcal{F}(nd + m)|_{D_{+}(f)},
$$
and the maps (\ref{equation-multiply-more-generally})
restricted to $D_{+}(f)$
$$
\widetilde M(nd)|_{D_{+}(f)}
=
\widetilde M|_{D_{+}(f)}
\otimes_{\mathcal{O}_{D_{+}(f)}}
\mathcal{O}_X(nd)|_{D_{+}(f)}
\longrightarrow
\widetilde{M(nd)}|_{D_{+}(f)}
$$
are isomorphisms for all $n, m \in \mathbf{Z}$.
\end{lemma}

\begin{proof}
The (not graded) $S$-module maps $S \to S(nd)$, and $M \to M(nd)$, given by
$x \mapsto f^n x$ become isomorphisms after inverting $f$. The first shows that
$S_{(f)} \cong S(nd)_{(f)}$ which gives an isomorphism
$\mathcal{O}_{D_{+}(f)} \cong \mathcal{O}_X(nd)|_{D_{+}(f)}$.
The second shows that the map
$S(nd)_{(f)} \otimes_{S_{(f)}} M_{(f)} \to M(nd)_{(f)}$
is an isomorphism. The case of the map (\ref{equation-multiply-on-sheaf})
is a consequence of the case of the map (\ref{equation-multiply}).
\end{proof}

\begin{lemma}
\label{lemma-apply-modules}
Let $S$ be a graded ring. Let $M$ be a graded $S$-module.
Set $X = \text{Proj}(S)$. Assume $X$ is covered by the standard
opens $D_+(f)$ with $f \in S_1$, e.g., if $S$ is generated by $S_1$
over $S_0$. Then the sheaves $\mathcal{O}_X(n)$
are invertible and the maps
(\ref{equation-multiply}), (\ref{equation-multiply-on-sheaf}), and
(\ref{equation-multiply-more-generally}) are isomorphisms.
In particular, these maps induce isomorphisms
$$
\mathcal{O}_X(1)^{\otimes n} \cong
\mathcal{O}_X(n)
\quad
\text{and}
\quad
\widetilde{M} \otimes_{\mathcal{O}_X} \mathcal{O}_X(n) =
\widetilde{M}(n) \cong \widetilde{M(n)}
$$
Thus (\ref{equation-map-global-sections-degree-n}) becomes a map
\begin{equation}
\label{equation-map-global-sections-degree-n-simplified}
M_n \longrightarrow \Gamma(X, \widetilde{M}(n))
\end{equation}
and (\ref{equation-global-sections-more-generally}) becomes a map
\begin{equation}
\label{equation-global-sections-more-generally-simplified}
M \longrightarrow
\bigoplus\nolimits_{n \in \mathbf{Z}} \Gamma(X, \widetilde{M}(n)).
\end{equation}
\end{lemma}

\begin{proof}
Under the assumptions of the lemma $X$ is covered by the
open subsets $D_{+}(f)$ with $f \in S_1$ and the
lemma is a consequence of Lemma \ref{lemma-when-invertible} above.
\end{proof}

\begin{lemma}
\label{lemma-where-invertible}
Let $S$ be a graded ring. Set $X = \text{Proj}(S)$. Fix $d \geq 1$ an
integer. The following open subsets of $X$ are equal:
\begin{enumerate}
\item The largest open subset $W = W_d \subset X$ such that
each $\mathcal{O}_X(dn)|_W$ is invertible and all the
multiplication maps
$\mathcal{O}_X(nd)|_W \otimes_{\mathcal{O}_W} \mathcal{O}_X(md)|_W
\to \mathcal{O}_X(nd + md)|_W$
(see \ref{equation-multiply}) are isomorphisms.
\item The union of the open subsets $D_{+}(fg)$ with
$f, g \in S$ homogeneous and $\deg(f) = \deg(g) + d$.
\end{enumerate}
Moreover, all the maps
$\widetilde M(nd)|_W = \widetilde M|_W \otimes_{\mathcal{O}_W}
\mathcal{O}_X(nd)|_W \to \widetilde{M(nd)}|_W$
(see \ref{equation-multiply-more-generally}) are isomorphisms.
\end{lemma}

\begin{proof}
If $x \in D_{+}(fg)$ with $\deg(f) = \deg(g) + d$ then
on $D_{+}(fg)$ the sheaves $\mathcal{O}_X(dn)$
are generated by the element $(f/g)^n = f^{2n}/(fg)^n$. This implies $x$
is in the open subset $W$ defined in (1) by arguing as in the
proof of Lemma \ref{lemma-when-invertible}.

\medskip\noindent
Conversely, suppose that $\mathcal{O}_X(d)$ is free of rank 1
in an open neighbourhood $V$ of $x \in X$ and all the
multiplication maps
$\mathcal{O}_X(nd)|_V \otimes_{\mathcal{O}_V} \mathcal{O}_X(md)|_V
\to \mathcal{O}_X(nd + md)|_V$ are isomorphisms.
We may choose $h \in S_{+}$ homogeneous such that $x \in D_{+}(h) \subset V$.
By the definition of the twists of the structure sheaf we conclude there
exists an element $s$ of $(S_h)_d$ such that $s^n$ is a basis of $(S_h)_{nd}$
as a module over $S_{(h)}$ for all $n \in \mathbf{Z}$.
We may write
$s = f/h^m$ for some $m \geq 1$ and $f \in S_{d + m \deg(h)}$.
Set $g = h^m$ so $s = f/g$. Note that $x \in D_{+}(g)$ by construction.
Note that $g^d \in (S_h)_{-d\deg(g)}$.
By assumption we can write this as a multiple of
$s^{\deg(g)} = f^{\deg(g)}/g^{\deg(g)}$, say
$g^d = a/g^e \cdot f^{\deg(g)}/g^{\deg(g)}$.
Then we conclude that $g^{d + e + \deg(g)} = a f^{\deg(g)}$
and hence also $x \in D_{+}(f)$. So $x$ is an element of the set defined
in (2).

\medskip\noindent
The existence of the generating section $s = f/g$ over
the affine open $D_{+}(fg)$ whose
powers freely generate the sheaves of modules
$\mathcal{O}_X(nd)$ easily implies that the multiplication maps
$\widetilde M(nd)|_W = \widetilde M|_W \otimes_{\mathcal{O}_W}
\mathcal{O}_X(nd)|_W \to \widetilde{M(nd)}|_W$
(see \ref{equation-multiply-more-generally})
are isomorphisms. Compare with the proof of Lemma \ref{lemma-when-invertible}.
\end{proof}

\noindent
Recall from Modules, Lemma \ref{modules-lemma-s-open}
that given an invertible sheaf $\mathcal{L}$ on a locally ringed
space $X$, and given a global section $s$ of $\mathcal{L}$
the set $X_s = \{x \in X \mid s \not \in \mathfrak m_x\mathcal{L}_x\}$
is open.

\begin{lemma}
\label{lemma-principal-open}
Let $S$ be a graded ring. Set $X = \text{Proj}(S)$. Fix $d \geq 1$ an
integer. Let $W = W_d \subset X$ be the open subscheme defined in
Lemma \ref{lemma-where-invertible}. Let $n \geq 1$ and $f \in S_{nd}$.
Denote $s \in \Gamma(W, \mathcal{O}_W(nd))$ the section which is
the image of $f$ via (\ref{equation-global-sections}) restricted to $W$. Then
$$
W_s = D_{+}(f) \cap W.
$$
\end{lemma}

\begin{proof}
Let $D_{+}(ab) \subset W$ be a standard affine open with
$a, b \in S$ homogeneous and $\deg(a) = \deg(b) + d$.
Note that $D_{+}(ab) \cap D_{+}(f) = D_{+}(abf)$.
On the other hand the restriction of $s$ to $D_{+}(ab)$
corresponds to the element $f/1 = b^nf/a^n (a/b)^n \in (S_{ab})_{nd}$.
We have seen in the proof of Lemma \ref{lemma-where-invertible} that
$(a/b)^n$ is a generator for $\mathcal{O}_W(nd)$ over $D_{+}(ab)$.
We conclude that $W_s \cap D_{+}(ab)$ is the principal open
associated to $b^nf/a^n \in \mathcal{O}_X(D_{+}(ab))$.
Thus the result of the lemma is clear.
\end{proof}

\noindent
The following lemma states the properties that we will later use to
characterize schemes with an ample invertible sheaf.

\begin{lemma}
\label{lemma-ample-on-proj}
Let $S$ be a graded ring.
Let $X = \text{Proj}(S)$.
Let $Y \subset X$ be a quasi-compact open subscheme.
Denote $\mathcal{O}_Y(n)$ the restriction of
$\mathcal{O}_X(n)$ to $Y$.
There exists an integer $d \geq 1$ such that
\begin{enumerate}
\item the subscheme $Y$ is contained in the open $W_d$ defined
in Lemma \ref{lemma-where-invertible},
\item the sheaf $\mathcal{O}_Y(dn)$ is invertible for all $n \in \mathbf{Z}$,
\item all the maps
$\mathcal{O}_Y(nd) \otimes_{\mathcal{O}_Y} \mathcal{O}_Y(m)
\longrightarrow
\mathcal{O}_Y(nd + m)$
of Equation (\ref{equation-multiply}) are isomorphisms,
\item all the maps
$\widetilde M(nd)|_Y = \widetilde M|_Y \otimes_{\mathcal{O}_Y}
\mathcal{O}_X(nd)|_Y \to \widetilde{M(nd)}|_Y$
(see \ref{equation-multiply-more-generally}) are isomorphisms,
\item given $f \in S_{nd}$ denote $s \in \Gamma(Y, \mathcal{O}_Y(nd))$
the image of $f$ via (\ref{equation-global-sections})
restricted to $Y$, then $D_{+}(f) \cap Y = Y_s$,
\item a basis for the topology on $Y$ is given
by the collection of opens $Y_s$, where $s \in \Gamma(Y, \mathcal{O}_Y(nd))$,
$n \geq 1$, and
\item a basis for the topology of $Y$ is given
by those opens $Y_s \subset Y$, for
$s \in \Gamma(Y, \mathcal{O}_Y(nd))$, $n \geq 1$ which are affine.
\end{enumerate}
\end{lemma}

\begin{proof}
Since $Y$ is quasi-compact there exist finitely many homogeneous
$f_i \in S_{+}$, $i = 1, \ldots, n$ such that the standard opens
$D_{+}(f_i)$ give an open covering of $Y$. Let $d_i = \deg(f_i)$ and set
$d = d_1 \ldots d_n$. Note that $D_{+}(f_i) = D_{+}(f_i^{d/d_i})$
and hence we see immediately that $Y \subset W_d$, by characterization
(2) in Lemma \ref{lemma-where-invertible} or
by (1) using Lemma \ref{lemma-when-invertible}.
Note that (1) implies (2), (3) and (4) by Lemma \ref{lemma-where-invertible}.
(Note that (3) is a special case of (4).)
Assertion (5) follows from Lemma \ref{lemma-principal-open}.
Assertions (6) and (7) follow because the open subsets $D_{+}(f)$
form a basis for the topology of $X$ and are affine.
\end{proof}

\begin{lemma}
\label{lemma-comparison-proj-quasi-coherent}
Let $S$ be a graded ring. Set $X = \text{Proj}(S)$.
Let $\mathcal{F}$ be a quasi-coherent $\mathcal{O}_X$-module.
Set $M = \bigoplus_{n \in \mathbf{Z}} \Gamma(X, \mathcal{F}(n))$ as
a graded $S$-module, using
(\ref{equation-global-sections-module}) and (\ref{equation-global-sections}).
Then there is a canonical $\mathcal{O}_X$-module map
$$
\widetilde{M} \longrightarrow \mathcal{F}
$$
functorial in $\mathcal{F}$ such that the induced map
$M_0 \to \Gamma(X, \mathcal{F})$ is the identity.
\end{lemma}

\begin{proof}
Let $f \in S$ be homogeneous of degree $d > 0$. Recall that
$\widetilde{M}|_{D_{+}(f)}$ corresponds to the
$S_{(f)}$-module $M_{(f)}$ by Lemma \ref{lemma-proj-sheaves}.
Thus we can define a canonical map
$$
M_{(f)} \longrightarrow \Gamma(D_+(f), \mathcal{F}),\quad
m/f^n \longmapsto m|_{D_+(f)} \otimes f|_{D_+(f)}^{-n}
$$
which makes sense because $f|_{D_+(f)}$ is a trivializing
section of the invertible sheaf $\mathcal{O}_X(d)|_{D_+(f)}$, see
Lemma \ref{lemma-when-invertible} and its proof.
Since $\widetilde{M}$ is quasi-coherent, this leads to a canonical
map
$$
\widetilde{M}|_{D_+(f)} \longrightarrow \mathcal{F}|_{D_+(f)}
$$
via Schemes, Lemma \ref{schemes-lemma-compare-constructions}.
We obtain a global map if we prove that the displayed maps glue on overlaps.
Proof of this is omitted. We also omit the proof of the final statement.
\end{proof}










\section{Functoriality of Proj}
\label{section-functoriality-proj}

\noindent
A graded ring map $\psi : A \to B$ does not always give rise to a morphism of
associated projective homogeneous spectra. The reason is that
the inverse image $\psi^{-1}(\mathfrak q)$
of a homogeneous prime $\mathfrak q \subset B$ may
contain the irrelevant prime $A_{+}$ even if $\mathfrak q$ does not
contain $B_{+}$.
The correct result is stated as follows.

\begin{lemma}
\label{lemma-morphism-proj}
Let $A$, $B$ be two graded rings.
Set $X = \text{Proj}(A)$ and $Y = \text{Proj}(B)$.
Let $\psi : A \to B$ be a graded ring map.
Set
$$
U(\psi)
=
\bigcup\nolimits_{f \in A_{+}\ \text{homogeneous}} D_{+}(\psi(f))
\subset Y.
$$
Then there is a canonical morphism of schemes
$$
r_\psi :
U(\psi)
\longrightarrow
X
$$
and a map of $\mathbf{Z}$-graded $\mathcal{O}_{U(\psi)}$-algebras
$$
\theta = \theta_\psi :
r_\psi^*\left(
\bigoplus\nolimits_{d \in \mathbf{Z}} \mathcal{O}_X(d)
\right)
\longrightarrow
\bigoplus\nolimits_{d \in \mathbf{Z}} \mathcal{O}_{U(\psi)}(d).
$$
The triple $(U(\psi), r_\psi, \theta)$ is
characterized by the following properties:
\begin{enumerate}
\item For every $d \geq 0$ the diagram
$$
\xymatrix{
A_d \ar[d] \ar[rr]_{\psi} & &
B_d \ar[d] \\
\Gamma(X, \mathcal{O}_X(d)) \ar[r]^-\theta &
\Gamma(U(\psi), \mathcal{O}_Y(d)) &
\Gamma(Y, \mathcal{O}_Y(d)) \ar[l]
}
$$
is commutative.
\item For any $f \in A_{+}$ homogeneous
we have $r_\psi^{-1}(D_{+}(f)) = D_{+}(\psi(f))$ and
the restriction of $r_\psi$ to $D_{+}(\psi(f))$
corresponds to the ring map
$A_{(f)} \to B_{(\psi(f))}$ induced by $\psi$.
\end{enumerate}
\end{lemma}

\begin{proof}
Clearly condition (2) uniquely determines the morphism of schemes
and the open subset $U(\psi)$. Pick $f \in A_d$ with $d \geq 1$.
Note that
$\mathcal{O}_X(n)|_{D_{+}(f)}$ corresponds to the
$A_{(f)}$-module $(A_f)_n$ and that
$\mathcal{O}_Y(n)|_{D_{+}(\psi(f))}$ corresponds to the
$B_{(\psi(f))}$-module $(B_{\psi(f)})_n$. In other words $\theta$
when restricted to $D_{+}(\psi(f))$ corresponds to a map of
$\mathbf{Z}$-graded $B_{(\psi(f))}$-algebras
$$
A_f \otimes_{A_{(f)}} B_{(\psi(f))}
\longrightarrow
B_{\psi(f)}
$$
Condition (1) determines the images of all elements of $A$.
Since $f$ is an invertible element which is mapped to $\psi(f)$
we see that $1/f^m$ is mapped to $1/\psi(f)^m$. It easily follows
from this that $\theta$ is uniquely determined, namely it is
given by the rule
$$
a/f^m \otimes b/\psi(f)^e \longmapsto \psi(a)b/\psi(f)^{m + e}.
$$
To show existence we remark that the proof of uniqueness above gave
a well defined prescription for the morphism $r$ and the map $\theta$
when restricted to every standard open of the form
$D_{+}(\psi(f)) \subset U(\psi)$ into $D_{+}(f)$.
Call these $r_f$ and $\theta_f$.
Hence we only need to verify that if $D_{+}(f) \subset D_{+}(g)$
for some $f, g \in A_{+}$ homogeneous, then the restriction of
$r_g$ to $D_{+}(\psi(f))$ matches $r_f$. This is clear from the
formulas given for $r$ and $\theta$ above.
\end{proof}

\begin{lemma}
\label{lemma-morphism-proj-transitive}
Let $A$, $B$, and $C$ be graded rings.
Set $X = \text{Proj}(A)$, $Y = \text{Proj}(B)$ and $Z = \text{Proj}(C)$.
Let $\varphi : A \to B$, $\psi : B \to C$ be graded ring maps.
Then we have
$$
U(\psi \circ \varphi) = r_\psi^{-1}(U(\varphi))
\quad
\text{and}
\quad
r_{\psi \circ \varphi}
=
r_\varphi \circ r_\psi|_{U(\psi \circ \varphi)}.
$$
In addition we have
$$
\theta_\psi \circ r_\psi^*\theta_\varphi
=
\theta_{\psi \circ \varphi}
$$
with obvious notation.
\end{lemma}

\begin{proof}
Omitted.
\end{proof}

\begin{lemma}
\label{lemma-surjective-graded-rings-map-proj}
With hypotheses and notation as in Lemma \ref{lemma-morphism-proj} above.
Assume $A_d \to B_d$ is surjective for all $d \gg 0$. Then
\begin{enumerate}
\item $U(\psi) = Y$,
\item $r_\psi : Y \to X$ is a closed immersion, and
\item the maps $\theta : r_\psi^*\mathcal{O}_X(n) \to \mathcal{O}_Y(n)$
are surjective but not isomorphisms in general (even if $A \to B$ is
surjective).
\end{enumerate}
\end{lemma}

\begin{proof}
Part (1) follows from the definition of $U(\psi)$ and the fact that
$D_{+}(f) = D_{+}(f^n)$ for any $n > 0$. For $f \in A_{+}$ homogeneous
we see that $A_{(f)} \to B_{(\psi(f))}$ is surjective because
any element of $B_{(\psi(f))}$ can be represented by a fraction
$b/\psi(f)^n$ with $n$ arbitrarily large (which forces the degree of
$b \in B$ to be large). This proves (2).
The same argument shows the map
$$
A_f \to B_{\psi(f)}
$$
is surjective which proves the surjectivity of $\theta$.
For an example where this map is not an isomorphism
consider the graded ring $A = k[x, y]$ where $k$ is a field
and $\deg(x) = 1$, $\deg(y) = 2$. Set $I = (x)$, so that
$B = k[y]$. Note that $\mathcal{O}_Y(1) = 0$ in this case.
But it is easy to see that $r_\psi^*\mathcal{O}_X(1)$
is not zero. (There are less silly examples.)
\end{proof}

\begin{lemma}
\label{lemma-eventual-iso-graded-rings-map-proj}
With hypotheses and notation as in Lemma \ref{lemma-morphism-proj} above.
Assume $A_d \to B_d$ is an isomorphism for all $d \gg 0$. Then
\begin{enumerate}
\item $U(\psi) = Y$,
\item $r_\psi : Y \to X$ is an isomorphism, and
\item the maps $\theta : r_\psi^*\mathcal{O}_X(n) \to \mathcal{O}_Y(n)$
are isomorphisms.
\end{enumerate}
\end{lemma}

\begin{proof}
We have (1) by Lemma \ref{lemma-surjective-graded-rings-map-proj}.
Let $f \in A_{+}$ be homogeneous. The assumption on $\psi$ implies that
$A_f \to B_f$ is an isomorphism (details omitted). Thus it is clear that
$r_\psi$ and $\theta$ restrict to isomorphisms over $D_{+}(f)$.
The lemma follows.
\end{proof}

\begin{lemma}
\label{lemma-surjective-graded-rings-generated-degree-1-map-proj}
With hypotheses and notation as in Lemma \ref{lemma-morphism-proj} above.
Assume $A_d \to B_d$ is surjective for $d \gg 0$ and that $A$ is generated
by $A_1$ over $A_0$. Then
\begin{enumerate}
\item $U(\psi) = Y$,
\item $r_\psi : Y \to X$ is a closed immersion, and
\item the maps $\theta : r_\psi^*\mathcal{O}_X(n) \to \mathcal{O}_Y(n)$
are isomorphisms.
\end{enumerate}
\end{lemma}

\begin{proof}
By Lemmas \ref{lemma-eventual-iso-graded-rings-map-proj} and
\ref{lemma-morphism-proj-transitive}
we may replace $B$ by the image of $A \to B$
without changing $X$ or the sheaves $\mathcal{O}_X(n)$.
Thus we may assume that $A \to B$ is surjective. By
Lemma \ref{lemma-surjective-graded-rings-map-proj} we get (1) and (2)
and surjectivity in (3).
By Lemma \ref{lemma-apply-modules} we see that both
$\mathcal{O}_X(n)$ and $\mathcal{O}_Y(n)$
are invertible. Hence $\theta$ is an isomorphism.
\end{proof}

\begin{lemma}
\label{lemma-base-change-map-proj}
With hypotheses and notation as in Lemma \ref{lemma-morphism-proj} above.
Assume there exists a ring map $R \to A_0$ and a ring map
$R \to R'$ such that $B = R' \otimes_R A$. Then
\begin{enumerate}
\item $U(\psi) = Y$,
\item the diagram
$$
\xymatrix{
Y = \text{Proj}(B) \ar[r]_{r_\psi} \ar[d] &
\text{Proj}(A) = X \ar[d] \\
\Spec(R') \ar[r] &
\Spec(R)
}
$$
is a fibre product square, and
\item the maps $\theta : r_\psi^*\mathcal{O}_X(n) \to \mathcal{O}_Y(n)$
are isomorphisms.
\end{enumerate}
\end{lemma}

\begin{proof}
This follows immediately by looking at what happens over the standard
opens $D_{+}(f)$ for $f \in A_{+}$.
\end{proof}

\begin{lemma}
\label{lemma-localization-map-proj}
With hypotheses and notation as in Lemma \ref{lemma-morphism-proj} above.
Assume there exists a $g \in A_0$ such that $\psi$ induces an
isomorphism $A_g \to B$. Then
$U(\psi) = Y$, $r_\psi : Y \to X$ is an open immersion
which induces an isomorphism of $Y$ with the inverse image
of $D(g) \subset \Spec(A_0)$. Moreover the map $\theta$
is an isomorphism.
\end{lemma}

\begin{proof}
This is a special case of Lemma \ref{lemma-base-change-map-proj} above.
\end{proof}

\begin{lemma}
\label{lemma-d-uple}
Let $S$ be a graded ring. Let $d \geq 1$. Set $S' = S^{(d)}$ with notation
as in Algebra, Section \ref{algebra-section-graded}. Set
$X = \text{Proj}(S)$ and $X' = \text{Proj}(S')$. There is a canonical
isomorphism $i : X \to X'$ of schemes such that
\begin{enumerate}
\item for any graded $S$-module $M$ setting $M' = M^{(d)}$,
we have a canonical isomorphism $\widetilde{M} \to i^*\widetilde{M'}$,
\item we have canonical isomorphisms
$\mathcal{O}_{X}(nd) \to i^*\mathcal{O}_{X'}(n)$
\end{enumerate}
and these isomorphisms are compatible with the multiplication maps
of Lemma \ref{lemma-widetilde-tensor} and hence with the maps
(\ref{equation-multiply}),
(\ref{equation-multiply-on-sheaf}),
(\ref{equation-global-sections}),
(\ref{equation-global-sections-module}),
(\ref{equation-multiply-more-generally}), and
(\ref{equation-global-sections-more-generally}) (see proof for precise
statements.
\end{lemma}

\begin{proof}
The injective ring map $S' \to S$ (which is not a homomorphism of graded rings
due to our conventions), induces a map $j : \Spec(S) \to \Spec(S')$.
Given a graded prime ideal $\mathfrak p \subset S$ we see that
$\mathfrak p' = j(\mathfrak p) = S' \cap \mathfrak p$
is a graded prime ideal of $S'$.
Moreover, if $f \in S_+$ is homogeneous and $f \not \in \mathfrak p$, then
$f^d \in S'_+$ and $f^d \not \in \mathfrak p'$. Conversely, if
$\mathfrak p' \subset S'$ is a graded prime ideal not containing some
homogeneous element $f \in S'_+$, then
$\mathfrak p = \{g \in S \mid g^d \in \mathfrak p'\}$ is a
graded prime ideal of $S$ not containing $f$ whose image under $j$
is $\mathfrak p'$. To see that $\mathfrak p$ is an ideal, note
that if $g, h \in \mathfrak p$, then
$(g + h)^{2d} \in \mathfrak p'$ by the binomial formula
and hence $g + h \in \mathfrak p'$ as $\mathfrak p'$ is a prime.
In this way we see that $j$ induces a homeomorphism $i : X \to X'$.
Moreover, given $f \in S_+$ homogeneous, then we have
$S_{(f)} \cong S'_{(f^d)}$. Since these isomorphisms are compatible
with the restrictions mappings of
Lemma \ref{lemma-standard-open}, we see that there exists an
isomorphism $i^\sharp : i^{-1}\mathcal{O}_{X'} \to \mathcal{O}_X$ of
structure sheaves on $X$ and $X'$, hence $i$ is an isomorphism
of schemes.

\medskip\noindent
Let $M$ be a graded $S$-module. Given $f \in S_+$ homogeneous, we have
$M_{(f)} \cong M'_{(f^d)}$, hence in exactly the same manner as above
we obtain the isomorphism in (1). The isomorphisms in (2) are a special
case of (1) for $M = S(nd)$ which gives $M' = S'(n)$. Let $M$ and $N$
be graded $S$-modules. Then we have
$$
M' \otimes_{S'} N' =
(M \otimes_S N)^{(d)} =
(M \otimes_S N)'
$$
as can be verified directly from the definitions. Having said this
the compatibility with the multiplication maps of
Lemma \ref{lemma-widetilde-tensor} is the commutativity of the diagram
$$
\xymatrix{
\widetilde M \otimes_{\mathcal{O}_X} \widetilde N
\ar[d]_{(1) \otimes (1)} \ar[r] &
\widetilde{M \otimes_S N} \ar[d]^{(1)} \\
i^*(\widetilde{M'} \otimes_{\mathcal{O}_{X'}} \widetilde{N'}) \ar[r] &
i^*(\widetilde{M' \otimes_{S'} N'})
}
$$
This can be seen by looking at the construction of the maps
over the open $D_+(f) = D_+(f^d)$ where the top horizontal
arrow is given by the map
$M_{(f)} \times N_{(f)} \to (M \otimes_S N)_{(f)}$
and the lower horizontal arrow by the map
$M'_{(f^d)} \times N'_{(f^d)} \to (M' \otimes_{S'} N')_{(f^d)}$.
Since these maps agree via the identifications
$M_{(f)} = M'_{(f^d)}$, etc, we get the desired compatibility.
We omit the proof of the other compatibilities.
\end{proof}











\section{Morphisms into Proj}
\label{section-morphisms-proj}

\noindent
Let $S$ be a graded ring.
Let $X = \text{Proj}(S)$ be the homogeneous spectrum of $S$.
Let $d \geq 1$ be an integer.
Consider the open subscheme
\begin{equation}
\label{equation-Ud}
U_d = \bigcup\nolimits_{f  \in S_d} D_{+}(f)
\quad\subset\quad
X = \text{Proj}(S)
\end{equation}
Note that $d | d' \Rightarrow U_d \subset U_{d'}$ and
$X = \bigcup_d U_d$. Neither $X$ nor $U_d$ need
be quasi-compact, see Algebra, Lemma \ref{algebra-lemma-topology-proj}.
Let us write $\mathcal{O}_{U_d}(n) = \mathcal{O}_X(n)|_{U_d}$.
By Lemma \ref{lemma-when-invertible}
we know that $\mathcal{O}_{U_d}(nd)$, $n \in \mathbf{Z}$
is an invertible $\mathcal{O}_{U_d}$-module and
that all the multiplication maps
$\mathcal{O}_{U_d}(nd) \otimes_{\mathcal{O}_{U_d}} \mathcal{O}_{U_d}(m)
\to \mathcal{O}_{U_d}(nd + m)$ of
(\ref{equation-multiply}) are isomorphisms. In particular we have
$\mathcal{O}_{U_d}(nd) \cong \mathcal{O}_{U_d}(d)^{\otimes n}$.
The graded ring map (\ref{equation-global-sections}) on global sections
combined with restriction to $U_d$ give a homomorphism of graded rings
\begin{equation}
\label{equation-psi-d}
\psi^d : S^{(d)} \longrightarrow \Gamma_*(U_d, \mathcal{O}_{U_d}(d)).
\end{equation}
For the notation $S^{(d)}$, see Algebra, Section \ref{algebra-section-graded}.
For the notation $\Gamma_*$ see
Modules, Definition \ref{modules-definition-gamma-star}.
Moreover, since $U_d$ is covered by the opens $D_{+}(f)$, $f \in S_d$
we see that $\mathcal{O}_{U_d}(d)$ is globally generated
by the sections in the image of
$\psi^d_1 : S^{(d)}_1 = S_d \to \Gamma(U_d, \mathcal{O}_{U_d}(d))$, see
Modules, Definition \ref{modules-definition-globally-generated}.

\medskip\noindent
Let $Y$ be a scheme, and let $\varphi : Y \to X$ be a morphism of schemes.
Assume the image $\varphi(Y)$ is contained in the open subscheme
$U_d$ of $X$.
By the discussion following
Modules, Definition \ref{modules-definition-gamma-star}
we obtain a homomorphism of graded rings
$$
\Gamma_*(U_d, \mathcal{O}_{U_d}(d))
\longrightarrow
\Gamma_*(Y, \varphi^*\mathcal{O}_X(d)).
$$
The composition of this and $\psi^d$ gives a graded ring
homomorphism
\begin{equation}
\label{equation-psi-phi-d}
\psi_\varphi^d :
S^{(d)}
\longrightarrow
\Gamma_*(Y, \varphi^*\mathcal{O}_X(d))
\end{equation}
which has the property that the invertible sheaf
$\varphi^*\mathcal{O}_X(d)$ is globally generated
by the sections in the image of
$(S^{(d)})_1 = S_d \to \Gamma(Y, \varphi^*\mathcal{O}_X(d))$.

\begin{lemma}
\label{lemma-converse-construction}
Let $S$ be a graded ring, and $X = \text{Proj}(S)$.
Let $d \geq 1$ and $U_d \subset X$ as above.
Let $Y$ be a scheme.
Let $\mathcal{L}$ be an invertible sheaf on $Y$.
Let $\psi : S^{(d)} \to \Gamma_*(Y, \mathcal{L})$ be
a graded ring homomorphism such that $\mathcal{L}$ is
generated by the sections in the image of
$\psi|_{S_d} : S_d \to \Gamma(Y, \mathcal{L})$.
Then there exist a morphism
$\varphi : Y \to X$ such that $\varphi(Y) \subset U_d$ and
an isomorphism $\alpha : \varphi^*\mathcal{O}_{U_d}(d) \to \mathcal{L}$
such that $\psi_\varphi^d$ agrees with $\psi$ via $\alpha$:
$$
\xymatrix{
\Gamma_*(Y, \mathcal{L}) &
\Gamma_*(Y, \varphi^*\mathcal{O}_{U_d}(d)) \ar[l]^-\alpha &
\Gamma_*(U_d, \mathcal{O}_{U_d}(d)) \ar[l]^-{\varphi^*} \\
S^{(d)} \ar[u]^\psi & &
S^{(d)} \ar[u]^{\psi^d} \ar[ul]^{\psi^d_\varphi} \ar[ll]_{\text{id}}
}
$$
commutes. Moreover, the pair $(\varphi, \alpha)$ is unique.
\end{lemma}

\begin{proof}
Pick $f \in S_d$. Denote $s = \psi(f) \in \Gamma(Y, \mathcal{L})$.
On the open set $Y_s$ where $s$ does not vanish multiplication
by $s$ induces an isomorphism $\mathcal{O}_{Y_s} \to \mathcal{L}|_{Y_s}$,
see Modules, Lemma \ref{modules-lemma-s-open}. We will denote
the inverse of this map $x \mapsto x/s$, and similarly for
powers of $\mathcal{L}$. Using this we
define a ring map $\psi_{(f)} : S_{(f)} \to \Gamma(Y_s, \mathcal{O})$
by mapping the fraction $a/f^n$ to $\psi(a)/s^n$.
By Schemes, Lemma \ref{schemes-lemma-morphism-into-affine}
this corresponds to a morphism
$\varphi_f : Y_s \to \Spec(S_{(f)}) = D_{+}(f)$.
We also introduce the isomorphism
$\alpha_f : \varphi_f^*\mathcal{O}_{D_{+}(f)}(d) \to \mathcal{L}|_{Y_s}$
which maps the pullback of the trivializing section
$f$ over $D_{+}(f)$ to the trivializing section $s$ over $Y_s$.
With this choice the commutativity of the diagram in the lemma
holds with $Y$ replaced by $Y_s$, $\varphi$ replaced by $\varphi_f$,
and $\alpha$ replaced by $\alpha_f$; verification omitted.

\medskip\noindent
Suppose that $f' \in S_d$ is a second element, and denote
$s' = \psi(f') \in \Gamma(Y, \mathcal{L})$. Then
$Y_s \cap Y_{s'} = Y_{ss'}$ and similarly
$D_{+}(f) \cap D_{+}(f') = D_{+}(ff')$.
In Lemma \ref{lemma-ample-on-proj} we saw that
$D_{+}(f') \cap D_{+}(f)$ is the same as the set
of points of $D_{+}(f)$ where the section of
$\mathcal{O}_X(d)$ defined by $f'$ does not vanish.
Hence
$\varphi_f^{-1}(D_{+}(f') \cap D_{+}(f)) = Y_s \cap Y_{s'}
= \varphi_{f'}^{-1}(D_{+}(f') \cap D_{+}(f))$.
On $D_{+}(f) \cap D_{+}(f')$ the fraction $f/f'$ is an
invertible section of the structure sheaf with inverse
$f'/f$. Note that $\psi_{(f')}(f/f') = \psi(f)/s' = s/s'$
and $\psi_{(f)}(f'/f) = \psi(f')/s = s'/s$. We claim there
is a unique ring map
$S_{(ff')} \to \Gamma(Y_{ss'}, \mathcal{O})$ making the
following diagram commute
$$
\xymatrix{
\Gamma(Y_s, \mathcal{O}) \ar[r] &
\Gamma(Y_{ss'}, \mathcal{O}) &
\Gamma(Y_{s, '} \mathcal{O}) \ar[l]\\
S_{(f)} \ar[r] \ar[u]^{\psi_{(f)}} &
S_{(ff')} \ar[u] &
S_{(f')} \ar[l] \ar[u]^{\psi_{(f')}}
}
$$
It exists because we may use the rule
$x/(ff')^n \mapsto \psi(x)/(ss')^n$, which ``works'' by the formulas
above. Uniqueness follows as $\text{Proj}(S)$ is separated, see
Lemma \ref{lemma-proj-separated} and its proof. This shows that the
morphisms $\varphi_f$ and $\varphi_{f'}$ agree over $Y_s \cap Y_{s'}$.
The restrictions of $\alpha_f$ and $\alpha_{f'}$ agree over
$Y_s \cap Y_{s'}$ because the regular functions $s/s'$ and
$\psi_{(f')}(f)$ agree. This proves that the morphisms $\psi_f$
glue to a global morphism from $Y$ into $U_d \subset X$, and
that the maps $\alpha_f$ glue to an isomorphism satisfying the
conditions of the lemma.

\medskip\noindent
We still have to show the pair $(\varphi, \alpha)$ is unique.
Suppose $(\varphi', \alpha')$ is a second such pair.
Let $f \in S_d$. By the commutativity of the diagrams in the lemma we have
that the inverse images of $D_{+}(f)$ under both $\varphi$ and $\varphi'$
are equal to $Y_{\psi(f)}$. Since the opens $D_{+}(f)$ are a basis
for the topology on $X$, and since $X$ is a sober topological
space (see Schemes, Lemma \ref{schemes-lemma-scheme-sober})
this means the maps $\varphi$ and $\varphi'$ are the same
on underlying topological spaces. Let us use $s = \psi(f)$ to
trivialize the invertible sheaf $\mathcal{L}$ over $Y_{\psi(f)}$.
By the commutativity of the diagrams we have that
$\alpha^{\otimes n}(\psi^d_{\varphi}(x)) =
\psi(x) = (\alpha')^{\otimes n}(\psi^d_{\varphi'}(x))$
for all $x \in S_{nd}$. By construction of $\psi^d_{\varphi}$
and $\psi^d_{\varphi'}$ we have
$\psi^d_{\varphi}(x) = \varphi^\sharp(x/f^n) \psi^d_{\varphi}(f^n)$
over $Y_{\psi(f)}$,
and similarly for $\psi^d_{\varphi'}$. By the commutativity of
the diagrams of the lemma we deduce that
$\varphi^\sharp(x/f^n) = (\varphi')^\sharp(x/f^n)$.
This proves that $\varphi$ and $\varphi'$ induce the same morphism
from $Y_{\psi(f)}$ into the affine scheme $D_{+}(f) = \Spec(S_{(f)})$.
Hence $\varphi$ and $\varphi'$ are the same as morphisms.
Finally, it remains to show that the commutativity of the
diagram of the lemma singles out, given $\varphi$, a unique
$\alpha$. We omit the verification.
\end{proof}

\noindent
We continue the discussion from above the lemma.
Let $S$ be a graded ring.
Let $Y$ be a scheme. We will consider {\it triples}
$(d, \mathcal{L}, \psi)$ where
\begin{enumerate}
\item $d \geq 1$ is an integer,
\item $\mathcal{L}$ is an invertible $\mathcal{O}_Y$-module, and
\item $\psi : S^{(d)} \to \Gamma_*(Y, \mathcal{L})$ is a graded
ring homomorphism such that $\mathcal{L}$ is generated by
the global sections $\psi(f)$, with $f \in S_d$.
\end{enumerate}
Given a morphism $h : Y' \to Y$ and a triple
$(d, \mathcal{L}, \psi)$ over $Y$ we can pull it back to the
triple $(d, h^*\mathcal{L}, h^* \circ \psi)$.
Given two triples $(d, \mathcal{L}, \psi)$ and
$(d, \mathcal{L}', \psi')$ with the same integer $d$
we say they are {\it strictly equivalent} if there exists
an isomorphism $\beta : \mathcal{L} \to \mathcal{L}'$
such that $\beta \circ \psi = \psi'$ as graded
ring maps $S^{(d)} \to \Gamma_*(Y, \mathcal{L}')$.

\medskip\noindent
For each integer $d \geq 1$ we define
\begin{eqnarray*}
F_d : \Sch^{opp} & \longrightarrow & \textit{Sets}, \\
Y & \longmapsto &
\{\text{strict equivalence classes of triples }
(d, \mathcal{L}, \psi)
\text{ as above}\}
\end{eqnarray*}
with pullbacks as defined above.

\begin{lemma}
\label{lemma-proj-functor-strict}
Let $S$ be a graded ring.
Let $X = \text{Proj}(S)$.
The open subscheme $U_d \subset X$ (\ref{equation-Ud}) represents the
functor $F_d$ and the triple $(d, \mathcal{O}_{U_d}(d), \psi^d)$
defined above is the universal family (see
Schemes, Section \ref{schemes-section-representable}).
\end{lemma}

\begin{proof}
This is a reformulation of Lemma \ref{lemma-converse-construction}
\end{proof}

\begin{lemma}
\label{lemma-apply}
Let $S$ be a graded ring generated as an $S_0$-algebra by
the elements of $S_1$. In this case the scheme $X = \text{Proj}(S)$
represents the functor which associates to a scheme
$Y$ the set of pairs $(\mathcal{L}, \psi)$, where
\begin{enumerate}
\item $\mathcal{L}$ is an invertible $\mathcal{O}_Y$-module, and
\item $\psi : S \to \Gamma_*(Y, \mathcal{L})$ is a graded
ring homomorphism such that $\mathcal{L}$ is generated by
the global sections $\psi(f)$, with $f \in S_1$
\end{enumerate}
up to strict equivalence as above.
\end{lemma}

\begin{proof}
Under the assumptions of the lemma we have $X = U_1$ and the
lemma is a reformulation of Lemma \ref{lemma-proj-functor-strict} above.
\end{proof}

\noindent
We end this section with a discussion of a functor corresponding
to $\text{Proj}(S)$ for a general graded ring $S$.
We advise the reader to skip the rest of this section.

\medskip\noindent
Fix an arbitrary graded ring $S$. Let $T$ be a scheme.
We will say two triples $(d, \mathcal{L}, \psi)$ and
$(d', \mathcal{L}', \psi')$ over $T$ with possibly different integers
$d$, $d'$ are {\it equivalent} if there exists
an isomorphism
$\beta : \mathcal{L}^{\otimes d'} \to (\mathcal{L}')^{\otimes d}$
of invertible sheaves over $T$
such that $\beta \circ \psi|_{S^{(dd')}}$ and $\psi'|_{S^{(dd')}}$ agree
as graded ring maps $S^{(dd')} \to \Gamma_*(Y, (\mathcal{L}')^{\otimes dd'})$.

\begin{lemma}
\label{lemma-equivalent}
Let $S$ be a graded ring. Set $X = \text{Proj}(S)$. Let $T$ be a scheme.
Let $(d, \mathcal{L}, \psi)$ and $(d', \mathcal{L}', \psi')$
be two triples over $T$. The following are equivalent:
\begin{enumerate}
\item Let $n = \text{lcm}(d, d')$. Write $n = ad = a'd'$. There exists
an isomorphism
$\beta : \mathcal{L}^{\otimes a} \to (\mathcal{L}')^{\otimes a'}$
with the property that
$\beta \circ \psi|_{S^{(n)}}$ and $\psi'|_{S^{(n)}}$ agree
as graded ring maps $S^{(n)} \to \Gamma_*(Y, (\mathcal{L}')^{\otimes n})$.
\item The triples $(d, \mathcal{L}, \psi)$ and $(d', \mathcal{L}', \psi')$
are equivalent.
\item For some positive integer $n = ad = a'd'$ there exists
an isomorphism
$\beta : \mathcal{L}^{\otimes a} \to (\mathcal{L}')^{\otimes a'}$
with the property that
$\beta \circ \psi|_{S^{(n)}}$ and $\psi'|_{S^{(n)}}$ agree
as graded ring maps $S^{(n)} \to \Gamma_*(Y, (\mathcal{L}')^{\otimes n})$.
\item The morphisms $\varphi : T \to X$ and $\varphi' : T \to X$
associated to $(d, \mathcal{L}, \psi)$ and $(d', \mathcal{L}', \psi')$
are equal.
\end{enumerate}
\end{lemma}

\begin{proof}
Clearly (1) implies (2) and (2) implies (3) by restricting to
more divisible degrees and powers of invertible sheaves.
Also (3) implies (4) by the uniqueness statement
in Lemma \ref{lemma-converse-construction}.
Thus we have to prove that (4) implies (1). Assume (4),
in other words $\varphi = \varphi'$.
Note that this implies that we may write
$\mathcal{L} = \varphi^*\mathcal{O}_X(d)$ and
$\mathcal{L}' = \varphi^*\mathcal{O}_X(d')$.
Moreover, via these identifications we have that the graded ring
maps $\psi$ and $\psi'$ correspond to the restriction of the canonical
graded ring map
$$
S \longrightarrow \bigoplus\nolimits_{n \geq 0} \Gamma(X, \mathcal{O}_X(n))
$$
to $S^{(d)}$ and $S^{(d')}$ composed with pullback by $\varphi$
(by Lemma \ref{lemma-converse-construction} again). Hence taking
$\beta$ to be the isomorphism
$$
(\varphi^*\mathcal{O}_X(d))^{\otimes a} =
\varphi^*\mathcal{O}_X(n) =
(\varphi^*\mathcal{O}_X(d'))^{\otimes a'}
$$
works.
\end{proof}

\noindent
Let $S$ be a graded ring.
Let $X = \text{Proj}(S)$.
Over the open subscheme scheme $U_d \subset X = \text{Proj}(S)$
(\ref{equation-Ud}) we have the triple
$(d, \mathcal{O}_{U_d}(d), \psi^d)$. Clearly, if $d | d'$ the triples
$(d, \mathcal{O}_{U_d}(d), \psi^d)$ and
$(d', \mathcal{O}_{U_{d'}}(d'), \psi^{d'})$ are equivalent
when restricted to the open $U_d$ (which is a subset of $U_{d'}$).
This, combined with Lemma \ref{lemma-converse-construction} shows
that morphisms $Y \to X$ correspond roughly to
equivalence classes of triples over $Y$. This is not quite true since if $Y$ is
not quasi-compact, then there may not be a single triple which works.
Thus we have to be slightly careful in defining the corresponding functor.

\medskip\noindent
Here is one possible way to do this. Suppose $d' = ad$.
Consider the transformation of functors $F_d \to F_{d'}$
which assigns to the triple $(d, \mathcal{L}, \psi)$ over
$T$ the triple $(d', \mathcal{L}^{\otimes a}, \psi|_{S^{(d')}})$.
One of the implications of Lemma \ref{lemma-equivalent} is that the
transformation $F_d \to F_{d'}$ is injective!
For a quasi-compact scheme $T$ we define
$$
F(T) = \bigcup\nolimits_{d \in \mathbf{N}} F_d(T)
$$
with transition maps as explained above. This clearly defines a
contravariant functor on the category of quasi-compact schemes
with values in sets. For a general scheme
$T$ we define
$$
F(T)
=
\lim_{V \subset T\text{ quasi-compact open}} F(V).
$$
In other words, an element $\xi$ of $F(T)$ corresponds to a compatible system
of choices of elements $\xi_V \in F(V)$ where $V$ ranges over the
quasi-compact opens of $T$.
We omit the definition of the pullback map $F(T) \to F(T')$
for a morphism $T' \to T$ of schemes.
Thus we have defined our functor
\begin{eqnarray*}
F : \Sch^{opp} & \longrightarrow & \textit{Sets}
\end{eqnarray*}

\begin{lemma}
\label{lemma-proj-functor}
Let $S$ be a graded ring.
Let $X = \text{Proj}(S)$.
The functor $F$ defined above is representable by the scheme $X$.
\end{lemma}

\begin{proof}
We have seen above that the functor $F_d$ corresponds to the
open subscheme $U_d \subset X$. Moreover the transformation
of functors $F_d \to F_{d'}$ (if $d | d'$) defined above
corresponds to the inclusion morphism $U_d \to U_{d'}$
(see discussion above). Hence to show that $F$ is represented
by $X$ it suffices to show that $T \to X$ for a quasi-compact
scheme $T$ ends up in some $U_d$, and that for a general scheme
$T$ we have
$$
\Mor(T, X)
=
\lim_{V \subset T\text{ quasi-compact open}} \Mor(V, X).
$$
These verifications are omitted.
\end{proof}







\section{Projective space}
\label{section-projective-space}

\noindent
Projective space is one of the fundamental objects studied in
algebraic geometry. In this section we just give its construction
as $\text{Proj}$ of a polynomial ring. Later we will discover many
of its beautiful properties.

\begin{lemma}
\label{lemma-projective-space}
Let $S = \mathbf{Z}[T_0, \ldots, T_n]$ with $\deg(T_i) = 1$.
The scheme
$$
\mathbf{P}^n_{\mathbf{Z}} = \text{Proj}(S)
$$
represents the functor which associates to a scheme $Y$ the pairs
$(\mathcal{L}, (s_0, \ldots, s_n))$ where
\begin{enumerate}
\item $\mathcal{L}$ is an invertible $\mathcal{O}_Y$-module, and
\item $s_0, \ldots, s_n$ are global sections of $\mathcal{L}$
which generate $\mathcal{L}$
\end{enumerate}
up to the following equivalence:
$(\mathcal{L}, (s_0, \ldots, s_n)) \sim
(\mathcal{N}, (t_0, \ldots, t_n))$ $\Leftrightarrow$ there exists
an isomorphism $\beta : \mathcal{L} \to \mathcal{N}$
with $\beta(s_i) = t_i$ for $i = 0, \ldots, n$.
\end{lemma}

\begin{proof}
This is a special case of Lemma \ref{lemma-apply} above.
Namely, for any graded ring $A$ we have
\begin{eqnarray*}
\Mor_{graded rings}(\mathbf{Z}[T_0, \ldots, T_n], A)
& = &
A_1 \times \ldots \times A_1 \\
\psi & \mapsto & (\psi(T_0), \ldots, \psi(T_n))
\end{eqnarray*}
and the degree $1$ part of $\Gamma_*(Y, \mathcal{L})$ is
just $\Gamma(Y, \mathcal{L})$.
\end{proof}

\begin{definition}
\label{definition-projective-space}
The scheme
$\mathbf{P}^n_{\mathbf{Z}} = \text{Proj}(\mathbf{Z}[T_0, \ldots, T_n])$
is called {\it projective $n$-space over $\mathbf{Z}$}.
Its base change $\mathbf{P}^n_S$ to a scheme $S$ is called
{\it projective $n$-space over $S$}. If $R$ is a ring the base change
to $\Spec(R)$ is denoted $\mathbf{P}^n_R$ and called
{\it projective $n$-space over $R$}.
\end{definition}

\noindent
Given a scheme $Y$ over $S$
and a pair $(\mathcal{L}, (s_0, \ldots, s_n))$ as in
Lemma \ref{lemma-projective-space}
the induced morphism to $\mathbf{P}^n_S$ is denoted
$$
\varphi_{(\mathcal{L}, (s_0, \ldots, s_n))} :
Y \longrightarrow \mathbf{P}^n_S
$$
This makes sense since the pair defines a morphism into
$\mathbf{P}^n_{\mathbf{Z}}$ and we already have the structure
morphism into $S$ so combined we get a morphism into
$\mathbf{P}^n_S = \mathbf{P}^n_{\mathbf{Z}} \times S$.
Note that this is the $S$-morphism characterized by
$$
\mathcal{L} =
\varphi_{(\mathcal{L}, (s_0, \ldots, s_n))}^*\mathcal{O}_{\mathbf{P}^n_R}(1)
\quad
\text{and}
\quad
s_i = \varphi_{(\mathcal{L}, (s_0, \ldots, s_n))}^*T_i
$$
where we think of $T_i$ as a global section of
$\mathcal{O}_{\mathbf{P}^n_S}(1)$ via (\ref{equation-global-sections}).

\begin{lemma}
\label{lemma-standard-covering-projective-space}
Projective $n$-space over $\mathbf{Z}$ is covered by
$n + 1$ standard opens
$$
\mathbf{P}^n_{\mathbf{Z}} =
\bigcup\nolimits_{i = 0, \ldots, n} D_{+}(T_i)
$$
where each $D_{+}(T_i)$ is isomorphic to $\mathbf{A}^n_{\mathbf{Z}}$
affine $n$-space over $\mathbf{Z}$.
\end{lemma}

\begin{proof}
This is true because
$\mathbf{Z}[T_0, \ldots, T_n]_{+} = (T_0, \ldots, T_n)$ and
since
$$
\Spec
\left(
\mathbf{Z}
\left[\frac{T_0}{T_i}, \ldots, \frac{T_n}{T_i}
\right]
\right)
\cong
\mathbf{A}^n_{\mathbf{Z}}
$$
in an obvious way.
\end{proof}

\begin{lemma}
\label{lemma-projective-space-separated}
Let $S$ be a scheme.
The structure morphism $\mathbf{P}^n_S \to S$ is
\begin{enumerate}
\item separated,
\item quasi-compact,
\item satisfies the existence and uniqueness parts of the valuative criterion,
and
\item universally closed.
\end{enumerate}
\end{lemma}

\begin{proof}
All these properties are stable under base change (this is clear for the
last two and for the other two see
Schemes, Lemmas
\ref{schemes-lemma-separated-permanence} and
\ref{schemes-lemma-quasi-compact-preserved-base-change}).
Hence it suffices to prove them for the morphism
$\mathbf{P}^n_{\mathbf{Z}} \to \Spec(\mathbf{Z})$.
Separatedness is Lemma \ref{lemma-proj-separated}. Quasi-compactness follows
from Lemma \ref{lemma-standard-covering-projective-space}.
Existence and uniqueness of the valuative criterion follow from
Lemma \ref{lemma-proj-valuative-criterion}.
Universally closed follows from the above and
Schemes, Proposition \ref{schemes-proposition-characterize-universally-closed}.
\end{proof}

\begin{remark}
\label{remark-missing-finite-type}
What's missing in the list of properties above? Well to be sure the property
of being of finite type. The reason we do not list this here is that we have
not yet defined the notion of finite type at this point. (Another property
which is missing is ``smoothness''. And I'm sure there are many more you can
think of.)
\end{remark}

\begin{lemma}[Segre embedding]
\label{lemma-segre-embedding}
Let $S$ be a scheme. There exists a closed immersion
$$
\mathbf{P}^n_S \times_S \mathbf{P}^m_S
\longrightarrow
\mathbf{P}^{nm + n + m}_S
$$
called the {\it Segre embedding}.
\end{lemma}

\begin{proof}
It suffices to prove this when $S = \Spec(\mathbf{Z})$.
Hence we will drop the index $S$ and work in the absolute setting.
Write $\mathbf{P}^n = \text{Proj}(\mathbf{Z}[X_0, \ldots, X_n])$,
$\mathbf{P}^m = \text{Proj}(\mathbf{Z}[Y_0, \ldots, Y_m])$,
and
$\mathbf{P}^{nm + n + m} =
\text{Proj}(\mathbf{Z}[Z_0, \ldots, Z_{nm + n + m}])$.
In order to map into $\mathbf{P}^{nm + n + m}$ we have to
write down an invertible sheaf $\mathcal{L}$ on the left hand
side and $(n + 1)(m + 1)$ sections $s_i$ which generate it.
See Lemma \ref{lemma-projective-space}.
The invertible sheaf we take is
$$
\mathcal{L} =
\text{pr}_1^*\mathcal{O}_{\mathbf{P}^n}(1)
\otimes
\text{pr}_2^*\mathcal{O}_{\mathbf{P}^m}(1)
$$
The sections we take are
$$
s_0 = X_0Y_0, \ s_1 = X_1Y_0, \ldots, \ s_n = X_nY_0,
\ s_{n + 1} = X_0Y_1, \ldots, \ s_{nm + n + m} = X_nY_m.
$$
These generate $\mathcal{L}$ since the sections $X_i$ generate
$\mathcal{O}_{\mathbf{P}^n}(1)$ and the sections $Y_j$ generate
$\mathcal{O}_{\mathbf{P}^m}(1)$. The induced morphism
$\varphi$ has the property that
$$
\varphi^{-1}(D_{+}(Z_{i + (n + 1)j})) = D_{+}(X_i) \times D_{+}(Y_j).
$$
Hence it is an affine morphism. The corresponding ring map in case
$(i, j) = (0, 0)$ is the map
$$
\mathbf{Z}[Z_1/Z_0, \ldots, Z_{nm + n + m}/Z_0]
\longrightarrow
\mathbf{Z}[X_1/X_0, \ldots, X_n/X_0, Y_1/Y_0, \ldots, Y_n/Y_0]
$$
which maps $Z_i/Z_0$ to the element $X_i/X_0$ for $i \leq n$ and
the element $Z_{(n + 1)j}/Z_0$ to the element $Y_j/Y_0$. Hence it
is surjective. A similar argument works for the other affine
open subsets. Hence the morphism $\varphi$ is a closed immersion
(see Schemes, Lemma \ref{schemes-lemma-closed-local-target} and
Example \ref{schemes-example-closed-immersion-affines}.)
\end{proof}

\noindent
The following two lemmas are special cases of more general results later, but
perhaps it makes sense to prove these directly here now.

\begin{lemma}
\label{lemma-closed-in-projective-space}
Let $R$ be a ring. Let $Z \subset \mathbf{P}^n_R$ be a closed subscheme.
Let
$$
I_d = \Ker\left(
R[T_0, \ldots, T_n]_d
\longrightarrow
\Gamma(Z, \mathcal{O}_{\mathbf{P}^n_R}(d)|_Z)\right)
$$
Then $I = \bigoplus I_d \subset R[T_0, \ldots, T_n]$ is
a graded ideal and $Z = \text{Proj}(R[T_0, \ldots, T_n]/I)$.
\end{lemma}

\begin{proof}
It is clear that $I$ is a graded ideal.
Set $Z' = \text{Proj}(R[T_0, \ldots, T_n]/I)$.
By Lemma \ref{lemma-surjective-graded-rings-generated-degree-1-map-proj}
we see that $Z'$ is a closed subscheme of $\mathbf{P}^n_R$.
To see the equality $Z = Z'$
it suffices to check on an standard affine open
$D_{+}(T_i)$. By renumbering the homogeneous coordinates we
may assume $i = 0$. Say $Z \cap D_{+}(T_0)$, resp.\ $Z' \cap D_{+}(T_0)$
is cut out by the ideal $J$, resp.\ $J'$ of $R[T_1/T_0, \ldots, T_n/T_0]$.
Then $J'$ is the ideal generated by the elements $F/T_0^{\deg(F)}$ where
$F \in I$ is homogeneous.
Suppose the degree of $F \in I$ is $d$. Since $F$ vanishes as a section
of $\mathcal{O}_{\mathbf{P}^n_R}(d)$ restricted to $Z$ we see that
$F/T_0^d$ is an element of $J$. Thus $J' \subset J$.

\medskip\noindent
Conversely, suppose that $f \in J$. If $f$ has total degree
$d$ in $T_1/T_0, \ldots, T_n/T_0$, then we can write
$f = F/T_0^d$ for some $F \in R[T_0, \ldots, T_n]_d$.
Pick $i \in \{1, \ldots, n\}$. Then $Z \cap D_{+}(T_i)$ is
cut out by some ideal $J_i \subset R[T_0/T_i, \ldots, T_n/T_i]$.
Moreover,
$$
J \cdot
R\left[
\frac{T_1}{T_0}, \ldots, \frac{T_n}{T_0},
\frac{T_0}{T_i}, \ldots, \frac{T_n}{T_i}
\right]
=
J_i \cdot
R\left[
\frac{T_1}{T_0}, \ldots, \frac{T_n}{T_0},
\frac{T_0}{T_i}, \ldots, \frac{T_n}{T_i}
\right]
$$
The left hand side is the localization of $J$ with respect to the
element $T_i/T_0$ and the right hand side is the localization of $J_i$
with respect to the element $T_0/T_i$. It follows that
$T_0^{d_i}F/T_i^{d + d_i}$ is an element of $J_i$ for some $d_i$
sufficiently large. This proves that $T_0^{\max(d_i)}F$ is an
element of $I$, because its restriction to each standard affine
open $D_{+}(T_i)$ vanishes on the closed subscheme
$Z \cap D_{+}(T_i)$. Hence $f \in J'$ and we conclude $J \subset J'$
as desired.
\end{proof}

\noindent
The following lemma is a special case of the more general
Properties, Lemmas \ref{properties-lemma-ample-quasi-coherent} or
\ref{properties-lemma-proj-quasi-coherent}.

\begin{lemma}
\label{lemma-quasi-coherent-projective-space}
Let $R$ be a ring.
Let $\mathcal{F}$ be a quasi-coherent sheaf on $\mathbf{P}^n_R$.
For $d \geq 0$ set
$$
M_d
=
\Gamma(\mathbf{P}^n_R,
\mathcal{F} \otimes_{\mathcal{O}_{\mathbf{P}^n_R}}
\mathcal{O}_{\mathbf{P}^n_R}(d))
=
\Gamma(\mathbf{P}^n_R, \mathcal{F}(d))
$$
Then $M = \bigoplus_{d \geq 0} M_d$ is a graded $R[T_0, \ldots, R_n]$-module
and there is a canonical isomorphism $\mathcal{F} = \widetilde{M}$.
\end{lemma}

\begin{proof}
The multiplication maps
$$
R[T_0, \ldots, R_n]_e \times M_d \longrightarrow M_{d + e}
$$
come from the natural isomorphisms
$$
\mathcal{O}_{\mathbf{P}^n_R}(e)
\otimes_{\mathcal{O}_{\mathbf{P}^n_R}}
\mathcal{F}(d)
\longrightarrow
\mathcal{F}(e + d)
$$
see Equation (\ref{equation-global-sections-module}). Let us construct the
map $c : \widetilde{M} \to \mathcal{F}$. On each of the standard affines
$U_i = D_{+}(T_i)$ we see that $\Gamma(U_i, \widetilde{M}) = (M[1/T_i])_0$
where the subscript ${}_0$ means degree $0$ part. An element of this
can be written as $m/T_i^d$ with $m \in M_d$. Since $T_i$ is a generator
of $\mathcal{O}(1)$ over $U_i$ we can always write
$m|_{U_i} = m_i \otimes T_i^d$ where $m_i \in \Gamma(U_i, \mathcal{F})$
is a unique section. Thus a natural guess is $c(m/T_i^d) = m_i$.
A small argument, which is omitted here, shows that this gives a
well defined map $c : \widetilde{M} \to \mathcal{F}$ if we can
show that
$$
(T_i/T_j)^d m_i|_{U_i \cap U_j} = m_j|_{U_i \cap U_j}
$$
in $M[1/T_iT_j]$.
But this is clear since on the overlap the generators $T_i$ and
$T_j$ of $\mathcal{O}(1)$ differ by the invertible function $T_i/T_j$.

\medskip\noindent
Injectivity of $c$. We may check for injectivity over the affine opens
$U_i$. Let $i \in \{0, \ldots, n\}$
and let $s$ be an element $s = m/T_i^d \in \Gamma(U_i, \widetilde{M})$
such that $c(m/T_i^d) = 0$. By the description of $c$ above this means that
$m_i = 0$, hence $m|_{U_i} = 0$. Hence $T_i^em = 0$ in $M$ for some
$e$. Hence $s = m/T_i^d = T_i^e/T_i^{e + d} = 0$ as desired.

\medskip\noindent
Surjectivity of $c$. We may check for surjectivity over the affine opens
$U_i$. By renumbering it suffices to check it over $U_0$.
Let $s \in \mathcal{F}(U_0)$.
Let us write $\mathcal{F}|_{U_i} = \widetilde{N_i}$ for some
$R[T_0/T_i, \ldots, T_0/T_i]$-module $N_i$, which is possible because
$\mathcal{F}$ is quasi-coherent. So $s$ corresponds to an element
$x \in N_0$. Then we have that
$$
(N_i)_{T_j/T_i} \cong (N_j)_{T_i/T_j}
$$
(where the subscripts mean ``principal localization at'')
as modules over the ring
$$
R\left[
\frac{T_0}{T_i}, \ldots, \frac{T_n}{T_i},
\frac{T_0}{T_j}, \ldots, \frac{T_n}{T_j}
\right].
$$
This means that for some large integer $d$ there exist elements
$s_i \in N_i$, $i = 1, \ldots, n$ such that
$$
s = (T_i/T_0)^d s_i
$$
on $U_0 \cap U_i$. Next, we look at the difference
$$
t_{ij} = s_i - (T_j/T_i)^d s_j
$$
on $U_i \cap U_j$, $0 < i < j$. By our choice of $s_i$ we know that
$t_{ij}|_{U_0 \cap U_i \cap U_j} = 0$. Hence there exists a large integer
$e$ such that $(T_0/T_i)^et_{ij} = 0$. Set $s_i' = (T_0/T_i)^es_i$,
and $s_0' = s$. Then we will have
$$
s_a' = (T_b/T_a)^{e + d} s_b'
$$
on $U_a \cap U_b$ for all $a, b$. This is exactly the condition that the
elements $s'_a$ glue to a global section
$m \in \Gamma(\mathbf{P}^n_R, \mathcal{F}(e + d))$.
And moreover $c(m/T_0^{e + d}) = s$ by construction. Hence $c$ is
surjective and we win.
\end{proof}

\begin{lemma}
\label{lemma-globally-generated-omega-twist-1}
Let $X$ be a scheme. Let $\mathcal{L}$ be an invertible sheaf
and let $s_0, \ldots, s_n$ be global sections of $\mathcal{L}$
which generate it. Let $\mathcal{F}$ be the kernel of the induced
map $\mathcal{O}_X^{\oplus n + 1} \to \mathcal{L}$.
Then $\mathcal{F} \otimes \mathcal{L}$ is globally generated.
\end{lemma}

\begin{proof}
In fact the result is true if $X$ is any locally ringed space.
The sheaf $\mathcal{F}$ is a finite locally free $\mathcal{O}_X$-module
of rank $n$. The elements
$$
s_{ij} = (0, \ldots, 0, s_j, 0, \ldots, 0, -s_i, 0, \ldots, 0)
\in \Gamma(X, \mathcal{L}^{\oplus n + 1})
$$
with $s_j$ in the $i$th spot and $-s_i$ in the $j$th spot map to zero
in $\mathcal{L}^{\otimes 2}$. Hence
$s_{ij} \in \Gamma(X, \mathcal{F} \otimes_{\mathcal{O}_X} \mathcal{L})$.
A local computation shows that these sections generate
$\mathcal{F} \otimes \mathcal{L}$.

\medskip\noindent
Alternative proof. Consider the morphism
$\varphi : X \to \mathbf{P}^n_\mathbf{Z}$ associated to
the pair $(\mathcal{L}, (s_0, \ldots, s_n))$. Since the pullback
of $\mathcal{O}(1)$ is $\mathcal{L}$ and since the pullback
of $T_i$ is $s_i$, it suffices to prove the lemma in the
case of $\mathbf{P}^n_\mathbf{Z}$. In this case the sheaf
$\mathcal{F}$ corresponds to the graded $S = \mathbf{Z}[T_0, \ldots, T_n]$
module $M$ which fits into the short exact sequence
$$
0 \to M \to S^{\oplus n + 1} \to S(1) \to 0
$$
where the second map is given by $T_0, \ldots, T_n$. In this
case the statement above translates into the statement that
the elements
$$
T_{ij} = (0, \ldots, 0, T_j, 0, \ldots, 0, -T_i, 0, \ldots, 0)
\in M(1)_0
$$
generate the graded module $M(1)$ over $S$. We omit the details.
\end{proof}








\section{Invertible sheaves and morphisms into Proj}
\label{section-invertible-proj}

\noindent
Let $T$ be a scheme and let $\mathcal{L}$ be an invertible sheaf
on $T$. For a section $s \in \Gamma(T, \mathcal{L})$ we denote
$T_s$ the open subset of points where $s$ does not vanish. See
Modules, Lemma \ref{modules-lemma-s-open}. We can view the following
lemma as a slight generalization of Lemma \ref{lemma-apply}.
It also is a generalization of Lemma \ref{lemma-morphism-proj}.

\begin{lemma}
\label{lemma-invertible-map-into-proj}
Let $A$ be a graded ring.
Set $X = \text{Proj}(A)$.
Let $T$ be a scheme.
Let $\mathcal{L}$ be an invertible $\mathcal{O}_T$-module.
Let $\psi : A \to \Gamma_*(T, \mathcal{L})$ be a homomorphism
of graded rings. Set
$$
U(\psi) = \bigcup\nolimits_{f \in A_{+}\text{ homogeneous}} T_{\psi(f)}
$$
The morphism $\psi$ induces a canonical morphism of schemes
$$
r_{\mathcal{L}, \psi} :
U(\psi) \longrightarrow X
$$
together with a map of $\mathbf{Z}$-graded $\mathcal{O}_T$-algebras
$$
\theta :
r_{\mathcal{L}, \psi}^*\left(
\bigoplus\nolimits_{d \in \mathbf{Z}} \mathcal{O}_X(d)
\right)
\longrightarrow
\bigoplus\nolimits_{d \in \mathbf{Z}} \mathcal{L}^{\otimes d}|_{U(\psi)}.
$$
The triple $(U(\psi), r_{\mathcal{L}, \psi}, \theta)$ is
characterized by the following properties:
\begin{enumerate}
\item For $f \in A_{+}$ homogeneous we have
$r_{\mathcal{L}, \psi}^{-1}(D_{+}(f)) = T_{\psi(f)}$.
\item For every $d \geq 0$ the diagram
$$
\xymatrix{
A_d \ar[d]_{(\ref{equation-global-sections})} \ar[r]_{\psi} &
\Gamma(T, \mathcal{L}^{\otimes d}) \ar[d]^{restrict} \\
\Gamma(X, \mathcal{O}_X(d)) \ar[r]^{\theta} &
\Gamma(U(\psi), \mathcal{L}^{\otimes d})
}
$$
is commutative.
\end{enumerate}
Moreover, for any $d \geq 1$ and any open subscheme $V \subset T$
such that the sections in $\psi(A_d)$ generate $\mathcal{L}^{\otimes d}|_V$
the morphism $r_{\mathcal{L}, \psi}|_V$ agrees with the morphism
$\varphi : V \to \text{Proj}(A)$ and the map $\theta|_V$ agrees with the map
$\alpha : \varphi^*\mathcal{O}_X(d) \to \mathcal{L}^{\otimes d}|_V$
where $(\varphi, \alpha)$ is the pair
of Lemma \ref{lemma-converse-construction}
associated to
$\psi|_{A^{(d)}} : A^{(d)} \to \Gamma_*(V, \mathcal{L}^{\otimes d})$.
\end{lemma}

\begin{proof}
Suppose that we have two triples $(U, r : U \to X, \theta)$
and $(U', r' : U' \to X, \theta')$ satisfying (1) and (2).
Property (1) implies that $U = U' = U(\psi)$ and that
$r = r'$ as maps of underlying topological
spaces, since the opens $D_{+}(f)$ form a basis for the topology
on $X$, and since $X$ is a sober topological space (see
Algebra, Section \ref{algebra-section-proj}
and
Schemes, Lemma \ref{schemes-lemma-scheme-sober}).
Let $f \in A_{+}$ be homogeneous. Note that
$\Gamma(D_{+}(f), \bigoplus_{n \in \mathbf{Z}} \mathcal{O}_X(n)) = A_f$
as a $\mathbf{Z}$-graded algebra. Consider the two
$\mathbf{Z}$-graded ring maps
$$
\theta, \theta' :
A_f
\longrightarrow
\Gamma(T_{\psi(f)}, \bigoplus \mathcal{L}^{\otimes n}).
$$
We know that multiplication by $f$ (resp.\ $\psi(f)$)
is an isomorphism on the left (resp.\ right) hand side.
We also know that $\theta(x/1) = \theta'(x/1) = \psi(x)|_{T_{\psi(f)}}$
by (2) for all $x \in A$. Hence we deduce easily that $\theta = \theta'$
as desired. Considering the degree $0$ parts we deduce that
$r^\sharp = (r')^\sharp$, i.e., that $r = r'$ as morphisms of schemes.
This proves the uniqueness.

\medskip\noindent
Now we come to existence. By the uniqueness just proved, it is enough to
construct the pair $(r, \theta)$ locally on $T$. Hence we may assume
that $T = \Spec(R)$ is affine, that $\mathcal{L} = \mathcal{O}_T$
and that for some $f \in A_{+}$ homogeneous we have
$\psi(f)$ generates $\mathcal{O}_T = \mathcal{O}_T^{\otimes \deg(f)}$.
In other words, $\psi(f) = u \in R^*$ is a unit. In this case the map
$\psi$ is a graded ring map
$$
A \longrightarrow R[x] = \Gamma_*(T, \mathcal{O}_T)
$$
which maps $f$ to $ux^{\deg(f)}$. Clearly this extends (uniquely) to
a $\mathbf{Z}$-graded ring map $\theta : A_f \to R[x, x^{-1}]$ by
mapping $1/f$ to $u^{-1}x^{-\deg(f)}$. This map in degree zero gives
the ring map $A_{(f)} \to R$ which gives the morphism
$r : T = \Spec(R) \to \Spec(A_{(f)}) = D_{+}(f) \subset X$.
Hence we have constructed $(r, \theta)$ in this special case.

\medskip\noindent
Let us show the last statement of the lemma.
According to Lemma \ref{lemma-converse-construction}
the morphism constructed there is the unique one such that
the displayed diagram in its statement commutes.
The commutativity of the diagram in the lemma implies the
commutativity when restricted to $V$ and $A^{(d)}$.
Whence the result.
\end{proof}

\begin{remark}
\label{remark-not-in-invertible-locus}
Assumptions as in Lemma \ref{lemma-invertible-map-into-proj} above.
The image of the morphism $r_{\mathcal{L}, \psi}$ need not be
contained in the locus where the sheaf $\mathcal{O}_X(1)$
is invertible.
Here is an example.
Let $k$ be a field.
Let $S = k[A, B, C]$ graded by $\deg(A) = 1$, $\deg(B) = 2$, $\deg(C) = 3$.
Set $X = \text{Proj}(S)$.
Let $T = \mathbf{P}^2_k = \text{Proj}(k[X_0, X_1, X_2])$.
Recall that $\mathcal{L} = \mathcal{O}_T(1)$ is invertible
and that $\mathcal{O}_T(n) = \mathcal{L}^{\otimes n}$.
Consider the composition $\psi$ of the maps
$$
S \to k[X_0, X_1, X_2] \to \Gamma_*(T, \mathcal{L}).
$$
Here the first map is $A \mapsto X_0$, $B \mapsto X_1^2$,
$C \mapsto X_2^3$ and the second map is (\ref{equation-global-sections}).
By the lemma this corresponds to a morphism
$r_{\mathcal{L}, \psi} : T \to X = \text{Proj}(S)$
which is easily seen to be surjective. On the other hand, in
Remark \ref{remark-not-isomorphism} we showed that the sheaf
$\mathcal{O}_X(1)$ is not invertible at all points of $X$.
\end{remark}










\section{Relative Proj via glueing}
\label{section-relative-proj-via-glueing}

\begin{situation}
\label{situation-relative-proj}
Here $S$ is a scheme, and $\mathcal{A}$
is a quasi-coherent graded $\mathcal{O}_S$-algebra.
\end{situation}

\noindent
In this section we outline how to construct a morphism
of schemes
$$
\underline{\text{Proj}}_S(\mathcal{A}) \longrightarrow S
$$
by glueing the homogeneous spectra $\text{Proj}(\Gamma(U, \mathcal{A}))$
where $U$ ranges over the affine opens of $S$. We first show that the
homogeneous spectra of the values of $\mathcal{A}$ over affines form a
suitable collection of schemes, as in Lemma \ref{lemma-relative-glueing}.

\begin{lemma}
\label{lemma-proj-inclusion}
In Situation \ref{situation-relative-proj}.
Suppose $U \subset U' \subset S$ are affine opens.
Let $A = \mathcal{A}(U)$ and $A' = \mathcal{A}(U')$.
The map of graded rings $A' \to A$ induces a morphism
$r : \text{Proj}(A) \to \text{Proj}(A')$, and the diagram
$$
\xymatrix{
\text{Proj}(A) \ar[r] \ar[d] &
\text{Proj}(A') \ar[d] \\
U \ar[r] &
U'
}
$$
is cartesian. Moreover there are canonical isomorphisms
$\theta : r^*\mathcal{O}_{\text{Proj}(A')}(n) \to
\mathcal{O}_{\text{Proj}(A)}(n)$ compatible with multiplication maps.
\end{lemma}

\begin{proof}
Let $R = \mathcal{O}_S(U)$ and $R' = \mathcal{O}_S(U')$.
Note that the map $R \otimes_{R'} A' \to A$ is an isomorphism as
$\mathcal{A}$ is quasi-coherent
(see Schemes, Lemma \ref{schemes-lemma-widetilde-pullback} for example).
Hence the lemma follows from
Lemma \ref{lemma-base-change-map-proj}.
\end{proof}

\noindent
In particular the morphism $\text{Proj}(A) \to \text{Proj}(A')$
of the lemma is an open immersion.

\begin{lemma}
\label{lemma-transitive-proj}
In Situation \ref{situation-relative-proj}.
Suppose $U \subset U' \subset U'' \subset S$ are affine opens.
Let $A = \mathcal{A}(U)$, $A' = \mathcal{A}(U')$ and $A'' = \mathcal{A}(U'')$.
The composition of the morphisms
$r : \text{Proj}(A) \to \text{Proj}(A')$, and
$r' : \text{Proj}(A') \to \text{Proj}(A'')$ of
Lemma \ref{lemma-proj-inclusion} gives the
morphism $r'' : \text{Proj}(A) \to \text{Proj}(A'')$
of Lemma \ref{lemma-proj-inclusion}. A similar statement
holds for the isomorphisms $\theta$.
\end{lemma}

\begin{proof}
This follows from Lemma \ref{lemma-morphism-proj-transitive} since
the map $A'' \to A$ is the composition of $A'' \to A'$ and
$A' \to A$.
\end{proof}

\begin{lemma}
\label{lemma-glue-relative-proj}
In Situation \ref{situation-relative-proj}.
There exists a morphism of schemes
$$
\pi : \underline{\text{Proj}}_S(\mathcal{A}) \longrightarrow S
$$
with the following properties:
\begin{enumerate}
\item for every affine open $U \subset S$ there exists an isomorphism
$i_U : \pi^{-1}(U) \to \text{Proj}(A)$ with $A = \mathcal{A}(U)$, and
\item for $U \subset U' \subset S$ affine open the composition
$$
\xymatrix{
\text{Proj}(A) \ar[r]^{i_U^{-1}} &
\pi^{-1}(U) \ar[rr]^{inclusion} & &
\pi^{-1}(U') \ar[r]^{i_{U'}} &
\text{Proj}(A')
}
$$
with $A = \mathcal{A}(U)$, $A' = \mathcal{A}(U')$
is the open immersion of Lemma \ref{lemma-proj-inclusion} above.
\end{enumerate}
\end{lemma}

\begin{proof}
Follows immediately from
Lemmas \ref{lemma-relative-glueing},
\ref{lemma-proj-inclusion}, and
\ref{lemma-transitive-proj}.
\end{proof}

\begin{lemma}
\label{lemma-glue-relative-proj-twists}
In Situation \ref{situation-relative-proj}.
The morphism $\pi : \underline{\text{Proj}}_S(\mathcal{A}) \to S$
of Lemma \ref{lemma-glue-relative-proj} comes with the following
additional structure.
There exists a quasi-coherent $\mathbf{Z}$-graded sheaf
of $\mathcal{O}_{\underline{\text{Proj}}_S(\mathcal{A})}$-algebras
$\bigoplus\nolimits_{n \in \mathbf{Z}}
\mathcal{O}_{\underline{\text{Proj}}_S(\mathcal{A})}(n)$,
and a morphism of graded $\mathcal{O}_S$-algebras
$$
\psi :
\mathcal{A}
\longrightarrow
\bigoplus\nolimits_{n \geq 0}
\pi_*\left(\mathcal{O}_{\underline{\text{Proj}}_S(\mathcal{A})}(n)\right)
$$
uniquely determined by the following property:
For every affine open $U \subset S$ with $A = \mathcal{A}(U)$
there is an isomorphism
$$
\theta_U :
i_U^*\left(
\bigoplus\nolimits_{n \in \mathbf{Z}} \mathcal{O}_{\text{Proj}(A)}(n)
\right)
\longrightarrow
\left(
\bigoplus\nolimits_{n \in \mathbf{Z}}
\mathcal{O}_{\underline{\text{Proj}}_S(\mathcal{A})}(n)
\right)|_{\pi^{-1}(U)}
$$
of $\mathbf{Z}$-graded $\mathcal{O}_{\pi^{-1}(U)}$-algebras
such that
$$
\xymatrix{
A_n
\ar[rr]_\psi
\ar[dr]_-{(\ref{equation-global-sections})}
& &
\Gamma(\pi^{-1}(U),
\mathcal{O}_{\underline{\text{Proj}}_S(\mathcal{A})}(n)) \\
&
\Gamma(\text{Proj}(A),
\mathcal{O}_{\text{Proj}(A)}(n))
\ar[ru]_-{\theta_U}
&
}
$$
is commutative.
\end{lemma}

\begin{proof}
We are going to use Lemma \ref{lemma-relative-glueing-sheaves}
to glue the sheaves of $\mathbf{Z}$-graded algebras
$\bigoplus_{n \in \mathbf{Z}} \mathcal{O}_{\text{Proj}(A)}(n)$
for $A = \mathcal{A}(U)$, $U \subset S$ affine open
over the scheme $\underline{\text{Proj}}_S(\mathcal{A})$.
We have constructed the data necessary for this in
Lemma \ref{lemma-proj-inclusion} and we have checked condition (d) of
Lemma \ref{lemma-relative-glueing-sheaves} in
Lemma \ref{lemma-transitive-proj}. Hence we get the
sheaf of $\mathbf{Z}$-graded
$\mathcal{O}_{\underline{\text{Proj}}_S(\mathcal{A})}$-algebras
$\bigoplus_{n \in \mathbf{Z}}
\mathcal{O}_{\underline{\text{Proj}}_S(\mathcal{A})}(n)$
together with the isomorphisms $\theta_U$ for all
$U \subset S$ affine open and all $n \in \mathbf{Z}$.
For every affine open $U \subset S$ with $A = \mathcal{A}(U)$ we have a map
$A \to \Gamma(\text{Proj}(A),
\bigoplus_{n \geq 0} \mathcal{O}_{\text{Proj}(A)}(n))$.
Hence the map $\psi$ exists by functoriality
of relative glueing, see Remark \ref{remark-relative-glueing-functorial}.
The diagram of the lemma commutes by construction.
This characterizes the sheaf of $\mathbf{Z}$-graded
$\mathcal{O}_{\underline{\text{Proj}}_S(\mathcal{A})}$-algebras
$\bigoplus \mathcal{O}_{\underline{\text{Proj}}_S(\mathcal{A})}(n)$
because the proof of Lemma \ref{lemma-morphism-proj} shows that
having these diagrams commute uniquely determines the maps $\theta_U$.
Some details omitted.
\end{proof}















\section{Relative Proj as a functor}
\label{section-relative-proj}

\noindent
We place ourselves in Situation \ref{situation-relative-proj}.
So $S$ is a scheme and $\mathcal{A} = \bigoplus_{d \geq 0} \mathcal{A}_d$
is a quasi-coherent graded $\mathcal{O}_S$-algebra.
In this section we relativize the construction of
$\text{Proj}$ by constructing a functor which the relative
homogeneous spectrum will represent.
As a result we will construct a morphism of schemes
$$
\underline{\text{Proj}}_S(\mathcal{A}) \longrightarrow S
$$
which above affine opens of $S$ will look like the homogeneous spectrum
of a graded ring. The discussion will be modeled after our
discussion of the relative spectrum in Section \ref{section-spec}.
The easier method using glueing schemes of the form
$\text{Proj}(A)$, $A = \Gamma(U, \mathcal{A})$, $U \subset S$
affine open, is explained in Section \ref{section-relative-proj-via-glueing},
and the result in this section will be shown to be isomorphic to that one.

\medskip\noindent
Fix for the moment an integer $d \geq 1$.
We denote $\mathcal{A}^{(d)} = \bigoplus_{n \geq 0} \mathcal{A}_{nd}$
similarly to the notation in Algebra, Section \ref{algebra-section-graded}.
Let $T$ be a scheme.
Let us consider {\it quadruples $(d, f : T \to S, \mathcal{L}, \psi)$
over $T$} where
\begin{enumerate}
\item $d$ is the integer we fixed above,
\item $f : T \to S$ is a morphism of schemes,
\item $\mathcal{L}$ is an invertible $\mathcal{O}_T$-module, and
\item
$\psi : f^*\mathcal{A}^{(d)} \to \bigoplus_{n \geq 0}\mathcal{L}^{\otimes n}$
is a homomorphism of graded $\mathcal{O}_T$-algebras
such that $f^*\mathcal{A}_d \to \mathcal{L}$ is surjective.
\end{enumerate}
Given a morphism $h : T' \to T$ and a quadruple
$(d, f, \mathcal{L}, \psi)$ over $T$ we can pull it back to the
quadruple $(d, f \circ h, h^*\mathcal{L}, h^*\psi)$ over $T'$.
Given two quadruples $(d, f, \mathcal{L}, \psi)$ and
$(d, f', \mathcal{L}', \psi')$ over $T$ with the same integer $d$
we say they are {\it strictly equivalent} if $f = f'$ and there exists
an isomorphism $\beta : \mathcal{L} \to \mathcal{L}'$
such that $\beta \circ \psi = \psi'$ as graded $\mathcal{O}_T$-algebra maps
$f^*\mathcal{A}^{(d)} \to \bigoplus_{n \geq 0} (\mathcal{L}')^{\otimes n}$.

\medskip\noindent
For each integer $d \geq 1$ we define
\begin{eqnarray*}
F_d : \Sch^{opp} & \longrightarrow & \textit{Sets}, \\
T & \longmapsto &
\{\text{strict equivalence classes of }
(d, f : T \to S, \mathcal{L}, \psi)
\text{ as above}\}
\end{eqnarray*}
with pullbacks as defined above.

\begin{lemma}
\label{lemma-proj-base-change}
In Situation \ref{situation-relative-proj}. Let $d \geq 1$.
Let $F_d$ be the functor
associated to $(S, \mathcal{A})$ above.
Let $g : S' \to S$ be a morphism of schemes.
Set $\mathcal{A}' = g^*\mathcal{A}$. Let $F_d'$ be the
functor associated to $(S', \mathcal{A}')$ above.
Then there is a canonical isomorphism
$$
F'_d \cong h_{S'} \times_{h_S} F_d
$$
of functors.
\end{lemma}

\begin{proof}
A quadruple
$(d, f' : T \to S', \mathcal{L}',
\psi' : (f')^*(\mathcal{A}')^{(d)} \to
\bigoplus_{n \geq 0} (\mathcal{L}')^{\otimes n})$
is the same as a quadruple
$(d, f, \mathcal{L},
\psi : f^*\mathcal{A}^{(d)} \to
\bigoplus_{n \geq 0} \mathcal{L}^{\otimes n})$
together with a factorization of $f$ as $f = g \circ f'$. Namely,
the correspondence is $f = g \circ f'$, $\mathcal{L} = \mathcal{L}'$
and $\psi = \psi'$ via the identifications
$(f')^*(\mathcal{A}')^{(d)} = (f')^*g^*(\mathcal{A}^{(d)}) =
f^*\mathcal{A}^{(d)}$. Hence the lemma.
\end{proof}

\begin{lemma}
\label{lemma-relative-proj-affine}
In Situation \ref{situation-relative-proj}. Let $F_d$ be the functor
associated to $(d, S, \mathcal{A})$ above.
If $S$ is affine, then $F_d$ is representable by the open subscheme
$U_d$ (\ref{equation-Ud})
of the scheme $\text{Proj}(\Gamma(S, \mathcal{A}))$.
\end{lemma}

\begin{proof}
Write $S = \Spec(R)$ and $A = \Gamma(S, \mathcal{A})$.
Then $A$ is a graded $R$-algebra and $\mathcal{A} = \widetilde A$.
To prove the lemma we have to identify the functor $F_d$
with the functor $F_d^{triples}$ of triples defined in Section
\ref{section-morphisms-proj}.

\medskip\noindent
Let $(d, f : T \to S, \mathcal{L}, \psi)$ be a quadruple.
We may think of $\psi$ as a $\mathcal{O}_S$-module map
$\mathcal{A}^{(d)} \to \bigoplus_{n \geq 0} f_*\mathcal{L}^{\otimes n}$.
Since $\mathcal{A}^{(d)}$ is quasi-coherent this is the same
thing as an $R$-linear homomorphism of graded rings
$A^{(d)} \to \Gamma(S, \bigoplus_{n \geq 0} f_*\mathcal{L}^{\otimes n})$.
Clearly, $\Gamma(S, \bigoplus_{n \geq 0} f_*\mathcal{L}^{\otimes n}) =
\Gamma_*(T, \mathcal{L})$. Thus we may associate to
the quadruple the triple $(d, \mathcal{L}, \psi)$.

\medskip\noindent
Conversely, let $(d, \mathcal{L}, \psi)$ be a triple.
The composition $R \to A_0 \to \Gamma(T, \mathcal{O}_T)$
determines a morphism $f : T \to S = \Spec(R)$, see
Schemes, Lemma \ref{schemes-lemma-morphism-into-affine}.
With this choice of $f$ the map
$A^{(d)} \to \Gamma(S, \bigoplus_{n \geq 0} f_*\mathcal{L}^{\otimes n})$
is $R$-linear, and hence corresponds to a $\psi$ which we
can use for a quadruple $(d, f : T \to S, \mathcal{L}, \psi)$.
We omit the verification that this establishes an isomorphism
of functors $F_d = F_d^{triples}$.
\end{proof}

\begin{lemma}
\label{lemma-relative-proj-d}
In Situation \ref{situation-relative-proj}.
The functor $F_d$ is representable by a scheme.
\end{lemma}

\begin{proof}
We are going to use Schemes, Lemma \ref{schemes-lemma-glue-functors}.

\medskip\noindent
First we check that $F_d$ satisfies the sheaf property for the
Zariski topology. Namely, suppose that $T$ is a scheme,
that $T = \bigcup_{i \in I} U_i$ is an open covering,
and that $(d, f_i, \mathcal{L}_i, \psi_i) \in F_d(U_i)$ such that
$(d, f_i, \mathcal{L}_i, \psi_i)|_{U_i \cap U_j}$ and
$(d, f_j, \mathcal{L}_j, \psi_j)|_{U_i \cap U_j}$ are strictly
equivalent. This implies that the morphisms $f_i : U_i \to S$
glue to a morphism of schemes $f : T \to S$ such that
$f|_{I_i} = f_i$, see Schemes, Section \ref{schemes-section-glueing-schemes}.
Thus $f_i^*\mathcal{A}^{(d)} = f^*\mathcal{A}^{(d)}|_{U_i}$.
It also implies there exist isomorphisms
$\beta_{ij} : \mathcal{L}_i|_{U_i \cap U_j} \to \mathcal{L}_j|_{U_i \cap U_j}$
such that $\beta_{ij} \circ \psi_i = \psi_j$ on $U_i \cap U_j$.
Note that the isomorphisms $\beta_{ij}$ are uniquely determined
by this requirement because the maps $f_i^*\mathcal{A}_d \to \mathcal{L}_i$
are surjective. In particular we see that
$\beta_{jk} \circ \beta_{ij} = \beta_{ik}$ on $U_i \cap U_j \cap U_k$.
Hence by Sheaves,
Section \ref{sheaves-section-glueing-sheaves} the invertible sheaves
$\mathcal{L}_i$ glue to an invertible $\mathcal{O}_T$-module
$\mathcal{L}$ and the morphisms $\psi_i$ glue to
morphism of $\mathcal{O}_T$-algebras
$\psi : f^*\mathcal{A}^{(d)} \to \bigoplus_{n \geq 0} \mathcal{L}^{\otimes n}$.
This proves that $F_d$ satisfies the sheaf condition with respect to
the Zariski topology.

\medskip\noindent
Let $S = \bigcup_{i \in I} U_i$ be an affine open covering.
Let $F_{d, i} \subset F_d$ be the subfunctor consisting of
those pairs $(f : T \to S, \varphi)$ such that
$f(T) \subset U_i$.

\medskip\noindent
We have to show each $F_{d, i}$ is representable.
This is the case because $F_{d, i}$ is identified with
the functor associated to $U_i$ equipped with
the quasi-coherent graded $\mathcal{O}_{U_i}$-algebra
$\mathcal{A}|_{U_i}$ by Lemma \ref{lemma-proj-base-change}.
Thus the result follows from Lemma \ref{lemma-relative-proj-affine}.

\medskip\noindent
Next we show that $F_{d, i} \subset F_d$ is representable by open immersions.
Let $(f : T \to S, \varphi) \in F_d(T)$. Consider $V_i = f^{-1}(U_i)$.
It follows from the definition of $F_{d, i}$ that given $a : T' \to T$
we gave $a^*(f, \varphi) \in F_{d, i}(T')$ if and only if $a(T') \subset V_i$.
This is what we were required to show.

\medskip\noindent
Finally, we have to show that the collection $(F_{d, i})_{i \in I}$
covers $F_d$. Let $(f : T \to S, \varphi) \in F_d(T)$.
Consider $V_i = f^{-1}(U_i)$. Since $S = \bigcup_{i \in I} U_i$
is an open covering of $S$ we see that $T = \bigcup_{i \in I} V_i$
is an open covering of $T$. Moreover $(f, \varphi)|_{V_i} \in F_{d, i}(V_i)$.
This finishes the proof of the lemma.
\end{proof}

\noindent
At this point we can redo the material at the end of
Section \ref{section-morphisms-proj} in the current
relative setting and define a functor
which is representable by
$\underline{\text{Proj}}_S(\mathcal{A})$. To do this we introduce the
notion of equivalence between two
quadruples $(d, f : T \to S, \mathcal{L}, \psi)$ and
$(d', f' : T \to S, \mathcal{L}', \psi')$ with possibly different
values of the integers $d, d'$. Namely, we say these
are {\it equivalent} if $f = f'$, and there exists an
isomorphism $\beta : \mathcal{L}^{\otimes d'} \to (\mathcal{L}')^{\otimes d}$
such that
$\beta \circ \psi|_{f^*\mathcal{A}^{(dd')}} = \psi'|_{f^*\mathcal{A}^{(dd')}}$.
The following lemma implies that this defines an equivalence relation.
(This is not a complete triviality.)

\begin{lemma}
\label{lemma-equivalent-relative}
In Situation \ref{situation-relative-proj}.
Let $T$ be a scheme.
Let $(d, f, \mathcal{L}, \psi)$, $(d', f', \mathcal{L}', \psi')$
be two quadruples over $T$. The following are equivalent:
\begin{enumerate}
\item Let $m = \text{lcm}(d, d')$. Write $m = ad = a'd'$.
We have $f = f'$ and there exists
an isomorphism
$\beta : \mathcal{L}^{\otimes a} \to (\mathcal{L}')^{\otimes a'}$
with the property that $\beta \circ \psi|_{f^*\mathcal{A}^{(m)}}$
and $\psi'|_{f^*\mathcal{A}^{(m)}}$ agree
as graded ring maps
$f^*\mathcal{A}^{(m)} \to \bigoplus_{n \geq 0} (\mathcal{L}')^{\otimes mn}$.
\item The quadruples $(d, f, \mathcal{L}, \psi)$ and
$(d', f', \mathcal{L}', \psi')$ are equivalent.
\item We have $f = f'$ and
for some positive integer $m = ad = a'd'$ there exists an isomorphism
$\beta : \mathcal{L}^{\otimes a} \to (\mathcal{L}')^{\otimes a'}$
with the property that $\beta \circ \psi|_{f^*\mathcal{A}^{(m)}}$
and $\psi'|_{f^*\mathcal{A}^{(m)}}$ agree
as graded ring maps
$f^*\mathcal{A}^{(m)} \to \bigoplus_{n \geq 0} (\mathcal{L}')^{\otimes mn}$.
\end{enumerate}
\end{lemma}

\begin{proof}
Clearly (1) implies (2) and (2) implies (3) by restricting to
more divisible degrees and powers of invertible sheaves.
Assume (3) for some integer $m = ad = a'd'$. Let
$m_0 = \text{lcm}(d, d')$ and write it as $m_0 = a_0d = a'_0d'$.
We are given an isomorphism
$\beta : \mathcal{L}^{\otimes a} \to (\mathcal{L}')^{\otimes a'}$
with the property described in (3). We want to find an isomorphism
$\beta_0 : \mathcal{L}^{\otimes a_0} \to (\mathcal{L}')^{\otimes a'_0}$
having that property as well.
Since by assumption the maps $\psi : f^*\mathcal{A}_d \to \mathcal{L}$
and $\psi' : (f')^*\mathcal{A}_{d'} \to \mathcal{L}'$ are surjective the
same is true for the maps
$\psi : f^*\mathcal{A}_{m_0} \to \mathcal{L}^{\otimes a_0}$
and $\psi' : (f')^*\mathcal{A}_{m_0} \to (\mathcal{L}')^{\otimes a_0}$.
Hence if $\beta_0$ exists it is uniquely determined by the
condition that $\beta_0 \circ \psi = \psi'$. This means that
we may work locally on $T$. Hence we may assume that
$f = f' : T \to S$ maps into an affine open, in other words
we may assume that $S$ is affine. In this case the result follows
from the corresponding result for triples (see Lemma \ref{lemma-equivalent})
and the fact that triples and quadruples correspond in the
affine base case (see proof of Lemma \ref{lemma-relative-proj-affine}).
\end{proof}

\noindent
Suppose $d' = ad$. Consider the transformation of functors $F_d \to F_{d'}$
which assigns to the quadruple $(d, f, \mathcal{L}, \psi)$ over
$T$ the quadruple
$(d', f, \mathcal{L}^{\otimes a}, \psi|_{f^*\mathcal{A}^{(d')}})$.
One of the implications of Lemma \ref{lemma-equivalent-relative} is that the
transformation $F_d \to F_{d'}$ is injective!
For a quasi-compact scheme $T$ we define
$$
F(T) = \bigcup\nolimits_{d \in \mathbf{N}} F_d(T)
$$
with transition maps as explained above. This clearly defines a
contravariant functor on the category of quasi-compact schemes
with values in sets. For a general scheme
$T$ we define
$$
F(T)
=
\lim_{V \subset T\text{ quasi-compact open}} F(V).
$$
In other words, an element $\xi$ of $F(T)$ corresponds to a compatible system
of choices of elements $\xi_V \in F(V)$ where $V$ ranges over the
quasi-compact opens of $T$.
We omit the definition of the pullback map $F(T) \to F(T')$
for a morphism $T' \to T$ of schemes.
Thus we have defined our functor
\begin{equation}
\label{equation-proj}
F : \Sch^{opp} \longrightarrow \textit{Sets}
\end{equation}

\begin{lemma}
\label{lemma-relative-proj}
In Situation \ref{situation-relative-proj}.
The functor $F$ above is representable by a scheme.
\end{lemma}

\begin{proof}
Let $U_d \to S$ be the scheme representing the functor $F_d$
defined above. Let $\mathcal{L}_d$,
$\psi^d : \pi_d^*\mathcal{A}^{(d)} \to
\bigoplus_{n \geq 0} \mathcal{L}_d^{\otimes n}$ be the universal object.
If $d | d'$, then we may consider the quadruple
$(d', \pi_d, \mathcal{L}_d^{\otimes d'/d}, \psi^d|_{\mathcal{A}^{(d')}})$
which determines a canonical morphism $U_d \to U_{d'}$ over $S$.
By construction this morphism corresponds to the transformation
of functors $F_d \to F_{d'}$ defined above.

\medskip\noindent
For every affine open $\Spec(R) = V \subset S$
setting $A = \Gamma(V, \mathcal{A})$ we have a canonical
identification of the base change $U_{d, V}$ with the corresponding open
subscheme of $\text{Proj}(A)$, see Lemma \ref{lemma-relative-proj-affine}.
Moreover, the morphisms $U_{d, V} \to U_{d', V}$ constructed above
correspond to the inclusions of opens in $\text{Proj}(A)$.
Thus we conclude that $U_d \to U_{d'}$ is an open immersion.

\medskip\noindent
This allows us to construct $X$
by glueing the schemes $U_d$ along the open immersions $U_d \to U_{d'}$.
Technically, it is convenient to choose a sequence
$d_1 | d_2 | d_3 | \ldots$ such that every positive integer
divides one of the $d_i$ and to simply take
$X = \bigcup U_{d_i}$ using the open immersions above.
It is then a simple matter to prove that $X$ represents the
functor $F$.
\end{proof}

\begin{lemma}
\label{lemma-glueing-gives-functor-proj}
In Situation \ref{situation-relative-proj}.
The scheme $\pi : \underline{\text{Proj}}_S(\mathcal{A}) \to S$
constructed in Lemma \ref{lemma-glue-relative-proj}
and the scheme representing the functor $F$
are canonically isomorphic as schemes over $S$.
\end{lemma}

\begin{proof}
Let $X$ be the scheme representing the functor $F$.
Note that $X$ is a scheme over $S$ since the functor $F$
comes equipped with a natural transformation $F \to h_S$.
Write $Y = \underline{\text{Proj}}_S(\mathcal{A})$.
We have to show that $X \cong Y$ as $S$-schemes.
We give two arguments.

\medskip\noindent
The first argument uses the construction of $X$ as the union
of the schemes $U_d$ representing $F_d$ in the
proof of Lemma \ref{lemma-relative-proj}.
Over each affine open of $S$ we can identify $X$ with the homogeneous spectrum
of the sections of $\mathcal{A}$ over that open, since this was
true for the opens $U_d$. Moreover, these identifications
are compatible with further restrictions to smaller affine opens.
On the other hand, $Y$ was constructed by glueing these
homogeneous spectra.
Hence we can glue these isomorphisms to an isomorphism
between $X$ and $\underline{\text{Proj}}_S(\mathcal{A})$ as
desired. Details omitted.

\medskip\noindent
Here is the second argument.
Lemma \ref{lemma-glue-relative-proj-twists}
shows that there exists a morphism of graded algebras
$$
\psi : \pi^*\mathcal{A}
\longrightarrow
\bigoplus\nolimits_{n \geq 0} \mathcal{O}_Y(n)
$$
over $Y$ which on sections over affine opens of $S$ agrees with
(\ref{equation-global-sections}). Hence for every $y \in Y$
there exists an open neighbourhood $V \subset Y$ of $y$
and an integer $d \geq 1$ such that for $d | n$ the sheaf
$\mathcal{O}_Y(n)|_V$ is invertible and the multiplication maps
$\mathcal{O}_Y(n)|_V \otimes_{\mathcal{O}_V} \mathcal{O}_Y(m)|_V
\to \mathcal{O}_Y(n + m)|_V$ are isomorphisms. Thus
$\psi$ restricted to the sheaf $\pi^*\mathcal{A}^{(d)}|_V$
gives an element of $F_d(V)$. Since the opens $V$ cover $Y$
we see ``$\psi$'' gives rise to an element of $F(Y)$.
Hence a canonical morphism $Y \to X$ over $S$.
Because this construction is completely canonical to see
that it is an isomorphism we may work locally on $S$.
Hence we reduce to the case $S$ affine where the result is clear.
\end{proof}

\begin{definition}
\label{definition-relative-proj}
Let $S$ be a scheme. Let $\mathcal{A}$ be a quasi-coherent sheaf of
graded $\mathcal{O}_S$-algebras. The
{\it relative homogeneous spectrum of $\mathcal{A}$ over $S$},
or the {\it homogeneous spectrum of $\mathcal{A}$ over $S$}, or the
{\it relative Proj of $\mathcal{A}$ over $S$} is the scheme
constructed in Lemma \ref{lemma-glue-relative-proj} which represents the
functor $F$ (\ref{equation-proj}), see
Lemma \ref{lemma-glueing-gives-functor-proj}.
We denote it $\pi : \underline{\text{Proj}}_S(\mathcal{A}) \to S$.
\end{definition}

\noindent
The relative Proj comes equipped with a quasi-coherent
sheaf of $\mathbf{Z}$-graded algebras
$\bigoplus_{n \in \mathbf{Z}}
\mathcal{O}_{\underline{\text{Proj}}_S(\mathcal{A})}(n)$
(the twists of the structure sheaf) and
a ``universal'' homomorphism of graded algebras
$$
\psi_{univ} :
\mathcal{A}
\longrightarrow
\pi_*\left(
\bigoplus\nolimits_{n \geq 0}
\mathcal{O}_{\underline{\text{Proj}}_S(\mathcal{A})}(n)
\right)
$$
see Lemma \ref{lemma-glue-relative-proj-twists}. We may also think of this
as a homomorphism
$$
\psi_{univ} :
\pi^*\mathcal{A}
\longrightarrow
\bigoplus\nolimits_{n \geq 0}
\mathcal{O}_{\underline{\text{Proj}}_S(\mathcal{A})}(n)
$$
if we like. The following lemma is a formulation of the
universality of this object.

\begin{lemma}
\label{lemma-tie-up-psi}
In Situation \ref{situation-relative-proj}.
Let $(f : T \to S, d, \mathcal{L}, \psi)$
be a quadruple. Let
$r_{d, \mathcal{L}, \psi} : T \to \underline{\text{Proj}}_S(\mathcal{A})$
be the associated $S$-morphism.
There exists an isomorphism
of $\mathbf{Z}$-graded $\mathcal{O}_T$-algebras
$$
\theta :
r_{d, \mathcal{L}, \psi}^*\left(
\bigoplus\nolimits_{n \in \mathbf{Z}}
\mathcal{O}_{\underline{\text{Proj}}_S(\mathcal{A})}(nd)
\right)
\longrightarrow
\bigoplus\nolimits_{n \in \mathbf{Z}} \mathcal{L}^{\otimes n}
$$
such that the following diagram commutes
$$
\xymatrix{
\mathcal{A}^{(d)} \ar[rr]_-{\psi}
 \ar[rd]_-{\psi_{univ}} & &
f_*\left(
\bigoplus\nolimits_{n \in \mathbf{Z}}
\mathcal{L}^{\otimes n}
\right) \\
 &
\pi_*\left(
\bigoplus\nolimits_{n \geq 0}
\mathcal{O}_{\underline{\text{Proj}}_S(\mathcal{A})}(nd)
\right) \ar[ru]_\theta
}
$$
The commutativity of this diagram uniquely determines $\theta$.
\end{lemma}

\begin{proof}
Note that the quadruple $(f : T \to S, d, \mathcal{L}, \psi)$
defines an element of $F_d(T)$. Let
$U_d \subset \underline{\text{Proj}}_S(\mathcal{A})$
be the locus
where the sheaf $\mathcal{O}_{\underline{\text{Proj}}_S(\mathcal{A})}(d)$
is invertible and generated by the image of
$\psi_{univ} : \pi^*\mathcal{A}_d \to
\mathcal{O}_{\underline{\text{Proj}}_S(\mathcal{A})}(d)$.
Recall that $U_d$ represents the functor $F_d$, see the proof
of Lemma \ref{lemma-relative-proj}. Hence the result will follow
if we can show the quadruple
$(U_d \to S, d, \mathcal{O}_{U_d}(d), \psi_{univ}|_{\mathcal{A}^{(d)}})$
is the universal family, i.e., the representing object in $F_d(U_d)$.
We may do this after restricting to an affine open of $S$ because
(a) the formation of the functors $F_d$ commutes with base change
(see Lemma \ref{lemma-proj-base-change}), and (b) the pair
$(\bigoplus_{n \in \mathbf{Z}}
\mathcal{O}_{\underline{\text{Proj}}_S(\mathcal{A})}(n),
\psi_{univ})$
is constructed by glueing over affine opens in $S$
(see Lemma \ref{lemma-glue-relative-proj-twists}).
Hence we may assume that $S$ is affine. In this case the functor
of quadruples $F_d$ and the functor of triples $F_d$ agree
(see proof of Lemma \ref{lemma-relative-proj-affine}) and moreover
Lemma \ref{lemma-proj-functor-strict}
shows that $(d, \mathcal{O}_{U_d}(d), \psi^d)$
is the universal triple over $U_d$.
Going backwards through the identifications in the proof of
Lemma \ref{lemma-relative-proj-affine} shows that
$(U_d \to S, d, \mathcal{O}_{U_d}(d), \psi_{univ}|_{\mathcal{A}^{(d)}})$
is the universal quadruple as desired.
\end{proof}

\begin{lemma}
\label{lemma-relative-proj-separated}
Let $S$ be a scheme and $\mathcal{A}$ be a quasi-coherent sheaf
of graded $\mathcal{O}_S$-algebras. The morphism
$\pi : \underline{\text{Proj}}_S(\mathcal{A}) \to S$
is separated.
\end{lemma}

\begin{proof}
To prove a morphism is separated we may work locally on the base,
see Schemes, Section \ref{schemes-section-separation-axioms}.
By construction $\underline{\text{Proj}}_S(\mathcal{A})$ is
over any affine $U \subset S$ isomorphic to
$\text{Proj}(A)$ with $A = \mathcal{A}(U)$. By
Lemma \ref{lemma-proj-separated} we see that $\text{Proj}(A)$ is separated.
Hence $\text{Proj}(A) \to U$ is separated (see
Schemes, Lemma \ref{schemes-lemma-compose-after-separated}) as desired.
\end{proof}

\begin{lemma}
\label{lemma-relative-proj-base-change}
Let $S$ be a scheme and $\mathcal{A}$ be a quasi-coherent sheaf
of graded $\mathcal{O}_S$-algebras. Let $g : S' \to S$ be any morphism
of schemes. Then there is a canonical isomorphism
$$
r :
\underline{\text{Proj}}_{S'}(g^*\mathcal{A})
\longrightarrow
S' \times_S \underline{\text{Proj}}_S(\mathcal{A})
$$
as well as a corresponding isomorphism
$$
\theta :
r^*\text{pr}_2^*\left(\bigoplus\nolimits_{d \in \mathbf{Z}}
\mathcal{O}_{\underline{\text{Proj}}_S(\mathcal{A})}(d)\right)
\longrightarrow
\bigoplus\nolimits_{d \in \mathbf{Z}}
\mathcal{O}_{\underline{\text{Proj}}_{S'}(g^*\mathcal{A})}(d)
$$
of $\mathbf{Z}$-graded
$\mathcal{O}_{\underline{\text{Proj}}_{S'}(g^*\mathcal{A})}$-algebras.
\end{lemma}

\begin{proof}
This follows from Lemma \ref{lemma-proj-base-change} and the construction
of $\underline{\text{Proj}}_S(\mathcal{A})$ in
Lemma \ref{lemma-relative-proj} as the union
of the schemes $U_d$ representing the functors $F_d$.
In terms of the construction of relative Proj via glueing
this isomorphism is given by the isomorphisms constructed
in Lemma \ref{lemma-base-change-map-proj} which provides us with
the isomorphism $\theta$. Some details omitted.
\end{proof}

\begin{lemma}
\label{lemma-apply-relative}
Let $S$ be a scheme.
Let $\mathcal{A}$ be a quasi-coherent sheaf of graded $\mathcal{O}_S$-modules
generated as an $\mathcal{A}_0$-algebra by $\mathcal{A}_1$.
In this case the scheme $X = \underline{\text{Proj}}_S(\mathcal{A})$
represents the functor $F_1$ which associates to a scheme
$f : T \to S$ over $S$ the set of pairs $(\mathcal{L}, \psi)$, where
\begin{enumerate}
\item $\mathcal{L}$ is an invertible $\mathcal{O}_T$-module, and
\item $\psi : f^*\mathcal{A} \to \bigoplus_{n \geq 0} \mathcal{L}^{\otimes n}$
is a graded $\mathcal{O}_T$-algebra homomorphism such that
$f^*\mathcal{A}_1 \to \mathcal{L}$ is surjective
\end{enumerate}
up to strict equivalence as above. Moreover, in this case all the
quasi-coherent sheaves
$\mathcal{O}_{\underline{\text{Proj}}(\mathcal{A})}(n)$
are invertible
$\mathcal{O}_{\underline{\text{Proj}}(\mathcal{A})}$-modules
and the multiplication maps induce isomorphisms
$
\mathcal{O}_{\underline{\text{Proj}}(\mathcal{A})}(n)
\otimes_{\mathcal{O}_{\underline{\text{Proj}}(\mathcal{A})}}
\mathcal{O}_{\underline{\text{Proj}}(\mathcal{A})}(m) =
\mathcal{O}_{\underline{\text{Proj}}(\mathcal{A})}(n + m)$.
\end{lemma}

\begin{proof}
Under the assumptions of the lemma the sheaves
$\mathcal{O}_{\underline{\text{Proj}}(\mathcal{A})}(n)$
are invertible and the multiplication maps isomorphisms
by Lemma \ref{lemma-relative-proj} and
Lemma \ref{lemma-apply}
over affine opens of $S$. Thus $X$ actually represents the
functor $F_1$, see proof of Lemma \ref{lemma-relative-proj}.
\end{proof}












\section{Quasi-coherent sheaves on relative Proj}
\label{section-quasi-coherent-relative-proj}

\noindent
We briefly discuss how to deal with graded modules in the relative
setting.

\medskip\noindent
We place ourselves in Situation \ref{situation-relative-proj}.
So $S$ is a scheme, and
$\mathcal{A}$ is a quasi-coherent graded $\mathcal{O}_S$-algebra.
Let $\mathcal{M} = \bigoplus_{n \in \mathbf{Z}} \mathcal{M}_n$
be a graded $\mathcal{A}$-module, quasi-coherent as an $\mathcal{O}_S$-module.
We are going to describe the associated quasi-coherent sheaf
of modules on $\underline{\text{Proj}}_S(\mathcal{A})$.
We first describe the value of this sheaf on schemes $T$ mapping
into the relative Proj.

\medskip\noindent
Let $T$ be a scheme. Let $(d, f : T \to S, \mathcal{L}, \psi)$
be a quadruple over $T$, as in Section \ref{section-relative-proj}.
We define a quasi-coherent sheaf
$\widetilde{\mathcal{M}}_T$ of $\mathcal{O}_T$-modules
as follows
\begin{equation}
\label{equation-widetilde-M}
\widetilde{\mathcal{M}}_T =
\left(
f^*\mathcal{M}^{(d)}
\otimes_{f^*\mathcal{A}^{(d)}}
\left(\bigoplus\nolimits_{n \in \mathbf{Z}} \mathcal{L}^{\otimes n}\right)
\right)_0
\end{equation}
So $\widetilde{\mathcal{M}}_T$ is the degree $0$ part
of the tensor product of the graded $f^*\mathcal{A}^{(d)}$-modules
$\mathcal{M}^{(d)}$ and
$\bigoplus\nolimits_{n \in \mathbf{Z}} \mathcal{L}^{\otimes n}$.
Note that the sheaf $\widetilde{\mathcal{M}}_T$ depends on the quadruple
even though we suppressed this in the notation.
This construction has the pleasing property that
given any morphism $g : T' \to T$ we have
$\widetilde{\mathcal{M}}_{T'} = g^*\widetilde{\mathcal{M}}_T$
where $\widetilde{\mathcal{M}}_{T'}$ denotes the quasi-coherent
sheaf associated to the pullback quadruple
$(d, f \circ g, g^*\mathcal{L}, g^*\psi)$.

\medskip\noindent
Since all sheaves in (\ref{equation-widetilde-M}) are quasi-coherent
we can spell out the construction
over an affine open $\Spec(C) = V \subset T$
which maps into an affine open $\Spec(R) = U \subset S$.
Namely, suppose that $\mathcal{A}|_U$ corresponds
to the graded $R$-algebra $A$, that $\mathcal{M}|_U$ corresponds to the
graded $A$-module $M$, and that $\mathcal{L}|_V$ corresponds to the
invertible $C$-module $L$. The map $\psi$ gives
rise to a graded $R$-algebra map
$\gamma : A^{(d)} \to \bigoplus_{n \geq 0} L^{\otimes n}$.
(Tensor powers of $L$ over $C$.)
Then $(\widetilde{\mathcal{M}}_T)|_V$
is the quasi-coherent sheaf associated to the $C$-module
$$
N_{R, C, A, M, \gamma} =
\left(
M^{(d)} \otimes_{A^{(d)}, \gamma}
\left(\bigoplus\nolimits_{n \in \mathbf{Z}} L^{\otimes n}\right)
\right)_0
$$
By assumption we may even cover $T$ by affine opens
$V$ such that there exists some $a \in A_d$ such that
$\gamma(a) \in L$ is a $C$-basis for the module $L$.
In that case any element of $N_{R, C, A, M, \gamma}$ is a sum
of pure tensors $\sum m_i \otimes \gamma(a)^{-n_i}$ with $m \in M_{n_id}$.
In fact we may multiply each $m_i$ with a suitable positive power
of $a$ and collect terms to see that each element of $N_{R, C, A, M, \gamma}$
can be written as $m \otimes \gamma(a)^{-n}$ with $m \in M_{nd}$ and
$n \gg 0$. In other words we see that in this case
$$
N_{R, C, A, M, \gamma} = M_{(a)} \otimes_{A_{(a)}} C
$$
where the map $A_{(a)} \to C$ is the map
$x/a^n \mapsto \gamma(x)/\gamma(a)^n$. In other words, this is
the value of $\widetilde{M}$ on $D_{+}(a) \subset \text{Proj}(A)$
pulled back to $\Spec(C)$ via the morphism
$\Spec(C) \to D_{+}(a)$ coming from $\gamma$.

\begin{lemma}
\label{lemma-relative-proj-modules}
In Situation \ref{situation-relative-proj}.
For any quasi-coherent sheaf of graded $\mathcal{A}$-modules
$\mathcal{M}$ on $S$, there exists a canonical associated sheaf
of $\mathcal{O}_{\underline{\text{Proj}}_S(\mathcal{A})}$-modules
$\widetilde{\mathcal{M}}$ with the following properties:
\begin{enumerate}
\item Given a scheme $T$ and a quadruple
$(T \to S, d, \mathcal{L}, \psi)$ over $T$
corresponding to a morphism
$h : T \to \underline{\text{Proj}}_S(\mathcal{A})$ there is
a canonical isomorphism
$\widetilde{\mathcal{M}}_T = h^*\widetilde{\mathcal{M}}$
where $\widetilde{\mathcal{M}}_T$ is defined by (\ref{equation-widetilde-M}).
\item The isomorphisms of (1) are compatible with pullbacks.
\item There is a canonical map
$$
\pi^*\mathcal{M}_0 \longrightarrow \widetilde{\mathcal{M}}.
$$
\item The construction $\mathcal{M} \mapsto \widetilde{\mathcal{M}}$
is functorial in $\mathcal{M}$.
\item The construction $\mathcal{M} \mapsto \widetilde{\mathcal{M}}$
is exact.
\item There are canonical maps
$$
\widetilde{\mathcal{M}}
\otimes_{\mathcal{O}_{\underline{\text{Proj}}_S(\mathcal{A})}}
\widetilde{\mathcal{N}}
\longrightarrow
\widetilde{\mathcal{M} \otimes_\mathcal{A} \mathcal{N}}
$$
as in
Lemma \ref{lemma-widetilde-tensor}.
\item There exist canonical maps
$$
\pi^*\mathcal{M}
\longrightarrow
\bigoplus\nolimits_{n \in \mathbf{Z}}
\widetilde{\mathcal{M}(n)}
$$
generalizing (\ref{equation-global-sections-more-generally}).
\item The formation of $\widetilde{\mathcal{M}}$ commutes with base change.
\end{enumerate}
\end{lemma}

\begin{proof}
Omitted. We should split this lemma into parts and prove the parts separately.
\end{proof}












\section{Functoriality of relative Proj}
\label{section-functoriality-relative-proj}

\noindent
This section is the analogue of Section \ref{section-functoriality-proj}
for the relative Proj. Let $S$ be a scheme. A graded $\mathcal{O}_S$-algebra
map $\psi : \mathcal{A} \to \mathcal{B}$ does not always give rise to a
morphism of associated relative Proj. The correct result is stated as follows.

\begin{lemma}
\label{lemma-morphism-relative-proj}
Let $S$ be a scheme. Let $\mathcal{A}$, $\mathcal{B}$ be two graded
quasi-coherent $\mathcal{O}_S$-algebras. Set
$p : X = \underline{\text{Proj}}_S(\mathcal{A}) \to S$ and
$q : Y = \underline{\text{Proj}}_S(\mathcal{B}) \to S$. Let
$\psi : \mathcal{A} \to \mathcal{B}$ be a homomorphism of
graded $\mathcal{O}_S$-algebras. There is a canonical open
$U(\psi) \subset Y$ and a canonical morphism of schemes
$$
r_\psi :
U(\psi)
\longrightarrow
X
$$
over $S$ and a map of $\mathbf{Z}$-graded $\mathcal{O}_{U(\psi)}$-algebras
$$
\theta = \theta_\psi :
r_\psi^*\left(
\bigoplus\nolimits_{d \in \mathbf{Z}} \mathcal{O}_X(d)
\right)
\longrightarrow
\bigoplus\nolimits_{d \in \mathbf{Z}} \mathcal{O}_{U(\psi)}(d).
$$
The triple $(U(\psi), r_\psi, \theta)$ is characterized by the property
that for any affine open $W \subset S$ the triple
$$
(U(\psi) \cap p^{-1}W,\quad
r_\psi|_{U(\psi) \cap p^{-1}W} : U(\psi) \cap p^{-1}W \to q^{-1}W,\quad
\theta|_{U(\psi) \cap p^{-1}W})
$$
is equal to the triple associated to
$\psi : \mathcal{A}(W) \to \mathcal{B}(W)$ in
Lemma \ref{lemma-morphism-proj} via the identifications
$p^{-1}W = \text{Proj}(\mathcal{A}(W))$ and
$q^{-1}W = \text{Proj}(\mathcal{B}(W))$ of
Section \ref{section-relative-proj-via-glueing}.
\end{lemma}

\begin{proof}
This lemma proves itself by glueing the local triples.
\end{proof}

\begin{lemma}
\label{lemma-morphism-relative-proj-transitive}
Let $S$ be a scheme. Let $\mathcal{A}$, $\mathcal{B}$, and $\mathcal{C}$ be
quasi-coherent graded $\mathcal{O}_S$-algebras.
Set $X = \underline{\text{Proj}}_S(\mathcal{A})$,
$Y = \underline{\text{Proj}}_S(\mathcal{B})$ and
$Z = \underline{\text{Proj}}_S(\mathcal{C})$.
Let $\varphi : \mathcal{A} \to \mathcal{B}$,
$\psi : \mathcal{B} \to \mathcal{C}$ be graded $\mathcal{O}_S$-algebra maps.
Then we have
$$
U(\psi \circ \varphi) = r_\varphi^{-1}(U(\psi))
\quad
\text{and}
\quad
r_{\psi \circ \varphi}
=
r_\varphi \circ r_\psi|_{U(\psi \circ \varphi)}.
$$
In addition we have
$$
\theta_\psi \circ r_\psi^*\theta_\varphi
=
\theta_{\psi \circ \varphi}
$$
with obvious notation.
\end{lemma}

\begin{proof}
Omitted.
\end{proof}

\begin{lemma}
\label{lemma-surjective-graded-rings-map-relative-proj}
With hypotheses and notation as in Lemma \ref{lemma-morphism-relative-proj}
above. Assume $\mathcal{A}_d \to \mathcal{B}_d$ is surjective for
$d \gg 0$. Then
\begin{enumerate}
\item $U(\psi) = Y$,
\item $r_\psi : Y \to X$ is a closed immersion, and
\item the maps $\theta : r_\psi^*\mathcal{O}_X(n) \to \mathcal{O}_Y(n)$
are surjective but not isomorphisms in general (even if
$\mathcal{A} \to \mathcal{B}$ is surjective).
\end{enumerate}
\end{lemma}

\begin{proof}
Follows on combining
Lemma \ref{lemma-morphism-relative-proj}
with
Lemma \ref{lemma-surjective-graded-rings-map-proj}.
\end{proof}

\begin{lemma}
\label{lemma-eventual-iso-graded-rings-map-relative-proj}
With hypotheses and notation as in Lemma \ref{lemma-morphism-relative-proj}
above. Assume $\mathcal{A}_d \to \mathcal{B}_d$ is an isomorphism for all
$d \gg 0$. Then
\begin{enumerate}
\item $U(\psi) = Y$,
\item $r_\psi : Y \to X$ is an isomorphism, and
\item the maps $\theta : r_\psi^*\mathcal{O}_X(n) \to \mathcal{O}_Y(n)$
are isomorphisms.
\end{enumerate}
\end{lemma}

\begin{proof}
Follows on combining
Lemma \ref{lemma-morphism-relative-proj}
with
Lemma \ref{lemma-eventual-iso-graded-rings-map-proj}.
\end{proof}

\begin{lemma}
\label{lemma-surjective-generated-degree-1-map-relative-proj}
With hypotheses and notation as in Lemma \ref{lemma-morphism-relative-proj}
above. Assume $\mathcal{A}_d \to \mathcal{B}_d$ is surjective for $d \gg 0$
and that $\mathcal{A}$ is generated by $\mathcal{A}_1$ over $\mathcal{A}_0$.
Then
\begin{enumerate}
\item $U(\psi) = Y$,
\item $r_\psi : Y \to X$ is a closed immersion, and
\item the maps $\theta : r_\psi^*\mathcal{O}_X(n) \to \mathcal{O}_Y(n)$
are isomorphisms.
\end{enumerate}
\end{lemma}

\begin{proof}
Follows on combining
Lemma \ref{lemma-morphism-relative-proj}
with
Lemma \ref{lemma-surjective-graded-rings-generated-degree-1-map-proj}.
\end{proof}

















\section{Invertible sheaves and morphisms into relative Proj}
\label{section-invertible-relative-proj}

\noindent
It seems that we may need the following lemma somewhere.
The situation is the following:
\begin{enumerate}
\item Let $S$ be a scheme.
\item Let $\mathcal{A}$ be a quasi-coherent graded $\mathcal{O}_S$-algebra.
\item Denote $\pi : \underline{\text{Proj}}_S(\mathcal{A}) \to S$ the relative
homogeneous spectrum over $S$.
\item Let $f : X \to S$ be a morphism of schemes.
\item Let $\mathcal{L}$ be an invertible $\mathcal{O}_X$-module.
\item Let $\psi : f^*\mathcal{A} \to
\bigoplus_{d \geq 0} \mathcal{L}^{\otimes d}$
be a homomorphism of graded $\mathcal{O}_X$-algebras.
\end{enumerate}
Given this data set
$$
U(\psi) = \bigcup\nolimits_{(U, V, a)} U_{\psi(a)}
$$
where $(U, V, a)$ satisfies:
\begin{enumerate}
\item $V \subset S$ affine open,
\item $U = f^{-1}(V)$, and
\item $a \in \mathcal{A}(V)_{+}$ is homogeneous.
\end{enumerate}
Namely, then $\psi(a) \in \Gamma(U, \mathcal{L}^{\otimes \deg(a)})$
and $U_{\psi(a)}$ is the corresponding open (see
Modules, Lemma \ref{modules-lemma-s-open}).

\begin{lemma}
\label{lemma-invertible-map-into-relative-proj}
With assumptions and notation as above. The morphism
$\psi$ induces a canonical morphism of schemes over $S$
$$
r_{\mathcal{L}, \psi} :
U(\psi) \longrightarrow \underline{\text{Proj}}_S(\mathcal{A})
$$
together with a map of graded $\mathcal{O}_{U(\psi)}$-algebras
$$
\theta :
r_{\mathcal{L}, \psi}^*\left(
\bigoplus\nolimits_{d \geq 0}
\mathcal{O}_{\underline{\text{Proj}}_S(\mathcal{A})}(d)
\right)
\longrightarrow
\bigoplus\nolimits_{d \geq 0} \mathcal{L}^{\otimes d}|_{U(\psi)}
$$
characterized by the following properties:
\begin{enumerate}
\item For every open $V \subset S$ and every $d \geq 0$ the diagram
$$
\xymatrix{
\mathcal{A}_d(V) \ar[d]_{\psi} \ar[r]_{\psi} &
\Gamma(f^{-1}(V), \mathcal{L}^{\otimes d}) \ar[d]^{restrict} \\
\Gamma(\pi^{-1}(V),
\mathcal{O}_{\underline{\text{Proj}}_S(\mathcal{A})}(d)) \ar[r]^{\theta} &
\Gamma(f^{-1}(V) \cap U(\psi), \mathcal{L}^{\otimes d})
}
$$
is commutative.
\item For any $d \geq 1$ and any open subscheme $W \subset X$
such that $\psi|_W : f^*\mathcal{A}_d|_W \to \mathcal{L}^{\otimes d}|_W$
is surjective the restriction of the morphism $r_{\mathcal{L}, \psi}$
agrees with the morphism $W \to \underline{\text{Proj}}_S(\mathcal{A})$
which exists by the construction of the relative homogeneous spectrum,
see Definition \ref{definition-relative-proj}.
\item For any affine open $V \subset S$, the restriction
$$
(U(\psi) \cap f^{-1}(V), r_{\mathcal{L}, \psi}|_{U(\psi) \cap f^{-1}(V)},
\theta|_{U(\psi) \cap f^{-1}(V)})
$$
agrees via $i_V$ (see Lemma \ref{lemma-glue-relative-proj}) with the triple
$(U(\psi'), r_{\mathcal{L}, \psi'}, \theta')$
of Lemma \ref{lemma-invertible-map-into-proj} associated to the map
$\psi' : A = \mathcal{A}(V) \to \Gamma_*(f^{-1}(V), \mathcal{L}|_{f^{-1}(V)})$
induced by $\psi$.
\end{enumerate}
\end{lemma}

\begin{proof}
Use characterization (3) to construct the morphism $r_{\mathcal{L}, \psi}$
and $\theta$ locally over $S$. Use the uniqueness of
Lemma \ref{lemma-invertible-map-into-proj}
to show that the construction glues. Details omitted.
\end{proof}







\section{Twisting by invertible sheaves and relative Proj}
\label{section-twisting-and-proj}

\noindent
Let $S$ be a scheme.
Let $\mathcal{A} = \bigoplus_{d \geq 0} \mathcal{A}_d$ be a
quasi-coherent graded $\mathcal{O}_S$-algebra.
Let $\mathcal{L}$ be an invertible sheaf on $S$.
In this situation we obtain another quasi-coherent graded
$\mathcal{O}_S$-algebra, namely
$$
\mathcal{B}
=
\bigoplus\nolimits_{d \geq 0}
\mathcal{A}_d \otimes_{\mathcal{O}_S} \mathcal{L}^{\otimes d}
$$
It turns out that $\mathcal{A}$ and $\mathcal{B}$ have
isomorphic relative homogeneous spectra.

\begin{lemma}
\label{lemma-twisting-and-proj}
With notation $S$, $\mathcal{A}$, $\mathcal{L}$ and $\mathcal{B}$ as
above. There is a canonical isomorphism
$$
\xymatrix{
P = \underline{\text{Proj}}_S(\mathcal{A})
\ar[rr]_g \ar[rd]_\pi & &
\underline{\text{Proj}}_S(\mathcal{B}) = P'
\ar[ld]^{\pi'} \\
& S &
}
$$
with the following properties
\begin{enumerate}
\item There are isomorphisms
$\theta_n : g^*\mathcal{O}_{P'}(n)
\to
\mathcal{O}_P(n) \otimes \pi^*\mathcal{L}^{\otimes n}$
which fit together to give an isomorphism of $\mathbf{Z}$-graded
algebras
$$
\theta :
g^*\left(
\bigoplus\nolimits_{n \in \mathbf{Z}} \mathcal{O}_{P'}(n)
\right)
\longrightarrow
\bigoplus\nolimits_{n \in \mathbf{Z}} \mathcal{O}_P(n)
\otimes \pi^*\mathcal{L}^{\otimes n}
$$
\item For every open $V \subset S$ the diagrams
$$
\xymatrix{
\mathcal{A}_n(V) \otimes \mathcal{L}^{\otimes n}(V)
\ar[r]_{multiply} \ar[d]^{\psi \otimes \pi^*}
&
\mathcal{B}_n(V) \ar[dd]^\psi \\
\Gamma(\pi^{-1}V, \mathcal{O}_P(n)) \otimes
\Gamma(\pi^{-1}V, \pi^*\mathcal{L}^{\otimes n})
\ar[d]^{multiply} \\
\Gamma(\pi^{-1}V, \mathcal{O}_P(n) \otimes \pi^*\mathcal{L}^{\otimes n})
&
\Gamma(\pi'^{-1}V, \mathcal{O}_{P'}(n)) \ar[l]_-{\theta_n}
}
$$
are commutative.
\item Add more here as necessary.
\end{enumerate}
\end{lemma}

\begin{proof}
This is the identity map when $\mathcal{L} \cong \mathcal{O}_S$.
In general choose an open covering of $S$ such that $\mathcal{L}$
is trivialized over the pieces and glue the corresponding maps.
Details omitted.
\end{proof}
















\section{Projective bundles}
\label{section-projective-bundle}

\noindent
Let $S$ be a scheme.
Let $\mathcal{E}$ be a quasi-coherent sheaf of $\mathcal{O}_S$-modules.
By Modules, Lemma \ref{modules-lemma-whole-tensor-algebra-permanence}
the symmetric algebra $\text{Sym}(\mathcal{E})$ of
$\mathcal{E}$ over $\mathcal{O}_S$
is a quasi-coherent sheaf of $\mathcal{O}_S$-algebras.
Note that it is generated in degree $1$ over $\mathcal{O}_S$.
Hence it makes sense to apply the construction of the
previous section to it, specifically Lemmas
\ref{lemma-relative-proj} and \ref{lemma-apply-relative}.

\begin{definition}
\label{definition-projective-bundle}
Let $S$ be a scheme. Let $\mathcal{E}$ be a quasi-coherent
$\mathcal{O}_S$-module\footnote{The reader may expect here
the condition that $\mathcal{E}$ is finite locally free. We do not
do so in order to be consistent with
\cite[II, Definition 4.1.1]{EGA}.}.
We denote
$$
\pi :
\mathbf{P}(\mathcal{E}) = \underline{\text{Proj}}_S(\text{Sym}(\mathcal{E}))
\longrightarrow
S
$$
and we call it the {\it projective bundle associated to $\mathcal{E}$}.
The symbol $\mathcal{O}_{\mathbf{P}(\mathcal{E})}(n)$
indicates the invertible $\mathcal{O}_{\mathbf{P}(\mathcal{E})}$-module
of Lemma \ref{lemma-apply-relative} and is called the $n$th
{\it twist of the structure sheaf}.
\end{definition}

\noindent
According to Lemma \ref{lemma-glue-relative-proj-twists} there are
canonical $\mathcal{O}_S$-module homomorphisms
$$
\text{Sym}^n(\mathcal{E})
\longrightarrow
\pi_*\mathcal{O}_{\mathbf{P}(\mathcal{E})}(n)
\quad\text{equivalently}\quad
\pi^*\text{Sym}^n(\mathcal{E})
\longrightarrow
\mathcal{O}_{\mathbf{P}(\mathcal{E})}(n)
$$
for all $n \geq 0$. In particular, for $n = 1$ we have
$$
\mathcal{E}
\longrightarrow
\pi_*\mathcal{O}_{\mathbf{P}(\mathcal{E})}(1)
\quad\text{equivalently}\quad
\pi^*\mathcal{E}
\longrightarrow
\mathcal{O}_{\mathbf{P}(\mathcal{E})}(1)
$$
and the map $\pi^*\mathcal{E} \to \mathcal{O}_{\mathbf{P}(\mathcal{E})}(1)$
is a surjection by Lemma \ref{lemma-apply-relative}.
This is a good way to remember how we have normalized
our construction of $\mathbf{P}(\mathcal{E})$.

\medskip\noindent
Warning: In some references the scheme $\mathbf{P}(\mathcal{E})$
is only defined for $\mathcal{E}$ finite locally free on $S$.
Moreover sometimes $\mathbf{P}(\mathcal{E})$ is actually defined as our
$\mathbf{P}(\mathcal{E}^\vee)$ where $\mathcal{E}^\vee$
is the dual of $\mathcal{E}$ (and this is done only when $\mathcal{E}$ is
finite locally free).

\medskip\noindent
Let $S$, $\mathcal{E}$, $\mathbf{P}(\mathcal{E}) \to S$ be as in
Definition \ref{definition-projective-bundle}. Let $f : T \to S$
be a scheme over $S$. Let $\psi : f^*\mathcal{E} \to \mathcal{L}$
be a surjection where $\mathcal{L}$ is an invertible $\mathcal{O}_T$-module.
The induced graded $\mathcal{O}_T$-algebra map
$$
f^*\text{Sym}(\mathcal{E}) = \text{Sym}(f^*\mathcal{E}) \to
\text{Sym}(\mathcal{L}) = \bigoplus\nolimits_{n \geq 0} \mathcal{L}^{\otimes n}
$$
corresponds to a morphism
$$
\varphi_{\mathcal{L}, \psi} : T \longrightarrow \mathbf{P}(\mathcal{E})
$$
over $S$ by our construction of the relative Proj as the scheme representing
the functor $F$ in Section \ref{section-relative-proj}. On the other hand,
given a morphism $\varphi : T \to \mathbf{P}(\mathcal{E})$ over $S$
we can set $\mathcal{L} = \varphi^*\mathcal{O}_{\mathbf{P}(\mathcal{E})}(1)$
and $\psi : f^*\mathcal{E} \to \mathcal{L}$ equal to the pullback
by $\varphi$ of the canonical surjection
$\pi^*\mathcal{E} \to \mathcal{O}_{\mathbf{P}(\mathcal{E})}(1)$.
By Lemma \ref{lemma-apply-relative} these constructions
are inverse bijections between the set of isomorphism classes of pairs
$(\mathcal{L}, \psi)$ and the set of morphisms
$\varphi : T \to \mathbf{P}(\mathcal{E})$ over $S$.
Thus we see that $\mathbf{P}(\mathcal{E})$ represents the functor
which associates to $f : T \to S$ the set of $\mathcal{O}_T$-module
quotients of $f^*\mathcal{E}$ which are locally free of rank $1$.

\begin{example}[Projective space of a vector space]
\label{example-projective-space}
Let $k$ be a field. Let $V$ be a $k$-vector space. The corresponding
{\it projective space} is the $k$-scheme
$$
\mathbf{P}(V) = \text{Proj}(\text{Sym}(V))
$$
where $\text{Sym}(V)$ is the symmetric algebra on $V$ over $k$.
Of course we have $\mathbf{P}(V) \cong \mathbf{P}^n_k$ if $\dim(V) = n + 1$
because then the symmetric algebra on $V$ is isomorphic to a polynomial
ring in $n + 1$ variables. If we
think of $V$ as a quasi-coherent module on $\Spec(k)$, then $\mathbf{P}(V)$
is the corresponding projective space bundle over $\Spec(k)$. By the
discussion above a $k$-valued point $p$ of $\mathbf{P}(V)$ corresponds to
a surjection of $k$-vector spaces $V \to L_p$ with $\dim(L_p) = 1$.
More generally, let $X$ be a scheme over $k$, let $\mathcal{L}$ be an
invertible $\mathcal{O}_X$-module, and let
$\psi : V \to \Gamma(X, \mathcal{L})$ be a $k$-linear map
such that $\mathcal{L}$ is generated as an $\mathcal{O}_X$-module
by the sections in the image of $\psi$. Then the discussion above
gives a canonical morphism
$$
\varphi_{\mathcal{L}, \psi} : X \longrightarrow \mathbf{P}(V)
$$
of schemes over $k$ such that there is an isomorphism
$\theta : \varphi_{\mathcal{L}, \psi}^*\mathcal{O}_{\mathbf{P}(V)}(1)
\to \mathcal{L}$ and such that $\psi$ agrees with the composition
$$
V \to
\Gamma(\mathbf{P}(V), \mathcal{O}_{\mathbf{P}(V)}(1))
\to
\Gamma(X, \varphi_{\mathcal{L}, \psi}^*\mathcal{O}_{\mathbf{P}(V)}(1))
\to
\Gamma(X, \mathcal{L})
$$
See Lemma \ref{lemma-invertible-map-into-proj}. If
$V \subset \Gamma(X, \mathcal{L})$ is a subspace, then we will
denote the morphism constructed above simply as
$\varphi_{\mathcal{L}, V}$.
If $\dim(V) = n + 1$ and we choose a basis $v_0, \ldots, v_n$ of $V$
then the diagram
$$
\xymatrix{
X \ar@{=}[d] \ar[rr]_{\varphi_{\mathcal{L}, \psi}} & &
\mathbf{P}(V) \ar[d]^{\cong} \\
X \ar[rr]^{\varphi_{(\mathcal{L}, (s_0, \ldots, s_n))}} & &
\mathbf{P}^n_k
}
$$
is commutative, where $s_i = \psi(v_i) \in \Gamma(X, \mathcal{L})$, where
$\varphi_{(\mathcal{L}, (s_0, \ldots, s_n))}$
is as in Section \ref{section-projective-space},
and where the right vertical arrow corresponds
to the isomorphism $k[T_0, \ldots, T_n] \to \text{Sym}(V)$ sending
$T_i$ to $v_i$.
\end{example}

\begin{example}
\label{example-projective-bundle}
The map $\text{Sym}^n(\mathcal{E}) \to
\pi_*(\mathcal{O}_{\mathbf{P}(\mathcal{E})}(n))$
is an isomorphism if $\mathcal{E}$ is locally free, but in general
need not be an isomorphism. In fact we will give an example where
this map is not injective for $n = 1$. Set $S = \Spec(A)$ with
$$
A = k[u, v, s_1, s_2, t_1, t_2]/I
$$
where $k$ is a field and
$$
I = (-us_1 + vt_1 + ut_2, vs_1 + us_2 - vt_2, vs_2, ut_1).
$$
Denote $\overline{u}$ the class of $u$ in $A$ and similarly for
the other variables.
Let $M = (Ax \oplus Ay)/A(\overline{u}x + \overline{v}y)$ so that
$$
\text{Sym}(M) = A[x, y]/(\overline{u}x + \overline{v}y)
= k[x, y, u, v, s_1, s_2, t_1, t_2]/J
$$
where
$$
J = (-us_1 + vt_1 + ut_2, vs_1 + us_2 - vt_2, vs_2, ut_1, ux + vy).
$$
In this case the projective bundle associated to the quasi-coherent
sheaf $\mathcal{E} = \widetilde{M}$ on $S = \Spec(A)$ is the scheme
$$
P =
\text{Proj}(\text{Sym}(M)).
$$
Note that this scheme as an affine open covering
$P = D_{+}(x) \cup D_{+}(y)$.
Consider the element
$m \in M$ which is the image of the element
$us_1x + vt_2y$. Note that
$$
x(us_1x + vt_2y) = (s_1x + s_2y)(ux + vy) \bmod I
$$
and
$$
y(us_1x + vt_2y) = (t_1x + t_2y)(ux + vy) \bmod I.
$$
The first equation implies that $m$ maps to zero as a
section of $\mathcal{O}_P(1)$ on $D_{+}(x)$ and the second
that it maps to zero as a section of $\mathcal{O}_P(1)$ on $D_{+}(y)$.
This shows that $m$ maps to zero in $\Gamma(P, \mathcal{O}_P(1))$.
On the other hand we claim that $m \not = 0$, so that $m$ gives
an example of a nonzero global section of $\mathcal{E}$ mapping to zero
in $\Gamma(P, \mathcal{O}_P(1))$. Assume $m = 0$
to get a contradiction. In this case there exists
an element $f \in k[u, v, s_1, s_2, t_1, t_2]$ such that
$$
us_1x + vt_2y = f(ux + vy) \bmod I
$$
Since $I$ is generated by homogeneous polynomials of degree $2$ we
may decompose $f$ into its homogeneous components and take the
degree 1 component. In other words we may assume that
$$
f = au + bv + \alpha_1s_1 + \alpha_2s_2 + \beta_1t_1 + \beta_2t_2
$$
for some $a, b, \alpha_1, \alpha_2, \beta_1, \beta_2 \in k$.
The resulting conditions are that
$$
\begin{matrix}
us_1 - u(au + bv + \alpha_1s_1 + \alpha_2s_2 + \beta_1t_1 + \beta_2t_2)
\in I \\
vt_2 - v(au + bv + \alpha_1s_1 + \alpha_2s_2 + \beta_1t_1 + \beta_2t_2)
\in I
\end{matrix}
$$
There are no terms $u^2, uv, v^2$ in the generators of $I$ and
hence we see $a = b = 0$. Thus we get the relations
$$
\begin{matrix}
us_1 - u(\alpha_1s_1 + \alpha_2s_2 + \beta_1t_1 + \beta_2t_2)
\in I \\
vt_2 - v(\alpha_1s_1 + \alpha_2s_2 + \beta_1t_1 + \beta_2t_2)
\in I
\end{matrix}
$$
We may use the first generator of $I$ to replace any occurrence of
$us_1$ by $vt_1 + ut_2$, the second generator of $I$ to replace any
occurrence of $vs_1$ by $-us_2 + vt_2$, the third generator
to remove occurrences of $vs_2$ and the third to remove occurrences
of $ut_1$. Then we get the relations
$$
\begin{matrix}
(1 - \alpha_1)vt_1 + (1 - \alpha_1)ut_2 - \alpha_2us_2 - \beta_2ut_2 = 0 \\
(1 - \alpha_1)vt_2 + \alpha_1us_2 - \beta_1vt_1 - \beta_2vt_2 = 0
\end{matrix}
$$
This implies that $\alpha_1$ should be both $0$ and $1$ which is
a contradiction as desired.
\end{example}


\begin{lemma}
\label{lemma-projective-bundle-separated}
Let $S$ be a scheme.
The structure morphism $\mathbf{P}(\mathcal{E}) \to S$ of a
projective bundle over $S$ is separated.
\end{lemma}

\begin{proof}
Immediate from Lemma \ref{lemma-relative-proj-separated}.
\end{proof}

\begin{lemma}
\label{lemma-projective-space-bundle}
Let $S$ be a scheme. Let $n \geq 0$. Then
$\mathbf{P}^n_S$ is a projective bundle over $S$.
\end{lemma}

\begin{proof}
Note that
$$
\mathbf{P}^n_{\mathbf{Z}} =
\text{Proj}(\mathbf{Z}[T_0, \ldots, T_n]) =
\underline{\text{Proj}}_{\Spec(\mathbf{Z})}
\left(\widetilde{\mathbf{Z}[T_0, \ldots, T_n]}\right)
$$
where the grading on the ring $\mathbf{Z}[T_0, \ldots, T_n]$ is given by
$\deg(T_i) = 1$ and the elements of $\mathbf{Z}$ are in degree $0$.
Recall that $\mathbf{P}^n_S$ is defined as
$\mathbf{P}^n_{\mathbf{Z}} \times_{\Spec(\mathbf{Z})} S$.
Moreover, forming the relative homogeneous spectrum commutes with base change,
see Lemma \ref{lemma-relative-proj-base-change}.
For any scheme $g : S \to \Spec(\mathbf{Z})$ we have
$g^*\mathcal{O}_{\Spec(\mathbf{Z})}[T_0, \ldots, T_n]
= \mathcal{O}_S[T_0, \ldots, T_n]$.
Combining the above we see that
$$
\mathbf{P}^n_S = \underline{\text{Proj}}_S(\mathcal{O}_S[T_0, \ldots, T_n]).
$$
Finally, note that
$\mathcal{O}_S[T_0, \ldots, T_n] = \text{Sym}(\mathcal{O}_S^{\oplus n + 1})$.
Hence we see that $\mathbf{P}^n_S$ is a projective bundle over $S$.
\end{proof}



\section{Grassmannians}
\label{section-grassmannian}

\noindent
In this section we introduce the standard Grassmannian functors and
we show that they are represented by schemes. Pick integers $k$, $n$
with $0 < k < n$. We will construct a functor
\begin{equation}
\label{equation-gkn}
G(k, n) : \Sch \longrightarrow \textit{Sets}
\end{equation}
which will loosely speaking parametrize $k$-dimensional subspaces
of $n$-space. However, for technical reasons it is more convenient
to parametrize $(n - k)$-dimensional quotients and this is what we will
do.

\medskip\noindent
More precisely, $G(k, n)$ associates to a scheme $S$ the set $G(k, n)(S)$
of isomorphism classes of surjections
$$
q : \mathcal{O}_S^{\oplus n} \longrightarrow \mathcal{Q}
$$
where $\mathcal{Q}$ is a finite locally free $\mathcal{O}_S$-module
of rank $n - k$. Note that this is indeed a set, for example by
Modules, Lemma \ref{modules-lemma-set-isomorphism-classes-finite-type-modules}
or by the observation that the isomorphism class of the surjection $q$
is determined by the kernel of $q$ (and given a sheaf there is a set
of subsheaves). Given a morphism of schemes $f : T \to S$ we let
$G(k, n)(f) : G(k, n)(S) \to G(k, n)(T)$ which sends the
isomorphism class of $q : \mathcal{O}_S^{\oplus n} \longrightarrow \mathcal{Q}$
to the isomorphism class of
$f^*q : \mathcal{O}_T^{\oplus n} \longrightarrow f^*\mathcal{Q}$.
This makes sense since  (1) $f^*\mathcal{O}_S = \mathcal{O}_T$,
(2) $f^*$ is additive, (3) $f^*$ preserves locally free modules
(Modules, Lemma \ref{modules-lemma-pullback-locally-free}),
and (4) $f^*$ is right exact
(Modules, Lemma \ref{modules-lemma-exactness-pushforward-pullback}).

\begin{lemma}
\label{lemma-gkn-representable}
Let $0 < k < n$.
The functor $G(k, n)$ of (\ref{equation-gkn}) is representable by a scheme.
\end{lemma}

\begin{proof}
Set $F = G(k, n)$. To prove the lemma we will use the criterion of
Schemes, Lemma \ref{schemes-lemma-glue-functors}.
The reason $F$ satisfies the sheaf property for the
Zariski topology is that we can glue sheaves, see Sheaves,
Section \ref{sheaves-section-glueing-sheaves} (some details omitted).

\medskip\noindent
The family of subfunctors $F_i$.
Let $I$ be the set of subsets of $\{1, \ldots, n\}$ of cardinality $n - k$.
Given a scheme $S$ and $j \in \{1, \ldots, n\}$ we denote $e_j$
the global section
$$
e_j = (0, \ldots, 0, 1, 0, \ldots, 0)\quad(1\text{ in }j\text{th spot})
$$
of $\mathcal{O}_S^{\oplus n}$. Of course these sections freely generate
$\mathcal{O}_S^{\oplus n}$. Similarly, for $j \in \{1, \ldots, n - k\}$
we denote $f_j$ the global section of $\mathcal{O}_S^{\oplus n - k}$
which is zero in all summands except the $j$th where we put a $1$.
For $i \in I$ we let
$$
s_i : \mathcal{O}_S^{\oplus n - k} \longrightarrow \mathcal{O}_S^{\oplus n}
$$
which is the direct sum of the coprojections
$\mathcal{O}_S \to \mathcal{O}_S^{\oplus n}$ corresponding to elements of $I$.
More precisely, if $i = \{i_1, \ldots, i_{n - k}\}$ with
$i_1 < i_2 < \ldots < i_{n - k}$
then $s_i$ maps $f_j$ to $e_{i_j}$ for $j \in \{1, \ldots, n - k\}$.
With this notation we can set
$$
F_i(S) = \{q : \mathcal{O}_S^{\oplus n} \to \mathcal{Q} \in F(S) \mid
q \circ s_i \text{ is surjective}\}
\subset F(S)
$$
Given a morphism $f : T \to S$ of schemes the pullback $f^*s_i$
is the corresponding map over $T$. Since $f^*$ is right exact
(Modules, Lemma \ref{modules-lemma-exactness-pushforward-pullback})
we conclude that $F_i$ is a subfunctor of $F$.

\medskip\noindent
Representability of $F_i$. To prove this we may assume (after renumbering)
that $i = \{1, \ldots, n - k\}$. This means $s_i$ is the inclusion of
the first $n - k$ summands. Observe that if $q \circ s_i$ is surjective,
then $q \circ s_i$ is an isomorphism as a surjective map between finite
locally free modules of the same rank
(Modules, Lemma \ref{modules-lemma-map-finite-locally-free}).
Thus if $q : \mathcal{O}_S^{\oplus n} \to \mathcal{Q}$ is an element of
$F_i(S)$, then we can use $q \circ s_i$ to identify $\mathcal{Q}$ with
$\mathcal{O}_S^{\oplus n - k}$. After doing so we obtain
$$
q : \mathcal{O}_S^{\oplus n} \longrightarrow \mathcal{O}_S^{\oplus n - k}
$$
mapping $e_j$ to $f_j$ (notation as above) for $j = 1, \ldots, n - k$.
To determine $q$ completely we have to fix the images
$q(e_{n - k + 1}), \ldots, q(e_n)$ in
$\Gamma(S, \mathcal{O}_S^{\oplus n - k})$.
It follows that $F_i$ is isomorphic to the functor
$$
S \longmapsto
\prod\nolimits_{j = n - k + 1, \ldots, n}
\Gamma(S,  \mathcal{O}_S^{\oplus n - k})
$$
This functor is isomorphic to the $k(n - k)$-fold self product of the functor
$S \mapsto \Gamma(S, \mathcal{O}_S)$. By
Schemes, Example \ref{schemes-example-global-sections}
the latter is representable by $\mathbf{A}^1_\mathbf{Z}$. It follows $F_i$
is representable by $\mathbf{A}^{k(n - k)}_\mathbf{Z}$ since fibred product
over $\Spec(\mathbf{Z})$ is the product in the category of schemes.

\medskip\noindent
The inclusion $F_i \subset F$ is representable by open immersions.
Let $S$ be a scheme and let
$q : \mathcal{O}_S^{\oplus n} \to \mathcal{Q}$ be an element of
$F(S)$. By
Modules, Lemma \ref{modules-lemma-finite-type-surjective-on-stalk}.
the set $U_i = \{s \in S \mid (q \circ s_i)_s\text{ surjective}\}$
is open in $S$. Since $\mathcal{O}_{S, s}$ is a local ring
and $\mathcal{Q}_s$ a finite $\mathcal{O}_{S, s}$-module
by Nakayama's lemma (Algebra, Lemma \ref{algebra-lemma-NAK}) we have
$$
s \in U_i \Leftrightarrow
\left(
\text{the map }
\kappa(s)^{\oplus n - k} \to \mathcal{Q}_s/\mathfrak m_s\mathcal{Q}_s
\text{ induced by }
(q \circ s_i)_s
\text{ is surjective}
\right)
$$
Let $f : T \to S$ be a morphism of schemes and let $t \in T$ be a point
mapping to $s \in S$. We have
$(f^*\mathcal{Q})_t =
\mathcal{Q}_s \otimes_{\mathcal{O}_{S, s}} \mathcal{O}_{T, t}$
(Sheaves, Lemma \ref{sheaves-lemma-stalk-pullback-modules})
and so on. Thus the map
$$
\kappa(t)^{\oplus n - k} \to (f^*\mathcal{Q})_t/\mathfrak m_t(f^*\mathcal{Q})_t
$$
induced by $(f^*q \circ f^*s_i)_t$ is the base change of the map
$\kappa(s)^{\oplus n - k} \to \mathcal{Q}_s/\mathfrak m_s\mathcal{Q}_s$
above by the field extension $\kappa(t)/\kappa(s)$. It follows
that $s \in U_i$ if and only if $t$ is in the corresponding open
for $f^*q$. In particular $T \to S$ factors through $U_i$ if
and only if $f^*q \in F_i(T)$ as desired.

\medskip\noindent
The collection $F_i$, $i \in I$ covers $F$. Let
$q : \mathcal{O}_S^{\oplus n} \to \mathcal{Q}$ be an element of
$F(S)$. We have to show that for every point $s$ of $S$ there exists
an $i \in I$ such that $s_i$ is surjective in a neighbourhood of $s$.
Thus we have to show that one of the compositions
$$
\kappa(s)^{\oplus n - k} \xrightarrow{s_i}
\kappa(s)^{\oplus n} \rightarrow
\mathcal{Q}_s/\mathfrak m_s\mathcal{Q}_s
$$
is surjective (see previous paragraph). As
$\mathcal{Q}_s/\mathfrak m_s\mathcal{Q}_s$ is a vector space of
dimension $n - k$ this follows from the theory of vector spaces.
\end{proof}

\begin{definition}
\label{definition-grassmannian}
Let $0 < k < n$. The scheme $\mathbf{G}(k, n)$ representing the functor
$G(k, n)$ is called {\it Grassmannian over $\mathbf{Z}$}.
Its base change $\mathbf{G}(k, n)_S$ to a scheme $S$ is called
{\it Grassmannian over $S$}. If $R$ is a ring the base change
to $\Spec(R)$ is denoted $\mathbf{G}(k, n)_R$ and called
{\it Grassmannian over $R$}.
\end{definition}

\noindent
The definition makes sense as we've shown in
Lemma \ref{lemma-gkn-representable}
that these functors are indeed representable.

\begin{lemma}
\label{lemma-projective-space-grassmannian}
Let $n \geq 1$. There is a canonical isomorphism
$\mathbf{G}(n, n + 1) = \mathbf{P}^n_\mathbf{Z}$.
\end{lemma}

\begin{proof}
According to Lemma \ref{lemma-projective-space} the scheme
$\mathbf{P}^n_\mathbf{Z}$ represents the functor
which assigns to a scheme $S$ the set of isomorphisms classes
of pairs $(\mathcal{L}, (s_0, \ldots, s_n))$ consisting of
an invertible module $\mathcal{L}$ and an $(n + 1)$-tuple
of global sections generating $\mathcal{L}$.
Given such a pair we obtain a quotient
$$
\mathcal{O}_S^{\oplus n + 1} \longrightarrow \mathcal{L},\quad
(h_0, \ldots, h_n) \longmapsto \sum h_i s_i.
$$
Conversely, given an element
$q : \mathcal{O}_S^{\oplus n + 1} \to \mathcal{Q}$ of $G(n, n + 1)(S)$
we obtain such a pair, namely $(\mathcal{Q}, (q(e_1), \ldots, q(e_{n + 1})))$.
Here $e_i$, $i = 1, \ldots, n + 1$ are the standard generating sections
of the free module $\mathcal{O}_S^{\oplus n + 1}$.
We omit the verification that these constructions define mutually
inverse transformations of functors.
\end{proof}


\begin{multicols}{2}[\section{Other chapters}]
\noindent
Preliminaries
\begin{enumerate}
\item \hyperref[introduction-section-phantom]{Introduction}
\item \hyperref[conventions-section-phantom]{Conventions}
\item \hyperref[sets-section-phantom]{Set Theory}
\item \hyperref[categories-section-phantom]{Categories}
\item \hyperref[topology-section-phantom]{Topology}
\item \hyperref[sheaves-section-phantom]{Sheaves on Spaces}
\item \hyperref[sites-section-phantom]{Sites and Sheaves}
\item \hyperref[stacks-section-phantom]{Stacks}
\item \hyperref[fields-section-phantom]{Fields}
\item \hyperref[algebra-section-phantom]{Commutative Algebra}
\item \hyperref[brauer-section-phantom]{Brauer Groups}
\item \hyperref[homology-section-phantom]{Homological Algebra}
\item \hyperref[derived-section-phantom]{Derived Categories}
\item \hyperref[simplicial-section-phantom]{Simplicial Methods}
\item \hyperref[more-algebra-section-phantom]{More on Algebra}
\item \hyperref[smoothing-section-phantom]{Smoothing Ring Maps}
\item \hyperref[modules-section-phantom]{Sheaves of Modules}
\item \hyperref[sites-modules-section-phantom]{Modules on Sites}
\item \hyperref[injectives-section-phantom]{Injectives}
\item \hyperref[cohomology-section-phantom]{Cohomology of Sheaves}
\item \hyperref[sites-cohomology-section-phantom]{Cohomology on Sites}
\item \hyperref[dga-section-phantom]{Differential Graded Algebra}
\item \hyperref[dpa-section-phantom]{Divided Power Algebra}
\item \hyperref[hypercovering-section-phantom]{Hypercoverings}
\end{enumerate}
Schemes
\begin{enumerate}
\setcounter{enumi}{24}
\item \hyperref[schemes-section-phantom]{Schemes}
\item \hyperref[constructions-section-phantom]{Constructions of Schemes}
\item \hyperref[properties-section-phantom]{Properties of Schemes}
\item \hyperref[morphisms-section-phantom]{Morphisms of Schemes}
\item \hyperref[coherent-section-phantom]{Cohomology of Schemes}
\item \hyperref[divisors-section-phantom]{Divisors}
\item \hyperref[limits-section-phantom]{Limits of Schemes}
\item \hyperref[varieties-section-phantom]{Varieties}
\item \hyperref[topologies-section-phantom]{Topologies on Schemes}
\item \hyperref[descent-section-phantom]{Descent}
\item \hyperref[perfect-section-phantom]{Derived Categories of Schemes}
\item \hyperref[more-morphisms-section-phantom]{More on Morphisms}
\item \hyperref[flat-section-phantom]{More on Flatness}
\item \hyperref[groupoids-section-phantom]{Groupoid Schemes}
\item \hyperref[more-groupoids-section-phantom]{More on Groupoid Schemes}
\item \hyperref[etale-section-phantom]{\'Etale Morphisms of Schemes}
\end{enumerate}
Topics in Scheme Theory
\begin{enumerate}
\setcounter{enumi}{40}
\item \hyperref[chow-section-phantom]{Chow Homology}
\item \hyperref[intersection-section-phantom]{Intersection Theory}
\item \hyperref[pic-section-phantom]{Picard Schemes of Curves}
\item \hyperref[adequate-section-phantom]{Adequate Modules}
\item \hyperref[dualizing-section-phantom]{Dualizing Complexes}
\item \hyperref[duality-section-phantom]{Duality for Schemes}
\item \hyperref[discriminant-section-phantom]{Discriminants and Differents}
\item \hyperref[local-cohomology-section-phantom]{Local Cohomology}
\item \hyperref[curves-section-phantom]{Algebraic Curves}
\item \hyperref[resolve-section-phantom]{Resolution of Surfaces}
\item \hyperref[models-section-phantom]{Semistable Reduction}
\item \hyperref[pione-section-phantom]{Fundamental Groups of Schemes}
\item \hyperref[etale-cohomology-section-phantom]{\'Etale Cohomology}
\item \hyperref[ssgroups-section-phantom]{Linear Algebraic Groups}
\item \hyperref[crystalline-section-phantom]{Crystalline Cohomology}
\item \hyperref[proetale-section-phantom]{Pro-\'etale Cohomology}
\end{enumerate}
Algebraic Spaces
\begin{enumerate}
\setcounter{enumi}{56}
\item \hyperref[spaces-section-phantom]{Algebraic Spaces}
\item \hyperref[spaces-properties-section-phantom]{Properties of Algebraic Spaces}
\item \hyperref[spaces-morphisms-section-phantom]{Morphisms of Algebraic Spaces}
\item \hyperref[decent-spaces-section-phantom]{Decent Algebraic Spaces}
\item \hyperref[spaces-cohomology-section-phantom]{Cohomology of Algebraic Spaces}
\item \hyperref[spaces-limits-section-phantom]{Limits of Algebraic Spaces}
\item \hyperref[spaces-divisors-section-phantom]{Divisors on Algebraic Spaces}
\item \hyperref[spaces-over-fields-section-phantom]{Algebraic Spaces over Fields}
\item \hyperref[spaces-topologies-section-phantom]{Topologies on Algebraic Spaces}
\item \hyperref[spaces-descent-section-phantom]{Descent and Algebraic Spaces}
\item \hyperref[spaces-perfect-section-phantom]{Derived Categories of Spaces}
\item \hyperref[spaces-more-morphisms-section-phantom]{More on Morphisms of Spaces}
\item \hyperref[spaces-flat-section-phantom]{Flatness on Algebraic Spaces}
\item \hyperref[spaces-groupoids-section-phantom]{Groupoids in Algebraic Spaces}
\item \hyperref[spaces-more-groupoids-section-phantom]{More on Groupoids in Spaces}
\item \hyperref[bootstrap-section-phantom]{Bootstrap}
\item \hyperref[spaces-pushouts-section-phantom]{Pushouts of Algebraic Spaces}
\end{enumerate}
Topics in Geometry
\begin{enumerate}
\setcounter{enumi}{73}
\item \hyperref[spaces-chow-section-phantom]{Chow Groups of Spaces}
\item \hyperref[groupoids-quotients-section-phantom]{Quotients of Groupoids}
\item \hyperref[spaces-more-cohomology-section-phantom]{More on Cohomology of Spaces}
\item \hyperref[spaces-simplicial-section-phantom]{Simplicial Spaces}
\item \hyperref[spaces-duality-section-phantom]{Duality for Spaces}
\item \hyperref[formal-spaces-section-phantom]{Formal Algebraic Spaces}
\item \hyperref[restricted-section-phantom]{Restricted Power Series}
\item \hyperref[spaces-resolve-section-phantom]{Resolution of Surfaces Revisited}
\end{enumerate}
Deformation Theory
\begin{enumerate}
\setcounter{enumi}{81}
\item \hyperref[formal-defos-section-phantom]{Formal Deformation Theory}
\item \hyperref[defos-section-phantom]{Deformation Theory}
\item \hyperref[cotangent-section-phantom]{The Cotangent Complex}
\item \hyperref[examples-defos-section-phantom]{Deformation Problems}
\end{enumerate}
Algebraic Stacks
\begin{enumerate}
\setcounter{enumi}{85}
\item \hyperref[algebraic-section-phantom]{Algebraic Stacks}
\item \hyperref[examples-stacks-section-phantom]{Examples of Stacks}
\item \hyperref[stacks-sheaves-section-phantom]{Sheaves on Algebraic Stacks}
\item \hyperref[criteria-section-phantom]{Criteria for Representability}
\item \hyperref[artin-section-phantom]{Artin's Axioms}
\item \hyperref[quot-section-phantom]{Quot and Hilbert Spaces}
\item \hyperref[stacks-properties-section-phantom]{Properties of Algebraic Stacks}
\item \hyperref[stacks-morphisms-section-phantom]{Morphisms of Algebraic Stacks}
\item \hyperref[stacks-limits-section-phantom]{Limits of Algebraic Stacks}
\item \hyperref[stacks-cohomology-section-phantom]{Cohomology of Algebraic Stacks}
\item \hyperref[stacks-perfect-section-phantom]{Derived Categories of Stacks}
\item \hyperref[stacks-introduction-section-phantom]{Introducing Algebraic Stacks}
\item \hyperref[stacks-more-morphisms-section-phantom]{More on Morphisms of Stacks}
\item \hyperref[stacks-geometry-section-phantom]{The Geometry of Stacks}
\end{enumerate}
Topics in Moduli Theory
\begin{enumerate}
\setcounter{enumi}{99}
\item \hyperref[moduli-section-phantom]{Moduli Stacks}
\item \hyperref[moduli-curves-section-phantom]{Moduli of Curves}
\end{enumerate}
Miscellany
\begin{enumerate}
\setcounter{enumi}{101}
\item \hyperref[examples-section-phantom]{Examples}
\item \hyperref[exercises-section-phantom]{Exercises}
\item \hyperref[guide-section-phantom]{Guide to Literature}
\item \hyperref[desirables-section-phantom]{Desirables}
\item \hyperref[coding-section-phantom]{Coding Style}
\item \hyperref[obsolete-section-phantom]{Obsolete}
\item \hyperref[fdl-section-phantom]{GNU Free Documentation License}
\item \hyperref[index-section-phantom]{Auto Generated Index}
\end{enumerate}
\end{multicols}


\bibliography{my}
\bibliographystyle{amsalpha}

\end{document}
