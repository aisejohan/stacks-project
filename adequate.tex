\IfFileExists{stacks-project.cls}{%
\documentclass{stacks-project}
}{%
\documentclass{amsart}
}

% The following AMS packages are automatically loaded with
% the amsart documentclass:
%\usepackage{amsmath}
%\usepackage{amssymb}
%\usepackage{amsthm}

\usepackage{graphicx}

% For dealing with references we use the comment environment
\usepackage{verbatim}
\newenvironment{reference}{\comment}{\endcomment}
%\newenvironment{reference}{}{}
\newenvironment{slogan}{\comment}{\endcomment}
\newenvironment{history}{\comment}{\endcomment}

% For commutative diagrams you can use
% \usepackage{amscd}
\usepackage[all]{xy}

% We use 2cell for 2-commutative diagrams.
\xyoption{2cell}
\UseAllTwocells

% To put source file link in headers.
% Change "template.tex" to "this_filename.tex"
% \usepackage{fancyhdr}
% \pagestyle{fancy}
% \lhead{}
% \chead{}
% \rhead{Source file: \url{template.tex}}
% \lfoot{}
% \cfoot{\thepage}
% \rfoot{}
% \renewcommand{\headrulewidth}{0pt}
% \renewcommand{\footrulewidth}{0pt}
% \renewcommand{\headheight}{12pt}

\usepackage{multicol}

% For cross-file-references
\usepackage{xr-hyper}

% Package for hypertext links:
\usepackage{hyperref}

% For any local file, say "hello.tex" you want to link to please
% use \externaldocument[hello-]{hello}
\externaldocument[introduction-]{introduction}
\externaldocument[conventions-]{conventions}
\externaldocument[sets-]{sets}
\externaldocument[categories-]{categories}
\externaldocument[topology-]{topology}
\externaldocument[sheaves-]{sheaves}
\externaldocument[sites-]{sites}
\externaldocument[stacks-]{stacks}
\externaldocument[fields-]{fields}
\externaldocument[algebra-]{algebra}
\externaldocument[brauer-]{brauer}
\externaldocument[homology-]{homology}
\externaldocument[derived-]{derived}
\externaldocument[simplicial-]{simplicial}
\externaldocument[more-algebra-]{more-algebra}
\externaldocument[smoothing-]{smoothing}
\externaldocument[modules-]{modules}
\externaldocument[sites-modules-]{sites-modules}
\externaldocument[injectives-]{injectives}
\externaldocument[cohomology-]{cohomology}
\externaldocument[sites-cohomology-]{sites-cohomology}
\externaldocument[dga-]{dga}
\externaldocument[dpa-]{dpa}
\externaldocument[hypercovering-]{hypercovering}
\externaldocument[schemes-]{schemes}
\externaldocument[constructions-]{constructions}
\externaldocument[properties-]{properties}
\externaldocument[morphisms-]{morphisms}
\externaldocument[coherent-]{coherent}
\externaldocument[divisors-]{divisors}
\externaldocument[limits-]{limits}
\externaldocument[varieties-]{varieties}
\externaldocument[topologies-]{topologies}
\externaldocument[descent-]{descent}
\externaldocument[perfect-]{perfect}
\externaldocument[more-morphisms-]{more-morphisms}
\externaldocument[flat-]{flat}
\externaldocument[groupoids-]{groupoids}
\externaldocument[more-groupoids-]{more-groupoids}
\externaldocument[etale-]{etale}
\externaldocument[chow-]{chow}
\externaldocument[intersection-]{intersection}
\externaldocument[pic-]{pic}
\externaldocument[adequate-]{adequate}
\externaldocument[dualizing-]{dualizing}
\externaldocument[duality-]{duality}
\externaldocument[discriminant-]{discriminant}
\externaldocument[local-cohomology-]{local-cohomology}
\externaldocument[curves-]{curves}
\externaldocument[resolve-]{resolve}
\externaldocument[models-]{models}
\externaldocument[pione-]{pione}
\externaldocument[etale-cohomology-]{etale-cohomology}
\externaldocument[ssgroups-]{ssgroups}
\externaldocument[proetale-]{proetale}
\externaldocument[crystalline-]{crystalline}
\externaldocument[spaces-]{spaces}
\externaldocument[spaces-properties-]{spaces-properties}
\externaldocument[spaces-morphisms-]{spaces-morphisms}
\externaldocument[decent-spaces-]{decent-spaces}
\externaldocument[spaces-cohomology-]{spaces-cohomology}
\externaldocument[spaces-limits-]{spaces-limits}
\externaldocument[spaces-divisors-]{spaces-divisors}
\externaldocument[spaces-over-fields-]{spaces-over-fields}
\externaldocument[spaces-topologies-]{spaces-topologies}
\externaldocument[spaces-descent-]{spaces-descent}
\externaldocument[spaces-perfect-]{spaces-perfect}
\externaldocument[spaces-more-morphisms-]{spaces-more-morphisms}
\externaldocument[spaces-flat-]{spaces-flat}
\externaldocument[spaces-groupoids-]{spaces-groupoids}
\externaldocument[spaces-more-groupoids-]{spaces-more-groupoids}
\externaldocument[bootstrap-]{bootstrap}
\externaldocument[spaces-pushouts-]{spaces-pushouts}
\externaldocument[spaces-chow-]{spaces-chow}
\externaldocument[groupoids-quotients-]{groupoids-quotients}
\externaldocument[spaces-more-cohomology-]{spaces-more-cohomology}
\externaldocument[spaces-simplicial-]{spaces-simplicial}
\externaldocument[spaces-duality-]{spaces-duality}
\externaldocument[formal-spaces-]{formal-spaces}
\externaldocument[restricted-]{restricted}
\externaldocument[spaces-resolve-]{spaces-resolve}
\externaldocument[formal-defos-]{formal-defos}
\externaldocument[defos-]{defos}
\externaldocument[cotangent-]{cotangent}
\externaldocument[examples-defos-]{examples-defos}
\externaldocument[algebraic-]{algebraic}
\externaldocument[examples-stacks-]{examples-stacks}
\externaldocument[stacks-sheaves-]{stacks-sheaves}
\externaldocument[criteria-]{criteria}
\externaldocument[artin-]{artin}
\externaldocument[quot-]{quot}
\externaldocument[stacks-properties-]{stacks-properties}
\externaldocument[stacks-morphisms-]{stacks-morphisms}
\externaldocument[stacks-limits-]{stacks-limits}
\externaldocument[stacks-cohomology-]{stacks-cohomology}
\externaldocument[stacks-perfect-]{stacks-perfect}
\externaldocument[stacks-introduction-]{stacks-introduction}
\externaldocument[stacks-more-morphisms-]{stacks-more-morphisms}
\externaldocument[stacks-geometry-]{stacks-geometry}
\externaldocument[moduli-]{moduli}
\externaldocument[moduli-curves-]{moduli-curves}
\externaldocument[examples-]{examples}
\externaldocument[exercises-]{exercises}
\externaldocument[guide-]{guide}
\externaldocument[desirables-]{desirables}
\externaldocument[coding-]{coding}
\externaldocument[obsolete-]{obsolete}
\externaldocument[fdl-]{fdl}
\externaldocument[index-]{index}

% Theorem environments.
%
\theoremstyle{plain}
\newtheorem{theorem}[subsection]{Theorem}
\newtheorem{proposition}[subsection]{Proposition}
\newtheorem{lemma}[subsection]{Lemma}

\theoremstyle{definition}
\newtheorem{definition}[subsection]{Definition}
\newtheorem{example}[subsection]{Example}
\newtheorem{exercise}[subsection]{Exercise}
\newtheorem{situation}[subsection]{Situation}

\theoremstyle{remark}
\newtheorem{remark}[subsection]{Remark}
\newtheorem{remarks}[subsection]{Remarks}

\numberwithin{equation}{subsection}

% Macros
%
\def\lim{\mathop{\mathrm{lim}}\nolimits}
\def\colim{\mathop{\mathrm{colim}}\nolimits}
\def\Spec{\mathop{\mathrm{Spec}}}
\def\Hom{\mathop{\mathrm{Hom}}\nolimits}
\def\Ext{\mathop{\mathrm{Ext}}\nolimits}
\def\SheafHom{\mathop{\mathcal{H}\!\mathit{om}}\nolimits}
\def\SheafExt{\mathop{\mathcal{E}\!\mathit{xt}}\nolimits}
\def\Sch{\mathit{Sch}}
\def\Mor{\mathop{Mor}\nolimits}
\def\Ob{\mathop{\mathrm{Ob}}\nolimits}
\def\Sh{\mathop{\mathit{Sh}}\nolimits}
\def\NL{\mathop{N\!L}\nolimits}
\def\proetale{{pro\text{-}\acute{e}tale}}
\def\etale{{\acute{e}tale}}
\def\QCoh{\mathit{QCoh}}
\def\Ker{\mathop{\mathrm{Ker}}}
\def\Im{\mathop{\mathrm{Im}}}
\def\Coker{\mathop{\mathrm{Coker}}}
\def\Coim{\mathop{\mathrm{Coim}}}
\def\id{\mathop{\mathrm{id}}\nolimits}

%
% Macros for linear algebraic groups
%
\def\SL{\mathop{\mathrm{SL}}\nolimits}
\def\GL{\mathop{\mathrm{GL}}\nolimits}
\def\ltimes{{\mathchar"256E}}
\def\rtimes{{\mathchar"256F}}
\def\Rrightarrow{{\mathchar"3456}}

%
% Macros for moduli stacks/spaces
%
\def\QCohstack{\mathcal{QC}\!\mathit{oh}}
\def\Cohstack{\mathcal{C}\!\mathit{oh}}
\def\Spacesstack{\mathcal{S}\!\mathit{paces}}
\def\Quotfunctor{\mathrm{Quot}}
\def\Hilbfunctor{\mathrm{Hilb}}
\def\Curvesstack{\mathcal{C}\!\mathit{urves}}
\def\Polarizedstack{\mathcal{P}\!\mathit{olarized}}
\def\Complexesstack{\mathcal{C}\!\mathit{omplexes}}
% \Pic is the operator that assigns to X its picard group, usage \Pic(X)
% \Picardstack_{X/B} denotes the Picard stack of X over B
% \Picardfunctor_{X/B} denotes the Picard functor of X over B
\def\Pic{\mathop{\mathrm{Pic}}\nolimits}
\def\Picardstack{\mathcal{P}\!\mathit{ic}}
\def\Picardfunctor{\mathrm{Pic}}
\def\Deformationcategory{\mathcal{D}\!\mathit{ef}}


% OK, start here.
%
\begin{document}

\title{Adequate Modules}


\maketitle

\phantomsection
\label{section-phantom}

\tableofcontents

\section{Introduction}
\label{section-introduction}

\noindent
For any scheme $X$ the category $\QCoh(\mathcal{O}_X)$
of quasi-coherent modules is abelian and a weak Serre subcategory
of the abelian category of all $\mathcal{O}_X$-modules. The same
thing works for the category of quasi-coherent modules on
an algebraic space $X$ viewed as a subcategory of the category
of all $\mathcal{O}_X$-modules on the small \'etale site of $X$.
Moreover, for a quasi-compact and quasi-separated morphism
$f : X \to Y$ the pushforward $f_*$ and higher direct images
preserve quasi-coherence.

\medskip\noindent
Next, let $X$ be a scheme and let $\mathcal{O}$ be the structure
sheaf on one of the big sites of $X$, say, the big fppf site.
The category of quasi-coherent $\mathcal{O}$-modules is abelian
(in fact it is equivalent to the category of usual quasi-coherent
$\mathcal{O}_X$-modules on the scheme $X$ we mentioned above)
but its imbedding into $\textit{Mod}(\mathcal{O})$ is not exact.
An example is the map of quasi-coherent modules
$$
\mathcal{O}_{\mathbf{A}^1_k}
\longrightarrow
\mathcal{O}_{\mathbf{A}^1_k}
$$
on $\mathbf{A}^1_k = \Spec(k[x])$ given by multiplication by $x$.
In the abelian category of quasi-coherent sheaves this map is injective,
whereas in the abelian category of all $\mathcal{O}$-modules on the
big site of $\mathbf{A}^1_k$ this map has a nontrivial kernel as we
see by evaluating on sections over $\Spec(k[x]/(x)) = \Spec(k)$.
Moreover, for a quasi-compact and quasi-separated morphism
$f : X \to Y$ the functor $f_{big, *}$ does not preserve quasi-coherence.

\medskip\noindent
In this chapter we introduce the category of what we will call
adequate modules, closely related to quasi-coherent modules, which
``fixes'' the two problems mentioned above. Another solution,
which we will implement when we talk about quasi-coherent modules
on algebraic stacks, is to consider $\mathcal{O}$-modules which
are locally quasi-coherent and satisfy the flat base change property.
See Cohomology of Stacks, Section
\ref{stacks-cohomology-section-loc-qcoh-flat-base-change},
Cohomology of Stacks, Remark
\ref{stacks-cohomology-remark-bousfield-colocalization}, and
Derived Categories of Stacks, Section
\ref{stacks-perfect-section-derived}.






\section{Conventions}
\label{section-conventions}

\noindent
In this chapter we fix
$\tau \in \{Zar, \etale, smooth, syntomic, fppf\}$
and we fix a big $\tau$-site $\Sch_\tau$ as in
Topologies, Section \ref{topologies-section-procedure}.
All schemes will be objects of $\Sch_\tau$.
In particular, given a scheme $S$ we obtain sites
$(\textit{Aff}/S)_\tau \subset (\Sch/S)_\tau$.
The structure sheaf $\mathcal{O}$ on these sites is defined by
the rule $\mathcal{O}(T) = \Gamma(T, \mathcal{O}_T)$.

\medskip\noindent
All rings $A$ will be such that $\Spec(A)$ is (isomorphic to) an
object of $\Sch_\tau$. Given a ring $A$ we denote
$\textit{Alg}_A$ the category of $A$-algebras whose objects are the
$A$-algebras $B$ of the form $B = \Gamma(U, \mathcal{O}_U)$
where $U$ is an affine object of $\Sch_\tau$. Thus given an
affine scheme $S = \Spec(A)$ the functor
$$
(\textit{Aff}/S)_\tau \longrightarrow \textit{Alg}_A,
\quad
U \longmapsto \mathcal{O}(U)
$$
is an equivalence.





\section{Adequate functors}
\label{section-quasi-coherent}

\noindent
In this section we discuss a topic closely related to
direct images of quasi-coherent sheaves. Most of this material
was taken from the paper \cite{Jaffe}.

\begin{definition}
\label{definition-module-valued-functor}
Let $A$ be a ring. A {\it module-valued functor} is a functor
$F : \textit{Alg}_A \to \textit{Ab}$ such that
\begin{enumerate}
\item for every object $B$ of $\textit{Alg}_A$ the group
$F(B)$ is endowed with the structure of a $B$-module, and
\item for any morphism $B \to B'$ of $\textit{Alg}_A$ the map
$F(B) \to F(B')$ is $B$-linear.
\end{enumerate}
A {\it morphism of module-valued functors} is a transformation of
functors $\varphi : F \to G$ such that $F(B) \to G(B)$ is $B$-linear
for all $B \in \Ob(\textit{Alg}_A)$.
\end{definition}

\noindent
Let $S = \Spec(A)$ be an affine scheme.
The category of module-valued functors on $\textit{Alg}_A$ is
equivalent to the category
$\textit{PMod}((\textit{Aff}/S)_\tau, \mathcal{O})$
of presheaves of $\mathcal{O}$-modules. The equivalence is given
by the rule which assigns to the module-valued functor $F$ the
presheaf $\mathcal{F}$ defined by the rule
$\mathcal{F}(U) = F(\mathcal{O}(U))$.
This is clear from the equivalence
$(\textit{Aff}/S)_\tau \to \textit{Alg}_A$, $U \mapsto \mathcal{O}(U)$
given in Section \ref{section-conventions}.
The quasi-inverse sets $F(B) = \mathcal{F}(\Spec(B))$.

\medskip\noindent
An important special case of a module-valued functor comes about as follows.
Let $M$ be an $A$-module. Then we will denote $\underline{M}$ the
module-valued functor $B \mapsto M \otimes_A B$ (with obvious $B$-module
structure). Note that if $M \to N$ is a map of $A$-modules then there is an
associated morphism $\underline{M} \to \underline{N}$ of module-valued
functors. Conversely, any morphism of module-valued functors
$\underline{M} \to \underline{N}$ comes from an $A$-module map $M \to N$
as the reader can see by evaluating on $B = A$. In other words
$\text{Mod}_A$ is a full
subcategory of the category of module-valued functors on $\textit{Alg}_A$.

\medskip\noindent
Given and $A$-module map $\varphi : M \to N$ then
$\Coker(\underline{M} \to \underline{N}) =
\underline{Q}$ where $Q = \Coker(M \to N)$ because $\otimes$
is right exact. But this isn't the case
for the kernel in general: for example an injective map of
$A$-modules need not be injective after base change. Thus the following
definition makes sense.

\begin{definition}
\label{definition-adequate-functor}
Let $A$ be a ring. A module-valued functor $F$ on $\textit{Alg}_A$ is
called
\begin{enumerate}
\item {\it adequate} if there exists a
map of $A$-modules $M \to N$ such that $F$ is isomorphic to
$\Ker(\underline{M} \to \underline{N})$.
\item {\it linearly adequate} if $F$ is isomorphic to the
kernel of a map $\underline{A^{\oplus n}} \to \underline{A^{\oplus m}}$.
\end{enumerate}
\end{definition}

\noindent
Note that $F$ is adequate if and only if there exists an
exact sequence $0 \to F \to \underline{M} \to \underline{N}$ and
$F$ is linearly adequate if and only if there exists an exact sequence
$0 \to F \to \underline{A^{\oplus n}} \to \underline{A^{\oplus m}}$.

\medskip\noindent
Let $A$ be a ring. In this section we will show the category of adequate
functors on $\textit{Alg}_A$ is abelian
(Lemmas \ref{lemma-cokernel-adequate} and \ref{lemma-kernel-adequate})
and has a set of generators
(Lemma \ref{lemma-adequate-surjection-from-linear}).
We will also see that it is a weak Serre subcategory of the category
of all module-valued functors on $\textit{Alg}_A$
(Lemma \ref{lemma-extension-adequate})
and that it has arbitrary colimits
(Lemma \ref{lemma-colimit-adequate}).

\begin{lemma}
\label{lemma-adequate-finite-presentation}
Let $A$ be a ring.
Let $F$ be an adequate functor on $\textit{Alg}_A$.
If $B = \colim B_i$ is a filtered
colimit of $A$-algebras, then $F(B) = \colim F(B_i)$.
\end{lemma}

\begin{proof}
This holds because for any $A$-module $M$ we have
$M \otimes_A B = \colim M \otimes_A B_i$ (see
Algebra, Lemma \ref{algebra-lemma-tensor-products-commute-with-limits})
and because filtered colimits commute with exact sequences, see
Algebra, Lemma \ref{algebra-lemma-directed-colimit-exact}.
\end{proof}

\begin{remark}
\label{remark-settheoretic}
Consider the category $\textit{Alg}_{fp, A}$ whose objects are $A$-algebras
$B$ of the form $B = A[x_1, \ldots, x_n]/(f_1, \ldots, f_m)$ and whose
morphisms are $A$-algebra maps. Every $A$-algebra $B$ is a filtered colimit
of finitely presented $A$-algebra, i.e., a filtered colimit of objects of
$\textit{Alg}_{fp, A}$. By
Lemma \ref{lemma-adequate-finite-presentation}
we conclude every adequate functor $F$ is determined by its restriction to
$\textit{Alg}_{fp, A}$. For some questions we can therefore restrict to
functors on $\textit{Alg}_{fp, A}$. For example, the category of adequate
functors does not depend on the choice of the big $\tau$-site
chosen in
Section \ref{section-conventions}.
\end{remark}

\begin{lemma}
\label{lemma-adequate-flat}
Let $A$ be a ring.
Let $F$ be an adequate functor on $\textit{Alg}_A$.
If $B \to B'$ is flat, then $F(B) \otimes_B B' \to F(B')$
is an isomorphism.
\end{lemma}

\begin{proof}
Choose an exact sequence $0 \to F \to \underline{M} \to \underline{N}$.
This gives the diagram
$$
\xymatrix{
0 \ar[r] & F(B) \otimes_B B' \ar[r] \ar[d] &
(M \otimes_A B)\otimes_B B' \ar[r] \ar[d] &
(N \otimes_A B)\otimes_B B' \ar[d] \\
0 \ar[r] & F(B') \ar[r] &
M \otimes_A B' \ar[r] &
N \otimes_A B'
}
$$
where the rows are exact (the top one because $B \to B'$ is flat).
Since the right two vertical arrows are isomorphisms, so is the
left one.
\end{proof}

\begin{lemma}
\label{lemma-adequate-surjection-from-linear}
Let $A$ be a ring.
Let $F$ be an adequate functor on $\textit{Alg}_A$. Then there exists a
surjection $L \to F$ with $L$ a direct sum of linearly adequate functors.
\end{lemma}

\begin{proof}
Choose an exact sequence $0 \to F \to \underline{M} \to \underline{N}$
where $\underline{M} \to \underline{N}$ is given by
$\varphi : M \to N$. By
Lemma \ref{lemma-adequate-finite-presentation}
it suffices to construct $L \to F$ such that $L(B) \to F(B)$ is surjective
for every finitely presented $A$-algebra $B$. Hence it suffices to construct,
given a finitely presented $A$-algebra $B$ and an element $\xi \in F(B)$
a map $L \to F$ with $L$ linearly adequate such that $\xi$ is in the image
of $L(B) \to F(B)$.
(Because there is a set worth of such pairs $(B, \xi)$ up to isomorphism.)

\medskip\noindent
To do this write $\sum_{i = 1, \ldots, n} m_i \otimes b_i$ the image of
$\xi$ in $\underline{M}(B) = M \otimes_A B$. We know that
$\sum \varphi(m_i) \otimes b_i = 0$ in $N \otimes_A B$.
As $N$ is a filtered colimit of finitely presented $A$-modules, we can
find a finitely presented $A$-module $N'$, a commutative diagram
of $A$-modules
$$
\xymatrix{
A^{\oplus n} \ar[r] \ar[d]_{m_1, \ldots, m_n} & N' \ar[d] \\
M \ar[r] & N
}
$$
such that $(b_1, \ldots, b_n)$ maps to zero in $N' \otimes_A B$.
Choose a presentation $A^{\oplus l} \to A^{\oplus k} \to N' \to 0$.
Choose a lift $A^{\oplus n} \to A^{\oplus k}$ of the map
$A^{\oplus n} \to N'$ of the diagram. Then we see that there exist
$(c_1, \ldots, c_l) \in B^{\oplus l}$ such that
$(b_1, \ldots, b_n, c_1, \ldots, c_l)$ maps to zero in $B^{\oplus k}$
under the map $B^{\oplus n} \oplus B^{\oplus l} \to B^{\oplus k}$.
Consider the commutative diagram
$$
\xymatrix{
A^{\oplus n} \oplus A^{\oplus l} \ar[r] \ar[d] & A^{\oplus k} \ar[d] \\
M \ar[r] & N
}
$$
where the left vertical arrow is zero on the summand $A^{\oplus l}$.
Then we see that $L$ equal to the kernel of $\underline{A^{\oplus n + l}}
\to \underline{A^{\oplus k}}$ works because the element
$(b_1, \ldots, b_n, c_1, \ldots, c_l) \in L(B)$ maps to $\xi$.
\end{proof}

\noindent
Consider a graded $A$-algebra $B = \bigoplus_{d \geq 0} B_d$. Then there are
two $A$-algebra maps $p, a : B \to B[t, t^{-1}]$, namely $p : b \mapsto b$ and
$a : b \mapsto t^{\deg(b)} b$ where $b$ is homogeneous. If $F$ is a
module-valued functor on $\textit{Alg}_A$, then we define
\begin{equation}
\label{equation-weight-k}
F(B)^{(k)} = \{\xi \in F(B) \mid t^k F(p)(\xi) = F(a)(\xi)\}.
\end{equation}
For functors which behave well with respect to flat ring extensions
this gives a direct sum decomposition. This amounts to the fact that
representations of $\mathbf{G}_m$ are completely reducible.

\begin{lemma}
\label{lemma-flat-functor-split}
Let $A$ be a ring.
Let $F$ be a module-valued functor on $\textit{Alg}_A$.
Assume that for $B \to B'$ flat the map
$F(B) \otimes_B B' \to F(B')$ is an isomorphism.
Let $B$ be a graded $A$-algebra. Then
\begin{enumerate}
\item $F(B) = \bigoplus_{k \in \mathbf{Z}} F(B)^{(k)}$, and
\item the map $B \to B_0 \to B$ induces map $F(B) \to F(B)$
whose image is contained in $F(B)^{(0)}$.
\end{enumerate}
\end{lemma}

\begin{proof}
Let $x \in F(B)$. The map $p : B \to B[t, t^{-1}]$ is free
hence we know that
$$
F(B[t, t^{-1}]) =
\bigoplus\nolimits_{k \in \mathbf{Z}} F(p)(F(B)) \cdot t^k =
\bigoplus\nolimits_{k \in \mathbf{Z}} F(B) \cdot t^k
$$
as indicated we drop the $F(p)$ in the rest of the proof.
Write $F(a)(x) = \sum t^k x_k$ for some $x_k \in F(B)$.
Denote $\epsilon : B[t, t^{-1}] \to B$
the $B$-algebra map $t \mapsto 1$. Note that the compositions
$\epsilon \circ p, \epsilon \circ a : B \to B[t, t^{-1}] \to B$ are
the identity. Hence we see that
$$
x = F(\epsilon)(F(a)(x)) = F(\epsilon)(\sum t^k x_k) = \sum x_k.
$$
On the other hand, we claim that $x_k \in F(B)^{(k)}$. Namely, consider
the commutative diagram
$$
\xymatrix{
B \ar[r]_a \ar[d]_{a'} &
B[t, t^{-1}] \ar[d]^f \\
B[s, s^{-1}] \ar[r]^-g &
B[t, s, t^{-1}, s^{-1}]
}
$$
where $a'(b) = s^{\deg(b)}b$, $f(b) = b$, $f(t) = st$ and
$g(b) = t^{\deg(b)}b$ and $g(s) = s$. Then
$$
F(g)(F(a'))(x) = F(g)(\sum s^k x_k) =
\sum s^k F(a)(x_k)
$$
and going the other way we see
$$
F(f)(F(a))(x) = F(f)(\sum t^k x_k) = \sum (st)^k x_k.
$$
Since $B \to B[s, t, s^{-1}, t^{-1}]$ is free we see that
$F(B[t, s, t^{-1}, s^{-1}]) =
\bigoplus_{k, l \in \mathbf{Z}} F(B) \cdot t^ks^l$ and
comparing coefficients in the expressions above we find
$F(a)(x_k) = t^k x_k$ as desired.

\medskip\noindent
Finally, the image of $F(B_0) \to F(B)$ is contained in $F(B)^{(0)}$
because $B_0 \to B \xrightarrow{a} B[t, t^{-1}]$ is equal to
$B_0 \to B \xrightarrow{p} B[t, t^{-1}]$.
\end{proof}

\noindent
As a particular case of
Lemma \ref{lemma-flat-functor-split}
note that
$$
\underline{M}(B)^{(k)} = M \otimes_A B_k
$$
where $B_k$ is the degree $k$ part of the graded $A$-algebra $B$.

\begin{lemma}
\label{lemma-lift-map}
Let $A$ be a ring. Given a solid diagram
$$
\xymatrix{
0 \ar[r] &
L \ar[d]_\varphi \ar[r] &
\underline{A^{\oplus n}} \ar[r] \ar@{..>}[ld] &
\underline{A^{\oplus m}} \\
& \underline{M}
}
$$
of module-valued functors on $\textit{Alg}_A$
with exact row there exists a dotted arrow making the diagram commute.
\end{lemma}

\begin{proof}
Suppose that the map $A^{\oplus n} \to A^{\oplus m}$ is given by the
$m \times n$-matrix $(a_{ij})$. Consider the ring
$B = A[x_1, \ldots, x_n]/(\sum a_{ij}x_j)$. The element
$(x_1, \ldots, x_n) \in \underline{A^{\oplus n}}(B)$ maps to zero in
$\underline{A^{\oplus m}}(B)$ hence is the image of a unique element
$\xi \in L(B)$. Note that $\xi$ has the following universal property:
for any $A$-algebra $C$ and any $\xi' \in L(C)$ there exists an $A$-algebra
map $B \to C$ such that $\xi$ maps to $\xi'$ via the map $L(B) \to L(C)$.

\medskip\noindent
Note that $B$ is a graded $A$-algebra, hence we can use
Lemmas \ref{lemma-flat-functor-split} and \ref{lemma-adequate-flat}
to decompose the values of our functors on $B$ into graded pieces.
Note that $\xi \in L(B)^{(1)}$ as $(x_1, \ldots, x_n)$ is an element
of degree one in $\underline{A^{\oplus n}}(B)$. Hence we see that
$\varphi(\xi) \in \underline{M}(B)^{(1)} = M \otimes_A B_1$.
Since $B_1$ is generated by $x_1, \ldots, x_n$ as an $A$-module we
can write $\varphi(\xi) = \sum m_i \otimes x_i$. Consider the map
$A^{\oplus n} \to M$ which maps the $i$th basis vector to $m_i$.
By construction the associated map
$\underline{A^{\oplus n}} \to \underline{M}$
maps the element $\xi$ to $\varphi(\xi)$. It follows from the
universal property mentioned above that the diagram commutes.
\end{proof}

\begin{lemma}
\label{lemma-cokernel-into-module}
Let $A$ be a ring.
Let $\varphi : F \to \underline{M}$ be a map of module-valued functors
on $\textit{Alg}_A$ with $F$ adequate.
Then $\Coker(\varphi)$ is adequate.
\end{lemma}

\begin{proof}
By
Lemma \ref{lemma-adequate-surjection-from-linear}
we may assume that $F = \bigoplus L_i$ is a direct sum of linearly adequate
functors. Choose exact sequences
$0 \to L_i \to \underline{A^{\oplus n_i}} \to \underline{A^{\oplus m_i}}$.
For each $i$ choose a map $A^{\oplus n_i} \to M$ as in
Lemma \ref{lemma-lift-map}.
Consider the diagram
$$
\xymatrix{
0 \ar[r] &
\bigoplus L_i  \ar[r] \ar[d] &
\bigoplus \underline{A^{\oplus n_i}} \ar[r] \ar[ld] &
\bigoplus \underline{A^{\oplus m_i}} \\
& \underline{M}
}
$$
Consider the $A$-modules
$$
Q =
\Coker(\bigoplus A^{\oplus n_i} \to M \oplus \bigoplus A^{\oplus m_i})
\quad\text{and}\quad
P = \Coker(\bigoplus A^{\oplus n_i} \to \bigoplus A^{\oplus m_i}).
$$
Then we see that $\Coker(\varphi)$ is isomorphic to the
kernel of $\underline{Q} \to \underline{P}$.
\end{proof}

\begin{lemma}
\label{lemma-cokernel-adequate}
\begin{slogan}
The cokernel of a map of adequate functors on the category of algebras
over a ring is adequate.
\end{slogan}
Let $A$ be a ring.
Let $\varphi : F \to G$ be a map of adequate functors on $\textit{Alg}_A$.
Then $\Coker(\varphi)$ is adequate.
\end{lemma}

\begin{proof}
Choose an injection $G \to \underline{M}$.
Then we have an injection $G/F \to \underline{M}/F$. By
Lemma \ref{lemma-cokernel-into-module}
we see that $\underline{M}/F$ is adequate, hence we can find an injection
$\underline{M}/F \to \underline{N}$.
Composing we obtain an injection $G/F \to \underline{N}$. By
Lemma \ref{lemma-cokernel-into-module}
the cokernel of the induced map $G \to \underline{N}$ is adequate
hence we can find an injection $\underline{N}/G \to \underline{K}$.
Then $0 \to G/F \to \underline{N} \to \underline{K}$ is exact and
we win.
\end{proof}

\begin{lemma}
\label{lemma-kernel-adequate}
Let $A$ be a ring.
Let $\varphi : F \to G$ be a map of adequate functors on $\textit{Alg}_A$.
Then $\Ker(\varphi)$ is adequate.
\end{lemma}

\begin{proof}
Choose an injection $F \to \underline{M}$ and an injection
$G \to \underline{N}$. Denote $F \to \underline{M \oplus N}$
the diagonal map so that
$$
\xymatrix{
F \ar[d] \ar[r] & G \ar[d] \\
\underline{M \oplus N} \ar[r] & \underline{N}
}
$$
commutes. By
Lemma \ref{lemma-cokernel-adequate}
we can find a module map $M \oplus N \to K$ such that
$F$ is the kernel of $\underline{M \oplus N} \to \underline{K}$.
Then $\Ker(\varphi)$ is the kernel of
$\underline{M \oplus N} \to \underline{K \oplus N}$.
\end{proof}

\begin{lemma}
\label{lemma-colimit-adequate}
Let $A$ be a ring.
An arbitrary direct sum of adequate functors on $\textit{Alg}_A$
is adequate. A colimit of adequate functors is adequate.
\end{lemma}

\begin{proof}
The statement on direct sums is immediate.
A general colimit can be written as a kernel of a map between
direct sums, see
Categories, Lemma \ref{categories-lemma-colimits-coproducts-coequalizers}.
Hence this follows from
Lemma \ref{lemma-kernel-adequate}.
\end{proof}

\begin{lemma}
\label{lemma-flat-linear-functor}
Let $A$ be a ring.
Let $F, G$ be module-valued functors on $\textit{Alg}_A$.
Let $\varphi : F \to G$ be a transformation of functors. Assume
\begin{enumerate}
\item $\varphi$ is additive,
\item for every $A$-algebra $B$ and $\xi \in F(B)$ and unit
$u \in B^*$ we have $\varphi(u\xi) = u\varphi(\xi)$ in $G(B)$, and
\item for any flat ring map $B \to B'$ we have
$G(B) \otimes_B B' = G(B')$.
\end{enumerate}
Then $\varphi$ is a morphism of module-valued functors.
\end{lemma}

\begin{proof}
Let $B$ be an $A$-algebra, $\xi \in F(B)$, and $b \in B$. We have to show
that $\varphi(b \xi) = b \varphi(\xi)$. Consider the ring map
$$
B \to B' = B[x, y, x^{-1}, y^{-1}]/(x + y - b).
$$
This ring map is faithfully flat, hence $G(B) \subset G(B')$. On the
other hand
$$
\varphi(b\xi) = \varphi((x + y)\xi) =
\varphi(x\xi) + \varphi(y\xi) = x\varphi(\xi) + y\varphi(\xi)
= (x + y)\varphi(\xi) = b\varphi(\xi)
$$
because $x, y$ are units in $B'$. Hence we win.
\end{proof}

\begin{lemma}
\label{lemma-extension-adequate-key}
Let $A$ be a ring.
Let $0 \to \underline{M} \to G \to L \to 0$ be a short exact sequence
of module-valued functors on $\textit{Alg}_A$ with $L$ linearly adequate.
Then $G$ is adequate.
\end{lemma}

\begin{proof}
We first point out that for any flat $A$-algebra map
$B \to B'$ the map $G(B) \otimes_B B' \to G(B')$ is an isomorphism.
Namely, this holds for $\underline{M}$ and $L$, see
Lemma \ref{lemma-adequate-flat}
and hence follows for $G$ by the five lemma. In particular, by
Lemma \ref{lemma-flat-functor-split}
we see that $G(B) = \bigoplus_{k \in \mathbf{Z}} G(B)^{(k)}$
for any graded $A$-algebra $B$.

\medskip\noindent
Choose an exact sequence
$0 \to L \to \underline{A^{\oplus n}} \to \underline{A^{\oplus m}}$.
Suppose that the map $A^{\oplus n} \to A^{\oplus m}$ is given by the
$m \times n$-matrix $(a_{ij})$. Consider the graded $A$-algebra
$B = A[x_1, \ldots, x_n]/(\sum a_{ij}x_j)$. The element
$(x_1, \ldots, x_n) \in \underline{A^{\oplus n}}(B)$ maps to zero in
$\underline{A^{\oplus m}}(B)$ hence is the image of a unique element
$\xi \in L(B)$. Observe that $\xi \in L(B)^{(1)}$. The map
$$
\Hom_A(B, C) \longrightarrow L(C), \quad
f \longmapsto L(f)(\xi)
$$
defines an isomorphism of functors. The reason is that $f$ is
determined by the images $c_i = f(x_i) \in C$ which have to
satisfy the relations $\sum a_{ij}c_j = 0$. And $L(C)$ is the
set of $n$-tuples $(c_1, \ldots, c_n)$ satisfying the relations
$\sum a_{ij} c_j = 0$.

\medskip\noindent
Since the value of each of the functors $\underline{M}$, $G$, $L$
on $B$ is a direct sum of its weight spaces (by the lemma mentioned
above) exactness of $0 \to \underline{M} \to G \to L \to 0$ implies
the sequence $0 \to \underline{M}(B)^{(1)} \to G(B)^{(1)} \to L(B)^{(1)} \to 0$
is exact. Thus we may choose an element $\theta \in G(B)^{(1)}$ mapping
to $\xi$.

\medskip\noindent
Consider the graded $A$-algebra
$$
C = A[x_1, \ldots, x_n, y_1, \ldots, y_n]/
(\sum a_{ij}x_j, \sum a_{ij}y_j)
$$
There are three graded $A$-algebra homomorphisms $p_1, p_2, m : B \to C$
defined by the rules
$$
p_1(x_i) = x_i, \quad
p_1(x_i) = y_i, \quad
m(x_i) = x_i + y_i.
$$
We will show that the element
$$
\tau = G(m)(\theta) - G(p_1)(\theta) - G(p_2)(\theta) \in G(C)
$$
is zero. First, $\tau$ maps to zero in $L(C)$ by a direct calculation.
Hence $\tau$ is an element of $\underline{M}(C)$.
Moreover, since $m$, $p_1$, $p_2$ are graded algebra maps we see
that $\tau \in G(C)^{(1)}$ and since $\underline{M} \subset G$
we conclude
$$
\tau \in \underline{M}(C)^{(1)} = M \otimes_A C_1.
$$
We may write uniquely
$\tau = \underline{M}(p_1)(\tau_1) + \underline{M}(p_2)(\tau_2)$ with
$\tau_i \in M \otimes_A B_1 = \underline{M}(B)^{(1)}$ because
$C_1 = p_1(B_1) \oplus p_2(B_1)$.
Consider the ring map $q_1 : C \to B$ defined by $x_i \mapsto x_i$ and
$y_i \mapsto 0$. Then
$\underline{M}(q_1)(\tau) =
\underline{M}(q_1)(\underline{M}(p_1)(\tau_1) + \underline{M}(p_2)(\tau_2)) =
\tau_1$.
On the other hand, because
$q_1 \circ m = q_1 \circ p_1$ we see that
$G(q_1)(\tau) = - G(q_1 \circ p_2)(\tau)$. Since $q_1 \circ p_2$ factors as
$B \to A \to B$ we see that $G(q_1 \circ p_2)(\tau)$ is in
$G(B)^{(0)}$, see
Lemma \ref{lemma-flat-functor-split}.
Hence $\tau_1 = 0$ because it is in
$G(B)^{(0)} \cap \underline{M}(B)^{(1)} \subset
G(B)^{(0)} \cap G(B)^{(1)} = 0$.
Similarly $\tau_2 = 0$, whence $\tau = 0$.

\medskip\noindent
Since $\theta \in G(B)$ we obtain a transformation of functors
$$
\psi : L(-) = \Hom_A(B, - ) \longrightarrow G(-)
$$
by mapping $f : B \to C$ to $G(f)(\theta)$. Since $\theta$ is a lift of
$\xi$ the map $\psi$ is a right inverse of $G \to L$. In terms of
$\psi$ the statements proved above have the following meaning:
$\tau = 0$ means that $\psi$ is additive and
$\theta \in G(B)^{(1)}$ implies that for any $A$-algebra $D$ we have
$\psi(ul) = u\psi(l)$ in $G(D)$ for $l \in L(D)$ and $u \in D^*$ a unit.
This implies that $\psi$ is a morphism of module-valued functors, see
Lemma \ref{lemma-flat-linear-functor}.
Clearly this implies that $G \cong \underline{M} \oplus L$ and we win.
\end{proof}

\begin{remark}
\label{remark-linearly-adequate}
Let $A$ be a ring.
The proof of
Lemma \ref{lemma-extension-adequate-key}
shows that any extension $0 \to \underline{M} \to E \to L \to 0$
of module-valued functors on $\textit{Alg}_A$
with $L$ linearly adequate splits. It uses only the following properties
of the module-valued functor $F = \underline{M}$:
\begin{enumerate}
\item $F(B) \otimes_B B' \to F(B')$ is an isomorphism
for a flat ring map $B \to B'$, and
\item
$F(C)^{(1)} = F(p_1)(F(B)^{(1)}) \oplus F(p_2)(F(B)^{(1)})$
where $B = A[x_1, \ldots, x_n]/(\sum a_{ij}x_j)$ and
$C = A[x_1, \ldots, x_n, y_1, \ldots, y_n]/
(\sum a_{ij}x_j, \sum a_{ij}y_j)$.
\end{enumerate}
These two properties hold for any adequate functor $F$; details omitted.
Hence we see that $L$ is a projective object of the abelian category of
adequate functors.
\end{remark}

\begin{lemma}
\label{lemma-extension-adequate}
Let $A$ be a ring.
Let $0 \to F \to G \to H \to 0$ be a short exact sequence of
module-valued functors on $\textit{Alg}_A$.
If $F$ and $H$ are adequate, so is $G$.
\end{lemma}

\begin{proof}
Choose an exact sequence $0 \to F \to \underline{M} \to \underline{N}$.
If we can show that $(\underline{M} \oplus G)/F$ is adequate, then
$G$ is the kernel of the map of adequate functors
$(\underline{M} \oplus G)/F \to \underline{N}$, hence
adequate by
Lemma \ref{lemma-kernel-adequate}.
Thus we may assume $F = \underline{M}$.

\medskip\noindent
We can choose a surjection $L \to H$ where $L$ is a direct sum of
linearly adequate functors, see
Lemma \ref{lemma-adequate-surjection-from-linear}.
If we can show that the pullback $G \times_H L$ is adequate, then
$G$ is the cokernel of the map $\Ker(L \to H) \to G \times_H L$
hence adequate by
Lemma \ref{lemma-cokernel-adequate}.
Thus we may assume that $H = \bigoplus L_i$ is a direct sum of
linearly adequate functors. By
Lemma \ref{lemma-extension-adequate-key}
each of the pullbacks $G \times_H L_i$ is adequate. By
Lemma \ref{lemma-colimit-adequate}
we see that $\bigoplus G \times_H L_i$ is adequate.
Then $G$ is the cokernel of
$$
\bigoplus\nolimits_{i \not = i'} F \longrightarrow
\bigoplus G \times_H L_i
$$
where $\xi$ in the summand $(i, i')$ maps to
$(0, \ldots, 0, \xi, 0, \ldots, 0, -\xi, 0, \ldots, 0)$
with nonzero entries in the summands $i$ and $i'$.
Thus $G$ is adequate by
Lemma \ref{lemma-cokernel-adequate}.
\end{proof}

\begin{lemma}
\label{lemma-base-change-adequate}
Let $A \to A'$ be a ring map. If $F$ is an adequate functor on
$\textit{Alg}_A$, then its restriction $F'$ to
$\textit{Alg}_{A'}$ is adequate too.
\end{lemma}

\begin{proof}
Choose an exact sequence $0 \to F \to \underline{M} \to \underline{N}$.
Then $F'(B') = F(B') = \Ker(M \otimes_A B' \to N \otimes_A B')$.
Since $M \otimes_A B' = M \otimes_A A' \otimes_{A'} B'$ and similarly
for $N$ we see that $F'$ is the kernel of
$\underline{M \otimes_A A'} \to \underline{N \otimes_A A'}$.
\end{proof}

\begin{lemma}
\label{lemma-pushforward-adequate}
Let $A \to A'$ be a ring map. If $F'$ is an adequate functor on
$\textit{Alg}_{A'}$, then the module-valued functor
$F : B \mapsto F'(A' \otimes_A B)$ on $\textit{Alg}_A$ is adequate too.
\end{lemma}

\begin{proof}
Choose an exact sequence $0 \to F' \to \underline{M'} \to \underline{N'}$.
Then
\begin{align*}
F(B) & = F'(A' \otimes_A B) \\
& = \Ker(M' \otimes_{A'} (
A' \otimes_A B) \to N' \otimes_{A'} (A' \otimes_A B)) \\
& = \Ker(M' \otimes_A B \to N' \otimes_A B)
\end{align*}
Thus $F$ is the kernel of
$\underline{M} \to \underline{N}$
where $M = M'$ and $N = N'$ viewed as $A$-modules.
\end{proof}

\begin{lemma}
\label{lemma-adequate-product}
Let $A = A_1 \times \ldots \times A_n$ be a product of rings.
An adequate functor over $A$ is the same thing as a sequence
$F_1, \ldots, F_n$ of adequate functors $F_i$ over $A_i$.
\end{lemma}

\begin{proof}
This is true because an $A$-algebra $B$ is canonically a product
$B_1 \times \ldots \times B_n$ and the same thing holds for $A$-modules.
Setting $F(B) = \coprod F_i(B_i)$ gives the correspondence.
Details omitted.
\end{proof}

\begin{lemma}
\label{lemma-adequate-descent}
Let $A \to A'$ be a ring map and let $F$ be a module-valued functor on
$\textit{Alg}_A$ such that
\begin{enumerate}
\item the restriction $F'$ of $F$ to the category of $A'$-algebras is
adequate, and
\item for any $A$-algebra $B$ the sequence
$$
0 \to F(B) \to F(B \otimes_A A') \to F(B \otimes_A A' \otimes_A A')
$$
is exact.
\end{enumerate}
Then $F$ is adequate.
\end{lemma}

\begin{proof}
The functors $B \to F(B \otimes_A A')$ and
$B \mapsto F(B \otimes_A A' \otimes_A A')$ are adequate, see
Lemmas \ref{lemma-pushforward-adequate} and
\ref{lemma-base-change-adequate}.
Hence $F$ as a kernel of a map of adequate functors is adequate, see
Lemma \ref{lemma-kernel-adequate}.
\end{proof}






\section{Higher exts of adequate functors}
\label{section-higher-ext}

\noindent
Let $A$ be a ring. In
Lemma \ref{lemma-extension-adequate}
we have seen that any extension of adequate functors in the category
of module-valued functors on $\textit{Alg}_A$ is adequate. In this
section we show that the same remains true for higher ext groups.

\begin{lemma}
\label{lemma-adjoint}
Let $A$ be a ring.
For every module-valued functor $F$ on $\textit{Alg}_A$
there exists a morphism $Q(F) \to F$ of module-valued functors on
$\textit{Alg}_A$ such that (1) $Q(F)$ is adequate and (2) for every
adequate functor $G$ the map $\Hom(G, Q(F)) \to \Hom(G, F)$
is a bijection.
\end{lemma}

\begin{proof}
Choose a set $\{L_i\}_{i \in I}$ of linearly adequate functors such that
every linearly adequate functor is isomorphic to one of the $L_i$.
This is possible. Suppose that we can find $Q(F) \to F$ with (1) and
(2)' or every $i \in I$ the map $\Hom(L_i, Q(F)) \to \Hom(L_i, F)$
is a bijection. Then (2) holds. Namely, combining
Lemmas \ref{lemma-adequate-surjection-from-linear} and
\ref{lemma-kernel-adequate}
we see that every adequate functor $G$ sits in an exact sequence
$$
K \to L \to G \to 0
$$
with $K$ and $L$ direct sums of linearly adequate functors. Hence (2)'
implies that
$\Hom(L, Q(F)) \to \Hom(L, F)$
and
$\Hom(K, Q(F)) \to \Hom(K, F)$
are bijections, whence the same thing for $G$.

\medskip\noindent
Consider the category $\mathcal{I}$ whose objects are pairs
$(i, \varphi)$ where $i \in I$ and $\varphi : L_i \to F$ is a morphism.
A morphism $(i, \varphi) \to (i', \varphi')$ is a map
$\psi : L_i \to L_{i'}$ such that $\varphi' \circ \psi = \varphi$.
Set
$$
Q(F) = \colim_{(i, \varphi) \in \Ob(\mathcal{I})} L_i
$$
There is a natural map $Q(F) \to F$, by
Lemma \ref{lemma-colimit-adequate}
it is adequate, and by construction it has property (2)'.
\end{proof}

\begin{lemma}
\label{lemma-enough-injectives}
Let $A$ be a ring. Denote $\mathcal{P}$ the category of module-valued
functors on $\textit{Alg}_A$ and $\mathcal{A}$ the category of adequate
functors on $\textit{Alg}_A$. Denote $i : \mathcal{A} \to \mathcal{P}$
the inclusion functor. Denote $Q : \mathcal{P} \to \mathcal{A}$
the construction of Lemma \ref{lemma-adjoint}.
Then
\begin{enumerate}
\item $i$ is fully faithful, exact, and its image is a weak Serre subcategory,
\item $\mathcal{P}$ has enough injectives,
\item the functor $Q$ is a right adjoint to $i$ hence left exact,
\item $Q$ transforms injectives into injectives,
\item $\mathcal{A}$ has enough injectives.
\end{enumerate}
\end{lemma}

\begin{proof}
This lemma just collects some facts we have already seen so far.
Part (1) is clear from the definitions, the characterization of
weak Serre subcategories (see
Homology, Lemma \ref{homology-lemma-characterize-weak-serre-subcategory}),
and
Lemmas \ref{lemma-cokernel-adequate}, \ref{lemma-kernel-adequate},
and \ref{lemma-extension-adequate}.
Recall that $\mathcal{P}$ is equivalent to the category
$\textit{PMod}((\textit{Aff}/\Spec(A))_\tau, \mathcal{O})$.
Hence (2) by
Injectives, Proposition \ref{injectives-proposition-presheaves-modules}.
Part (3) follows from
Lemma \ref{lemma-adjoint}
and
Categories, Lemma \ref{categories-lemma-adjoint-exact}.
Parts (4) and (5) follow from
Homology, Lemmas \ref{homology-lemma-adjoint-preserve-injectives} and
\ref{homology-lemma-adjoint-enough-injectives}.
\end{proof}

\noindent
Let $A$ be a ring. As in
Formal Deformation Theory, Section
\ref{formal-defos-section-tangent-spaces-functors}
given an $A$-algebra $B$ and an $B$-module $N$ we set $B[N]$ equal to
the $R$-algebra with underlying $B$-module $B \oplus N$ with multiplication
given by $(b, m)(b', m ') = (bb', bm' + b'm)$. Note that this construction
is functorial in the pair $(B, N)$ where morphism $(B, N) \to (B', N')$
is given by an $A$-algebra map $B \to B'$ and an $B$-module map
$N \to N'$. In some sense the functor $TF$ of pairs defined in the following
lemma is the tangent space of $F$.
Below we will only consider pairs $(B, N)$ such that
$B[N]$ is an object of $\textit{Alg}_A$.

\begin{lemma}
\label{lemma-tangent-functor}
Let $A$ be a ring. Let $F$ be a module valued functor.
For every $B \in \Ob(\textit{Alg}_A)$ and $B$-module $N$
there is a canonical decomposition
$$
F(B[N]) = F(B) \oplus TF(B, N)
$$
characterized by the following properties
\begin{enumerate}
\item $TF(B, N) = \Ker(F(B[N]) \to F(B))$,
\item there is a $B$-module structure $TF(B, N)$
compatible with $B[N]$-module structure on $F(B[N])$,
\item $TF$ is a functor from the category of pairs $(B, N)$,
\item
\label{item-mult-map}
there are canonical maps $N \otimes_B F(B) \to TF(B, N)$
inducing a transformation between functors defined on the category
of pairs $(B, N)$,
\item $TF(B, 0) = 0$ and the map $TF(B, N) \to TF(B, N')$ is
zero when $N \to N'$ is the zero map.
\end{enumerate}
\end{lemma}

\begin{proof}
Since $B \to B[N] \to B$ is the identity we see that $F(B) \to F(B[N])$
is a direct summand whose complement is $TF(N, B)$ as defined in (1).
This construction is functorial in the pair $(B, N)$ simply because
given a morphism of pairs $(B, N) \to (B', N')$ we obtain a commutative
diagram
$$
\xymatrix{
B' \ar[r] & B'[N'] \ar[r] & B' \\
B \ar[r] \ar[u] & B[N] \ar[r] \ar[u] & B \ar[u]
}
$$
in $\textit{Alg}_A$. The $B$-module structure comes from the $B[N]$-module
structure and the ring map $B \to B[N]$. The map in (4) is the
composition
$$
N \otimes_B F(B) \longrightarrow
B[N] \otimes_{B[N]} F(B[N]) \longrightarrow F(B[N])
$$
whose image is contained in $TF(B, N)$. (The first arrow uses the inclusions
$N \to B[N]$ and $F(B) \to F(B[N])$ and the second arrow is the multiplication
map.) If $N = 0$, then $B = B[N]$
hence $TF(B, 0) = 0$. If $N \to N'$ is zero then it factors as
$N \to 0 \to N'$ hence the induced map is zero since $TF(B, 0) = 0$.
\end{proof}

\noindent
Let $A$ be a ring. Let $M$ be an $A$-module. Then the module-valued functor
$\underline{M}$ has tangent space $T\underline{M}$ given by the rule
$T\underline{M}(B, N) = N \otimes_A M$. In particular, for $B$ given, the
functor $N \mapsto T\underline{M}(B, N)$ is additive and right exact. It turns
out this also holds for injective module-valued functors.

\begin{lemma}
\label{lemma-tangent-injective}
Let $A$ be a ring. Let $I$ be an injective object of the category
of module-valued functors. Then for any $B \in \Ob(\textit{Alg}_A)$
and short exact sequence
$0 \to N_1 \to N \to N_2 \to 0$
of $B$-modules the sequence
$$
TI(B, N_1) \to TI(B, N) \to TI(B, N_2) \to 0
$$
is exact.
\end{lemma}

\begin{proof}
We will use the results of
Lemma \ref{lemma-tangent-functor}
without further mention.
Denote $h : \textit{Alg}_A \to \textit{Sets}$ the functor given by
$h(C) = \Mor_A(B[N], C)$. Similarly for $h_1$ and $h_2$.
The map $B[N] \to B[N_2]$ corresponding to the surjection $N \to N_2$
is surjective. It corresponds to a map $h_2 \to h$ such that
$h_2(C) \to h(C)$ is injective for all $A$-algebras $C$. On the other
hand, there are two maps $p, q : h \to h_1$, corresponding to the
zero map $N_1 \to N$ and the injection $N_1 \to N$. Note that
$$
\xymatrix{
h_2 \ar[r] & h \ar@<1ex>[r] \ar@<-1ex>[r] & h_1
}
$$
is an equalizer diagram. Denote $\mathcal{O}_h$ the module-valued functor
$C \mapsto \bigoplus_{h(C)} C$. Similarly for $\mathcal{O}_{h_1}$ and
$\mathcal{O}_{h_2}$. Note that
$$
\Hom_\mathcal{P}(\mathcal{O}_h, F) = F(B[N])
$$
where $\mathcal{P}$ is the category of module-valued functors on
$\textit{Alg}_A$. We claim there is an equalizer diagram
$$
\xymatrix{
\mathcal{O}_{h_2} \ar[r] &
\mathcal{O}_h \ar@<1ex>[r] \ar@<-1ex>[r] &
\mathcal{O}_{h_1}
}
$$
in $\mathcal{P}$. Namely, suppose that $C \in \Ob(\textit{Alg}_A)$
and $\xi = \sum_{i = 1, \ldots, n} c_i \cdot f_i$ where $c_i \in C$ and
$f_i : B[N] \to C$ is an element of
$\mathcal{O}_h(C)$. If $p(\xi) = q(\xi)$, then
we see that
$$
\sum c_i \cdot f_i \circ z = \sum c_i \cdot f_i \circ y
$$
where $z, y : B[N_1] \to B[N]$ are the maps $z : (b, m_1) \mapsto (b, 0)$
and $y : (b, m_1) \mapsto (b, m_1)$. This means that for every $i$
there exists a $j$ such that $f_j \circ z = f_i \circ y$. Clearly, this
implies that $f_i(N_1) = 0$, i.e., $f_i$ factors through a unique map
$\overline{f}_i : B[N_2] \to C$. Hence $\xi$ is the image of
$\overline{\xi} = \sum c_i \cdot \overline{f}_i$.
Since $I$ is injective, it transforms this equalizer diagram
into a coequalizer diagram
$$
\xymatrix{
I(B[N_1]) \ar@<1ex>[r] \ar@<-1ex>[r] &
I(B[N]) \ar[r] &
I(B[N_2])
}
$$
This diagram is compatible with the direct sum decompositions
$I(B[N]) = I(B) \oplus TI(B, N)$ and $I(B[N_i]) = I(B) \oplus TI(B, N_i)$.
The zero map $N \to N_1$ induces the zero map $TI(B, N) \to TI(B, N_1)$.
Thus we see that the coequalizer property
above means we have an exact sequence
$TI(B, N_1) \to TI(B, N) \to TI(B, N_2) \to 0$
as desired.
\end{proof}

\begin{lemma}
\label{lemma-exactness-implies}
Let $A$ be a ring. Let $F$ be a module-valued functor
such that for any $B \in \Ob(\textit{Alg}_A)$ the
functor $TF(B, -)$ on $B$-modules transforms a short exact sequence
of $B$-modules into a right exact sequence. Then
\begin{enumerate}
\item $TF(B, N_1 \oplus N_2) = TF(B, N_1) \oplus TF(B, N_2)$,
\item there is a second functorial $B$-module structure on $TF(B, N)$
defined by setting $x \cdot b = TF(B, b\cdot 1_N)(x)$ for $x \in TF(B, N)$
and $b \in B$,
\item
\label{item-mult-map-linear}
the canonical map $N \otimes_B F(B) \to TF(B, N)$ of
Lemma \ref{lemma-tangent-functor}
is $B$-linear also with respect to the second $B$-module structure,
\item
\label{item-tangent-right-exact}
given a finitely presented $B$-module $N$ there is a canonical
isomorphism $TF(B, B) \otimes_B N \to TF(B, N)$ where the tensor
product uses the second $B$-module structure on $TF(B, B)$.
\end{enumerate}
\end{lemma}

\begin{proof}
We will use the results of
Lemma \ref{lemma-tangent-functor}
without further mention.
The maps $N_1 \to N_1 \oplus N_2$ and $N_2 \to N_1 \oplus N_2$ give
a map $TF(B, N_1) \oplus TF(B, N_2) \to TF(B, N_1 \oplus N_2)$
which is injective since the maps $N_1 \oplus N_2 \to N_1$ and
$N_1 \oplus N_2 \to N_2$ induce an inverse.
Since $TF$ is right exact we see that
$TF(B, N_1) \to TF(B, N_1 \oplus N_2) \to TF(B, N_2) \to 0$ is exact.
Hence $TF(B, N_1) \oplus TF(B, N_2) \to TF(B, N_1 \oplus N_2)$ is an
isomorphism. This proves (1).

\medskip\noindent
To see (2) the only thing we need to show is that
$x \cdot (b_1 + b_2) = x \cdot b_1 + x \cdot b_2$.
(Associativity and additivity are clear.) To see this consider
$$
N \xrightarrow{(b_1, b_2)} N \oplus N \xrightarrow{+} N
$$
and apply $TF(B, -)$.

\medskip\noindent
Part (3) follows immediately from the fact that
$N \otimes_B F(B) \to TF(B, N)$ is functorial in the pair $(B, N)$.

\medskip\noindent
Suppose $N$ is a finitely presented $B$-module. Choose a presentation
$B^{\oplus m} \to B^{\oplus n} \to N \to 0$. This gives an exact
sequence
$$
TF(B, B^{\oplus m}) \to TF(B, B^{\oplus n}) \to TF(B, N) \to 0
$$
by right exactness of $TF(B, -)$. By part (1) we can write
$TF(B, B^{\oplus m}) = TF(B, B)^{\oplus m}$ and
$TF(B, B^{\oplus n}) = TF(B, B)^{\oplus n}$. Next, suppose that
$B^{\oplus m} \to B^{\oplus n}$ is given by the matrix $T = (b_{ij})$.
Then the induced map $TF(B, B)^{\oplus m} \to TF(B, B)^{\oplus n}$
is given by the matrix with entries $TF(B, b_{ij} \cdot 1_B)$.
This combined with right exactness of $\otimes$ proves (4).
\end{proof}

\begin{example}
\label{example-module-structure-different}
Let $F$ be a module-valued functor as in
Lemma \ref{lemma-exactness-implies}.
It is not always the case that the two module structures on
$TF(B, N)$ agree. Here is an example. Suppose $A = \mathbf{F}_p$
where $p$ is a prime. Set $F(B) = B$ but with $B$-module structure
given by $b \cdot x = b^px$. Then $TF(B, N) = N$ with $B$-module structure
given by $b \cdot x = b^px$ for $x \in N$. However, the second $B$-module
structure is given by $x \cdot b = bx$. Note that in this case the canonical
map $N \otimes_B F(B) \to TF(B, N)$ is zero as raising an element
$n \in B[N]$ to the $p$th power is zero.
\end{example}

\noindent
In the following lemma we will frequently use the observation that
if $0 \to F \to G \to H \to 0$ is an exact sequence of module-valued
functors on $\textit{Alg}_A$, then for any pair $(B, N)$ the
sequence $0 \to TF(B, N) \to TG(B, N) \to TH(B, N) \to 0$ is exact.
This follows from the fact that $0 \to F(B[N]) \to G(B[N]) \to H(B[N]) \to 0$
is exact.

\begin{lemma}
\label{lemma-exactness-permanence}
Let $A$ be a ring. For $F$ a module-valued functor on $\textit{Alg}_A$
say $(*)$ holds if for all $B \in \Ob(\textit{Alg}_A)$ the
functor $TF(B, -)$ on $B$-modules transforms a short exact sequence
of $B$-modules into a right exact sequence. Let
$0 \to F \to G \to H \to 0$ be a short exact sequence of
module-valued functors on $\textit{Alg}_A$.
\begin{enumerate}
\item If $(*)$ holds for $F, G$ then $(*)$ holds for $H$.
\item If $(*)$ holds for $F, H$ then $(*)$ holds for $G$.
\item If $H' \to H$ is morphism of module-valued functors on $\textit{Alg}_A$
and $(*)$ holds for $F$, $G$, $H$, and $H'$, then $(*)$ holds for
$G \times_H H'$.
\end{enumerate}
\end{lemma}

\begin{proof}
Let $B$ be given. Let $0 \to N_1 \to N_2 \to N_3 \to 0$ be a short exact
sequence of $B$-modules. Part (1) follows from a diagram chase in
the diagram
$$
\xymatrix{
0 \ar[r] &
TF(B, N_1) \ar[r] \ar[d] &
TG(B, N_1) \ar[r] \ar[d] &
TH(B, N_1) \ar[r] \ar[d] & 0 \\
0 \ar[r] &
TF(B, N_2) \ar[r] \ar[d] &
TG(B, N_2) \ar[r] \ar[d] &
TH(B, N_2) \ar[r] \ar[d] & 0 \\
0 \ar[r] &
TF(B, N_3) \ar[r] \ar[d] &
TG(B, N_3) \ar[r] \ar[d] &
TH(B, N_3) \ar[r] & 0 \\
& 0 & 0
}
$$
with exact horizontal rows and exact columns involving $TF$ and $TG$.
To prove part (2) we do a diagram chase in the diagram
$$
\xymatrix{
0 \ar[r] &
TF(B, N_1) \ar[r] \ar[d] &
TG(B, N_1) \ar[r] \ar[d] &
TH(B, N_1) \ar[r] \ar[d] & 0 \\
0 \ar[r] &
TF(B, N_2) \ar[r] \ar[d] &
TG(B, N_2) \ar[r] \ar[d] &
TH(B, N_2) \ar[r] \ar[d] & 0 \\
0 \ar[r] &
TF(B, N_3) \ar[r] \ar[d] &
TG(B, N_3) \ar[r] &
TH(B, N_3) \ar[r] \ar[d] & 0 \\
& 0 & & 0
}
$$
with exact horizontal rows and exact columns involving $TF$ and $TH$.
Part (3) follows from part (2) as $G \times_H H'$ sits in the exact
sequence $0 \to F \to G \times_H H' \to H' \to 0$.
\end{proof}

\noindent
Most of the work in this section was done in order to prove the
following key vanishing result.

\begin{lemma}
\label{lemma-ext-group-zero-key}
Let $A$ be a ring. Let $M$, $P$ be $A$-modules with $P$ of finite
presentation. Then
$\Ext^i_\mathcal{P}(\underline{P}, \underline{M}) = 0$
for $i > 0$ where $\mathcal{P}$ is the category of module-valued
functors on $\textit{Alg}_A$.
\end{lemma}

\begin{proof}
Choose an injective resolution $\underline{M} \to I^\bullet$ in
$\mathcal{P}$, see
Lemma \ref{lemma-enough-injectives}.
By
Derived Categories, Lemma \ref{derived-lemma-compute-ext-resolutions}
any element of $\Ext^i_\mathcal{P}(\underline{P}, \underline{M})$
comes from a morphism $\varphi : \underline{P} \to I^i$ with
$d^i \circ \varphi = 0$. We will prove that the
Yoneda extension
$$
E : 0 \to \underline{M} \to I^0 \to \ldots \to
I^{i - 1} \times_{\Ker(d^i)} \underline{P} \to \underline{P} \to 0
$$
of $\underline{P}$ by $\underline{M}$
associated to $\varphi$ is trivial, which will prove the lemma by
Derived Categories, Lemma \ref{derived-lemma-yoneda-extension}.

\medskip\noindent
For $F$ a module-valued functor on $\textit{Alg}_A$
say $(*)$ holds if for all $B \in \Ob(\textit{Alg}_A)$ the
functor $TF(B, -)$ on $B$-modules transforms a short exact sequence
of $B$-modules into a right exact sequence.
Recall that the module-valued functors $\underline{M}, I^n, \underline{P}$
each have property $(*)$, see
Lemma \ref{lemma-tangent-injective}
and the remarks preceding it.
By splitting $0 \to \underline{M} \to I^\bullet$ into short
exact sequences we find that each of the functors
$\Im(d^{n - 1}) = \Ker(d^n) \subset I^n$ has property $(*)$ by
Lemma \ref{lemma-exactness-permanence}
and also that $I^{i - 1} \times_{\Ker(d^i)} \underline{P}$ has property
$(*)$.

\medskip\noindent
Thus we may assume the Yoneda extension is given as
$$
E : 0 \to \underline{M} \to F_{i - 1} \to \ldots \to
F_0 \to \underline{P} \to 0
$$
where each of the module-valued functors $F_j$ has property $(*)$.
Set $G_j(B) = TF_j(B, B)$ viewed as a $B$-module via the {\it second}
$B$-module structure defined in
Lemma \ref{lemma-exactness-implies}.
Since $TF_j$ is a functor on pairs we see that $G_j$ is a module-valued
functor on $\textit{Alg}_A$. Moreover, since $E$ is an exact sequence
the sequence $G_{j + 1} \to G_j \to G_{j - 1}$ is exact (see remark
preceding
Lemma \ref{lemma-exactness-permanence}).
Observe that $T\underline{M}(B, B) = M \otimes_A B = \underline{M}(B)$
and that the two $B$-module structures agree on this.
Thus we obtain a Yoneda extension
$$
E' :  0 \to \underline{M} \to G_{i - 1} \to \ldots \to
G_0 \to \underline{P} \to 0
$$
Moreover, the canonical maps
$$
F_j(B) = B \otimes_B F_j(B) \longrightarrow TF_j(B, B) = G_j(B)
$$
of
Lemma \ref{lemma-tangent-functor} (\ref{item-mult-map})
are $B$-linear by
Lemma \ref{lemma-exactness-implies} (\ref{item-mult-map-linear})
and functorial in $B$. Hence a map
$$
\xymatrix{
0 \ar[r] &
\underline{M} \ar[r] \ar[d]^1 &
F_{i - 1} \ar[r] \ar[d] &
\ldots \ar[r] &
F_0 \ar[r] \ar[d] &
\underline{P} \ar[r] \ar[d]^1 & 0 \\
0 \ar[r] &
\underline{M} \ar[r] &
G_{i - 1} \ar[r] &
\ldots \ar[r] &
G_0 \ar[r] &
\underline{P} \ar[r] & 0
}
$$
of Yoneda extensions. In particular we see that $E$ and $E'$ have the
same class in $\Ext^i_\mathcal{P}(\underline{P}, \underline{M})$
by the lemma on Yoneda Exts mentioned above. Finally, let $N$ be a
$A$-module of finite presentation. Then we see that
$$
0 \to T\underline{M}(A, N) \to TF_{i - 1}(A, N) \to \ldots \to
TF_0(A, N) \to T\underline{P}(A, N) \to 0
$$
is exact. By
Lemma \ref{lemma-exactness-implies} (\ref{item-tangent-right-exact})
with $B = A$ this translates into the exactness of the sequence of
$A$-modules
$$
0 \to M \otimes_A N \to G_{i - 1}(A) \otimes_A N \to \ldots \to
G_0(A) \otimes_A N \to P \otimes_A N \to 0
$$
Hence the sequence of $A$-modules
$0 \to M \to G_{i - 1}(A) \to \ldots \to G_0(A) \to P \to 0$
is universally exact, in the sense that it remains exact on tensoring
with any finitely presented $A$-module $N$. Let
$K = \Ker(G_0(A) \to P)$ so that we have exact sequences
$$
0 \to K \to G_0(A) \to P \to 0
\quad\text{and}\quad
G_2(A) \to G_1(A) \to K \to 0
$$
Tensoring the second sequence with $N$ we obtain that
$K \otimes_A N = \Coker(G_2(A) \otimes_A N \to G_1(A) \otimes_A N)$.
Exactness of $G_2(A) \otimes_A N \to G_1(A) \otimes_A N \to G_0(A) \otimes_A N$
then implies that $K \otimes_A N \to G_0(A) \otimes_A N$ is injective.
By
Algebra, Theorem \ref{algebra-theorem-universally-exact-criteria}
this means that the $A$-module extension $0 \to K \to G_0(A) \to P \to 0$
is exact, and because $P$ is assumed of finite presentation this means
the sequence is split, see
Algebra, Lemma \ref{algebra-lemma-universally-exact-split}.
Any splitting $P \to G_0(A)$ defines a map $\underline{P} \to G_0$
which splits the surjection $G_0 \to \underline{P}$. Thus the
Yoneda extension $E'$ is equivalent to the trivial Yoneda extension
and we win.
\end{proof}

\begin{lemma}
\label{lemma-ext-group-zero}
Let $A$ be a ring. Let $M$ be an $A$-module. Let $L$ be a linearly
adequate functor on $\textit{Alg}_A$. Then
$\Ext^i_\mathcal{P}(L, \underline{M}) = 0$
for $i > 0$ where $\mathcal{P}$ is the category of module-valued
functors on $\textit{Alg}_A$.
\end{lemma}

\begin{proof}
Since $L$ is linearly adequate there exists an exact sequence
$$
0 \to L \to \underline{A^{\oplus m}} \to \underline{A^{\oplus n}} \to
\underline{P} \to 0
$$
Here $P = \Coker(A^{\oplus m} \to A^{\oplus n})$ is the cokernel
of the map of finite free $A$-modules which is given by the definition
of linearly adequate functors. By
Lemma \ref{lemma-ext-group-zero-key}
we have the vanishing of
$\Ext^i_\mathcal{P}(\underline{P}, \underline{M})$
and
$\Ext^i_\mathcal{P}(\underline{A}, \underline{M})$
for $i > 0$.
Let $K = \Ker(\underline{A^{\oplus n}} \to \underline{P})$.
By the long exact sequence of Ext groups associated to the exact sequence
$0 \to K \to \underline{A^{\oplus n}} \to \underline{P} \to 0$
we conclude that
$\Ext^i_\mathcal{P}(K, \underline{M}) = 0$ for $i > 0$.
Repeating with the sequence
$0 \to L \to \underline{A^{\oplus m}} \to K \to 0$
we win.
\end{proof}

\begin{lemma}
\label{lemma-RQ-zero}
With notation as in
Lemma \ref{lemma-enough-injectives}
we have $R^pQ(F) = 0$ for all $p > 0$ and any adequate functor $F$.
\end{lemma}

\begin{proof}
Choose an exact sequence $0 \to F \to \underline{M^0} \to \underline{M^1}$.
Set $M^2 = \Coker(M^0 \to M^1)$ so that
$0 \to F \to \underline{M^0} \to \underline{M^1}
\to \underline{M^2} \to 0$ is a resolution. By
Derived Categories, Lemma \ref{derived-lemma-two-ss-complex-functor}
we obtain a spectral sequence
$$
R^pQ(\underline{M^q}) \Rightarrow R^{p + q}Q(F)
$$
Since $Q(\underline{M^q}) = \underline{M^q}$ it suffices to prove
$R^pQ(\underline{M}) = 0$, $p > 0$ for any $A$-module $M$.

\medskip\noindent
Choose an injective resolution $\underline{M} \to I^\bullet$ in
the category $\mathcal{P}$. Suppose that $R^iQ(\underline{M})$ is nonzero.
Then $\Ker(Q(I^i) \to Q(I^{i + 1}))$ is strictly bigger
than the image of $Q(I^{i - 1}) \to Q(I^i)$. Hence by
Lemma \ref{lemma-adequate-surjection-from-linear}
there exists a linearly adequate functor $L$ and a map
$\varphi : L \to Q(I^i)$ mapping into the kernel of $Q(I^i) \to Q(I^{i + 1})$
which does not factor through the image of $Q(I^{i - 1}) \to Q(I^i)$.
Because $Q$ is a left adjoint to the inclusion functor the map
$\varphi$ corresponds to a map $\varphi' : L \to I^i$ with the same properties.
Thus $\varphi'$ gives a nonzero element of
$\Ext^i_\mathcal{P}(L, \underline{M})$ contradicting
Lemma \ref{lemma-ext-group-zero}.
\end{proof}














\section{Adequate modules}
\label{section-adequate}

\noindent
In
Descent, Section \ref{descent-section-quasi-coherent-sheaves}
we have seen that quasi-coherent modules on a scheme $S$
are the same as quasi-coherent modules on any of the big
sites $(\Sch/S)_\tau$ associated to $S$. We have seen that there
are two issues with this identification:
\begin{enumerate}
\item $\QCoh(\mathcal{O}_S) \to
\textit{Mod}((\Sch/S)_\tau, \mathcal{O})$,
$\mathcal{F} \mapsto \mathcal{F}^a$ is not exact in general
(Descent, Lemma \ref{descent-lemma-equivalence-quasi-coherent-limits}), and
\item given a quasi-compact and quasi-separated morphism $f : X \to S$
the functor $f_*$ does not preserve quasi-coherent sheaves on the
big sites in general (Descent, Proposition
\ref{descent-proposition-equivalence-quasi-coherent-functorial}).
\end{enumerate}
Part (1) means that we cannot define a triangulated subcategory
of $D(\mathcal{O})$ consisting of complexes whose cohomology sheaves
are quasi-coherent. Part (2) means that $Rf_*\mathcal{F}$ isn't a
complex with quasi-coherent cohomology sheaves even when $\mathcal{F}$
is quasi-coherent and $f$ is quasi-compact and quasi-separated.
Moreover, the examples given in the proofs of
Descent, Lemma
\ref{descent-lemma-equivalence-quasi-coherent-limits}
and
Descent, Proposition
\ref{descent-proposition-equivalence-quasi-coherent-functorial}
are not of a pathological nature.

\medskip\noindent
In this section we discuss a slightly larger category
of $\mathcal{O}$-modules on $(\Sch/S)_\tau$ with contains the
quasi-coherent modules, is abelian, and is preserved under $f_*$ when
$f$ is quasi-compact and quasi-separated.
To do this, suppose that $S$ is a scheme. Let $\mathcal{F}$ be a presheaf
of $\mathcal{O}$-modules on $(\Sch/S)_\tau$.
For any affine object $U = \Spec(A)$ of $(\Sch/S)_\tau$
we can restrict $\mathcal{F}$ to $(\textit{Aff}/U)_\tau$ to get
a presheaf of $\mathcal{O}$-modules on this site. The corresponding
module-valued functor, see
Section \ref{section-quasi-coherent},
will be denoted
$$
F = F_{\mathcal{F}, A} :
\textit{Alg}_A \longrightarrow \textit{Ab},
\quad
B \longmapsto \mathcal{F}(\Spec(B))
$$
The assignment $\mathcal{F} \mapsto F_{\mathcal{F}, A}$ is an exact
functor of abelian categories.

\begin{definition}
\label{definition-adequate}
A sheaf of $\mathcal{O}$-modules $\mathcal{F}$ on $(\Sch/S)_\tau$ is
{\it adequate} if there exists a $\tau$-covering
$\{\Spec(A_i) \to S\}_{i \in I}$ such that $F_{\mathcal{F}, A_i}$ is
adequate for all $i \in I$.
\end{definition}

\noindent
We will see below that the category of adequate $\mathcal{O}$-modules
is independent of the chosen topology $\tau$.

\begin{lemma}
\label{lemma-adequate-local}
Let $S$ be a scheme. Let $\mathcal{F}$ be an adequate $\mathcal{O}$-module on
$(\Sch/S)_\tau$. For any affine scheme $\Spec(A)$ over $S$
the functor $F_{\mathcal{F}, A}$ is adequate.
\end{lemma}

\begin{proof}
Let $\{\Spec(A_i) \to S\}_{i \in I}$ be a $\tau$-covering
such that $F_{\mathcal{F}, A_i}$ is adequate for all $i \in I$.
We can find a standard affine $\tau$-covering
$\{\Spec(A'_j) \to \Spec(A)\}_{j = 1, \ldots, m}$
such that $\Spec(A'_j) \to \Spec(A) \to S$ factors
through $\Spec(A_{i(j)})$ for some $i(j) \in I$. Then we see that
$F_{\mathcal{F}, A'_j}$ is the restriction of
$F_{\mathcal{F}, A_{i(j)}}$ to the category of $A'_j$-algebras.
Hence $F_{\mathcal{F}, A'_j}$ is adequate by
Lemma \ref{lemma-base-change-adequate}.
By
Lemma \ref{lemma-adequate-product}
the sequence
$F_{\mathcal{F}, A'_j}$ corresponds to an adequate ``product'' functor
$F'$ over $A' = A'_1 \times \ldots \times A'_m$. As $\mathcal{F}$ is a
sheaf (for the Zariski topology) this product functor $F'$ is equal
to $F_{\mathcal{F}, A'}$, i.e., is the restriction of $F$ to $A'$-algebras.
Finally,  $\{\Spec(A') \to \Spec(A)\}$ is a $\tau$-covering.
It follows from
Lemma \ref{lemma-adequate-descent}
that $F_{\mathcal{F}, A}$ is adequate.
\end{proof}

\begin{lemma}
\label{lemma-adequate-affine}
Let $S = \Spec(A)$ be an affine scheme. The category of adequate
$\mathcal{O}$-modules on $(\Sch/S)_\tau$ is equivalent to the
category of adequate module-valued functors on $\textit{Alg}_A$.
\end{lemma}

\begin{proof}
Given an adequate module $\mathcal{F}$ the functor $F_{\mathcal{F}, A}$
is adequate by Lemma \ref{lemma-adequate-local}.
Given an adequate functor $F$ we choose an exact sequence
$0 \to F \to \underline{M} \to \underline{N}$ and we consider
the $\mathcal{O}$-module $\mathcal{F} = \Ker(M^a \to N^a)$ where
$M^a$ denotes the quasi-coherent $\mathcal{O}$-module on
$(\Sch/S)_\tau$ associated to the quasi-coherent sheaf
$\widetilde{M}$ on $S$. Note that $F = F_{\mathcal{F}, A}$, in particular
the module $\mathcal{F}$ is adequate by definition.
We omit the proof that the constructions define mutually inverse
equivalences of categories.
\end{proof}

\begin{lemma}
\label{lemma-pullback-adequate}
Let $f : T \to S$ be a morphism of schemes.
The pullback $f^*\mathcal{F}$ of an adequate $\mathcal{O}$-module
$\mathcal{F}$ on $(\Sch/S)_\tau$ is an adequate
$\mathcal{O}$-module on $(\Sch/T)_\tau$.
\end{lemma}

\begin{proof}
The pullback map
$f^* : \textit{Mod}((\Sch/S)_\tau, \mathcal{O}) \to
\textit{Mod}((\Sch/T)_\tau, \mathcal{O})$
is given by restriction, i.e., $f^*\mathcal{F}(V) = \mathcal{F}(V)$
for any scheme $V$ over $T$. Hence this lemma follows immediately from
Lemma \ref{lemma-adequate-local}
and the definition.
\end{proof}

\noindent
Here is a characterization of the category of adequate $\mathcal{O}$-modules.
To understand the significance, consider a map $\mathcal{G} \to \mathcal{H}$
of quasi-coherent $\mathcal{O}_S$-modules on a scheme $S$.
The cokernel of the associated map $\mathcal{G}^a \to \mathcal{H}^a$
of $\mathcal{O}$-modules is quasi-coherent because it is equal to
$(\mathcal{H}/\mathcal{G})^a$. But the kernel of
$\mathcal{G}^a \to \mathcal{H}^a$ in general isn't
quasi-coherent. However, it is adequate.

\begin{lemma}
\label{lemma-adequate-characterize}
Let $S$ be a scheme. Let $\mathcal{F}$ be an $\mathcal{O}$-module on
$(\Sch/S)_\tau$. The following are equivalent
\begin{enumerate}
\item $\mathcal{F}$ is adequate,
\item there exists an affine open covering $S = \bigcup S_i$ and
maps of quasi-coherent $\mathcal{O}_{S_i}$-modules
$\mathcal{G}_i \to \mathcal{H}_i$
such that $\mathcal{F}|_{(\Sch/S_i)_\tau}$ is the
kernel of $\mathcal{G}_i^a \to \mathcal{H}_i^a$
\item there exists a $\tau$-covering $\{S_i \to S\}_{i \in I}$ and
maps of $\mathcal{O}_{S_i}$-quasi-coherent modules
$\mathcal{G}_i \to \mathcal{H}_i$
such that $\mathcal{F}|_{(\Sch/S_i)_\tau}$ is the
kernel of $\mathcal{G}_i^a \to \mathcal{H}_i^a$,
\item there exists a $\tau$-covering $\{f_i : S_i \to S\}_{i \in I}$
such that each $f_i^*\mathcal{F}$ is adequate,
\item for any affine scheme $U$ over $S$ the restriction
$\mathcal{F}|_{(\Sch/U)_\tau}$ is the kernel
of a map $\mathcal{G}^a \to \mathcal{H}^a$ of quasi-coherent
$\mathcal{O}_U$-modules.
\end{enumerate}
\end{lemma}

\begin{proof}
Let $U = \Spec(A)$ be an affine scheme over $S$.
Set $F = F_{\mathcal{F}, A}$. By definition, the functor
$F$ is adequate if and only if there exists a map of $A$-modules
$M \to N$ such that $F = \Ker(\underline{M} \to \underline{N})$.
Combining with
Lemmas \ref{lemma-adequate-local} and
\ref{lemma-adequate-affine}
we see that (1) and (5) are equivalent.

\medskip\noindent
It is clear that (5) implies (2) and (2) implies (3).
If (3) holds then we can refine the covering
$\{S_i \to S\}$ such that each $S_i = \Spec(A_i)$ is affine.
Then we see, by the preliminary remarks of the proof, that
$F_{\mathcal{F}, A_i}$ is adequate. Thus $\mathcal{F}$
is adequate by definition. Hence (3) implies (1).

\medskip\noindent
Finally, (4) is equivalent to (1) using
Lemma \ref{lemma-pullback-adequate}
for one direction and that
a composition of $\tau$-coverings is a $\tau$-covering for the other.
\end{proof}

\noindent
Just like is true for quasi-coherent sheaves the category of
adequate modules is independent of the topology.

\begin{lemma}
\label{lemma-adequate-fpqc}
Let $\mathcal{F}$ be an adequate $\mathcal{O}$-module on
$(\Sch/S)_\tau$. For any surjective flat morphism
$\Spec(B) \to \Spec(A)$ of affines over $S$
the extended {\v C}ech complex
$$
0 \to \mathcal{F}(\Spec(A)) \to
\mathcal{F}(\Spec(B)) \to
\mathcal{F}(\Spec(B \otimes_A B)) \to \ldots
$$
is exact. In particular $\mathcal{F}$ satisfies the sheaf condition
for fpqc coverings, and is a sheaf of $\mathcal{O}$-modules
on $(\Sch/S)_{fppf}$.
\end{lemma}

\begin{proof}
With $A \to B$ as in the lemma let $F = F_{\mathcal{F}, A}$. This functor
is adequate by
Lemma \ref{lemma-adequate-local}.
By
Lemma \ref{lemma-adequate-flat}
since $A \to B$, $A \to B \otimes_A B$, etc are flat we see that
$F(B) = F(A) \otimes_A B$,
$F(B \otimes_A B) = F(A) \otimes_A B \otimes_A B$, etc.
Exactness follows from
Descent, Lemma \ref{descent-lemma-ff-exact}.

\medskip\noindent
Thus $\mathcal{F}$ satisfies the sheaf condition for
$\tau$-coverings (in particular Zariski coverings) and any faithfully
flat covering of an affine by an affine. Arguing as in the proofs of
Descent, Lemma \ref{descent-lemma-standard-fpqc-covering}
and
Descent, Proposition \ref{descent-proposition-fpqc-descent-quasi-coherent}
we conclude that $\mathcal{F}$ satisfies the sheaf condition for all
fpqc coverings (made out of objects of $(\Sch/S)_\tau$).
Details omitted.
\end{proof}

\noindent
Lemma \ref{lemma-adequate-fpqc} shows in particular that
for any pair of topologies $\tau, \tau'$ the collection
of adequate modules for the $\tau$-topology and the $\tau'$-topology
are identical (as presheaves of modules on the underlying category $\Sch/S$).

\begin{definition}
\label{definition-category-adequate-modules}
Let $S$ be a scheme. The category of adequate $\mathcal{O}$-modules on
$(\Sch/S)_\tau$ is denoted {\it $\textit{Adeq}(\mathcal{O})$} or
{\it $\textit{Adeq}((\Sch/S)_\tau, \mathcal{O})$}. If we want to think just
about the abelian category of adequate modules without choosing a
topology we simply write {\it $\textit{Adeq}(S)$}.
\end{definition}

\begin{lemma}
\label{lemma-same-cohomology-adequate}
Let $S$ be a scheme. Let $\mathcal{F}$ be an adequate
$\mathcal{O}$-module on $(\Sch/S)_\tau$.
\begin{enumerate}
\item The restriction $\mathcal{F}|_{S_{Zar}}$ is a quasi-coherent
$\mathcal{O}_S$-module on the scheme $S$.
\item The restriction $\mathcal{F}|_{S_\etale}$ is the
quasi-coherent module associated to $\mathcal{F}|_{S_{Zar}}$.
\item For any affine scheme $U$ over $S$ we have $H^q(U, \mathcal{F}) = 0$
for all $q > 0$.
\item There is a canonical isomorphism
$$
H^q(S, \mathcal{F}|_{S_{Zar}}) =
H^q((\Sch/S)_\tau, \mathcal{F}).
$$
\end{enumerate}
\end{lemma}

\begin{proof}
By
Lemma \ref{lemma-adequate-flat}
and
Lemma \ref{lemma-adequate-local}
we see that for any flat morphism of affines $U \to V$ over $S$
we have
$\mathcal{F}(U) = \mathcal{F}(V) \otimes_{\mathcal{O}(V)} \mathcal{O}(U)$.
This works in particular if $U \subset V \subset S$ are affine opens of
$S$, hence $\mathcal{F}|_{S_{Zar}}$ is quasi-coherent.
Thus (1) holds.

\medskip\noindent
Let $S' \to S$ be an \'etale morphism of schemes.
Then for $U \subset S'$ affine open mapping into an affine open
$V \subset S$ we see that
$\mathcal{F}(U) = \mathcal{F}(V) \otimes_{\mathcal{O}(V)} \mathcal{O}(U)$
because $U \to V$ is \'etale, hence flat. Therefore
$\mathcal{F}|_{S'_{Zar}}$ is the pullback of $\mathcal{F}|_{S_{Zar}}$.
This proves (2).

\medskip\noindent
We are going to apply
Cohomology on Sites,
Lemma \ref{sites-cohomology-lemma-cech-vanish-collection}
to the site $(\Sch/S)_\tau$ with
$\mathcal{B}$ the set of affine schemes over $S$ and
$\text{Cov}$ the set of standard affine $\tau$-coverings.
Assumption (3) of the lemma is satisfied by
Descent, Lemma \ref{descent-lemma-standard-covering-Cech}
and
Lemma \ref{lemma-adequate-fpqc}
for the case of a covering by a single affine.
Hence we conclude that $H^p(U, \mathcal{F}) = 0$ for every
affine scheme $U$ over $S$. This proves (3).
In exactly the same way as in the proof of
Descent, Proposition \ref{descent-proposition-same-cohomology-quasi-coherent}
this implies the equality of cohomologies (4).
\end{proof}

\begin{remark}
\label{remark-compare}
Let $S$ be a scheme. We have functors
$u : \QCoh(\mathcal{O}_S) \to \textit{Adeq}(\mathcal{O})$
and
$v : \textit{Adeq}(\mathcal{O}) \to \QCoh(\mathcal{O}_S)$.
Namely, the functor $u : \mathcal{F} \mapsto \mathcal{F}^a$
comes from taking the associated $\mathcal{O}$-module which is
adequate by
Lemma \ref{lemma-adequate-characterize}.
Conversely, the functor $v$ comes from restriction
$v : \mathcal{G} \mapsto \mathcal{G}|_{S_{Zar}}$, see
Lemma \ref{lemma-same-cohomology-adequate}.
Since $\mathcal{F}^a$ can be described as the pullback of
$\mathcal{F}$ under a morphism of ringed topoi
$((\Sch/S)_\tau, \mathcal{O}) \to (S_{Zar}, \mathcal{O}_S)$, see
Descent, Remark \ref{descent-remark-change-topologies-ringed-sites}
and since restriction is the pushforward we see that $u$ and $v$
are adjoint as follows
$$
\SheafHom_{\mathcal{O}_S}(\mathcal{F}, v\mathcal{G})
=
\SheafHom_\mathcal{O}(u\mathcal{F}, \mathcal{G})
$$
where $\mathcal{O}$ denotes the structure sheaf on the big site.
It is immediate from the description that the adjunction mapping
$\mathcal{F} \to vu\mathcal{F}$ is an isomorphism for all quasi-coherent
sheaves.
\end{remark}

\begin{lemma}
\label{lemma-sheafification-adequate}
Let $S$ be a scheme. Let $\mathcal{F}$ be a presheaf of $\mathcal{O}$-modules
on $(\Sch/S)_\tau$. If for every affine scheme
$\Spec(A)$ over $S$ the functor $F_{\mathcal{F}, A}$ is
adequate, then the sheafification of $\mathcal{F}$ is an adequate
$\mathcal{O}$-module.
\end{lemma}

\begin{proof}
Let $U = \Spec(A)$ be an affine scheme over $S$.
Set $F = F_{\mathcal{F}, A}$.
The sheafification $\mathcal{F}^\# = (\mathcal{F}^+)^+$, see
Sites, Section \ref{sites-section-sheafification}.
By construction
$$
(\mathcal{F})^+(U) =
\colim_\mathcal{U} \check{H}^0(\mathcal{U}, \mathcal{F})
$$
where the colimit is over coverings in the site $(\Sch/S)_\tau$.
Since $U$ is affine it suffices to take the limit over standard
affine $\tau$-coverings
$\mathcal{U} = \{U_i \to U\}_{i \in I} =
\{\Spec(A_i) \to \Spec(A)\}_{i \in I}$ of $U$.
Since each $A \to A_i$ and $A \to A_i \otimes_A A_j$ is flat we see that
$$
\check{H}^0(\mathcal{U}, \mathcal{F}) =
\Ker(\prod F(A) \otimes_A A_i \to \prod F(A) \otimes_A A_i \otimes_A A_j)
$$
by
Lemma \ref{lemma-adequate-flat}.
Since $A \to \prod A_i$ is faithfully flat we see that this always
is canonically isomorphic to $F(A)$ by
Descent, Lemma \ref{descent-lemma-ff-exact}.
Thus the presheaf $(\mathcal{F})^+$ has the same value as
$\mathcal{F}$ on all affine schemes over $S$. Repeating the argument
once more we deduce the same thing for $\mathcal{F}^\# = ((\mathcal{F})^+)^+$.
Thus $F_{\mathcal{F}, A} = F_{\mathcal{F}^\#, A}$ and we conclude
that $\mathcal{F}^\#$ is adequate.
\end{proof}

\begin{lemma}
\label{lemma-abelian-adequate}
Let $S$ be a scheme.
\begin{enumerate}
\item The category $\textit{Adeq}(\mathcal{O})$ is abelian.
\item The functor
$\textit{Adeq}(\mathcal{O}) \to
\textit{Mod}((\Sch/S)_\tau, \mathcal{O})$
is exact.
\item If $0 \to \mathcal{F}_1 \to \mathcal{F}_2 \to \mathcal{F}_3 \to 0$
is a short exact sequence of $\mathcal{O}$-modules and
$\mathcal{F}_1$ and $\mathcal{F}_3$ are adequate, then
$\mathcal{F}_2$ is adequate.
\item The category $\textit{Adeq}(\mathcal{O})$ has colimits and
$\textit{Adeq}(\mathcal{O}) \to
\textit{Mod}((\Sch/S)_\tau, \mathcal{O})$
commutes with them.
\end{enumerate}
\end{lemma}

\begin{proof}
Let $\varphi : \mathcal{F} \to \mathcal{G}$ be a map of adequate
$\mathcal{O}$-modules. To prove (1) and (2) it suffices to show that
$\mathcal{K} = \Ker(\varphi)$ and
$\mathcal{Q} = \Coker(\varphi)$ computed in
$\textit{Mod}((\Sch/S)_\tau, \mathcal{O})$ are adequate.
Let $U = \Spec(A)$ be an affine scheme over $S$.
Let $F = F_{\mathcal{F}, A}$ and $G = F_{\mathcal{G}, A}$.
By
Lemmas \ref{lemma-kernel-adequate} and
\ref{lemma-cokernel-adequate}
the kernel $K$ and cokernel $Q$ of the induced map
$F \to G$ are adequate functors.
Because the kernel is computed on the level of presheaves, we see
that $K = F_{\mathcal{K}, A}$ and we conclude $\mathcal{K}$ is adequate.
To prove the result for the cokernel, denote $\mathcal{Q}'$ the presheaf
cokernel of $\varphi$. Then $Q = F_{\mathcal{Q}', A}$ and
$\mathcal{Q} = (\mathcal{Q}')^\#$. Hence $\mathcal{Q}$
is adequate by
Lemma \ref{lemma-sheafification-adequate}.

\medskip\noindent
Let $0 \to \mathcal{F}_1 \to \mathcal{F}_2 \to \mathcal{F}_3 \to 0$
is a short exact sequence of $\mathcal{O}$-modules and
$\mathcal{F}_1$ and $\mathcal{F}_3$ are adequate.
Let $U = \Spec(A)$ be an affine scheme over $S$.
Let $F_i = F_{\mathcal{F}_i, A}$. The sequence of functors
$$
0 \to F_1 \to F_2 \to F_3 \to 0
$$
is exact, because for $V = \Spec(B)$ affine over $U$ we have
$H^1(V, \mathcal{F}_1) = 0$ by
Lemma \ref{lemma-same-cohomology-adequate}.
Since $F_1$ and $F_3$ are adequate functors by
Lemma \ref{lemma-adequate-local}
we see that $F_2$ is adequate by
Lemma \ref{lemma-extension-adequate}.
Thus $\mathcal{F}_2$ is adequate.

\medskip\noindent
Let $\mathcal{I} \to \textit{Adeq}(\mathcal{O})$, $i \mapsto \mathcal{F}_i$
be a diagram. Denote $\mathcal{F} = \colim_i \mathcal{F}_i$
the colimit computed in
$\textit{Mod}((\Sch/S)_\tau, \mathcal{O})$.
To prove (4) it suffices to show that $\mathcal{F}$ is adequate.
Let $\mathcal{F}' = \colim_i \mathcal{F}_i$ be the colimit computed
in presheaves of $\mathcal{O}$-modules. Then
$\mathcal{F} = (\mathcal{F}')^\#$.
Let $U = \Spec(A)$ be an affine scheme over $S$.
Let $F_i = F_{\mathcal{F}_i, A}$. By
Lemma \ref{lemma-colimit-adequate}
the functor $\colim_i F_i = F_{\mathcal{F}', A}$ is adequate.
Lemma \ref{lemma-sheafification-adequate}
shows that $\mathcal{F}$ is adequate.
\end{proof}

\noindent
The following lemma tells us that the total direct image
$Rf_*\mathcal{F}$ of an adequate module under a quasi-compact and
quasi-separated morphism is a complex whose cohomology sheaves
are adequate.

\begin{lemma}
\label{lemma-direct-image-adequate}
Let $f : T \to S$ be a quasi-compact and quasi-separated morphism
of schemes. For any adequate $\mathcal{O}_T$-module on
$(\Sch/T)_\tau$ the pushforward
$f_*\mathcal{F}$ and the higher direct images $R^if_*\mathcal{F}$
are adequate $\mathcal{O}_S$-modules on $(\Sch/S)_\tau$.
\end{lemma}

\begin{proof}
First we explain how to compute the higher direct images.
Choose an injective resolution $\mathcal{F} \to \mathcal{I}^\bullet$.
Then $R^if_*\mathcal{F}$ is the $i$th cohomology sheaf of the
complex $f_*\mathcal{I}^\bullet$.
Hence $R^if_*\mathcal{F}$ is the sheaf associated to the presheaf
which associates to an object $U/S$ of $(\Sch/S)_\tau$
the module
\begin{align*}
\frac{\Ker(f_*\mathcal{I}^i(U) \to f_*\mathcal{I}^{i + 1}(U))}
{\Im(f_*\mathcal{I}^{i - 1}(U) \to f_*\mathcal{I}^i(U))}
& =
\frac{\Ker(\mathcal{I}^i(U \times_S T) \to
\mathcal{I}^{i + 1}(U \times_S T))}
{\Im(\mathcal{I}^{i - 1}(U \times_S T) \to \mathcal{I}^i(U \times_S T))}
\\
& =
H^i(U \times_S T, \mathcal{F}) \\
& = H^i((\Sch/U \times_S T)_\tau,
\mathcal{F}|_{(\Sch/U \times_S T)_\tau}) \\
& = H^i(U \times_S T, \mathcal{F}|_{(U \times_S T)_{Zar}})
\end{align*}
The first equality by
Topologies, Lemma \ref{topologies-lemma-morphism-big-fppf}
(and its analogues for other topologies),
the second equality by definition of cohomology of $\mathcal{F}$
over an object of $(\Sch/T)_\tau$,
the third equality by
Cohomology on Sites, Lemma \ref{sites-cohomology-lemma-cohomology-of-open},
and the last equality by
Lemma \ref{lemma-same-cohomology-adequate}.
Thus by
Lemma \ref{lemma-sheafification-adequate}
it suffices to prove the claim stated in the following paragraph.

\medskip\noindent
Let $A$ be a ring. Let $T$ be a scheme quasi-compact and quasi-separated
over $A$. Let $\mathcal{F}$ be an adequate $\mathcal{O}_T$-module on
$(\Sch/T)_\tau$. For an $A$-algebra $B$ set
$T_B = T \times_{\Spec(A)} \Spec(B)$ and denote
$\mathcal{F}_B = \mathcal{F}|_{(T_B)_{Zar}}$ the restriction of
$\mathcal{F}$ to the small Zariski site of $T_B$.
(Recall that this is a ``usual'' quasi-coherent sheaf on the scheme
$T_B$, see
Lemma \ref{lemma-same-cohomology-adequate}.)
Claim: The functor
$$
B \longmapsto H^q(T_B, \mathcal{F}_B)
$$
is adequate. We will prove the lemma by the usual
procedure of cutting $T$ into pieces.

\medskip\noindent
Case I: $T$ is affine. In this case the schemes $T_B$ are all affine
and $H^q(T_B, \mathcal{F}_B) = 0$ for all $q \geq 1$.
The functor $B \mapsto H^0(T_B, \mathcal{F}_B)$ is adequate by
Lemma \ref{lemma-pushforward-adequate}.

\medskip\noindent
Case II: $T$ is separated. Let $n$ be the minimal number of affines needed
to cover $T$. We argue by induction on $n$. The base case is Case I.
Choose an affine open covering $T = V_1 \cup \ldots \cup V_n$.
Set $V = V_1 \cup \ldots \cup V_{n - 1}$ and $U = V_n$. Observe that
$$
U \cap V = (V_1 \cap V_n) \cup \ldots \cup (V_{n - 1} \cap V_n)
$$
is also a union of $n - 1$ affine opens as $T$ is separated, see
Schemes, Lemma \ref{schemes-lemma-characterize-separated}.
Note that for each $B$ the base changes $U_B$, $V_B$ and
$(U \cap V)_B = U_B \cap V_B$ behave in the same way. Hence we see that
for each $B$ we have a long exact sequence
$$
0 \to
H^0(T_B, \mathcal{F}_B) \to
H^0(U_B, \mathcal{F}_B) \oplus H^0(V_B, \mathcal{F}_B) \to
H^0((U \cap V)_B, \mathcal{F}_B) \to
H^1(T_B, \mathcal{F}_B) \to \ldots
$$
functorial in $B$, see
Cohomology, Lemma \ref{cohomology-lemma-mayer-vietoris}.
By induction hypothesis the functors
$B \mapsto H^q(U_B, \mathcal{F}_B)$,
$B \mapsto H^q(V_B, \mathcal{F}_B)$, and
$B \mapsto H^q((U \cap V)_B, \mathcal{F}_B)$
are adequate. Using
Lemmas \ref{lemma-kernel-adequate} and
\ref{lemma-cokernel-adequate}
we see that our functor $B \mapsto H^q(T_B, \mathcal{F}_B)$ sits in the
middle of a short exact sequence whose outer terms are adequate.
Thus the claim follows from
Lemma \ref{lemma-extension-adequate}.

\medskip\noindent
Case III: General quasi-compact and quasi-separated case.
The proof is again by induction on the number $n$ of affines needed to
cover $T$. The base case $n = 1$ is Case I.
Choose an affine open covering $T = V_1 \cup \ldots \cup V_n$.
Set $V = V_1 \cup \ldots \cup V_{n - 1}$ and $U = V_n$. Note that
since $T$ is quasi-separated $U \cap V$ is a quasi-compact open of an
affine scheme, hence Case II applies to it. The rest of the argument
proceeds in exactly the same manner as in the paragraph above and is
omitted.
\end{proof}







\section{Parasitic adequate modules}
\label{section-parasitic}

\noindent
In this section we start comparing adequate modules and quasi-coherent
modules on a scheme $S$. Recall that there are functors
$u : \QCoh(\mathcal{O}_S) \to \textit{Adeq}(\mathcal{O})$
and
$v : \textit{Adeq}(\mathcal{O}) \to \QCoh(\mathcal{O}_S)$
satisfying the adjunction
$$
\SheafHom_{\QCoh(\mathcal{O}_S)}(\mathcal{F}, v\mathcal{G})
=
\SheafHom_{\textit{Adeq}(\mathcal{O})}(u\mathcal{F}, \mathcal{G})
$$
and such that $\mathcal{F} \to vu\mathcal{F}$ is an isomorphism for
every quasi-coherent sheaf $\mathcal{F}$, see
Remark \ref{remark-compare}.
Hence $u$ is a fully faithful embedding and we can identify
$\QCoh(\mathcal{O}_S)$ with a full subcategory of
$\textit{Adeq}(\mathcal{O})$.
The functor $v$ is exact but $u$ is not left exact in general.
The kernel of $v$ is the subcategory of parasitic adequate modules.

\medskip\noindent
In Descent, Definition \ref{descent-definition-parasitic}
we give the definition of a parasitic module.
For adequate modules the notion does not depend
on the chosen topology.

\begin{lemma}
\label{lemma-parasitic-adequate}
Let $S$ be a scheme.
Let $\mathcal{F}$ be an adequate $\mathcal{O}$-module on
$(\Sch/S)_\tau$. The following are equivalent:
\begin{enumerate}
\item $v\mathcal{F} = 0$,
\item $\mathcal{F}$ is parasitic,
\item $\mathcal{F}$ is parasitic for the $\tau$-topology,
\item $\mathcal{F}(U) = 0$ for all $U \subset S$ open, and
\item there exists an affine open covering $S = \bigcup U_i$
such that $\mathcal{F}(U_i) = 0$ for all $i$.
\end{enumerate}
\end{lemma}

\begin{proof}
The implications (2) $\Rightarrow$ (3) $\Rightarrow$ (4) $\Rightarrow$ (5)
are immediate from the definitions. Assume (5). Suppose that
$S = \bigcup U_i$ is an affine open covering such that $\mathcal{F}(U_i) = 0$
for all $i$. Let $V \to S$ be a flat morphism. There exists an affine
open covering $V = \bigcup V_j$ such that each $V_j$ maps into some
$U_i$. As the morphism $V_j \to S$ is flat, also $V_j \to U_i$ is flat. 
Hence the corresponding ring map
$A_i = \mathcal{O}(U_i) \to \mathcal{O}(V_j) = B_j$ is flat. Thus by
Lemma \ref{lemma-adequate-local}
and
Lemma \ref{lemma-adequate-flat}
we see that $\mathcal{F}(U_i) \otimes_{A_i} B_j \to \mathcal{F}(V_j)$
is an isomorphism. Hence $\mathcal{F}(V_j) = 0$. Since $\mathcal{F}$ is
a sheaf for the Zariski topology we conclude that $\mathcal{F}(V) = 0$.
In this way we see that (5) implies (2).

\medskip\noindent
This proves the equivalence of (2), (3), (4), and (5).
As (1) is equivalent to (3) (see
Remark \ref{remark-compare})
we conclude that all five conditions are equivalent.
\end{proof}

\noindent
Let $S$ be a scheme.
The subcategory of parasitic adequate modules is a Serre subcategory of
$\textit{Adeq}(\mathcal{O})$. The quotient is the category of
quasi-coherent modules.

\begin{lemma}
\label{lemma-adequate-by-parasitic}
Let $S$ be a scheme. The subcategory
$\mathcal{C} \subset \textit{Adeq}(\mathcal{O})$ of parasitic adequate
modules is a Serre subcategory. Moreover, the functor $v$ induces
an equivalence of categories
$$
\textit{Adeq}(\mathcal{O}) / \mathcal{C} = \QCoh(\mathcal{O}_S).
$$
\end{lemma}

\begin{proof}
The category $\mathcal{C}$ is the kernel of the exact functor
$v : \textit{Adeq}(\mathcal{O}) \to \QCoh(\mathcal{O}_S)$, see
Lemma \ref{lemma-parasitic-adequate}.
Hence it is a Serre subcategory by
Homology, Lemma \ref{homology-lemma-kernel-exact-functor}.
By
Homology, Lemma \ref{homology-lemma-serre-subcategory-is-kernel}
we obtain an induced exact functor
$\overline{v} :
\textit{Adeq}(\mathcal{O}) / \mathcal{C}
\to
\QCoh(\mathcal{O}_S)$.
Because $u$ is a right inverse to $v$ we see right away that
$\overline{v}$ is essentially surjective.
We see that $\overline{v}$ is faithful by
Homology, Lemma \ref{homology-lemma-quotient-by-kernel-exact-functor}.
Because $u$ is a right inverse to $v$ we finally conclude that
$\overline{v}$ is fully faithful.
\end{proof}

\begin{lemma}
\label{lemma-direct-image-parasitic-adequate}
Let $f : T \to S$ be a quasi-compact and quasi-separated morphism
of schemes. For any parasitic adequate $\mathcal{O}_T$-module on
$(\Sch/T)_\tau$ the pushforward
$f_*\mathcal{F}$ and the higher direct images $R^if_*\mathcal{F}$
are parasitic adequate $\mathcal{O}_S$-modules on $(\Sch/S)_\tau$.
\end{lemma}

\begin{proof}
We have already seen in
Lemma \ref{lemma-direct-image-adequate}
that these higher direct images are adequate.
Hence it suffices to show that
$(R^if_*\mathcal{F})(U_i) = 0$ for any $\tau$-covering
$\{U_i \to S\}$ open. And $R^if_*\mathcal{F}$
is parasitic by
Descent, Lemma \ref{descent-lemma-direct-image-parasitic}.
\end{proof}














\section{Derived categories of adequate modules, I}
\label{section-comparison}


\noindent
Let $S$ be a scheme. We continue the discussion started in
Section \ref{section-parasitic}.
The exact functor $v$ induces a functor
$$
D(\textit{Adeq}(\mathcal{O}))
\longrightarrow
D(\QCoh(\mathcal{O}_S))
$$
and similarly for bounded versions.

\begin{lemma}
\label{lemma-quotient-easy}
Let $S$ be a scheme. Let
$\mathcal{C} \subset \textit{Adeq}(\mathcal{O})$ denote the
full subcategory consisting of parasitic adequate modules.
Then
$$
D(\textit{Adeq}(\mathcal{O}))/D_\mathcal{C}(\textit{Adeq}(\mathcal{O}))
= D(\QCoh(\mathcal{O}_S))
$$
and similarly for the bounded versions.
\end{lemma}

\begin{proof}
Follows immediately from
Derived Categories, Lemma \ref{derived-lemma-quotient-by-serre-easy}.
\end{proof}

\noindent
Next, we look for a description the other way around by looking at
the functors
$$
K^+(\QCoh(\mathcal{O}_S))
\longrightarrow
K^+(\textit{Adeq}(\mathcal{O}))
\longrightarrow
D^+(\textit{Adeq}(\mathcal{O}))
\longrightarrow
D^+(\QCoh(\mathcal{O}_S)).
$$
In some cases the derived category of adequate modules is a localization
of the homotopy category of complexes of quasi-coherent modules at
universal quasi-isomorphisms. Let $S$ be a scheme. A map of complexes
$\varphi : \mathcal{F}^\bullet \to \mathcal{G}^\bullet$
of quasi-coherent $\mathcal{O}_S$-modules is said to be a
{\it universal quasi-isomorphism} if for every morphism of schemes
$f : T \to S$ the pullback $f^*\varphi$ is a quasi-isomorphism.

\begin{lemma}
\label{lemma-describe-Dplus-adequate}
Let $U = \Spec(A)$ be an affine scheme.
The bounded below derived category
$D^+(\textit{Adeq}(\mathcal{O}))$ is the localization
of $K^+(\QCoh(\mathcal{O}_U))$ at the multiplicative subset of
universal quasi-isomorphisms.
\end{lemma}

\begin{proof}
If $\varphi : \mathcal{F}^\bullet \to \mathcal{G}^\bullet$
is a morphism of complexes of quasi-coherent
$\mathcal{O}_U$-modules, then $u\varphi : u\mathcal{F}^\bullet \to
u\mathcal{G}^\bullet$ is a quasi-isomorphism if and only if $\varphi$ is
a universal quasi-isomorphism. Hence the collection $S$
of universal quasi-isomorphisms is a saturated multiplicative
system compatible with the triangulated structure by
Derived Categories, Lemma \ref{derived-lemma-triangle-functor-localize}.
Hence $S^{-1}K^+(\QCoh(\mathcal{O}_U))$ exists and is a
triangulated category, see
Derived Categories, Proposition
\ref{derived-proposition-construct-localization}.
We obtain a canonical functor
$can : S^{-1}K^+(\QCoh(\mathcal{O}_U)) \to
D^{+}(\textit{Adeq}(\mathcal{O}))$ by
Derived Categories, Lemma \ref{derived-lemma-universal-property-localization}.

\medskip\noindent
Note that, almost by definition, every adequate module on $U$ has an
embedding into a quasi-coherent sheaf, see
Lemma \ref{lemma-adequate-characterize}. Hence by
Derived Categories, Lemma \ref{derived-lemma-subcategory-right-resolution}
given $\mathcal{F}^\bullet \in \Ob(K^+(\textit{Adeq}(\mathcal{O})))$
there exists a quasi-isomorphism
$\mathcal{F}^\bullet \to u\mathcal{G}^\bullet$
where $\mathcal{G}^\bullet \in \Ob(K^+(\QCoh(\mathcal{O}_U)))$.
This proves that $can$ is essentially surjective.

\medskip\noindent
Similarly, suppose that $\mathcal{F}^\bullet$ and $\mathcal{G}^\bullet$
are bounded below complexes of quasi-coherent $\mathcal{O}_U$-modules.
A morphism in $D^+(\textit{Adeq}(\mathcal{O}))$ between these
consists of a pair $f : u\mathcal{F}^\bullet \to \mathcal{H}^\bullet$
and $s : u\mathcal{G}^\bullet \to \mathcal{H}^\bullet$ where $s$
is a quasi-isomorphism. Pick a quasi-isomorphism
$s' : \mathcal{H}^\bullet \to u\mathcal{E}^\bullet$. Then we see that
$s' \circ f : \mathcal{F} \to \mathcal{E}^\bullet$ and the
universal quasi-isomorphism
$s' \circ s : \mathcal{G}^\bullet \to \mathcal{E}^\bullet$ give
a morphism in $S^{-1}K^{+}(\QCoh(\mathcal{O}_U))$ mapping
to the given morphism. This proves the "fully" part of full faithfulness.
Faithfulness is proved similarly.
\end{proof}

\begin{lemma}
\label{lemma-right-adjoint-adequate}
Let $U = \Spec(A)$ be an affine scheme.
The inclusion functor
$$
\textit{Adeq}(\mathcal{O}) \to
\textit{Mod}((\Sch/U)_\tau, \mathcal{O})
$$
has a right adjoint $A$\footnote{This is the ``adequator''.}.
Moreover, the adjunction mapping
$A(\mathcal{F}) \to \mathcal{F}$ is an isomorphism for every
adequate module $\mathcal{F}$.
\end{lemma}

\begin{proof}
By
Topologies, Lemma \ref{topologies-lemma-affine-big-site-fppf}
(and similarly for the other topologies)
we may work with $\mathcal{O}$-modules on $(\textit{Aff}/U)_\tau$.
Denote $\mathcal{P}$ the category of module-valued
functors on $\textit{Alg}_A$ and $\mathcal{A}$ the category of adequate
functors on $\textit{Alg}_A$. Denote $i : \mathcal{A} \to \mathcal{P}$
the inclusion functor. Denote $Q : \mathcal{P} \to \mathcal{A}$
the construction of Lemma \ref{lemma-adjoint}.
We have the commutative diagram
\begin{equation}
\label{equation-categories}
\vcenter{
\xymatrix{
\textit{Adeq}(\mathcal{O}) \ar[r]_-k \ar@{=}[d] &
\textit{Mod}((\textit{Aff}/U)_\tau, \mathcal{O}) \ar[r]_-j &
\textit{PMod}((\textit{Aff}/U)_\tau, \mathcal{O}) \ar@{=}[d] \\
\mathcal{A} \ar[rr]^-i & & \mathcal{P}
}
}
\end{equation}
The left vertical equality is
Lemma \ref{lemma-adequate-affine}
and the right vertical equality was explained in
Section \ref{section-quasi-coherent}.
Define $A(\mathcal{F}) = Q(j(\mathcal{F}))$.
Since $j$ is fully faithful it follows immediately that $A$
is a right adjoint of the inclusion functor $k$. Also, since
$k$ is fully faithful too, the final assertion follows formally.
\end{proof}

\noindent
The functor $A$ is a right adjoint hence left exact. Since the inclusion
functor is exact, see
Lemma \ref{lemma-abelian-adequate}
we conclude that $A$ transforms injectives into injectives, and that
the category $\textit{Adeq}(\mathcal{O})$ has enough injectives, see
Homology, Lemma \ref{homology-lemma-adjoint-enough-injectives}
and
Injectives, Theorem \ref{injectives-theorem-sheaves-modules-injectives}.
This also follows from the equivalence in
(\ref{equation-categories})
and
Lemma \ref{lemma-enough-injectives}.

\begin{lemma}
\label{lemma-RA-zero}
Let $U = \Spec(A)$ be an affine scheme.
For any object $\mathcal{F}$ of $\textit{Adeq}(\mathcal{O})$
we have $R^pA(\mathcal{F}) = 0$ for all $p > 0$ where $A$ is
as in
Lemma \ref{lemma-right-adjoint-adequate}.
\end{lemma}

\begin{proof}
With notation as in the proof of
Lemma \ref{lemma-right-adjoint-adequate}
choose an injective resolution $k(\mathcal{F}) \to \mathcal{I}^\bullet$
in the category of $\mathcal{O}$-modules on $(\textit{Aff}/U)_\tau$.
By
Cohomology on Sites, Lemmas \ref{sites-cohomology-lemma-include-modules}
and
Lemma \ref{lemma-same-cohomology-adequate}
the complex $j(\mathcal{I}^\bullet)$ is exact.
On the other hand, each $j(\mathcal{I}^n)$ is an injective object
of the category of presheaves of modules by
Cohomology on Sites, Lemma
\ref{sites-cohomology-lemma-injective-module-injective-presheaf}.
It follows that $R^pA(\mathcal{F}) = R^pQ(j(k(\mathcal{F})))$.
Hence the result now follows from
Lemma \ref{lemma-RQ-zero}.
\end{proof}

\noindent
Let $S$ be a scheme. By the discussion in
Section \ref{section-adequate}
the embedding
$\textit{Adeq}(\mathcal{O}) \subset
\textit{Mod}((\Sch/S)_\tau, \mathcal{O})$
exhibits $\textit{Adeq}(\mathcal{O})$ as a weak Serre subcategory of
the category of all $\mathcal{O}$-modules. Denote
$$
D_{\textit{Adeq}}(\mathcal{O}) \subset
D(\mathcal{O}) = D(\textit{Mod}((\Sch/S)_\tau, \mathcal{O}))
$$
the triangulated subcategory of complexes whose cohomology sheaves
are adequate, see
Derived Categories, Section \ref{derived-section-triangulated-sub}.
We obtain a canonical functor
$$
D(\textit{Adeq}(\mathcal{O}))
\longrightarrow
D_{\textit{Adeq}}(\mathcal{O})
$$
see
Derived Categories, Equation (\ref{derived-equation-compare}).

\begin{lemma}
\label{lemma-bounded-below}
If $U = \Spec(A)$ is an affine scheme, then the bounded
below version
\begin{equation}
\label{equation-compare-bounded-adequate}
D^+(\textit{Adeq}(\mathcal{O}))
\longrightarrow
D^+_{\textit{Adeq}}(\mathcal{O})
\end{equation}
of the functor above is an equivalence.
\end{lemma}

\begin{proof}
Let $A : \textit{Mod}(\mathcal{O}) \to \textit{Adeq}(\mathcal{O})$
be the right adjoint to the inclusion functor constructed in
Lemma \ref{lemma-right-adjoint-adequate}.
Since $A$ is left exact and since $\textit{Mod}(\mathcal{O})$
has enough injectives, $A$ has a right derived functor
$RA : D^+_{\textit{Adeq}}(\mathcal{O}) \to D^+(\textit{Adeq}(\mathcal{O}))$.
We claim that $RA$ is a quasi-inverse to
(\ref{equation-compare-bounded-adequate}).
To see this the key fact is that if $\mathcal{F}$ is an adequate module, then
the adjunction map $\mathcal{F} \to RA(\mathcal{F})$ is a
quasi-isomorphism by Lemma \ref{lemma-RA-zero}.

\medskip\noindent
Namely, to prove the lemma in full it suffices to show:
\begin{enumerate}
\item Given $\mathcal{F}^\bullet \in K^+(\textit{Adeq}(\mathcal{O}))$
the canonical map $\mathcal{F}^\bullet \to RA(\mathcal{F}^\bullet)$
is a quasi-isomorphism, and
\item given $\mathcal{G}^\bullet \in K^+(\textit{Mod}(\mathcal{O}))$
the canonical map $RA(\mathcal{G}^\bullet) \to \mathcal{G}^\bullet$
is a quasi-isomorphism.
\end{enumerate}
Both (1) and (2) follow from the key fact via a spectral sequence
argument using one of the spectral sequences of
Derived Categories, Lemma \ref{derived-lemma-two-ss-complex-functor}.
Some details omitted.
\end{proof}

\begin{lemma}
\label{lemma-ext-adequate}
Let $U = \Spec(A)$ be an affine scheme.
Let $\mathcal{F}$ and $\mathcal{G}$ be adequate $\mathcal{O}$-modules.
For any $i \geq 0$ the natural map
$$
\Ext^i_{\textit{Adeq}(\mathcal{O})}(\mathcal{F}, \mathcal{G})
\longrightarrow
\Ext^i_{\textit{Mod}(\mathcal{O})}(\mathcal{F}, \mathcal{G})
$$
is an isomorphism.
\end{lemma}

\begin{proof}
By definition these ext groups are computed as hom sets in the
derived category. Hence this follows immediately from
Lemma \ref{lemma-bounded-below}.
\end{proof}









\section{Pure extensions}
\label{section-pure}

\noindent
We want to characterize extensions of quasi-coherent sheaves on
the big site of an affine schemes in terms of algebra. To do this
we introduce the following notion.

\begin{definition}
\label{definition-pure}
Let $A$ be a ring.
\begin{enumerate}
\item An $A$-module $P$ is said to be {\it pure projective}
if for every universally exact sequence
$0 \to K \to M \to N \to 0$ of $A$-module the sequence
$0 \to \Hom_A(P, K) \to \Hom_A(P, M) \to \Hom_A(P, N) \to 0$
is exact.
\item An $A$-module $I$ is said to be {\it pure injective}
if for every universally exact sequence
$0 \to K \to M \to N \to 0$ of $A$-module the sequence
$0 \to \Hom_A(N, I) \to \Hom_A(M, I) \to \Hom_A(K, I) \to 0$
is exact.
\end{enumerate}
\end{definition}

\noindent
Let's characterize pure projectives.

\begin{lemma}
\label{lemma-pure-projective}
Let $A$ be a ring.
\begin{enumerate}
\item A module is pure projective if and only if
it is a direct summand of a direct sum of finitely presented $A$-modules.
\item For any module $M$ there exists a universally exact sequence
$0 \to N \to P \to M \to 0$ with $P$ pure projective.
\end{enumerate}
\end{lemma}

\begin{proof}
First note that a finitely presented $A$-module is pure projective by
Algebra, Theorem \ref{algebra-theorem-universally-exact-criteria}.
Hence a direct summand of a direct sum of finitely presented $A$-modules
is indeed pure projective. Let $M$ be any $A$-module.
Write $M = \colim_{i \in I} P_i$ as a filtered colimit of
finitely presented $A$-modules. Consider the sequence
$$
0 \to N \to \bigoplus P_i \to M \to 0.
$$
For any finitely presented $A$-module $P$ the map
$\Hom_A(P, \bigoplus P_i) \to \Hom_A(P, M)$
is surjective, as any map $P \to M$ factors through some $P_i$.
Hence by
Algebra, Theorem \ref{algebra-theorem-universally-exact-criteria}
this sequence is universally exact. This proves (2).
If now $M$ is pure projective, then the sequence is split and
we see that $M$ is a direct summand of $\bigoplus P_i$.
\end{proof}

\noindent
Let's characterize pure injectives.

\begin{lemma}
\label{lemma-pure-injective}
Let $A$ be a ring. For any $A$-module $M$ set
$M^\vee = \Hom_\mathbf{Z}(M, \mathbf{Q}/\mathbf{Z})$.
\begin{enumerate}
\item For any $A$-module $M$ the $A$-module $M^\vee$ is pure injective.
\item An $A$-module $I$ is pure injective if and only if the map
$I \to (I^\vee)^\vee$ splits.
\item For any module $M$ there exists a universally exact sequence
$0 \to M \to I \to N \to 0$ with $I$ pure injective.
\end{enumerate}
\end{lemma}

\begin{proof}
We will use the properties of the functor $M \mapsto M^\vee$ found in
More on Algebra, Section \ref{more-algebra-section-injectives-modules}
without further mention. Part (1) holds because
$\Hom_A(N, M^\vee) =
\Hom_\mathbf{Z}(N \otimes_A M, \mathbf{Q}/\mathbf{Z})$
and because $\mathbf{Q}/\mathbf{Z}$ is injective in the category of
abelian groups. Hence if $I \to (I^\vee)^\vee$ is split, then
$I$ is pure injective. We claim that for any $A$-module $M$ the
evaluation map $ev : M \to (M^\vee)^\vee$ is universally injective.
To see this note that $ev^\vee : ((M^\vee)^\vee)^\vee \to M^\vee$
has a right inverse, namely $ev' : M^\vee \to ((M^\vee)^\vee)^\vee$.
Then for any $A$-module $N$ applying the exact faithful functor
${}^\vee$ to the map $N \otimes_A M \to N \otimes_A (M^\vee)^\vee$
gives
$$
\Hom_A(N, ((M^\vee)^\vee)^\vee) =
\Big(N \otimes_A (M^\vee)^\vee\Big)^\vee
\to
\Big(N \otimes_A M\Big)^\vee =
\Hom_A(N, M^\vee)
$$
which is surjective by the existence of the right inverse. The claim follows.
The claim implies (3) and the necessity of the condition in (2).
\end{proof}

\noindent
Before we continue we make the following observation which we will
use frequently in the rest of this section.

\begin{lemma}
\label{lemma-split-universally-exact-sequence}
Let $A$ be a ring.
\begin{enumerate}
\item Let $L \to M \to N$ be a universally exact sequence
of $A$-modules. Let $K = \Im(M \to N)$.
Then $K \to N$ is universally injective.
\item Any universally exact complex
can be split into universally exact short exact sequences.
\end{enumerate}
\end{lemma}

\begin{proof}
Proof of (1). For any $A$-module $T$ the sequence
$L \otimes_A T \to M \otimes_A T \to K \otimes_A T \to 0$ is exact
by right exactness of $\otimes$. By assumption the sequence
$L \otimes_A T \to M \otimes_A T \to N \otimes_A T$ is exact.
Combined this shows that $K \otimes_A T \to N \otimes_A T$ is injective.

\medskip\noindent
Part (2) means the following: Suppose that $M^\bullet$ is a universally
exact complex of $A$-modules. Set $K^i = \Ker(d^i) \subset M^i$.
Then the short exact sequences $0 \to K^i \to M^i \to K^{i + 1} \to 0$
are universally exact. This follows immediately from part (1).
\end{proof}

\begin{definition}
\label{definition-pure-resolution}
Let $A$ be a ring. Let $M$ be an $A$-module.
\begin{enumerate}
\item A {\it pure projective resolution} $P_\bullet \to M$
is a universally exact sequence
$$
\ldots \to P_1 \to P_0 \to M \to 0
$$
with each $P_i$ pure projective.
\item A {\it pure injective resolution} $M \to I^\bullet$ is a universally
exact sequence
$$
0 \to M \to I^0 \to I^1 \to \ldots
$$
with each $I^i$ pure injective.
\end{enumerate}
\end{definition}

\noindent
These resolutions satisfy the usual uniqueness properties among the class
of all universally exact left or right resolutions.

\begin{lemma}
\label{lemma-pure-projective-resolutions}
Let $A$ be a ring.
\begin{enumerate}
\item Any $A$-module has a pure projective resolution.
\end{enumerate}
Let $M \to N$ be a map of $A$-modules.
Let $P_\bullet \to M$ be a pure projective resolution and
let $N_\bullet \to N$ be a universally exact resolution.
\begin{enumerate}
\item[(2)] There exists a map of complexes $P_\bullet \to N_\bullet$
inducing the given map
$$
M = \Coker(P_1 \to P_0) \to \Coker(N_1 \to N_0) = N
$$
\item[(3)] two maps $\alpha, \beta : P_\bullet \to N_\bullet$
inducing the same map $M \to N$ are homotopic.
\end{enumerate}
\end{lemma}

\begin{proof}
Part (1) follows immediately from
Lemma \ref{lemma-pure-projective}.
Before we prove (2) and (3) note that by
Lemma \ref{lemma-split-universally-exact-sequence}
we can split the universally exact complex $N_\bullet \to N \to 0$
into universally exact short exact sequences $0 \to K_0 \to N_0 \to N \to 0$
and $0 \to K_i \to N_i \to K_{i - 1} \to 0$.

\medskip\noindent
Proof of (2). Because $P_0$ is pure projective
we can find a map $P_0 \to N_0$ lifting the map $P_0 \to M \to N$.
We obtain an induced map $P_1 \to F_0 \to N_0$ which ends up in $K_0$.
Since $P_1$ is pure projective we may lift this
to a map $P_1 \to N_1$. This in turn induces a map
$P_2 \to P_1 \to N_1$ which maps to zero into
$N_0$, i.e., into $K_1$. Hence we may lift to get a map
$P_2 \to N_2$. Repeat.

\medskip\noindent
Proof of (3). To show that $\alpha, \beta$ are homotopic it suffices
to show the difference $\gamma = \alpha - \beta$ is homotopic
to zero. Note that the image of $\gamma_0 : P_0 \to N_0$
is contained in $K_0$. Hence we may lift
$\gamma_0$ to a map $h_0 : P_0 \to N_1$. Consider the map
$\gamma_1' = \gamma_1 - h_0 \circ d_{P, 1} : P_1 \to N_1$.
By our choice of $h_0$ we see that the image of $\gamma_1'$
is contained in $K_1$. Since $P_1$ is pure projective may lift
$\gamma_1'$ to a map $h_1 : P_1 \to N_2$. At this point we have
$\gamma_1 = h_0 \circ d_{F, 1} + d_{N, 2} \circ h_1$. Repeat.
\end{proof}

\begin{lemma}
\label{lemma-pure-injective-resolutions}
Let $A$ be a ring.
\begin{enumerate}
\item Any $A$-module has a pure injective resolution.
\end{enumerate}
Let $M \to N$ be a map of $A$-modules.
Let $M \to M^\bullet$ be a universally exact resolution and
let $N \to I^\bullet$ be a pure injective resolution.
\begin{enumerate}
\item[(2)] There exists a map of complexes $M^\bullet \to I^\bullet$
inducing the given map
$$
M = \Ker(M^0 \to M^1) \to \Ker(I^0 \to I^1) = N
$$
\item[(3)] two maps $\alpha, \beta : M^\bullet \to I^\bullet$
inducing the same map $M \to N$ are homotopic.
\end{enumerate}
\end{lemma}

\begin{proof}
This lemma is dual to
Lemma \ref{lemma-pure-projective-resolutions}.
The proof is identical, except one has to reverse all the arrows.
\end{proof}

\noindent
Using the material above we can define pure extension groups as
follows. Let $A$ be a ring and let $M$, $N$ be $A$-modules.
Choose a pure injective resolution $N \to I^\bullet$. By
Lemma \ref{lemma-pure-injective-resolutions}
the complex
$$
\Hom_A(M, I^\bullet)
$$
is well defined up to homotopy. Hence its $i$th cohomology module
is a well defined invariant of $M$ and $N$.

\begin{definition}
\label{definition-pure-ext}
Let $A$ be a ring and let $M$, $N$ be $A$-modules.
The $i$th {\it pure extension module} $\text{Pext}^i_A(M, N)$
is the $i$th cohomology module of the complex
$\Hom_A(M, I^\bullet)$ where $I^\bullet$ is a pure injective
resolution of $N$.
\end{definition}

\noindent
Warning: It is not true that an exact sequence of $A$-modules gives
rise to a long exact sequence of pure extensions groups. (You need
a universally exact sequence for this.)
We collect some facts which are obvious from the material above.

\begin{lemma}
\label{lemma-facts-pext}
Let $A$ be a ring.
\begin{enumerate}
\item $\text{Pext}^i_A(M, N) = 0$ for $i > 0$ whenever $N$ is pure injective,
\item $\text{Pext}^i_A(M, N) = 0$ for $i > 0$ whenever $M$ is pure projective,
in particular if $M$ is an $A$-module of finite presentation,
\item $\text{Pext}^i_A(M, N)$ is also the $i$th cohomology module
of the complex $\Hom_A(P_\bullet, N)$ where $P_\bullet$
is a pure projective resolution of $M$.
\end{enumerate}
\end{lemma}

\begin{proof}
To see (3) consider the double complex
$$
A^{\bullet, \bullet} = \Hom_A(P_\bullet, I^\bullet)
$$
Each of its rows is exact except in degree $0$ where its cohomology
is $\Hom_A(M, I^q)$. Each of its columns is exact except in degree $0$
where its cohomology is $\Hom_A(P_p, N)$. Hence the two spectral
sequences associated to this complex in
Homology, Section \ref{homology-section-double-complex}
degenerate, giving the equality.
\end{proof}





\section{Higher exts of quasi-coherent sheaves on the big site}
\label{section-big}

\noindent
It turns out that the module-valued functor $\underline{I}$ associated to
a pure injective module $I$ gives rise to an injective object in the
category of adequate functors on $\textit{Alg}_A$.
Warning: It is not true that a pure projective module gives rise to
a projective object in the category of adequate functors. We do have
plenty of projective objects, namely, the linearly adequate functors.

\begin{lemma}
\label{lemma-pure-injective-injective-adequate}
Let $A$ be a ring.
Let $\mathcal{A}$ be the category of adequate functors on $\textit{Alg}_A$.
The injective objects of $\mathcal{A}$ are exactly the functors
$\underline{I}$ where $I$ is a pure injective $A$-module.
\end{lemma}

\begin{proof}
Let $I$ be an injective object of $\mathcal{A}$.
Choose an embedding $I \to \underline{M}$ for some $A$-module $M$.
As $I$ is injective we see that $\underline{M} = I \oplus F$ for some
module-valued functor $F$. Then $M = I(A) \oplus F(A)$ and it follows
that $I = \underline{I(A)}$. Thus we see that any injective object
is of the form $\underline{I}$ for some $A$-module $I$.
It is clear that the module $I$ has to be pure injective
since any universally exact sequence $0 \to M \to N \to L \to 0$
gives rise to an exact sequence
$0 \to \underline{M} \to \underline{N} \to \underline{L} \to 0$
of $\mathcal{A}$.

\medskip\noindent
Finally, suppose that $I$ is a pure injective
$A$-module. Choose an embedding $\underline{I} \to J$
into an injective object of $\mathcal{A}$ (see
Lemma \ref{lemma-enough-injectives}).
We have seen above that $J = \underline{I'}$
for some $A$-module $I'$ which is pure injective. As
$\underline{I} \to \underline{I'}$ is injective
the map $I \to I'$ is universally injective. By assumption on $I$
it splits. Hence $\underline{I}$ is a summand of $J = \underline{I'}$
whence an injective object of the category $\mathcal{A}$.
\end{proof}

\noindent
Let $U = \Spec(A)$ be an affine scheme. Let $M$ be an $A$-module.
We will use the notation $M^a$ to denote the quasi-coherent sheaf
of $\mathcal{O}$-modules on $(\Sch/U)_\tau$ associated to
the quasi-coherent sheaf $\widetilde{M}$ on $U$.
Now we have all the notation in place to formulate the following lemma.

\begin{lemma}
\label{lemma-big-ext}
Let $U = \Spec(A)$ be an affine scheme. Let $M$, $N$ be $A$-modules.
For all $i$ we have a canonical isomorphism
$$
\Ext^i_{\textit{Mod}(\mathcal{O})}(M^a, N^a) = \text{Pext}^i_A(M, N)
$$
functorial in $M$ and $N$.
\end{lemma}

\begin{proof}
Let us construct a canonical arrow from right to left. Namely, if
$N \to I^\bullet$ is a pure injective resolution, then
$M^a \to (I^\bullet)^a$ is an exact complex of (adequate)
$\mathcal{O}$-modules. Hence any element of $\text{Pext}^i_A(M, N)$
gives rise to a map $N^a \to M^a[i]$ in $D(\mathcal{O})$, i.e.,
an element of the group on the left.

\medskip\noindent
To prove this map is an isomorphism, note that we may replace
$\Ext^i_{\textit{Mod}(\mathcal{O})}(M^a, N^a)$ by
$\Ext^i_{\textit{Adeq}(\mathcal{O})}(M^a, N^a)$, see
Lemma \ref{lemma-ext-adequate}.
Let $\mathcal{A}$ be the category of adequate functors
on $\textit{Alg}_A$. We have seen that $\mathcal{A}$ is
equivalent to $\textit{Adeq}(\mathcal{O})$, see
Lemma \ref{lemma-adequate-affine}; see also the proof of
Lemma \ref{lemma-right-adjoint-adequate}.
Hence now it suffices to prove that
$$
\Ext^i_\mathcal{A}(\underline{M}, \underline{N}) =
\text{Pext}^i_A(M, N)
$$
However, this is clear from
Lemma \ref{lemma-pure-injective-injective-adequate}
as a pure injective resolution $N \to I^\bullet$ exactly corresponds
to an injective resolution of $\underline{N}$ in $\mathcal{A}$.
\end{proof}







\section{Derived categories of adequate modules, II}
\label{section-derived-categories}

\noindent
Let $S$ be a scheme. Denote $\mathcal{O}_S$ the structure sheaf of $S$
and $\mathcal{O}$ the structure sheaf of the big site $(\Sch/S)_\tau$.
In
Descent, Remark \ref{descent-remark-change-topologies-ringed}
we constructed a morphism of ringed sites
\begin{equation}
\label{equation-compare-big-small}
f :
((\Sch/S)_\tau, \mathcal{O})
\longrightarrow
(S_{Zar}, \mathcal{O}_S).
\end{equation}
In the previous sections have seen that the functor
$f_* : \textit{Mod}(\mathcal{O}) \to \textit{Mod}(\mathcal{O}_S)$
transforms adequate sheaves into quasi-coherent sheaves, and
induces an exact functor
$v : \textit{Adeq}(\mathcal{O}) \to \QCoh(\mathcal{O}_S)$, and
in fact that $f_* = v$ induces an equivalence
$\textit{Adeq}(\mathcal{O})/\mathcal{C} \to \QCoh(\mathcal{O}_S)$
where $\mathcal{C}$ is the subcategory of parasitic adequate modules.
Moreover, the functor $f^*$ transforms quasi-coherent modules
into adequate modules, and induces a functor
$u : \QCoh(\mathcal{O}_S) \to \textit{Adeq}(\mathcal{O})$
which is a left adjoint to $v$.

\medskip\noindent
There is a very similar relationship between
$D_{\textit{Adeq}}(\mathcal{O})$ and $D_\QCoh(S)$.
First we explain why the category $D_{\textit{Adeq}}(\mathcal{O})$
is independent of the chosen topology.

\begin{remark}
\label{remark-D-adeq-independence-topology}
Let $S$ be a scheme.
Let $\tau, \tau' \in \{Zar, \etale, smooth, syntomic, fppf\}$.
Denote $\mathcal{O}_\tau$, resp.\ $\mathcal{O}_{\tau'}$
the structure sheaf $\mathcal{O}$ viewed as a sheaf on
$(\Sch/S)_\tau$, resp.\ $(\Sch/S)_{\tau'}$.
Then $D_{\textit{Adeq}}(\mathcal{O}_\tau)$ and
$D_{\textit{Adeq}}(\mathcal{O}_{\tau'})$ are canonically isomorphic.
This follows from Cohomology on Sites, Lemma
\ref{sites-cohomology-lemma-compare-topologies-derived-adequate-modules}.
Namely, assume $\tau$ is stronger than the topology $\tau'$, let
$\mathcal{C} = (\Sch/S)_{fppf}$, and let $\mathcal{B}$ the collection
of affine schemes over $S$. Assumptions (1) and (2) we've seen above.
Assumption (3) is clear and assumption (4) follows from
Lemma \ref{lemma-same-cohomology-adequate}.
\end{remark}

\begin{remark}
\label{remark-D-adeq-and-D-QCoh}
Let $S$ be a scheme. The morphism $f$ see
(\ref{equation-compare-big-small}) induces
adjoint functors
$Rf_* : D_{\textit{Adeq}}(\mathcal{O}) \to D_\QCoh(S)$
and
$Lf^* : D_\QCoh(S) \to D_{\textit{Adeq}}(\mathcal{O})$.
Moreover $Rf_* Lf^* \cong \text{id}_{D_\QCoh(S)}$.

\medskip\noindent
We sketch the proof. By
Remark \ref{remark-D-adeq-independence-topology}
we may assume the topology $\tau$ is the Zariski topology.
We will use the existence of the unbounded total derived
functors $Lf^*$ and $Rf_*$ on $\mathcal{O}$-modules and their
adjointness, see
Cohomology on Sites, Lemma \ref{sites-cohomology-lemma-adjoint}.
In this case $f_*$ is just the restriction to the subcategory
$S_{Zar}$ of $(\Sch/S)_{Zar}$. Hence it is clear that
$Rf_* = f_*$ induces
$Rf_* : D_{\textit{Adeq}}(\mathcal{O}) \to D_\QCoh(S)$.
Suppose that $\mathcal{G}^\bullet$ is an object of
$D_\QCoh(S)$. We may choose a system
$\mathcal{K}_1^\bullet \to \mathcal{K}_2^\bullet \to \ldots$
of bounded above complexes of flat $\mathcal{O}_S$-modules whose
transition maps are termwise split injectives and a diagram
$$
\xymatrix{
\mathcal{K}_1^\bullet \ar[d] \ar[r] &
\mathcal{K}_2^\bullet \ar[d] \ar[r] & \ldots \\
\tau_{\leq 1}\mathcal{G}^\bullet \ar[r] &
\tau_{\leq 2}\mathcal{G}^\bullet \ar[r] & \ldots
}
$$
with the properties (1), (2), (3) listed in
Derived Categories, Lemma \ref{derived-lemma-special-direct-system}
where $\mathcal{P}$ is the collection of flat $\mathcal{O}_S$-modules.
Then $Lf^*\mathcal{G}^\bullet$ is computed by
$\colim f^*\mathcal{K}_n^\bullet$, see
Cohomology on Sites, Lemmas \ref{sites-cohomology-lemma-pullback-K-flat} and
\ref{sites-cohomology-lemma-derived-base-change}
(note that our sites have enough points by
\'Etale Cohomology, Lemma \ref{etale-cohomology-lemma-points-fppf}).
We have to see that $H^i(Lf^*\mathcal{G}^\bullet) =
\colim H^i(f^*\mathcal{K}_n^\bullet)$ is adequate for each $i$. By
Lemma \ref{lemma-abelian-adequate}
we conclude that it suffices to show that
each $H^i(f^*\mathcal{K}_n^\bullet)$ is adequate.

\medskip\noindent
The adequacy of $H^i(f^*\mathcal{K}_n^\bullet)$ is local on $S$, hence
we may assume that $S = \Spec(A)$ is affine. Because $S$ is affine
$D_\QCoh(S) = D(\QCoh(\mathcal{O}_S))$, see
the discussion in
Derived Categories of Schemes, Section
\ref{perfect-section-derived-quasi-coherent}.
Hence there exists a quasi-isomorphism
$\mathcal{F}^\bullet \to \mathcal{K}_n^\bullet$
where $\mathcal{F}^\bullet$ is a bounded above complex of flat
quasi-coherent modules.
Then $f^*\mathcal{F}^\bullet \to f^*\mathcal{K}_n^\bullet$ is a
quasi-isomorphism, and the cohomology sheaves of
$f^*\mathcal{F}^\bullet$ are adequate.

\medskip\noindent
The final assertion
$Rf_* Lf^* \cong \text{id}_{D_\QCoh(S)}$
follows from the explicit description of the functors above.
(In plain English: if $\mathcal{F}$ is quasi-coherent and $p > 0$, then
$L_pf^*\mathcal{F}$ is a parasitic adequate module.)
\end{remark}


\begin{remark}
\label{remark-conclusion}
Remark \ref{remark-D-adeq-and-D-QCoh}
above implies we have an equivalence of derived categories
$$
D_{\textit{Adeq}}(\mathcal{O})/D_\mathcal{C}(\mathcal{O})
\longrightarrow
D_\QCoh(S)
$$
where $\mathcal{C}$ is the category of parasitic adequate modules.
Namely, it is clear that $D_\mathcal{C}(\mathcal{O})$ is the kernel
of $Rf_*$, hence a functor as indicated. For any object $X$ of
$D_{\textit{Adeq}}(\mathcal{O})$ the map $Lf^*Rf_*X \to X$ maps
to a quasi-isomorphism in $D_\QCoh(S)$, hence
$Lf^*Rf_*X \to X$ is an isomorphism in
$D_{\textit{Adeq}}(\mathcal{O})/D_\mathcal{C}(\mathcal{O})$.
Finally, for $X, Y$ objects of $D_{\textit{Adeq}}(\mathcal{O})$
the map
$$
Rf_* :
\Hom_{D_{\textit{Adeq}}(\mathcal{O})/D_\mathcal{C}(\mathcal{O})}(X, Y)
\to
\Hom_{D_\QCoh(S)}(Rf_*X, Rf_*Y)
$$
is bijective as $Lf^*$ gives an inverse (by the remarks above).
\end{remark}









\begin{multicols}{2}[\section{Other chapters}]
\noindent
Preliminaries
\begin{enumerate}
\item \hyperref[introduction-section-phantom]{Introduction}
\item \hyperref[conventions-section-phantom]{Conventions}
\item \hyperref[sets-section-phantom]{Set Theory}
\item \hyperref[categories-section-phantom]{Categories}
\item \hyperref[topology-section-phantom]{Topology}
\item \hyperref[sheaves-section-phantom]{Sheaves on Spaces}
\item \hyperref[sites-section-phantom]{Sites and Sheaves}
\item \hyperref[stacks-section-phantom]{Stacks}
\item \hyperref[fields-section-phantom]{Fields}
\item \hyperref[algebra-section-phantom]{Commutative Algebra}
\item \hyperref[brauer-section-phantom]{Brauer Groups}
\item \hyperref[homology-section-phantom]{Homological Algebra}
\item \hyperref[derived-section-phantom]{Derived Categories}
\item \hyperref[simplicial-section-phantom]{Simplicial Methods}
\item \hyperref[more-algebra-section-phantom]{More on Algebra}
\item \hyperref[smoothing-section-phantom]{Smoothing Ring Maps}
\item \hyperref[modules-section-phantom]{Sheaves of Modules}
\item \hyperref[sites-modules-section-phantom]{Modules on Sites}
\item \hyperref[injectives-section-phantom]{Injectives}
\item \hyperref[cohomology-section-phantom]{Cohomology of Sheaves}
\item \hyperref[sites-cohomology-section-phantom]{Cohomology on Sites}
\item \hyperref[dga-section-phantom]{Differential Graded Algebra}
\item \hyperref[dpa-section-phantom]{Divided Power Algebra}
\item \hyperref[hypercovering-section-phantom]{Hypercoverings}
\end{enumerate}
Schemes
\begin{enumerate}
\setcounter{enumi}{24}
\item \hyperref[schemes-section-phantom]{Schemes}
\item \hyperref[constructions-section-phantom]{Constructions of Schemes}
\item \hyperref[properties-section-phantom]{Properties of Schemes}
\item \hyperref[morphisms-section-phantom]{Morphisms of Schemes}
\item \hyperref[coherent-section-phantom]{Cohomology of Schemes}
\item \hyperref[divisors-section-phantom]{Divisors}
\item \hyperref[limits-section-phantom]{Limits of Schemes}
\item \hyperref[varieties-section-phantom]{Varieties}
\item \hyperref[topologies-section-phantom]{Topologies on Schemes}
\item \hyperref[descent-section-phantom]{Descent}
\item \hyperref[perfect-section-phantom]{Derived Categories of Schemes}
\item \hyperref[more-morphisms-section-phantom]{More on Morphisms}
\item \hyperref[flat-section-phantom]{More on Flatness}
\item \hyperref[groupoids-section-phantom]{Groupoid Schemes}
\item \hyperref[more-groupoids-section-phantom]{More on Groupoid Schemes}
\item \hyperref[etale-section-phantom]{\'Etale Morphisms of Schemes}
\end{enumerate}
Topics in Scheme Theory
\begin{enumerate}
\setcounter{enumi}{40}
\item \hyperref[chow-section-phantom]{Chow Homology}
\item \hyperref[intersection-section-phantom]{Intersection Theory}
\item \hyperref[pic-section-phantom]{Picard Schemes of Curves}
\item \hyperref[adequate-section-phantom]{Adequate Modules}
\item \hyperref[dualizing-section-phantom]{Dualizing Complexes}
\item \hyperref[duality-section-phantom]{Duality for Schemes}
\item \hyperref[discriminant-section-phantom]{Discriminants and Differents}
\item \hyperref[local-cohomology-section-phantom]{Local Cohomology}
\item \hyperref[curves-section-phantom]{Algebraic Curves}
\item \hyperref[resolve-section-phantom]{Resolution of Surfaces}
\item \hyperref[models-section-phantom]{Semistable Reduction}
\item \hyperref[pione-section-phantom]{Fundamental Groups of Schemes}
\item \hyperref[etale-cohomology-section-phantom]{\'Etale Cohomology}
\item \hyperref[ssgroups-section-phantom]{Linear Algebraic Groups}
\item \hyperref[crystalline-section-phantom]{Crystalline Cohomology}
\item \hyperref[proetale-section-phantom]{Pro-\'etale Cohomology}
\end{enumerate}
Algebraic Spaces
\begin{enumerate}
\setcounter{enumi}{56}
\item \hyperref[spaces-section-phantom]{Algebraic Spaces}
\item \hyperref[spaces-properties-section-phantom]{Properties of Algebraic Spaces}
\item \hyperref[spaces-morphisms-section-phantom]{Morphisms of Algebraic Spaces}
\item \hyperref[decent-spaces-section-phantom]{Decent Algebraic Spaces}
\item \hyperref[spaces-cohomology-section-phantom]{Cohomology of Algebraic Spaces}
\item \hyperref[spaces-limits-section-phantom]{Limits of Algebraic Spaces}
\item \hyperref[spaces-divisors-section-phantom]{Divisors on Algebraic Spaces}
\item \hyperref[spaces-over-fields-section-phantom]{Algebraic Spaces over Fields}
\item \hyperref[spaces-topologies-section-phantom]{Topologies on Algebraic Spaces}
\item \hyperref[spaces-descent-section-phantom]{Descent and Algebraic Spaces}
\item \hyperref[spaces-perfect-section-phantom]{Derived Categories of Spaces}
\item \hyperref[spaces-more-morphisms-section-phantom]{More on Morphisms of Spaces}
\item \hyperref[spaces-flat-section-phantom]{Flatness on Algebraic Spaces}
\item \hyperref[spaces-groupoids-section-phantom]{Groupoids in Algebraic Spaces}
\item \hyperref[spaces-more-groupoids-section-phantom]{More on Groupoids in Spaces}
\item \hyperref[bootstrap-section-phantom]{Bootstrap}
\item \hyperref[spaces-pushouts-section-phantom]{Pushouts of Algebraic Spaces}
\end{enumerate}
Topics in Geometry
\begin{enumerate}
\setcounter{enumi}{73}
\item \hyperref[spaces-chow-section-phantom]{Chow Groups of Spaces}
\item \hyperref[groupoids-quotients-section-phantom]{Quotients of Groupoids}
\item \hyperref[spaces-more-cohomology-section-phantom]{More on Cohomology of Spaces}
\item \hyperref[spaces-simplicial-section-phantom]{Simplicial Spaces}
\item \hyperref[spaces-duality-section-phantom]{Duality for Spaces}
\item \hyperref[formal-spaces-section-phantom]{Formal Algebraic Spaces}
\item \hyperref[restricted-section-phantom]{Restricted Power Series}
\item \hyperref[spaces-resolve-section-phantom]{Resolution of Surfaces Revisited}
\end{enumerate}
Deformation Theory
\begin{enumerate}
\setcounter{enumi}{81}
\item \hyperref[formal-defos-section-phantom]{Formal Deformation Theory}
\item \hyperref[defos-section-phantom]{Deformation Theory}
\item \hyperref[cotangent-section-phantom]{The Cotangent Complex}
\item \hyperref[examples-defos-section-phantom]{Deformation Problems}
\end{enumerate}
Algebraic Stacks
\begin{enumerate}
\setcounter{enumi}{85}
\item \hyperref[algebraic-section-phantom]{Algebraic Stacks}
\item \hyperref[examples-stacks-section-phantom]{Examples of Stacks}
\item \hyperref[stacks-sheaves-section-phantom]{Sheaves on Algebraic Stacks}
\item \hyperref[criteria-section-phantom]{Criteria for Representability}
\item \hyperref[artin-section-phantom]{Artin's Axioms}
\item \hyperref[quot-section-phantom]{Quot and Hilbert Spaces}
\item \hyperref[stacks-properties-section-phantom]{Properties of Algebraic Stacks}
\item \hyperref[stacks-morphisms-section-phantom]{Morphisms of Algebraic Stacks}
\item \hyperref[stacks-limits-section-phantom]{Limits of Algebraic Stacks}
\item \hyperref[stacks-cohomology-section-phantom]{Cohomology of Algebraic Stacks}
\item \hyperref[stacks-perfect-section-phantom]{Derived Categories of Stacks}
\item \hyperref[stacks-introduction-section-phantom]{Introducing Algebraic Stacks}
\item \hyperref[stacks-more-morphisms-section-phantom]{More on Morphisms of Stacks}
\item \hyperref[stacks-geometry-section-phantom]{The Geometry of Stacks}
\end{enumerate}
Topics in Moduli Theory
\begin{enumerate}
\setcounter{enumi}{99}
\item \hyperref[moduli-section-phantom]{Moduli Stacks}
\item \hyperref[moduli-curves-section-phantom]{Moduli of Curves}
\end{enumerate}
Miscellany
\begin{enumerate}
\setcounter{enumi}{101}
\item \hyperref[examples-section-phantom]{Examples}
\item \hyperref[exercises-section-phantom]{Exercises}
\item \hyperref[guide-section-phantom]{Guide to Literature}
\item \hyperref[desirables-section-phantom]{Desirables}
\item \hyperref[coding-section-phantom]{Coding Style}
\item \hyperref[obsolete-section-phantom]{Obsolete}
\item \hyperref[fdl-section-phantom]{GNU Free Documentation License}
\item \hyperref[index-section-phantom]{Auto Generated Index}
\end{enumerate}
\end{multicols}


\bibliography{my}
\bibliographystyle{amsalpha}

\end{document}
