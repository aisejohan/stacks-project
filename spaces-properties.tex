\IfFileExists{stacks-project.cls}{%
\documentclass{stacks-project}
}{%
\documentclass{amsart}
}

% The following AMS packages are automatically loaded with
% the amsart documentclass:
%\usepackage{amsmath}
%\usepackage{amssymb}
%\usepackage{amsthm}

\usepackage{graphicx}

% For dealing with references we use the comment environment
\usepackage{verbatim}
\newenvironment{reference}{\comment}{\endcomment}
%\newenvironment{reference}{}{}
\newenvironment{slogan}{\comment}{\endcomment}
\newenvironment{history}{\comment}{\endcomment}

% For commutative diagrams you can use
% \usepackage{amscd}
\usepackage[all]{xy}

% We use 2cell for 2-commutative diagrams.
\xyoption{2cell}
\UseAllTwocells

% To put source file link in headers.
% Change "template.tex" to "this_filename.tex"
% \usepackage{fancyhdr}
% \pagestyle{fancy}
% \lhead{}
% \chead{}
% \rhead{Source file: \url{template.tex}}
% \lfoot{}
% \cfoot{\thepage}
% \rfoot{}
% \renewcommand{\headrulewidth}{0pt}
% \renewcommand{\footrulewidth}{0pt}
% \renewcommand{\headheight}{12pt}

\usepackage{multicol}

% For cross-file-references
\usepackage{xr-hyper}

% Package for hypertext links:
\usepackage{hyperref}

% For any local file, say "hello.tex" you want to link to please
% use \externaldocument[hello-]{hello}
\externaldocument[introduction-]{introduction}
\externaldocument[conventions-]{conventions}
\externaldocument[sets-]{sets}
\externaldocument[categories-]{categories}
\externaldocument[topology-]{topology}
\externaldocument[sheaves-]{sheaves}
\externaldocument[sites-]{sites}
\externaldocument[stacks-]{stacks}
\externaldocument[fields-]{fields}
\externaldocument[algebra-]{algebra}
\externaldocument[brauer-]{brauer}
\externaldocument[homology-]{homology}
\externaldocument[derived-]{derived}
\externaldocument[simplicial-]{simplicial}
\externaldocument[more-algebra-]{more-algebra}
\externaldocument[smoothing-]{smoothing}
\externaldocument[modules-]{modules}
\externaldocument[sites-modules-]{sites-modules}
\externaldocument[injectives-]{injectives}
\externaldocument[cohomology-]{cohomology}
\externaldocument[sites-cohomology-]{sites-cohomology}
\externaldocument[dga-]{dga}
\externaldocument[dpa-]{dpa}
\externaldocument[hypercovering-]{hypercovering}
\externaldocument[schemes-]{schemes}
\externaldocument[constructions-]{constructions}
\externaldocument[properties-]{properties}
\externaldocument[morphisms-]{morphisms}
\externaldocument[coherent-]{coherent}
\externaldocument[divisors-]{divisors}
\externaldocument[limits-]{limits}
\externaldocument[varieties-]{varieties}
\externaldocument[topologies-]{topologies}
\externaldocument[descent-]{descent}
\externaldocument[perfect-]{perfect}
\externaldocument[more-morphisms-]{more-morphisms}
\externaldocument[flat-]{flat}
\externaldocument[groupoids-]{groupoids}
\externaldocument[more-groupoids-]{more-groupoids}
\externaldocument[etale-]{etale}
\externaldocument[chow-]{chow}
\externaldocument[intersection-]{intersection}
\externaldocument[pic-]{pic}
\externaldocument[adequate-]{adequate}
\externaldocument[dualizing-]{dualizing}
\externaldocument[duality-]{duality}
\externaldocument[discriminant-]{discriminant}
\externaldocument[local-cohomology-]{local-cohomology}
\externaldocument[curves-]{curves}
\externaldocument[resolve-]{resolve}
\externaldocument[models-]{models}
\externaldocument[pione-]{pione}
\externaldocument[etale-cohomology-]{etale-cohomology}
\externaldocument[ssgroups-]{ssgroups}
\externaldocument[proetale-]{proetale}
\externaldocument[crystalline-]{crystalline}
\externaldocument[spaces-]{spaces}
\externaldocument[spaces-properties-]{spaces-properties}
\externaldocument[spaces-morphisms-]{spaces-morphisms}
\externaldocument[decent-spaces-]{decent-spaces}
\externaldocument[spaces-cohomology-]{spaces-cohomology}
\externaldocument[spaces-limits-]{spaces-limits}
\externaldocument[spaces-divisors-]{spaces-divisors}
\externaldocument[spaces-over-fields-]{spaces-over-fields}
\externaldocument[spaces-topologies-]{spaces-topologies}
\externaldocument[spaces-descent-]{spaces-descent}
\externaldocument[spaces-perfect-]{spaces-perfect}
\externaldocument[spaces-more-morphisms-]{spaces-more-morphisms}
\externaldocument[spaces-flat-]{spaces-flat}
\externaldocument[spaces-groupoids-]{spaces-groupoids}
\externaldocument[spaces-more-groupoids-]{spaces-more-groupoids}
\externaldocument[bootstrap-]{bootstrap}
\externaldocument[spaces-pushouts-]{spaces-pushouts}
\externaldocument[spaces-chow-]{spaces-chow}
\externaldocument[groupoids-quotients-]{groupoids-quotients}
\externaldocument[spaces-more-cohomology-]{spaces-more-cohomology}
\externaldocument[spaces-simplicial-]{spaces-simplicial}
\externaldocument[spaces-duality-]{spaces-duality}
\externaldocument[formal-spaces-]{formal-spaces}
\externaldocument[restricted-]{restricted}
\externaldocument[spaces-resolve-]{spaces-resolve}
\externaldocument[formal-defos-]{formal-defos}
\externaldocument[defos-]{defos}
\externaldocument[cotangent-]{cotangent}
\externaldocument[examples-defos-]{examples-defos}
\externaldocument[algebraic-]{algebraic}
\externaldocument[examples-stacks-]{examples-stacks}
\externaldocument[stacks-sheaves-]{stacks-sheaves}
\externaldocument[criteria-]{criteria}
\externaldocument[artin-]{artin}
\externaldocument[quot-]{quot}
\externaldocument[stacks-properties-]{stacks-properties}
\externaldocument[stacks-morphisms-]{stacks-morphisms}
\externaldocument[stacks-limits-]{stacks-limits}
\externaldocument[stacks-cohomology-]{stacks-cohomology}
\externaldocument[stacks-perfect-]{stacks-perfect}
\externaldocument[stacks-introduction-]{stacks-introduction}
\externaldocument[stacks-more-morphisms-]{stacks-more-morphisms}
\externaldocument[stacks-geometry-]{stacks-geometry}
\externaldocument[moduli-]{moduli}
\externaldocument[moduli-curves-]{moduli-curves}
\externaldocument[examples-]{examples}
\externaldocument[exercises-]{exercises}
\externaldocument[guide-]{guide}
\externaldocument[desirables-]{desirables}
\externaldocument[coding-]{coding}
\externaldocument[obsolete-]{obsolete}
\externaldocument[fdl-]{fdl}
\externaldocument[index-]{index}

% Theorem environments.
%
\theoremstyle{plain}
\newtheorem{theorem}[subsection]{Theorem}
\newtheorem{proposition}[subsection]{Proposition}
\newtheorem{lemma}[subsection]{Lemma}

\theoremstyle{definition}
\newtheorem{definition}[subsection]{Definition}
\newtheorem{example}[subsection]{Example}
\newtheorem{exercise}[subsection]{Exercise}
\newtheorem{situation}[subsection]{Situation}

\theoremstyle{remark}
\newtheorem{remark}[subsection]{Remark}
\newtheorem{remarks}[subsection]{Remarks}

\numberwithin{equation}{subsection}

% Macros
%
\def\lim{\mathop{\mathrm{lim}}\nolimits}
\def\colim{\mathop{\mathrm{colim}}\nolimits}
\def\Spec{\mathop{\mathrm{Spec}}}
\def\Hom{\mathop{\mathrm{Hom}}\nolimits}
\def\Ext{\mathop{\mathrm{Ext}}\nolimits}
\def\SheafHom{\mathop{\mathcal{H}\!\mathit{om}}\nolimits}
\def\SheafExt{\mathop{\mathcal{E}\!\mathit{xt}}\nolimits}
\def\Sch{\mathit{Sch}}
\def\Mor{\mathop{Mor}\nolimits}
\def\Ob{\mathop{\mathrm{Ob}}\nolimits}
\def\Sh{\mathop{\mathit{Sh}}\nolimits}
\def\NL{\mathop{N\!L}\nolimits}
\def\proetale{{pro\text{-}\acute{e}tale}}
\def\etale{{\acute{e}tale}}
\def\QCoh{\mathit{QCoh}}
\def\Ker{\mathop{\mathrm{Ker}}}
\def\Im{\mathop{\mathrm{Im}}}
\def\Coker{\mathop{\mathrm{Coker}}}
\def\Coim{\mathop{\mathrm{Coim}}}
\def\id{\mathop{\mathrm{id}}\nolimits}

%
% Macros for linear algebraic groups
%
\def\SL{\mathop{\mathrm{SL}}\nolimits}
\def\GL{\mathop{\mathrm{GL}}\nolimits}
\def\ltimes{{\mathchar"256E}}
\def\rtimes{{\mathchar"256F}}
\def\Rrightarrow{{\mathchar"3456}}

%
% Macros for moduli stacks/spaces
%
\def\QCohstack{\mathcal{QC}\!\mathit{oh}}
\def\Cohstack{\mathcal{C}\!\mathit{oh}}
\def\Spacesstack{\mathcal{S}\!\mathit{paces}}
\def\Quotfunctor{\mathrm{Quot}}
\def\Hilbfunctor{\mathrm{Hilb}}
\def\Curvesstack{\mathcal{C}\!\mathit{urves}}
\def\Polarizedstack{\mathcal{P}\!\mathit{olarized}}
\def\Complexesstack{\mathcal{C}\!\mathit{omplexes}}
% \Pic is the operator that assigns to X its picard group, usage \Pic(X)
% \Picardstack_{X/B} denotes the Picard stack of X over B
% \Picardfunctor_{X/B} denotes the Picard functor of X over B
\def\Pic{\mathop{\mathrm{Pic}}\nolimits}
\def\Picardstack{\mathcal{P}\!\mathit{ic}}
\def\Picardfunctor{\mathrm{Pic}}
\def\Deformationcategory{\mathcal{D}\!\mathit{ef}}


% OK, start here.
%
\begin{document}

\title{Properties of Algebraic Spaces}


\maketitle

\phantomsection
\label{section-phantom}

\tableofcontents

\section{Introduction}
\label{section-introduction}

\noindent
Please see
Spaces, Section \ref{spaces-section-introduction}
for a brief introduction to algebraic spaces, and please read
some of that chapter for our basic definitions and conventions
concerning algebraic spaces. In this chapter we start introducing
some basic notions and properties of algebraic spaces. A fundamental
reference for the case of quasi-separated algebraic spaces is
\cite{Kn}.

\medskip\noindent
The discussion is somewhat awkward at times since we made the design
decision to first talk about properties of algebraic spaces by
themselves, and only later about properties of morphisms of algebraic
spaces. We make an exception for this rule regarding
{\it \'etale morphisms} of algebraic spaces, which we introduce in
Section \ref{section-etale-morphisms}.
But until that section whenever we say a morphism has a certain property, it
automatically means the source of the morphism is a scheme (or perhaps the
morphism is representable).

\medskip\noindent
Some of the material in the chapter (especially regarding points)
will be improved upon in the chapter on decent algebraic spaces.


\section{Conventions}
\label{section-conventions}

\noindent
The standing assumption is that all schemes are contained in
a big fppf site $\Sch_{fppf}$. And all rings $A$ considered
have the property that $\Spec(A)$ is (isomorphic) to an
object of this big site.

\medskip\noindent
Let $S$ be a scheme and let $X$ be an algebraic space over $S$.
In this chapter and the following we will write $X \times_S X$
for the product of $X$ with itself (in the category of algebraic
spaces over $S$), instead of $X \times X$. The reason is that we
want to avoid confusion when changing base schemes, as in
Spaces, Section \ref{spaces-section-change-base-scheme}.


\section{Separation axioms}
\label{section-separation}

\noindent
In this section we collect all the ``absolute'' separation conditions
of algebraic spaces. Since in our language any algebraic space is an
algebraic space over some definite base scheme, any absolute property
of $X$ over $S$ corresponds to a conditions imposed on $X$ viewed
as an algebraic space over $\Spec(\mathbf{Z})$. Here is the
precise formulation.

\begin{definition}
\label{definition-separated}
(Compare Spaces, Definition \ref{spaces-definition-separated}.)
Consider a big fppf site
$\Sch_{fppf} = (\Sch/\Spec(\mathbf{Z}))_{fppf}$.
Let $X$ be an algebraic space over
$\Spec(\mathbf{Z})$. Let $\Delta : X \to X \times X$
be the diagonal morphism.
\begin{enumerate}
\item We say $X$ is {\it separated} if $\Delta$ is a closed immersion.
\item We say $X$ is {\it locally separated}\footnote{In the
literature this often refers to quasi-separated and locally
separated algebraic spaces.} if $\Delta$ is an
immersion.
\item We say $X$ is {\it quasi-separated} if $\Delta$ is quasi-compact.
\item We say $X$ is {\it Zariski locally quasi-separated}\footnote{
This notion was suggested by B.\ Conrad.} if there
exists a Zariski covering $X = \bigcup_{i \in I} X_i$ (see Spaces,
Definition \ref{spaces-definition-Zariski-open-covering}) such that
each $X_i$ is quasi-separated.
\end{enumerate}
Let $S$ is a scheme contained in $\Sch_{fppf}$, and let
$X$ be an algebraic space over $S$. Then we say $X$ is {\it separated},
{\it locally separated}, {\it quasi-separated}, or
{\it Zariski locally quasi-separated}
if $X$ viewed as an algebraic space over $\Spec(\mathbf{Z})$ (see
Spaces, Definition \ref{spaces-definition-base-change})
has the corresponding property.
\end{definition}

\noindent
It is true that an algebraic space $X$ over $S$ which is separated
(in the absolute sense above) is separated over $S$ (and similarly
for the other absolute separation properties above). This will be discussed
in great detail in
Morphisms of Spaces, Section \ref{spaces-morphisms-section-separation-axioms}.
We will see in
Lemma \ref{lemma-quasi-separated-quasi-compact-pieces}
that being Zariski locally separated is independent of the base scheme
(hence equivalent to the absolute notion).

\begin{lemma}
\label{lemma-trivial-implications}
Let $S$ be a scheme.
Let $X$ be an algebraic space over $S$.
We have the following implications among the separation axioms
of Definition \ref{definition-separated}:
\begin{enumerate}
\item separated implies all the others,
\item quasi-separated implies Zariski locally quasi-separated.
\end{enumerate}
\end{lemma}

\begin{proof}
Omitted.
\end{proof}

\begin{lemma}
\label{lemma-characterize-quasi-separated}
Let $S$ be a scheme. Let $X$ be an algebraic space over $S$.
The following are equivalent
\begin{enumerate}
\item $X$ is a quasi-separated algebraic space,
\item for $U \to X$, $V \to X$ with $U$, $V$ quasi-compact schemes
the fibre product $U \times_X V$ is quasi-compact,
\item for $U \to X$, $V \to X$ with $U$, $V$ affine
the fibre product $U \times_X V$ is quasi-compact.
\end{enumerate}
\end{lemma}

\begin{proof}
Using Spaces, Lemma
\ref{spaces-lemma-category-of-spaces-over-smaller-base-scheme}
we see that we may assume $S = \Spec(\mathbf{Z})$.
Since $U \times_X V = X \times_{X \times X} (U \times V)$
and since $U \times V$ is quasi-compact if $U$ and $V$ are so,
we see that (1) implies (2). It is clear that (2) implies (3).
Assume (3). Choose a scheme $W$ and a surjective \'etale morphism
$W \to X$. Then $W \times W \to X \times X$ is surjective \'etale.
Hence it suffices to show that
$$
j : W \times_X W = X \times_{(X \times X)} (W \times W) \to W \times W
$$
is quasi-compact, see Spaces, Lemma
\ref{spaces-lemma-descent-representable-transformations-property}.
If $U \subset W$ and $V \subset W$ are affine opens, then
$j^{-1}(U \times V) = U \times_X V$ is quasi-compact by assumption.
Since the affine opens $U \times V$ form an affine open covering of
$W \times W$ (Schemes, Lemma \ref{schemes-lemma-affine-covering-fibre-product})
we conclude by
Schemes, Lemma \ref{schemes-lemma-quasi-compact-affine}.
\end{proof}

\begin{lemma}
\label{lemma-characterize-separated}
Let $S$ be a scheme. Let $X$ be an algebraic space over $S$.
The following are equivalent
\begin{enumerate}
\item $X$ is a separated algebraic space,
\item for $U \to X$, $V \to X$ with $U$, $V$ affine
the fibre product $U \times_X V$ is affine and
$$
\mathcal{O}(U) \otimes_\mathbf{Z} \mathcal{O}(V)
\longrightarrow
\mathcal{O}(U \times_X V)
$$
is surjective.
\end{enumerate}
\end{lemma}

\begin{proof}
Using Spaces, Lemma
\ref{spaces-lemma-category-of-spaces-over-smaller-base-scheme}
we see that we may assume $S = \Spec(\mathbf{Z})$.
Since $U \times_X V = X \times_{X \times X} (U \times V)$
and since $U \times V$ is affine if $U$ and $V$ are so,
we see that (1) implies (2).
Assume (2). Choose a scheme $W$ and a surjective \'etale morphism
$W \to X$. Then $W \times W \to X \times X$ is surjective \'etale.
Hence it suffices to show that
$$
j : W \times_X W = X \times_{(X \times X)} (W \times W) \to W \times W
$$
is a closed immersion, see Spaces, Lemma
\ref{spaces-lemma-descent-representable-transformations-property}.
If $U \subset W$ and $V \subset W$ are affine opens, then
$j^{-1}(U \times V) = U \times_X V$ is affine by assumption
and the map $U \times_X V \to U \times V$ is a closed immersion
because the corresponding ring map is surjective.
Since the affine opens $U \times V$ form an affine open covering
of $W \times W$
(Schemes, Lemma \ref{schemes-lemma-affine-covering-fibre-product})
we conclude by
Morphisms, Lemma \ref{morphisms-lemma-closed-immersion}.
\end{proof}







\section{Points of algebraic spaces}
\label{section-points}

\noindent
As is clear from
Spaces, Example \ref{spaces-example-affine-line-translation}
a point of an algebraic space should not be defined as a monomorphism
from the spectrum of a field.
Instead we define them as equivalence classes of morphisms of spectra
of fields exactly as explained in
Schemes, Section \ref{schemes-section-points}.

\medskip\noindent
Let $S$ be a scheme.
Let $F$ be a presheaf on $(\Sch/S)_{fppf}$.
Let $K$ be a field. Consider a morphism
$$
\Spec(K) \longrightarrow F.
$$
By the Yoneda Lemma this is given by an
element $p \in F(\Spec(K))$. We say that two such
pairs $(\Spec(K), p)$ and $(\Spec(L), q)$
are {\it equivalent} if there exists
a third field $\Omega$ and a commutative diagram
$$
\xymatrix{
\Spec(\Omega) \ar[r] \ar[d] &
\Spec(L) \ar[d]^q \\
\Spec(K) \ar[r]^p &
F.
}
$$
In other words, there are field extensions
$K \to \Omega$ and $L \to \Omega$ such that
$p$ and $q$ map to the same element
of $F(\Spec(\Omega))$. We omit the verification that this
defines an equivalence relation.

\begin{definition}
\label{definition-points}
Let $S$ be a scheme. Let $X$ be an algebraic space over $S$.
A {\it point} of $X$ is an equivalence class of morphisms
from spectra of fields into $X$.
The set of points of $X$ is denoted $|X|$.
\end{definition}

\noindent
Note that if $f : X \to Y$ is a morphism of algebraic spaces
over $S$, then there is an induced map $|f| : |X| \to |Y|$ which
maps a representative $x : \Spec(K) \to X$ to the representative
$f \circ x : \Spec(K) \to Y$.

\begin{lemma}
\label{lemma-scheme-points}
Let $S$ be a scheme. Let $X$ be a scheme over $S$.
The points of $X$ as a scheme are in canonical 1-1 correspondence
with the points of $X$ as an algebraic space.
\end{lemma}

\begin{proof}
This is Schemes, Lemma \ref{schemes-lemma-characterize-points}.
\end{proof}

\begin{lemma}
\label{lemma-points-cartesian}
Let $S$ be a scheme. Let
$$
\xymatrix{
Z \times_Y X \ar[r] \ar[d] & X \ar[d] \\
Z \ar[r] & Y
}
$$
be a cartesian diagram of algebraic spaces over $S$. Then the map of sets
of points
$$
|Z \times_Y X|
\longrightarrow
|Z| \times_{|Y|} |X|
$$
is surjective.
\end{lemma}

\begin{proof}
Namely, suppose given fields $K$, $L$ and morphisms
$\Spec(K) \to X$, $\Spec(L) \to Z$, then the
assumption that they agree as elements of $|Y|$ means that
there is a common extension $M/K$ and $M/L$
such that
$\Spec(M) \to \Spec(K) \to X \to Y$ and
$\Spec(M) \to \Spec(L) \to Z \to Y$ agree.
And this is exactly the condition which says you get a
morphism $\Spec(M) \to Z \times_Y X$.
\end{proof}

\begin{lemma}
\label{lemma-characterize-surjective}
Let $S$ be a scheme.
Let $X$ be an algebraic space over $S$.
Let $f : T \to X$ be a morphism from a scheme to $X$.
The following are equivalent
\begin{enumerate}
\item $f : T \to X$ is surjective (according to
Spaces, Definition \ref{spaces-definition-relative-representable-property}),
and
\item $|f| : |T| \to |X|$ is surjective.
\end{enumerate}
\end{lemma}

\begin{proof}
Assume (1). Let $x : \Spec(K) \to X$ be a morphism
from the spectrum of a field into $X$. By assumption the morphism of
schemes $\Spec(K) \times_X T \to \Spec(K)$ is surjective.
Hence there exists a field extension $K'/K$ and a morphism
$\Spec(K') \to \Spec(K) \times_X T$ such that the left
square in the diagram
$$
\xymatrix{
\Spec(K') \ar[r] \ar[d] &
\Spec(K) \times_X T \ar[d] \ar[r] &
T \ar[d] \\
\Spec(K) \ar@{=}[r] &
\Spec(K) \ar[r]^-x & X
}
$$
is commutative. This shows that $|f| : |T| \to |X|$ is surjective.

\medskip\noindent
Assume (2). Let $Z \to X$ be a morphism where $Z$ is
a scheme. We have to show that the morphism of schemes $Z \times_X T \to T$
is surjective, i.e., that $|Z \times_X T| \to |Z|$ is surjective.
This follows from (2) and
Lemma \ref{lemma-points-cartesian}.
\end{proof}

\begin{lemma}
\label{lemma-points-presentation}
Let $S$ be a scheme.
Let $X$ be an algebraic space over $S$.
Let $X = U/R$ be a presentation of $X$, see
Spaces, Definition \ref{spaces-definition-presentation}.
Then the image of $|R| \to |U| \times |U|$ is an equivalence relation
and $|X|$ is the quotient of $|U|$ by this equivalence relation.
\end{lemma}

\begin{proof}
The assumption means that $U$ is a scheme, $p : U \to X$ is a surjective,
\'etale morphism, $R = U \times_X U$ is a scheme and defines an \'etale
equivalence relation on $U$ such that $X = U/R$ as sheaves. By
Lemma \ref{lemma-characterize-surjective}
we see that $|U| \to |X|$ is surjective. By
Lemma \ref{lemma-points-cartesian}
the map
$$
|R| \longrightarrow |U| \times_{|X|} |U|
$$
is surjective. Hence the image of $|R| \to |U| \times |U|$ is
exactly the set of pairs $(u_1, u_2) \in |U| \times |U|$
such that $u_1$ and $u_2$ have the same image in $|X|$.
Combining these two statements we get the result of the lemma.
\end{proof}

\begin{lemma}
\label{lemma-topology-points}
Let $S$ be a scheme. There exists a unique topology on the sets of points
of algebraic spaces over $S$ with the following properties:
\begin{enumerate}
\item if $X$ is a scheme over $S$, then the topology on $|X|$ is the usual one
(via the identification of Lemma \ref{lemma-scheme-points}),
\item for every morphism of algebraic spaces $X \to Y$ over $S$
the map $|X| \to |Y|$ is continuous, and
\item for every \'etale morphism $U \to X$ with $U$ a scheme
the map of topological spaces $|U| \to |X|$ is continuous and open.
\end{enumerate}
\end{lemma}

\begin{proof}
Let $X$ be an algebraic space over $S$. Let $p : U \to X$ be a
surjective \'etale morphism where $U$ is a scheme over $S$.
We define $W \subset |X|$ is open if and only if $|p|^{-1}(W)$
is an open subset of $|U|$. This is a topology on $|X|$
(it is the quotient topology on $|X|$, see
Topology, Lemma \ref{topology-lemma-quotient}).

\medskip\noindent
Let us prove that the topology is independent of the choice of
the presentation. To do this it suffices to show that if $U'$ is a scheme,
and $U' \to X$ is an \'etale morphism, then the map $|U'| \to |X|$
(with topology on $|X|$ defined using $U \to X$ as above)
is open and continuous; which in addition will prove that (3) holds.
Set $U'' = U \times_X U'$, so that we have the commutative diagram
$$
\xymatrix{
U'' \ar[r] \ar[d] & U' \ar[d] \\
U \ar[r] & X
}
$$
As $U \to X$ and $U' \to X$ are \'etale we see that
both $U'' \to U$ and $U'' \to U'$ are \'etale morphisms of schemes.
Moreover, $U'' \to U'$ is surjective. Hence
we get a commutative diagram of maps of sets
$$
\xymatrix{
|U''| \ar[r] \ar[d] & |U'| \ar[d] \\
|U| \ar[r] & |X|
}
$$
The lower horizontal arrow is surjective (see
Lemma \ref{lemma-characterize-surjective}
or
Lemma \ref{lemma-points-presentation})
and continuous by definition of the topology on $|X|$.
The top horizontal arrow is surjective, continuous, and open by
Morphisms, Lemma \ref{morphisms-lemma-etale-open}.
The left vertical arrow is continuous and open (by
Morphisms, Lemma \ref{morphisms-lemma-etale-open}
again.) Hence it follows formally that the right vertical
arrow is continuous and open.

\medskip\noindent
To finish the proof we prove (2).
Let $a : X \to Y$ be a morphism of algebraic spaces. According to
Spaces, Lemma \ref{spaces-lemma-lift-morphism-presentations}
we can find a diagram
$$
\xymatrix{
U \ar[d]_p \ar[r]_\alpha & V \ar[d]^q \\
X \ar[r]^a & Y
}
$$
where $U$ and $V$ are schemes, and $p$ and $q$ are surjective and \'etale.
This gives rise to the diagram
$$
\xymatrix{
|U| \ar[d]_p \ar[r]_\alpha & |V| \ar[d]^q \\
|X| \ar[r]^a & |Y|
}
$$
where all but the lower horizontal arrows are known to be continuous and
the two vertical arrows are surjective and open. It follows that the
lower horizontal arrow is continuous as desired.
\end{proof}

\begin{definition}
\label{definition-topological-space}
Let $S$ be a scheme. Let $X$ be an algebraic space over $S$.
The underlying {\it topological space} of $X$ is the set of points
$|X|$ endowed with the topology constructed in
Lemma \ref{lemma-topology-points}.
\end{definition}

\noindent
It turns out that this topological space carries the same information
as the small Zariski site $X_{Zar}$ of
Spaces, Definition \ref{spaces-definition-small-Zariski-site}.

\begin{lemma}
\label{lemma-open-subspaces}
Let $S$ be a scheme.
Let $X$ be an algebraic space over $S$.
\begin{enumerate}
\item The rule $X' \mapsto |X'|$ defines an inclusion preserving
bijection between open subspaces $X'$ (see
Spaces, Definition \ref{spaces-definition-immersion})
of $X$, and opens of the topological space $|X|$.
\item A family $\{X_i \subset X\}_{i \in I}$ of open subspaces of $X$
is a Zariski covering (see
Spaces, Definition \ref{spaces-definition-Zariski-open-covering})
if and only if $|X| = \bigcup |X_i|$.
\end{enumerate}
In other words, the small Zariski site $X_{Zar}$ of $X$ is canonically
identified with a site associated to the topological space $|X|$ (see
Sites, Example \ref{sites-example-site-topological}).
\end{lemma}

\begin{proof}
In order to prove (1) let us construct the inverse of the rule.
Namely, suppose that $W \subset |X|$ is open. Choose a presentation
$X = U/R$ corresponding to the surjective \'etale map
$p : U \to X$ and \'etale maps $s, t : R \to U$.
By construction we see that $|p|^{-1}(W)$ is an
open of $U$. Denote $W' \subset U$ the corresponding open subscheme.
It is clear that $R' = s^{-1}(W') = t^{-1}(W')$ is a Zariski open
of $R$ which defines an \'etale equivalence relation on $W'$.
By Spaces, Lemma \ref{spaces-lemma-finding-opens} the morphism
$X' = W'/R' \to X$ is an open immersion. Hence $X'$ is an algebraic space
by Spaces, Lemma \ref{spaces-lemma-representable-over-space}.
By construction $|X'| = W$, i.e., $X'$ is a subspace of $X$
corresponding to $W$. Thus (1) is proved.

\medskip\noindent
To prove (2), note that if $\{X_i \subset X\}_{i \in I}$ is a collection
of open subspaces, then it is a Zariski covering if and only if the
$U = \bigcup U \times_X X_i$ is an open covering. This follows from
the definition of a Zariski covering and the fact that the morphism
$U \to X$ is surjective as a map of presheaves on $(\Sch/S)_{fppf}$.
On the other hand, we see that $|X| = \bigcup |X_i|$ if and only if
$U = \bigcup U \times_X X_i$ by Lemma \ref{lemma-points-presentation}
(and the fact that the projections $U \times_X X_i \to X_i$ are surjective
and \'etale). Thus the equivalence of (2) follows.
\end{proof}

\begin{lemma}
\label{lemma-factor-through-open-subspace}
Let $S$ be a scheme.
Let $X$, $Y$ be algebraic spaces over $S$.
Let $X' \subset X$ be an open subspace.
Let $f : Y \to X$ be a morphism of algebraic spaces over $S$.
Then $f$ factors through $X'$ if and only if $|f| : |Y| \to |X|$
factors through $|X'| \subset |X|$.
\end{lemma}

\begin{proof}
By Spaces, Lemma \ref{spaces-lemma-base-change-immersions}
we see that $Y' = Y \times_X X' \to Y$ is an open immersion.
If $|f|(|Y|) \subset |X'|$, then clearly $|Y'| = |Y|$. Hence $Y' = Y$ by
Lemma \ref{lemma-open-subspaces}.
\end{proof}

\begin{lemma}
\label{lemma-etale-image-open}
Let $S$ be a scheme. Let $X$ be an algebraic spaces over $S$.
Let $U$ be a scheme and let $f : U \to X$ be an \'etale morphism.
Let $X' \subset X$ be the open subspace corresponding to
the open $|f|(|U|) \subset |X|$ via
Lemma \ref{lemma-open-subspaces}.
Then $f$ factors through a surjective \'etale morphism $f' : U \to X'$.
Moreover, if $R = U \times_X U$, then $R = U \times_{X'} U$ and $X'$ has
the presentation $X' = U/R$.
\end{lemma}

\begin{proof}
The existence of the factorization follows from
Lemma \ref{lemma-factor-through-open-subspace}.
The morphism $f'$ is surjective according to
Lemma \ref{lemma-characterize-surjective}.
To see $f'$ is \'etale, suppose that $T \to X'$ is a morphism
where $T$ is a scheme. Then $T \times_X U = T \times_{X'} U$
as $X' \to X$ is a monomorphism of sheaves. Thus the projection
$T \times_{X'} U \to T$ is \'etale as we assumed $f$ \'etale.
We have $U \times_X U = U \times_{X'} U$ as $X' \to X$ is a monomorphism.
Then $X' = U/R$ follows from
Spaces, Lemma \ref{spaces-lemma-space-presentation}.
\end{proof}

\begin{lemma}
\label{lemma-equivalence-class-point-monomorphism}
Let $S$ be a scheme. Let $X$ be an algebraic space over $S$.
Let $p : \Spec(K) \to X$ and $q : \Spec(L) \to X$
be morphisms where $K$ and $L$ are fields. Assume $p$ and $q$
determine the same point of $|X|$ and $p$ is a monomorphism.
Then $q$ factors uniquely through $p$.
\end{lemma}

\begin{proof}
Since $p$ and $q$ define the same point of $|X|$, we see that the scheme
$$
Y = \Spec(K) \times_{p, X, q} \Spec(L)
$$
is nonempty. Since the base change of a monomorphism is a monomorphism
this means that the projection morphism $Y \to \Spec(L)$
is a monomorphism. Hence $Y = \Spec(L)$, see
Schemes, Lemma \ref{schemes-lemma-mono-towards-spec-field}.
We conclude that $q$ factors through $p$. Uniqueness
comes from the fact that $p$ is a monomorphism.
\end{proof}

\begin{lemma}
\label{lemma-points-monomorphism}
Let $S$ be a scheme. Let $X$ be an algebraic space over $S$.
Consider the map
$$
\{\Spec(k) \to X \text{ monomorphism where }k\text{ is a field}\}
\longrightarrow
|X|
$$
This map is injective.
\end{lemma}

\begin{proof}
This follows from Lemma \ref{lemma-equivalence-class-point-monomorphism}.
\end{proof}

\noindent
We will see in Decent Spaces,
Lemma \ref{decent-spaces-lemma-decent-points-monomorphism}
that the map of Lemma \ref{lemma-points-monomorphism}
is a bijection when $X$ is decent.

















\section{Quasi-compact spaces}
\label{section-quasi-compact}

\begin{definition}
\label{definition-quasi-compact}
Let $S$ be a scheme.
Let $X$ be an algebraic space over $S$.
We say $X$ is {\it quasi-compact} if there exists a surjective
\'etale morphism $U \to X$ with $U$ quasi-compact.
\end{definition}

\begin{lemma}
\label{lemma-quasi-compact-space}
Let $S$ be a scheme.
Let $X$ be an algebraic space over $S$.
Then $X$ is quasi-compact if and only if $|X|$ is quasi-compact.
\end{lemma}

\begin{proof}
Choose a scheme $U$ and an \'etale surjective morphism $U \to X$.
We will use Lemma \ref{lemma-characterize-surjective}.
If $U$ is quasi-compact, then since $|U| \to |X|$ is surjective
we conclude that $|X|$ is quasi-compact.
If $|X|$ is quasi-compact, then since $|U| \to |X|$ is open
we see that there exists a quasi-compact open $U' \subset U$
such that $|U'| \to |X|$ is surjective (and still \'etale).
Hence we win.
\end{proof}

\begin{lemma}
\label{lemma-finite-disjoint-quasi-compact}
A finite disjoint union of quasi-compact algebraic spaces is
a quasi-compact algebraic space.
\end{lemma}

\begin{proof}
This is clear from
Lemma \ref{lemma-quasi-compact-space}
and the corresponding topological fact.
\end{proof}

\begin{example}
\label{example-quasi-compact-not-very-reasonable}
The space $\mathbf{A}^1_{\mathbf{Q}}/\mathbf{Z}$ is a quasi-compact
algebraic space.
\end{example}

\begin{lemma}
\label{lemma-space-locally-quasi-compact}
Let $S$ be a scheme.
Let $X$ be an algebraic space over $S$.
Every point of $|X|$ has a fundamental system of open
quasi-compact neighbourhoods.
In particular $|X|$ is locally quasi-compact in the sense of
Topology, Definition \ref{topology-definition-locally-quasi-compact}.
\end{lemma}

\begin{proof}
This follows formally from the fact that there exists a scheme
$U$ and a surjective, open, continuous map
$U \to |X|$ of topological spaces. To be a bit more precise, if
$u \in U$ maps to $x \in |X|$, then the images of the affine
neighbourhoods of $u$ will give a fundamental system of quasi-compact
open neighbourhoods of $x$.
\end{proof}







\section{Special coverings}
\label{section-special-coverings}

\noindent
In this section we collect some straightforward lemmas on the existence
of \'etale surjective coverings of algebraic spaces.

\begin{lemma}
\label{lemma-cover-by-union-affines}
Let $S$ be a scheme.
Let $X$ be an algebraic space over $S$.
There exists a surjective \'etale morphism $U \to X$ where
$U$ is a disjoint union of affine schemes.
We may in addition assume each of these affines
maps into an affine open of $S$.
\end{lemma}

\begin{proof}
Let $V \to X$ be a surjective \'etale morphism.
Let $V = \bigcup_{i \in I} V_i$ be a Zariski open covering
such that each $V_i$ maps into an affine open of $S$.
Then set $U = \coprod_{i \in I} V_i$ with induced morphism
$U \to V \to X$. This is \'etale and surjective as a composition
of \'etale and surjective representable
transformations of functors (via the general principle
Spaces, Lemma
\ref{spaces-lemma-composition-representable-transformations-property}
and
Morphisms, Lemmas \ref{morphisms-lemma-composition-surjective} and
\ref{morphisms-lemma-composition-etale}).
\end{proof}

\begin{lemma}
\label{lemma-union-of-quasi-compact}
Let $S$ be a scheme.
Let $X$ be an algebraic space over $S$.
There exists a Zariski covering $X = \bigcup X_i$
such that each algebraic space $X_i$ has a surjective
\'etale covering by an affine scheme. We may in addition assume
each $X_i$ maps into an affine open of $S$.
\end{lemma}

\begin{proof}
By Lemma \ref{lemma-cover-by-union-affines} we can find a surjective
\'etale morphism $U = \coprod U_i \to X$, with $U_i$ affine and mapping
into an affine open of $S$. Let $X_i \subset X$ be the open subspace
of $X$ such that $U_i \to X$ factors through an \'etale surjective morphism
$U_i \to X_i$, see
Lemma \ref{lemma-etale-image-open}.
Since $U = \bigcup U_i$ we see that $X = \bigcup X_i$.
As $U_i \to X_i$ is surjective it follows that $X_i \to S$ maps into
an affine open of $S$.
\end{proof}

\begin{lemma}
\label{lemma-quasi-compact-affine-cover}
Let $S$ be a scheme.
Let $X$ be an algebraic space over $S$.
Then $X$ is quasi-compact if and only if
there exists an \'etale surjective morphism $U \to X$
with $U$ an affine scheme.
\end{lemma}

\begin{proof}
If there exists an \'etale surjective morphism $U \to X$ with $U$
affine then $X$ is quasi-compact by Definition \ref{definition-quasi-compact}.
Conversely, if $X$ is quasi-compact, then $|X|$ is quasi-compact.
Let $U = \coprod_{i \in I} U_i$ be a disjoint union of affine schemes
with an \'etale and surjective map $\varphi : U \to X$
(Lemma \ref{lemma-cover-by-union-affines}).
Then $|X| = \bigcup \varphi(|U_i|)$ and
by quasi-compactness there is a finite subset $i_1, \ldots, i_n$
such that $|X| = \bigcup \varphi(|U_{i_j}|)$. Hence
$U_{i_1} \cup \ldots \cup U_{i_n}$ is an affine scheme with a
finite surjective morphism towards $X$.
\end{proof}

\noindent
The following lemma will be obsoleted by the discussion of
separated morphisms in the chapter on morphisms of algebraic spaces.

\begin{lemma}
\label{lemma-separated-cover}
Let $S$ be a scheme.
Let $X$ be an algebraic space over $S$.
Let $U$ be a separated scheme and $U \to X$ \'etale.
Then $U \to X$ is separated, and $R = U \times_X U$ is a separated scheme.
\end{lemma}

\begin{proof}
Let $X' \subset X$ be the open subscheme such that $U \to X$ factors
through an \'etale surjection $U \to X'$, see
Lemma \ref{lemma-etale-image-open}.
If $U \to X'$ is separated, then so is $U \to X$, see
Spaces, Lemma
\ref{spaces-lemma-composition-representable-transformations-property}
(as the open immersion $X' \to X$ is separated by
Spaces, Lemma
\ref{spaces-lemma-representable-transformations-property-implication}
and
Schemes, Lemma \ref{schemes-lemma-immersions-monomorphisms}).
Moreover, since $U \times_{X'} U = U \times_X U$ it suffices
to prove the result after replacing $X$ by $X'$, i.e., we may
assume $U \to X$ surjective.
Consider the commutative diagram
$$
\xymatrix{
R = U \times_X U \ar[r] \ar[d] & U \ar[d] \\
U \ar[r] & X
}
$$
In the proof of
Spaces, Lemma \ref{spaces-lemma-properties-diagonal}
we have seen that $j : R \to U \times_S U$ is separated.
The morphism of schemes $U \to S$ is separated as $U$ is a separated
scheme, see
Schemes, Lemma \ref{schemes-lemma-compose-after-separated}.
Hence $U \times_S U \to U$ is separated as a base change, see
Schemes, Lemma \ref{schemes-lemma-separated-permanence}.
Hence the scheme $U \times_S U$ is separated (by the same lemma).
Since $j$ is separated we see in the same way that $R$ is separated.
Hence $R \to U$ is a separated morphism (by
Schemes, Lemma \ref{schemes-lemma-compose-after-separated}
again). Thus by
Spaces, Lemma \ref{spaces-lemma-representable-morphisms-spaces-property}
and the diagram above we conclude that $U \to X$ is separated.
\end{proof}

\begin{lemma}
\label{lemma-quasi-separated}
Let $S$ be a scheme. Let $X$ be an algebraic space over $S$.
If there exists a quasi-separated scheme $U$ and a surjective
\'etale morphism $U \to X$ such that either of the projections
$U \times_X U \to U$ is quasi-compact, then $X$ is quasi-separated.
\end{lemma}

\begin{proof}
We may think of $X$ as an algebraic space over $\mathbf{Z}$.
Consider the cartesian diagram
$$
\xymatrix{
U \times_X U \ar[r] \ar[d]_j & X \ar[d]^\Delta \\
U \times U \ar[r] & X \times X
}
$$
Since $U$ is quasi-separated the projection $U \times U \to U$ is
quasi-separated (as a base change of a quasi-separated morphism
of schemes, see Schemes, Lemma \ref{schemes-lemma-separated-permanence}).
Hence the assumption in the lemma implies $j$ is quasi-compact by
Schemes, Lemma \ref{schemes-lemma-quasi-compact-permanence}.
By Spaces, Lemma \ref{spaces-lemma-representable-morphisms-spaces-property}
we see that $\Delta$ is quasi-compact as desired.
\end{proof}

\begin{lemma}
\label{lemma-quasi-separated-quasi-compact-pieces}
Let $S$ be a scheme.
Let $X$ be an algebraic space over $S$.
The following are equivalent
\begin{enumerate}
\item $X$ is Zariski locally quasi-separated over $S$,
\item $X$ is Zariski locally quasi-separated,
\item there exists a Zariski open covering $X = \bigcup X_i$
such that for each $i$ there exists an affine scheme
$U_i$ and a quasi-compact surjective \'etale
morphism $U_i \to X_i$, and
\item there exists a Zariski open covering $X = \bigcup X_i$
such that for each $i$ there exists an affine scheme
$U_i$ which maps into an affine open of $S$ and a quasi-compact
surjective \'etale morphism $U_i \to X_i$.
\end{enumerate}
\end{lemma}

\begin{proof}
Assume  $U_i \to X_i \subset X$ are as in (3). To prove (4)
choose for each $i$ a finite affine open covering $U_i =
U_{i1} \cup \ldots \cup U_{in_i}$ such that each $U_{ij}$ maps
into an affine open of $S$. The compositions
$U_{ij} \to U_i \to X_i$ are \'etale and quasi-compact (see
Spaces, Lemma
\ref{spaces-lemma-composition-representable-transformations-property}).
Let $X_{ij} \subset X_i$ be the open subspace corresponding to
the image of $|U_{ij}| \to |X_i|$, see
Lemma \ref{lemma-etale-image-open}.
Note that $U_{ij} \to X_{ij}$ is quasi-compact as $X_{ij} \subset X_i$
is a monomorphism and as $U_{ij} \to X$ is quasi-compact.
Then $X = \bigcup X_{ij}$ is a covering as in (4).
The implication (4) $\Rightarrow$ (3) is immediate.

\medskip\noindent
Assume (4). To show that $X$ is Zariski locally quasi-separated over $S$
it suffices to show that $X_i$ is quasi-separated over $S$.
Hence we may assume there exists an affine scheme $U$ mapping into
an affine open of $S$ and a quasi-compact surjective \'etale
morphism $U \to X$. Consider the fibre product square
$$
\xymatrix{
U \times_X U \ar[r] \ar[d] & U \times_S U \ar[d] \\
X \ar[r]^-{\Delta_{X/S}} & X \times_S X
}
$$
The right vertical arrow is surjective \'etale (see
Spaces, Lemma
\ref{spaces-lemma-product-representable-transformations-property})
and $U \times_S U$ is affine (as $U$ maps into an affine open of $S$, see
Schemes, Section \ref{schemes-section-fibre-products}),
and $U \times_X U$ is quasi-compact
because the projection $U \times_X U \to U$ is quasi-compact as a
base change of $U \to X$. It follows from
Spaces, Lemma \ref{spaces-lemma-representable-morphisms-spaces-property}
that $\Delta_{X/S}$ is quasi-compact as desired.

\medskip\noindent
Assume (1). To prove (3) there is an immediate reduction to the case
where $X$ is quasi-separated over $S$. By
Lemma \ref{lemma-union-of-quasi-compact}
we can find a Zariski open covering $X = \bigcup X_i$ such that each
$X_i$ maps into an affine open of $S$, and such that there exist affine
schemes $U_i$ and surjective \'etale morphisms $U_i \to X_i$.
Since $U_i \to S$ maps into an affine open of $S$ we see that
$U_i \times_S U_i$ is affine, see
Schemes, Section \ref{schemes-section-fibre-products}.
As $X$ is quasi-separated over $S$, the morphisms
$$
R_i = U_i \times_{X_i} U_i = U_i \times_X U_i
\longrightarrow
U_i \times_S U_i
$$
as base changes of $\Delta_{X/S}$ are quasi-compact. Hence we conclude
that $R_i$ is a quasi-compact scheme. This in turn implies that each
projection $R_i \to U_i$ is quasi-compact. Hence, applying
Spaces, Lemma \ref{spaces-lemma-representable-morphisms-spaces-property}
to the covering $U_i \to X_i$ and the morphism $U_i \to X_i$
we conclude that the morphisms $U_i \to X_i$ are quasi-compact as desired.

\medskip\noindent
At this point we see that (1), (3), and (4) are equivalent. Since (3) does
not refer to the base scheme we conclude that these are also equivalent
with (2).
\end{proof}

\noindent
The following lemma will turn out to be quite useful.

\begin{lemma}
\label{lemma-finite-fibres-presentation}
Let $S$ be a scheme.
Let $X$ be an algebraic space over $S$.
Let $U$ be a scheme. Let $\varphi : U \to X$ be an \'etale morphism such that
the projections $R = U \times_X U \to U$ are quasi-compact; for example if
$\varphi$ is quasi-compact. Then the fibres of
$$
|U| \to |X|
\quad\text{and}\quad
|R| \to |X|
$$
are finite.
\end{lemma}

\begin{proof}
Denote $R = U \times_X U$, and $s, t : R \to U$ the projections.
Let $u \in U$ be a point, and let $x \in |X|$ be its image.
The fibre of $|U| \to |X|$ over $x$ is equal to
$s(t^{-1}(\{u\}))$ by
Lemma \ref{lemma-points-cartesian},
and the fibre of $|R| \to |X|$ over $x$ is $t^{-1}(s(t^{-1}(\{u\})))$.
Since $t : R \to U$ is \'etale and quasi-compact, it has finite fibres
(as its fibres are disjoint unions of spectra of fields by
Morphisms, Lemma \ref{morphisms-lemma-etale-over-field}
and quasi-compact). Hence we win.
\end{proof}








\section{Properties of Spaces defined by properties of schemes}
\label{section-types-properties}

\noindent
Any \'etale local property of schemes gives rise to a corresponding
property of algebraic spaces via the following lemma.

\begin{lemma}
\label{lemma-type-property}
Let $S$ be a scheme.
Let $X$ be an algebraic space over $S$.
Let $\mathcal{P}$ be a property of schemes which is local in the \'etale
topology, see
Descent, Definition \ref{descent-definition-property-local}.
The following are equivalent
\begin{enumerate}
\item for some scheme $U$ and surjective \'etale morphism $U \to X$
the scheme $U$ has property $\mathcal{P}$, and
\item for every scheme $U$ and every \'etale morphism $U \to X$
the scheme $U$ has property $\mathcal{P}$.
\end{enumerate}
If $X$ is representable this is equivalent to $\mathcal{P}(X)$.
\end{lemma}

\begin{proof}
The implication (2) $\Rightarrow$ (1) is immediate.
For the converse, choose a surjective \'etale morphism $U \to X$
with $U$ a scheme that has $\mathcal{P}$ and let $V$ be an \'etale
$X$-scheme. Then $U \times_X V \rightarrow V$ is an \'etale surjection
of schemes, so $V$ inherits $\mathcal{P}$ from $U \times_X V$, which in
turn inherits $\mathcal{P}$ from $U$ (see discussion following
Descent, Definition \ref{descent-definition-property-local}).
The last claim is clear from (1) and
Descent, Definition \ref{descent-definition-property-local}.
\end{proof}

\begin{definition}
\label{definition-type-property}
Let $\mathcal{P}$ be a property of schemes which is
local in the \'etale topology.
Let $S$ be a scheme.
Let $X$ be an algebraic space over $S$.
We say $X$ {\it has property $\mathcal{P}$}
if any of the equivalent conditions of
Lemma \ref{lemma-type-property}
hold.
\end{definition}

\begin{remark}
\label{remark-list-properties-local-etale-topology}
Here is a list of properties which are local for the \'etale topology
(keep in mind that the fpqc, fppf, syntomic, and smooth topologies are
stronger than the \'etale topology):
\begin{enumerate}
\item locally Noetherian, see
Descent, Lemma \ref{descent-lemma-Noetherian-local-fppf},
\item Jacobson, see
Descent, Lemma \ref{descent-lemma-Jacobson-local-fppf},
\item locally Noetherian and $(S_k)$, see
Descent, Lemma \ref{descent-lemma-Sk-local-syntomic},
\item Cohen-Macaulay, see
Descent, Lemma \ref{descent-lemma-CM-local-syntomic},
\item Gorenstein, see
Duality for Schemes, Lemma \ref{duality-lemma-gorenstein-local-syntomic},
\item reduced, see
Descent, Lemma \ref{descent-lemma-reduced-local-smooth},
\item normal, see
Descent, Lemma \ref{descent-lemma-normal-local-smooth},
\item locally Noetherian and $(R_k)$, see
Descent, Lemma \ref{descent-lemma-Rk-local-smooth},
\item regular, see
Descent, Lemma \ref{descent-lemma-regular-local-smooth},
\item Nagata, see
Descent, Lemma \ref{descent-lemma-Nagata-local-smooth}.
\end{enumerate}
\end{remark}

\noindent
Any \'etale local property of germs of schemes gives rise to a corresponding
property of algebraic spaces. Here is the obligatory lemma.

\begin{lemma}
\label{lemma-local-source-target-at-point}
Let $\mathcal{P}$ be a property of germs of schemes which is \'etale local, see
Descent, Definition \ref{descent-definition-local-at-point}.
Let $S$ be a scheme.
Let $X$ be an algebraic space over $S$.
Let $x \in |X|$ be a point of $X$.
Consider \'etale morphisms $a : U \to X$ where $U$ is a scheme.
The following are equivalent
\begin{enumerate}
\item for any $U \to X$ as above and $u \in U$ with $a(u) = x$ we have
$\mathcal{P}(U, u)$, and
\item for some $U \to X$ as above and $u \in U$ with $a(u) = x$ we have
$\mathcal{P}(U, u)$.
\end{enumerate}
If $X$ is representable, then this is equivalent to $\mathcal{P}(X, x)$.
\end{lemma}

\begin{proof}
Omitted.
\end{proof}

\begin{definition}
\label{definition-property-at-point}
Let $S$ be a scheme. Let $X$ be an algebraic space over $S$.
Let $x \in |X|$. Let $\mathcal{P}$ be a property of germs of schemes which is
\'etale local.
We say $X$ {\it has property $\mathcal{P}$ at $x$} if any of the
equivalent conditions of
Lemma \ref{lemma-local-source-target-at-point}
hold.
\end{definition}

\begin{remark}
\label{remark-list-properties-local-ring-local-etale-topology}
Let $P$ be a property of local rings. Assume that for any
\'etale ring map $A \to B$ and $\mathfrak q$ is a prime of $B$ lying over
the prime $\mathfrak p$ of $A$, then
$P(A_\mathfrak p) \Leftrightarrow P(B_\mathfrak q)$.
Then we obtain an \'etale local property of germs $(U, u)$ of schemes
by setting $\mathcal{P}(U, u) = P(\mathcal{O}_{U, u})$.
In this situation we will use the terminology
``the local ring of $X$ at $x$ has $P$'' to mean
$X$ has property $\mathcal{P}$ at $x$.
Here is a list of such properties $P$:
\begin{enumerate}
\item Noetherian, see
More on Algebra, Lemma \ref{more-algebra-lemma-Noetherian-etale-extension},
\item dimension $d$, see
More on Algebra, Lemma \ref{more-algebra-lemma-dimension-etale-extension},
\item regular, see
More on Algebra, Lemma \ref{more-algebra-lemma-regular-etale-extension},
\item discrete valuation ring, follows from (2), (3), and
Algebra, Lemma \ref{algebra-lemma-characterize-dvr},
\item reduced, see
More on Algebra, Lemma \ref{more-algebra-lemma-henselization-reduced},
\item normal, see
More on Algebra, Lemma \ref{more-algebra-lemma-henselization-normal},
\item Noetherian and depth $k$, see
More on Algebra, Lemma \ref{more-algebra-lemma-henselization-depth},
\item Noetherian and Cohen-Macaulay, see
More on Algebra, Lemma \ref{more-algebra-lemma-henselization-CM},
\item Noetherian and Gorenstein, see
Dualizing Complexes, Lemma \ref{dualizing-lemma-flat-under-gorenstein}.
\end{enumerate}
There are more properties for which this holds, for example G-ring and
Nagata. If we every need these we will add them here
as well as references to detailed proofs of the corresponding
algebra facts.
\end{remark}







\section{Constructible sets}
\label{section-constructible}

\begin{lemma}
\label{lemma-locally-constructible}
Let $S$ be a scheme. Let $X$ be an algebraic space over $S$.
Let $E \subset |X|$ be a subset. The following are equivalent
\begin{enumerate}
\item for every \'etale morphism $U \to X$ where $U$ is a
scheme the inverse image of $E$ in $U$ is a locally constructible
subset of $U$,
\item for every \'etale morphism $U \to X$ where $U$ is an
affine scheme the inverse image of $E$ in $U$ is a constructible
subset of $U$,
\item for some surjective \'etale morphism $U \to X$ where $U$ is a
scheme the inverse image of $E$ in $U$ is a locally constructible
subset of $U$.
\end{enumerate}
\end{lemma}

\begin{proof}
By Properties, Lemma \ref{properties-lemma-locally-constructible}
we see that (1) and (2) are equivalent. It is immediate that (1)
implies (3). Thus we assume we have a surjective \'etale morphism
$\varphi : U \to X$ where $U$ is a scheme such that $\varphi^{-1}(E)$
is locally constructible. Let $\varphi' : U' \to X$ be another
\'etale morphism where $U'$ is a scheme. Then we have
$$
E'' = \text{pr}_1^{-1}(\varphi^{-1}(E)) = \text{pr}_2^{-1}((\varphi')^{-1}(E))
$$
where $\text{pr}_1 : U \times_X U' \to U$ and
$\text{pr}_2 : U \times_X U' \to U'$ are the projections.
By Morphisms, Lemma \ref{morphisms-lemma-inverse-image-constructible}
we see that $E''$ is locally constructible in $U \times_X U'$.
Let $W' \subset U'$ be an affine open. Since $\text{pr}_2$ is
\'etale and hence open, we can choose a quasi-compact open
$W'' \subset U \times_X U'$ with $\text{pr}_2(W'') = W'$.
Then $\text{pr}_2|_{W''} : W'' \to W'$ is quasi-compact.
We have $W' \cap (\varphi')^{-1}(E) = \text{pr}_2(E'' \cap W'')$
as $\varphi$ is surjective, see Lemma \ref{lemma-points-cartesian}.
Thus $W' \cap (\varphi')^{-1}(E) = \text{pr}_2(E'' \cap W'')$
is locally constructible by
Morphisms, Theorem \ref{morphisms-theorem-chevalley} as desired.
\end{proof}

\begin{definition}
\label{definition-locally-constructible}
Let $S$ be a scheme. Let $X$ be an algebraic space over $S$.
Let $E \subset |X|$ be a subset. We say $E$ is
{\it \'etale locally constructible} if the equivalent
conditions of Lemma \ref{lemma-locally-constructible} are satisfied.
\end{definition}

\noindent
Of course, if $X$ is representable, i.e., $X$ is a scheme,
then this just means $E$ is a locally constructible subset
of the underlying topological space.







\section{Dimension at a point}
\label{section-dimension}

\noindent
We can use
Descent, Lemma \ref{descent-lemma-dimension-at-point-local}
to define the dimension of an algebraic
space $X$ at a point $x$. This will give us a different notion than the
topological one (i.e., the dimension of $|X|$ at $x$).

\begin{definition}
\label{definition-dimension-at-point}
Let $S$ be a scheme.
Let $X$ be an algebraic space over $S$.
Let $x \in |X|$ be a point of $X$.
We define the {\it dimension of $X$ at $x$} to be
the element $\dim_x(X) \in \{0, 1, 2, \ldots, \infty\}$
such that $\dim_x(X) = \dim_u(U)$ for any (equivalently some)
pair $(a : U \to X, u)$ consisting of an \'etale morphism $a : U \to X$
from a scheme to $X$ and a point $u \in U$ with $a(u) = x$.
See
Definition \ref{definition-property-at-point},
Lemma \ref{lemma-local-source-target-at-point}, and
Descent, Lemma \ref{descent-lemma-dimension-at-point-local}.
\end{definition}

\noindent
Warning: It is {\bf not} the case that $\dim_x(X) = \dim_x(|X|)$
in general. A counter example is the algebraic space $X$ of
Spaces, Example \ref{spaces-example-infinite-product}.
Namely, let $x \in |X|$ be a point not equal to the generic point $x_0$
of $|X|$. Then we have $\dim_x(X) = 0$ but $\dim_x(|X|) = 1$.
In particular, the dimension of $X$ (as defined
below) is different from the dimension of $|X|$.

\begin{definition}
\label{definition-dimension}
Let $S$ be a scheme. Let $X$ be an algebraic space over $S$.
The {\it dimension} $\dim(X)$ of $X$ is defined by the rule
$$
\dim(X) = \sup\nolimits_{x \in |X|} \dim_x(X)
$$
\end{definition}

\noindent
By
Properties, Lemma \ref{properties-lemma-dimension}
we see that this is the usual notion if $X$ is a scheme.
There is another integer that measures the dimension of a scheme
at a point, namely the dimension of the local ring. This invariant
is compatible with \'etale morphisms also, see
Section \ref{section-dimension-local-ring}.






\section{Dimension of local rings}
\label{section-dimension-local-ring}

\noindent
The dimension of the local ring of an algebraic space is a well defined
concept.

\begin{lemma}
\label{lemma-pre-dimension-local-ring}
Let $S$ be a scheme.
Let $X$ be an algebraic space over $S$.
Let $x \in |X|$ be a point.
Let $d \in \{0, 1, 2, \ldots, \infty\}$.
The following are equivalent
\begin{enumerate}
\item for some scheme $U$ and \'etale morphism $a : U \to X$ and point
$u \in U$ with $a(u) = x$ we have $\dim(\mathcal{O}_{U, u}) = d$,
\item for any scheme $U$, any \'etale morphism $a : U \to X$, and any point
$u \in U$ with $a(u) = x$ we have $\dim(\mathcal{O}_{U, u}) = d$.
\end{enumerate}
If $X$ is a scheme, this is equivalent to $\dim(\mathcal{O}_{X, x}) = d$.
\end{lemma}

\begin{proof}
Combine
Lemma \ref{lemma-local-source-target-at-point} and
Descent, Lemma \ref{descent-lemma-dimension-local-ring-local}.
\end{proof}

\begin{definition}
\label{definition-dimension-local-ring}
Let $S$ be a scheme. Let $X$ be an algebraic space over $S$. Let $x \in |X|$
be a point. The {\it dimension of the local ring of $X$ at $x$} is
the element $d \in \{0, 1, 2, \ldots, \infty\}$ satisfying the equivalent
conditions of Lemma \ref{lemma-pre-dimension-local-ring}. In this case we
will also say {\it $x$ is a point of codimension $d$ on $X$}.
\end{definition}

\noindent
Besides the lemma below we also point the reader to
Lemmas \ref{lemma-dimension-local-ring} and
\ref{lemma-dimension-decent-invariant-under-etale}.

\begin{lemma}
\label{lemma-dimension}
Let $S$ be a scheme. Let $X$ be an algebraic space over $S$.
The following quantities are equal:
\begin{enumerate}
\item The dimension of $X$.
\item The supremum of the dimensions of the local rings of $X$.
\item The supremum of $\dim_x(X)$ for $x \in |X|$.
\end{enumerate}
\end{lemma}

\begin{proof}
The numbers in (1) and (3) are equal by Definition \ref{definition-dimension}.
Let $U \to X$ be a surjective \'etale morphism from a scheme $U$.
The supremum of $\dim_x(X)$ for $x \in |X|$ is the same as the
supremum of $\dim_u(U)$ for points $u$ of $U$ by definition.
This is the same as the supremum of $\dim(\mathcal{O}_{U, u})$ by
Properties, Lemma \ref{properties-lemma-dimension}. This in turn
is the same as (2) by definition.
\end{proof}




\section{Generic points}
\label{section-generic-points}

\noindent
Let $T$ be a topological space. According to the second edition of
EGA I, a {\it maximal point of $T$} is a generic point of an irreducible
component of $T$. If $T = |X|$ is the topological space associated to
an algebraic space $X$, there are at least two notions of maximal points:
we can look at maximal points of $T$ viewed as a topological space, or
we can look at images of maximal points of $U$ where $U \to X$ is an
\'etale morphism and $U$ is a scheme. The second notion corresponds to
the set of points of codimension $0$ (Lemma \ref{lemma-codimension-0-points}).
The codimension $0$ points are easier
to work with for general algebraic spaces; the two notions
agree for quasi-separated and more generally decent algebraic spaces
(Decent Spaces, Lemma \ref{decent-spaces-lemma-decent-generic-points}).

\begin{lemma}
\label{lemma-codimension-0-points}
Let $S$ be a scheme and let $X$ be an algebraic space over $S$.
Let $x \in |X|$. Consider \'etale morphisms $a : U \to X$ where
$U$ is a scheme. The following are equivalent
\begin{enumerate}
\item $x$ is a point of codimension $0$ on $X$,
\item for some $U \to X$ as above and $u \in U$ with $a(u) = x$,
the point $u$ is the generic point of an irreducible component of $U$, and
\item for any $U \to X$ as above and any $u \in U$ mapping to $x$,
the point $u$ is the generic point of an irreducible component of $U$.
\end{enumerate}
If $X$ is representable, this is equivalent to $x$ being a generic
point of an irreducible component of $|X|$.
\end{lemma}

\begin{proof}
Observe that a point $u$ of a scheme $U$ is a generic point of an
irreducible component of $U$ if and only if $\dim(\mathcal{O}_{U, u}) = 0$
(Properties, Lemma \ref{properties-lemma-generic-point}).
Hence this follows from the definition of the codimension of a
point on $X$ (Definition \ref{definition-dimension-local-ring}).
\end{proof}

\begin{lemma}
\label{lemma-codimension-0-points-dense}
Let $S$ be a scheme and let $X$ be an algebraic space over $S$.
The set of codimension $0$ points of $X$ is dense in $|X|$.
\end{lemma}

\begin{proof}
If $U$ is a scheme, then the set of generic points of irreducible
components is dense in $U$ (holds for any quasi-sober topological space).
Thus if $U \to X$ is a surjective \'etale morphism, then the set
of codimension $0$ points of $X$ is the image of a dense subset of
$|U|$ (Lemma \ref{lemma-codimension-0-points}). Since $|X|$ has the
quotient topology for $|U| \to |X|$ we conclude.
\end{proof}





\section{Reduced spaces}
\label{section-reduced}

\noindent
We have already defined reduced algebraic spaces in
Section \ref{section-types-properties}.
Here we just prove some simple lemmas regarding reduced algebraic
spaces.

\begin{lemma}
\label{lemma-subspace-induced-topology}
Let $S$ be a scheme. Let $Z \to X$ be an immersion of algebraic spaces.
Then $|Z| \to |X|$ is a homeomorphism of $|Z|$ onto a locally closed subset
of $|X|$.
\end{lemma}

\begin{proof}
Let $U$ be a scheme and $U \to X$ a surjective \'etale morphism.
Then $Z \times_X U \to U$ is an immersion of schemes, hence gives a
homeomorphism of $|Z \times_X U|$ with a locally closed subset $T'$
of $|U|$. By Lemma \ref{lemma-points-cartesian} the subset
$T'$ is the inverse image of the image $T$ of $|Z| \to |X|$.
The map $|Z| \to |X|$ is injective because the transformation of
functors $Z \to X$ is injective, see
Spaces, Section \ref{spaces-section-Zariski}. By
Topology, Lemma \ref{topology-lemma-open-morphism-quotient-topology}
we see that $T$ is locally closed in $|X|$. Moreover, the continuous
map $|Z| \to T$ is a homeomorphism as the map $|Z \times_X U| \to T'$
is a homeomorphism and $|Z \times_Y U| \to |Z|$ is submersive.
\end{proof}

\noindent
The following lemma will help us construct
(locally) closed subspaces.

\begin{lemma}
\label{lemma-subspaces-presentation}
Let $S$ be a scheme. Let $j : R \to U \times_S U$ be an \'etale equivalence
relation. Let $X = U/R$ be the associated algebraic space
(Spaces, Theorem \ref{spaces-theorem-presentation}). There is a
canonical bijection
$$
R\text{-invariant locally closed subschemes }Z'\text{ of }U
\leftrightarrow
\text{locally closed subspaces }Z\text{ of }X
$$
Moreover, if $Z \to X$ is closed (resp.\ open) if and only if
$Z' \to U$ is closed (resp.\ open).
\end{lemma}

\begin{proof}
Denote $\varphi : U \to X$ the canonical map. The bijection sends
$Z \to X$ to $Z' = Z \times_X U \to U$. It is immediate from the definition
that $Z' \to U$ is an immersion, resp.\ closed immersion, resp.\ open
immersion if $Z \to X$ is so. It is also clear that $Z'$ is $R$-invariant
(see Groupoids, Definition \ref{groupoids-definition-invariant-open}).

\medskip\noindent
Conversely, assume that $Z' \to U$ is an immersion which is $R$-invariant.
Let $R'$ be the restriction of $R$ to $Z'$, see
Groupoids, Definition \ref{groupoids-definition-restrict-groupoid}.
Since $R' = R \times_{s, U} Z' = Z' \times_{U, t} R$ in this case
we see that $R'$ is an \'etale equivalence relation on $Z'$. By
Spaces, Theorem \ref{spaces-theorem-presentation} we see
$Z = Z'/R'$ is an algebraic space. By construction we have
$U \times_X Z = Z'$, so $U \times_X Z \to Z$ is an immersion.
Note that the property ``immersion'' is preserved under base change
and fppf local on the base (see Spaces, Section \ref{spaces-section-lists}).
Moreover, immersions are separated and locally quasi-finite (see
Schemes, Lemma \ref{schemes-lemma-immersions-monomorphisms}
and
Morphisms, Lemma \ref{morphisms-lemma-immersion-locally-quasi-finite}).
Hence by More on Morphisms, Lemma
\ref{more-morphisms-lemma-separated-locally-quasi-finite-morphisms-fppf-descend}
immersions satisfy descent for fppf covering. This means all the hypotheses of
Spaces,
Lemma \ref{spaces-lemma-morphism-sheaves-with-P-effective-descent-etale}
are satisfied for $Z \to X$, $\mathcal{P}=$``immersion'',
and the \'etale surjective morphism $U \to X$. We conclude that $Z \to X$
is representable and an immersion, which is the
definition of a subspace (see
Spaces, Definition \ref{spaces-definition-immersion}).

\medskip\noindent
It is clear that these constructions are inverse to each other and we win.
\end{proof}

\begin{lemma}
\label{lemma-reduced-closed-subspace}
Let $S$ be a scheme.
Let $X$ be an algebraic space over $S$.
Let $T \subset |X|$ be a closed subset.
There exists a unique closed subspace $Z \subset X$ with
the following properties: (a) we have $|Z| = T$, and (b) $Z$ is reduced.
\end{lemma}

\begin{proof}
Let $U \to X$ be a surjective \'etale morphism, where $U$ is a scheme.
Set $R = U \times_X U$, so that $X = U/R$, see
Spaces, Lemma \ref{spaces-lemma-space-presentation}.
As usual we denote $s, t : R \to U$ the two projection morphisms.
By Lemma \ref{lemma-points-presentation}
we see that $T$ corresponds to a closed subset $T' \subset |U|$ such
that $s^{-1}(T') = t^{-1}(T')$.
Let $Z' \subset U$ be the reduced induced scheme structure on $T'$.
In this case the fibre products
$Z' \times_{U, t} R$ and $Z' \times_{U, s} R$ are closed subschemes
of $R$
(Schemes, Lemma \ref{schemes-lemma-base-change-immersion})
which are \'etale over $Z'$
(Morphisms, Lemma \ref{morphisms-lemma-base-change-etale}),
and hence reduced
(because being reduced is local in the \'etale topology, see
Remark \ref{remark-list-properties-local-etale-topology}).
Since they have the same underlying topological space (see above)
we conclude that $Z' \times_{U, t} R = Z' \times_{U, s} R$.
Thus we can apply Lemma \ref{lemma-subspaces-presentation}
to obtain a closed subspace $Z \subset X$ whose pullback to $U$ is $Z'$.
By construction $|Z| = T$ and $Z$ is reduced. This proves existence.
We omit the proof of uniqueness.
\end{proof}

\begin{lemma}
\label{lemma-map-into-reduction}
Let $S$ be a scheme.
Let $X$, $Y$ be algebraic spaces over $S$.
Let $Z \subset X$ be a closed subspace.
Assume $Y$ is reduced.
A morphism $f : Y \to X$ factors through $Z$ if and only if
$f(|Y|) \subset |Z|$.
\end{lemma}

\begin{proof}
Assume $f(|Y|) \subset |Z|$. Choose a diagram
$$
\xymatrix{
V \ar[d]_b \ar[r]_h & U \ar[d]^a \\
Y \ar[r]^f & X
}
$$
where $U$, $V$ are schemes, and the vertical arrows are surjective and
\'etale. The scheme $V$ is reduced, see
Lemma \ref{lemma-type-property}.
Hence $h$ factors through $a^{-1}(Z)$ by
Schemes, Lemma \ref{schemes-lemma-map-into-reduction}.
So $a \circ h$ factors through $Z$.
As $Z \subset X$ is a subsheaf, and $V \to Y$ is a surjection of sheaves
on $(\Sch/S)_{fppf}$ we conclude that $X \to Y$ factors
through $Z$.
\end{proof}

\begin{definition}
\label{definition-reduced-induced-space}
Let $S$ be a scheme, and let $X$ be an algebraic space over $S$.
Let $Z \subset |X|$ be a closed subset.
An {\it algebraic space structure on $Z$} is given by a closed subspace
$Z'$ of $X$ with $|Z'|$ equal to $Z$.
The {\it reduced induced algebraic space structure}
on $Z$ is the one constructed in
Lemma \ref{lemma-reduced-closed-subspace}.
The {\it reduction $X_{red}$ of $X$} is the reduced induced algebraic
space structure on $|X|$.
\end{definition}











\section{The schematic locus}
\label{section-schematic}

\noindent
Every algebraic space has a largest open subspace which is a
scheme; this is more or less clear but we also write out the proof below.
Of course this subspace may be empty, for example if
$X = \mathbf{A}^1_{\mathbf{Q}}/\mathbf{Z}$ (the universal
counter example). On the other hand, if $X$ is for example quasi-separated,
then this largest open subscheme is actually dense in $X$!

\begin{lemma}
\label{lemma-subscheme}
Let $S$ be a scheme.
Let $X$ be an algebraic space over $S$.
There exists a largest open subspace $X' \subset X$ which is a scheme.
\end{lemma}

\begin{proof}
Let $U \to X$ be an \'etale surjective morphism, where $U$ is a scheme.
Let $R = U \times_X U$. The open subspaces of $X$ correspond $1 - 1$
with open subschemes of $U$ which are $R$-invariant. Hence there is a
set of them. Let $X_i$, $i \in I$ be the set of open subspaces
of $X$ which are schemes, i.e., are representable. Consider the
open subspace $X' \subset X$ whose underlying set of points is
the open $\bigcup |X_i|$ of $|X|$. By
Lemma \ref{lemma-characterize-surjective}
we see that
$$
\coprod X_i \longrightarrow X'
$$
is a surjective map of sheaves on $(\Sch/S)_{fppf}$.
But since each $X_i \to X'$ is representable by open immersions
we see that in fact the map is surjective in the Zariski
topology. Namely, if $T \to X'$ is a morphism from a scheme
into $X'$, then $X_i \times_{X'} T$ is an open subscheme of $T$.
Hence we can apply
Schemes, Lemma \ref{schemes-lemma-glue-functors}
to see that $X'$ is a scheme.
\end{proof}

\noindent
In the rest of this section we say that an open subspace
$X'$ of an algebraic space $X$ is {\it dense} if the corresponding
open subset $|X'| \subset |X|$ is dense.

\begin{lemma}
\label{lemma-quasi-separated-finite-etale-cover-dense-open-scheme}
Let $S$ be a scheme. Let $X$ be an algebraic space over $S$.
If there exists a finite, \'etale, surjective morphism
$U \to X$ where $U$ is a quasi-separated scheme, then
there exists a dense open subspace $X'$ of $X$ which is a scheme.
More precisely, every point $x \in |X|$ of codimension $0$ in $X$
is contained in $X'$.
\end{lemma}

\begin{proof}
Let $X' \subset X$ be the maximal open subspace which is a scheme
(Lemma \ref{lemma-subscheme}).
Let $x \in |X|$ be a point of codimension $0$ on $X$.
By Lemma \ref{lemma-codimension-0-points-dense}
it suffices to show $x \in X'$.
Let $U \to X$ be as in the statement of the lemma.
Write $R = U \times_X U$ and denote $s, t : R \to U$ the projections as usual.
Note that $s, t$ are surjective, finite and \'etale.
By Lemma \ref{lemma-finite-fibres-presentation}
the fibre of $|U| \to |X|$ over $x$ is finite, say
$\{\eta_1, \ldots, \eta_n\}$. By Lemma \ref{lemma-codimension-0-points}
each $\eta_i$ is the generic point of an irreducible component of $U$.
By Properties, Lemma \ref{properties-lemma-maximal-points-affine}
we can find an affine open $W \subset U$ containing
$\{\eta_1, \ldots, \eta_n\}$
(this is where we use that $U$ is quasi-separated). By
Groupoids, Lemma \ref{groupoids-lemma-find-invariant-affine}
we may assume that $W$ is $R$-invariant.
Since $W \subset U$ is an $R$-invariant affine open, the restriction
$R_W$ of $R$ to $W$ equals $R_W = s^{-1}(W) = t^{-1}(W)$ (see
Groupoids, Definition \ref{groupoids-definition-invariant-open}
and discussion following it). In particular the maps $R_W \to W$ are
finite \'etale also. It follows that $R_W$ is affine.
Thus we see that $W/R_W$ is a scheme, by
Groupoids, Proposition \ref{groupoids-proposition-finite-flat-equivalence}.
On the other hand, $W/R_W$ is an open subspace of $X$ by
Spaces, Lemma \ref{spaces-lemma-finding-opens} and it contains
$x$ by construction.
\end{proof}

\noindent
We will improve the following proposition to the case of
decent algebraic spaces in
Decent Spaces, Theorem
\ref{decent-spaces-theorem-decent-open-dense-scheme}.

\begin{proposition}
\label{proposition-locally-quasi-separated-open-dense-scheme}
Let $S$ be a scheme. Let $X$ be an algebraic space over $S$. If $X$ is
Zariski locally quasi-separated (for example if $X$ is quasi-separated), then
there exists a dense open subspace $X'$ of $X$ which is a scheme.
More precisely, every point $x \in |X|$ of codimension $0$ on $X$
is contained in $X'$.
\end{proposition}

\begin{proof}
The question is local on $X$ by Lemma \ref{lemma-subscheme}.
Thus by Lemma \ref{lemma-quasi-separated-quasi-compact-pieces}
we may assume that there exists an affine scheme $U$ and a
surjective, quasi-compact, \'etale morphism $U \to X$.
Moreover $U \to X$ is separated (Lemma \ref{lemma-separated-cover}).
Set $R = U \times_X U$ and denote $s, t : R \to U$ the projections
as usual. Then $s, t$ are surjective, quasi-compact, separated, and
\'etale. Hence $s, t$ are also quasi-finite and have finite fibres
(Morphisms,
Lemmas \ref{morphisms-lemma-etale-locally-quasi-finite},
\ref{morphisms-lemma-quasi-finite-locally-quasi-compact}, and
\ref{morphisms-lemma-quasi-finite}).
By Morphisms, Lemma \ref{morphisms-lemma-generically-finite}
for every $\eta \in U$ which is the generic point of an
irreducible component of $U$, there exists an open neighbourhood
$V \subset U$ of $\eta$ such that $s^{-1}(V) \to V$ is finite. By
Descent, Lemma \ref{descent-lemma-descending-property-finite}
being finite is fpqc (and in particular \'etale) local on the target.
Hence we may apply
More on Groupoids, Lemma \ref{more-groupoids-lemma-property-invariant}
which says that the largest open $W \subset U$ over which $s$ is
finite is $R$-invariant. By the above $W$ contains every
generic point of an irreducible component of $U$.
The restriction $R_W$ of $R$ to $W$ equals $R_W = s^{-1}(W) = t^{-1}(W)$
(see Groupoids, Definition \ref{groupoids-definition-invariant-open}
and discussion following it).
By construction $s_W, t_W : R_W \to W$ are finite \'etale.
Consider the open subspace $X' = W/R_W \subset X$ (see
Spaces, Lemma \ref{spaces-lemma-finding-opens}).
By construction the inclusion map $X' \to X$
induces a bijection on points of codimension $0$.
This reduces us to
Lemma \ref{lemma-quasi-separated-finite-etale-cover-dense-open-scheme}.
\end{proof}




\section{Obtaining a scheme}
\label{section-getting-a-scheme}

\noindent
We have used in the previous section that the quotient $U/R$ of an
affine scheme $U$ by an equivalence relation $R$ is a scheme if the
morphisms $s, t : R \to U$ are finite \'etale. This is a special case
of the following result.

\begin{proposition}
\label{proposition-finite-flat-equivalence-global}
Let $S$ be a scheme.
Let $(U, R, s, t, c)$ be a groupoid scheme over $S$.
Assume
\begin{enumerate}
\item $s, t : R \to U$ finite locally free,
\item $j = (t, s)$ is an equivalence relation, and
\item every nonempty closed subset $Z$ of $U$ contains a point
$u$ whose $R$-equivalence class $t(s^{-1}(\{u\}))$ is contained
in an affine open of $U$\footnote{Let $E \subset U$ be the
set of points $u$ such that $t(s^{-1}(\{u\}))$
is contained in an affine open of $U$. Condition (3) holds
if $E = U$, or if every finite type point of $U$ is in $E$, or
if every $u \in U$ specializes to a point of $E$.}.
\end{enumerate}
Then there exists a finite locally free morphism $U \to M$
of schemes over $S$ such that $R = U \times_M U$ and such that $M$
represents the quotient sheaf $U/R$ in the fppf topology.
\end{proposition}

\begin{proof}
By assumption (3) and
Groupoids, Lemma \ref{groupoids-lemma-find-invariant-affine}
we can find an open covering $U = \bigcup U_i$ such that each $U_i$
is an $R$-invariant affine open of $U$. Set $R_i = R|_{U_i}$.
Consider the fppf sheaves $F = U/R$ and $F_i = U_i/R_i$.
By Spaces, Lemma \ref{spaces-lemma-finding-opens} the morphisms
$F_i \to F$ are representable and open immersions.
By Groupoids, Proposition \ref{groupoids-proposition-finite-flat-equivalence}
the sheaves $F_i$ are representable by affine schemes.
If $T$ is a scheme and $T \to F$ is a morphism, then $V_i = F_i \times_F T$
is open in $T$ and we claim that $T = \bigcup V_i$. Namely,
fppf locally on $T$ we can lift $T \to F$ to a morphism
$f : T \to U$ and in that case $f^{-1}(U_i) \subset V_i$.
Hence we conclude that $F$ is representable by a scheme, see
Schemes, Lemma \ref{schemes-lemma-glue-functors}.
\end{proof}

\noindent
For example, if $U$ is isomorphic to a locally closed subscheme of an
affine scheme or isomorphic to a locally closed subscheme of
$\text{Proj}(A)$ for some graded ring $A$, then the third assumption holds by
Properties, Lemma \ref{properties-lemma-ample-finite-set-in-affine}.
In particular we can apply this to free actions of finite groups and
finite group schemes on quasi-affine or quasi-projective schemes.
For example, the quotient $X/G$ of a quasi-projective variety
$X$ by a free action of a finite group $G$ is a scheme. Here is a
detailed statement.

\begin{lemma}
\label{lemma-quotient-scheme}
Let $S$ be a scheme. Let $G \to S$ be a group scheme. Let $X \to S$ be
a morphism of schemes. Let $a : G \times_S X \to X$ be an action. Assume that
\begin{enumerate}
\item $G \to S$ is finite locally free,
\item the action $a$ is free,
\item $X \to S$ is affine, or quasi-affine, or projective, or
quasi-projective, or $X$ is isomorphic to an open subscheme of an
affine scheme, or $X$ is isomorphic to an open subscheme of $\text{Proj}(A)$
for some graded ring $A$, or $G \to S$ is radicial.
\end{enumerate}
Then the fppf quotient sheaf $X/G$ is a scheme and $X \to X/G$
is an fppf $G$-torsor.
\end{lemma}

\begin{proof}
We first show that $X/G$ is a scheme. Since the action is free the morphism
$j = (a, \text{pr}) : G \times_S X \to X \times_S X$
is a monomorphism and hence an equivalence relation, see
Groupoids, Lemma \ref{groupoids-lemma-free-action}. The maps
$s, t : G \times_S X \to X$
are finite locally free as we've assumed that $G \to S$ is finite locally
free. To conclude it now suffices to prove the last assumption of
Proposition \ref{proposition-finite-flat-equivalence-global} holds.
Since the action of $G$ is over $S$ it suffices to prove that
any finite set of points in a fibre of $X \to S$ is contained in an
affine open of $X$. If $X$ is isomorphic to an open subscheme of an
affine scheme or isomorphic to an open subscheme of $\text{Proj}(A)$
for some graded ring $A$ this follows from
Properties, Lemma \ref{properties-lemma-ample-finite-set-in-affine}.
If $X \to S$ is affine, or quasi-affine, or projective, or
quasi-projective, we may replace $S$ by an affine open and we
get back to the case we just dealt with. If $G \to S$ is radicial,
then the orbits of points on $X$ under the action of $G$ are singletons
and the condition trivially holds. Some details omitted.

\medskip\noindent
To see that $X \to X/G$ is an fppf $G$-torsor
(Groupoids, Definition \ref{groupoids-definition-principal-homogeneous-space})
we have to show that $G \times_S X \to X \times_{X/G} X$
is an isomorphism and that $X \to X/G$ fppf locally has sections.
The second part is clear from the fact that $X \to X/G$ is surjective
as a map of fppf sheaves (by construction). The first part follows from
the isomorphism $R = U \times_M U$ in the conclusion of
Proposition \ref{proposition-finite-flat-equivalence-global}
(note that $R = G \times_S X$ in our case).
\end{proof}

\begin{lemma}
\label{lemma-quotient-separated}
Notation and assumptions as in
Proposition \ref{proposition-finite-flat-equivalence-global}. Then
\begin{enumerate}
\item if $U$ is quasi-separated over $S$, then $U/R$ is quasi-separated
over $S$,
\item if $U$ is quasi-separated, then $U/R$ is quasi-separated,
\item if $U$ is separated over $S$, then $U/R$ is separated over $S$,
\item if $U$ is separated, then $U/R$ is separated, and
\item add more here.
\end{enumerate}
Similar results hold in the setting of Lemma \ref{lemma-quotient-scheme}.
\end{lemma}

\begin{proof}
Since $M$ represents the quotient sheaf we have a cartesian diagram
$$
\xymatrix{
R \ar[r]_-j \ar[d] & U \times_S U \ar[d] \\
M \ar[r] & M \times_S M
}
$$
of schemes. Since $U \times_S U \to M \times_S M$ is surjective finite locally
free, to show that $M \to M \times_S M$ is quasi-compact, resp.\ a closed
immersion, it suffices to show that $j : R \to U \times_S U$ is
quasi-compact, resp.\ a closed immersion, see
Descent, Lemmas \ref{descent-lemma-descending-property-quasi-compact} and
\ref{descent-lemma-descending-property-closed-immersion}.
Since $j : R \to U \times_S U$ is a morphism over $U$ and since
$R$ is finite over $U$, we see that $j$ is quasi-compact as soon
as the projection $U \times_S U \to U$ is quasi-separated
(Schemes, Lemma \ref{schemes-lemma-quasi-compact-permanence}).
Since $j$ is a monomorphism and locally of finite type, we see that
$j$ is a closed immersion as soon as it is proper
(\'Etale Morphisms, Lemma \ref{etale-lemma-characterize-closed-immersion})
which will be the case as soon as the projection
$U \times_S U \to U$ is separated
(Morphisms, Lemma \ref{morphisms-lemma-image-proper-scheme-closed}).
This proves (1) and (3). To prove (2) and (4) we replace $S$ by
$\Spec(\mathbf{Z})$, see Definition \ref{definition-separated}.
Since Lemma \ref{lemma-quotient-scheme} is proved through an application of
Proposition \ref{proposition-finite-flat-equivalence-global}
the final statement is clear too.
\end{proof}




\section{Points on quasi-separated spaces}
\label{section-points-quasi-separated}

\noindent
Points can behave very badly on algebraic spaces in the generality introduced
in the Stacks project. However, for quasi-separated spaces their behaviour
is mostly like the behaviour of points on schemes. We prove a few results
on this in this section; the chapter on decent spaces contains many more
results on this, see for example
Decent Spaces, Section \ref{decent-spaces-section-points}.

\begin{lemma}
\label{lemma-quasi-separated-sober}
Let $S$ be a scheme. Let $X$ be a Zariski locally quasi-separated
algebraic space over $S$. Then the topological space $|X|$ is sober (see
Topology, Definition \ref{topology-definition-generic-point}).
\end{lemma}

\begin{proof}
Combining
Topology, Lemma \ref{topology-lemma-sober-local}
and
Lemma \ref{lemma-quasi-separated-quasi-compact-pieces}
we see that we may assume that there exists an affine scheme $U$
and a surjective, quasi-compact, \'etale morphism $U \to X$.
Set $R = U \times_X U$ with projection maps $s, t : R \to U$. Applying
Lemma \ref{lemma-finite-fibres-presentation}
we see that the fibres of $s, t$ are finite. It follows all the assumptions of
Topology, Lemma \ref{topology-lemma-quotient-kolmogorov}
are met, and we conclude that $|X|$ is Kolmogorov\footnote{
Actually we use here also
Schemes, Lemma \ref{schemes-lemma-scheme-sober} (soberness schemes),
Morphisms, Lemmas \ref{morphisms-lemma-etale-flat}
and \ref{morphisms-lemma-generalizations-lift-flat} (generalizations
lift along \'etale morphisms),
Lemma \ref{lemma-points-presentation} (points on an algebraic space in
terms of a presentation), and
Lemma \ref{lemma-topology-points} (openness quotient map).}.

\medskip\noindent
It remains to show that every irreducible closed subset
$T \subset |X|$ has a generic point. By
Lemma \ref{lemma-reduced-closed-subspace}
there exists a closed subspace $Z \subset X$ with $|Z| = |T|$.
Note that $U \times_X Z \to Z$ is a quasi-compact, surjective, \'etale
morphism from an affine scheme to $Z$, hence $Z$ is Zariski locally
quasi-separated by
Lemma \ref{lemma-quasi-separated-quasi-compact-pieces}.
By
Proposition \ref{proposition-locally-quasi-separated-open-dense-scheme}
we see that there exists an open dense subspace $Z' \subset Z$
which is a scheme. This means that $|Z'| \subset T$ is open dense.
Hence the topological space $|Z'|$ is irreducible, which means that
$Z'$ is an irreducible scheme. By
Schemes, Lemma \ref{schemes-lemma-scheme-sober}
we conclude that $|Z'|$ is the closure of a single point
$\eta \in |Z'| \subset T$ and hence also $T = \overline{\{\eta\}}$, and we win.
\end{proof}

\begin{lemma}
\label{lemma-quasi-compact-quasi-separated-spectral}
Let $S$ be a scheme. Let $X$ be a quasi-compact and quasi-separated
algebraic space over $S$. The topological space $|X|$ is a spectral space.
\end{lemma}

\begin{proof}
By Topology, Definition \ref{topology-definition-spectral-space}
we have to check that $|X|$ is sober, quasi-compact, has a basis
of quasi-compact opens, and the intersection of any two
quasi-compact opens is quasi-compact. By
Lemma \ref{lemma-quasi-separated-sober} we see that $|X|$ is sober.
By Lemma \ref{lemma-quasi-compact-space} we see that $|X|$ is quasi-compact.
By Lemma \ref{lemma-quasi-compact-affine-cover} there exists an affine scheme
$U$ and a surjective \'etale morphism $f : U \to X$.
Since $|f| : |U| \to |X|$ is open and continuous and since $|U|$ has
a basis of quasi-compact opens, we conclude that $|X|$ has a basis
of quasi-compact opens. Finally, suppose that
$A, B \subset |X|$ are quasi-compact open. Then $A = |X'|$ and $B = |X''|$
for some open subspaces $X', X'' \subset X$ (Lemma \ref{lemma-open-subspaces})
and we can choose affine schemes $V$ and $W$ and surjective
\'etale morphisms $V \to X'$ and $W \to X''$
(Lemma \ref{lemma-quasi-compact-affine-cover}).
Then $A \cap B$ is the image of
$|V \times_X W| \to |X|$ (Lemma \ref{lemma-points-cartesian}).
Since $V \times_X W$ is quasi-compact as $X$ is quasi-separated
(Lemma \ref{lemma-characterize-quasi-separated})
we conclude that $A \cap B$ is quasi-compact and the proof is finished.
\end{proof}

\noindent
The following lemma can be used to prove that
an algebraic space is isomorphic to the spectrum of a field.

\begin{lemma}
\label{lemma-point-like-spaces}
Let $S$ be a scheme. Let $k$ be a field.
Let $X$ be an algebraic space over $S$ and assume that there exists
a surjective \'etale morphism $\Spec(k) \to X$.
If $X$ is quasi-separated, then $X \cong \Spec(k')$
where $k/k'$ is a finite separable extension.
\end{lemma}

\begin{proof}
Set $R = \Spec(k) \times_X \Spec(k)$, so that we have a
fibre product diagram
$$
\xymatrix{
R \ar[r]_-s \ar[d]_-t & \Spec(k) \ar[d] \\
\Spec(k) \ar[r] & X
}
$$
By
Spaces, Lemma \ref{spaces-lemma-space-presentation}
we know $X = \Spec(k)/R$ is the quotient sheaf.
Because $\Spec(k) \to X$ is \'etale, the morphisms $s$ and $t$ are
\'etale. Hence $R = \coprod_{i \in I} \Spec(k_i)$ is a disjoint
union of spectra of fields, and both $s$ and $t$
induce finite separable field extensions $s, t : k \subset k_i$, see
Morphisms, Lemma \ref{morphisms-lemma-etale-over-field}.
Because
$$
R = \Spec(k) \times_X \Spec(k)
= (\Spec(k) \times_S \Spec(k)) \times_{X \times_S X, \Delta} X
$$
and since $\Delta$ is quasi-compact by assumption we conclude that
$R \to \Spec(k) \times_S \Spec(k)$ is quasi-compact.
Hence $R$ is quasi-compact as $\Spec(k) \times_S \Spec(k)$ is
affine. We conclude that $I$ is finite. This implies
that $s$ and $t$ are finite locally free morphisms. Hence by
Groupoids, Proposition \ref{groupoids-proposition-finite-flat-equivalence}
we conclude that $\Spec(k)/R$ is
represented by $\Spec(k')$, with $k' \subset k$ finite locally free
where
$$
k' = \{x \in k \mid s_i(x) = t_i(x)\text{ for all }i \in I\}
$$
It is easy to see that $k'$ is a field.
\end{proof}

\begin{remark}
\label{remark-cannot-decide-yet}
Lemma \ref{lemma-point-like-spaces} holds for decent algebraic spaces, see
Decent Spaces, Lemma \ref{decent-spaces-lemma-decent-point-like-spaces}.
In fact a decent algebraic space with one point is a scheme, see
Decent Spaces, Lemma \ref{decent-spaces-lemma-when-field}.
This also holds when $X$ is locally separated, because a
locally separated algebraic space is decent, see
Decent Spaces, Lemma \ref{decent-spaces-lemma-locally-separated-decent}.
\end{remark}







\section{\'Etale morphisms of algebraic spaces}
\label{section-etale-morphisms}

\noindent
This section really belongs in the chapter on morphisms of algebraic
spaces, but we need the notion of an algebraic space \'etale over another
in order to define the small \'etale site of an algebraic space.
Thus we need to do some preliminary work on \'etale morphisms from schemes to
algebraic spaces, and \'etale morphisms between algebraic spaces.
For more about \'etale morphisms of algebraic spaces, see
Morphisms of Spaces, Section \ref{spaces-morphisms-section-etale}.

\begin{lemma}
\label{lemma-etale-over-space}
Let $S$ be a scheme.
Let $X$ be an algebraic space over $S$.
Let $U$, $U'$ be schemes over $S$.
\begin{enumerate}
\item If $U \to U'$ is an \'etale morphism of schemes, and
if $U' \to X$ is an \'etale morphism from $U'$ to $X$, then the
composition $U \to X$ is an \'etale morphism from $U$ to $X$.
\item If $\varphi : U \to X$ and $\varphi' : U' \to X$ are
\'etale morphisms towards $X$, and if $\chi : U \to U'$ is a
morphism of schemes such that $\varphi = \varphi' \circ \chi$,
then $\chi$ is an \'etale morphism of schemes.
\item If $\chi : U \to U'$ is a surjective \'etale morphism
of schemes and $\varphi' : U' \to X$ is a morphism such that
$\varphi = \varphi' \circ \chi$ is \'etale, then $\varphi'$
is \'etale.
\end{enumerate}
\end{lemma}

\begin{proof}
Recall that our definition of an \'etale morphism from a scheme into an
algebraic space comes from
Spaces, Definition
\ref{spaces-definition-relative-representable-property}
via the fact that any morphism from a scheme into an algebraic space
is representable.

\medskip\noindent
Part (1) of the lemma follows from this, the fact that
\'etale morphisms are preserved under composition
(Morphisms, Lemma
\ref{morphisms-lemma-composition-etale})
and
Spaces, Lemmas
\ref{spaces-lemma-composition-representable-transformations-property} and
\ref{spaces-lemma-morphism-schemes-gives-representable-transformation-property}
(which are formal).

\medskip\noindent
To prove part (2) choose a scheme $W$ over $S$ and a
surjective \'etale morphism $W \to X$. Consider the base change
$\chi_W : W \times_X U \to W \times_X U'$ of $\chi$.
As $W \times_X U$ and $W \times_X U'$ are \'etale over $W$, we conclude that
$\chi_W$ is \'etale, by
Morphisms, Lemma \ref{morphisms-lemma-etale-permanence}.
On the other hand, in the commutative diagram
$$
\xymatrix{
W \times_X U \ar[r] \ar[d] & W \times_X U' \ar[d] \\
U \ar[r] & U'
}
$$
the two vertical arrows are \'etale and surjective.
Hence by
Descent, Lemma \ref{descent-lemma-syntomic-smooth-etale-permanence}
we conclude that $U \to U'$ is \'etale.

\medskip\noindent
To prove part (3) choose a scheme $W$ over $S$ and a morphism $W \to X$.
As above we consider the diagram
$$
\xymatrix{
W \times_X U \ar[r] \ar[d] & W \times_X U' \ar[d] \ar[r] & W \ar[d] \\
U \ar[r] & U' \ar[r] & X
}
$$
Now we know that $W \times_X U \to W \times_X U'$ is surjective \'etale
(as a base change of $U \to U'$)
and that $W \times_X U \to W$ is \'etale. Thus $W \times_X U' \to W$
is \'etale by Descent, Lemma
\ref{descent-lemma-syntomic-smooth-etale-permanence}. By definition
this means that $\varphi'$ is \'etale.
\end{proof}

\begin{definition}
\label{definition-etale}
Let $S$ be a scheme.
A morphism $f : X \to Y$ between algebraic spaces over $S$ is
called {\it \'etale} if and only if for every \'etale morphism
$\varphi : U \to X$ where $U$ is a scheme, the composition
$f \circ \varphi$ is \'etale also.
\end{definition}

\noindent
If $X$ and $Y$ are schemes, then this agree with the usual notion of an
\'etale morphism of schemes. In fact, whenever $X \to Y$ is a representable
morphism of algebraic spaces, then this agrees with the notion defined via
Spaces, Definition \ref{spaces-definition-relative-representable-property}.
This follows by combining Lemma \ref{lemma-etale-local} below and
Spaces, Lemma \ref{spaces-lemma-representable-morphisms-spaces-property}.

\begin{lemma}
\label{lemma-etale-local}
Let $S$ be a scheme.
Let $f : X \to Y$ be a morphism of algebraic spaces over $S$.
The following are equivalent:
\begin{enumerate}
\item $f$ is \'etale,
\item there exists a surjective \'etale morphism $\varphi : U \to X$,
where $U$ is a scheme, such that the composition $f \circ \varphi$ is
\'etale (as a morphism of algebraic spaces),
\item there exists a surjective \'etale morphism $\psi : V \to Y$,
where $V$ is a scheme, such that the base change $V \times_Y X \to V$
is \'etale (as a morphism of algebraic spaces),
\item there exists a commutative diagram
$$
\xymatrix{
U \ar[d] \ar[r] & V \ar[d] \\
X \ar[r] & Y
}
$$
where $U$, $V$ are schemes, the vertical arrows are \'etale, and the
left vertical arrow is surjective such that the horizontal arrow is \'etale.
\end{enumerate}
\end{lemma}

\begin{proof}
Let us prove that (4) implies (1). Assume a diagram as in (4) given.
Let $W \to X$ be an \'etale morphism with $W$ a scheme. Then we see
that $W \times_X U \to U$ is \'etale. Hence $W \times_X U \to V$ is \'etale
as the composition of the \'etale morphisms of schemes $W \times_X U \to U$
and $U \to V$. Therefore $W \times_X U \to Y$ is \'etale by
Lemma \ref{lemma-etale-over-space} (1). Since also
the projection $W \times_X U \to W$ is surjective and \'etale, we conclude
from Lemma \ref{lemma-etale-over-space} (3) that $W \to Y$ is \'etale.

\medskip\noindent
Let us prove that (1) implies (4). Assume (1). Choose a commutative diagram
$$
\xymatrix{
U \ar[d] \ar[r] & V \ar[d] \\
X \ar[r] & Y
}
$$
where $U \to X$ and $V \to Y$ are surjective and \'etale, see
Spaces, Lemma \ref{spaces-lemma-lift-morphism-presentations}.
By assumption the morphism $U \to Y$ is \'etale,
and hence $U \to V$ is \'etale by Lemma \ref{lemma-etale-over-space} (2).

\medskip\noindent
We omit the proof that (2) and (3) are also equivalent to (1).
\end{proof}

\begin{lemma}
\label{lemma-composition-etale}
The composition of two \'etale morphisms of algebraic spaces
is \'etale.
\end{lemma}

\begin{proof}
This is immediate from the definition.
\end{proof}

\begin{lemma}
\label{lemma-base-change-etale}
The base change of an \'etale morphism of algebraic spaces
by any morphism of algebraic spaces is \'etale.
\end{lemma}

\begin{proof}
Let $X \to Y$ be an \'etale morphism of algebraic spaces over $S$.
Let $Z \to Y$ be a morphism of algebraic spaces.
Choose a scheme $U$ and a surjective \'etale morphism $U \to X$.
Choose a scheme $W$ and a surjective \'etale morphism $W \to Z$.
Then $U \to Y$ is \'etale, hence in the diagram
$$
\xymatrix{
W \times_Y U \ar[d] \ar[r] & W \ar[d] \\
Z \times_Y X \ar[r] & Z
}
$$
the top horizontal arrow is \'etale.
Moreover, the left vertical arrow is surjective
and \'etale (verification omitted). Hence we conclude that the lower
horizontal arrow is \'etale by Lemma \ref{lemma-etale-local}.
\end{proof}

\begin{lemma}
\label{lemma-etale-permanence}
Let $S$ be a scheme. Let $X, Y, Z$ be algebraic spaces.
Let $g : X \to Z$, $h : Y \to Z$ be \'etale morphisms and let
$f : X \to Y$ be a morphism such that $h \circ f = g$.
Then $f$ is \'etale.
\end{lemma}

\begin{proof}
Choose a commutative diagram
$$
\xymatrix{
U \ar[d] \ar[r]_\chi & V \ar[d] \\
X \ar[r] & Y
}
$$
where $U \to X$ and $V \to Y$ are surjective and \'etale, see
Spaces, Lemma \ref{spaces-lemma-lift-morphism-presentations}.
By assumption the morphisms $\varphi : U \to X \to Z$ and
$\psi : V \to Y \to Z$ are \'etale. Moreover, $\psi \circ \chi = \varphi$
by our assumption on $f, g, h$.
Hence $U \to V$ is \'etale by Lemma \ref{lemma-etale-over-space}
part (2).
\end{proof}

\begin{lemma}
\label{lemma-etale-open}
Let $S$ be a scheme.
If $X \to Y$ is an \'etale morphism of algebraic spaces over $S$,
then the associated map $|X| \to |Y|$ of topological spaces
is open.
\end{lemma}

\begin{proof}
This is clear from the diagram in
Lemma \ref{lemma-etale-local}
and
Lemma \ref{lemma-topology-points}.
\end{proof}

\noindent
Finally, here is a fun lemma. It is not true that an algebraic space
with an \'etale morphism towards a scheme is a scheme, see
Spaces, Example \ref{spaces-example-non-representable-descent}.
But it is true if the target is the spectrum of a field.

\begin{lemma}
\label{lemma-etale-over-field-scheme}
Let $S$ be a scheme. Let $X \to \Spec(k)$
be \'etale morphism over $S$, where $k$ is a field.
Then $X$ is a scheme.
\end{lemma}

\begin{proof}
Let $U$ be an affine scheme, and let $U \to X$ be an \'etale morphism. By
Definition \ref{definition-etale}
we see that $U \to \Spec(k)$ is an \'etale
morphism. Hence $U = \coprod_{i = 1, \ldots, n} \Spec(k_i)$
is a finite disjoint union of spectra of finite separable extensions
$k_i$ of $k$, see
Morphisms, Lemma \ref{morphisms-lemma-etale-over-field}.
The $R = U \times_X U \to U \times_{\Spec(k)} U$ is a monomorphism
and $U \times_{\Spec(k)} U$ is also a finite disjoint union of
spectra of finite separable extensions of $k$. Hence by
Schemes, Lemma \ref{schemes-lemma-mono-towards-spec-field}
we see that $R$ is similarly a finite disjoint union of
spectra of finite separable extensions of $k$.
This $U$ and $R$ are affine and
both projections $R \to U$ are finite locally free.
Hence $U/R$ is a scheme by
Groupoids, Proposition \ref{groupoids-proposition-finite-flat-equivalence}.
By
Spaces, Lemma \ref{spaces-lemma-finding-opens}
it is also an open subspace of $X$. By
Lemma \ref{lemma-subscheme}
we conclude that $X$ is a scheme.
\end{proof}













\section{Spaces and fpqc coverings}
\label{section-fpqc}

\noindent
Let $S$ be a scheme.
An algebraic space over $S$ is defined as a sheaf in the fppf topology with
additional properties. Hence it is not immediately clear that it satisfies
the sheaf property for the fpqc topology (see
Topologies, Definition \ref{topologies-definition-sheaf-property-fpqc}).
In this section we give Gabber's argument showing this is true.
However, when we say that the algebraic space $X$ satisfies the
sheaf property for the fpqc topology we really only consider fpqc
coverings $\{f_i : T_i \to T\}_{i \in I}$ such that $T, T_i$ are
objects of the big site $(\Sch/S)_{fppf}$ (as per
our conventions, see Section \ref{section-conventions}).

\begin{proposition}[Gabber]
\label{proposition-sheaf-fpqc}
Let $S$ be a scheme. Let $X$ be an algebraic space over $S$. Then
$X$ satisfies the sheaf property for the fpqc topology.
\end{proposition}

\begin{proof}
Since $X$ is a sheaf for the Zariski topology it suffices to show
the following. Given a surjective flat morphism of affines
$f : T' \to T$ we have:
$X(T)$ is the equalizer of the two maps $X(T') \to X(T' \times_T T')$.
See Topologies, Lemma \ref{topologies-lemma-sheaf-property-fpqc}
(there is a little argument omitted here because the lemma cited
is formulated for functors defined on the category of all schemes).

\medskip\noindent
Let $a, b : T \to X$ be two morphisms such that $a \circ f = b \circ f$.
We have to show $a = b$. Consider the fibre product
$$
E = X \times_{\Delta_{X/S}, X \times_S X, (a, b)} T.
$$
By Spaces, Lemma \ref{spaces-lemma-properties-diagonal}
the morphism $\Delta_{X/S}$ is a representable monomorphism. Hence
$E \to T$ is a monomorphism of schemes. Our assumption that
$a \circ f = b \circ f$ implies that $T' \to T$ factors (uniquely) through $E$.
Consider the commutative diagram
$$
\xymatrix{
T' \times_T E \ar[r] \ar[d] & E \ar[d] \\
T' \ar[r] \ar@/^5ex/[u] \ar[ru] & T
}
$$
Since the projection $T' \times_T E \to T'$ is a monomorphism
with a section we conclude it is an isomorphism. Hence we conclude that
$E \to T$ is an isomorphism by
Descent, Lemma \ref{descent-lemma-descending-property-isomorphism}.
This means $a = b$ as desired.

\medskip\noindent
Next, let $c : T' \to X$ be a morphism such that the two compositions
$T' \times_T T' \to T' \to X$ are the same. We have to find a morphism
$a : T \to X$ whose composition with $T' \to T$ is $c$. Choose an
affine scheme $U$ and an \'etale morphism $U \to X$ such that the image
of $|U| \to |X|$ contains the image of $|c| : |T'| \to |X|$.
This is possible by Lemmas \ref{lemma-topology-points} and
\ref{lemma-cover-by-union-affines}, the fact that a finite disjoint union of
affines is affine, and the fact that $|T'|$ is quasi-compact
(small argument omitted). Since $U \to X$ is separated
(Lemma \ref{lemma-separated-cover}), we see that
$$
V = U \times_{X, c} T' \longrightarrow T'
$$
is a surjective, \'etale, separated morphism of schemes
(to see that it is surjective use Lemma \ref{lemma-points-cartesian}
and our choice of $U \to X$). The fact that
$c \circ \text{pr}_0 = c \circ \text{pr}_1$ means that we obtain a
descent datum on $V/T'/T$
(Descent, Definition \ref{descent-definition-descent-datum})
because
\begin{align*}
V \times_{T'} (T' \times_T T')
& =
U \times_{X, c \circ \text{pr}_0} (T' \times_T T') \\
& =
(T' \times_T T') \times_{c \circ \text{pr}_1, X} U \\
& =
(T' \times_T T') \times_{T'} V
\end{align*}
The morphism $V \to T'$ is ind-quasi-affine by
More on Morphisms, Lemma
\ref{more-morphisms-lemma-etale-separated-ind-quasi-affine}
(because \'etale morphisms are locally quasi-finite, see
Morphisms, Lemma \ref{morphisms-lemma-etale-locally-quasi-finite}).
By More on Groupoids, Lemma \ref{more-groupoids-lemma-ind-quasi-affine}
the descent datum is effective. Say $W \to T$ is a morphism
such that there is an isomorphism $\alpha : T' \times_T W \to V$
compatible with the given descent datum on $V$ and the canonical descent
datum on $T' \times_T W$. Then $W \to T$ is surjective and \'etale
(Descent, Lemmas \ref{descent-lemma-descending-property-surjective} and
\ref{descent-lemma-descending-property-etale}).
Consider the composition
$$
b' : T' \times_T W \longrightarrow V = U \times_{X, c} T' \longrightarrow U
$$
The two compositions
$b' \circ (\text{pr}_0, 1), 
b' \circ (\text{pr}_1, 1) :
(T' \times_T T') \times_T W \to T' \times_T W \to U$
agree by our choice of $\alpha$ and the corresponding property of $c$
(computation omitted). Hence $b'$ descends to a morphism $b : W \to U$ by
Descent, Lemma \ref{descent-lemma-fpqc-universal-effective-epimorphisms}.
The diagram
$$
\xymatrix{
T' \times_T W \ar[r] \ar[d] & W \ar[r]_b & U \ar[d] \\
T' \ar[rr]^c &  & X
}
$$
is commutative. What this means is that we have proved the existence
of $a$ \'etale locally on $T$, i.e., we have an $a' : W \to X$.
However, since we have proved uniqueness
in the first paragraph, we find that this \'etale local solution
satisfies the glueing condition, i.e., we have
$\text{pr}_0^*a' = \text{pr}_1^*a'$ as elements of $X(W \times_T W)$.
Since $X$ is an \'etale sheaf we find a unique $a \in X(T)$ restricting
to $a'$ on $W$.
\end{proof}









\section{The \'etale site of an algebraic space}
\label{section-etale-site}

\noindent
In this section we define the small \'etale site of an algebraic space.
This is the analogue of the small \'etale site $S_\etale$ of a scheme.
Lemma \ref{lemma-etale-over-space} implies that in the definition below
any morphism between objects of the \'etale site of $X$ is \'etale, and that
any scheme \'etale over an object of $X_\etale$ is also an object of
$X_\etale$.

\begin{definition}
\label{definition-etale-site}
Let $S$ be a scheme.
Let $\Sch_{fppf}$ be a big fppf site containing $S$,
and let $\Sch_\etale$ be the corresponding big \'etale site
(i.e., having the same underlying category).
Let $X$ be an algebraic space over $S$.
The {\it small \'etale site $X_\etale$} of $X$ is defined as follows:
\begin{enumerate}
\item An object of $X_\etale$ is a morphism $\varphi : U \to X$
where $U \in \Ob((\Sch/S)_\etale)$ is a scheme and
$\varphi$ is an \'etale morphism,
\item a morphism $(\varphi : U \to X) \to (\varphi' : U' \to X)$
is given by a morphism of schemes $\chi : U \to U'$ such that
$\varphi = \varphi' \circ \chi$, and
\item a family of morphisms $\{(U_i \to X) \to (U \to X)\}_{i \in I}$
of $X_\etale$ is a covering if and only if $\{U_i \to U\}_{i \in I}$
is a covering of $(\Sch/S)_\etale$.
\end{enumerate}
\end{definition}

\noindent
A consequence of our choice is that the \'etale site of an algebraic space
in general does not have a final object! On the other hand, if $X$ happens
to be a scheme, then the definition above agrees with
Topologies, Definition \ref{topologies-definition-big-small-etale}.

\medskip\noindent
The above is our default site, but there are a couple of variants
which we will also use. Namely, we can consider all {\it algebraic spaces}
$U$ which are \'etale over $X$ and this produces the site
$X_{spaces, \etale}$ we define below or we can consider all {\it affine schemes}
$U$ which are \'etale over $X$ and this produces the site
$X_{affine, \etale}$ we define below. The first of these two notions
is used when discussing functoriality of the small \'etale site, see
Lemma \ref{lemma-functoriality-etale-site}.

\begin{definition}
\label{definition-spaces-etale-site}
Let $S$ be a scheme.
Let $\Sch_{fppf}$ be a big fppf site containing $S$,
and let $\Sch_\etale$ be the corresponding big \'etale site
(i.e., having the same underlying category).
Let $X$ be an algebraic space over $S$.
The site {\it $X_{spaces, \etale}$} of $X$ is defined as follows:
\begin{enumerate}
\item An object of $X_{spaces, \etale}$ is a morphism
$\varphi : U \to X$ where $U$ is an algebraic space over $S$ and
$\varphi$ is an \'etale morphism of algebraic spaces over $S$,
\item a morphism $(\varphi : U \to X) \to (\varphi' : U' \to X)$ of
$X_{spaces, \etale}$ is given by a morphism of algebraic spaces
$\chi : U \to U'$ such that $\varphi = \varphi' \circ \chi$, and
\item a family of morphisms
$\{\varphi_i : (U_i \to X) \to (U \to X)\}_{i \in I}$
of $X_{spaces, \etale}$ is a covering if and only if
$|U| = \bigcup \varphi_i(|U_i|)$.
\end{enumerate}
As usual we choose a set of coverings of this type, including at least
the coverings in $X_\etale$, as in
Sets, Lemma \ref{sets-lemma-coverings-site}
to turn $X_{spaces, \etale}$ into a site.
\end{definition}

\noindent
Since the identity morphism of $X$ is \'etale it is clear that
$X_{spaces, \etale}$ does have a final object.
Let us show right away that the corresponding topos equals the
small \'etale topos of $X$.

\begin{lemma}
\label{lemma-compare-etale-sites}
The functor
$$
X_\etale \longrightarrow X_{spaces, \etale}, \quad
U/X \longmapsto U/X
$$
is a special cocontinuous functor
(Sites, Definition \ref{sites-definition-special-cocontinuous-functor})
and hence induces an equivalence of topoi
$\Sh(X_\etale) \to \Sh(X_{spaces, \etale})$.
\end{lemma}

\begin{proof}
We have to show that the functor satisfies the assumptions (1) -- (5) of
Sites, Lemma \ref{sites-lemma-equivalence}.
It is clear that the functor is continuous and cocontinuous, which
proves assumptions (1) and (2).
Assumptions (3) and (4) hold simply because the functor is fully faithful.
Assumption (5) holds, because an algebraic space by definition has
a covering by a scheme.
\end{proof}

\begin{remark}
\label{remark-explain-equivalence}
Let us explain the meaning of Lemma \ref{lemma-compare-etale-sites}.
Let $S$ be a scheme, and let $X$ be an algebraic space over $S$.
Let $\mathcal{F}$ be a sheaf on the small \'etale site $X_\etale$ of
$X$. The lemma says that there exists a unique sheaf $\mathcal{F}'$ on
$X_{spaces, \etale}$ which restricts back to $\mathcal{F}$ on the
subcategory $X_\etale$. If $U \to X$ is an \'etale morphism of
algebraic spaces, then how do we compute $\mathcal{F}'(U)$? Well, by definition
of an algebraic space there exists a scheme $U'$ and a surjective
\'etale morphism $U' \to U$. Then $\{U' \to U\}$ is a covering in
$X_{spaces, \etale}$ and hence we get an equalizer diagram
$$
\xymatrix{
\mathcal{F}'(U) \ar[r] &
\mathcal{F}(U') \ar@<1ex>[r] \ar@<-1ex>[r] &
\mathcal{F}(U' \times_U U').
}
$$
Note that $U' \times_U U'$ is a scheme, and hence we may
write $\mathcal{F}$ and not $\mathcal{F}'$.
Thus we see how to compute $\mathcal{F}'$
when given the sheaf $\mathcal{F}$.
\end{remark}

\begin{definition}
\label{definition-affine-etale-site}
Let $S$ be a scheme.
Let $\Sch_{fppf}$ be a big fppf site containing $S$,
and let $\Sch_\etale$ be the corresponding big \'etale site
(i.e., having the same underlying category).
Let $X$ be an algebraic space over $S$.
The site {\it $X_{affine, \etale}$} of $X$ is defined as follows:
\begin{enumerate}
\item An object of $X_{affine, \etale}$ is a morphism
$\varphi : U \to X$ where $U \in \Ob((\Sch/S)_\etale)$ is an affine scheme and
$\varphi$ is an \'etale morphism,
\item a morphism $(\varphi : U \to X) \to (\varphi' : U' \to X)$ of
$X_{affine, \etale}$ is given by a morphism of schemes
$\chi : U \to U'$ such that $\varphi = \varphi' \circ \chi$, and
\item a family of morphisms
$\{\varphi_i : (U_i \to X) \to (U \to X)\}_{i \in I}$
of $X_{affine, \etale}$ is a covering if and only if
$\{U_i \to U\}$ is a standard \'etale covering, see
Topologies, Definition \ref{topologies-definition-standard-etale}.
\end{enumerate}
As usual we choose a set of coverings of this type, as in
Sets, Lemma \ref{sets-lemma-coverings-site}
to turn $X_{affine, \etale}$ into a site.
\end{definition}

\begin{lemma}
\label{lemma-alternative}
Let $S$ be a scheme. Let $X$ be an algebraic space over $S$.
The functor $X_{affine, \etale} \to X_\etale$
is special cocontinuous and induces an equivalence of topoi from
$\Sh(X_{affine, \etale})$ to $\Sh(X_\etale)$.
\end{lemma}

\begin{proof}
Omitted. Hint: compare with the proof of
Topologies, Lemma \ref{topologies-lemma-affine-big-site-etale}.
\end{proof}

\begin{definition}
\label{definition-etale-topos}
Let $S$ be a scheme. Let $X$ be an algebraic space over $S$.
The {\it \'etale topos} of $X$, or more precisely the
{\it small \'etale topos} of $X$ is the category
$\Sh(X_\etale)$
of sheaves of sets on $X_\etale$.
\end{definition}

\noindent
By
Lemma \ref{lemma-compare-etale-sites}
we have
$\Sh(X_\etale) = \Sh(X_{spaces, \etale})$,
so we can also think of this as the category of sheaves of sets on
$X_{spaces, \etale}$. Similarly, by
Lemma \ref{lemma-alternative}
we see that
$\Sh(X_\etale) = \Sh(X_{affine, \etale})$.
It turns out that the topos is functorial with respect to morphisms
of algebraic spaces. Here is a precise statement.

\begin{lemma}
\label{lemma-functoriality-etale-site}
Let $S$ be a scheme.
Let $f : X \to Y$ be a morphism of algebraic spaces over $S$.
\begin{enumerate}
\item The continuous functor
$$
Y_{spaces, \etale} \longrightarrow X_{spaces, \etale}, \quad
V \longmapsto X \times_Y V
$$
induces a morphism of sites
$$
f_{spaces, \etale} :
X_{spaces, \etale}
\to
Y_{spaces, \etale}.
$$
\item The rule $f \mapsto f_{spaces, \etale}$ is compatible with
compositions, in other words $(f \circ g)_{spaces, \etale}
= f_{spaces, \etale} \circ g_{spaces, \etale}$ (see
Sites, Definition \ref{sites-definition-composition-morphisms-sites}).
\item The morphism of topoi associated to $f_{spaces, \etale}$
induces, via Lemma \ref{lemma-compare-etale-sites}, a morphism of topoi
$f_{small} : \Sh(X_\etale) \to \Sh(Y_\etale)$
whose construction is compatible with compositions.
\item If $f$ is a representable morphism of algebraic spaces,
then $f_{small}$ comes from a morphism of sites
$X_\etale \to Y_\etale$,
corresponding to the continuous functor $V \mapsto X \times_Y V$.
\end{enumerate}
\end{lemma}

\begin{proof}
Let us show that the functor described in (1) satisfies the assumptions
of Sites, Proposition \ref{sites-proposition-get-morphism}.
Thus we have to show that
$Y_{spaces, \etale}$ has a final object (namely $Y$) and that
the functor transforms this into a final object in $X_{spaces, \etale}$
(namely $X$). This is clear as $X \times_Y Y = X$ in any category.
Next, we have to show that $Y_{spaces, \etale}$ has fibre products.
This is true since the category of algebraic spaces has fibre products,
and since $V \times_Y V'$ is \'etale over $Y$ if $V$ and $V'$ are \'etale
over $Y$ (see Lemmas \ref{lemma-composition-etale} and
\ref{lemma-base-change-etale} above).
OK, so the proposition applies and we see that we get a morphism
of sites as described in (1).

\medskip\noindent
Part (2) you get by unwinding the definitions.
Part (3) is clear by using the equivalences for $X$ and $Y$
from Lemma \ref{lemma-compare-etale-sites} above.
Part (4) follows, because if $f$ is representable, then the
functors above fit into a commutative diagram
$$
\xymatrix{
X_\etale \ar[r] &
X_{spaces, \etale} \\
Y_\etale \ar[r] \ar[u] &
Y_{spaces, \etale} \ar[u]
}
$$
of categories.
\end{proof}

\noindent
We can do a little bit better than the lemma above in describing
the relationship between sheaves on $X$ and sheaves on $Y$.
Namely, we can formulate this in turns of $f$-maps, compare
Sheaves, Definition \ref{sheaves-definition-f-map}, as follows.

\begin{definition}
\label{definition-f-map}
Let $S$ be a scheme.
Let $f : X \to Y$ be a morphism of algebraic spaces over $S$.
Let $\mathcal{F}$ be a sheaf of sets on $X_\etale$ and
let $\mathcal{G}$ be a sheaf of sets on $Y_\etale$.
An {\it $f$-map $\varphi : \mathcal{G} \to \mathcal{F}$}
is a collection of maps
$\varphi_{(U, V, g)} : \mathcal{G}(V) \to \mathcal{F}(U)$
indexed by commutative diagrams
$$
\xymatrix{
U \ar[d]_g \ar[r] & X \ar[d]^f \\
V \ar[r] & Y
}
$$
where $U \in X_\etale$, $V \in Y_\etale$ such that whenever
given an extended diagram
$$
\xymatrix{
U' \ar[r] \ar[d]_{g'} & U \ar[d]_g \ar[r] & X \ar[d]^f \\
V' \ar[r] & V \ar[r] & Y
}
$$
with $V' \to V$ and $U' \to U$ \'etale morphisms of schemes the diagram
$$
\xymatrix{
\mathcal{G}(V)
\ar[rr]_{\varphi_{(U, V, g)}}
\ar[d]_{\text{restriction of }\mathcal{G}} & &
\mathcal{F}(U)
\ar[d]^{\text{restriction of }\mathcal{F}} \\
\mathcal{G}(V')
\ar[rr]^{\varphi_{(U', V', g')}} & &
\mathcal{F}(U')
}
$$
commutes.
\end{definition}

\begin{lemma}
\label{lemma-f-map}
Let $S$ be a scheme.
Let $f : X \to Y$ be a morphism of algebraic spaces over $S$.
Let $\mathcal{F}$ be a sheaf of sets on $X_\etale$ and
let $\mathcal{G}$ be a sheaf of sets on $Y_\etale$.
There are canonical bijections between the following three sets:
\begin{enumerate}
\item The set of maps $\mathcal{G} \to f_{small, *}\mathcal{F}$.
\item The set of maps $f_{small}^{-1}\mathcal{G} \to \mathcal{F}$.
\item The set of $f$-maps $\varphi : \mathcal{G} \to \mathcal{F}$.
\end{enumerate}
\end{lemma}

\begin{proof}
Note that (1) and (2) are the same because the functors $f_{small, *}$
and $f_{small}^{-1}$ are a pair of adjoint functors.
Suppose that $\alpha : f_{small}^{-1}\mathcal{G} \to \mathcal{F}$
is a map of sheaves on $Y_\etale$. Let a diagram
$$
\xymatrix{
U \ar[d]_g \ar[r]_{j_U} & X \ar[d]^f \\
V \ar[r]^{j_V} & Y
}
$$
as in Definition \ref{definition-f-map} be given.
By the commutativity of the diagram we also get a map
$g_{small}^{-1}(j_V)^{-1}\mathcal{G} \to (j_U)^{-1}\mathcal{F}$
(compare Sites, Section \ref{sites-section-localize} for the
description of the localization functors). Hence we certainly
get a map
$\varphi_{(V, U, g)} :
\mathcal{G}(V) = (j_V)^{-1}\mathcal{G}(V)
\to
(j_U)^{-1}\mathcal{F}(U) = \mathcal{F}(U)$.
We omit the verification that this rule is compatible with
further restrictions and defines an $f$-map from $\mathcal{G}$ to
$\mathcal{F}$.

\medskip\noindent
Conversely, suppose that we are given an $f$-map
$\varphi = (\varphi_{(U, V, g)})$.
Let $\mathcal{G}'$ (resp.\ $\mathcal{F}'$) denote the extension of
$\mathcal{G}$ (resp.\ $\mathcal{F}$) to $Y_{spaces, \etale}$
(resp.\ $X_{spaces, \etale}$), see
Lemma \ref{lemma-compare-etale-sites}.
Then we have to construct a map of sheaves
$$
\mathcal{G}' \longrightarrow (f_{spaces, \etale})_*\mathcal{F}'
$$
To do this, let $V \to Y$ be an \'etale morphism of algebraic spaces.
We have to construct a map of sets
$$
\mathcal{G}'(V) \to \mathcal{F}'(X \times_Y V)
$$
Choose an \'etale surjective morphism $V' \to V$ with $V'$ a scheme,
and after that choose an \'etale surjective morphism
$U' \to X \times_U V'$ with $U'$ a scheme. We get a morphism of
schemes $g' : U' \to V'$ and also a morphism of schemes
$$
g'' : U' \times_{X \times_Y V} U' \longrightarrow V' \times_V V'
$$
Consider the following diagram
$$
\xymatrix{
\mathcal{F}'(X \times_Y V) \ar[r] &
\mathcal{F}(U') \ar@<1ex>[r] \ar@<-1ex>[r] &
\mathcal{F}(U' \times_{X \times_Y V} U') \\
\mathcal{G}'(X \times_Y V) \ar[r] \ar@{..>}[u] &
\mathcal{G}(V') \ar@<1ex>[r] \ar@<-1ex>[r] \ar[u]_{\varphi_{(U', V', g')}} &
\mathcal{G}(V' \times_V V') \ar[u]_{\varphi_{(U'', V'', g'')}}
}
$$
The compatibility of the maps $\varphi_{...}$
with restriction shows that the two right squares commute.
The definition of coverings in $X_{spaces, \etale}$ shows that
the horizontal rows are equalizer diagrams. Hence we get
the dotted arrow. We leave it to the reader to show that these
arrows are compatible with the restriction mappings.
\end{proof}

\noindent
If the morphism of algebraic spaces $X \to Y$ is \'etale, then the morphism
of topoi $\Sh(X_\etale) \to \Sh(Y_\etale)$
is a localization. Here is a statement.

\begin{lemma}
\label{lemma-etale-morphism-topoi}
Let $S$ be a scheme, and let $f : X \to Y$ be a morphism of algebraic spaces
over $S$. Assume $f$ is \'etale. In this case there is a functor
$$
j : X_\etale \to Y_\etale, \quad
(\varphi : U \to X) \mapsto (f \circ \varphi : U \to Y)
$$
which is cocontinuous. The morphism of topoi $f_{small}$ is the
morphism of topoi associated to $j$, see
Sites, Lemma \ref{sites-lemma-cocontinuous-morphism-topoi}.
Moreover, $j$ is continuous as well, hence
Sites, Lemma \ref{sites-lemma-when-shriek}
applies. In particular $f_{small}^{-1}\mathcal{G}(U) = \mathcal{G}(jU)$
for all sheaves $\mathcal{G}$ on $Y_\etale$.
\end{lemma}

\begin{proof}
Note that by our very definition of an \'etale morphism of algebraic spaces
(Definition \ref{definition-etale}) it is
indeed the case that the rule given defines a functor $j$ as indicated.
It is clear that $j$ is cocontinuous and continuous, simply because a covering
$\{U_i \to U\}$ of $j(\varphi : U \to X)$ in $Y_\etale$ is the
same thing as a covering of $(\varphi : U \to X)$ in $X_\etale$. It
remains to show that $j$ induces the same morphism of topoi as $f_{small}$.
To see this we consider the diagram
$$
\xymatrix{
X_\etale \ar[r] \ar[d]^j &
X_{spaces, \etale} \ar@/_/[d]_{j_{spaces}} \\
Y_\etale \ar[r] &
Y_{spaces, \etale} \ar@/_/[u]_{v : V \mapsto X \times_Y V}
}
$$
of categories. Here the functor $j_{spaces}$ is the obvious extension of $j$
to the category $X_{spaces, \etale}$. Thus the inner square is
commutative. In fact $j_{spaces}$ can be identified with the
localization functor
$j_X : Y_{spaces, \etale}/X \to Y_{spaces, \etale}$
discussed in
Sites, Section \ref{sites-section-localize}.
Hence, by
Sites, Lemma \ref{sites-lemma-localize-given-products}
the cocontinuous functor $j_{spaces}$ and the functor $v$ of the diagram
induce the same morphism of topoi. By
Sites, Lemma \ref{sites-lemma-composition-cocontinuous}
the commutativity of the inner square (consisting of cocontinuous functors
between sites) gives a commutative diagram of associated morphisms of topoi.
Hence, by the construction of $f_{small}$ in
Lemma \ref{lemma-functoriality-etale-site} we win.
\end{proof}

\noindent
The lemma above says that the pullback of $\mathcal{G}$ via an \'etale morphism
$f : X \to Y$ of algebraic spaces is simply the restriction of $\mathcal{G}$
to the category $X_\etale$. We will often use the short hand
\begin{equation}
\label{equation-restrict}
\mathcal{G}|_{X_\etale} = f_{small}^{-1}\mathcal{G}
\end{equation}
to indicate this. Note that the functor
$j : X_\etale \to Y_\etale$
of the lemma in this situation is faithful, but not fully faithful in
general. We will discuss this in a more technical fashion in
Section \ref{section-localize}.

\begin{lemma}
\label{lemma-pushforward-etale-base-change}
Let $S$ be a scheme. Let
$$
\xymatrix{
X' \ar[r] \ar[d]_{f'} & X \ar[d]^f \\
Y' \ar[r]^g & Y
}
$$
be a cartesian square of algebraic spaces over $S$. Let
$\mathcal{F}$ be a sheaf on $X_\etale$. If $g$ is \'etale, then
\begin{enumerate}
\item $f'_{small, *}(\mathcal{F}|_{X'}) = (f_{small, *}\mathcal{F})|_{Y'}$
in $\Sh(Y'_\etale)$\footnote{Also
$(f')_{small}^{-1}(\mathcal{G}|_{Y'}) = (f_{small}^{-1}\mathcal{G})|_{X'}$
because of commutativity of the diagram and (\ref{equation-restrict})}, and
\item if $\mathcal{F}$ is an abelian sheaf, then
$R^if'_{small, *}(\mathcal{F}|_{X'}) = (R^if_{small, *}\mathcal{F})|_{Y'}$.
\end{enumerate}
\end{lemma}

\begin{proof}
Consider the following diagram of functors
$$
\xymatrix{
X'_{spaces, \etale} \ar[r]_j &
X_{spaces, \etale} \\
Y'_{spaces, \etale} \ar[r]^j \ar[u]^{V' \mapsto V' \times_{Y'} X'} &
Y_{spaces, \etale} \ar[u]_{V \mapsto V \times_Y X}
}
$$
The horizontal arrows are localizations and the vertical arrows induce
morphisms of sites. Hence the last statement of
Sites, Lemma \ref{sites-lemma-localize-morphism}
gives (1). To see (2) apply (1) to an injective resolution of $\mathcal{F}$
and use that restriction is exact and preserves injectives (see
Cohomology on Sites, Lemma \ref{sites-cohomology-lemma-cohomology-of-open}).
\end{proof}

\noindent
The following lemma says that you can think of a sheaf on the small
\'etale site of an algebraic space as a compatible collection of sheaves
on the small \'etale sites of schemes \'etale over the space. Please note
that all the comparison mappings $c_f$ in the lemma are isomorphisms,
which is compatible with
Topologies, Lemma \ref{topologies-lemma-characterize-sheaf-big-etale}
and the fact that all morphisms between objects of $X_\etale$
are \'etale.

\begin{lemma}
\label{lemma-characterize-sheaf-small-etale}
Let $S$ be a scheme. Let $X$ be an algebraic space over $S$.
A sheaf $\mathcal{F}$ on $X_\etale$ is given by the following data:
\begin{enumerate}
\item for every $U \in \Ob(X_\etale)$ a sheaf
$\mathcal{F}_U$ on $U_\etale$,
\item for every $f : U' \to U$ in $X_\etale$ an isomorphism
$c_f : f_{small}^{-1}\mathcal{F}_U \to \mathcal{F}_{U'}$.
\end{enumerate}
These data are subject to the condition that given any $f : U' \to U$
and $g : U'' \to U'$ in $X_\etale$ the composition
$c_g \circ g_{small}^{-1} c_f$ is equal to $c_{f \circ g}$.
\end{lemma}

\begin{proof}
We may interpret $g_{small}^{-1}$ as in Lemma \ref{lemma-etale-morphism-topoi}.
Then the lemma follows from a general fact about sites, see
Sites, Lemma \ref{sites-lemma-glue-sheaves-absolute}.
\end{proof}

\noindent
Let $S$ be a scheme. Let $X$ be an algebraic space over $S$.
Let $X = U/R$ be a presentation of $X$ coming from any surjective
\'etale morphism $\varphi : U \to X$, see
Spaces, Definition \ref{spaces-definition-presentation}.
In particular, we obtain a groupoid $(U, R, s, t, c, e, i)$ such that
$j = (t, s) : R \to U \times_S U$, see
Groupoids, Lemma \ref{groupoids-lemma-equivalence-groupoid}.

\begin{lemma}
\label{lemma-descent-sheaf}
With $S$, $\varphi : U \to X$, and $(U, R, s, t, c, e, i)$ as above.
For any sheaf $\mathcal{F}$ on $X_\etale$ the
sheaf\footnote{In this lemma
and its proof we write simply $\varphi^{-1}$ instead of $\varphi_{small}^{-1}$
and similarly for all the other pullbacks.}
$\mathcal{G} = \varphi^{-1}\mathcal{F}$ comes equipped with a canonical
isomorphism
$$
\alpha :
t^{-1}\mathcal{G}
\longrightarrow
s^{-1}\mathcal{G}
$$
such that the diagram
$$
\xymatrix{
& \text{pr}_1^{-1}t^{-1}\mathcal{G} \ar[r]_-{\text{pr}_1^{-1}\alpha} &
\text{pr}_1^{-1}s^{-1}\mathcal{G} \ar@{=}[rd] & \\
\text{pr}_0^{-1}s^{-1}\mathcal{G} \ar@{=}[ru] & & &
c^{-1}s^{-1}\mathcal{G} \\
&
\text{pr}_0^{-1}t^{-1}\mathcal{G} \ar[lu]^{\text{pr}_0^{-1}\alpha} \ar@{=}[r] &
c^{-1}t^{-1}\mathcal{G} \ar[ru]_{c^{-1}\alpha}
}
$$
is a commutative. The functor $\mathcal{F} \mapsto (\mathcal{G}, \alpha)$
defines an equivalence of categories between sheaves on
$X_\etale$ and pairs $(\mathcal{G}, \alpha)$ as above.
\end{lemma}

\begin{proof}[First proof of Lemma \ref{lemma-descent-sheaf}]
Let $\mathcal{C} = X_{spaces, \etale}$. By
Lemma \ref{lemma-etale-morphism-topoi}
and its proof we have $U_{spaces, \etale} = \mathcal{C}/U$
and the pullback functor $\varphi^{-1}$ is just the restriction functor.
Moreover, $\{U \to X\}$ is a covering of the site $\mathcal{C}$ and
$R = U \times_X U$. The isomorphism $\alpha$ is just the canonical
identification
$$
\left(\mathcal{F}|_{\mathcal{C}/U}\right)|_{\mathcal{C}/U \times_X U}
=
\left(\mathcal{F}|_{\mathcal{C}/U}\right)|_{\mathcal{C}/U \times_X U}
$$
and the commutativity of the diagram is the cocycle condition for glueing
data. Hence this lemma is a special case of glueing of sheaves, see
Sites, Section \ref{sites-section-glueing-sheaves}.
\end{proof}

\begin{proof}[Second proof of Lemma \ref{lemma-descent-sheaf}]
The existence of $\alpha$ comes from the fact that
$\varphi \circ t = \varphi \circ s$ and that pullback is
functorial in the morphism, see
Lemma \ref{lemma-functoriality-etale-site}.
In exactly the same way, i.e., by functoriality of pullback, we see
that the isomorphism $\alpha$ fits into the commutative diagram.
The construction $\mathcal{F} \mapsto (\varphi^{-1}\mathcal{F}, \alpha)$
is clearly functorial in the sheaf $\mathcal{F}$.
Hence we obtain the functor.

\medskip\noindent
Conversely, suppose that $(\mathcal{G}, \alpha)$ is a pair.
Let $V \to X$ be an object of $X_\etale$.
In this case the morphism $V' = U \times_X V \to V$ is a surjective \'etale
morphism of schemes, and hence $\{V' \to V\}$ is an \'etale
covering of $V$. Set $\mathcal{G}' = (V' \to V)^{-1}\mathcal{G}$.
Since $R = U \times_X U$ with $t = \text{pr}_0$
and $s = \text{pr}_0$ we see that $V' \times_V V' = R \times_X V$
with projection maps $s', t' : V' \times_V V' \to V'$ equal to the pullbacks
of $t$ and $s$. Hence $\alpha$ pulls back to an isomorphism
$\alpha' : (t')^{-1}\mathcal{G}' \to (s')^{-1}\mathcal{G}'$. Having said this
we simply define
$$
\xymatrix{
\mathcal{F}(V) \ar@{=}[r] &
\text{Equalizer}(\mathcal{G}(V') \ar@<1ex>[r] \ar@<-1ex>[r] &
\mathcal{G}(V' \times_V V').
}
$$
We omit the verification that this defines a sheaf. To see that
$\mathcal{G}(V) = \mathcal{F}(V)$ if there exists a morphism $V \to U$
note that in this case the equalizer is
$H^0(\{V' \to V\}, \mathcal{G}) = \mathcal{G}(V)$.
\end{proof}







\section{Points of the small \'etale site}
\label{section-points-small-etale-site}

\noindent
This section is the analogue of
\'Etale Cohomology, Section
\ref{etale-cohomology-section-stalks}.

\begin{definition}
\label{definition-geometric-point}
Let $S$ be a scheme. Let $X$ be an algebraic space over $S$.
\begin{enumerate}
\item A {\it geometric point} of $X$ is a morphism
$\overline{x} : \Spec(k) \to X$, where $k$ is an algebraically
closed field. We often abuse notation and
write $\overline{x} = \Spec(k)$.
\item For every geometric point $\overline{x}$ we have the corresponding
``image'' point $x \in |X|$. We say that $\overline{x}$ is a
{\it geometric point lying over $x$}.
\end{enumerate}
\end{definition}

\noindent
It turns out that we can take stalks of sheaves on $X_\etale$
at geometric points exactly in the same way as was done in the
case of the small \'etale site of a scheme. In order to do this we
define the notion of an \'etale neighbourhood as follows.

\begin{definition}
\label{definition-etale-neighbourhood}
Let $S$ be a scheme. Let $X$ be an algebraic space over $S$.
Let $\overline{x}$ be a geometric point of $X$.
\begin{enumerate}
\item An {\it \'etale neighborhood} of $\overline{x}$
of $X$ is a commutative diagram
$$
\xymatrix{
& U \ar[d]^\varphi \\
{\bar x} \ar[r]^{\bar x} \ar[ur]^{\bar u} & X
}
$$
where $\varphi$ is an \'etale morphism of algebraic spaces over $S$.
We will use the notation $\varphi : (U, \overline{u}) \to (X, \overline{x})$
to indicate this situation.
\item A {\it morphism of \'etale neighborhoods}
$(U, \overline{u}) \to (U', \overline{u}')$
is an $X$-morphism $h : U \to U'$
such that $\overline{u}' = h \circ \overline{u}$.
\end{enumerate}
\end{definition}

\noindent
Note that we allow $U$ to be an algebraic space. When we take stalks
of a sheaf on $X_\etale$ we have to restrict to those $U$ which
are in $X_\etale$, and so in this case we will only consider the case
where $U$ is a scheme. Alternately we can work with the site
$X_{space, \etale}$ and consider all \'etale neighbourhoods. And there
won't be any difference because of the last assertion in the following lemma.

\begin{lemma}
\label{lemma-cofinal-etale}
Let $S$ be a scheme. Let $X$ be an algebraic space over $S$.
Let $\overline{x}$ be a geometric point of $X$.
The category of \'etale neighborhoods is cofiltered. More precisely:
\begin{enumerate}
\item Let $(U_i, \overline{u}_i)_{i = 1, 2}$ be two \'etale neighborhoods of
$\overline{x}$ in $X$. Then there exists a third \'etale neighborhood
$(U, \overline{u})$ and morphisms
$(U, \overline{u}) \to (U_i, \overline{u}_i)$, $i = 1, 2$.
\item Let $h_1, h_2: (U, \overline{u}) \to (U', \overline{u}')$ be two
morphisms between \'etale neighborhoods of $\overline{s}$. Then there exist an
\'etale neighborhood $(U'', \overline{u}'')$ and a morphism
$h : (U'', \overline{u}'') \to (U, \overline{u})$
which equalizes $h_1$ and $h_2$, i.e., such that
$h_1 \circ h = h_2 \circ h$.
\end{enumerate}
Moreover, given any \'etale neighbourhood
$(U, \overline{u}) \to (X, \overline{x})$
there exists a morphism of \'etale neighbourhoods
$(U', \overline{u}') \to (U, \overline{u})$
where $U'$ is a scheme.
\end{lemma}

\begin{proof}
For part (1), consider the fibre product $U = U_1 \times_X U_2$.
It is \'etale over both $U_1$ and $U_2$ because \'etale morphisms are
preserved under base change and composition, see
Lemmas \ref{lemma-base-change-etale} and \ref{lemma-composition-etale}.
The map $\overline{u} \to U$ defined by $(\overline{u}_1, \overline{u}_2)$
gives it the structure of an \'etale neighborhood mapping to both
$U_1$ and $U_2$.

\medskip\noindent
For part (2), define $U''$ as the fibre product
$$
\xymatrix{
U'' \ar[r] \ar[d] & U \ar[d]^{(h_1, h_2)} \\
U' \ar[r]^-\Delta & U' \times_X U'.
}
$$
Since $\overline{u}$ and $\overline{u}'$ agree over $X$ with $\overline{x}$,
we see that $\overline{u}'' = (\overline{u}, \overline{u}')$ is a geometric
point of $U''$. In particular $U'' \not = \emptyset$.
Moreover, since $U'$ is \'etale over $X$, so is the fibre product
$U'\times_X U'$ (as seen above in the case of $U_1 \times_X U_2$).
Hence the vertical arrow $(h_1, h_2)$ is \'etale by
Lemma \ref{lemma-etale-permanence}.
Therefore $U''$ is \'etale over $U'$ by base change, and hence also
\'etale over $X$ (because compositions of \'etale morphisms are \'etale).
Thus $(U'', \overline{u}'')$ is a solution to the problem posed by (2).

\medskip\noindent
To see the final assertion, choose any surjective \'etale morphism
$U' \to U$ where $U'$ is a scheme. Then
$U' \times_U \overline{u}$ is a scheme surjective and \'etale over
$\overline{u} = \Spec(k)$ with $k$ algebraically closed.
It follows (see
Morphisms, Lemma \ref{morphisms-lemma-etale-over-field})
that $U' \times_U \overline{u} \to \overline{u}$ has a section
which gives us the desired $\overline{u}'$.
\end{proof}

\begin{lemma}
\label{lemma-geometric-lift-to-usual}
Let $S$ be a scheme. Let $X$ be an algebraic space over $S$.
Let $\overline{x} : \Spec(k) \to X$ be a geometric point of $X$
lying over $x \in |X|$. Let $\varphi : U \to X$ be an \'etale morphism
of algebraic spaces and let $u \in |U|$ with $\varphi(u) = x$.
Then there exists a geometric point
$\overline{u} : \Spec(k) \to U$ lying over $u$ with
$\overline{x} = \varphi \circ \overline{u}$.
\end{lemma}

\begin{proof}
Choose an affine scheme $U'$ with $u' \in U'$ and an \'etale morphism
$U' \to U$ which maps $u'$ to $u$. If we can prove the lemma for
$(U', u') \to (X, x)$ then the lemma follows. Hence we may assume that
$U$ is a scheme, in particular that $U \to X$ is representable.
Then look at the cartesian diagram
$$
\xymatrix{
\Spec(k) \times_{\overline{x}, X, \varphi} U
\ar[d]_{\text{pr}_1} \ar[r]_-{\text{pr}_2} & U
\ar[d]^\varphi \\
\Spec(k) \ar[r]^-{\overline{x}} & X
}
$$
The projection $\text{pr}_1$ is the base change of an \'etale morphisms
so it is \'etale, see
Lemma \ref{lemma-base-change-etale}.
Therefore, the scheme $\Spec(k) \times_{\overline{x}, X, \varphi} U$
is a disjoint union of finite separable extensions of $k$, see
Morphisms, Lemma \ref{morphisms-lemma-etale-over-field}.
But $k$ is algebraically closed, so all these extensions are trivial,
so $\Spec(k) \times_{\overline{x}, X, \varphi} U$
is a disjoint union of copies of $\Spec(k)$ and each of
these corresponds to a geometric point $\overline{u}$ with
$\varphi \circ \overline{u} = \overline{x}$. By
Lemma \ref{lemma-points-cartesian}
the map
$$
|\Spec(k) \times_{\overline{x}, X, \varphi} U|
\longrightarrow
|\Spec(k)| \times_{|X|} |U|
$$
is surjective, hence we can pick $\overline{u}$ to lie over $u$.
\end{proof}

\begin{lemma}
\label{lemma-geometric-lift-to-cover}
Let $S$ be a scheme. Let $X$ be an algebraic space over $S$.
Let $\overline{x}$ be a geometric point of $X$.
Let $(U, \overline{u})$ an \'etale neighborhood of $\overline{x}$.
Let $\{\varphi_i : U_i \to U\}_{i \in I}$ be an \'etale covering in
$X_{spaces, \etale}$.
Then there exist $i \in I$ and $\overline{u}_i : \overline{x} \to U_i$
such that $\varphi_i : (U_i, \overline{u}_i) \to (U, \overline{u})$
is a morphism of \'etale neighborhoods.
\end{lemma}

\begin{proof}
Let $u \in |U|$ be the image of $\overline{u}$.
As $|U| = \bigcup_{i \in I} \varphi_i(|U_i|)$ there exists an
$i$ and a point $u_i \in U_i$ mapping to $x$. Apply
Lemma \ref{lemma-geometric-lift-to-usual}
to $(U_i, u_i) \to (U, u)$ and $\overline{u}$ to
get the desired geometric point.
\end{proof}

\begin{definition}
\label{definition-stalk}
Let $S$ be a scheme. Let $X$ be an algebraic space over $S$.
Let $\mathcal{F}$ be a presheaf on $X_\etale$.
Let $\overline{x}$ be a geometric point of $X$.
The {\it stalk} of $\mathcal{F}$ at $\overline{x}$ is
$$
\mathcal{F}_{\bar x}
=
\colim_{(U, \overline{u})} \mathcal{F}(U)
$$
where $(U, \overline{u})$ runs over all \'etale neighborhoods
of $\overline{x}$ in $X$ with $U \in \Ob(X_\etale)$.
\end{definition}

\noindent
By
Lemma \ref{lemma-cofinal-etale},
this colimit is over a filtered
index category, namely the opposite of the category of \'etale neighborhoods
in $X_\etale$. More precisely
Lemma \ref{lemma-cofinal-etale}
says the opposite of the category of all \'etale neighbourhoods is filtered,
and the full subcategory of those which are in $X_\etale$ is a cofinal
subcategory hence also filtered.

\medskip\noindent
This means an element of $\mathcal{F}_{\overline{x}}$ can be
thought of as a triple $(U, \overline{u}, \sigma)$ where
$U \in \Ob(X_\etale)$ and $\sigma \in \mathcal{F}(U)$.
Two triples $(U, \overline{u}, \sigma)$, $(U', \overline{u}', \sigma')$
define the same element of the stalk if there exists a third
\'etale neighbourhood
$(U'', \overline{u}'')$, $U'' \in \Ob(X_\etale)$
and morphisms of \'etale neighbourhoods
$h : (U'', \overline{u}'') \to (U, \overline{u})$,
$h' : (U'', \overline{u}'') \to (U', \overline{u}')$ such that
$h^*\sigma = (h')^*\sigma'$ in $\mathcal{F}(U'')$. See
Categories, Section \ref{categories-section-directed-colimits}.

\medskip\noindent
This also implies that if $\mathcal{F}'$ is the sheaf on
$X_{spaces, \etale}$ corresponding to $\mathcal{F}$ on
$X_\etale$, then
\begin{equation}
\label{equation-stalk-spaces-etale}
\mathcal{F}_{\overline{x}} = \colim_{(U, \overline{u})} \mathcal{F}'(U)
\end{equation}
where now the colimit is over all the \'etale neighbourhoods of $\overline{x}$.
We will often jump between the point of view of using $X_\etale$
and $X_{spaces, \etale}$ without further mention.

\medskip\noindent
In particular this means that if $\mathcal{F}$ is a presheaf of
abelian groups, rings, etc then $\mathcal{F}_{\overline{x}}$ is
an abelian group, ring, etc simply by the usual way of defining the
group structure on a directed colimit of abelian groups, rings, etc.

\begin{lemma}
\label{lemma-stalk-gives-point}
\begin{slogan}
A geometric point of an algebraic space gives a point of its \'etale topos.
\end{slogan}
Let $S$ be a scheme.
Let $X$ be an algebraic space over $S$.
Let $\overline{x}$ be a geometric point of $X$.
Consider the functor
$$
u : X_\etale \longrightarrow \textit{Sets}, \quad
U \longmapsto |U_{\overline{x}}|
$$
Then $u$ defines a point $p$ of the site $X_\etale$
(Sites, Definition \ref{sites-definition-point})
and its associated stalk functor $\mathcal{F} \mapsto \mathcal{F}_p$
(Sites, Equation \ref{sites-equation-stalk})
is the functor $\mathcal{F} \mapsto \mathcal{F}_{\overline{x}}$
defined above.
\end{lemma}

\begin{proof}
In the proof of
Lemma \ref{lemma-geometric-lift-to-usual}
we have seen that the scheme $U_{\overline{x}} = \overline{x} \times_X U$
is a disjoint union of
schemes isomorphic to $\overline{x}$. Thus we can also think of
$|U_{\overline{x}}|$ as the set of geometric points of $U$ lying over
$\overline{x}$, i.e., as the collection of morphisms
$\overline{u} : \overline{x} \to U$ fitting into the diagram of
Definition \ref{definition-etale-neighbourhood}.
From this it follows that $u(X)$ is a singleton, and that
$u(U \times_V W) = u(U) \times_{u(V)} u(W)$
whenever $U \to V$ and $W \to V$ are morphisms in $X_\etale$.
And, given a covering $\{U_i \to U\}_{i \in I}$ in $X_\etale$ we see
that $\coprod u(U_i) \to u(U)$ is surjective by
Lemma \ref{lemma-geometric-lift-to-cover}.
Hence
Sites, Proposition \ref{sites-proposition-point-limits}
applies, so $p$ is a point of the site $X_\etale$.
Finally, the our functor $\mathcal{F} \mapsto \mathcal{F}_{\overline{s}}$
is given by exactly the same colimit as the functor
$\mathcal{F} \mapsto \mathcal{F}_p$ associated to $p$ in
Sites, Equation \ref{sites-equation-stalk}
which proves the final assertion.
\end{proof}

\begin{lemma}
\label{lemma-stalk-exact}
Let $S$ be a scheme.
Let $X$ be an algebraic space over $S$.
Let $\overline{x}$ be a geometric point of $X$.
\begin{enumerate}
\item The stalk functor
$\textit{PAb}(X_\etale) \to \textit{Ab}$,
$\mathcal{F}  \mapsto  \mathcal{F}_{\overline{x}}$
is exact.
\item We have $(\mathcal{F}^\#)_{\overline{x}} = \mathcal{F}_{\overline{x}}$
for any presheaf of sets $\mathcal{F}$ on $X_\etale$.
\item The functor
$\textit{Ab}(X_\etale) \to \textit{Ab}$,
$\mathcal{F} \mapsto \mathcal{F}_{\overline{x}}$ is exact.
\item Similarly the functors
$\textit{PSh}(X_\etale) \to \textit{Sets}$ and
$\Sh(X_\etale) \to \textit{Sets}$ given by the stalk functor
$\mathcal{F} \mapsto \mathcal{F}_{\overline{x}}$ are exact (see
Categories, Definition \ref{categories-definition-exact})
and commute with arbitrary colimits.
\end{enumerate}
\end{lemma}

\begin{proof}
This result follows from the general material in
Modules on Sites, Section \ref{sites-modules-section-stalks}.
This is true because $\mathcal{F} \mapsto \mathcal{F}_{\overline{x}}$
comes from a point of the small \'etale site of $X$, see
Lemma \ref{lemma-stalk-gives-point}. See the proof of
\'Etale Cohomology, Lemma \ref{etale-cohomology-lemma-stalk-exact}
for a direct proof of some of these statements in the setting of
the small \'etale site of a scheme.
\end{proof}

\noindent
We will see below that the stalk functor
$\mathcal{F} \mapsto \mathcal{F}_{\overline{x}}$
is really the pullback along the morphism $\overline{x}$. In that sense
the following lemma is a generalization of the lemma above.

\begin{lemma}
\label{lemma-stalk-pullback}
Let $S$ be a scheme.
Let $f : X \to Y$ be a morphism of algebraic spaces over $S$.
\begin{enumerate}
\item The functor
$f_{small}^{-1} :
\textit{Ab}(Y_\etale)
\to
\textit{Ab}(X_\etale)$
is exact.
\item The functor
$f_{small}^{-1} :
\Sh(Y_\etale)
\to
\Sh(X_\etale)$
is exact, i.e., it commutes with finite limits and colimits, see
Categories, Definition \ref{categories-definition-exact}.
\item For any \'etale morphism $V \to Y$ of algebraic spaces
we have $f_{small}^{-1}h_V = h_{X \times_Y V}$.
\item Let $\overline{x} \to X$ be a geometric point.
Let $\mathcal{G}$ be a sheaf on $Y_\etale$.
Then there is a canonical identification
$$
(f_{small}^{-1}\mathcal{G})_{\overline{x}} = \mathcal{G}_{\overline{y}}.
$$
where $\overline{y} = f \circ \overline{x}$.
\end{enumerate}
\end{lemma}

\begin{proof}
Recall that $f_{small}$ is defined via $f_{spaces, small}$ in
Lemma \ref{lemma-functoriality-etale-site}.
Parts (1), (2) and (3) are general consequences of the fact that
$f_{spaces, \etale} :
X_{spaces, \etale}
\to
Y_{spaces, \etale}$
is a morphism of sites, see
Sites, Definition \ref{sites-definition-morphism-sites}
for (2),
Modules on Sites, Lemma \ref{sites-modules-lemma-flat-pullback-exact}
for (1), and
Sites, Lemma \ref{sites-lemma-pullback-representable-sheaf}
for (3).

\medskip\noindent
Proof of (4). This statement is a special case of
Sites, Lemma \ref{sites-lemma-point-morphism-sites}
via
Lemma \ref{lemma-stalk-gives-point}.
We also provide a direct proof. Note that by
Lemma \ref{lemma-stalk-exact}.
taking stalks commutes with sheafification.
Let $\mathcal{G}'$ be the sheaf on $Y_{spaces, \etale}$ whose restriction
to $Y_\etale$ is $\mathcal{G}$.
Recall that $f_{spaces, \etale}^{-1}\mathcal{G}'$ is the sheaf
associated to the presheaf
$$
U \longrightarrow \colim_{U \to X \times_Y V} \mathcal{G}'(V),
$$
see
Sites, Sections \ref{sites-section-continuous-functors} and
\ref{sites-section-functoriality-PSh}.
Thus we have
\begin{align*}
(f_{spaces, \etale}^{-1}\mathcal{G}')_{\overline{x}}
& =
\colim_{(U, \overline{u})} f_{spaces, \etale}^{-1}\mathcal{G}'(U)
\\
& = \colim_{(U, \overline{u})}
\colim_{a : U \to X \times_Y V} \mathcal{G}'(V) \\
& = \colim_{(V, \overline{v})} \mathcal{G}'(V) \\
& = \mathcal{G}'_{\overline{y}}
\end{align*}
in the third equality the pair $(U, \overline{u})$ and the map
$a : U \to X \times_Y V$ corresponds to the pair $(V, a \circ \overline{u})$.
Since the stalk of $\mathcal{G}'$
(resp.\ $f_{spaces, \etale}^{-1}\mathcal{G}'$)
agrees with the stalk of $\mathcal{G}$ (resp.\ $f_{small}^{-1}\mathcal{G}$),
see
Equation (\ref{equation-stalk-spaces-etale})
the result follows.
\end{proof}

\begin{remark}
\label{remark-stalk-pullback}
This remark is the analogue of
\'Etale Cohomology, Remark \ref{etale-cohomology-remark-stalk-pullback}.
Let $S$ be a scheme.
Let $X$ be an algebraic space over $S$.
Let $\overline{x} : \Spec(k) \to X$ be a geometric point of $X$.
By
\'Etale Cohomology,
Theorem \ref{etale-cohomology-theorem-equivalence-sheaves-point}
the category of sheaves on $\Spec(k)_\etale$ is
equivalent to the category of sets (by taking a sheaf to its global sections).
Hence it follows from
Lemma \ref{lemma-stalk-pullback} part (4)
applied to the morphism $\overline{x}$ that the functor
$$
\Sh(X_\etale) \longrightarrow \textit{Sets}, \quad
\mathcal{F} \longmapsto \mathcal{F}_{\overline{x}}
$$
is isomorphic to the functor
$$
\Sh(X_\etale)
\longrightarrow
\Sh(\Spec(k)_\etale) = \textit{Sets},
\quad
\mathcal{F} \longmapsto \overline{x}^*\mathcal{F}
$$
Hence we may view the stalk functors as pullback functors along
geometric morphisms (and not just some abstract morphisms of topoi
as in the result of
Lemma \ref{lemma-stalk-gives-point}).
\end{remark}

\begin{remark}
\label{remark-map-stalks}
Let $S$ be a scheme.
Let $X$ be an algebraic space over $S$.
Let $x \in |X|$.
We claim that for any pair of geometric points $\overline{x}$ and
$\overline{x}'$ lying over $x$ the stalk functors are isomorphic.
By definition of $|X|$ we can find a third geometric point
$\overline{x}''$ so that there exists a commutative diagram
$$
\xymatrix{
\overline{x}'' \ar[r] \ar[d] \ar[rd]^{\overline{x}''} &
\overline{x}' \ar[d]^{\overline{x}'} \\
\overline{x} \ar[r]^{\overline{x}} & X.
}
$$
Since the stalk functor $\mathcal{F} \mapsto \mathcal{F}_{\overline{x}}$
is given by pullback along the morphism $\overline{x}$ (and similarly for
the others) we conclude by functoriality of pullbacks.
\end{remark}




\noindent
The following theorem says that the small \'etale site of an algebraic
space has enough points.

\begin{theorem}
\label{theorem-exactness-stalks}
Let $S$ be a scheme. Let $X$ be an algebraic space over $S$.
A map $a : \mathcal{F} \to \mathcal{G}$ of sheaves of sets is injective
(resp.\ surjective) if and only if the map on stalks
$a_{\overline{x}} : \mathcal{F}_{\overline{x}} \to \mathcal{G}_{\overline{x}}$
is injective (resp.\ surjective) for all geometric points of $X$.
A sequence of abelian sheaves on $X_\etale$ is exact
if and only if it is exact on all stalks at geometric points of $S$.
\end{theorem}

\begin{proof}
We know the theorem is true if $X$ is a scheme, see
\'Etale Cohomology, Theorem \ref{etale-cohomology-theorem-exactness-stalks}.
Choose a surjective \'etale morphism $f : U \to X$ where $U$ is a scheme.
Since $\{U \to X\}$ is a covering (in $X_{spaces, \etale}$) we can check
whether a map of sheaves is injective, or surjective by restricting
to $U$. Now if $\overline{u} : \Spec(k) \to U$ is a geometric
point of $U$, then
$(\mathcal{F}|_U)_{\overline{u}} = \mathcal{F}_{\overline{x}}$
where $\overline{x} = f \circ \overline{u}$. (This is clear from the
colimits defining the stalks at $\overline{u}$ and $\overline{x}$, but
it also follows from
Lemma \ref{lemma-stalk-pullback}.)
Hence the result for $U$ implies the result for $X$ and we win.
\end{proof}

\noindent
The following lemma should be skipped on a first reading.

\begin{lemma}
\label{lemma-points-small-etale-site}
Let $S$ be a scheme.
Let $X$ be an algebraic space over $S$.
Let $p : \Sh(pt) \to \Sh(X_\etale)$
be a point of the small \'etale topos of $X$.
Then there exists a geometric point $\overline{x}$ of $X$
such that the stalk functor $\mathcal{F} \mapsto \mathcal{F}_p$
is isomorphic to the stalk functor
$\mathcal{F} \mapsto \mathcal{F}_{\overline{x}}$.
\end{lemma}

\begin{proof}
By
Sites, Lemma \ref{sites-lemma-point-site-topos}
there is a one to one correspondence between points of the site and points
of the associated topos. Hence we may assume that $p$ is given by
a functor $u : X_\etale \to \textit{Sets}$ which defines a point
of the site $X_\etale$.
Let $U \in \Ob(X_\etale)$ be an object whose structure morphism
$j : U \to X$ is surjective. Note that $h_U$ is a sheaf
which surjects onto the final sheaf. Since taking stalks is exact
we see that $(h_U)_p = u(U)$ is not empty (use
Sites, Lemma \ref{sites-lemma-points-recover}).
Pick $x \in u(U)$. By
Sites, Lemma \ref{sites-lemma-point-localize}
we obtain a point $q : \Sh(pt) \to \Sh(U_\etale)$
such that $p = j_{small} \circ q$, so that
$\mathcal{F}_p = (\mathcal{F}|_U)_q$ functorially.
By
\'Etale Cohomology, Lemma \ref{etale-cohomology-lemma-points-small-etale-site}
there is a geometric point $\overline{u}$ of $U$ and a functorial
isomorphism $\mathcal{G}_q = \mathcal{G}_{\overline{u}}$
for $\mathcal{G} \in \Sh(U_\etale)$. Set
$\overline{x} = j \circ \overline{u}$. Then we see that
$\mathcal{F}_{\overline{x}} \cong (\mathcal{F}|_U)_{\overline{u}}$
functorially in $\mathcal{F}$ on $X_\etale$ by
Lemma \ref{lemma-stalk-pullback}
and we win.
\end{proof}





\section{Supports of abelian sheaves}
\label{section-support}

\noindent
First we talk about supports of local sections.

\begin{lemma}
\label{lemma-support-subsheaf-final}
Let $S$ be a scheme. Let $X$ be an algebraic space over $S$.
Let $\mathcal{F}$ be a subsheaf of the final object of the \'etale
topos of $X$ (see
Sites, Example \ref{sites-example-singleton-sheaf}).
Then there exists a unique open
$W \subset X$ such that $\mathcal{F} = h_W$.
\end{lemma}

\begin{proof}
The condition means that $\mathcal{F}(U)$ is a singleton or
empty for all $\varphi : U \to X$ in $\Ob(X_{spaces, \etale})$.
In particular local sections always glue. If
$\mathcal{F}(U) \not = \emptyset$, then
$\mathcal{F}(\varphi(U)) \not = \emptyset$ because
$\varphi(U) \subset X$ is an open subspace
(Lemma \ref{lemma-etale-open})
and
$\{\varphi : U \to \varphi(U)\}$ is a covering in $X_{spaces, \etale}$.
Take
$W = \bigcup_{\varphi : U \to S, \mathcal{F}(U) \not = \emptyset} \varphi(U)$
to conclude.
\end{proof}

\begin{lemma}
\label{lemma-zero-over-image}
Let $S$ be a scheme.
Let $X$ be an algebraic space over $S$.
Let $\mathcal{F}$ be an abelian sheaf on $X_{spaces, \etale}$.
Let $\sigma \in \mathcal{F}(U)$ be a local section.
There exists an open subspace $W \subset U$ such that
\begin{enumerate}
\item $W \subset U$ is the largest open subspace of $U$ such
that $\sigma|_W = 0$,
\item for every $\varphi : V \to U$ in $X_{spaces, \etale}$ we have
$$
\sigma|_V = 0 \Leftrightarrow \varphi(V) \subset W,
$$
\item for every geometric point $\overline{u}$ of $U$ we have
$$
(U, \overline{u}, \sigma) = 0\text{ in }\mathcal{F}_{\overline{x}}
\Leftrightarrow
\overline{u} \in W
$$
where $\overline{x} = (U \to X) \circ \overline{u}$.
\end{enumerate}
\end{lemma}

\begin{proof}
Since $\mathcal{F}$ is a sheaf in the \'etale topology the restriction of
$\mathcal{F}$ to $U_{Zar}$ is a sheaf on $U$ in the Zariski topology.
Hence there exists a Zariski open $W$ having property (1), see
Modules, Lemma \ref{modules-lemma-support-section-closed}. Let
$\varphi : V \to U$ be an arrow of $X_{spaces, \etale}$. Note that
$\varphi(V) \subset U$ is an open subspace
(Lemma \ref{lemma-etale-open})
and that $\{V \to \varphi(V)\}$ is an \'etale covering. Hence if
$\sigma|_V = 0$, then by the sheaf condition for $\mathcal{F}$ we
see that $\sigma|_{\varphi(V)} = 0$. This proves (2).
To prove (3) we have to show that if $(U, \overline{u}, \sigma)$
defines the zero element of $\mathcal{F}_{\overline{x}}$, then
$\overline{u} \in W$. This is true because the assumption means
there exists a morphism of \'etale neighbourhoods
$(V, \overline{v}) \to (U, \overline{u})$ such that
$\sigma|_V = 0$. Hence by (2) we see that $V \to U$ maps into $W$, and
hence $\overline{u} \in W$.
\end{proof}

\noindent
Let $S$ be a scheme.
Let $X$ be an algebraic space over $S$.
Let $x \in |X|$.
Let $\mathcal{F}$ be a sheaf on $X_\etale$. By
Remark \ref{remark-map-stalks}
the isomorphism class of the stalk of the sheaf $\mathcal{F}$
at a geometric points lying over $x$ is well defined.

\begin{definition}
\label{definition-support}
Let $S$ be a scheme.
Let $X$ be an algebraic space over $S$.
Let $\mathcal{F}$ be an abelian sheaf on $X_\etale$.
\begin{enumerate}
\item The {\it support of $\mathcal{F}$} is the set of
points $x \in |X|$ such that $\mathcal{F}_{\overline{x}} \not = 0$
for any (some) geometric point $\overline{x}$ lying over $x$.
\item Let $\sigma \in \mathcal{F}(U)$ be a section.
The {\it support of $\sigma$} is the closed subset $U \setminus W$, where
$W \subset U$ is the largest open subset of $U$ on which $\sigma$
restricts to zero (see
Lemma \ref{lemma-zero-over-image}).
\end{enumerate}
\end{definition}

\begin{lemma}
\label{lemma-support-section-closed}
Let $S$ be a scheme.
Let $X$ be an algebraic space over $S$.
Let $\mathcal{F}$ be an abelian sheaf on $X_\etale$.
Let $U \in \Ob(X_\etale)$ and $\sigma \in \mathcal{F}(U)$.
\begin{enumerate}
\item The support of $\sigma$ is closed in $|X|$.
\item The support of $\sigma + \sigma'$ is contained in the union of
the supports of $\sigma, \sigma' \in \mathcal{F}(X)$.
\item If $\varphi : \mathcal{F} \to \mathcal{G}$ is a map of
abelian sheaves on $X_\etale$, then the support of $\varphi(\sigma)$ is
contained in the support of $\sigma \in \mathcal{F}(U)$.
\item The support of $\mathcal{F}$ is the union of the images of the
supports of all local sections of $\mathcal{F}$.
\item If $\mathcal{F} \to \mathcal{G}$ is surjective then the support
of $\mathcal{G}$ is a subset of the support of $\mathcal{F}$.
\item If $\mathcal{F} \to \mathcal{G}$ is injective then the support
of $\mathcal{F}$ is a subset of the support of $\mathcal{G}$.
\end{enumerate}
\end{lemma}

\begin{proof}
Part (1) holds by definition.
Parts (2) and (3) hold because they holds for the restriction of
$\mathcal{F}$ and $\mathcal{G}$ to $U_{Zar}$, see
Modules, Lemma \ref{modules-lemma-support-section-closed}.
Part (4) is a direct consequence of
Lemma \ref{lemma-zero-over-image} part (3).
Parts (5) and (6) follow from the other parts.
\end{proof}

\begin{lemma}
\label{lemma-support-sheaf-rings-closed}
The support of a sheaf of rings on the small \'etale site of an
algebraic space is closed.
\end{lemma}

\begin{proof}
This is true because (according to our conventions)
a ring is $0$ if and only if
$1 = 0$, and hence the support of a sheaf of rings
is the support of the unit section.
\end{proof}






\section{The structure sheaf of an algebraic space}
\label{section-structure-sheaf}

\noindent
The structure sheaf of an algebraic space is the sheaf of rings of the
following lemma.

\begin{lemma}
\label{lemma-sheaf-condition-holds}
Let $S$ be a scheme. Let $X$ be an algebraic space over $S$.
The rule $U \mapsto \Gamma(U, \mathcal{O}_U)$ defines
a sheaf of rings on $X_\etale$.
\end{lemma}

\begin{proof}
Immediate from the definition of a covering and
Descent, Lemma \ref{descent-lemma-sheaf-condition-holds}.
\end{proof}

\begin{definition}
\label{definition-structure-sheaf}
Let $S$ be a scheme.
Let $X$ be an algebraic space over $S$.
The {\it structure sheaf} of $X$
is the sheaf of rings $\mathcal{O}_X$
on the small \'etale site $X_\etale$ described in
Lemma \ref{lemma-sheaf-condition-holds}.
\end{definition}

\noindent
According to Lemma \ref{lemma-characterize-sheaf-small-etale} the sheaf
$\mathcal{O}_X$ corresponds to a system of \'etale sheaves $(\mathcal{O}_X)_U$
for $U$ ranging through the objects of $X_\etale$. It is clear from
the proof of that lemma and our definition that we have simply
$(\mathcal{O}_X)_U = \mathcal{O}_U$ where $\mathcal{O}_U$ is the structure
sheaf of $U_\etale$ as introduced in
Descent, Definition \ref{descent-definition-structure-sheaf}.
In particular, if $X$ is a scheme we recover the sheaf $\mathcal{O}_X$
on the small \'etale site of $X$.

\medskip\noindent
Via the equivalence
$\Sh(X_\etale) = \Sh(X_{spaces, \etale})$
of Lemma \ref{lemma-compare-etale-sites} we may also think of $\mathcal{O}_X$
as a sheaf of rings on $X_{spaces, \etale}$. It is explained in
Remark \ref{remark-explain-equivalence}
how to compute $\mathcal{O}_X(Y)$, and in particular $\mathcal{O}_X(X)$, when
$Y \to X$ is an object of $X_{spaces, \etale}$.

\begin{lemma}
\label{lemma-morphism-ringed-topoi}
Let $S$ be a scheme.
Let $f : X \to Y$ be a morphism of algebraic spaces over $S$.
Then there is a canonical map
$f^\sharp : f_{small}^{-1}\mathcal{O}_Y \to \mathcal{O}_X$ such that
$$
(f_{small}, f^\sharp) :
(\Sh(X_\etale), \mathcal{O}_X)
\longrightarrow
(\Sh(Y_\etale), \mathcal{O}_Y)
$$
is a morphism of ringed topoi. Furthermore,
\begin{enumerate}
\item The construction $f \mapsto (f_{small}, f^\sharp)$ is compatible with
compositions.
\item If $f$ is a morphism of schemes, then $f^\sharp$ is the map described in
Descent, Remark \ref{descent-remark-change-topologies-ringed}.
\end{enumerate}
\end{lemma}

\begin{proof}
By Lemma \ref{lemma-f-map} it suffices to give an $f$-map from
$\mathcal{O}_Y$ to $\mathcal{O}_X$. In other words, for every
commutative diagram
$$
\xymatrix{
U \ar[d]_g \ar[r] & X \ar[d]^f \\
V \ar[r] & Y
}
$$
where $U \in X_\etale$, $V \in Y_\etale$ we have to give a
map of rings
$
(f^\sharp)_{(U, V, g)} :
\Gamma(V, \mathcal{O}_V)
\to
\Gamma(U, \mathcal{O}_U).
$
Of course we just take $(f^\sharp)_{(U, V, g)} = g^\sharp$.
It is clear that this is compatible with restriction mappings
and hence indeed gives an $f$-map.
We omit checking compatibility with compositions and agreement with the
construction in
Descent, Remark \ref{descent-remark-change-topologies-ringed}.
\end{proof}

\begin{lemma}
\label{lemma-reduced-space}
Let $S$ be a scheme. Let $X$ be an algebraic space over $S$.
The following are equivalent
\begin{enumerate}
\item $X$ is reduced,
\item for every $x \in |X|$ the local ring of $X$ at $x$ is
reduced (Remark \ref{remark-list-properties-local-ring-local-etale-topology}).
\end{enumerate}
In this case $\Gamma(X, \mathcal{O}_X)$ is a reduced ring and
if $f \in \Gamma(X, \mathcal{O}_X)$ has $X = V(f)$, then $f = 0$.
\end{lemma}

\begin{proof}
The equivalence of (1) and (2) follows from
Properties, Lemma \ref{properties-lemma-characterize-reduced}
applied to affine schemes \'etale over $X$. The final statements
follow the cited lemma and fact that $\Gamma(X, \mathcal{O}_X)$ is
a subring of $\Gamma(U, \mathcal{O}_U)$ for some
reduced scheme $U$ \'etale over $X$.
\end{proof}







\section{Stalks of the structure sheaf}
\label{section-stalks-structure-sheaf}

\noindent
This section is the analogue of
\'Etale Cohomology, Section
\ref{etale-cohomology-section-stalks-structure-sheaf}.

\begin{lemma}
\label{lemma-describe-etale-local-ring}
Let $S$ be a scheme.
Let $X$ be an algebraic space over $S$.
Let $\overline{x}$ be a geometric point of $X$.
Let $(U, \overline{u})$ be an \'etale neighbourhood of $\overline{x}$
where $U$ is a scheme. Then we have
$$
\mathcal{O}_{X, \overline{x}} =
\mathcal{O}_{U, \overline{u}} =
\mathcal{O}_{U, u}^{sh}
$$
where the left hand side is the stalk of the structure sheaf of $X$,
and the right hand side is the strict henselization of the local ring
of $U$ at the point $u$ at which $\overline{u}$ is centered.
\end{lemma}

\begin{proof}
We know that the structure sheaf $\mathcal{O}_U$ on
$U_\etale$ is the restriction of the structure sheaf of $X$.
Hence the first equality follows from
Lemma \ref{lemma-stalk-pullback} part (4).
The second equality is explained in
\'Etale Cohomology,
Lemma \ref{etale-cohomology-lemma-describe-etale-local-ring}.
\end{proof}

\begin{definition}
\label{definition-etale-local-rings}
Let $S$ be a scheme.
Let $X$ be an algebraic space over $S$.
Let $\overline{x}$ be a geometric point of $X$ lying over the point
$x \in |X|$.
\begin{enumerate}
\item The {\it \'etale local ring of $X$ at $\overline{x}$}
is the stalk of the structure sheaf $\mathcal{O}_X$ on $X_\etale$
at $\overline{x}$.
Notation: $\mathcal{O}_{X, \overline{x}}$.
\item The {\it strict henselization of $X$ at $\overline{x}$}
is the scheme $\Spec(\mathcal{O}_{X, \overline{x}})$.
\end{enumerate}
\end{definition}

\noindent
The isomorphism type of the strict henselization of $X$ at $\overline{x}$
(as a scheme over $X$) depends only on the point $x \in |X|$ and not on
the choice of the geometric point lying over $x$, see
Remark \ref{remark-map-stalks}.

\begin{lemma}
\label{lemma-etale-site-locally-ringed}
Let $S$ be a scheme.
Let $X$ be an algebraic space over $S$.
The small \'etale site $X_\etale$ endowed with its
structure sheaf $\mathcal{O}_X$ is a locally ringed site, see
Modules on Sites, Definition \ref{sites-modules-definition-locally-ringed}.
\end{lemma}

\begin{proof}
This follows because the stalks
$\mathcal{O}_{X, \overline{x}}$ are
local, and because $S_\etale$ has enough points, see
Lemmas \ref{lemma-describe-etale-local-ring} and
Theorem \ref{theorem-exactness-stalks}.
See
Modules on Sites, Lemma \ref{sites-modules-lemma-locally-ringed-stalk} and
\ref{sites-modules-lemma-ringed-stalk-not-zero}
for the fact that this implies the small \'etale site is locally ringed.
\end{proof}

\begin{lemma}
\label{lemma-dimension-local-ring}
Let $S$ be a scheme. Let $X$ be an algebraic space over $S$.
Let $x \in |X|$ be a point. Let $d \in \{0, 1, 2, \ldots, \infty\}$.
The following are equivalent
\begin{enumerate}
\item the dimension of the local ring of $X$ at $x$
(Definition \ref{definition-dimension-local-ring}) is $d$,
\item $\dim(\mathcal{O}_{X, \overline{x}}) = d$ for some geometric
point $\overline{x}$ lying over $x$, and
\item $\dim(\mathcal{O}_{X, \overline{x}}) = d$ for any geometric
point $\overline{x}$ lying over $x$.
\end{enumerate}
\end{lemma}

\begin{proof}
The equivalence of (2) and (3) follows from the fact that the
isomorphism type of $\mathcal{O}_{X, \overline{x}}$ only depends
on $x \in |X|$, see Remark \ref{remark-map-stalks}.
Using Lemma \ref{lemma-describe-etale-local-ring}
the equivalence of (1) and (2)$+$(3) comes down to the
following statement: Given any local ring $R$ we have
$\dim(R) = \dim(R^{sh})$. This is
More on Algebra, Lemma \ref{more-algebra-lemma-henselization-dimension}.
\end{proof}

\begin{lemma}
\label{lemma-dimension-decent-invariant-under-etale}
Let $S$ be a scheme. Let $f : X \to Y$ be an \'etale morphism
of algebraic spaces over $S$. Let $x \in X$. Then
(1) $\dim_x(X) = \dim_{f(x)}(Y)$ and (2) the dimension of
the local ring of $X$ at $x$ equals the dimension of
the local ring of $Y$ at $f(x)$. If $f$ is surjective, then
(3) $\dim(X) = \dim(Y)$.
\end{lemma}

\begin{proof}
Choose a scheme $U$ and a point $u \in U$ and an \'etale morphism
$U \to X$ which maps $u$ to $x$. Then the composition $U \to Y$
is also \'etale and maps $u$ to $f(x)$. Thus the statements (1) and (2)
follow as the relevant integers are defined in terms of the behaviour
of the scheme $U$ at $u$. See
Definition \ref{definition-dimension-at-point} for (1). Part (3) is
an immediate consequence of (1), see Definition \ref{definition-dimension}.
\end{proof}

\begin{lemma}
\label{lemma-reduced-local-ring}
Let $S$ be a scheme. Let $X$ be an algebraic space over $S$.
Let $x \in |X|$ be a point. The following are equivalent
\begin{enumerate}
\item the local ring of $X$ at $x$ is reduced
(Remark \ref{remark-list-properties-local-ring-local-etale-topology}),
\item $\mathcal{O}_{X, \overline{x}}$ is reduced for some geometric
point $\overline{x}$ lying over $x$, and
\item $\mathcal{O}_{X, \overline{x}}$ is reduced for any geometric
point $\overline{x}$ lying over $x$.
\end{enumerate}
\end{lemma}

\begin{proof}
The equivalence of (2) and (3) follows from the fact that the
isomorphism type of $\mathcal{O}_{X, \overline{x}}$ only depends
on $x \in |X|$, see Remark \ref{remark-map-stalks}.
Using Lemma \ref{lemma-describe-etale-local-ring}
the equivalence of (1) and (2)$+$(3) comes down to the
following statement: a local ring is reduced if and only if
its strict henselization is reduced. This is
More on Algebra, Lemma \ref{more-algebra-lemma-henselization-reduced}.
\end{proof}











\section{Local irreducibility}
\label{section-irreducible-local-ring}

\noindent
A point on an algebraic space has a well defined \'etale local ring, which
corresponds to the strict henselization of the local ring in the case of a
scheme. In general we cannot see how many irreducible components of a
scheme or an algebraic space pass through the given point from the
\'etale local ring. We can only count the number of geometric branches.

\begin{lemma}
\label{lemma-irreducible-local-ring}
Let $S$ be a scheme.
Let $X$ be an algebraic space over $S$.
Let $x \in |X|$ be a point.
The following are equivalent
\begin{enumerate}
\item for any scheme $U$ and \'etale morphism $a : U \to X$ and
$u \in U$ with $a(u) = x$ the local ring $\mathcal{O}_{U, u}$ has a
unique minimal prime,
\item for any scheme $U$ and \'etale morphism $a : U \to X$ and
$u \in U$ with $a(u) = x$ there is a unique irreducible component of $U$
through $u$,
\item for any scheme $U$ and \'etale morphism $a : U \to X$ and
$u \in U$ with $a(u) = x$ the local ring $\mathcal{O}_{U, u}$
is unibranch,
\item for any scheme $U$ and \'etale morphism $a : U \to X$ and
$u \in U$ with $a(u) = x$ the local ring $\mathcal{O}_{U, u}$
is geometrically unibranch,
\item $\mathcal{O}_{X, \overline{x}}$ has a unique minimal prime
for any geometric point $\overline{x}$ lying over $x$.
\end{enumerate}
\end{lemma}

\begin{proof}
The equivalence of (1) and (2) follows from the fact that irreducible
components of $U$ passing through $u$ are in $1$-$1$ correspondence with
minimal primes of the local ring of $U$ at $u$. Let $a : U \to X$ and
$u \in U$ be as in (1). Then $\mathcal{O}_{X, \overline{x}}$ is
the strict henselization of $\mathcal{O}_{U, u}$ by
Lemma \ref{lemma-describe-etale-local-ring}.
In particular (4) and (5) are equivalent by
More on Algebra, Lemma \ref{more-algebra-lemma-geometrically-unibranch}.
The equivalence of (2), (3), and (4) follows from
More on Morphisms, Lemma \ref{more-morphisms-lemma-nr-branches}.
\end{proof}

\begin{definition}
\label{definition-unibranch}
Let $S$ be a scheme. Let $X$ be an algebraic space over $S$.
Let $x \in |X|$. We say that $X$ is {\it geometrically unibranch
at $x$} if the equivalent conditions of
Lemma \ref{lemma-irreducible-local-ring}
hold. We say that $X$ is {\it geometrically unibranch} if $X$ is
geometrically unibranch at every $x \in |X|$.
\end{definition}

\noindent
This is consistent with the definition for schemes
(Properties, Definition \ref{properties-definition-unibranch}).

\begin{lemma}
\label{lemma-nr-branches-local-ring}
Let $S$ be a scheme. Let $X$ be an algebraic space over $S$.
Let $x \in |X|$ be a point. Let $n \in \{1, 2, \ldots\}$ be an integer.
The following are equivalent
\begin{enumerate}
\item for any scheme $U$ and \'etale morphism $a : U \to X$ and
$u \in U$ with $a(u) = x$ the number of minimal primes
of the local ring $\mathcal{O}_{U, u}$ is $\leq n$
and for at least one choice of $U, a, u$ it is $n$,
\item for any scheme $U$ and \'etale morphism $a : U \to X$ and
$u \in U$ with $a(u) = x$ the number irreducible components of
$U$ passing through $u$ is $\leq n$ and for at least one choice
of $U, a, u$ it is $n$,
\item for any scheme $U$ and \'etale morphism $a : U \to X$ and $u \in U$
with $a(u) = x$ the number of branches of $U$ at $u$ is $\leq n$
and for at least one choice of $U, a, u$ it is $n$,
\item for any scheme $U$ and \'etale morphism $a : U \to X$ and $u \in U$
with $a(u) = x$ the number of geometric branches of $U$ at $u$ is $n$, and
\item the number of minimal prime ideals of
$\mathcal{O}_{X, \overline{x}}$ is $n$.
\end{enumerate}
\end{lemma}

\begin{proof}
The equivalence of (1) and (2) follows from the fact that irreducible
components of $U$ passing through $u$ are in $1$-$1$ correspondence with
minimal primes of the local ring of $U$ at $u$. Let $a : U \to X$ and
$u \in U$ be as in (1). Then $\mathcal{O}_{X, \overline{x}}$ is the
strict henselization of $\mathcal{O}_{U, u}$ by
Lemma \ref{lemma-describe-etale-local-ring}. Recall that
the (geometric) number of branches of $U$ at $u$ is the number
of minimal prime ideals of the (strict) henselization of $\mathcal{O}_{U, u}$.
In particular (4) and (5) are equivalent.
The equivalence of (2), (3), and (4) follows from
More on Morphisms, Lemma \ref{more-morphisms-lemma-nr-branches}.
\end{proof}

\begin{definition}
\label{definition-number-of-geometric-branches}
Let $S$ be a scheme.
Let $X$ be an algebraic space over $S$.
Let $x \in |X|$. The {\it number of geometric branches of $X$ at $x$} is
either $n \in \mathbf{N}$ if the equivalent conditions
of Lemma \ref{lemma-nr-branches-local-ring}
hold, or else $\infty$.
\end{definition}








\section{Noetherian spaces}
\label{section-noetherian}

\noindent
We have already defined locally Noetherian algebraic spaces in
Section \ref{section-types-properties}.

\begin{definition}
\label{definition-noetherian}
Let $S$ be a scheme. Let $X$ be an algebraic space over $S$.
We say $X$ is {\it Noetherian} if $X$ is quasi-compact, quasi-separated
and locally Noetherian.
\end{definition}

\noindent
Note that a Noetherian algebraic space $X$ is not just quasi-compact
and locally Noetherian, but also quasi-separated. This does not conflict
with the definition of a Noetherian scheme, as a locally Noetherian
scheme is quasi-separated, see
Properties, Lemma \ref{properties-lemma-locally-Noetherian-quasi-separated}.
This does not hold for algebraic spaces. Namely,
$X = \mathbf{A}^1_k/\mathbf{Z}$, see
Spaces, Example \ref{spaces-example-affine-line-translation}
is locally Noetherian and quasi-compact but not quasi-separated
(hence not Noetherian according to our definitions).

\medskip\noindent
A consequence of the choice made above is that an algebraic space
of finite type over a Noetherian algebraic space is not automatically
Noetherian, i.e., the analogue of
Morphisms, Lemma \ref{morphisms-lemma-finite-type-noetherian}
does not hold. The correct statement is that an algebraic space of
finite presentation over a Noetherian algebraic space is Noetherian
(see
Morphisms of Spaces,
Lemma \ref{spaces-morphisms-lemma-finite-presentation-noetherian}).

\medskip\noindent
A Noetherian algebraic space $X$ is very close to being a scheme.
In the rest of this section we collect some lemmas to illustrate this.

\begin{lemma}
\label{lemma-Noetherian-topology}
Let $S$ be a scheme. Let $X$ be an algebraic space over $S$.
\begin{enumerate}
\item If $X$ is locally Noetherian then $|X|$ is a locally Noetherian
topological space.
\item If $X$ is quasi-compact and locally Noetherian, then $|X|$
is a Noetherian topological space.
\end{enumerate}
\end{lemma}

\begin{proof}
Assume $X$ is locally Noetherian.
Choose a scheme $U$ and a surjective \'etale morphism
$U \to X$. As $X$ is locally Noetherian we see that $U$ is locally
Noetherian. By
Properties, Lemma \ref{properties-lemma-Noetherian-topology}
this means that $|U|$ is a locally Noetherian topological space.
Since $|U| \to |X|$ is open and surjective we conclude that
$|X|$ is locally Noetherian by
Topology, Lemma \ref{topology-lemma-image-Noetherian}.
This proves (1). If $X$ is quasi-compact and locally Noetherian,
then $|X|$ is quasi-compact and locally Noetherian. Hence $|X|$
is Noetherian by
Topology,
Lemma \ref{topology-lemma-quasi-compact-locally-Noetherian-Noetherian}.
\end{proof}

\begin{lemma}
\label{lemma-Noetherian-sober}
Let $S$ be a scheme. Let $X$ be an algebraic space over $S$.
If $X$ is Noetherian, then $|X|$ is a sober Noetherian topological space.
\end{lemma}

\begin{proof}
A quasi-separated algebraic space has an underlying sober topological
space, see
Lemma \ref{lemma-quasi-separated-sober}.
It is Noetherian by
Lemma \ref{lemma-Noetherian-topology}.
\end{proof}

\begin{lemma}
\label{lemma-Noetherian-local-ring-Noetherian}
Let $S$ be a scheme. Let $X$ be a Noetherian algebraic space over $S$.
Let $\overline{x}$ be a geometric point of $X$. Then
$\mathcal{O}_{X, \overline{x}}$ is a Noetherian local ring.
\end{lemma}

\begin{proof}
Choose an \'etale neighbourhood $(U, \overline{u})$ of $\overline{x}$
where $U$ is a scheme. Then $\mathcal{O}_{X, \overline{x}}$ is the
strict henselization of the local ring of $U$ at $u$, see
Lemma \ref{lemma-describe-etale-local-ring}.
By our definition of Noetherian spaces the scheme $U$ is locally Noetherian.
Hence we conclude by
More on Algebra, Lemma \ref{more-algebra-lemma-henselization-noetherian}.
\end{proof}








\section{Regular algebraic spaces}
\label{section-regular}

\noindent
We have already defined regular algebraic spaces in
Section \ref{section-types-properties}.

\begin{lemma}
\label{lemma-regular}
Let $S$ be a scheme.
Let $X$ be a locally Noetherian algebraic space over $S$.
The following are equivalent
\begin{enumerate}
\item $X$ is regular, and
\item every \'etale local ring $\mathcal{O}_{X, \overline{x}}$ is
regular.
\end{enumerate}
\end{lemma}

\begin{proof}
Let $U$ be a scheme and let $U \to X$ be a surjective \'etale morphism.
By assumption $U$ is locally Noetherian. Moreover, every \'etale local
ring $\mathcal{O}_{X, \overline{x}}$ is the strict henselization of
a local ring on $U$ and conversely, see
Lemma \ref{lemma-describe-etale-local-ring}.
Thus by
More on Algebra, Lemma \ref{more-algebra-lemma-henselization-regular}
we see that (2) is equivalent to every local ring of $U$ being
regular, i.e., $U$ being a regular scheme (see
Properties, Lemma \ref{properties-lemma-characterize-regular}).
This equivalent to (1) by
Definition \ref{definition-type-property}.
\end{proof}

\noindent
We can use Descent, Lemma \ref{descent-lemma-regular-local-ring-local}
to define what it means for an algebraic space $X$ to be regular at a
point $x$.

\begin{definition}
\label{definition-regular-at-point}
Let $S$ be a scheme. Let $X$ be an algebraic space over $S$.
Let $x \in |X|$ be a point. We say {\it $X$ is regular at $x$}
if $\mathcal{O}_{U, u}$ is a regular local ring for any
(equivalently some) pair $(a : U \to X, u)$ consisting of an
\'etale morphism $a : U \to X$ from a scheme to $X$ and a point
$u \in U$ with $a(u) = x$.
\end{definition}

\noindent
See Definition \ref{definition-property-at-point},
Lemma \ref{lemma-local-source-target-at-point}, and
Descent, Lemma \ref{descent-lemma-regular-local-ring-local}.

\begin{lemma}
\label{lemma-regular-at-x}
Let $S$ be a scheme. Let $X$ be an algebraic space over $S$.
Let $x \in |X|$ be a point. The following are equivalent
\begin{enumerate}
\item $X$ is regular at $x$, and
\item the \'etale local ring $\mathcal{O}_{X, \overline{x}}$ is
regular for any (equivalently some) geometric point $\overline{x}$
lying over $x$.
\end{enumerate}
\end{lemma}

\begin{proof}
Let $U$ be a scheme, $u \in U$ a point, and let $a : U \to X$ be an
\'etale morphism mapping $u$ to $x$. For any geometric point
$\overline{x}$ of $X$ lying over $x$, the \'etale local
ring $\mathcal{O}_{X, \overline{x}}$ is the strict henselization of
a local ring on $U$ at $u$, see
Lemma \ref{lemma-describe-etale-local-ring}.
Thus we conclude by
More on Algebra, Lemma \ref{more-algebra-lemma-henselization-regular}.
\end{proof}

\begin{lemma}
\label{lemma-regular-normal}
A regular algebraic space is normal.
\end{lemma}

\begin{proof}
This follows from the definitions and the case of schemes
See Properties, Lemma \ref{properties-lemma-regular-normal}.
\end{proof}

 








\section{Sheaves of modules on algebraic spaces}
\label{section-modules}

\noindent
If $X$ is an algebraic space, then a sheaf of modules on $X$ is
a sheaf of $\mathcal{O}_X$-modules on the small \'etale site of $X$
where $\mathcal{O}_X$ is the structure sheaf of $X$. The category
of sheaves of modules is denoted $\textit{Mod}(\mathcal{O}_X)$.

\medskip\noindent
Given a morphism $f : X \to Y$ of algebraic spaces, by
Lemma \ref{lemma-morphism-ringed-topoi}
we get a morphism of ringed topoi and hence by
Modules on Sites, Definition \ref{sites-modules-definition-pushforward}
we get well defined pullback and direct image functors
\begin{equation}
\label{equation-push-pull}
f^* :
\textit{Mod}(\mathcal{O}_Y)
\longrightarrow
\textit{Mod}(\mathcal{O}_X), \quad
f_* :
\textit{Mod}(\mathcal{O}_X)
\longrightarrow
\textit{Mod}(\mathcal{O}_Y)
\end{equation}
which are adjoint in the usual way. If $g : Y \to Z$ is another morphism
of algebraic spaces over $S$, then we have
$(g \circ f)^* = f^* \circ g^*$ and $(g \circ f)_* = g_* \circ f_*$
simply because the morphisms of ringed topoi compose in the corresponding
way (by the lemma).

\begin{lemma}
\label{lemma-etale-exact-pullback}
Let $S$ be a scheme.
Let $f : X \to Y$ be an \'etale morphism of algebraic spaces over $S$.
Then $f^{-1}\mathcal{O}_Y = \mathcal{O}_X$, and
$f^*\mathcal{G} = f_{small}^{-1}\mathcal{G}$ for any sheaf of
$\mathcal{O}_Y$-modules $\mathcal{G}$. In particular,
$f^* : \textit{Mod}(\mathcal{O}_Y) \to \textit{Mod}(\mathcal{O}_X)$
is exact.
\end{lemma}

\begin{proof}
By the description of inverse image in Lemma \ref{lemma-etale-morphism-topoi}
and the definition of the structure sheaves it is clear that
$f_{small}^{-1}\mathcal{O}_Y = \mathcal{O}_X$. Since the pullback
$$
f^*\mathcal{G} =
f_{small}^{-1}\mathcal{G} \otimes_{f_{small}^{-1}\mathcal{O}_Y}
\mathcal{O}_X
$$
by definition we conclude that $f^*\mathcal{G} = f_{small}^{-1}\mathcal{G}$.
The exactness is clear because $f_{small}^{-1}$ is exact, as $f_{small}$
is a morphism of topoi.
\end{proof}

\noindent
We continue our abuse of notation introduced in
Equation (\ref{equation-restrict})
by writing
\begin{equation}
\label{equation-restrict-modules}
\mathcal{G}|_{X_\etale}
= f^*\mathcal{G}
= f_{small}^{-1}\mathcal{G}
\end{equation}
in the situation of the lemma above. We will discuss this in a more
technical fashion in
Section \ref{section-localize}.

\begin{lemma}
\label{lemma-pushforward-etale-base-change-modules}
Let $S$ be a scheme. Let
$$
\xymatrix{
X' \ar[r] \ar[d]_{f'} & X \ar[d]^f \\
Y' \ar[r]^g & Y
}
$$
be a cartesian square of algebraic spaces over $S$. Let
$\mathcal{F} \in \textit{Mod}(\mathcal{O}_X)$. If $g$ is \'etale, then
$f'_*(\mathcal{F}|_{X'}) = (f_*\mathcal{F})|_{Y'}$\footnote{Also
$(f')^*(\mathcal{G}|_{Y'}) = (f^*\mathcal{G})|_{X'}$
by commutativity of the diagram and (\ref{equation-restrict-modules})} and
$R^if'_*(\mathcal{F}|_{X'}) = (R^if_*\mathcal{F})|_{Y'}$ in
$\textit{Mod}(\mathcal{O}_{Y'})$.
\end{lemma}

\begin{proof}
This is a reformulation of
Lemma \ref{lemma-pushforward-etale-base-change}
in the case of modules.
\end{proof}

\begin{lemma}
\label{lemma-characterize-module-small-etale}
Let $S$ be a scheme. Let $X$ be an algebraic space over $S$.
A sheaf $\mathcal{F}$ of $\mathcal{O}_X$-modules is given by the following
data:
\begin{enumerate}
\item for every $U \in \Ob(X_\etale)$ a sheaf
$\mathcal{F}_U$ of $\mathcal{O}_U$-modules on $U_\etale$,
\item for every $f : U' \to U$ in $X_\etale$ an isomorphism
$c_f : f_{small}^*\mathcal{F}_U \to \mathcal{F}_{U'}$.
\end{enumerate}
These data are subject to the condition that given any $f : U' \to U$
and $g : U'' \to U'$ in $X_\etale$ the composition
$c_g \circ g_{small}^*c_f$ is equal to $c_{f \circ g}$.
\end{lemma}

\begin{proof}
Combine Lemmas \ref{lemma-etale-exact-pullback}
and \ref{lemma-characterize-sheaf-small-etale}, and use the fact that
any morphism between objects of $X_\etale$ is an \'etale morphism
of schemes.
\end{proof}







\section{\'Etale localization}
\label{section-localize}

\noindent
Reading this section should be avoided at all cost.

\medskip\noindent
Let $X \to Y$ be an \'etale morphism of algebraic spaces.
Then $X$ is an object of $Y_{spaces, \etale}$ and it is
immediate from the definitions, see also the proof of
Lemma \ref{lemma-etale-morphism-topoi},
that
\begin{equation}
\label{equation-localize}
X_{spaces, \etale} = Y_{spaces, \etale}/X
\end{equation}
where the right hand side is the localization of the site
$Y_{spaces, \etale}$ at the object $X$, see
Sites, Definition \ref{sites-definition-localize}.
Moreover, this identification is compatible with the structure sheaves by
Lemma \ref{lemma-etale-exact-pullback}.
Hence the ringed site $(X_{spaces, \etale}, \mathcal{O}_X)$
is identified with the localization of the ringed site
$(Y_{spaces, \etale}, \mathcal{O}_Y)$ at the object $X$:
\begin{equation}
\label{equation-localize-ringed}
(X_{spaces, \etale}, \mathcal{O}_X) =
(Y_{spaces, \etale}/X, \mathcal{O}_Y|_{Y_{spaces, \etale}/X})
\end{equation}
The localization of a ringed site used on the right hand side is defined in
Modules on Sites,
Definition \ref{sites-modules-definition-localize-ringed-site}.

\medskip\noindent
Assume now $X \to Y$ is an \'etale morphism of algebraic spaces and $X$ is
a scheme. Then $X$ is an object of $Y_\etale$ and it follows that
\begin{equation}
\label{equation-localize-at-scheme}
X_\etale = Y_\etale/X
\end{equation}
and
\begin{equation}
\label{equation-localize-at-scheme-ringed}
(X_\etale, \mathcal{O}_X) =
(Y_\etale/X, \mathcal{O}_Y|_{Y_\etale/X})
\end{equation}
as above.

\medskip\noindent
Finally, if $X \to Y$ is an \'etale morphism of algebraic spaces and $X$ is
an affine scheme, then $X$ is an object of $Y_{affine, \etale}$ and
\begin{equation}
\label{equation-localize-at-affine}
X_{affine, \etale} = Y_{affine, \etale}/X
\end{equation}
and
\begin{equation}
\label{equation-localize-at-affine-ringed}
(X_{affine, \etale}, \mathcal{O}_X) =
(Y_{affine, \etale}/X, \mathcal{O}_Y|_{Y_{affine, \etale}/X})
\end{equation}
as above.

\medskip\noindent
Next, we show that these localizations are compatible with morphisms.

\begin{lemma}
\label{lemma-relocalize-morphism}
Let $S$ be a scheme. Let
$$
\xymatrix{
U \ar[d]_p \ar[r]_g & V \ar[d]^q \\
X \ar[r]^f & Y
}
$$
be a commutative diagram of algebraic spaces over $S$ with $p$ and $q$ \'etale.
Via the identifications
(\ref{equation-localize-ringed}) for $U \to X$ and $V \to Y$
the morphism of ringed topoi
$$
(g_{spaces, \etale}, g^\sharp) :
(\Sh(U_{spaces, \etale}), \mathcal{O}_U)
\longrightarrow
(\Sh(V_{spaces, \etale}), \mathcal{O}_V)
$$
is $2$-isomorphic to the morphism $(f_{spaces, \etale, c}, f_c^\sharp)$
constructed in
Modules on Sites,
Lemma \ref{sites-modules-lemma-relocalize-morphism-ringed-sites}
starting with the morphism of ringed sites
$(f_{spaces, \etale}, f^\sharp)$ and
the map $c : U \to V \times_Y X$ corresponding to $g$.
\end{lemma}

\begin{proof}
The morphism $(f_{spaces, \etale, c}, f_c^\sharp)$ is defined as a
composition $f' \circ j$
of a localization and a base change map. Similarly $g$ is a composition
$U \to V \times_Y X \to V$. Hence it suffices to prove
the lemma in the following two cases: (1) $f = \text{id}$, and
(2) $U = X \times_Y V$. In case (1) the morphism $g : U \to V$ is
\'etale, see
Lemma \ref{lemma-etale-permanence}.
Hence $(g_{spaces, \etale}, g^\sharp)$ is a localization morphism
by the discussion surrounding
Equations (\ref{equation-localize}) and
(\ref{equation-localize-ringed})
which is exactly the content of the lemma in this case.
In case (2) the morphism $g_{spaces, \etale}$
comes from the morphism of ringed sites given by the functor
$V_{spaces, \etale} \to U_{spaces, \etale}$,
$V'/V \mapsto V' \times_V U/U$
which is also what the morphism $f'$ is defined by, see
Sites, Lemma \ref{sites-lemma-localize-morphism}.
We omit the verification that $(f')^\sharp = g^\sharp$
in this case (both are the restriction of $f^\sharp$
to $U_{spaces, \etale}$).
\end{proof}

\begin{lemma}
\label{lemma-relocalize-morphism-at-schemes}
Same notation and assumptions as in
Lemma \ref{lemma-relocalize-morphism}
except that we also assume $U$ and $V$ are schemes.
Via the identifications
(\ref{equation-localize-at-scheme-ringed})
for $U \to X$ and $V \to Y$ the morphism of ringed topoi
$$
(g_{small}, g^\sharp) :
(\Sh(U_\etale), \mathcal{O}_U)
\longrightarrow
(\Sh(V_\etale), \mathcal{O}_V)
$$
is $2$-isomorphic to the morphism $(f_{small, s}, f_s^\sharp)$
constructed in
Modules on Sites,
Lemma \ref{sites-modules-lemma-relocalize-morphism-ringed-topoi}
starting with $(f_{small}, f^\sharp)$ and
the map $s : h_U \to f_{small}^{-1}h_V$ corresponding to $g$.
\end{lemma}

\begin{proof}
Note that $(g_{small}, g^\sharp)$ is $2$-isomorphic as a
morphism of ringed topoi to the morphism of ringed topoi
associated to the morphism of ringed sites
$(g_{spaces, \etale}, g^\sharp)$. Hence we conclude by
Lemma \ref{lemma-relocalize-morphism}
and
Modules on Sites,
Lemma \ref{sites-modules-lemma-relocalize-morphism-compare}.
\end{proof}

\noindent
Finally, we discuss the relationship between sheaves of sets on the small
\'etale site $Y_\etale$ of an algebraic space $Y$ and algebraic spaces
\'etale over $Y$. Let $S$ be a scheme and let $Y$ be an algebraic space
over $S$. Let $\mathcal{F}$ be an object of $\Sh(Y_\etale)$. Consider
the functor
$$
X : (\Sch/S)_{fppf}^{opp} \longrightarrow \textit{Sets}
$$
defined by the rule
$$
X(T) = \{(y, s) \mid y : T \to Y\text{ is a morphism over }S\text{ and }
s \in \Gamma(T, y_{small}^{-1}\mathcal{F})\}
$$
Given a morphism $g : T' \to T$ the restriction map sends
$(y, s)$ to $(y \circ g, g_{small}^{-1}s)$. This makes sense
as $y_{small} \circ g_{small} = (y \circ g)_{small}$ by
Lemma \ref{lemma-functoriality-etale-site}. There is a canonical map
$X \to Y$ sending the pair $(y, s)$ to $y$.

\begin{lemma}
\label{lemma-sheaf-gives-space}
Let $S$ be a scheme and let $Y$ be an algebraic space over $S$.
Let $\mathcal{F}$ be a sheaf of sets on $Y_\etale$.
Provided a set theoretic condition is satisfied (see proof) we have
\begin{enumerate}
\item the functor $X$ associated to $\mathcal{F}$ above is an
algebraic space,
\item the map $X \to Y$ is an \'etale morphism of algebraic spaces,
\item via the identification $\Sh(Y_\etale) = \Sh(Y_{spaces, \etale})$
we have $\mathcal{F} \cong h_X$,
\item we have $\mathcal{F} \cong f_{small, !}*$. Here $*$ is the final object
of the category $\Sh(X_\etale)$ and $f_{small, !}$ exists by
Lemma \ref{lemma-etale-morphism-topoi}.
\end{enumerate}
\end{lemma}

\begin{proof}
Let us prove that $X$ is a sheaf for the fppf topology. Namely, suppose
that $\{g_i : T_i \to T\}$ is a covering of $(\Sch/S)_{fppf}$ and
$(y_i, s_i) \in X(T_i)$ satisfy the glueing condition, i.e., the
restriction of $(y_i, s_i)$ and $(y_j, s_j)$ to $T_i \times_T T_j$ agree.
Then since $Y$ is a sheaf for the fppf topology, we see that
the $y_i$ give rise to a unique morphism $y : T \to Y$ such that
$y_i = y \circ g_i$. Then we see that
$y_{i, small}^{-1}\mathcal{F} = g_{i, small}^{-1}y_{small}^{-1}\mathcal{F}$.
Hence the sections $s_i$ glue uniquely to a section of
$y_{small}^{-1}\mathcal{F}$ by
\'Etale Cohomology, Lemma \ref{etale-cohomology-lemma-describe-pullback}.

\medskip\noindent
The construction that sends $\mathcal{F} \in \Ob(\Sh(Y_\etale))$
to $X \in \Ob((\Sch/S)_{fppf})$ preserves finite limits and
all colimits since each of the functors $y_{small}^{-1}$ have
this property. Of course, if $V \in \Ob(Y_\etale)$, then
the construction sends the representable sheaf $h_V$ on $Y_\etale$
to the representable functor represented by $V$.

\medskip\noindent
By Sites, Lemma \ref{sites-lemma-sheaf-coequalizer-representable}
we can find a set $I$, for each $i \in I$ an object $V_i$ of $Y_\etale$
and a surjective map of sheaves
$$
\coprod h_{V_i} \longrightarrow \mathcal{F}
$$
on $Y_\etale$. The set theoretic condition we need is that the index
set $I$ is not too large\footnote{It suffices if the supremum of the
cardinalities of the stalks of $\mathcal{F}$ at geometric points of $Y$
is bounded by the size of some object of $(\Sch/S)_{fppf}$.}.
Then $V = \coprod V_i$ is an object of $(\Sch/S)_{fppf}$ and
therefore an object of $Y_\etale$ and we have a surjective map
$h_V \to \mathcal{F}$.

\medskip\noindent
Observe that the product of $h_V$ with itself in $\Sh(Y_\etale)$
is $h_{V \times_Y V}$. Consider the fibre product
$$
h_V \times_\mathcal{F} h_V \subset h_{V \times_Y V}
$$
There is an open subscheme $R$ of $V \times_Y V$ such that
$h_V \times_\mathcal{F} h_V = h_R$, see
Lemma \ref{lemma-support-subsheaf-final} (small detail omitted).
By the Yoneda lemma we obtain two morphisms $s, t : R \to V$ in $Y_\etale$
and we find a coequalizer diagram
$$
\xymatrix{
h_R \ar@<1ex>[r] \ar@<-1ex>[r] &
h_V \ar[r] &
\mathcal{F}
}
$$
in $\Sh(Y_\etale)$. Of course the morphisms $s, t$ are \'etale
and define an \'etale equivalence relation $(t, s) : R \to V \times_S V$.

\medskip\noindent
By the discussion in the preceding two paragraphs we find a
coequalizer diagram
$$
\xymatrix{
R \ar@<1ex>[r] \ar@<-1ex>[r] &
V \ar[r] &
X
}
$$
in $(\Sch/S)_{fppf}$. Thus $X = V/R$ is an algebraic space by
Spaces, Theorem \ref{spaces-theorem-presentation}. This proves (1).
Part (2) follows because $V \to Y$ is \'etale. Part (3) is immediate
from the definition of $X$ and $h_X$. We omit the proof of part (4);
it follows by matching the morphism associated to the cocontinuous
functor $j$ of Lemma \ref{lemma-etale-morphism-topoi}
with the description of $X_{spaces, \etale}$ as the localization
of $Y_{spaces, \etale}$ at $X$ discussed above and
Sites, Lemma \ref{sites-lemma-describe-j-shriek-representable}.
\end{proof}







\section{Recovering morphisms}
\label{section-morphisms}

\noindent
In this section we prove that the rule which associates to an algebraic space
its locally ringed small \'etale topos is fully faithful in a suitable
sense, see
Theorem \ref{theorem-fully-faithful}.

\begin{lemma}
\label{lemma-morphism-locally-ringed}
Let $S$ be a scheme.
Let $f : X \to Y$ be a morphism of algebraic spaces over $S$.
The morphism of ringed topoi $(f_{small}, f^\sharp)$
associated to $f$ is a morphism of locally ringed topoi, see
Modules on Sites,
Definition \ref{sites-modules-definition-morphism-locally-ringed-topoi}.
\end{lemma}

\begin{proof}
Note that the assertion makes sense since we have seen that
$(X_\etale, \mathcal{O}_{X_\etale})$ and
$(Y_\etale, \mathcal{O}_{Y_\etale})$
are locally ringed sites, see
Lemma \ref{lemma-etale-site-locally-ringed}.
Moreover, we know that $X_\etale$ has enough points, see
Theorem \ref{theorem-exactness-stalks}.
Hence it suffices to prove that $(f_{small}, f^\sharp)$
satisfies condition (3) of
Modules on Sites,
Lemma \ref{sites-modules-lemma-locally-ringed-morphism}.
To see this take a point $p$ of $X_\etale$. By
Lemma \ref{lemma-points-small-etale-site}
$p$ corresponds to a geometric point $\overline{x}$ of $X$.
By
Lemma \ref{lemma-stalk-pullback}
the point $q = f_{small} \circ p$ corresponds to the
geometric point $\overline{y} = f \circ \overline{x}$ of $Y$.
Hence the assertion we have to prove is that the induced map
of \'etale local rings
$$
\mathcal{O}_{Y, \overline{y}} \longrightarrow \mathcal{O}_{X, \overline{x}}
$$
is a local ring map. You can prove this directly, but instead we deduce it
from the corresponding result for schemes. To do this choose a commutative
diagram
$$
\xymatrix{
U \ar[d] \ar[r]_\psi & V \ar[d] \\
X \ar[r] & Y
}
$$
where $U$ and $V$ are schemes, and the vertical arrows are surjective
\'etale (see
Spaces, Lemma \ref{spaces-lemma-lift-morphism-presentations}).
Choose a lift $\overline{u} : \overline{x} \to U$ (possible by
Lemma \ref{lemma-geometric-lift-to-cover}).
Set $\overline{v} = \psi \circ \overline{u}$. We obtain a commutative
diagram of \'etale local rings
$$
\xymatrix{
\mathcal{O}_{U, \overline{u}} &
\mathcal{O}_{V, \overline{v}} \ar[l] \\
\mathcal{O}_{X, \overline{x}} \ar[u] &
\mathcal{O}_{Y, \overline{y}}. \ar[l] \ar[u]
}
$$
By
\'Etale Cohomology, Lemma \ref{etale-cohomology-lemma-morphism-locally-ringed}
the top horizontal arrow is a local ring map. Finally by
Lemma \ref{lemma-describe-etale-local-ring}
the vertical arrows are isomorphisms. Hence we win.
\end{proof}

\begin{lemma}
\label{lemma-2-morphism}
Let $S$ be a scheme.
Let $X$, $Y$ be algebraic spaces over $S$.
Let $f : X \to Y$ be a morphism of algebraic spaces over $S$.
Let $t$ be a $2$-morphism from $(f_{small}, f^\sharp)$ to itself, see
Modules on Sites,
Definition \ref{sites-modules-definition-2-morphism-ringed-topoi}.
Then $t = \text{id}$.
\end{lemma}

\begin{proof}
Let $X'$, resp.\ $Y'$ be $X$ viewed as an algebraic space over
$\Spec(\mathbf{Z})$, see
Spaces, Definition \ref{spaces-definition-base-change}.
It is clear from the construction that $(X_{small}, \mathcal{O})$
is equal to $(X'_{small}, \mathcal{O})$ and similarly for $Y$.
Hence we may work with $X'$ and $Y'$. In other words we may
assume that $S = \Spec(\mathbf{Z})$.

\medskip\noindent
Assume $S = \Spec(\mathbf{Z})$, $f : X \to Y$ and $t$ are as in
the lemma. This means that $t : f^{-1}_{small} \to f^{-1}_{small}$
is a transformation of functors such that the diagram
$$
\xymatrix{
f_{small}^{-1}\mathcal{O}_Y
\ar[rd]_{f^\sharp}  & &
f_{small}^{-1}\mathcal{O}_Y \ar[ll]^t \ar[ld]^{f^\sharp} \\
& \mathcal{O}_X
}
$$
is commutative. Suppose $V \to Y$ is \'etale with $V$ affine.
Write $V = \Spec(B)$. Choose generators $b_j \in B$, $j \in J$
for $B$ as a $\mathbf{Z}$-algebra. Set
$T = \Spec(\mathbf{Z}[\{x_j\}_{j \in J}])$.
In the following we will use that
$\Mor_{\Sch}(U, T) = \prod_{j \in J} \Gamma(U, \mathcal{O}_U)$
for any scheme $U$ without further mention.
The surjective ring map $\mathbf{Z}[x_j] \to B$, $x_j \mapsto b_j$
corresponds to a closed immersion $V \to T$.
We obtain a monomorphism
$$
i : V \longrightarrow T_Y = T \times Y
$$
of algebraic spaces over $Y$. In terms of sheaves on $Y_\etale$
the morphism $i$ induces an injection
$h_i : h_V \to \prod_{j \in J} \mathcal{O}_Y$ of sheaves.
The base change $i' : X \times_Y V \to T_X$ of $i$ to $X$
is a monomorphism too
(Spaces,
Lemma \ref{spaces-lemma-base-change-representable-transformations-property}).
Hence $i' : X \times_Y V \to T_X$ is a monomorphism, which
in turn means that
$h_{i'} : h_{X \times_Y V} \to \prod_{j \in J} \mathcal{O}_X$
is an injection of sheaves.
Via the identification $f_{small}^{-1}h_V = h_{X \times_Y V}$ of
Lemma \ref{lemma-stalk-pullback}
the map $h_{i'}$ is equal to
$$
\xymatrix{
f_{small}^{-1}h_V \ar[r]^-{f^{-1}h_i} &
\prod_{j \in J} f_{small}^{-1}\mathcal{O}_Y
\ar[r]^{\prod f^\sharp} &
\prod_{j \in J} \mathcal{O}_X
}
$$
(verification omitted). This means that the map
$t : f_{small}^{-1}h_V \to f_{small}^{-1}h_V$
fits into the commutative diagram
$$
\xymatrix{
f_{small}^{-1}h_V \ar[r]^-{f^{-1}h_i} \ar[d]^t &
\prod_{j \in J} f_{small}^{-1}\mathcal{O}_Y
\ar[r]^-{\prod f^\sharp} \ar[d]^{\prod t} &
\prod_{j \in J} \mathcal{O}_X \ar[d]^{\text{id}}\\
f_{small}^{-1}h_V \ar[r]^-{f^{-1}h_i} &
\prod_{j \in J} f_{small}^{-1}\mathcal{O}_Y
\ar[r]^-{\prod f^\sharp} &
\prod_{j \in J} \mathcal{O}_X
}
$$
The commutativity of the right square holds by our assumption on $t$
explained above.
Since the composition of the horizontal arrows is injective
by the discussion above we conclude that the left vertical arrow
is the identity map as well. Any sheaf of sets on
$Y_\etale$ admits a surjection from a (huge) coproduct of sheaves
of the form $h_V$ with $V$ affine (combine
Lemma \ref{lemma-alternative}
with
Sites, Lemma \ref{sites-lemma-sheaf-coequalizer-representable}).
Thus we conclude that $t : f_{small}^{-1} \to f_{small}^{-1}$
is the identity transformation as desired.
\end{proof}

\begin{lemma}
\label{lemma-faithful}
Let $S$ be a scheme.
Let $X$, $Y$ be algebraic spaces over $S$.
Any two morphisms $a, b : X \to Y$ of algebraic spaces over $S$
for which there exists a $2$-isomorphism
$(a_{small}, a^\sharp) \cong (b_{small}, b^\sharp)$
in the $2$-category of ringed topoi are equal.
\end{lemma}

\begin{proof}
Let $t : a_{small}^{-1} \to b_{small}^{-1}$ be the $2$-isomorphism.
We may equivalently think of $t$ as a transformation
$t : a_{spaces, \etale}^{-1} \to b_{spaces, \etale}^{-1}$
since there is not difference between sheaves on $X_\etale$
and sheaves on $X_{spaces, \etale}$.
Choose a commutative diagram
$$
\xymatrix{
U \ar[d]_p \ar[r]_\alpha  & V \ar[d]^q \\
X \ar[r]^a & Y
}
$$
where $U$ and $V$ are schemes, and $p$ and $q$ are surjective \'etale.
Consider the diagram
$$
\xymatrix{
h_U \ar[r]_-\alpha \ar@{=}[d] & a_{spaces, \etale}^{-1}h_V \ar[d]^t \\
h_U \ar@{..>}[r] & b_{spaces, \etale}^{-1}h_V
}
$$
Since the sheaf $b_{spaces, \etale}^{-1}h_V$ is isomorphic to
$h_{V \times_{Y, b} X}$ we see that the dotted arrow comes from a
morphism of schemes
$\beta : U \to V$ fitting into a commutative diagram
$$
\xymatrix{
U \ar[d]_p \ar[r]_\beta  & V \ar[d]^q \\
X \ar[r]^b & Y
}
$$
We claim that there exists a sequence of $2$-isomorphisms
\begin{align*}
(\alpha_{small}, \alpha^\sharp)
& \cong
(\alpha_{spaces, \etale}, \alpha^\sharp) \\
& \cong
(a_{spaces, \etale, c}, a_c^\sharp) \\
& \cong
(b_{spaces, \etale, d}, b_d^\sharp) \\
& \cong
(\beta_{spaces, \etale}, \beta^\sharp) \\
& \cong
(\beta_{small}, \beta^\sharp)
\end{align*}
The first and the last $2$-isomorphisms come from the identifications
between sheaves on $U_{spaces, \etale}$ and sheaves on
$U_\etale$ and similarly for $V$. The second and fourth
$2$-isomorphisms are those of
Lemma \ref{lemma-relocalize-morphism}
with $c : U \to X \times_{a, Y} V$ induced by $\alpha$ and
$d : U \to X \times_{b, Y} V$ induced by $\beta$.
The middle $2$-isomorphism comes from the transformation $t$.
Namely, the functor $a_{spaces, \etale, c}^{-1}$ corresponds
to the functor
$$
(\mathcal{H} \to h_V) \longmapsto
(a_{spaces, \etale}^{-1}\mathcal{H}
\times_{a_{spaces, \etale}^{-1}h_V, \alpha}
h_U \to
h_U)
$$
and similarly for $b_{spaces, \etale, d}^{-1}$, see
Sites, Lemma \ref{sites-lemma-relocalize-morphism}.
This uses the identification of sheaves on $Y_{spaces, \etale}/V$
as arrows $(\mathcal{H} \to h_V)$ in $\Sh(Y_{spaces, \etale})$
and similarly for $U/X$, see
Sites, Lemma \ref{sites-lemma-essential-image-j-shriek}.
Via this identification the structure sheaf $\mathcal{O}_V$ corresponds to the
pair $(\mathcal{O}_Y \times h_V \to h_V)$ and similarly
for $\mathcal{O}_U$, see
Modules on Sites,
Lemma \ref{sites-modules-lemma-localize-compare}.
Since $t$ switches $\alpha$ and $\beta$ we see that $t$ induces an isomorphism
$$
t :
a_{spaces, \etale}^{-1}\mathcal{H}
\times_{a_{spaces, \etale}^{-1}h_V, \alpha}
h_U
\longrightarrow
b_{spaces, \etale}^{-1}\mathcal{H}
\times_{b_{spaces, \etale}^{-1}h_V, \beta}
h_U
$$
over $h_U$ functorially in $(\mathcal{H} \to h_V)$. Also, $t$ is compatible
with $a_c^\sharp$ and $b_d^\sharp$ as $t$ is
compatible with $a^\sharp$ and $b^\sharp$ by our description
of the structure sheaves $\mathcal{O}_U$ and $\mathcal{O}_V$
above. Hence, the morphisms of ringed topoi
$(\alpha_{small}, \alpha^\sharp)$ and $(\beta_{small}, \beta^\sharp)$
are $2$-isomorphic. By
\'Etale Cohomology, Lemma \ref{etale-cohomology-lemma-faithful}
we conclude $\alpha = \beta$! Since $p : U \to X$ is a surjection
of sheaves it follows that $a = b$.
\end{proof}

\noindent
Here is the main result of this section.

\begin{theorem}
\label{theorem-fully-faithful}
Let $X$, $Y$ be algebraic spaces over $\Spec(\mathbf{Z})$.
Let
$$
(g, g^\sharp) :
(\Sh(X_\etale), \mathcal{O}_X)
\longrightarrow
(\Sh(Y_\etale), \mathcal{O}_Y)
$$
be a morphism of locally ringed topoi. Then there exists a
unique morphism of algebraic spaces $f : X \to Y$ such that
$(g, g^\sharp)$ is isomorphic to $(f_{small}, f^\sharp)$.
In other words, the construction
$$
\textit{Spaces}/\Spec(\mathbf{Z})
\longrightarrow \textit{Locally ringed topoi},
\quad
X \longrightarrow (X_\etale, \mathcal{O}_X)
$$
is fully faithful (morphisms up to $2$-isomorphisms on the right hand side).
\end{theorem}

\begin{proof}
The uniqueness we have seen in
Lemma \ref{lemma-faithful}.
Thus it suffices to prove existence.
In this proof we will freely use the identifications of
Equation (\ref{equation-localize-at-scheme-ringed})
as well as the result of
Lemma \ref{lemma-relocalize-morphism-at-schemes}.

\medskip\noindent
Let $U \in \Ob(X_\etale)$, let
$V \in \Ob(Y_\etale)$
and let $s \in g^{-1}h_V(U)$ be a section. We may think of
$s$ as a map of sheaves $s : h_U \to g^{-1}h_V$. By
Modules on Sites,
Lemma \ref{sites-modules-lemma-relocalize-morphism-ringed-topoi}
we obtain a commutative diagram of morphisms of ringed topoi
$$
\xymatrix{
(\Sh(X_\etale/U), \mathcal{O}_U)
\ar[rr]_-{(j, j^\sharp)} \ar[d]_{(g_s, g_s^\sharp)} & &
(\Sh(X_\etale), \mathcal{O}_X) \ar[d]^{(g, g^\sharp)} \\
(\Sh(V_\etale), \mathcal{O}_V) \ar[rr] & &
(\Sh(Y_\etale), \mathcal{O}_Y).
}
$$
By
\'Etale Cohomology, Theorem \ref{etale-cohomology-theorem-fully-faithful}
we obtain a unique morphism of schemes $f_s : U \to V$ such that
$(g_s, g_s^\sharp)$ is $2$-isomorphic to $(f_{s, small}, f_s^\sharp)$.
The construction $(U, V, s) \leadsto f_s$ just explained satisfies
the following functoriality property: Suppose given morphisms
$a : U' \to U$ in $X_\etale$ and $b : V' \to V$ in $Y_\etale$
and a map $s' : h_{U'} \to g^{-1}h_{V'}$ such that the diagram
$$
\xymatrix{
h_{U'} \ar[d]_a \ar[r]_{s'} & g^{-1}h_{V'} \ar[d]^{g^{-1}b} \\
h_U \ar[r]^s & g^{-1}h_V
}
$$
commutes. Then the diagram
$$
\xymatrix{
U' \ar[r]_-{f_{s'}} \ar[d]_a & u(V') \ar[d]^{u(b)} \\
U \ar[r]^-{f_s} & u(V)
}
$$
of schemes commutes. The reason this is true is that the same condition
holds for the morphisms $(g_s, g_s^\sharp)$ constructed in
Modules on Sites,
Lemma \ref{sites-modules-lemma-relocalize-morphism-ringed-topoi}
and the uniqueness in
\'Etale Cohomology, Theorem \ref{etale-cohomology-theorem-fully-faithful}.

\medskip\noindent
The problem is to glue the morphisms $f_s$ to a morphism of algebraic
spaces. To do this first choose a scheme $V$ and a surjective \'etale
morphism $V \to Y$. This means that $h_V \to *$ is surjective and hence
$g^{-1}h_V \to *$ is surjective too. This means there exists a scheme $U$
and a surjective \'etale morphism $U \to X$ and a morphism
$s : h_U \to g^{-1}h_V$. Next, set $R = V \times_Y V$ and
$R' = U \times_X U$. Then we get
$g^{-1}h_R = g^{-1}h_V \times g^{-1}h_V$ as $g^{-1}$ is exact.
Thus $s$ induces a morphism $s \times s : h_{R'} \to g^{-1}h_R$.
Applying the constructions above we see that we get a
commutative diagram of morphisms of schemes
$$
\xymatrix{
R' \ar@<1ex>[d] \ar@<-1ex>[d] \ar[rr]_{f_{s \times s}} & &
R \ar@<1ex>[d] \ar@<-1ex>[d] \\
U \ar[rr]^{f_s} & &
V
}
$$
Since we have $X = U/R'$ and $Y = V/R$ (see
Spaces, Lemma \ref{spaces-lemma-space-presentation})
we conclude that this diagram
defines a morphism of algebraic spaces $f : X \to Y$ fitting
into an obvious commutative diagram.
Now we still have to show that $(f_{small}, f^\sharp)$ is
$2$-isomorphic to $(g, g^\sharp)$.
Let $t_V : f_{s, small}^{-1} \to g_s^{-1}$ and
$t_R : f_{s \times s, small}^{-1} \to g_{s \times s}^{-1}$ be
the $2$-isomorphisms which are given to us by the construction above.
Let $\mathcal{G}$ be a sheaf on $Y_\etale$. Then we see that
$t_V$ defines an isomorphism
$$
f_{small}^{-1}\mathcal{G}|_{U_\etale}
=
f_{s, small}^{-1}\mathcal{G}|_{V_\etale}
\xrightarrow{t_V}
g_s^{-1}\mathcal{G}|_{V_\etale}
=
g^{-1}\mathcal{G}|_{U_\etale}.
$$
Moreover, this isomorphism pulled back to $R'$ via either projection
$R' \to U$ is the isomorphism
$$
f_{small}^{-1}\mathcal{G}|_{R'_\etale}
=
f_{s \times s, small}^{-1}\mathcal{G}|_{R_\etale}
\xrightarrow{t_R}
g_{s \times s}^{-1}\mathcal{G}|_{R_\etale}
=
g^{-1}\mathcal{G}|_{R'_\etale}.
$$
Since $\{U \to X\}$ is a covering in the site $X_{spaces, \etale}$
this means the first displayed isomorphism descends to an isomorphism
$t : f_{small}^{-1}\mathcal{G} \to g^{-1}\mathcal{G}$
of sheaves (small detail omitted). The isomorphism is functorial
in $\mathcal{G}$ since $t_V$ and $t_R$ are transformations of functors.
Finally, $t$ is compatible with $f^\sharp$ and $g^\sharp$ as
$t_V$ and $t_R$ are (some details omitted).
This finishes the proof of the theorem.
\end{proof}

\begin{lemma}
\label{lemma-isomorphism-ringed-topoi}
Let $X$, $Y$ be algebraic spaces over $\mathbf{Z}$. If
$$
(g, g^\sharp) :
(\Sh(X_\etale), \mathcal{O}_X)
\longrightarrow
(\Sh(Y_\etale), \mathcal{O}_Y)
$$
is an isomorphism of ringed topoi, then there exists a unique
morphism $f : X \to Y$ of algebraic spaces such that
$(g, g^\sharp)$ is isomorphic to $(f_{small}, f^\sharp)$
and moreover $f$ is an isomorphism of algebraic spaces.
\end{lemma}

\begin{proof}
By
Theorem \ref{theorem-fully-faithful}
it suffices to show that $(g, g^\sharp)$ is a morphism of
locally ringed topoi. By
Modules on Sites, Lemma \ref{sites-modules-lemma-locally-ringed-morphism}
(and since the site $X_\etale$ has enough points)
it suffices to check that the map
$\mathcal{O}_{Y, q} \to \mathcal{O}_{X, p}$ induced by $g^\sharp$
is a local ring map where $q = f \circ p$ and $p$ is any point of
$X_\etale$. As it is an isomorphism this is clear.
\end{proof}














\section{Quasi-coherent sheaves on algebraic spaces}
\label{section-quasi-coherent}

\noindent
In
Descent, Sections \ref{descent-section-quasi-coherent-sheaves},
\ref{descent-section-quasi-coherent-cohomology}, and
\ref{descent-section-quasi-coherent-sheaves-bis}
we have seen that for a scheme $U$, there is no difference between a
quasi-coherent $\mathcal{O}_U$-module on $U$, or a quasi-coherent
$\mathcal{O}$-module on the small \'etale site of $U$. Hence the following
definition is compatible with our original notion of a quasi-coherent sheaf
on a scheme
(Schemes, Section \ref{schemes-section-quasi-coherent}),
when applied to a representable algebraic space.

\begin{definition}
\label{definition-quasi-coherent}
Let $S$ be a scheme. Let $X$ be an algebraic space over $S$.
A {\it quasi-coherent} $\mathcal{O}_X$-module
is a quasi-coherent module on the ringed site
$(X_\etale, \mathcal{O}_X)$ in the sense of
Modules on Sites,
Definition \ref{sites-modules-definition-site-local}.
The category of quasi-coherent sheaves on $X$ is denoted
$\QCoh(\mathcal{O}_X)$.
\end{definition}

\noindent
Note that as being quasi-coherent is an intrinsic notion (see
Modules on Sites, Lemma \ref{sites-modules-lemma-special-locally-free})
this is equivalent to saying that the corresponding $\mathcal{O}_X$-module
on $X_{spaces, \etale}$ is quasi-coherent.

\medskip\noindent
As usual, quasi-coherent sheaves behave well with respect to pullback.

\begin{lemma}
\label{lemma-pullback-quasi-coherent}
Let $S$ be a scheme.
Let $f : X \to Y$ be a morphism of algebraic spaces over $S$.
The pullback functor
$f^* : \textit{Mod}(\mathcal{O}_Y) \to \textit{Mod}(\mathcal{O}_X)$
preserves quasi-coherent sheaves.
\end{lemma}

\begin{proof}
This is a general fact, see
Modules on Sites, Lemma \ref{sites-modules-lemma-local-pullback}.
\end{proof}

\noindent
Note that this pullback functor agrees with the usual pullback functor
between quasi-coherent sheaves of modules if $X$ and $Y$ happen to be
schemes, see
Descent, Proposition
\ref{descent-proposition-equivalence-quasi-coherent-functorial}.
Here is the obligatory lemma comparing this with quasi-coherent sheaves
on the objects of the small \'etale site of $X$.

\begin{lemma}
\label{lemma-characterize-quasi-coherent-small-etale}
Let $S$ be a scheme. Let $X$ be an algebraic space over $S$.
A quasi-coherent $\mathcal{O}_X$-module $\mathcal{F}$
is given by the following data:
\begin{enumerate}
\item for every $U \in \Ob(X_\etale)$ a quasi-coherent
$\mathcal{O}_U$-module $\mathcal{F}_U$ on $U_\etale$,
\item for every $f : U' \to U$ in $X_\etale$ an isomorphism
$c_f : f_{small}^*\mathcal{F}_U \to \mathcal{F}_{U'}$.
\end{enumerate}
These data are subject to the condition that given any $f : U' \to U$
and $g : U'' \to U'$ in $X_\etale$ the composition
$c_g \circ g_{small}^*c_f$ is equal to $c_{f \circ g}$.
\end{lemma}

\begin{proof}
Combine Lemmas \ref{lemma-pullback-quasi-coherent} and
\ref{lemma-characterize-module-small-etale}.
\end{proof}

\begin{lemma}
\label{lemma-stalk-quasi-coherent}
Let $S$ be a scheme.
Let $X$ be an algebraic space over $S$.
Let $\mathcal{F}$ be a quasi-coherent $\mathcal{O}_X$-module.
Let $x \in |X|$ be a point and let $\overline{x}$ be a geometric
point lying over $x$. Finally, let
$\varphi : (U, \overline{u}) \to (X, \overline{x})$
be an \'etale neighbourhood where $U$ is a scheme.
Then
$$
(\varphi^*\mathcal{F})_u \otimes_{\mathcal{O}_{U, u}}
\mathcal{O}_{X, \overline{x}} =
\mathcal{F}_{\overline{x}}
$$
where $u \in U$ is the image of $\overline{u}$.
\end{lemma}

\begin{proof}
Note that $\mathcal{O}_{X, \overline{x}} = \mathcal{O}_{U, u}^{sh}$ by
Lemma \ref{lemma-describe-etale-local-ring}
hence the tensor product makes sense. Moreover, from
Definition \ref{definition-stalk}
it is clear that
$$
\mathcal{F}_{\overline{u}} = \colim (\varphi^*\mathcal{F})_u
$$
where the colimit is over $\varphi : (U, \overline{u}) \to (X, \overline{x})$
as in the lemma. Hence there is a canonical map from left to right in
the statement of the lemma. We have a similar colimit description for
$\mathcal{O}_{X, \overline{x}}$
and by
Lemma \ref{lemma-characterize-quasi-coherent-small-etale}
we have
$$
((\varphi')^*\mathcal{F})_{u'} =
(\varphi^*\mathcal{F})_u \otimes_{\mathcal{O}_{U, u}} \mathcal{O}_{U', u'}
$$
whenever $(U', \overline{u}') \to (U, \overline{u})$ is a morphism of
\'etale neighbourhoods. To complete the proof we use that
$\otimes$ commutes with colimits.
\end{proof}

\begin{lemma}
\label{lemma-stalk-pullback-quasi-coherent}
Let $S$ be a scheme. Let $f : X \to Y$ be a morphism of algebraic spaces
over $S$. Let $\mathcal{G}$ be a quasi-coherent $\mathcal{O}_Y$-module.
Let $\overline{x}$ be a geometric point of $X$ and let
$\overline{y} = f \circ \overline{x}$ be the image in $Y$.
Then there is a canonical isomorphism
$$
(f^*\mathcal{G})_{\overline{x}} =
\mathcal{G}_{\overline{y}} \otimes_{\mathcal{O}_{Y, \overline{y}}}
\mathcal{O}_{X, \overline{x}}
$$
of the stalk of the pullback with the tensor product of the stalk
with the local ring of $X$ at $\overline{x}$.
\end{lemma}

\begin{proof}
Since $f^*\mathcal{G} =
f_{small}^{-1}\mathcal{G} \otimes_{f_{small}^{-1}\mathcal{O}_Y} \mathcal{O}_X$
this follows from the description of stalks of pullbacks in
Lemma \ref{lemma-stalk-pullback}
and the fact that taking stalks commutes with tensor products.
A more direct way to see this is as follows.
Choose a commutative diagram
$$
\xymatrix{
U \ar[d]_p \ar[r]_\alpha  & V \ar[d]^q \\
X \ar[r]^a & Y
}
$$
where $U$ and $V$ are schemes, and $p$ and $q$ are surjective \'etale.
By
Lemma \ref{lemma-geometric-lift-to-usual}
we can choose a geometric point $\overline{u}$ of $U$ such that
$\overline{x} = p \circ \overline{u}$. Set
$\overline{v} = \alpha \circ \overline{u}$.
Then we see that
\begin{align*}
(f^*\mathcal{G})_{\overline{x}} & =
(p^*f^*\mathcal{G})_u \otimes_{\mathcal{O}_{U, u}}
\mathcal{O}_{X, \overline{x}} \\
& = (\alpha^*q^*\mathcal{G})_u \otimes_{\mathcal{O}_{U, u}}
\mathcal{O}_{X, \overline{x}} \\
& = (q^*\mathcal{G})_v \otimes_{\mathcal{O}_{V, v}}
\mathcal{O}_{U, u} \otimes_{\mathcal{O}_{U, u}}
\mathcal{O}_{X, \overline{x}} \\
& = (q^*\mathcal{G})_v \otimes_{\mathcal{O}_{V, v}}
\mathcal{O}_{X, \overline{x}} \\
& = (q^*\mathcal{G})_v \otimes_{\mathcal{O}_{V, v}}
\mathcal{O}_{Y, \overline{y}} \otimes_{\mathcal{O}_{Y, \overline{y}}}
\mathcal{O}_{X, \overline{x}} \\
& = \mathcal{G}_{\overline{y}} \otimes_{\mathcal{O}_{Y, \overline{y}}}
\mathcal{O}_{X, \overline{x}}
\end{align*}
Here we have used
Lemma \ref{lemma-stalk-quasi-coherent} (twice)
and the corresponding result for pullbacks of quasi-coherent sheaves
on schemes, see
Sheaves, Lemma \ref{sheaves-lemma-stalk-pullback-modules}.
\end{proof}

\begin{lemma}
\label{lemma-characterize-quasi-coherent}
Let $S$ be a scheme. Let $X$ be an algebraic space over $S$.
Let $\mathcal{F}$ be a sheaf of $\mathcal{O}_X$-modules.
The following are equivalent
\begin{enumerate}
\item $\mathcal{F}$ is a quasi-coherent $\mathcal{O}_X$-module,
\item there exists an \'etale morphism $f : Y \to X$ of
algebraic spaces over $S$ with $|f| : |Y| \to |X|$ surjective
such that $f^*\mathcal{F}$ is quasi-coherent on $Y$,
\item there exists a scheme $U$ and a surjective \'etale morphism
$\varphi : U \to X$ such that $\varphi^*\mathcal{F}$ is a quasi-coherent
$\mathcal{O}_U$-module, and
\item for every affine scheme $U$ and \'etale morphism $\varphi : U \to X$ the
restriction $\varphi^*\mathcal{F}$ is a quasi-coherent $\mathcal{O}_U$-module.
\end{enumerate}
\end{lemma}

\begin{proof}
It is clear that (1) implies (2) by considering $\text{id}_X$.
Assume $f : Y \to X$ is as in (2), and let $V \to Y$ be a surjective
\'etale morphism from a scheme towards $Y$. Then the composition $V \to X$ is
surjective \'etale as well
and by Lemma \ref{lemma-pullback-quasi-coherent} the pullback of $\mathcal{F}$
to $V$ is quasi-coherent as well. Hence we see that (2) implies (3).

\medskip\noindent
Let $U \to X$ be as in (3). Let us use the abuse of notation introduced
in Equation (\ref{equation-restrict-modules}).
As $\mathcal{F}|_{U_\etale}$ is quasi-coherent there exists an
\'etale covering $\{U_i \to U\}$ such that
$\mathcal{F}|_{U_{i, \etale}}$ has a global presentation, see
Modules on Sites, Definition \ref{sites-modules-definition-global} and
Lemma \ref{sites-modules-lemma-local-final-object}.
Let $V \to X$ be an object of $X_\etale$. Since $U \to X$ is
surjective and \'etale, the family of maps $\{U_i \times_X V \to V\}$ is an
\'etale covering
of $V$. Via the morphisms $U_i \times_X V \to U_i$ we can restrict the
global presentations of $\mathcal{F}|_{U_{i, \etale}}$ to get a global
presentation of $\mathcal{F}|_{(U_i \times_X V)_\etale}$
Hence the sheaf $\mathcal{F}$ on $X_\etale$ satisfies the condition of
Modules on Sites, Definition \ref{sites-modules-definition-site-local}
and hence is quasi-coherent.

\medskip\noindent
The equivalence of (3) and (4) comes from the fact that any scheme has
an affine open covering.
\end{proof}

\begin{lemma}
\label{lemma-properties-quasi-coherent}
Let $S$ be a scheme. Let $X$ be an algebraic space over $S$.
The category $\QCoh(\mathcal{O}_X)$ of quasi-coherent sheaves on $X$
has the following properties:
\begin{enumerate}
\item Any direct sum of quasi-coherent sheaves is quasi-coherent.
\item Any colimit of quasi-coherent sheaves is quasi-coherent.
\item The kernel and cokernel of a morphism of quasi-coherent sheaves
is quasi-coherent.
\item Given a short exact sequence of $\mathcal{O}_X$-modules
$0 \to \mathcal{F}_1 \to \mathcal{F}_2 \to \mathcal{F}_3 \to 0$
if two out of three are quasi-coherent so is the third.
\item Given two quasi-coherent $\mathcal{O}_X$-modules
the tensor product is quasi-coherent.
\item Given two quasi-coherent $\mathcal{O}_X$-modules
$\mathcal{F}$, $\mathcal{G}$ such that $\mathcal{F}$
is of finite presentation (see
Section \ref{section-properties-modules}),
then the internal hom
$\SheafHom_{\mathcal{O}_X}(\mathcal{F}, \mathcal{G})$
is quasi-coherent.
\end{enumerate}
\end{lemma}

\begin{proof}
If $X$ is a scheme, then this is
Descent, Lemma \ref{descent-lemma-properties-quasi-coherent}.
We will reduce the lemma to this case by \'etale localization.

\medskip\noindent
Choose a scheme $U$ and a surjective \'etale morphism $\varphi : U \to X$.
Our notation will be that $\textit{Mod}(\mathcal{O}_U) =
\textit{Mod}(U_\etale, \mathcal{O}_U)$ and
$\QCoh(\mathcal{O}_U) = \QCoh(U_\etale, \mathcal{O}_U)$; in other
words, even though $U$ is a scheme we think of quasi-coherent
modules on $U$ as modules on the small \'etale site of $U$.
By Lemma \ref{lemma-pullback-quasi-coherent} we have a commutative diagram
$$
\xymatrix{
\QCoh(\mathcal{O}_X) \ar[r]_{\varphi^*} \ar[d] &
\QCoh(\mathcal{O}_U) \ar[d] \\
\textit{Mod}(\mathcal{O}_X) \ar[r]^{\varphi^*} &
\textit{Mod}(\mathcal{O}_U)
}
$$
The bottom horizontal arrow is the restriction functor
(\ref{equation-restrict-modules})
$\mathcal{G} \mapsto \mathcal{G}|_{U_\etale}$.
This functor has both a left adjoint and a right adjoint, see
Modules on Sites, Section \ref{sites-modules-section-localize},
hence commutes with all limits and colimits.
Moreover, we know that an object of $\textit{Mod}(\mathcal{O}_X)$ is in
$\QCoh(\mathcal{O}_X)$ if and only if its restriction to $U$ is in
$\QCoh(\mathcal{O}_U)$, see
Lemma \ref{lemma-characterize-quasi-coherent}.
With these preliminaries out of the way we can start the proof.

\medskip\noindent
Proof of (1). Let $\mathcal{F}_i$, $i \in I$ be a family of quasi-coherent
$\mathcal{O}_X$-modules. By the discussion above we have
$$
\Big(\bigoplus \mathcal{F}_i\Big)|_{U_\etale} =
\bigoplus \mathcal{F}_i|_{U_\etale}
$$
Each of the modules $\mathcal{F}_i|_{U_\etale}$ is quasi-coherent.
Hence the direct sum is quasi-coherent by the case of schemes.
Hence $\bigoplus \mathcal{F}_i$ is quasi-coherent as a module
restricting to a quasi-coherent module on $U$.

\medskip\noindent
Proof of (2). Let $\mathcal{I} \to \QCoh(\mathcal{O}_X)$,
$i \mapsto \mathcal{F}_i$ be a diagram. Then
$$
(\colim \mathcal{F}_i)|_{U_\etale} = \colim \mathcal{F}_i|_{U_\etale}
$$
by the discussion above and we conclude in the same manner.

\medskip\noindent
Proof of (3). Let $a : \mathcal{F} \to \mathcal{F}'$
be an arrow of $\QCoh(\mathcal{O}_X)$. Then
we have $\Ker(a)|_{U_\etale} = \Ker(a|_{U_\etale})$
and $\Coker(a)|_{U_\etale} = \Coker(a|_{U_\etale})$
and we conclude in the same manner.

\medskip\noindent
Proof of (4). The restriction
$0 \to \mathcal{F}_1|_{U_\etale} \to \mathcal{F}_2|_{U_\etale}
\to \mathcal{F}_3|_{U_\etale} \to 0$
is short exact. Hence we have the 2-out-of-3 property
for this sequence and we conclude as before.

\medskip\noindent
Proof of (5). Let $\mathcal{F}$ and $\mathcal{G}$ be in $\QCoh(\mathcal{O}_X)$.
Then we have
$$
(\mathcal{F} \otimes_{\mathcal{O}_X} \mathcal{G})_{U_\etale} =
\mathcal{F}|_{U_\etale} \otimes_{\mathcal{O}_U} \mathcal{G}|_{U_\etale}
$$
and we conclude as before.

\medskip\noindent
Proof of (6). Let $\mathcal{F}$ and $\mathcal{G}$
be in $\QCoh(\mathcal{O}_X)$ with $\mathcal{F}$
of finite presentation. We have
$$
\SheafHom_{\mathcal{O}_X}(\mathcal{F}, \mathcal{G})|_{U_\etale} =
\SheafHom_{\mathcal{O}_U}(\mathcal{F}|_{U_\etale}, \mathcal{G}|_{U_\etale})
$$
Namely, restriction is a localization, see
Section \ref{section-localize}, especially formula
(\ref{equation-localize-at-scheme-ringed})) and formation of internal
hom commutes with localization, see
Modules on Sites, Lemma \ref{sites-modules-lemma-internal-hom-restriction}.
Thus we conclude as before.
\end{proof}

\noindent
It is in general not the case that the pushforward of a quasi-coherent sheaf
along a morphism of algebraic spaces is quasi-coherent. We will return to this
issue in
Morphisms of Spaces, Section \ref{spaces-morphisms-section-pushforward}.




\section{Properties of modules}
\label{section-properties-modules}

\noindent
In
Modules on Sites, Sections
\ref{sites-modules-section-global},
\ref{sites-modules-section-local}, and
Definition \ref{sites-modules-definition-flat}
we have defined a number of intrinsic properties of modules of
$\mathcal{O}$-module on any ringed topos. If $X$ is an algebraic
space, we will apply these notions freely to modules on the ringed
site $(X_\etale, \mathcal{O}_X)$, or equivalently on the ringed site
$(X_{spaces, \etale}, \mathcal{O}_X)$.

\medskip\noindent
Global properties $\mathcal{P}$:
\begin{enumerate}
\item[(a)] {\it free},
\item[(b)] {\it finite free},
\item[(c)] {\it generated by global sections},
\item[(d)] {\it generated by finitely many global sections},
\item[(e)] having a {\it global presentation}, and
\item[(f)] having a {\it global finite presentation}.
\end{enumerate}
Local properties $\mathcal{P}$:
\begin{enumerate}
\item[(g)] {\it locally free},
\item[(f)] {\it finite locally free},
\item[(h)] {\it locally generated by sections},
\item[(i)] {\it locally generated by $r$ sections},
\item[(j)] {\it finite type},
\item[(k)] {\it quasi-coherent} (see Section \ref{section-quasi-coherent}),
\item[(l)] {\it of finite presentation},
\item[(m)] {\it coherent}, and
\item[(n)] {\it flat}.
\end{enumerate}
Here are some results which follow immediately from the definitions:
\begin{enumerate}
\item In each case, except for $\mathcal{P}=$``coherent'', the property
is preserved under pullback, see
Modules on Sites, Lemmas \ref{sites-modules-lemma-global-pullback},
\ref{sites-modules-lemma-local-pullback}, and
\ref{sites-modules-lemma-pullback-flat}.
\item Each of the properties above (including coherent) are preserved under
pullbacks by \'etale morphisms of algebraic spaces (because in this
case pullback is given by restriction, see
Lemma \ref{lemma-etale-morphism-topoi}).
\item Assume $f : Y \to X$ is a surjective \'etale morphism of algebraic
spaces. For each of the local properties (g) -- (m), the fact that
$f^*\mathcal{F}$ has $\mathcal{P}$ implies that $\mathcal{F}$ has
$\mathcal{P}$. This follows as $\{Y \to X\}$ is a covering in
$X_{spaces, \etale}$ and
Modules on Sites, Lemma \ref{sites-modules-lemma-local-final-object}.
\item If $X$ is a scheme, $\mathcal{F}$ is a quasi-coherent module
on $X_\etale$, and $\mathcal{P}$ any property except ``coherent'' or
``locally free'', then $\mathcal{P}$ for $\mathcal{F}$ on $X_\etale$
is equivalent to the corresponding property for
$\mathcal{F}|_{X_{Zar}}$, i.e., it corresponds to $\mathcal{P}$
for $\mathcal{F}$ when we think of it as a quasi-coherent sheaf
on the scheme $X$. See
Descent, Lemma \ref{descent-lemma-equivalence-quasi-coherent-properties}.
\item If $X$ is a locally Noetherian scheme, $\mathcal{F}$ is a
quasi-coherent module on $X_\etale$, then $\mathcal{F}$
is coherent on $X_\etale$ if and only if
$\mathcal{F}|_{X_{Zar}}$ is coherent, i.e., it corresponds to
the usual notion of a coherent sheaf on the scheme $X$ being
coherent. See
Descent, Lemma \ref{descent-lemma-equivalence-quasi-coherent-properties}.
\end{enumerate}



\section{Locally projective modules}
\label{section-locally-projective}

\noindent
Recall that in
Properties, Section \ref{properties-section-locally-projective}
we defined the notion of a locally projective
quasi-coherent module.

\begin{lemma}
\label{lemma-locally-projective}
Let $S$ be a scheme. Let $X$ be an algebraic space over $S$.
Let $\mathcal{F}$ be a quasi-coherent $\mathcal{O}_X$-module.
The following are equivalent
\begin{enumerate}
\item for some scheme $U$ and surjective \'etale morphism
$U \to X$ the restriction $\mathcal{F}|_U$ is locally projective
on $U$, and
\item for any scheme $U$ and any \'etale morphism
$U \to X$ the restriction $\mathcal{F}|_U$ is locally projective
on $U$.
\end{enumerate}
\end{lemma}

\begin{proof}
Let $U \to X$ be as in (1) and let $V \to X$ be \'etale where
$V$ is a scheme. Then $\{U \times_X V \to V\}$ is an fppf covering
of schemes. Hence if $\mathcal{F}|_U$ is locally projective, then
$\mathcal{F}|_{U \times_X V}$ is locally projective (see
Properties, Lemma \ref{properties-lemma-locally-projective-pullback})
and hence $\mathcal{F}|_V$ is locally projective, see
Descent, Lemma \ref{descent-lemma-locally-projective-descends}.
\end{proof}

\begin{definition}
\label{definition-locally-projective}
Let $S$ be a scheme. Let $X$ be an algebraic space over $S$.
Let $\mathcal{F}$ be a quasi-coherent $\mathcal{O}_X$-module.
We say $\mathcal{F}$ is {\it locally projective}
if the equivalent conditions of
Lemma \ref{lemma-locally-projective}
are satisfied.
\end{definition}

\begin{lemma}
\label{lemma-locally-projective-pullback}
Let $S$ be a scheme.
Let $f : X \to Y$ be a morphism of algebraic spaces over $S$.
Let $\mathcal{G}$ be a quasi-coherent $\mathcal{O}_Y$-module.
If $\mathcal{G}$ is locally projective on $Y$, then $f^*\mathcal{G}$
is locally projective on $X$.
\end{lemma}

\begin{proof}
Choose a surjective \'etale morphism $V \to Y$ with $V$ a scheme.
Choose a surjective \'etale morphism $U \to V \times_Y X$ with
$U$ a scheme. Denote $\psi : U \to V$ the induced morphism.
Then
$$
f^*\mathcal{G}|_U = \psi^*(\mathcal{G}|_V)
$$
Hence the lemma follows from the definition and the result in the
case of schemes, see
Properties, Lemma \ref{properties-lemma-locally-projective-pullback}.
\end{proof}





\section{Quasi-coherent sheaves and presentations}
\label{section-quasi-coherent-presentation}

\noindent
Let $S$ be a scheme. Let $X$ be an algebraic space over $S$.
Let $X = U/R$ be a presentation of $X$ coming from any surjective
\'etale morphism $\varphi : U \to X$, see
Spaces, Definition \ref{spaces-definition-presentation}.
In particular, we obtain a groupoid $(U, R, s, t, c)$, such that
$j = (t, s) : R \to U \times_S U$, see
Groupoids, Lemma \ref{groupoids-lemma-equivalence-groupoid}.
In
Groupoids, Definition \ref{groupoids-definition-groupoid-module}
we have the defined the notion of a quasi-coherent sheaf
on an arbitrary groupoid. With these notions in place we have
the following observation.

\begin{proposition}
\label{proposition-quasi-coherent}
With $S$, $\varphi : U \to X$, and $(U, R, s, t, c)$ as above.
For any quasi-coherent $\mathcal{O}_X$-module $\mathcal{F}$ the
sheaf $\varphi^*\mathcal{F}$ comes equipped with a canonical
isomorphism
$$
\alpha : t^*\varphi^*\mathcal{F} \longrightarrow s^*\varphi^*\mathcal{F}
$$
which satisfies the conditions of
Groupoids, Definition \ref{groupoids-definition-groupoid-module}
and therefore defines a quasi-coherent sheaf on $(U, R, s, t, c)$.
The functor $\mathcal{F} \mapsto (\varphi^*\mathcal{F}, \alpha)$
defines an equivalence of categories
$$
\begin{matrix}
\text{Quasi-coherent} \\
\mathcal{O}_X\text{-modules}
\end{matrix}
\longleftrightarrow
\begin{matrix}
\text{Quasi-coherent modules}\\
\text{on }(U, R, s, t, c)
\end{matrix}
$$
\end{proposition}

\begin{proof}
In the statement of the proposition, and in this proof we think of a
quasi-coherent sheaf on a scheme as a quasi-coherent sheaf on the small
\'etale site of that scheme. This is permissible by the results of
Descent, Sections \ref{descent-section-quasi-coherent-sheaves},
\ref{descent-section-quasi-coherent-cohomology}, and
\ref{descent-section-quasi-coherent-sheaves-bis}.

\medskip\noindent
The existence of $\alpha$ comes from the fact that
$\varphi \circ t = \varphi \circ s$ and that pullback is
functorial in the morphism, see discussion surrounding
Equation (\ref{equation-push-pull}). In exactly the same way, i.e., by
functoriality of pullback, we see that the isomorphism $\alpha$ satisfies
condition (1) of
Groupoids, Definition \ref{groupoids-definition-groupoid-module}.
To see condition (2) of the definition it suffices to see that $\alpha$
is an isomorphism which is clear. The construction
$\mathcal{F} \mapsto (\varphi^*\mathcal{F}, \alpha)$
is clearly functorial in the quasi-coherent sheaf $\mathcal{F}$.
Hence we obtain the functor from left to right in the displayed
formula of the lemma.

\medskip\noindent
Conversely, suppose that $(\mathcal{F}, \alpha)$ is a quasi-coherent
sheaf on $(U, R, s, t, c)$. Let $V \to X$ be an object of $X_\etale$.
In this case the morphism $V' = U \times_X V \to V$ is a surjective \'etale
morphism of schemes, and hence $\{V' \to V\}$ is an \'etale
covering of $V$. Moreover, the quasi-coherent sheaf $\mathcal{F}$
pulls back to a quasi-coherent sheaf $\mathcal{F}'$ on $V'$.
Since $R = U \times_X U$ with $t = \text{pr}_0$ and $s = \text{pr}_0$
we see that $V' \times_V V' = R \times_X V$ with projection maps
$V' \times_V V' \to V'$ equal to the pullbacks of $t$ and $s$. Hence
$\alpha$ pulls back to an isomorphism
$\alpha' : \text{pr}_0^*\mathcal{F}' \to \text{pr}_1^*\mathcal{F}'$, and
the pair $(\mathcal{F}', \alpha')$ is a descend datum for quasi-coherent
sheaves with respect to $\{V' \to V\}$. By
Descent, Proposition
\ref{descent-proposition-fpqc-descent-quasi-coherent}
this descent datum is effective, and we obtain a quasi-coherent
$\mathcal{O}_V$-module $\mathcal{F}_V$ on $V_\etale$.
To see that this gives a quasi-coherent sheaf on $X_\etale$ we have
to show (by
Lemma \ref{lemma-characterize-quasi-coherent-small-etale})
that for any morphism $f : V_1 \to V_2$ in $X_\etale$
there is a canonical isomorphism
$c_f : \mathcal{F}_{V_1} \to \mathcal{F}_{V_2}$
compatible with compositions of morphisms. We omit the verification.
We also omit the verification that this defines a functor from the
category on the right to the category on the left which is inverse
to the functor described above.
\end{proof}

\begin{proposition}
\label{proposition-coherator}
Let $S$ be a scheme. Let $X$ be an algebraic space over $S$.
\begin{enumerate}
\item The category $\QCoh(\mathcal{O}_X)$ is a Grothendieck
abelian category. Consequently, $\QCoh(\mathcal{O}_X)$
has enough injectives and all limits.
\item The inclusion functor
$\QCoh(\mathcal{O}_X) \to \textit{Mod}(\mathcal{O}_X)$
has a right adjoint\footnote{This functor is sometimes called
the {\it coherator}.}
$$
Q : \textit{Mod}(\mathcal{O}_X) \longrightarrow \QCoh(\mathcal{O}_X)
$$
such that for every quasi-coherent sheaf $\mathcal{F}$ the adjunction mapping
$Q(\mathcal{F}) \to \mathcal{F}$ is an isomorphism.
\end{enumerate}
\end{proposition}

\begin{proof}
This proof is a repeat of the proof in the case of schemes, see
Properties, Proposition \ref{properties-proposition-coherator}.
We advise the reader to read that proof first.

\medskip\noindent
Part (1) means $\QCoh(\mathcal{O}_X)$ (a) has all colimits,
(b) filtered colimits are exact, and (c) has a generator, see
Injectives, Section \ref{injectives-section-grothendieck-conditions}.
By Lemma \ref{lemma-properties-quasi-coherent}
colimits in $\QCoh(\mathcal{O}_X)$ exist and agree
with colimits in $\textit{Mod}(\mathcal{O}_X)$. By
Modules on Sites, Lemma \ref{sites-modules-lemma-limits-colimits}
filtered colimits are exact. Hence (a) and (b) hold.

\medskip\noindent
To construct a generator, choose a presentation $X = U/R$ so that
$(U, R, s, t, c)$ is an
\'etale groupoid scheme and in particular $s$ and $t$ are flat morphisms
of schemes. Pick a cardinal $\kappa$ as in
Groupoids, Lemma \ref{groupoids-lemma-colimit-kappa}.
Pick a collection $(\mathcal{E}_t, \alpha_t)_{t \in T}$ of
$\kappa$-generated quasi-coherent modules on
$(U, R, s, t, c)$ as in
Groupoids, Lemma \ref{groupoids-lemma-set-of-iso-classes}.
Let $\mathcal{F}_t$ be the quasi-coherent module on $X$ which
corresponds to the quasi-coherent module $(\mathcal{E}_t, \alpha_t)$ via
the equivalence of categories of
Proposition \ref{proposition-quasi-coherent}.
Then we see that every quasi-coherent module $\mathcal{H}$ is the
directed colimit of its quasi-coherent submodules which are isomorphic
to one of the $\mathcal{F}_t$. Thus $\bigoplus_t \mathcal{F}_t$ is
a generator of $\QCoh(\mathcal{O}_X)$ and we conclude that (c) holds.
The assertions on limits and injectives hold in any
Grothendieck abelian category, see
Injectives, Theorem
\ref{injectives-theorem-injective-embedding-grothendieck} and
Lemma \ref{injectives-lemma-grothendieck-products}.

\medskip\noindent
Proof of (2). To construct $Q$ we use the following general procedure.
Given an object $\mathcal{F}$ of $\textit{Mod}(\mathcal{O}_X)$
we consider the functor
$$
\QCoh(\mathcal{O}_X)^{opp} \longrightarrow \textit{Sets},\quad
\mathcal{G} \longmapsto \Hom_X(\mathcal{G}, \mathcal{F})
$$
This functor transforms colimits into limits,
hence is representable, see
Injectives, Lemma \ref{injectives-lemma-grothendieck-brown}.
Thus there exists a quasi-coherent sheaf $Q(\mathcal{F})$
and a functorial isomorphism
$\Hom_X(\mathcal{G}, \mathcal{F}) = \Hom_X(\mathcal{G}, Q(\mathcal{F}))$
for $\mathcal{G}$ in $\QCoh(\mathcal{O}_X)$. By the Yoneda lemma
(Categories, Lemma \ref{categories-lemma-yoneda})
the construction $\mathcal{F} \leadsto Q(\mathcal{F})$ is
functorial in $\mathcal{F}$. By construction $Q$ is a right
adjoint to the inclusion functor.
The fact that $Q(\mathcal{F}) \to \mathcal{F}$ is an isomorphism
when $\mathcal{F}$ is quasi-coherent is a formal consequence of the fact
that the inclusion functor
$\QCoh(\mathcal{O}_X) \to \textit{Mod}(\mathcal{O}_X)$
is fully faithful.
\end{proof}






\section{Morphisms towards schemes}
\label{section-morphisms-to-schemes}

\noindent
Here is the analogue of
Schemes, Lemma \ref{schemes-lemma-morphism-into-affine}.

\begin{lemma}
\label{lemma-morphism-to-affine-scheme}
Let $X$ be an algebraic space over $\mathbf{Z}$.
Let $T$ be an affine scheme.
The map
$$
\Mor(X, T)
\longrightarrow
\Hom(\Gamma(T, \mathcal{O}_T), \Gamma(X, \mathcal{O}_X))
$$
which maps $f$ to $f^\sharp$ (on global sections) is bijective.
\end{lemma}

\begin{proof}
We construct the inverse of the map.
Let $\varphi : \Gamma(T, \mathcal{O}_T) \to \Gamma(X, \mathcal{O}_X)$
be a ring map. Choose a presentation $X = U/R$, see
Spaces, Definition \ref{spaces-definition-presentation}.
By
Schemes, Lemma \ref{schemes-lemma-morphism-into-affine}
the composition
$$
\Gamma(T, \mathcal{O}_T) \to \Gamma(X, \mathcal{O}_X) \to
\Gamma(U, \mathcal{O}_U)
$$
corresponds to a unique morphism of schemes $g : U \to T$. By the same lemma
the two compositions $R \to U \to T$ are equal. Hence we obtain a morphism
$f : X = U/R \to T$ such that $U \to X \to T$ equals $g$. By construction
the diagram
$$
\xymatrix{
\Gamma(U, \mathcal{O}_U) & \Gamma(X, \mathcal{O}_X) \ar[l] \\
& \Gamma(T, \mathcal{O}_T) \ar[lu]^{g^\sharp} \ar[u]^{\varphi}_{f^\sharp}
}
$$
commutes. Hence $f^\sharp$ equals $\varphi$ because $U \to X$ is an
\'etale covering and $\mathcal{O}_X$ is a sheaf on $X_\etale$.
The uniqueness of $f$ follows from the uniqueness of $g$.
\end{proof}





\section{Quotients by free actions}
\label{section-quotient-by-free}

\noindent
Let $S$ be a scheme.
Let $X$ be an algebraic space over $S$.
Let $G$ be an abstract group.
Let $a : G \to \text{Aut}(X)$ be a homomorphism, i.e., $a$ is an
{\it action} of $G$ on $X$. We will say the action is {\it free}
if for every scheme $T$ over $S$ the map
$$
G \times X(T) \longrightarrow X(T)
$$
is free. (We cannot use a criterion as in
Spaces, Lemma \ref{spaces-lemma-quotient}
because points may not have well defined residue fields.)
In case the action is free we're going to construct the quotient $X/G$
as an algebraic space. This is a special case of the general
Bootstrap, Lemma \ref{bootstrap-lemma-quotient-free-action}
that we will prove later.

\begin{lemma}
\label{lemma-quotient}
Let $S$ be a scheme.
Let $X$ be an algebraic space over $S$.
Let $G$ be an abstract group with a free action on $X$.
Then the quotient sheaf $X/G$ is an algebraic space.
\end{lemma}

\begin{proof}
The statement means that the sheaf $F$ associated to the presheaf
$$
T \longmapsto X(T)/G
$$
is an algebraic space. To see this we will construct a presentation.
Namely, choose a scheme $U$ and a surjective \'etale morphism
$\varphi : U \to X$. Set $V = \coprod_{g \in G} U$ and set
$\psi : V \to X$ equal to $a(g) \circ \varphi$ on the component corresponding
to $g \in G$. Let $G$ act on $V$ by permuting the components, i.e.,
$g_0 \in G$ maps the component corresponding to $g$ to the component
corresponding to $g_0g$ via the identity morphism of $U$.
Then $\psi$ is a $G$-equivariant morphism, i.e., we reduce to the
case dealt with in the next paragraph.

\medskip\noindent
Assume that there exists a $G$-action on $U$ and that $U \to X$ is surjective,
\'etale and $G$-equivariant. In this case there is an induced
action of $G$ on $R = U \times_X U$ compatible with the projection
mappings $t, s : R \to U$. Now we claim that
$$
X/G = U/\coprod\nolimits_{g \in G} R
$$
where the map
$$
j : \coprod\nolimits_{g \in G} R
\longrightarrow
U \times_S U
$$
is given by $(r, g) \mapsto (t(r), g(s(r)))$. Note that $j$ is a monomorphism:
If $(t(r), g(s(r))) = (t(r'), g'(s(r')))$, then
$t(r) = t(r')$, hence $r$ and $r'$ have the same image in $X$ under
both $s$ and $t$, hence $g = g'$ (as $G$ acts freely on $X$), hence
$s(r) = s(r')$, hence $r = r'$ (as $R$ is an equivalence relation on $U$).
Moreover $j$ is an equivalence relation (details omitted).
Both projections $\coprod\nolimits_{g \in G} R \to U$ are \'etale, as
$s$ and $t$ are \'etale. Thus $j$ is an \'etale equivalence relation
and $U/\coprod\nolimits_{g \in G} R$ is an algebraic space by
Spaces, Theorem \ref{spaces-theorem-presentation}.
There is a map
$$
U/\coprod\nolimits_{g \in G} R \longrightarrow X/G
$$
induced by the map $U \to X$. We omit the proof that it is an
isomorphism of sheaves.
\end{proof}






\begin{multicols}{2}[\section{Other chapters}]
\noindent
Preliminaries
\begin{enumerate}
\item \hyperref[introduction-section-phantom]{Introduction}
\item \hyperref[conventions-section-phantom]{Conventions}
\item \hyperref[sets-section-phantom]{Set Theory}
\item \hyperref[categories-section-phantom]{Categories}
\item \hyperref[topology-section-phantom]{Topology}
\item \hyperref[sheaves-section-phantom]{Sheaves on Spaces}
\item \hyperref[sites-section-phantom]{Sites and Sheaves}
\item \hyperref[stacks-section-phantom]{Stacks}
\item \hyperref[fields-section-phantom]{Fields}
\item \hyperref[algebra-section-phantom]{Commutative Algebra}
\item \hyperref[brauer-section-phantom]{Brauer Groups}
\item \hyperref[homology-section-phantom]{Homological Algebra}
\item \hyperref[derived-section-phantom]{Derived Categories}
\item \hyperref[simplicial-section-phantom]{Simplicial Methods}
\item \hyperref[more-algebra-section-phantom]{More on Algebra}
\item \hyperref[smoothing-section-phantom]{Smoothing Ring Maps}
\item \hyperref[modules-section-phantom]{Sheaves of Modules}
\item \hyperref[sites-modules-section-phantom]{Modules on Sites}
\item \hyperref[injectives-section-phantom]{Injectives}
\item \hyperref[cohomology-section-phantom]{Cohomology of Sheaves}
\item \hyperref[sites-cohomology-section-phantom]{Cohomology on Sites}
\item \hyperref[dga-section-phantom]{Differential Graded Algebra}
\item \hyperref[dpa-section-phantom]{Divided Power Algebra}
\item \hyperref[hypercovering-section-phantom]{Hypercoverings}
\end{enumerate}
Schemes
\begin{enumerate}
\setcounter{enumi}{24}
\item \hyperref[schemes-section-phantom]{Schemes}
\item \hyperref[constructions-section-phantom]{Constructions of Schemes}
\item \hyperref[properties-section-phantom]{Properties of Schemes}
\item \hyperref[morphisms-section-phantom]{Morphisms of Schemes}
\item \hyperref[coherent-section-phantom]{Cohomology of Schemes}
\item \hyperref[divisors-section-phantom]{Divisors}
\item \hyperref[limits-section-phantom]{Limits of Schemes}
\item \hyperref[varieties-section-phantom]{Varieties}
\item \hyperref[topologies-section-phantom]{Topologies on Schemes}
\item \hyperref[descent-section-phantom]{Descent}
\item \hyperref[perfect-section-phantom]{Derived Categories of Schemes}
\item \hyperref[more-morphisms-section-phantom]{More on Morphisms}
\item \hyperref[flat-section-phantom]{More on Flatness}
\item \hyperref[groupoids-section-phantom]{Groupoid Schemes}
\item \hyperref[more-groupoids-section-phantom]{More on Groupoid Schemes}
\item \hyperref[etale-section-phantom]{\'Etale Morphisms of Schemes}
\end{enumerate}
Topics in Scheme Theory
\begin{enumerate}
\setcounter{enumi}{40}
\item \hyperref[chow-section-phantom]{Chow Homology}
\item \hyperref[intersection-section-phantom]{Intersection Theory}
\item \hyperref[pic-section-phantom]{Picard Schemes of Curves}
\item \hyperref[adequate-section-phantom]{Adequate Modules}
\item \hyperref[dualizing-section-phantom]{Dualizing Complexes}
\item \hyperref[duality-section-phantom]{Duality for Schemes}
\item \hyperref[discriminant-section-phantom]{Discriminants and Differents}
\item \hyperref[local-cohomology-section-phantom]{Local Cohomology}
\item \hyperref[curves-section-phantom]{Algebraic Curves}
\item \hyperref[resolve-section-phantom]{Resolution of Surfaces}
\item \hyperref[models-section-phantom]{Semistable Reduction}
\item \hyperref[pione-section-phantom]{Fundamental Groups of Schemes}
\item \hyperref[etale-cohomology-section-phantom]{\'Etale Cohomology}
\item \hyperref[ssgroups-section-phantom]{Linear Algebraic Groups}
\item \hyperref[crystalline-section-phantom]{Crystalline Cohomology}
\item \hyperref[proetale-section-phantom]{Pro-\'etale Cohomology}
\end{enumerate}
Algebraic Spaces
\begin{enumerate}
\setcounter{enumi}{56}
\item \hyperref[spaces-section-phantom]{Algebraic Spaces}
\item \hyperref[spaces-properties-section-phantom]{Properties of Algebraic Spaces}
\item \hyperref[spaces-morphisms-section-phantom]{Morphisms of Algebraic Spaces}
\item \hyperref[decent-spaces-section-phantom]{Decent Algebraic Spaces}
\item \hyperref[spaces-cohomology-section-phantom]{Cohomology of Algebraic Spaces}
\item \hyperref[spaces-limits-section-phantom]{Limits of Algebraic Spaces}
\item \hyperref[spaces-divisors-section-phantom]{Divisors on Algebraic Spaces}
\item \hyperref[spaces-over-fields-section-phantom]{Algebraic Spaces over Fields}
\item \hyperref[spaces-topologies-section-phantom]{Topologies on Algebraic Spaces}
\item \hyperref[spaces-descent-section-phantom]{Descent and Algebraic Spaces}
\item \hyperref[spaces-perfect-section-phantom]{Derived Categories of Spaces}
\item \hyperref[spaces-more-morphisms-section-phantom]{More on Morphisms of Spaces}
\item \hyperref[spaces-flat-section-phantom]{Flatness on Algebraic Spaces}
\item \hyperref[spaces-groupoids-section-phantom]{Groupoids in Algebraic Spaces}
\item \hyperref[spaces-more-groupoids-section-phantom]{More on Groupoids in Spaces}
\item \hyperref[bootstrap-section-phantom]{Bootstrap}
\item \hyperref[spaces-pushouts-section-phantom]{Pushouts of Algebraic Spaces}
\end{enumerate}
Topics in Geometry
\begin{enumerate}
\setcounter{enumi}{73}
\item \hyperref[spaces-chow-section-phantom]{Chow Groups of Spaces}
\item \hyperref[groupoids-quotients-section-phantom]{Quotients of Groupoids}
\item \hyperref[spaces-more-cohomology-section-phantom]{More on Cohomology of Spaces}
\item \hyperref[spaces-simplicial-section-phantom]{Simplicial Spaces}
\item \hyperref[spaces-duality-section-phantom]{Duality for Spaces}
\item \hyperref[formal-spaces-section-phantom]{Formal Algebraic Spaces}
\item \hyperref[restricted-section-phantom]{Restricted Power Series}
\item \hyperref[spaces-resolve-section-phantom]{Resolution of Surfaces Revisited}
\end{enumerate}
Deformation Theory
\begin{enumerate}
\setcounter{enumi}{81}
\item \hyperref[formal-defos-section-phantom]{Formal Deformation Theory}
\item \hyperref[defos-section-phantom]{Deformation Theory}
\item \hyperref[cotangent-section-phantom]{The Cotangent Complex}
\item \hyperref[examples-defos-section-phantom]{Deformation Problems}
\end{enumerate}
Algebraic Stacks
\begin{enumerate}
\setcounter{enumi}{85}
\item \hyperref[algebraic-section-phantom]{Algebraic Stacks}
\item \hyperref[examples-stacks-section-phantom]{Examples of Stacks}
\item \hyperref[stacks-sheaves-section-phantom]{Sheaves on Algebraic Stacks}
\item \hyperref[criteria-section-phantom]{Criteria for Representability}
\item \hyperref[artin-section-phantom]{Artin's Axioms}
\item \hyperref[quot-section-phantom]{Quot and Hilbert Spaces}
\item \hyperref[stacks-properties-section-phantom]{Properties of Algebraic Stacks}
\item \hyperref[stacks-morphisms-section-phantom]{Morphisms of Algebraic Stacks}
\item \hyperref[stacks-limits-section-phantom]{Limits of Algebraic Stacks}
\item \hyperref[stacks-cohomology-section-phantom]{Cohomology of Algebraic Stacks}
\item \hyperref[stacks-perfect-section-phantom]{Derived Categories of Stacks}
\item \hyperref[stacks-introduction-section-phantom]{Introducing Algebraic Stacks}
\item \hyperref[stacks-more-morphisms-section-phantom]{More on Morphisms of Stacks}
\item \hyperref[stacks-geometry-section-phantom]{The Geometry of Stacks}
\end{enumerate}
Topics in Moduli Theory
\begin{enumerate}
\setcounter{enumi}{99}
\item \hyperref[moduli-section-phantom]{Moduli Stacks}
\item \hyperref[moduli-curves-section-phantom]{Moduli of Curves}
\end{enumerate}
Miscellany
\begin{enumerate}
\setcounter{enumi}{101}
\item \hyperref[examples-section-phantom]{Examples}
\item \hyperref[exercises-section-phantom]{Exercises}
\item \hyperref[guide-section-phantom]{Guide to Literature}
\item \hyperref[desirables-section-phantom]{Desirables}
\item \hyperref[coding-section-phantom]{Coding Style}
\item \hyperref[obsolete-section-phantom]{Obsolete}
\item \hyperref[fdl-section-phantom]{GNU Free Documentation License}
\item \hyperref[index-section-phantom]{Auto Generated Index}
\end{enumerate}
\end{multicols}


\bibliography{my}
\bibliographystyle{amsalpha}

\end{document}
